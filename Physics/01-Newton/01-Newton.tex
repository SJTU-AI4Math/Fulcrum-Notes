\documentclass[UTF8]{ctexart}

\makeatletter
\def\input@path{{../../Fulcrum-Template/}{../../Operator-List/}}
\makeatother

\usepackage{FulcrumCN}
\usepackage{OperatorListCN}
\usepackage{F4Math}
\usepackage{F4Physics}

% margin
\usepackage{geometry}
\geometry{
    paper =a4paper,
    top =3cm,
    bottom =3cm,
    left=2cm,
    right =2cm
}
\linespread{1.2}

\begin{document}


\begin{center}
    {\LARGE 经典力学笔记}
\end{center}

\section{物理量}

    \subsection{物理量}

        \begin{dfn}
            [Dimension]
            {量纲}
            [Dimension]
            [猫猫]
            定义\textbf{基本量纲} 为以下之一: 
            \begin{enumerate}
                \item \textbf{长度}, 记作 \(\Length\); 定义长度的单位为\textbf{米}, 记作 \(\m\); 
                \item \textbf{时间}, 记作 \(\Time\); 定义时间的单位为\textbf{秒}, 记作 \(\s\); 
                \item \textbf{质量}, 记作 \(\Mass\); 定义质量的单位为\textbf{千克}, 记作 \(\kg\); 
                \item \textbf{电流}, 记作 \(\ElectricCurrent\); 定义电流的单位为 \textbf{Ampere}, 记作 \(\A\); 
                \item \textbf{温度}, 记作 \(\Temperature\); 定义温度的单位为 \textbf{Kelvin}, 记作 \(\Klv\); 
                \item \textbf{物质的量}, 记作 \(\AmountOfSubstance\); 定义物质的量的单位为\textbf{摩尔}, 记作 \(\mol\); 
                \item \textbf{光强}, 记作 \(\LuminousIntensity\); 定义光强的单位为\textbf{坎德拉}, 记作 \(\cd\);
            \end{enumerate}

            定义\textbf{量纲}为基本量纲\(\to\Z\). 
        \end{dfn}

        \begin{str}
            [Quantity]
            {物理量}
            [Quantity]
            [猫猫]
            定义\textbf{物理量}包含以下信息: 
            \begin{enumerate}
                \item \(\R\); 
                \item 量纲; 
            \end{enumerate}
        \end{str}

        \begin{rmk}
            [猫猫]
            不引起混淆的情况下, 将物理量类型类型 \(\R\times\cdot\) 简记为 \(\cdot\). 
        \end{rmk}

        \begin{dfn}
            [Quantity-Vector]
            {向量}
            [Vector]
            [猫猫]
        \end{dfn}

    \subsection{世界模型}

        \begin{str}
            [Space]
            {空间}
            [Space]
            [猫猫]
            设 \(n:[3]\), 定义 \(n\) 维\textbf{空间}为: \(\Length_n\). 
        \end{str}

\section{运动学}

    \subsection{参考系}
        
        \begin{dfn}
            []
            {坐标系}
            [Coordinate System]
            [猫猫]
            设 \((S,\mathcal{T})\) 是\拓扑空间, \(n:\N\), \(\varphi:S\to\Length_n\), 定义 \((S,\varphi)\) 是 \(\Length_n\) 的\textbf{坐标系}, 当且仅当: \(\varphi\) 是一个同胚. 
        \end{dfn}

        \begin{xmp}
            [Cartesian-Coordinate-System]
            {直角坐标系}
            [Cartesian Coordinate System]
            [猫猫]
            \[P(x,y,z) : X\times Y\times Z\]
        \end{xmp}

        \begin{xmp}
            [Polar-Coordinate-System]
            {极坐标系}
            [Polar Coordinate System]
            [猫猫]
            \[(r,\theta)\]
            径向单位矢量 $\times$ 横向单位矢量
        \end{xmp}

        \begin{xmp}
            [Spherical-Coordinate-System]
            {球坐标系}
            [Spherical Coordinate System]
            [猫猫]
            \[(r,\theta,\varphi)\]

            其中 $r$ 为半径, $\theta$ 为平面角, $\varphi$ 为高度角. 
        \end{xmp}

        \begin{xmp}
            [Cylindrical-Coordinate-System]
            {柱坐标系}
            [Cylindrical Coordinate System]
            [猫猫]
            \[(\rho,\theta,z)\]
            
            其中 $(\rho,\theta)$ 构成二维平面上的极坐标系. 
        \end{xmp}

    \subsection{运动模型}

        \begin{str}
            [Object]
            {物体}
            [Object]
            [猫猫]
            设 \(n:[3]\), 定义 \(n\) 维\textbf{物体}类型. 
        \end{str}

        \begin{str}
            [Translation]
            {平动}
            [Translation]
            [猫猫]
            设 \(n:[3]\), \(O\) 是 \物体[\(n\)], \(T\subseteq\Time\), 定义 \(O\) 在 \(T\) 上\textbf{平动}包含以下信息: 
            \begin{enumerate}
                \item \textbf{轨迹} \(\bm{x}_O:T\to\Length_n\)
            \end{enumerate}
        \end{str}
        
        \begin{dfn}
            [Displacement]
            {位移}
            [Displacement]
            [猫猫]
            设 \(n:[3]\), \(O\) 是 \物体[\(n\)], \(a,b:\Time\), \(a<b\), \(O\) 在 \(\Icc{a}{b}\) 上\平动, 定义 \(O\) 在 \(\Icc{a}{b}\) 上的\textbf{位移}为: 
            \[\traj{O}(b)-\traj{O}(a)\]

            记作 \(\Delta\bm{x}_O\Icc{a}{b}\). 
        \end{dfn}
        
        \begin{dfn}
            [Velocity]
            {瞬时速度}
            [Instantaneous Velocity]
            [猫猫]
            设 \(n:[3]\), \(O\) 是 \物体[\(n\)], \(T\subseteq\Time\), \(O\) 在 \(T\) 上\平动, \(t_0\in T\), 定义 \(O\) 在 \(t_0\) 时刻的\textbf{瞬时速度}, 简称\textbf{速度}, 为: 
            \[\frac{\dd}{\dd t}\traj{O}(t_0)\]

            记作 \(\vlct{O}(t_0)\). 
        \end{dfn}
        
        \begin{dfn}
            [Average-Velocity]
            {平均速度}
            [Average Velocity]
            [猫猫]
            设 \(n:[3]\), \(O\) 是 \物体[\(n\)], \(T\subseteq\Time\), \(T\) Lebesgue 可测, 定义 \(O\) 在 \(T\) 上的\textbf{平均速度}为: 
            \[\frac{1}{\mu(T)}\int_T\vlct{O}(t)\dd t\]

            记作 \(\bar{\bm{v}}_O(T)\). 
        \end{dfn}
        
        \begin{dfn}
            {加速度}
            [Acceleration]
            [猫猫]
            设 \(n:[3]\), \(O\) 是 \物体[\(n\)], \(T\subseteq\Time\), \(O\) 在 \(T\) 上\平动, \(t_0\in T\), 定义 \(O\) 在 \(t_0\) 时刻的\textbf{加速度}为: 
            \[\frac{\dd}{\dd t}\vlct{O}(t_0)\]

            记作 \(\bm{a}_O(t_0)\). 
        \end{dfn}

    \subsection{Galileo 变换}
        
        \begin{dfn}
            []
            {参考系}
            [Reference Frame]
            [猫猫]
        \end{dfn}

\section{牛顿力学}

    \subsection{牛顿运动定律}
        
        \begin{str}
            [Massive-Object]
            {有质物体}
            [Massive Object]
            [猫猫]
            设 \(n:[3]\), 定义 \(n\) 维\textbf{有质物体}类型包含以下信息: 
            \begin{enumerate}
                \item \(n\) 维流形; 
                \item \textbf{密度} \(\rho:\Length_n\to\Mass\cdot\Length^{-n}\); 
            \end{enumerate}
        \end{str}

        \begin{str}
            [Field]
            {场}
            [Field]
            [猫猫]
            定义\textbf{场}是\物体 的子类型. 
        \end{str}
        
        \begin{dfn}
            [Force]
            {力}
            [Force]
            [猫猫]
            设 \(A,B\) 是\有质物体, 定义 \(A\) 施加在 \(B\) 上的\textbf{力}为 \(\Mass\cdot\Length\cdot\Time^{-2}\), 记为 \(\Force_{AB}\). 

            设 \(E\) 是\场, \(A\) 是\有质物体, 定义 \(E\) 施加在 \(A\) 上的\textbf{力}为 \(\Mass\cdot\Length\cdot\Time^{-2}\), 记为 \(\Force_{FA}\). 

            定义 \(\Force\) 的单位为 \textbf{Newton}, 记作 \(\Nt\). 
        \end{dfn}

        \begin{str}
            []
            {动力学系统}
            [Dynamical System]
            [猫猫]
            定义\textbf{动力学系统}类型包含以下信息: 
            \begin{enumerate}
                \item 参考系; 
                \item 场集 \(\mathcal{E}\); 
                \item 有质物体集 \(\mathcal{O}\); 
                \item 力集 \(\mathcal{F}\); 
            \end{enumerate}
        \end{str}
        
        \begin{dfn}
            []
            {惯性系}
            [Inertial Frame]
            [猫猫]
            质点在不受任何外力作用时, 其运动状态保持不变的参考系称为\textbf{惯性系}. 

            上述性质也称为\textbf{Newton 第一定律}. 
        \end{dfn}
        
        \begin{axm}
            []
            {Newton 第二定律}
            [Newton's Second Law]
            [猫猫]
            \[\bm{F}=m\bm{a}\]
        \end{axm}
        
        \begin{axm}
            []
            {Newton 第三定律}
            [Newton's Third Law]
            [猫猫]
            设 \((R,\varnothing,\{A,B\},\mathcal{F})\) 是动力学系统, \(\bm{F}:\Force_{AB}\), 则: 
            \[\bm{F}_{12}=-\bm{F}_{21}\]
        \end{axm}
        
        \begin{axm}
            []
            {Galileo 变换}
            [Galilean Transformation]
            [猫猫]
            将一惯性系映射到另一惯性系的变换称为\textbf{Galileo 变换}. 

            一切惯性系在力学上等价. 
        \end{axm}

    \subsection{质点模型}
        
        \begin{str}
            [Mass-Point]
            {质点}
            [Mass Point]
            [猫猫]
            定义\textbf{质点}类型包含以下信息: 
            \begin{enumerate}
                \item 位置 \(\bm{x}\). 
                \item 质量 \(m\); 
            \end{enumerate}
        \end{str}

    \subsection{常见力}
        
        \begin{axm}
            []
            {万有引力}
            [Gravitational Force]
            [猫猫]
            \[\bm{F}=-\frac{GMm}{r^3}\cdot\bm{r}\]
            \[\|\bm{F}\|=\frac{GMm}{r^2}\]
        \end{axm}
        
        \begin{axm}
            []
            {弹力}
            [Elastic Force]
            [猫猫]
            \[\bm{F}-k\bm{x}\]
        \end{axm}
        
        \begin{axm}
            []
            {摩擦力}
            [Frictional Force]
            [猫猫]
            \[\|\bm{f}\|=\mu N\]
        \end{axm}

    \subsection{非惯性系}

    \subsection{质点系运动学}
        
        \begin{dfn}
            []
            {质点系}
            [Mass System]
            [猫猫]
            同一参考系下质点的集合称为\textbf{质点系}. 
        \end{dfn}
        
        \begin{dfn}
            []
            {质心}
            [Center of Mass]
            [猫猫]
            离散质点系\(X\)的\textbf{质心}定义为如下质点: 
            \[m_C=\sum_{x\in X}m_X\]
            \[\bm{r}_C=\frac{\sum\limits_{x\in X}m_x\bm{r}_x}{m_C}\]

            连续质点系\(X\)的\textbf{质心}定义为如下质点:
            \[m_C=\int_X\dd m\]
            \[\bm{r}_C=\frac{\int_X\bm{r}\dd m}{m_C}\]
        \end{dfn}
        
        \begin{thm}
            []
            {质心运动定理}
            []
            [猫猫]
            质心运动状态只与外力有关, 且符合牛顿第二定律: 
            \[\bm{F}=m_C\bm{a}_C\]
        \end{thm}

\section{功与能}

    \subsection{功与动能定理}
        
        \begin{dfn}
            []
            {功}
            [Work]
            [猫猫]
            力对质点在一段路程 \(L\) 上做的\textbf{功}定义为: 
            \[W=\int_L\bm{F}\cdot\dd\bm{r}\]
        \end{dfn}
        
        \begin{dfn}
            []
            {功率}
            [Power]
            [猫猫]
            \[P=\frac{\dd W}{\dd t}=\bm{F}\cdot\bm{v}\]
        \end{dfn}
        
        \begin{dfn}
            []
            {动能}
            [Kinetic Energy]
            [猫猫]
            质点的\textbf{动能}定义为: 
            \[E_k=\frac{1}{2}m\bm{v}^2\]

            动能与参考系有关. 
        \end{dfn}
        
        \begin{thm}
            []
            {动能定理}
            [Work-Energy Theorem]
            [猫猫]
            力对质点在一段路程 \(L\) 上做功等于其动能变化量: 
            \[\Delta E_k|_{x\in L}=W_L\]
        \end{thm}
        
        \begin{dfn}
            []
            {保守力}
            [Conservative Force]
            [猫猫]
            保守力是指力的做功只与物体的初末位置有关, 与路径无关的力. 
        \end{dfn}
    
    \subsection{内力做功}
        

\section{动量与角动量}

    \subsection{动量与冲量}
        
        \begin{dfn}
            []
            {动量}
            [Momentum]
            [猫猫]
            \[\bm{p}=m\bm{v}\]
        \end{dfn}
        
        \begin{dfn}
            []
            {冲量}
            [Impulse]
            [猫猫]
            \[\bm{I}=\int_T\bm{F}\dd t\]
        \end{dfn}
        
        \begin{thm}
            []
            {动量守恒定理}
            [Conservation of Momentum]
            [猫猫]
        \end{thm}
    
    \subsection{碰撞问题}
        
        \begin{dfn}
            []
            {恢复系数}
            [Coefficient of Restitution]
            [猫猫]
            \[e=\frac{\bm{v}_2'-\bm{v}_1'}{\bm{v}_1-\bm{v}_2}\]

            \(e=0\) 时称为\textbf{完全非弹性碰撞}, \(e=1\) 时称为\textbf{完全弹性碰撞}, \(0<e<1\) 时称为\textbf{非弹性碰撞}. 
        \end{dfn}

    \subsection{力矩, 冲量矩与角动量}
        
        \begin{dfn}
            []
            {力矩}
            [Torque]
            [猫猫]
            力对质点关于位矢 \(\bm{r}_0\) 的\textbf{力矩}定义为: 
            \[\bm{M}=(\bm{r}-\bm{r}_0)\times\bm{F}\]
        \end{dfn}
        
        \begin{dfn}
            []
            {角动量}
            [Angular Momentum]
            [猫猫]
            质点关于位矢 \(\bm{r}_0\) 的\textbf{角动量}定义为: 
            \[\bm{L}=(\bm{r}-\bm{r}_0)\times\bm{p}\]
        \end{dfn}
        
        \begin{thm}
            []
            {角动量守恒定理}
            [Conservation of Angular Momentum]
            [猫猫]
            在无外力矩作用下, 质点的角动量守恒: 
            \[\bm{L}_1=\bm{L}_2\]
        \end{thm}

\section{刚体力学}
    
        \subsection{刚体的定义与性质}
            
            \begin{dfn}
                []
                {刚体}
                [Rigid Body]
                [猫猫]
                各质点间的相对位置不发生变化的质点系称为\textbf{刚体}: 
                \[\text{Rigid}:X\mapsto\forall t_1, t_2\in T, 
                \begin{cases}
                    \forall x_1,x_2\in X, d(x_1(t_1),x_2(t_1))=d(x_1(t_2), x_2(t_2))\\
                    \forall x_1,x_2,x_3,x_4\in X, \det(D)(t_1)\cdot\det(D)(t_2)\geq 0
                \end{cases}\]
                \[(D:=[x_4.\bm{r}-x_1.\bm{r}, x_3.\bm{r}-x_1.\bm{r}, x_2.\bm{r}-x_1.\bm{r}])\] 
            \end{dfn}
            
            \begin{dfn}
                []
                {角速度}
                [Angular Velocity]
                [猫猫]
                \[\bm{\omega}=\frac{\dd\varphi}{\dd t}\cdot\ihat\]
            \end{dfn}
            
            \begin{dfn}
                []
                {角加速度}
                [Angular Acceleration]
                [猫猫]
                \[\bm{\beta}=\frac{\dd\bm{\omega}}{\dd t}\]
            \end{dfn}
            
            \begin{dfn}
                []
                {转动惯量}
                [Moment of Inertia]
                [猫猫]
                \[J=\sum_{x\in X}\Delta m_x r_x^=2=\int_{x\in X}r_x^2\dd m_x\]
            \end{dfn}
            
            \begin{xmp}
                {均质球体关于直径的转动惯量}
                \[J=\frac{2}{5}mR^2\]
            \end{xmp}
            
            \begin{xmp}
                {均质滑轮关于轴线的转动惯量}
                \[J=\frac{1}{2}mR^2\]
            \end{xmp}
            
            \begin{xmp}
                {均质圆环关于轴线的转动惯量}
                \[J=\frac{1}{2}mR^2\]
            \end{xmp}
            
            \begin{xmp}
                {均质杆关于端点的转动惯量}
                \[J=\frac{1}{3}mL^2\]
            \end{xmp}
            
            \begin{dfn}
                {转动动能 (Rotational Kinetic Energy)}
            \end{dfn}
            
            \begin{thm}
                {转动动能计算公式}
                \[E_{k,\text{rot}}=\frac{1}{2}J\bm{\omega}^2\]
            \end{thm}
            
            \begin{thm}
                {刚体定轴转动动能定理}
                刚体合外力矩做功等于转动动能变化量
            \end{thm}

    \section{机械振动}

        \subsection{简谐振动}
            
            \begin{dfn}
                []
                {简谐振动}
                [Simple Harmonic Motion]
                [猫猫]
                称质点作\textbf{简谐运动}, 若在\textbf{回复力系数} \(k\) 下, 其受力情况关于某一位矢 \(\bm{r}_0\) 满足: 
                \[\bm{F}=-k(\bm{r}-\bm{r}_0)\]
            \end{dfn}
            
            \begin{thm}
                {简谐振动的位移函数是正弦函数}
                若质点作简谐振动, 则可用正弦函数表达: 
                \[x=A\sin(\omega t+\varphi)\]
            \end{thm}
            
            \begin{dfn}
                {角频率 / 圆频率}
                简谐振动的\textbf{角频率}定义为上式中的 \(\omega\). 
            \end{dfn}
            
            \begin{dfn}
                {固有圆频率}
                简谐振动的\textbf{固有圆频率}定义为: 
                \[\omega_0:=\sqrt{\frac{k}{m}}\]
            \end{dfn}
            
            \begin{dfn}
                {周期}
                简谐振动的\textbf{周期}定义为位移函数的一个完整周期, 记为 \(T\). 
                \[T:=\frac{2\pi}{\omega}\]
            \end{dfn}
            
            \begin{dfn}
                {频率}
                简谐振动的\textbf{频率}定义为周期的倒数, 记为 \(f\). 
                \[f:=\frac{1}{T}=\frac{\omega}{2\pi}\]
            \end{dfn}
            
            \begin{dfn}
                {振幅}
                简谐振动的\textbf{振幅}定义为位移函数的最大值, 记为 \(A\). 
            \end{dfn}
            
            \begin{dfn}
                {相位}
                简谐振动的\textbf{相位}函数, 记为 \(\Phi\), 定义为: 
                \[\Phi(t):=\omega t+\varphi\]
            \end{dfn}
            
            \begin{dfn}
                {初相位}
                简谐振动的\textbf{初相位}定义为 \(\Phi(0)\), 记作 \(\varphi\). 
            \end{dfn}
            
            \begin{ppt}
                {初相位由初始运动状态决定}
                \[\tan\varphi=-\frac{v_0}{\omega x_0}\]
            \end{ppt}

        \subsection{阻尼振动}
            
            \begin{dfn}
                {线性阻尼}
                设取值为正的物理量 \(\gamma\), 在机械振动中, 满足以下条件的力称为阻尼: 
                \[\bm{f}:=-\gamma\bm{v}\]
            \end{dfn}
            
            \begin{dfn}
                {阻尼系数}
                线性阻尼的\textbf{阻尼系数}定义为 \(\gamma\) 与质量之比的一半, 记作 \(\beta\): 
                \[\beta:=\frac{\gamma}{2m}\]
            \end{dfn}
            
            \begin{thm}
                {阻尼振动}
                称物体作阻尼振动, 若满足: 
                \[\bm{a}+2\beta\bm{v}+\omega_0^2\bm{x}=\bm{0}\]
            \end{thm}
            
            \begin{thm}
                {速率衰减公式}
                \[v=v_0 e^{-2\beta t}\]
            \end{thm}
            
            \begin{dfn}
                {临界阻尼条件}
                当 \(\beta=\omega_0\) 时, 称为\textbf{临界阻尼条件}. 

                若 \(\beta<\omega_0\), 则称为\textbf{弱阻尼条件}. 

                若 \(\beta>\omega_0\), 则称为\textbf{过阻尼条件}. 
            \end{dfn}
            
            \begin{thm}
                {弱阻尼条件下阻尼振动的运动学方程}
                \[x=A_0e^{-\beta t}\cos(\sqrt{\omega_0^2-\beta^2}t+\varphi_0)\]
            \end{thm}
            
            \begin{thm}
                {动能衰减公式}
                \[E_k=\frac{1}{2}mv_0^2 e^{-4\beta t}\]
            \end{thm}

        \subsection{单摆}
    
\end{document}