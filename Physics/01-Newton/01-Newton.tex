\documentclass[UTF8]{ctexart}

\makeatletter
\def\input@path{{Fulcrum-Template/}{Fulcrum-Template/OperatorList/}}
\makeatother

% ams package
\usepackage{amsfonts}
\usepackage{amssymb}
\usepackage{amsthm}
\usepackage{amsmath}

% margin
\usepackage{geometry}

% \dd
\usepackage{physics}

% Boldface
\usepackage{bm}

% Tikz
\usepackage{tikz}
\usetikzlibrary{calc}

% Gaussian Elimination
\usepackage{gauss}

% Commutative Graph
\usepackage[all]{xy}

% Comment
\usepackage{comment}

\title{Title}
\author{Fulcrum4Math}
\date{\today}

% General
\DeclareMathOperator{\N}{\mathbb{N}}                    % Set of Natural Numbers
\DeclareMathOperator{\Z}{\mathbb{Z}}                    % Set of Integers
\DeclareMathOperator{\Q}{\mathbb{Q}}                    % Set of Rational Numbers
\DeclareMathOperator{\R}{\mathbb{R}}                    % Set of Real Numbers
\DeclareMathOperator{\C}{\mathbb{C}}                    % Set of Complex Numbers

\DeclareMathOperator{\Id}{Id}                           % Identity

\DeclareMathOperator{\Ker}{Ker}                         % Kernel of a Homomorphism
\DeclareMathOperator{\Image}{Im}                        % Image of a mapping

% Set Theory
\DeclareMathOperator{\PP}{\mathcal{P}}                  % Power Sets
\DeclareMathOperator{\card}{card}                       % Cardinality

% Category Theory
\DeclareMathOperator{\Cat}{\mathcal{C}}                 % Category

\DeclareMathOperator{\Hom}{Hom}                         % Set of Homomorphisms
\DeclareMathOperator{\End}{End}                         % Set of Endomorphisms
\DeclareMathOperator{\Aut}{Aut}                         % Set of Automorphisms
\DeclareMathOperator{\Isom}{Isom}                       % Set of Isomorphisms

% Topology
\DeclareMathOperator{\T}{\mathcal{T}}                   % Topology

\DeclareMathOperator{\intr}{int}                        % Interior
\DeclareMathOperator{\cl}{cl}                           % Closure

\DeclareMathOperator{\U}{\overset{\circ}{\mathit{U}}}   % Deleted Neighbourhood

% Linear Algebra
\DeclareMathOperator{\K}{\mathbb{K}}                    % Number Field
\DeclareMathOperator{\F}{\mathbb{F}}                    % Number Field (F)

\DeclareMathOperator{\al}{\bm\alpha}                    % Boldfaced vector alpha
\DeclareMathOperator{\bt}{\bm\beta}                     % Boldfaced vector beta
\DeclareMathOperator{\x}{\bm{x}}                        % Boldfaced vector x

% \DeclareMathOperator{\A}{\bm{A}}                    % Boldfaced matrix A
% \DeclareMathOperator{\B}{\bm{B}}                    % Boldfaced matrix B
% \DeclareMathOperator{\Cc}{\bm{C}}                   % Boldfaced matrix C

\DeclareMathOperator{\CCol}{Col}                        % Column Space
\DeclareMathOperator{\RRow}{Row}                        % Row Space
\DeclareMathOperator{\Null}{Null}                       % Null Space

\renewcommand{\span}{\mathrm{span}\text{ }}             % Span

\DeclareMathOperator{\diag}{diag}                       % Diagonal Matrix

\newcommand{\<}{\langle}                                
\renewcommand{\>}{\rangle}                              % These two for ordinary Hilbert Inner Products <x,y>
\newcommand{\inprod}[2]{\<#1,#2\>}
\newcommand{\ocinterval}[2]{(#1,#2]}
\newcommand{\cointerval}[2]{[#1,#2)}
\newcommand{\ccinterval}[2]{[#1,#2]}
\newcommand{\oointerval}[2]{(#1,#2)}

% Mathematical Analysis

\DeclareMathOperator*{\ulim}{\overline{\lim}}
\DeclareMathOperator*{\llim}{\underline{\lim}}
\newcommand{\diff}[3]{\left. #1 \right|_{#2}^{#3}}

\newcommand{\Ball}[2]{\mathcal{B}\left(#1,#2\right)}	% Open Ball

% Theorem template below copied from https://zhuanlan.zhihu.com/p/763738880

% ————————————————————————————————————自定义颜色————————————————————————————————————
\definecolor{dfn_green1}{RGB}{0, 156, 39} % 深绿
\definecolor{dfn_green2}{RGB}{214, 254, 224} % 浅绿

\definecolor{thm_blue1}{RGB}{0, 91, 156} % 深蓝
\definecolor{thm_blue2}{RGB}{218, 240, 255} % 浅蓝

\definecolor{ppt_pink1}{RGB}{172, 0, 175} % 深粉
\definecolor{ppt_pink2}{RGB}{255, 237, 255} % 浅粉

\definecolor{crl_orange1}{RGB}{225, 124, 0} % 深橙
\definecolor{crl_orange2}{RGB}{255, 235, 210} % 浅橙

\definecolor{xmp_purple1}{RGB}{119, 0, 229} % 深紫
\definecolor{xmp_purple2}{RGB}{239, 223, 255} % 浅紫

\definecolor{cxmp_red1}{RGB}{211, 0, 35} % 深红
\definecolor{cxmp_red2}{RGB}{255, 214, 220} % 浅红

\definecolor{prf_grey1}{RGB}{120, 120, 120} % 深灰
\definecolor{prf_grey2}{RGB}{233, 233, 233} % 浅灰

\definecolor{axm_yellow1}{RGB}{192, 192, 0} % 深黄
\definecolor{axm_yellow2}{RGB}{255, 255, 172} % 浅黄

% 将RGB换为rgb,颜色数值取值范围改为0到1
% ————————————————————————————————————自定义颜色————————————————————————————————————

% ————————————————————————————————————盒子设置————————————————————————————————————

\usepackage{tcolorbox} % 盒子效果
\tcbuselibrary{most} % tcolorbox宏包的设置,详见宏包说明文档

% tolorbox提供了tcolorbox环境,其格式如下:
% 第一种格式:\begin{tcolorbox}[colback=⟨背景色⟩, colframe=⟨框线色⟩, arc=⟨转角弧度半径⟩, boxrule=⟨框线粗⟩]   \end{tcolorbox}
% 其中设置arc=0mm可得到直角;boxrule可换为toprule/bottomrule/leftrule/rightrule可分别设置对应边宽度,但是设置为0mm时仍有细边,若要绘制单边框线推荐使用第二种格式
% 方括号内加上title=⟨标题⟩, titlerule=⟨标题背景线粗⟩, colbacktitle=⟨标题背景线色⟩可为盒子加上标题及其背景线
% 第二种格式:\begin{tcolorbox}[enhanced, colback=⟨背景色⟩, boxrule=0pt, frame hidden, borderline={⟨框线粗⟩}{⟨偏移量⟩}{⟨框线色⟩}]   {\end{tcolorbox}}
% 将borderline换为borderline east/borderline west/borderline north/borderline south可分别为四边添加框线,同一边可以添加多条
% 加入breakable属性可以支持盒子拆分到两页中。
% 偏移量为正值时,框线向盒子内部移动相应距离,负值反之

\newenvironment{dfn_box}{
    \begin{tcolorbox}[enhanced, colback=dfn_green2, boxrule=0pt, frame hidden,
        borderline west={0.7mm}{0.1mm}{dfn_green1},breakable]
    }
    {\end{tcolorbox}}
    
\newenvironment{axm_box}{
    \begin{tcolorbox}[enhanced, colback=axm_yellow2, boxrule=0pt, frame hidden,
        borderline west={0.7mm}{0.1mm}{axm_yellow1},breakable]
    }
    {\end{tcolorbox}}
    
\newenvironment{thm_box}{
    \begin{tcolorbox}[enhanced, colback=thm_blue2, boxrule=0pt, frame hidden,
        borderline west={0.7mm}{0.1mm}{thm_blue1},breakable]
    }
    {\end{tcolorbox}}
    
\newenvironment{ppt_box}{
    \begin{tcolorbox}[enhanced, colback=ppt_pink2, boxrule=0pt, frame hidden,
        borderline west={0.7mm}{0.1mm}{ppt_pink1},breakable]
    }
    {\end{tcolorbox}}
    
\newenvironment{xmp_box}{
    \begin{tcolorbox}[enhanced, colback=xmp_purple2, boxrule=0pt, frame hidden,
        borderline west={0.7mm}{0.1mm}{xmp_purple1},breakable]
    }
    {\end{tcolorbox}}
    
\newenvironment{prf_box}{
    \begin{tcolorbox}[enhanced, colback=prf_grey2, boxrule=0pt, frame hidden,
        borderline west={0.7mm}{0.1mm}{prf_grey1},breakable]
    }
    {\end{tcolorbox}}

% tcolorbox宏包还提供了\tcbox指令,用于生成行内盒子,可制作高光效果

        % \newcommand{\hl}[1]{
        %     \tcbox[on line, arc=0pt, colback=hlan!5!white, colframe=hlan!5!white, boxsep=1pt, left=1pt, right=1pt, top=1.5pt, bottom=1.5pt, boxrule=0pt]
        % {\bfseries \color{hlan}#1}}
        
% 其中on line将盒子放置在本行(缺失会跳到下一行),boxsep用于控制文本内容和边框的距离,left、right、top、bottom则分别在boxsep的参数的基础上分别控制四边距离
% ————————————————————————————————————盒子设置————————————————————————————————————

% ————————————————————————————————————定理类环境设置————————————————————————————————————
\newtheoremstyle{MyStyle}{0pt}{}{}{\parindent}{\bfseries}{}{1em}{} % 定义新定理风格。格式如下:
%\newtheoremstyle{⟨风格名⟩}
%                {⟨上方间距⟩} % 若留空,则使用默认值
%                {⟨下方间距⟩} % 若留空,则使用默认值
%                {⟨主体字体⟩} % 如 \itshape
%                {⟨缩进长度⟩} % 若留空,则无缩进;可以使用 \parindent 进行正常段落缩进
%                {⟨定理头字体⟩} % 如 \bfseries
%                {⟨定理头后的标点符号⟩} % 如点号、冒号
%                {⟨定理头后的间距⟩} % 不可留空,若设置为 { },则表示正常词间间距;若设置为 {\newline},则环境内容开启新行
%                {⟨定理头格式指定⟩} % 一般留空
% 定理风格决定着由 \newtheorem 定义的环境的具体格式,有三种定理风格是预定义的,它们分别是:
% plain: 环境内容使用意大利斜体,环境上下方添加额外间距
% definition: 环境内容使用罗马正体,环境上下方添加额外间距
% remark: 环境内容使用罗马正体,环境上下方不添加额外间距
\theoremstyle{MyStyle} % 设置定理风格 

% 定义定义环境,格式为\newtheorem{⟨环境名⟩}{⟨定理头文本⟩}[⟨上级计数器⟩]或\newtheorem{⟨环境名⟩}[⟨共享计数器⟩]{⟨定理头文本⟩},其变体\newtheorem*不带编号

\newtheorem{definition}{概念}[subsection]
\newenvironment{cpt}{\begin{dfn_box}\begin{definition}}{\end{definition}\end{dfn_box}}

\newtheorem{axm}[definition]{定律}
\newenvironment{thr}{\begin{axm_box}\begin{axm}}{\end{axm}\end{axm_box}}

\newtheorem{theorem}[definition]{定理}
\newenvironment{thm}{\begin{thm_box}\begin{theorem}}{\end{theorem}\end{thm_box}}

\newtheorem{property}{性质}[definition]
\newenvironment{ppt}{\begin{ppt_box}\begin{property}}{\end{property}\end{ppt_box}}

\newtheorem{example}{例}[definition]
\newenvironment{xmp}{\begin{xmp_box}\begin{example}}{\end{example}\end{xmp_box}}

\newtheorem*{myproof}{推导: \newline}
\newenvironment{prf}{\begin{prf_box}\begin{myproof}}{\end{myproof}\end{prf_box}}

% ————————————————————————————————————定理类环境设置————————————————————————————————————
    % \newtheorem{xmp}{例}[subsection]
    % \newtheorem{thm}{定理}[subsection]
    % \newtheorem{crl}{推论}[thm]
    % \newtheorem{dfn}[thm]{定义}
    % \newtheorem{ppt}{性质}[thm]
    % \newtheorem{lma}[thm]{引理}
    % \newtheorem{axm}{公理}
    % \newtheorem{pbm}{题}
    % \newtheorem*{prf}{证明}
    % \newtheorem*{ans}{解答}

    % \newtheorem{dfn}{Definition}
    % \newtheorem{thm}{Theorem}
    % \newtheorem{lma}{Lemma}
    % \newtheorem{axm}{Axiom}
    % \newtheorem{pbm}{Problem}
    % \newtheorem*{prf}{Proof}
    % \newtheorem*{ans}{Answer}
    % English Version
% ------------------******-------------------

\geometry{
    paper =a4paper,
    top =3cm,
    bottom =3cm,
    left=2cm,
    right =2cm
}

\linespread{1.2}

\DeclareMathOperator{\kg}{\mathrm{kg}}
\DeclareMathOperator{\m}{\mathrm{m}}
\DeclareMathOperator{\s}{\mathrm{s}}
\DeclareMathOperator{\J}{\mathrm{J}}
\DeclareMathOperator{\New}{\mathrm{N}}
\DeclareMathOperator{\mol}{\mathrm{mol}}
\DeclareMathOperator{\Pa}{\mathrm{Pa}}
\DeclareMathOperator{\Ltr}{\mathrm{L}}
\DeclareMathOperator{\Kv}{\mathrm{K}}

\DeclareMathOperator{\ihat}{\bm{\hat{\imath}}}
\DeclareMathOperator{\jhat}{\bm{\hat{\jmath}}}
\DeclareMathOperator{\khat}{\bm{\hat{\mathit{k}}}}

\begin{document}


\begin{center}
    {\LARGE 经典力学笔记}
\end{center}

\section{质点运动学}

    \subsection{质点运动的描述}
        
        \begin{cpt}
            \textbf{参考系 (Reference Frame)}

            用来描述物体运动而选作为"静止"的参考标准. 
        
            惯性系
        
            坐标系
        
                通过 Cartesian 积坐标来对物理空间建立的模型. 
        
                直角坐标系
        
                    $$P(x,y,z) : X\times Y\times Z$$
        
                    位置矢量/矢径: 由原点指向
        
                极坐标系
        
                    $$(r,\theta)$$
        
                    径向单位矢量 $\times$ 横向单位矢量
                
                球坐标系
        
                    $$(r,\theta,\varphi) : $$
                    其中 $r$ 为半径, $\theta$ 为平面角, $\varphi$ 为高度角. 
        
                柱坐标系
        
                    $$(\rho,\theta,z) : $$
                    其中 $(\rho,\theta)$ 构成二维平面上的极坐标系. 
        
                自然坐标系
        \end{cpt}
        
        \begin{cpt}
            \textbf{位矢 (Position Vector)}

            某一参考系的元素, 记为 \(\bm{r}\). 

            一段连续的时间间隔 \(T\) 内位矢变化量称为\textbf{位移 (Displacement)}, 记为 \(\Delta\bm{r}_T\). 
        \end{cpt}
        
        \begin{cpt}
            \textbf{质量 (Mass)}

            记为 \(m\). 
        \end{cpt}
        
        \begin{cpt}
            \textbf{质点 (Mass Point)}

            质点是在时间参量下由质量和位矢构造而成的物理对象. 
            \[P=t\mapsto(m,\bm{r})\]
        \end{cpt}
        
        \begin{cpt}
            \textbf{平均速度 (Average Velocity)}

            质点在一段连续的时间间隔 \(T\) 内的\textbf{平均速度}定义为位移与时间间隔长度之比, 记为 \(\bar{\bm{v}}_T\): 
            \[\bar{\bm{v}}_T:=\frac{\Delta\bm{r}}{T}\]
        \end{cpt}
        
        \begin{cpt}
            \textbf{瞬时速度 (Instantaneous Velocity)}
            \[\bm{v}=\lim_{\Delta t\to 0+}\frac{\Delta\bm{r}}{\Delta t}=\frac{\dd}{\dd t}\bm{r}\]

            * \textit{常简称\textbf{速度}, 一般情况下不特别指明``平均''即为瞬时速度. }
        \end{cpt}
        
        \begin{cpt}
            \textbf{加速度 (Acceleration)}
            \[\bm{a}=\lim_{\Delta t\to 0+}\frac{\Delta\bm{v}}{\Delta t}=\frac{\dd}{\dd t}\bm{v}\]
            \[\Delta\bm{v}=\int_T \bm{v}\dd t\]

            * \textit{常简称\textbf{加速度}, 一般情况下不特别指明``平均''即为瞬时加速度. }
        \end{cpt}

    \subsection{牛顿运动定律}
        
        \begin{cpt}
            \textbf{力 (Force)}

            质点对另一质点施加作用, 使其改变运动状态的物理量称为\textbf{力}, 记为 \(\bm{F}\). 
        \end{cpt}
        
        \begin{cpt}
            \textbf{惯性系 (Inertial Frame) 与 Newton 第一定律 (Newton's First Law)}

            质点在不受任何外力作用时, 其运动状态保持不变的参考系称为\textbf{惯性系}. 

            上述性质也称为\textbf{Newton 第一定律}. 
        \end{cpt}
        
        \begin{thr}
            \textbf{Newton 第二定律 (Newton's Second Law)}
            \[\bm{F}=m\bm{a}\]
        \end{thr}
        
        \begin{thr}
            \textbf{Newton 第三定律 (Newton's Third Law)}
            \[\bm{F}_{12}=-\bm{F}_{21}\]
        \end{thr}
        
        \begin{thr}
            \textbf{Galileo 变换 (Galilean Transformation) 与力学相对性原理}

            将一惯性系映射到另一惯性系的变换称为\textbf{Galileo 变换}. 

            一切惯性系在力学上等价. 
        \end{thr}

    \subsection{常见力}
        
        \begin{thr}
            \textbf{万有引力 (Gravitational Force)}
            \[\bm{F}=-\frac{GMm}{r^3}\cdot\bm{r}\]
            \[\|\bm{F}\|=\frac{GMm}{r^2}\]
        \end{thr}
        
        \begin{thr}
            \textbf{弹力 (Elastic Force)}
            \[\bm{F}-k\bm{x}\]
        \end{thr}
        
        \begin{thr}
            \textbf{摩擦力 (Frictional Force)}
            \[\|\bm{f}\|=\mu N\]
        \end{thr}

    \subsection{非惯性系}

    \subsection{质点系运动学}
        
        \begin{cpt}
            \textbf{质点系 (Mass System)}

            同一参考系下质点的集合称为\textbf{质点系}. 
        \end{cpt}
        
        \begin{cpt}
            \textbf{质心 (Center of Mass)}

            离散质点系\(X\)的\textbf{质心}定义为如下质点: 
            \[m_C=\sum_{x\in X}m_X\]
            \[\bm{r}_C=\frac{\sum\limits_{x\in X}m_x\bm{r}_x}{m_C}\]

            连续质点系\(X\)的\textbf{质心}定义为如下质点:
            \[m_C=\int_X\dd m\]
            \[\bm{r}_C=\frac{\int_X\bm{r}\dd m}{m_C}\]
        \end{cpt}
        
        \begin{thm}
            \textbf{质心运动定理}

            质心运动状态只与外力有关, 且符合牛顿第二定律: 
            \[\bm{F}=m_C\bm{a}_C\]
        \end{thm}

\section{功与能}

    \subsection{功与动能定理}
        
        \begin{cpt}
            \textbf{功 (Work)}

            力对质点在一段路程 \(L\) 上做的\textbf{功}定义为: 
            \[W=\int_L\bm{F}\cdot\dd\bm{r}\]
        \end{cpt}
        
        \begin{cpt}
            \textbf{功率 (Power)}

            \[P=\frac{\dd W}{\dd t}=\bm{F}\cdot\bm{v}\]
        \end{cpt}
        
        \begin{cpt}
            \textbf{动能 (Kinetic Energy)}

            质点的\textbf{动能}定义为: 
            \[E_k=\frac{1}{2}m\bm{v}^2\]

            动能与参考系有关. 
        \end{cpt}
        
        \begin{thm}
            \textbf{动能定理 (Work-Energy Theorem)}

            力对质点在一段路程 \(L\) 上做功等于其动能变化量: 
            \[\Delta E_k|_{x\in L}=W_L\]
        \end{thm}
        
        \begin{cpt}
            保守力
        \end{cpt}
    
    \subsection{内力做功}
        

\section{动量与角动量}

    \subsection{动量与冲量}
        
        \begin{cpt}
            \textbf{动量 (Momentum)}
            \[\bm{p}=m\bm{v}\]
        \end{cpt}
        
        \begin{cpt}
            \textbf{冲量 (Impulse)}
            \[\bm{I}=\int_T\bm{F}\dd t\]
        \end{cpt}
        
        \begin{thm}
            \textbf{动量守恒定理 (Conservation of Momentum)}
        \end{thm}
    
    \subsection{碰撞问题}
        
        \begin{cpt}
            \textbf{恢复系数 (Coefficient of Restitution)}
            \[e=\frac{\bm{v}_2'-\bm{v}_1'}{\bm{v}_1-\bm{v}_2}\]

            \(e=0\) 时称为\textbf{完全非弹性碰撞}, \(e=1\) 时称为\textbf{完全弹性碰撞}, \(0<e<1\) 时称为\textbf{非弹性碰撞}. 
        \end{cpt}

    \subsection{力矩, 冲量矩与角动量}
        
        \begin{cpt}
            \textbf{力矩 (Torque)}

            力对质点关于位矢 \(\bm{r}_0\) 的\textbf{力矩}定义为: 
            \[\bm{M}=(\bm{r}-\bm{r}_0)\times\bm{F}\]
        \end{cpt}
        
        \begin{cpt}
            \textbf{角动量 (Angular Momentum)}

            质点关于位矢 \(\bm{r}_0\) 的\textbf{角动量}定义为: 
            \[\bm{L}=(\bm{r}-\bm{r}_0)\times\bm{p}\]
        \end{cpt}
        
        \begin{thm}
            \textbf{角动量守恒定理 (Conservation of Angular Momentum)}

            在无外力矩作用下, 质点的角动量守恒: 
            \[\bm{L}_1=\bm{L}_2\]
        \end{thm}

\section{刚体力学}
    
        \subsection{刚体的定义与性质}
            
            \begin{cpt}
                \textbf{刚体 (Rigid Body)}
    
                各质点间的相对位置不发生变化的质点系称为\textbf{刚体}: 
                \[\text{Rigid}:X\mapsto\forall t_1, t_2\in T, 
                \begin{cases}
                    \forall x_1,x_2\in X, d(x_1(t_1),x_2(t_1))=d(x_1(t_2), x_2(t_2))\\
                    \forall x_1,x_2,x_3,x_4\in X, \det(D)(t_1)\cdot\det(D)(t_2)\geq 0
                \end{cases}\]
                \[(D:=[x_4.\bm{r}-x_1.\bm{r}, x_3.\bm{r}-x_1.\bm{r}, x_2.\bm{r}-x_1.\bm{r}])\] 
            \end{cpt}
            
            \begin{cpt}
                \textbf{角速度 (Angular Velocity)}

                \[\bm{\omega}=\frac{\dd\varphi}{\dd t}\cdot\ihat\]
            \end{cpt}
            
            \begin{cpt}
                \textbf{角加速度 (Angular Acceleration)}
                \[\bm{\beta}=\frac{\dd\bm{\omega}}{\dd t}\]
            \end{cpt}
            
            \begin{cpt}
                \textbf{转动惯量}
                \[J=\sum_{x\in X}\Delta m_x r_x^=2=\int_{x\in X}r_x^2\dd m_x\]
            \end{cpt}
            
            \begin{xmp}
                \textbf{均质球体关于直径的转动惯量}
                \[J=\frac{2}{5}mR^2\]
            \end{xmp}
            
            \begin{xmp}
                \textbf{均质滑轮关于轴线的转动惯量}
                \[J=\frac{1}{2}mR^2\]
            \end{xmp}
            
            \begin{xmp}
                \textbf{均质圆环关于轴线的转动惯量}
                \[J=\frac{1}{2}mR^2\]
            \end{xmp}
            
            \begin{xmp}
                \textbf{均质杆关于端点的转动惯量}
                \[J=\frac{1}{3}mL^2\]
            \end{xmp}
            
            \begin{cpt}
                \textbf{转动动能 (Rotational Kinetic Energy)}
            \end{cpt}
            
            \begin{thm}
                \textbf{转动动能计算公式}
                \[E_{k,\text{rot}}=\frac{1}{2}J\bm{\omega}^2\]
            \end{thm}
            
            \begin{thm}
                \textbf{刚体定轴转动动能定理}

                刚体合外力矩做功等于转动动能变化量
            \end{thm}

    \section{机械振动}

        \subsection{简谐振动}
            
            \begin{cpt}
                \textbf{简谐振动 (Simple Harmonic Motion)}

                称质点作\textbf{简谐运动}, 若在\textbf{回复力系数} \(k\) 下, 其受力情况关于某一位矢 \(\bm{r}_0\) 满足: 
                \[\bm{F}=-k(\bm{r}-\bm{r}_0)\]
            \end{cpt}
            
            \begin{thm}
                \textbf{简谐振动的位移函数是正弦函数}

                若质点作简谐振动, 则可用正弦函数表达: 
                \[x=A\sin(\omega t+\varphi)\]
            \end{thm}
            
            \begin{cpt}
                \textbf{角频率 / 圆频率}

                简谐振动的\textbf{角频率}定义为上式中的 \(\omega\). 
            \end{cpt}
            
            \begin{cpt}
                \textbf{固有圆频率}

                简谐振动的\textbf{固有圆频率}定义为: 
                \[\omega_0:=\sqrt{\frac{k}{m}}\]
            \end{cpt}
            
            \begin{cpt}
                \textbf{周期}

                简谐振动的\textbf{周期}定义为位移函数的一个完整周期, 记为 \(T\). 
                \[T:=\frac{2\pi}{\omega}\]
            \end{cpt}
            
            \begin{cpt}
                \textbf{频率}

                简谐振动的\textbf{频率}定义为周期的倒数, 记为 \(f\). 
                \[f:=\frac{1}{T}=\frac{\omega}{2\pi}\]
            \end{cpt}
            
            \begin{cpt}
                \textbf{振幅}

                简谐振动的\textbf{振幅}定义为位移函数的最大值, 记为 \(A\). 
            \end{cpt}
            
            \begin{cpt}
                \textbf{相位}

                简谐振动的\textbf{相位}函数, 记为 \(\Phi\), 定义为: 
                \[\Phi(t):=\omega t+\varphi\]
            \end{cpt}
            
            \begin{cpt}
                \textbf{初相位}

                简谐振动的\textbf{初相位}定义为 \(\Phi(0)\), 记作 \(\varphi\). 
            \end{cpt}
            
            \begin{ppt}
                \textbf{初相位由初始运动状态决定}
                \[\tan\varphi=-\frac{v_0}{\omega x_0}\]
            \end{ppt}

        \subsection{阻尼振动}
            
            \begin{cpt}
                \textbf{线性阻尼}

                设取值为正的物理量 \(\gamma\), 在机械振动中, 满足以下条件的力称为阻尼: 
                \[\bm{f}:=-\gamma\bm{v}\]
            \end{cpt}
            
            \begin{cpt}
                \textbf{阻尼系数}

                线性阻尼的\textbf{阻尼系数}定义为 \(\gamma\) 与质量之比的一半, 记作 \(\beta\): 
                \[\beta:=\frac{\gamma}{2m}\]
            \end{cpt}
            
            \begin{thm}
                \textbf{阻尼振动}

                称物体作阻尼振动, 若满足: 
                \[\bm{a}+2\beta\bm{v}+\omega_0^2\bm{x}=\bm{0}\]
            \end{thm}
            
            \begin{thm}
                \textbf{速率衰减公式}
                \[v=v_0 e^{-2\beta t}\]
            \end{thm}
            
            \begin{cpt}
                \textbf{临界阻尼条件}
                当 \(\beta=\omega_0\) 时, 称为\textbf{临界阻尼条件}. 

                若 \(\beta<\omega_0\), 则称为\textbf{弱阻尼条件}. 

                若 \(\beta>\omega_0\), 则称为\textbf{过阻尼条件}. 
            \end{cpt}
            
            \begin{thm}
                \textbf{弱阻尼条件下阻尼振动的运动学方程}
                \[x=A_0e^{-\beta t}\cos(\sqrt{\omega_0^2-\beta^2}t+\varphi_0)\]
            \end{thm}
            
            \begin{thm}
                \textbf{动能衰减公式}
                \[E_k=\frac{1}{2}mv_0^2 e^{-4\beta t}\]
            \end{thm}

        \subsection{单摆}
    
\end{document}