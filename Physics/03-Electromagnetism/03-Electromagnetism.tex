\documentclass[UTF8]{ctexart}

\makeatletter
\def\input@path{{../../Fulcrum-Template/}{../../Operator-List/}}
\makeatother

\usepackage{FulcrumCN}

\usepackage{OperatorListCN}
\usepackage{F4Logic}
\usepackage{F4Set}
\usepackage{F4Topology}
\usepackage{F4Analysis}
\usepackage{F4Physics}

% margin
\usepackage{geometry}
\geometry{
    paper =a4paper,
    top =3cm,
    bottom =3cm,
    left=2cm,
    right =2cm
}
\linespread{1.2}

\begin{document}

\tableofcontents
\newpage

\section{经典电磁学}

    \subsection{静电场}

        \begin{dfn}
            [Electric-Charge]
            {电荷}
            [Electric Charge]
            [猫猫]
            定义\textbf{电荷}为实物理量, 量纲为 \(\ElectricCurrent\cdot\Time\). 

            定义\电荷 的量纲 \textbf{Column} 为 \(\A\cdot\s\), 记作 \(\Clm\). 
        \end{dfn}
        
        \begin{str}
            [Charged-Object]
            {带电体}
            [Charged Object]
            [猫猫]
            定义
        \end{str}
        
        \begin{xmp}
            []
            {电子电量}
            []
            [猫猫]
            \textbf{电子电量}约为 \(-1.602 \times 10^{-19} \Clm\), 记作 \(e\). 
        \end{xmp}
        
        \begin{dfn}
            [Coulomb-Force]
            {静电力 / Coulomb 力}
            [Coulomb Force]
            [猫猫]
        \end{dfn}
        
        \begin{str}
            [Electric-Field]
            {电场}
            [Electric Field]
            [猫猫]
            一种场. 
        \end{str}
        
        \begin{dfn}
            [Electric-Potential]
            {电势}
            []
            []
            设 \(E\) 是电场, \(x_0\) 是 \(E\) 中一点, \(x\) 是 \(E\) 中一点, 定义 \(E\) 在 \(x\) 处以 \(x_0\) 为零电势点的\textbf{电势}为实物理量, 量纲为 \(\V\), 记作 \(\varphi_E(x)\). 

            定义从空间到电势的映射 \(x\mapsto\varphi_E(x)\) 为\textbf{电势函数}, 记作 \(\Phi_E\). 
        \end{dfn}
        
        \begin{dfn}
            {等势线}
        \end{dfn}
        
        \begin{dfn}
            [Electric-Field-Intensity]
            {电场强度}
            [Electric Field Intensity]
            [猫猫]
            设 \(E\) 是电场, \(\bm{x}\) 是 \(E\) 中一点, 定义 \(E\) 在 \(\bm{x}\) 处的\textbf{电场强度}为: \(\nabla\Phi_E(\bm{x})\). 

            定义从空间到电场强度的映射 \(\bm{x}\mapsto\nabla\Phi_E(\bm{x})\) 为\textbf{电场强度函数}, 记作 \(\bm{E}\). 
        \end{dfn}
        
        \begin{dfn}
            []
            {电场线}
            []
            [猫猫]
        \end{dfn}
        
        \begin{dfn}
            []
            {电通量}
            [Electric Flux]
            [猫猫]
            设 \(E\) 是电场, \(S\) 是 \(\Length^3\) 上的 \(2\) 维流形, 定义 \(E\) 通过 \(S\) 的\textbf{电通量}为: 
            \[\int_S \bm{E}(\bm{r})\cdot\bm{S}(\bm{r})\dd{\bm{r}}\]
            记作 \(\varPhi_E(S)\).
        \end{dfn}
        
        \begin{thm}
            []
            {Gauss 定理}
            [Gauss's Theorem]
            [猫猫]
            设 \(E\) 是电场, \(V\) 是 \(\Length^3\) 上的 \(3\) 维流形, 则: 
            \[\varPhi_E(\partial V) = \frac{1}{\epsz}\int_V \rho(\bm{r})\dd\bm{r}\]
        \end{thm}
        
        \begin{prf}
            是 Stokes 定理的特例. 
        \end{prf}

    \subsection{点电荷}
        
        \begin{dfn}
            [Vacuum-Permittivity]
            {真空介电常数}
            [Vacuum Permittivity]
            [猫猫]
            定义\textbf{真空介电常数}为 \(8.8541\times{10}^{-12}(\Nt\cdot\m^2\cdot\Clm^{-2})\), 记作 \(\epsz\). 
        \end{dfn}
        
        \begin{dfn}
            [Coulomb-Constant]
            {静电力常数 / Coulomb 常数}
            [Coulomb's Constant]
            [猫猫]
            定义\textbf{静电力常数}为 \(\dfrac{1}{4\pi\epsz}\), 记作 \(\ke\). 
        \end{dfn}
        
        \begin{ppt}
            []
            {静电力常数估值}
            []
            [猫猫]
            \(\ke\) 的估值为: 
            \[\ke \approx 8.9875\times{10}^9(\Nt\cdot\m^2\cdot\Clm^{-2})\]
        \end{ppt}
        
        \begin{str}
            [Point-Charge]
            {点电荷}
            [Electric Point Charge]
            [猫猫]
            定义\textbf{点电荷}类型承载以下信息: 
            \begin{enumerate}
                \item \电荷 (\(q\))
                \item 位置 (\(\bm{r}\))
            \end{enumerate}
        \end{str}
        
        \begin{ppt}
            []
            {点电荷的电势分布}
            []
            [猫猫]
            \[\frac{\ke q}{r}\bm{e}\]
        \end{ppt}
        
        \begin{axm}
            [Coulomb-Law]
            {Coulomb 定律}
            [Coulomb's Law]
            [猫猫]
            设 \((q_1, \bm{r}_1), (q_2, \bm{r}_2)\) 是\点电荷, \(r:=\|\bm{r}_1-\bm{r}_2\|\), \(\bm{e}_r:=\dfrac{\bm{r}_1-\bm{r}_2}{\|\bm{r}_1-\bm{r}_2\|}\), 则: 
            \[\bm{f}=\ke\cdot\frac{q_1 q_2}{r^2}\bm{e}_r\]
        \end{axm}

    \subsection{电荷集}
        
        \begin{thm}
            []
            {离散点电荷集的电力叠加原理}
            [Principle of Superposition]
            [猫猫]
            设 \(Q\) 是\点电荷 集, \(\card Q\in\N\), 则 \(Q\) 激发的电场为: 
            \[\sum_{q\in Q}E_q\]
        \end{thm}

    \subsection{电偶极子}
        
        \begin{str}
            [Electric-Dipole]
            {电偶极子}
            [Electric Dipole]
            [猫猫]
            定义\textbf{电偶极子}类型承载以下信息: 
            \begin{enumerate}
                \item \textbf{电偶极矩} \(\bm{p}:\Length^3\cdot\ElectricCurrent\cdot\Time\), 简称\textbf{电矩}: 
                \item \textbf{位置} \(\bm{r}:\Length^3\)
            \end{enumerate}
        \end{str}
        
        \begin{ppt}
            []
            {电偶极子受合电场力为零}
            []
            [猫猫]
        \end{ppt}
        
        \begin{ppt}
            []
            {电偶极子受电场力力矩}
            []
            [猫猫]
            设 \(D=(\bm{p}, \bm{r})\) 是电偶极子, \(E\) 是电场, \(\bm{E}\) 是 \(E\) 的电场强度函数, 则 \(D\) 在电场 \(E\) 受力矩为: 
            \[\bm{M}_D = \bm{p}\times\bm{E}(\bm{r})\]
        \end{ppt}
        
        \begin{ppt}
            []
            {电偶极子在电场中的电势能}
            []
            [猫猫]
            设 \(D=(\bm{p}, \bm{r})\) 是电偶极子, \(E\) 是电场, \(\bm{E}\) 是 \(E\) 的电场强度函数, 则 \(D\) 在电场 \(E\) 中的电势能为: 
            \[E=\nabla(\bm{p}\cdot\bm{E})\]

            设 \(E\) 是匀强电场, 则 \(D\) 在电场 \(E\) 中的电势能为: 
            \[E=-\bm{p}\cdot\bm{E}\]
        \end{ppt}

    \subsection{连续带电体}
        
        \begin{dfn}
            []
            {连续带电体}
            []
            [猫猫]
            定义\textbf{连续带电体}包含以下信息: 
            \begin{enumerate}
                \item 维数 \(n\in\{1,2,3\}\); 
                \item \(n\) 维光滑流形 \(M\); 
                \item 电荷密度函数 \(\rho:M\to\Length^{-n}\cdot\ElectricCurrent\cdot\Time\). 
            \end{enumerate}
        \end{dfn}
        
        \begin{rmk}
            [猫猫]
            习惯上, 记线密度函数为 \(\lambda\), 面密度函数为 \(\sigma\), 体密度函数为 \(\rho\). 
        \end{rmk}
        
        \begin{ppt}
            []
            {连续带电体的总电荷量}
            []
            [猫猫]
            设 \(Q:=(M, \rho)\) 是连续带电体, 则 \(Q\) 的总电荷量为: 
            \[\int_M \rho(\bm{r})\dd\bm{r}\]
        \end{ppt}
        
        \begin{ppt}
            []
            {连续线带电体激发电场的电场强度分布}
            []
            [猫猫]
            设 \(Q:=(M, \rho)\) 是连续线带电体, 则 \(Q\) 激发电场在 \(\bm{x}\) 处的电场强度为: 
            \[\bm{E}_Q(\bm{x})= \ke\int_M\frac{\rho(\bm{r})(\bm{x}-\bm{r})}{\|\bm{x}-\bm{r}\|^3}\dd\bm{r}\]
        \end{ppt}
        
        \begin{ppt}
            []
            {连续带电体的电势分布}
            []
            [猫猫]
            设 \(Q:=(M, \rho)\) 是连续带电体, 则 \(Q\) 激发电场在 \(\bm{x}\) 处的电势为: 
            \[\Phi_Q(\bm{x})= \ke\int_M\frac{\rho(\bm{r})}{\|\bm{x}-\bm{r}\|}\dd\bm{r}\]
        \end{ppt}
        
        \begin{xmp}
            []
            {无限长带电直线的电场}
            []
            [猫猫]
            设 \(L:=(\{(x,0,0)|x:\Length\}, \cdot\mapsto\lambda)\) 是连续带电体, 则: 
            \[\bm{E}_L(d\cos\theta,d\sin\theta,\cdot)=\frac{2\ke\lambda}{d}(\cos\theta,\sin\theta,0)\]
            \[\Phi_L(d\cos\theta,d\sin\theta,\cdot)=2\ke\lambda\ln d+C\]
        \end{xmp}
        
        \begin{prf}
            \[
            \begin{aligned}
                \|\bm{E}_L(d\cos\theta,d\sin\theta,z)\|
                & = \|\bm{E}_L(d,0,0)\| \\
                & = \ke\lambda\left\|\int_{-\infty}^{+\infty}\frac{(d,0,-z)}{(\sqrt{d^2+z^2})^3}\dd z\right\| \\
                & = \ke\lambda d\int_{-\infty}^{\infty}\frac{1}{(\sqrt{d^2+z^2})^3}\dd z\\
                & = \frac{2\ke\lambda}{d}\\
            \end{aligned}\]
        \end{prf}
        
        \begin{xmp}
            []
            {圆环带电体的电场}
            []
            [猫猫]
        \end{xmp}
        
        \begin{xmp}
            []
            {无限大均匀带电平面的电场是匀强电场}
            []
            [猫猫]
            设 \(P:=(\{(x,y,0)|x,y:\Length\}, \cdot\mapsto\sigma)\) 是连续带电体, 则 \(P\) 激发的电场在 \(\bm{r}:\Length^3\) 处的电场强度为: 
            \[\bm{E}_P(\bm{r})=\dfrac{\sigma}{2\epsz}\]
        \end{xmp}
        
        \begin{xmp}
            []
            {均匀带电球壳的电场}
            []
            [猫猫]
            设 \(R:\Length\), \(B:=(\{\bm{r}:\Length^3|\|\bm{r}\|=R\}, \cdot\mapsto\sigma)\) 是连续带电体, 
        \end{xmp}
        
        \begin{xmp}
            {无限长均匀带电柱壳的电场}
            设 \(R:\Length\), \(C:=(\{(R\cos\theta, R\sin\theta, z)|z:\Length\}, \cdot\mapsto\sigma)\) 是连续带电体, 则: 
            \[\bm{E}_C(\bm{r})=\begin{cases}
                0, & \|\bm{r}\|<R\\
                \dfrac{\sigma R}{\epsz\|\bm{r}\|}\hat{\bm{r}}, & \|\bm{r}\|\ge R
            \end{cases}\]
        \end{xmp}
        
        \begin{xmp}
            {均匀带电球体的电场}
        \end{xmp}

    \subsection{电场力做功}

\section{静电感应}

    \subsection{导体的静电感应}
    
        \begin{dfn}
            [Conductor]
            {导体}
            [Conductor]
            [猫猫]
            定义\textbf{导体}是带电体类型的子类型. 
        \end{dfn}

        \begin{dfn}
            [Static-Electric-Equilibrium]
            {静电平衡}
            [Static Electric Equilibrium]
            [猫猫]
            设 \(E\) 是电场, \(C\) 是连续带电导体, 定义 \(C\) 在 \(E\) 中达到\textbf{静电平衡}当且仅当: 
            \begin{enumerate}
                \item 导体内部总场强为零: 
                \[\forall \bm{r} \in\intr C, \bm{E}(\bm{r}) + \bm{E}_C(\bm{r})=0\]
                
                \item 导体边界总场强与边界流形正交: 
                \[\forall \bm{r}\in\bd C, (\bm{E}(\bm{r}) + \bm{E}_C(\bm{r}))\perp T_{\bm{r}}\bd C\]
            \end{enumerate}
        \end{dfn}
        
        \begin{ppt}
            []
            {导体静电平衡时内部无电荷}
            []
            [猫猫]
            设 \(E\) 是电场, \(C\) 是连续带电导体, \(C\) 在 \(E\) 中达到静电平衡, 则: 
            \[\forall \bm{r}\in\intr C, \rho_C(\bm{r})=0\]
        \end{ppt}
        
        \begin{ppt}
            []
            {导体表面电场强度}
            []
            [猫猫]
            设 \(E\) 是电场, \(C\) 是连续带电导体, \(C\) 在 \(E\) 中达到静电平衡, 则: 
            \[\forall \bm{r}\in\bd C, \bm{E}(\bm{r})=\frac{\sigma}{\epsz}\bm{e}_{\bm{r}}\]
        \end{ppt}

    \subsection{电介质}
        
        \begin{dfn}
            [Dielectric]
            {电介质模型}
            [Dielectric]
            [猫猫]
            定义\textbf{电介质}是带电体类型的子类型, 包含以下信息: 
            \begin{enumerate}
                \item 流形 \(M\); 
                \item 极化强度函数 \(\bm{P}:M\to\Length^3\cdot\ElectricCurrent\cdot\Time\);
            \end{enumerate}
        \end{dfn}
        
        \begin{dfn}
            [Polarization-Intensity]
            {极化强度}
            [Polarization Intensity]
            [猫猫]
        \end{dfn}


\end{document}