\documentclass[UTF8]{ctexart}

\makeatletter
\def\input@path{{../../Fulcrum-Template/}{../../Operator-List/}}
\makeatother

\usepackage{FulcrumCN}

\usepackage{OperatorListCN}
\usepackage{F4Logic}
\usepackage{F4Set}
\usepackage{F4Topology}
\usepackage{F4Analysis}

% margin
\usepackage{geometry}
\geometry{
    paper =a4paper,
    top =3cm,
    bottom =3cm,
    left=2cm,
    right =2cm
}
\linespread{1.2}

\begin{document}

\tableofcontents
\newpage

\section{经典电磁学}

    \subsection{点电荷}

        \begin{dfn}
            [Electric-Charge]
            {电荷}
            [Electric Charge]
            [猫猫]
            定义电荷为实数物理量, 量纲为 \(\). 
        \end{dfn}
        
        \begin{axm}
            []
            {电子电量}
            []
            [猫猫]
            电子电量为 \(|e|\approx-1.602 \times 10^{-19} C\).
        \end{axm}
        
        \begin{dfn}
            [Point-Charge]
            {点电荷}
            [Electric Point Charge]
            [猫猫]
        \end{dfn}
        
        \begin{axm}
            [Coulomb-Law]
            {Coulomb 定律}
            [Coulomb's Law]
            [猫猫]
            设 \((q_1, \bm{r}_1), (q_2, \bm{r}_2)\) 是点电荷, \(r:=|\bm{r}_1-\bm{r}_2|\), 则: 
            \[\bm{f}=k_e\cdot\frac{q_1 q_2}{r^2}\bm{e}_r\]
        \end{axm}
        
        \begin{dfn}
            [Electric-Field]
            {电场}
            [Electric Field]
            [猫猫]
        \end{dfn}
        
        \begin{dfn}
            [Electric-Field-Intensity]
            {电场强度}
            [Electric Field Intensity]
            [猫猫]
        \end{dfn}
        
        \begin{thm}
            []
            {离散点集的电力叠加原理}
            [Principle of Superposition]
            [猫猫]
            设 \(Q\) 是点电荷集, 
        \end{thm}
        
        \begin{thm}
            []
            {连续点集的电力叠加原理}
            [Principle of Superposition]
            [猫猫]
        \end{thm}





\end{document}