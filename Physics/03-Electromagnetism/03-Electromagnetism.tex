\documentclass[UTF8]{ctexart}

\makeatletter
\def\input@path{{../../Fulcrum-Template/}{../../Operator-List/}}
\makeatother

\usepackage{FulcrumCN}

\usepackage{OperatorListCN}
\usepackage{F4Logic}
\usepackage{F4Set}
\usepackage{F4Topology}
\usepackage{F4Analysis}
\usepackage{F4Physics}

% margin
\usepackage{geometry}
\geometry{
    paper =a4paper,
    top =3cm,
    bottom =3cm,
    left=2cm,
    right =2cm
}
\linespread{1.2}

\begin{document}

\tableofcontents
\newpage

\section{经典电磁学}

    \subsection{静电场}

        \begin{dfn}
            [Electric-Charge]
            {电荷}
            [Electric Charge]
            [猫猫]
            定义\textbf{电荷}为实物理量, 量纲为 \(\A\cdot\s\). 

            定义\电荷 的量纲 \textbf{Column} 为 \(\A\cdot\s\), 记作 \(\Clm\). 
        \end{dfn}
        
        \begin{xmp}
            []
            {电子电量}
            []
            [猫猫]
            \textbf{电子电量}约为 \(-1.602 \times 10^{-19} \Clm\), 记作 \(e\). 
        \end{xmp}
        
        \begin{dfn}
            [Coulomb-Force]
            {静电力 / Coulomb 力}
            [Coulomb Force]
            [猫猫]
        \end{dfn}
        
        \begin{dfn}
            [Electric-Field]
            {电场}
            [Electric Field]
            [猫猫]
            一种场. 
        \end{dfn}
        
        \begin{dfn}
            [Electric-Potential]
            {电势}
            []
            []
            设 \(E\) 是电场, \(x\) 是 \(E\) 中一点, 定义 \(E\) 在 \(x\) 处的\textbf{电势}为实物理量, 量纲为 \(\V\), 记作 \(\varphi_E(x)\). 

            定义从空间到电势的映射 \(x\mapsto\varphi_E(x)\) 为\textbf{电势函数}, 记作 \(\Phi_E\). 
        \end{dfn}
        
        \begin{dfn}
            {等势线}
        \end{dfn}
        
        \begin{dfn}
            [Electric-Field-Intensity]
            {电场强度}
            [Electric Field Intensity]
            [猫猫]
            设 \(E\) 是电场, \(x\) 是 \(E\) 中一点, 定义 \(E\) 在 \(x\) 处的\textbf{电场强度}为: \(\nabla\Phi_E(x)\). 

            定义从空间到电场强度的映射 \(x\mapsto\nabla\Phi_E(x)\) 为\textbf{电场强度函数}, 记作 \(\bm{E}\). 
        \end{dfn}
        
        \begin{dfn}
            {电场线}
        \end{dfn}

    \subsection{点电荷的静电场}
        
        \begin{dfn}
            [Vacuum-Permittivity]
            {真空介电常数}
            [Vacuum Permittivity]
            [猫猫]
            定义\textbf{真空介电常数}为 \(8.8541\times{10}^{-12}(\Nt\cdot\m^2\cdot\Clm^{-2})\), 记作 \(\epsz\). 
        \end{dfn}
        
        \begin{dfn}
            [Coulomb-Constant]
            {静电力常数 / Coulomb 常数}
            [Coulomb's Constant]
            [猫猫]
            定义\textbf{静电力常数}为 \(\dfrac{1}{4\pi\epsz}\), 记作 \(\ke\). 
        \end{dfn}
        
        \begin{ppt}
            []
            {静电力常数估值}
            []
            [猫猫]
            \(\ke\) 的估值为: 
            \[\ke \approx 8.9875\times{10}^9(\Nt\cdot\m^2\cdot\Clm^{-2})\]
        \end{ppt}
        
        \begin{dfn}
            [Point-Charge]
            {点电荷}
            [Electric Point Charge]
            [猫猫]
            定义\textbf{点电荷}类型承载以下信息: 
            \begin{enumerate}
                \item \电荷 (\(q\))
                \item 位置 (\(\bm{r}\))
            \end{enumerate}
        \end{dfn}
        
        \begin{ppt}
            {点电荷的静电场}
        \end{ppt}
        
        \begin{axm}
            [Coulomb-Law]
            {Coulomb 定律}
            [Coulomb's Law]
            [猫猫]
            设 \((q_1, \bm{r}_1), (q_2, \bm{r}_2)\) 是\点电荷, \(r:=\|\bm{r}_1-\bm{r}_2\|\), \(\bm{e}_r:=\dfrac{\bm{r}_1-\bm{r}_2}{\|\bm{r}_1-\bm{r}_2\|}\), 则: 
            \[\bm{f}=\ke\cdot\frac{q_1 q_2}{r^2}\bm{e}_r\]
        \end{axm}

    \subsection{电荷集的静电场}
        
        \begin{thm}
            []
            {离散点电荷集的电力叠加原理}
            [Principle of Superposition]
            [猫猫]
            设 \(Q\) 是\点电荷 集, \(\card Q\in\N\), 则 \(Q\) 激发的电场为: 
            \[\sum_{q\in Q}E_q\]
        \end{thm}
        
        \begin{xmp}
            {偶极电场}
        \end{xmp}
        
        \begin{dfn}
            []
            {连续线带电体}
            []
            [猫猫]
            连续线带电体类型承载以下信息: 
            \begin{enumerate}
                \item 一维流形 \(L\); 
                \item 线电荷密度函数 \(\lambda:L\to ?\). 
            \end{enumerate}
        \end{dfn}
        
        \begin{ppt}
            []
            {连续线带电体的总电荷量}
            []
            [猫猫]
            设 \(Q:=(L, \lambda)\) 是连续线带电体, 则 \(Q\) 的总电荷量为: 
            \[\int_L \lambda(\bm{r})\dd\bm{r}\]
        \end{ppt}
        
        \begin{ppt}
            []
            {连续线带电体的电场}
            []
            [猫猫]
            设 \(Q:=(L, \lambda)\) 是连续线带电体, 则 \(Q\) 在空间一点 \(\bm{x}\) 处激发的电场为: 
            \[E_Q(\bm{x})=\int_L \ke\cdot\frac{\lambda(\bm{r})}{\|\bm{x}-\bm{r}\|^2}\bm{e}_{\bm{x}-\bm{r}}\dd \bm{r}\]
        \end{ppt}
        
        \begin{xmp}
            []
            {无限长带电直线的电场}
            []
            [猫猫]
        \end{xmp}
        
        \begin{xmp}
            []
            {圆环带电体的电场}
            []
            [猫猫]
        \end{xmp}
        
        \begin{dfn}
            []
            {连续面带电体}
            []
            [猫猫]
            连续面带电体类型承载以下信息: 
            \begin{enumerate}
                \item 二维流形; 
                \item 面电荷密度函数 \(\sigma\). 
            \end{enumerate}
        \end{dfn}
        
        \begin{ppt}
            []
            {连续面带电体的电场}
            []
            [猫猫]
        \end{ppt}
        
        \begin{xmp}
            []
            {无限大带电平面的电场}
            []
            [猫猫]
        \end{xmp}
        
        \begin{dfn}
            []
            {连续体带电体}
            []
            [猫猫]
            连续体带电体类型承载以下信息: 
            \begin{enumerate}
                \item 三维流形; 
                \item 体电荷密度函数 \(\rho\). 
            \end{enumerate}
        \end{dfn}
        
        \begin{ppt}
            []
            {连续体带电体的电场}
            []
            [猫猫]
        \end{ppt}

    \subsection{电通量}
        
        \begin{dfn}
            {电通量}
        \end{dfn}
        
        \begin{ppt}
            {匀强电场的电通量}
        \end{ppt}

    \subsection{电场力做功}



\end{document}