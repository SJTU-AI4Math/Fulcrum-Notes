\documentclass[UTF8]{ctexart}

\makeatletter
\def\input@path{{../../Fulcrum-Template/}{../../Operator-List/}}
\makeatother

\usepackage{FulcrumCN}

\usepackage{OperatorListCN}
\usepackage{F4Logic}
\usepackage{F4Set}
\usepackage{F4Topology}
\usepackage{F4Analysis}
\usepackage{F4Physics}

% margin
\usepackage{geometry}
\geometry{
    paper =a4paper,
    top =3cm,
    bottom =3cm,
    left=2cm,
    right =2cm
}
\linespread{1.2}

\begin{document}

\tableofcontents
\newpage

\section{经典电磁学}

    \subsection{带电体}

        \begin{dfn}
            [Electric-Charge]
            {电荷}
            [Electric Charge]
            [猫猫]
            定义\textbf{电荷}为 \(\ElectricCurrent\cdot\Time\), 记作 \(\ElectricCharge\). 

            定义\电荷 的单位 \textbf{Column} 为 \(\A\cdot\s\), 记作 \(\Clm\). 
        \end{dfn}
        
        \begin{str}
            [Single-Charged-Object]
            {单带电体}
            [Single Charged Object]
            [猫猫]
            定义\textbf{单带电体}类型派生自\物体, 承载以下信息: 
            \begin{enumerate}
                \item 维度 \(n:[3]\)
                \item \(n\) 维光滑流形 \(M_Q\); 
                \item \textbf{电荷分布} \(\rho_Q:M_Q\to\ElectricCharge\cdot\Length_{-n}\); 
            \end{enumerate}
        \end{str}
        
        \begin{dfn}
            [Charged-Object]
            {带电体}
            [Charged Object]
            [猫猫]
            定义\textbf{带电体}类型是\单带电体 集合类型. 
        \end{dfn}
        
        \begin{dfn}
            []
            {带电体的总电荷量}
            []
            [猫猫]
            设 \(Q\) 是\带电体, 定义 \(Q\) 的\textbf{总电荷量}为: 
            \[\int_M \rho_Q(\bm{r})\dd\bm{r}\]
        \end{dfn}

    \subsection{静电场}
        
        \begin{dfn}
            [Electric-Potential]
            {电势}
            [Electric Potential]
            [猫猫]
            定义\textbf{电势}为 \(\Energy\cdot\ElectricCharge^{-1}\), 记作 \(\ElectricPotential\). 

            定义\textbf{电势}的单位 \textbf{Volt} 为 \(\Nt\cdot\m\cdot\Clm^{-1}\), 记作 \(\Vlt\). 
        \end{dfn}
        
        \begin{str}
            [Electric-Field]
            {电场}
            [Electric Field]
            [猫猫]
            定义\textbf{电场}类型派生自\场, 包含以下信息: 
            \begin{enumerate}
                \item \textbf{电势函数} \(\EPot[E]:\Length_3\to\ElectricPotential\): 
            \end{enumerate}
        \end{str}
        
        \begin{dfn}
            []
            {等势线}
            []
            [猫猫]*
        \end{dfn}
        
        \begin{dfn}
            []
            {电场线}
            []
            [猫猫]*
        \end{dfn}
        
        \begin{dfn}
            [Voltage]
            {电势差 / 电压}
            [Voltage]
            [猫猫]
            设 \(E\) 是\电场, \(a,b:\Length_3\), 定义\textbf{电势差}为: \(\EPot[E](a)-\EPot[E](b)\), 记作 \(\Vltg{a}{b}\). 
        \end{dfn}
        
        \begin{dfn}
            [Electric-Field-Intensity]
            {电场强度}
            [Electric Field Intensity]
            [猫猫]
            设 \(E\) 是\电场, \(\bm{x}:\Length_3\), 定义 \(E\) 在 \(\bm{x}\) 处的\textbf{电场强度}为: \(-\nabla\EPot[E](\bm{x})\), 记作 \(\EInts[E](\bm{x})\). 

            定义 \(E\) 的\textbf{电场强度函数} 为 \(\bm{x}\mapsto\EInts[E](\bm{x})\), 记作 \(\EInts[E]\). 
        \end{dfn}
        
        \begin{ppt}
            []
            {电场强度场是保守场}
            []
            [猫猫]
        \end{ppt}
        
        \begin{ppt}
            []
            {电势差为电场强度的路径积分}
            []
            [猫猫]
            设 \(E\) 是\电场, \(a,b:\Length_3\), \(C\) 是 \(a\) 到 \(b\) 的路径, 则: 
            \[\Vltg{a}{b}=-\int_C\EInts[E](\bm{r})\cdot\dd{\bm{r}}\]
        \end{ppt}
        
        \begin{dfn}
            [Coulomb-Force]
            {电场力 / Coulomb 力 / 静电力}
            [Coulomb Force]
            [猫猫]
            设 \(E\) 是\电场, \(Q\) 是\带电体, 定义 \(E\) 作用在 \(Q\) 上的\textbf{电场力}. 
        \end{dfn}
        
        \begin{dfn}
            [Electric-Flux]
            {电通量}
            [Electric Flux]
            [猫猫]
            设 \(E\) 是\电场, \(S\) 是 \(\Length_3\) 上的 \(2\) 维流形, 定义 \(E\) 通过 \(S\) 的\textbf{电通量}为: 
            \[\int_S\EInts[E](\bm{r})\cdot\bm{S}(\bm{r})\dd{\bm{r}}\]
            记作 \(\EFlx[E](S)\). 
        \end{dfn}
        
        \begin{thm}
            []
            {Gauss 定理}
            [Gauss's Theorem]
            [猫猫]
            设 \(E\) 是电场, \(M\) 是 \(\Length_3\) 上的 \(3\) 维流形, 则: 
            \[\EFlx[E](\partial M) = \frac{1}{\epsz}\int_M \rho(\bm{r})\dd\bm{r}\]
        \end{thm}
        
        \begin{prf}
            是 Stokes 定理的特例. 
        \end{prf}
        
        \begin{dfn}
            [Electric-Potential-Energy]
            {电势能}
            [Electric Potential Energy]
            [猫猫]
            设 \(E\) 是\电场, \(Q\) 是\带电体, 定义 \(Q\) 在 \(E\) 中的\textbf{电势能}为: 
            \[\]
        \end{dfn}

    \subsection{带电体激发的电场}
        
        \begin{dfn}
            [Vacuum-Permittivity]
            {真空介电常数}
            [Vacuum Permittivity]
            [猫猫]
            定义\textbf{真空介电常数}为 \(8.8541\times{10}^{-12}(\Nt\cdot\m^2\cdot\Clm^{-2})\), 记作 \(\epsz\). 
        \end{dfn}
        
        \begin{dfn}
            [Coulomb-Constant]
            {静电力常数 / Coulomb 常数}
            [Coulomb's Constant]
            [猫猫]
            定义\textbf{静电力常数}为 \(\dfrac{1}{4\pi\epsz}\), 记作 \(\ke\). 
        \end{dfn}
        
        \begin{ppt}
            []
            {静电力常数估值}
            []
            [猫猫]*
            \[\ke \approx 8.9875\times{10}^9(\Nt\cdot\m^2\cdot\Clm^{-2})\]
        \end{ppt}
        
        \begin{dfn}
            []
            {带电体激发的电场}
            []
            [猫猫]
            设 \(Q\) 是\带电体, 定义由 \(Q\) \textbf{激发的电场}为: 
            \begin{enumerate}
            \item 电势函数: 
            \[\EPot[Q]=\bm{x}\mapsto\ke\int_{M_Q}\frac{\rho_Q(\bm{r})}{\|\bm{x}-\bm{r}\|}\dd\bm{r}\]
            \end{enumerate}

            记作 \(E_Q\). 
        \end{dfn}
        
        \begin{thm}
            []
            {带电体激发电场的电场强度分布}
            []
            [猫猫]
            设 \(Q\) 是\带电体, \(\bm{x}\)则: 
            \[\bm{E}_Q(\bm{x})=\ke\int_{M_Q}\frac{\rho_Q(\bm{r})}{\|\bm{x}-\bm{r}\|^2}(\widehat{\bm{x}-\bm{r}})\dd\bm{r}\]
        \end{thm}
        
        \begin{thm}
            []
            {电力叠加原理}
            [Principle of Superposition]
            [猫猫]
            设 \(\{Q_n\}:\N\to\带电体\), 则: 
            \[\Phi_{\Im\{Q_n\}}=\sum_{n:\N}\Phi_{Q_n}\]
        \end{thm}

    \subsection{常见带电体模型}
        
        \begin{str}
            [Point-Charge]
            {点电荷}
            [Electric Point Charge]
            [猫猫]
            定义\textbf{点电荷}类型是\单带电体 的子类型, 维数为 \(0\). 包含以下信息: 
            \begin{enumerate}
                \item 带电量 \(q:\ElectricCharge\)
                \item 位置 \(\bm{r}:\Length_3\)
            \end{enumerate}
        \end{str}
        
        \begin{ppt}
            []
            {点电荷的电势分布}
            []
            [猫猫]
            设 \((q, \bm{r})\) 是\点电荷, \(\bm{x}:\Length_3\), 则 \((q, \bm{r})\) 激发电场在 \(\bm{x}\) 处的电势为: 
            \[\frac{\ke q}{\|\bm{x}-\bm{r}\|}\widehat{(\bm{x}-\bm{r})}\]
        \end{ppt}
        
        \begin{xmp}
            []
            {电子的点电荷模型}
            []
            [猫猫]
            定义\textbf{电子}为\点电荷 类型的子类型. 
            \[q_e\approx -1.602 \times 10^{-19} \Clm\]
            
            电子电荷量记作 \(e\). 
        \end{xmp}
        
        \begin{thm}
            [Coulomb-Law]
            {Coulomb 定律}
            [Coulomb's Law]
            [猫猫]
            设 \((q_1, \bm{r}_1), (q_2, \bm{r}_2)\) 是\点电荷, \(r:=\|\bm{r}_1-\bm{r}_2\|\), \(\bm{e}_r:=\dfrac{\bm{r}_1-\bm{r}_2}{\|\bm{r}_1-\bm{r}_2\|}\), 则: 
            \[\bm{f}=\ke\cdot\frac{q_1 q_2}{r^2}\bm{e}_r\]
        \end{thm}
        
        \begin{str}
            [Discrete-Charged-Object]
            {离散带电体}
            []
            [猫猫]
        \end{str}
        
        \begin{ppt}
            {}
        \end{ppt}
        
        \begin{str}
            [Line-Charged-Object]
            {线带电体}
            []
            [猫猫]
            定义\textbf{线带电体}为\单带电体 类型的子类型, 维数为 \(1\). 

            包含以下信息: 
            \begin{enumerate}
                \item \(1\) 维光滑流形 \(M\); 
                \item 线密度函数 \(\lambda:M\to\ElectricCharge\cdot\Length^{-1}\). 
            \end{enumerate}
        \end{str}
        
        \begin{xmp}
            []
            {无限长带电直线的电场}
            []
            [猫猫]
            设 \(L:=(\{(x,0,0)|x:\Length\}, \cdot\mapsto\lambda)\) 是\线带电体, 则: 
            \begin{enumerate}
                \item
                    \[\bm{E}_L(d\cos\theta,d\sin\theta,\cdot)=\frac{2\ke\lambda}{d}(\cos\theta,\sin\theta,0)\]

                \item
                    \[\Phi_L(d\cos\theta,d\sin\theta,\cdot)=2\ke\lambda\ln d+C\]
            \end{enumerate}
        \end{xmp}
        
        \begin{prf}
            \[
            \begin{aligned}
                \|\bm{E}_L(d\cos\theta,d\sin\theta,z)\|
                & = \|\bm{E}_L(d,0,0)\| \\
                & = \ke\lambda\left\|\int_{-\infty}^{+\infty}\frac{(d,0,-z)}{(\sqrt{d^2+z^2})^3}\dd z\right\| \\
                & = \ke\lambda d\int_{-\infty}^{\infty}\frac{1}{(\sqrt{d^2+z^2})^3}\dd z\\
                & = \frac{2\ke\lambda}{d}\\
            \end{aligned}\]
        \end{prf}
        
        \begin{xmp}
            []
            {圆环带电体的电场}
            []
            [猫猫]*
        \end{xmp}
        
        \begin{str}
            [Surface-Charged-Object]
            {面带电体}
            []
            [猫猫]
            定义\textbf{面带电体}为\带电体 类型的子类型, 维数为 \(2\). 
            
            包含以下信息:
            \begin{enumerate}
                \item \(2\) 维光滑流形 \(M\); 
                \item 面密度函数 \(\sigma:M\to\ElectricCharge\cdot\Length^{-2}\). 
            \end{enumerate}
        \end{str}
        
        \begin{xmp}
            [Charged-Plane]
            {无限大均匀带电平面}
            []
            [猫猫]
            设 \(P:=(\{(x,y,0)|x,y:\Length\}, \cdot\mapsto\sigma)\) 是\面带电体, 则 \(P\) 激发的电场在 \(\bm{r}:\Length^3\) 处的电场强度为: 
            \[\bm{E}_P(\bm{r})=\dfrac{\sigma}{2\epsz}\]
        \end{xmp}
        
        \begin{xmp}
            []
            {均匀带电球壳的电场}
            []
            [猫猫]
            设 \(R:\Length\), \(B:=(\{\bm{r}:\Length^3|\|\bm{r}\|=R\}, \cdot\mapsto\sigma)\) 是连续带电体, 则: 
        \end{xmp}
        
        \begin{xmp}
            {无限长均匀带电柱壳的电场}
            设 \(R:\Length\), \(C:=(\{(R\cos\theta, R\sin\theta, z)|z:\Length\}, \cdot\mapsto\sigma)\) 是\面带电体, 则: 
            \[\bm{E}_C(\bm{r})=\begin{cases}
                0, & \|\bm{r}\|<R\\
                \dfrac{\sigma R}{\epsz\|\bm{r}\|}\hat{\bm{r}}, & \|\bm{r}\|\ge R
            \end{cases}\]
        \end{xmp}
        
        \begin{str}
            [Volume-Charged-Object]
            {体带电体}
            []
            [猫猫]
            定义\textbf{体带电体}为\带电体 类型的子类型, 维数为 \(3\). 
            
            包含以下信息: 
            \begin{enumerate}
                \item \(3\) 维光滑流形 \(M\); 
                \item 体密度函数 \(\rho:M\to\ElectricCharge\cdot\Length^{-3}\).
            \end{enumerate}
        \end{str}
        
        \begin{xmp}
            []
            {均匀带电球体的电场}
            []
            [猫猫]*
        \end{xmp}

    \subsection{电场力做功}

    \subsection{电偶极子}
        
        \begin{str}
            [Electric-Dipole]
            {电偶极子}
            [Electric Dipole]
            [猫猫]
            定义\textbf{电偶极子}类型承载以下信息: 
            \begin{enumerate}
                \item \textbf{电偶极矩} \(\bm{p}:\Length^3\cdot\ElectricCurrent\cdot\Time\), 简称\textbf{电矩}: 
                \item \textbf{位置} \(\bm{r}:\Length^3\)
            \end{enumerate}
        \end{str}
        
        \begin{ppt}
            []
            {电偶极子受合电场力为零}
            []
            [猫猫]
        \end{ppt}
        
        \begin{ppt}
            []
            {电偶极子受电场力力矩}
            []
            [猫猫]
            设 \(D=(\bm{p}, \bm{r})\) 是电偶极子, \(E\) 是电场, \(\bm{E}\) 是 \(E\) 的电场强度函数, 则 \(D\) 在电场 \(E\) 受力矩为: 
            \[\bm{M}_D = \bm{p}\times\bm{E}(\bm{r})\]
        \end{ppt}
        
        \begin{ppt}
            []
            {电偶极子在电场中的电势能}
            []
            [猫猫]
            设 \(D=(\bm{p}, \bm{r})\) 是电偶极子, \(E\) 是电场, \(\bm{E}\) 是 \(E\) 的电场强度函数, 则 \(D\) 在电场 \(E\) 中的电势能为: 
            \[E=\nabla(\bm{p}\cdot\bm{E})\]

            设 \(E\) 是匀强电场, 则 \(D\) 在电场 \(E\) 中的电势能为: 
            \[E=-\bm{p}\cdot\bm{E}\]
        \end{ppt}

\section{静电感应}

    \subsection{导体的静电感应}
    
        \begin{dfn}
            [Conductor]
            {导体}
            [Conductor]
            [猫猫]
            定义\textbf{导体}是\带电体 类型的子类型, 包含以下信息: 
            \begin{enumerate}
                \item 内部带电体 \(\rho_C\)
                \item 边界带电体 \(\sigma_C\)
            \end{enumerate}
        \end{dfn}

        \begin{dfn}
            [Static-Electric-Equilibrium]
            {静电平衡}
            [Static Electric Equilibrium]
            [猫猫]
            设 \(E\) 是\电场, \(C\) 是\导体, 定义 \(C\) 在 \(E\) 中达到\textbf{静电平衡}当且仅当: 
            \begin{enumerate}
                \item 导体内部总场强为零: 
                \[\forall \bm{r} \in\intr C, \bm{E}(\bm{r}) + \bm{E}_C(\bm{r})=0\]
                
                \item 导体边界总场强与边界流形正交: 
                \[\forall \bm{r}\in\bd C, (\bm{E}(\bm{r}) + \bm{E}_C(\bm{r}))\perp T_{\bm{r}}\bd C\]
            \end{enumerate}
        \end{dfn}
        
        \begin{ppt}
            []
            {导体静电平衡时内部无电荷}
            []
            [猫猫]
            设 \(E\) 是电场, \(C\) 是连续带电导体, \(C\) 在 \(E\) 中达到静电平衡, 则: 
            \[\forall \bm{r}\in\intr C, \rho_C(\bm{r})=0\]
        \end{ppt}
        
        \begin{ppt}
            []
            {导体表面电场强度}
            []
            [猫猫]
            设 \(E\) 是电场, \(C\) 是连续带电导体, \(C\) 在 \(E\) 中达到静电平衡, 则: 
            \[\forall \bm{r}\in\bd C, \bm{E}(\bm{r})=\frac{\sigma}{\epsz}\bm{e}_{\bm{r}}\]
        \end{ppt}

    \subsection{电介质的极化}
        
        \begin{str}
            [Dielectric]
            {电介质}
            [Dielectric]
            [猫猫]
            定义\textbf{电介质}类型包含以下信息: 
            \begin{enumerate}
                \item \textbf{极化率} \(\chi_e\); 
            \end{enumerate}
        \end{str}
        
        \begin{dfn}
            [Relative-Permittivity]
            {相对介电常数}
            [Relative Permittivity]
            [猫猫]
            设 \(X\) 是\电介质, 定义 \(X\) 的\textbf{相对介电常数} 为: \(1+\chi_e\), 记作 \(\varepsilon_r\). 
        \end{dfn}
        
        \begin{dfn}
            [Permittivity]
            {介电常数}
            [Permittivity]
            [猫猫]
            设 \(X\) 是\电介质, 定义 \(X\) 的\textbf{介电常数} 为: \(\epsz\varepsilon_r\), 记作 \(\varepsilon\). 
        \end{dfn}
        
        \begin{dfn}
            [Polarize]
            {极化强度}
            [Polarize]
            [猫猫]
            设 \(X\) 是\电介质, \(X\) 的\极化率 为 \(\chi_e\), \(E\) 是\电场, 定义 \(X\) 在 \(E\) 下的\textbf{极化强度}为: \(\chi_e\epsz\bm{E}\), 记作 \(\bm{P}\). 
        \end{dfn}
        
        \begin{dfn}
            [Polarization-Intensity]
            {极化电荷密度}
            [Polarization Intensity]
            [猫猫]
            设 \(X\) 是\电介质, \(E\) 是\电场, 定义 \(X\) 在 \(E\) 下的\textbf{极化电荷密度}为: 
            \[\bm{P}(\bm{x})\cdot\hat{\bm{n}}(\bm{x})\]
        \end{dfn}
        
        \begin{ppt}
            [Displacement]
            {电位移矢量}
            [Electric Displacement]
            [猫猫]
            \[\bm{D}=\epsz\varepsilon_r\bm{E}\]
        \end{ppt}

    \subsection{电容器}
        
        \begin{str}
            [Capacitor]
            {电容器}
            [Capacitor]
            [猫猫]
            定义\textbf{电容器}类型包含以下信息: 
            \begin{enumerate}
                \item \textbf{极板}: \(A, B\); 
                \item 电介质 \(D\); 
            \end{enumerate}
        \end{str}
        
        \begin{xmp}
            []
            {平行板电容器}
            []
            [猫猫]
            设 \(A,B\) 是\带电平面, \(A,B\) 面积为 \(S\), \(A,B\) 距离为 \(d\), \(D\) 是\电介质, \(D\) 的\介电常数 为 \(\varepsilon\), \(D\) 充满 \(A,B\) 之间, 则定义 \((A,B,D)\) 为\textbf{平行板电容器}. 
            \[C=\frac{\varepsilon S}{d}\]
        \end{xmp}
        
        \begin{xmp}
            []
            {柱形电容器}
            []
            [猫猫]
        \end{xmp}
        
        \begin{xmp}
            []
            {球形电容器}
            []
            [猫猫]
        \end{xmp}
        
        \begin{dfn}
            []
            {电容器的串联与并联}
            []
            [猫猫]
            设 \(C_1,C_2\) 是\电容器, 则: 
        \end{dfn}
        
        \begin{ppt}
            []
            {串联电容器的等效电容}
            []
            [猫猫]
            设 \(C_1,C_2\) 是\电容器, 
            \[\frac{1}{C}=\frac{1}{C_1}+\frac{1}{C_2}\]
        \end{ppt}

        \begin{ppt}
            []
            {并联电容器的等效电容}
            []
            [猫猫]
            \[C=C_1+C_2\]
        \end{ppt}

    \subsection{电势能}

\section{电流}

    \subsection{电流的描述}
        
        \begin{dfn}
            [Power-Supply]
            {电源}
            [Power Supply]
            [猫猫]
            \begin{enumerate}
                \item 非静电力场强
            \end{enumerate}
        \end{dfn}
        
        \begin{dfn}
            []
            {电动势}
            []
            [猫猫]
            设 \(X\) 是电源, 定义 \(X\) 的\textbf{电动势}为
            \[\int_{-}^{+}\bm{E}_k\cdot\dd\bm{l}\]

            记作 \(\mathscr{E}\). 
        \end{dfn}

        \begin{dfn}
            []
            {电流强度}
            []
            [猫猫]
            设 \(C\) 是导体, \(S\) 是 \(C\) 上的截面, 定义通过截面 \(S\) 的\textbf{电流强度}为: 
            \[\frac{\dd q}{\dd t}\]
        \end{dfn}
        
        \begin{dfn}
            []
            {电流密度}
            []
            [猫猫]
            设 \(C\) 是导体, \(S\) 是 \(C\) 上的截面, 定义通过截面 \(S\) 的\textbf{电流密度}为: 
            \[\frac{\dd I}{\dd S}\]
        \end{dfn}

    \subsection{磁场}
        
        \begin{dfn}
            [Magnetic-Induction-Intensity]
            {磁感应强度}
            [Magnetic-Induction-Intensity]
            [猫猫]
            定义\textbf{磁感应强度}为 \(\Nt\cdot\m^{-1}\cdot\A^{-1}\), 记作 \(\MagneticInductionIntensity\). 

            定义磁感应强度的单位 \textbf{Tesla} 为 \(\Nt\cdot\m^{-1}\cdot\A^{-1}\), 记作 \(\Tsl\). 
        \end{dfn}

        \begin{str}
            [Magnetic-Field]
            {磁场}
            [Magnetic Field]
            [猫猫]
            定义\textbf{磁场}是场, 包含以下信息: 
            \begin{enumerate}
                \item 磁感应强度函数 \(\bm{B}:\Length_3\to\); 
            \end{enumerate}
        \end{str}
        
        \begin{dfn}
            []
            {Lorentz 力 / 磁场力}
            []
            [猫猫]
            设 \(B\) 是磁场, \(Q\) 是\带电体, \(Q\) 的带电量为 \(q\), \(\bm{v}\) 是 \(Q\) 的速度, 定义 \(B\) 作用在 \(Q\) 上的\textbf{Lorentz 力 / 磁场力}为: 
            \[\bm{F} = q\bm{v}\times\bm{B}\]
        \end{dfn}
        
        \begin{ppt}
            {带电质点}
        \end{ppt}
        
        \begin{dfn}
            {运动电荷激发磁场}
        \end{dfn}
        
        \begin{axm}
            []
            {Biot-Savart 定律}
            [Biot-Savart Law]
            [猫猫]
            \[\bm{B}=\frac{\mu_0}{4\pi}\int_M\frac{I\dd\bm{l}\times\bm{r}}{r^3}\]
        \end{axm}

    \subsection{}
        
        \begin{dfn}
            []
            {}
            []
            []
        \end{dfn}
        
        \begin{dfn}
            []
            {电导率}
            []
            [猫猫]
            设 \(C\) 是导体, \(n\) 是载流子数密度, \(\tau\) 是平均自由时间, 定义 \(C\) 的\textbf{电导率}为: 
            \[\frac{ne^2\tau}{2m}\]

            记作 \(\gamma\). 
        \end{dfn}
        
        \begin{thm}
            []
            {Ohm 定律}
            [Ohm's Law]
            [猫猫]
            设 \(I\) 是电流, \(U\) 是电压, \(R\) 是电阻, 则有: 
            \[U=IR\]
        \end{thm}

        


\end{document}