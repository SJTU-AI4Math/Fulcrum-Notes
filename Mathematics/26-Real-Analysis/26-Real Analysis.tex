\documentclass[UTF8]{ctexart}

\makeatletter
\def\input@path{{../../Fulcrum-Template/}{../../Operator-List/}}
\makeatother

\usepackage{FulcrumCN}
\usepackage{OperatorListCN}
\usepackage{F4Topology}
\usepackage{F4Analysis}
\usepackage{F4Logic}
\usepackage{F4Set}

\usepackage{geometry}
\geometry{
    paper =a4paper,
    top =3cm,
    bottom =3cm,
    left=2cm,
    right =2cm
}

\linespread{1.2}

\begin{document}
    \begin{center}
        {\LARGE\textbf{实分析笔记}}

        Fulcrum4Math
    \end{center}

    \tableofcontents

    \newpage

    \section{Lebesgue 测度} % uuid:"Section-Lebesgue-Measure"

        \subsection{\(\sigma\)-代数与 Borel 集族} % uuid:"Subsection-Sigma-Algebra-Measure-Space"

            \begin{str}
                [Sigma-Algebra]
                {\(\sigma\)-代数}
                [\(\sigma\)-Algebra]
                [猫猫]
                设 \(S\) 是类型, \(\Sigma_S\subseteq\PS[S]\), 定义 \((S,\Sigma_S)\) 是 \(S\) 上的\textbf{\(\sigma\)-代数} 当且仅当: 
                \begin{enumerate}
                    \item 包含全集; 
                    \[S\in\Sigma_S\]
    
                    \item 对可数并封闭: 
                    \[\forall\{X_i|i\in\N\}\subseteq\Sigma_S, \bigcup_{i\in\N}X_i\in\Sigma_S\]
    
                    \item 对补集封闭: 
                    \[\forall X\in\Sigma_S, S-X\in\Sigma_S\]
                \end{enumerate}
            \end{str}

            \begin{ppt}
                [Sigma-Algebra-Equiv-Def]
                {\(\sigma\)-代数的等价定义}
                []
                [猫猫]
                (1) 可替换为 \(\varnothing\) 在 \(\Sigma_S\) 内; 
                \[\varnothing\in\Sigma_S\]

                (2) 可替换为 \(\Sigma_S\) 对可数交封闭: 
                \[\forall\{X_i|i\in\N\}\subseteq\Sigma_S, \bigcap_{i\in\N}X_i\in\Sigma_S\]
            \end{ppt}
            
            \begin{prf}
                (1) 由 \Sigma代数 对取补运算封闭性即证. 

                (2) 由 De Morgan 公式即证. 
            \end{prf}

            \begin{rmk}
                [猫猫]
                语境中若无歧义, 常省略类型 \(S\) 而直接称 \(\Sigma_S\) 为 \Sigma代数.
            \end{rmk}
            
            \begin{dfn}
                [Borel-Set]
                {Borel 集族}
                [Borel Sets]
                [猫猫]
                设 \((S,\mathcal{T})\) 是\拓扑空间, 定义由 \((S,\mathcal{T})\) 生成的 \textbf{Borel 集族} 为: 
                \[\bigcap_{\mathcal{T}\subseteq\Sigma_S}\Sigma_S\]

                记作 \(\Borel{S}\)
            \end{dfn}

            \begin{ppt}
                [Open-Closed-Borel]
                {开集与闭集是 Borel 集}
                []
                [猫猫]
                设 \((S,\mathcal{T})\) 是\拓扑空间, \(X\subseteq S\) 则:
                \[
                \begin{cases}
                    \开集[X]\implies X\in\Borel{S}\\
                    \闭集[X]\implies S-X\in\Borel{S}
                \end{cases}
                \]
            \end{ppt}

            \begin{xmp}
                [Real-Borel-Set]
                {\(\R\) 上的 Borel 集族}
                []
                [猫猫]
                由 \(\R\) 上的通常拓扑生成的 Borel 集族记作 \(\Borel{\R}\). 
                \begin{enumerate}
                    \item 奠基步: 定义 \(\Sigma_1^0\) 为: \(\{X\subseteq\R|\开集[X]\}\)
                        
                        定义 \(\Pi_1^0\) 为: \(\{X\subseteq\R|\闭集[X]\}\)

                    \item 定义 \(\Sigma_2^0\), 又称 \(F_\sigma\) 集, 为: 
                        \[\left\{\bigcup_{n=0}^{+\infty}X_i\middle|X_i\in\Pi_1^0\right\}\]

                        定义 \(\Pi_2^0\), 又称 \(G_\delta\) 集, 为: 
                        \[\left\{\bigcap_{n=0}^{+\infty}X_i\middle|X_i\in\Sigma_1^0\right\}\]

                    \item 后继步: 设 \(n\in\omega_1\), 

                        定义 \(\Sigma_{n+1}^0\) 为: 
                        \[\left\{\bigcup_{i=0}^{+\infty}X_i\middle|X_i\in\Pi_{n}^0\right\}\]

                        定义 \(\Pi_{n+1}^0\) 为: 
                        \[\left\{\bigcap_{i=0}^{+\infty}X_i\middle|X_i\in\Sigma_{n}^0\right\}\]

                    \item 极限步: 设 \(\lambda\in\omega_1\), 
                    
                        定义 \(\Sigma_\lambda^0\) 为: 
                        \[\left\{\bigcup_{i=0}^{+\infty}X_i\middle|X_i\in\bigcup_{n<\lambda}\Pi_{n}^0\right\}\]
                        
                        定义 \(\Pi_\lambda^0\) 为: 
                        \[\left\{\bigcap_{i=0}^{+\infty}X_i\middle|X_i\in\bigcup_{n<\lambda}\Sigma_{n}^0\right\}\]

                    \item 设 \(n\in\omega_1\), 定义 \(\Delta_\omega^0\) 为: \(\Sigma_\omega^0\cap\Pi_\omega^0\). 
                \end{enumerate}
            \end{xmp}

            \begin{ppt}
                [Real-Borel-Set-Hierarchy]
                {\(\Borel{\R}\) 的分层结构}
                []
                [猫猫]*
                \begin{enumerate}
                    \item 高层次集包含低层次集: 
                        \[\forall m,n\in\omega_1, m<n\implies
                        \begin{cases}
                            \Sigma_{m}^0\subsetneqq\Sigma_{n}^0\\
                            \Pi_{m}^0\subsetneqq\Pi_{n}^0\\
                        \end{cases}\]
                    
                    \item \(\Borel{\R}\) 可被全部层次集生成: 
                        \[\Borel{\R}=\bigcup_{n\in\omega_1}\Sigma_n^0\cup\Pi_n^0\]
                \end{enumerate}
            \end{ppt}

            \begin{ppt}
                [Real-Borel-Set-Uncountable]
                {\(\Borel{\R}\) 不可数}
                []
                [猫猫]*
                \[\card(\Borel{\R})=2^{\aleph_0}\]
            \end{ppt}

        \subsection{测度与测度空间} % uuid:"Subsection-Measure-Measure-Space"

            \begin{dfn}
                [Measure]
                {测度与测度空间}
                [Measure and Measure Space]
                [猫猫]
                设 \(\Sigma_S\) 是 \Sigma代数, \(\mu:\Sigma_S\to\bar{\R}\), 定义 \(\mu\) 是 \(\Sigma_S\) 上的\textbf{测度}, 当且仅当: 
                \begin{enumerate}
                    \item 非负性: 
                    \[\forall X\in\Sigma_S, \mu(X)\geq 0\]
    
                    \item 可数可加性: 
                    \[\forall\{X_n\}:\N\to\Sigma_S, (\forall i,j:\N, i\neq j\implies X_i\cap X_j=\varnothing)\implies\mu\left(\bigcup_{i\in\N}X_i\right)=\sum_{i\in\N}\mu(X_i)\]
                \end{enumerate}

                此时结构 \((S,\Sigma_S,\mu)\) 称为\textbf{测度空间}. 
            \end{dfn}
            
            \begin{ppt}
                [Empty-Set-Zero-Measure]
                {空集零测}
                []
                [猫猫]
                设 \((S,\Sigma_S,\mu)\) 是\测度空间, 则: 
                \[\mu(\varnothing)=0\]
            \end{ppt}
            
            \begin{prf}
                \[\mu(\varnothing)+\mu(\varnothing)=\mu(\varnothing\cup\varnothing)=\mu(\varnothing)\Longrightarrow\mu(\varnothing)=0\]
                \qed
            \end{prf}
            
            \begin{ppt}
                [Subset-Measure-Leq]
                {子集测度小于等于全集测度}
                []
                [猫猫]
                设 \((S,\Sigma_S,\mu)\) 是\测度空间, \(A,B\in\Sigma_S, A\subseteq B\), 则: \(\mu(A)\leq\mu(B)\). 
            \end{ppt}
            
            \begin{prf}
                用子集及其补表示全集即证: 
                \[B=A\cup(B-A)\Longrightarrow\mu(B)=\mu(A)+\mu(B-A)\geq\mu(A)+0=\mu(A)\]
                \qed
            \end{prf}

            \begin{xmp}
                [Trivial-Measure]
                {平凡测度}
                [Trivial Measure]
                [猫猫]
                设 \(\Sigma_S\) 是 \Sigma代数, 则: 
                \begin{enumerate}
                    \item \(\cdot\mapsto 0\) 是 \(\Sigma_S\) 上的\测度.

                    \item \(\cdot\mapsto +\infty\) 是 \(\Sigma_S\) 上的\测度.
                \end{enumerate}
            \end{xmp}

            \begin{xmp}
                [Cardinal-Measure]
                {基数测度}
                [Cardinal Measure]
                [猫猫]
                设 \(\Sigma_S\) 是 \Sigma代数, 则: \(X\mapsto\card(X)\) 是 \(\Sigma_S\) 上的\测度. 
            \end{xmp}

            \begin{dfn}
                [Measure-Regularity]
                {正则性}
                [Regularity]
                [猫猫]
                设 \((S,\mathcal{T})\) 是\拓扑空间, \((S,\Sigma_S,\mu)\) 是\测度空间, 
                
                定义 \(\mu\) \textbf{外正则}, 当且仅当: 
                \[\forall X\in\Sigma_S, \mu(X)=\inf\{\mu(O)|X\subseteq O\land\开集[O]\}\]

                定义 \(\mu\) \textbf{内正则}, 当且仅当: 
                \[\forall X\in\Sigma_S, \mu(X)=\sup\{\mu(K)|K\subseteq X\land\紧集[K]\}\]

                定义 \(\mu\) \textbf{正则}, 当且仅当: \(\mu\) 外正则且内正则. 
            \end{dfn}

            \begin{rmk}
                [猫猫]
                外正则性希望集合的测度可以用开集从外部良好逼近, 内正则性希望集合的测度可以用紧集从内部良好逼近. 
                
                在满足外正则性的前提下, 只要定义了开集的测度, 就能通过下确界生成所有集合的测度, 这样定义的测度称为外测度. 同理, 在满足内正则性的前提下, 只要定义了紧集的测度, 就能通过上确界生成所有集合的测度, 这样定义的测度称为内测度. 

                正则性要求两者皆成立, 也就是说任何集合的测度既可以用开集从外部逼近, 也可以用紧集从内部逼近, 且两者的值相等. 
            \end{rmk}

        \subsection{Borel 测度}
            
            \begin{dfn}
                [Borel-Measure-Real]
                {\(\R\) 上在通常拓扑意义下的 Borel 测度}
                []
                [猫猫]
                设 \(\mu:\Borel{\R}\to\bar{\R}\), \(\mu\) 是 \(\Borel{\R}\) 上的\测度, 定义 \(\mu\) 是 \textbf{Borel 测度} 当且仅当: 
                \[\forall a,b(a<b)\in\bar\R, \mu\Ioo{a}{b}=b-a\]
            \end{dfn}
            
            \begin{ppt}
                [Borel-Measure-Unique]
                {存在唯一的 Borel 测度}
                []
                [猫猫]
                    \(\exists!\mu:\Borel{\R}\to\bar{\R}\), \(\mu\) 是 \Borel测度. 
            \end{ppt}
            
            \begin{ppt}
                [Single-Point-Zero-Measure]
                {单点集零测}
                []
                [猫猫]
                设 \(\mu\) 是 \Borel测度, 则: 
                \[\forall x\in\R, \mu\{x\}=0\]
            \end{ppt}
            
            \begin{prf}
                考虑以点为中心的开球,并令半径任意小即可: 
                \[\forall x\in\R, \forall\varepsilon\in\R^+, \{x\}\subseteq\Ioo{x-\frac{\varepsilon}{2}}{x+\frac{\varepsilon}{2}}\]
                \[\Longrightarrow\forall\varepsilon\in\R^+, \mu\{x\}\leq\varepsilon\Longrightarrow\mu\{x\}=0\square\]
            \end{prf}
            
            \begin{ppt}
                [Interval-Borel-Measure]
                {区间上的 Borel 测度等于区间长度}
                []
                [猫猫]
                设 \(\mu\) 是 \Borel测度, 则: 
                \[\forall a,b\in\bar\R, a<b\implies
                \mu\Ioo{a}{b}=
                \mu\Ioc{a}{b}=
                \mu\Ico{a}{b}=
                \mu\Icc{a}{b}=b-a\]
            \end{ppt}

            \begin{ppt}
                {Borel 测度平移不变}
            \end{ppt}
            
            \begin{prf}
                由单点集零测性质, 用单点集的可数并表示可数集即证. \(\square\)
            \end{prf}
            
            \begin{ppt}
                [Countable-Zero-Measure]
                {可数集零测}
                []
                [猫猫]
                设 \(\mu\) 是 \Borel测度, 则: 
                \[\forall S\subseteq\R, \card S=\aleph_0\implies\mu(S)=0\]
            \end{ppt}
            
            \begin{cxmp}
                [Cantor-Set]
                {Cantor 集}
                [Cantor Set]
                [猫猫]
                Cantor 集是零测的, 但不可数. 
            \end{cxmp}
            
        \subsection{\(\R\) 上的 Lebesgue 测度} % uuid:"Lebesgue-Measure-On-R"
        
            \begin{dfn}
                [Complete-Measure-Space]
                {完备测度空间}
                [Complete Measure Space]
                [猫猫]
                设 \((S,\Sigma_S,\mu)\) 是\测度空间, 定义 \((S,\Sigma_S,\mu)\) 是\textbf{完备测度空间}, 当且仅当: 
                \[\forall N\in\Sigma_S, \mu(N)=0\implies\PS(N)\subseteq\Sigma_S\]
            \end{dfn}
            
            \begin{dfn}
                [Outer-Lebesgue-Measure]
                {Lebesgue 外测度}
                [Outer Lebesgue Measure]
                [猫猫]
                定义 \textbf{Lebesgue 外测度}, 记作 \(\mu^*\), 为: 
                \begin{enumerate}
                    \item 设 \(a,b:\bar{\R}\), \(a<b\), 则: 
                        \[\mu^*\Ioo{a}{b}=b-a\]
                    
                    \item 设 \(\开集[X]\), \(X=\bigcup\limits_{i=0}^{+\infty}\Ioo{a_i}{b_i}\) 则: 
                        \[\mu^*(X)=\sum_{i=0}^{+\infty}\mu^*\Ioo{a_i}{b_i}=\sum_{i=0}^{+\infty}(b_i-a_i)\]

                    \item 设 \(X\subseteq\R\), 则: 
                        \[\mu^*(X)=\inf\{\mu^*(O)|X\subseteq O\land\开集[O]\}\]
                \end{enumerate}
            \end{dfn}

            \begin{ppt}
                {外测度总是外正则的}
            \end{ppt}

            \begin{dfn}
                [Inner-Lebesgue-Measure]
                {Lebesgue 内测度}
                []
                [猫猫]

                用紧集内逼近得到的集函数 \(\mu_*:\PP(S)\to\R\) 称为是 \(\PP(S)\) 上的内测度: 
                \[\mu_*(X)=\sup_{C\text{是紧集}}\sum_{C\subseteq X}\mu(C)\]
            \end{dfn}
            
            \begin{dfn}
                [LebesgueMeasurableSet]
                {Lebesgue 可测集}
                [Lebesgue Measurable Set]
                [猫猫]
                设 \(X\subseteq\R\), 定义 \(X\) \textbf{Lebesgue 可测}当且仅当: 
                \[\mu^*(X)=\mu_*(X)\]
            \end{dfn}
            
            \begin{ppt}
                [ConstantinCaratheodory]
                {Constantin-Caratheodory 条件}
                []
                [猫猫]
                \(X\) 是 Lebesgue 可测的当且仅当 \(X\) 满足: 
                \[\forall A\subseteq S, \mu^*(A)=\mu^*((S-X)\cap A)+\mu^*(X\cap A)\]
            \end{ppt}
            
            \begin{ppt}
                [OuterMeasureZeroMeasurable]
                {外测度为零的集合 Lebesgue 可测}
                []
                [猫猫]
                \[\mu^*(S)=0\Longrightarrow S\text{ 是 Lebesgue 可测的}\]
            \end{ppt}
            
            \begin{cxmp}
                [VitaliSet]
                {Vitali 集}
                [Vitali Set]
                [猫猫]
                构造: 
            \end{cxmp}
            
            \begin{dfn}
                [MeasurableHullKernel]
                {可测包/核}
                [Measurable Hull/Kernel]
                [猫猫]
                对集合 \(S\subseteq\R\), 定义其\textbf{可测包}集族为包含 \(S\) 的最小可测集族, 其\textbf{可测核}
            \end{dfn}
            
            \begin{thm}
                [LebesgueMeasurableSetSigmaAlgebra]
                {Lebesgue 可测集族构成 \Sigma代数}
                []
                [猫猫]
                Lebesgue 可测集族构成 \(\R\) 上的一个 \Sigma代数. 
            \end{thm}
            
            \begin{crl}
                [LebesgueMeasurableSetGeneratedByBorelZero]
                {Lebesgue 可测集族由 Borel 集与零测集生成}
                []
                [猫猫]
                Lebesgue 可测集族是由 \(\mathcal{B}(S)\) 与全体零测集生成的 \Sigma代数. 
            \end{crl}
            
            \begin{dfn}
                [LebesgueMeasure]
                {Lebesgue 测度}
                [Lebesgue Measure]
                [猫猫]
                将 Lebesgue 内/外测度限制在 Lebesgue 可测集上得到的集函数称为 \textbf{Lebesgue 测度}. 
                \[\mu:=\mu^*|_\mathcal{L}=\mu_*|_\mathcal{L}\]
            \end{dfn}
            
            \begin{ppt}
                [BorelMeasureIsLebesgueMeasureOnBorel]
                {Borel 测度是 Lebesgue 测度在 Borel 集上的限制}
                []
                [猫猫]
                \[\mu_\mathcal{B}=\mu_\mathcal{L}|_{\mathcal{B}(S)}\]
            \end{ppt}
            
            \begin{ppt}
                [LebesgueMeasureComplete]
                {Lebesgue 测度完备}
                []
                [猫猫]
            \end{ppt}
            
            \begin{ppt}
                [LebesgueMeasureTranslationInvariant]
                {Lebesgue 测度平移不变}
                []
                [猫猫]
            \end{ppt}
            
            \begin{thm}
                [LittlewoodFirstPrinciple]
                {Littlewood 第一原理}
                []
                [猫猫]
                有限测度的可测集接近于区间的有限并: 
                
                设 \(E\) 为有限测度的可测集, 则 \(\forall\varepsilon\in\R^+\), \(\exists\) 有限空交的开区间族 \({\{I_i\}}_{i=1}^n\), 记其并为 \(T\), 满足: 
                \[\mu^*(E-F)+\mu^*(F-E)<\varepsilon\]
            \end{thm}

    \section{Lebesgue 可测函数} % uuid:"LebesgueMeasurableFunction"

        *本章的``可测''均指 Lebesgue 可测. 

        *本章默认实值函数的值域是扩充实数集 \(\bar{\R}\). 

        *称一个性质``几乎处处''成立指该性质在除了某零测集外处处成立. 

        \subsection{Lebesgue 可测函数} % uuid:"LebesgueMeasurableFunction"

            \begin{dfn}
                [LebesgueMeasurableFunction]
                {Lebesgue 可测函数}
                [Lebesgue Measurable Functions]
                [猫猫]
                设 \((S,\Sigma_S,\mu)\) 是测度空间, 函数 \(f:S\to\bar{\R}\) 称为是 \textbf{Lebesgue 可测函数}, 若 \(\forall\alpha\in\bar{\R}\), 集合 \(\{x\in S|f(x)>\alpha\}\) 是可测的. 
            \end{dfn}
            
            \begin{ppt}
                [LebesgueMeasurableFunctionEquivDef]
                {Lebesgue 可测函数的等价定义}
                []
                [猫猫]
                设 \(S\) 是可测集, 则下列叙述互相等价: 

                (1) \(\forall\alpha\in\bar{\R}, \{x\in S|f(x)>\alpha\}\) 是可测集. 

                (2) \(\forall\alpha\in\bar{\R}, \{x\in S|f(x)\geq\alpha\}\) 是可测集. 

                (3) \(\forall\alpha\in\bar{\R}, \{x\in S|f(x)<\alpha\}\) 是可测集. 

                (4) \(\forall\alpha\in\bar{\R}, \{x\in S|f(x)\leq\alpha\}\) 是可测集. 
            \end{ppt}
            
            \begin{ppt}
                [LebesgueMeasurableFunctionSinglePreimageMeasurable]
                {可测函数单点值的原像可测}
                []
                [猫猫]
                设 \(f:S\to\bar{\R}\) 是可测函数, 则 \(\forall\alpha\in\bar{\R}, \{x\in S|f(x)=\alpha\}\) 是可测集. 
            \end{ppt}
            
            \begin{ppt}
                [LebesgueMeasurableFunctionOpenPreimageMeasurable]
                {函数可测当且仅当开集的原像可测}
                []
                [猫猫]
                设 \(f:S\to\bar{\R}\) 是实值函数, 则 \(f\) 是可测函数当且仅当开集的原像是可测集. 
            \end{ppt}
            
            \begin{ppt}
                [LebesgueMeasurableFunctionCombination]
                {可测函数的拼合}
                []
                [猫猫]
                设可测集 \(S\) 有可测子集 \(T\), 则实值函数 \(f\) 在 \(S\) 上可测当且仅当 \(f|_T\) 和 \(f_{S-T}\) 均可测.
            \end{ppt}
            
            \begin{ppt}
                [LebesgueMeasurableFunctionTransitivity]
                {可测函数的传递性}
                []
                [猫猫]
                设 \(f\) 在 \(S\) 上可测, \(f=g\) 在 \(S\) 上几乎处处成立, 则 \(g\) 在 \(S\) 上可测. 
            \end{ppt}
            
            \begin{ppt}
                [LebesgueMeasurableFunctionAlgebra]
                {可测函数代数}
                []
                [猫猫]
                设 \(f,g:S\to\bar{\R}\) 是可测函数, 则: 
                
                (1) \(\forall k,l\in\bar{\R}\), 函数 \(k f+l g\) 是可测函数. 

                (2) 函数 \(fg\) 是可测函数. 
            \end{ppt}
            
            \begin{ppt}
                [LebesgueMeasurableFunctionComposition]
                {可测函数的复合稳定性}
                []
                [猫猫]
            \end{ppt}
            
            \begin{ppt}
                [LebesgueMeasurableFunctionPointwiseLimit]
                {可测函数的逐点收敛稳定性}
                []
                [猫猫]
                若定义在 \(D\) 上的可测函数序列 \(f_n\) 几乎处处逐点收敛于 \(f\), 则 \(f\) 可测. 
            \end{ppt}
            
            \begin{xmp}
                [ContinuousFunctionMeasurable]
                {可测集定义域上的连续实值函数可测}
                []
                [猫猫]
            \end{xmp}
            
            \begin{xmp}
                [MonotoneFunctionMeasurable]
                {定义在区间上的单调函数可测}
                []
                [猫猫]
            \end{xmp}
            
            \begin{cxmp}
                [UnmeasurableFunction]
                {不可测函数}
                []
                [猫猫]
                构造
            \end{cxmp}

        \subsection{简单逼近定理} % uuid:"SimpleFunctionApproximationTheorem"
            
            \begin{dfn}
                [IndicatorFunction]
                {指示/示性/特征函数}
                [Indicator/Characteristic Function]
                [猫猫]
                集合 \(A\subseteq S\) 的指示函数 \(\chi_A:S\to\{0,1\}\) 定义为:
                \[\chi_A=
                \begin{cases}
                    1, & x\in A\\
                    0, & x\notin A
                \end{cases}\]
            \end{dfn}
            
            \begin{ppt}
                [MeasurableSetIndicatorFunction]
                {集合可测当且仅当其指示函数可测}
                []
                [猫猫]
            \end{ppt}
            
            \begin{dfn}
                [SimpleFunction]
                {简单函数/单纯函数}
                [Simple Function]
                [猫猫]
                定义在可测集上的实值函数 \(f:E\to\bar{\R}\) 称为是\textbf{简单函数}, 若其值域是有限集. 
            \end{dfn}
            
            \begin{ppt}
                [SimpleFunctionAlgebra]
                {简单函数代数}
                [Simple Function Algebra]
                [猫猫]
            \end{ppt}
            
            \begin{ppt}
                [SimpleFunctionCanonicalRepresentation]
                {简单函数的典范表示}
                []
                [猫猫]
                简单函数能表示为有限个集合的指示函数的线性组合: 
                \[f=\sum_{i=1}^{n}C_i\cdot\chi_{A_i}(A_i=\{x\in A_i|f(x)=C_i\})\]
            \end{ppt}
            
            \begin{lma}
                [SimpleFunctionApproximationLemma]
                {简单逼近引理}
                [Simple Function Approximation Lemma]
                [猫猫]
                可测函数可以被简单函数良好控制: 

                设 \(f:E\to\bar{\R}\) 是可测函数, \(f\) 在 \(E\) 上有界, 则 \(\forall\varepsilon>0\), \(\exists E\) 上的简单函数 \(\varphi_\varepsilon, \psi_\varepsilon\) 满足: 
                \[\forall x\in E, \varphi_\varepsilon\leq f\leq\psi_\varepsilon\wedge 0\leq\psi_\varepsilon-\varphi_\varepsilon<\varepsilon\]
            \end{lma}
            
            \begin{thm}
                [SimpleFunctionApproximationTheorem]
                {简单逼近定理}
                [Simple Function Approximation Theorem]
                [猫猫]
                可测函数可以被简单函数逼近: 

                定义在可测集 \(E\) 上的扩充实值函数 \(E\) 是可测的 \(\iff\exists E\) 上的简单函数序列 \(\{\varphi_n\}\) 满足: 
                
                (1) \(\{\varphi_n\}\) 逐点收敛到 \(f\); 

                (2) \[\forall n\in\N, \forall x\in E, |\varphi_n|\leq|f|\]

                其中若 \(f\) 非负, 则可选取满足上述条件且递增的 \(\{\varphi_n\}\). 
            \end{thm}

        \subsection{Littlewood 原理} % uuid: "LittlewoodsPrinciples"

            \begin{lma}
                [EgoroffsLemma]
                {Egoroff 引理}
                [Egoroff's Lemma]
                [猫猫]
                设 \(E\) 是有限测度可测集, \(\{f_n\}\) 是 \(E\) 上逐点收敛于 \(f\) 的可测函数序列, 则: 
                \[\forall\varepsilon\in\R^+, \forall\delta\in\R^+, \exists E_\varepsilon\subseteq E(\mu(E-E_\varepsilon)<\delta), \exists N\in\N, \forall n\geq N, \forall x\in E_\varepsilon, |f_n-f|<\varepsilon\]
            \end{lma}

            \begin{thm}
                [EgoroffsTheorem]
                {Egoroff 定理 / Littlewood 第三原理}
                [Egoroff's Theorem]
                [猫猫]
                逐点收敛于某一的可测函数序列近似是一致收敛的: 

                设 \(E\) 是有限测度可测集, \(\{f_n\}\) 是 \(E\) 上逐点收敛于 \(f\) 的可测函数序列, 则: 
                \[\forall\varepsilon\in\R^+, \exists E_\varepsilon\subseteq E(\mu(E-E_\varepsilon)=0), \{f_n\}\text{ 在 \(E_\varepsilon\) 上一致收敛于} f\]
            \end{thm}
            
            \begin{lma}
                [LittlewoodsSecondPrinciple]
                {Littlewood 第二原理}
                [Littlewood's Second Principle]
                [猫猫]
                简单函数接近于连续函数: 
                
                设 \(f\) 为可测集 \(E\) 上的简单函数, 则: 
                \[\forall\varepsilon\in\R^+, \exists g\in\mathcal{C}(\R), \exists E_\varepsilon\subseteq E(\mu(E-E_\varepsilon)<\varepsilon), \forall x\in E_\varepsilon, f=g\]
            \end{lma}
            
            \begin{thm}
                [LusinsTheorem]
                {Lusin 定理}
                [Lusin's Theorem]
                [猫猫]
                可测函数接近于连续函数: 
                
                设 \(f\) 为可测集 \(E\) 上的可测函数, 则: 
                \[\forall\varepsilon\in\R^+, \exists g\in\mathcal{C}(\R), \exists E_\varepsilon\subseteq E(\mu(E-E_\varepsilon)<\varepsilon), \forall x\in E_\varepsilon, f=g\]
            \end{thm}

    \section{Lebesgue 积分} % uuid:"LebesgueIntegral"

        \subsection{有限测度集上有界可测函数的 Lebesgue 积分}
            
            \begin{dfn}
                [SimpleLebesgueIntegral]
                {简单函数的 Lebesgue 积分}
                [Lebesgue Integral of Simple Functions]
                [猫猫]
                设 \(f\) 是有限测度可测集 \(E\) 上的简单函数, 其典范表示为 \(f=\sum\limits_{i=1}^{n}C_i\cdot\chi_{E_i}(E_i=\{x\in E_i|f(x)=C_i\})\), 则可如下定义 \(f\) 在 \(E\) 上的 Lebesgue 积分: 
                \[\int_E f:=\sum_{i=1}^n C_i\cdot\mu(E_i)\]
            \end{dfn}
\end{document}