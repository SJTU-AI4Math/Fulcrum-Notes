\documentclass[UTF8]{ctexart}

\makeatletter
\def\input@path{{../../../Fulcrum-Template/}{../../../Operator-List/}}
\makeatother

\usepackage{FulcrumCN}
\usepackage{OperatorListCN}
\usepackage{geometry}
\geometry{
    paper =a4paper,
    top =3cm,
    bottom =3cm,
    left=2cm,
    right =2cm
}
\linespread{1.2}

\begin{document}
    \begin{center}
        {\LARGE\textbf{Fulcrum 自动生成示例}}

        SJTU AI4Math Team
    \end{center}

    \section{知识条目}
    \subsection{自动生成条目}

    
    \begin{thm}
        [Theorem-on-the-Convergence-of-$N-n-/-n$]
        {关于 $N\_n / n$ 收敛性的定理}
        [Theorem on the Convergence of $N_n / n$]
        [gpt-4.1]
        
Taking $b_n = n$ in Theorem 2.6, now we have
\[
N_n / n \to e^{-c} \quad \mathrm{in~probability}
\]

    \end{thm}
    
    
    
    \begin{ppt}
        [Absolute-Value-Inequality-for-Integrals]
        {积分绝对值不等式}
        [Absolute Value Inequality for Integrals]
        [gpt-4.1]
        对于任意可积函数 $f$,有
\[
|\int f \, d\mu| \leq \int |f| \, d\mu.
\]
    \end{ppt}
    
    
    
    \begin{thm}
        [Approximation-of-Integrable-Functions]
        {可积函数的近似性}
        [Approximation of Integrable Functions]
        [gpt-4.1]
        若 $f$ 满足 $\|f\|_p = \left( \int |f|^p \, d\mu \right)^{1/p} < \infty$, 则存在一列简单函数 $\varphi_n$ 使得 $\|\varphi_n - f\|_p \to 0$.
    \end{thm}
    
    
    
    \begin{dfn}
        [Definition-of-the-Distribution-of-X]
        {随机变量X的分布的定义}
        [Definition of the Distribution of X]
        [gpt-4.1]
        3 Definition of the distribution of $X$.
    \end{dfn}
    
    
    
    \begin{thm}
        [Bounded-Convergence-Theorem]
        {有界收敛定理}
        [Bounded Convergence Theorem]
        [gpt-4.1]
        The special case of Theorem 1.6.7 in which $Y$ is constant is called the bounded convergence theorem.
    \end{thm}
    
    
    
    \begin{dfn}
        [Definition-of-Independence-of-Events]
        {事件独立性的定义}
        [Definition of Independence of Events]
        [gpt-4.1]
        两个事件 $A$ 和 $B$ 是独立的,如果 $P(A \cap B) = P(A) P(B)$.
    \end{dfn}
    
    
    
    \begin{dfn}
        [Definition-of-$\pi$-system-for-Set-$\mathcal{A}-i$]
        {集合 $\mathcal{A}_i$ 的 $\pi$-系统定义}
        [Definition of $\pi$-system for Set $\mathcal{A}_i$]
        [gpt-4.1]
        $\mathcal{A}_i$ 是一个 $\pi$-系统.
    \end{dfn}
    
    
    
    \begin{dfn}
        [Definition-of-Measures-on-the-Real-Line]
        {实数轴上的测度的定义}
        [Definition of Measures on the Real Line]
        [gpt-4.1]
        在 $(\mathbf{R}, \mathcal{R})$ 上的测度通过给定一个满足以下性质的 Stieltjes 测度函数 $F$ 来定义:

(i) $F$ 是非减的;
(ii) $F$ 是右连续的,即 $\lim_{y \downarrow x} F(y) = F(x)$.
    \end{dfn}
    
    
    
    \begin{thm}
        [Necessary-Condition-for-the-Strong-Law-of-Large-Numbers]
        {强大数定律的必要条件}
        [Necessary Condition for the Strong Law of Large Numbers]
        [gpt-4.1]
        定理 2.3.8 说明 $E | X_i | < \infty$ 是强大数定律的必要条件.
    \end{thm}
    
    
    
    \begin{thm}
        [Theorem-of-Unique-Product-Measure]
        {唯一乘积测度定理}
        [Theorem of Unique Product Measure]
        [gpt-4.1]
        
存在唯一的测度 $\mu$ 在 $\mathcal{F}$ 上,使得
\[
\mu(A \times B) = \mu_{1}(A) \mu_{2}(B)
\]
通常将 $\mu$ 记作 $\mu_{1} \times \mu_{2}$.

    \end{thm}
    
    
    
    \begin{thm}
        [Proposition-on-Independence-and-Probability-of-Events-$A-k$]
        {关于事件$A\_k$的独立性及概率的命题}
        [Proposition on Independence and Probability of Events $A_k$]
        [gpt-4.1]
        事件$A_k$是相互独立的,且$P(A_k) = 1 / k$.
    \end{thm}
    
    
    
    \begin{thm}
        [Integral-Expression-and-Finite-Measure-Relationship]
        {积分表达式与有限测度的关系}
        [Integral Expression and Finite Measure Relationship]
        [gpt-4.1]
        
设 $\mu$ 是 $\mathbf{R}$ 上的有限测度,$F(x) = \mu((-\infty, x])$.则有
\[
\int \left( F(x + c) - F(x) \right) dx = c \boldsymbol{\mu}(\mathbf{R})
\]

    \end{thm}
    
    
    
    \begin{thm}
        [Monotone-Convergence-Theorem]
        {单调收敛定理}
        [Monotone Convergence Theorem]
        [gpt-4.1]
        
设 $f_{n} \geq 0$ 且 $f_{n} \uparrow f$,则有
\[
\int f_{n} \, d\mu \uparrow \int f \, d\mu
\]

    \end{thm}
    
    
    
    \begin{prf}
        [Proof-of-the-Monotone-Convergence-Theorem]
        {单调收敛定理的证明}
        [Proof of the Monotone Convergence Theorem]
        [gpt-4.1]
        
证明:由 Fatou 引理和定理 1. 得证.

    \end{prf}
    
    
    
    \begin{dfn}
        [Definition-of-Geometric-Distribution]
        {几何分布的定义}
        [Definition of Geometric Distribution]
        [gpt-4.1]
        $N$ 被称为具有成功概率 $p \in (0, 1)$ 的几何分布,如果
\[
P(N = k) = p (1 - p)^{k - 1} \quad \text{for } k = 1, 2, \ldots
\]
这里 $N$ 表示为观察到概率为 $p$ 的事件所需的独立试验次数.
    \end{dfn}
    
    
    
    \begin{dfn}
        [Definition-of-λ-system]
        {λ-系统的定义}
        [Definition of λ-system]
        [gpt-4.1]
        若集合族 $\mathcal{L}$ 满足以下条件,则称其为 $\lambda$-系统:
(i) $\Omega \in \mathcal{L}$;
(ii) 若 $A, B \in \mathcal{L}$ 且 $A \subset B$, 则 $B - A \in \mathcal{L}$;
(iii) 若 $A_n \in \mathcal{L}$ 且 $A_n \uparrow A$, 则 $A \in \mathcal{L}$.

    \end{dfn}
    
    
    
    \begin{dfn}
        [Definition-of-Bernoulli-Distribution]
        {Bernoulli 分布的定义}
        [Definition of Bernoulli Distribution]
        [gpt-4.1]
        我们称随机变量 $X$ 服从参数为 $p$ 的 Bernoulli 分布, 如果 $P(X = 1) = p$ 且 $P(X = 0) = 1 - p$.
    \end{dfn}
    
    
    
    \begin{thm}
        [Theorem-1.2.2]
        {定理 1.2.2}
        [Theorem 1.2.2]
        [gpt-4.1]
        定理 1.2.2
    \end{thm}
    
    
    
    \begin{dfn}
        [Definition-of-the-$\sigma$-field-generated-by-a-collection]
        {由集合生成的$\sigma$-域的定义}
        [Definition of the $\sigma$-field generated by a collection]
        [gpt-4.1]
        设$\Omega$为一个集合,$\mathcal{A}$为$\Omega$的子集的一个集合,则包含$\mathcal{A}$的最小$\sigma$-域称为由$\mathcal{A}$生成的$\sigma$-域,记作$\sigma(\mathcal{A})$.
    \end{dfn}
    
    
    
    \begin{thm}
        [Independence-of-Product-of-Independent-Random-Variables]
        {独立随机变量乘积的独立性}
        [Independence of Product of Independent Random Variables]
        [gpt-4.1]
        如果 $X_1, \ldots, X_n$ 是独立的随机变量, 那么 $X = X_1$ 与 $Y = X_2 \cdots X_n$ 也是独立的.
    \end{thm}
    
    
    
    \begin{thm}
        [Condition-for-Exchanging-Summation-and-Integration]
        {交换求和与积分的条件}
        [Condition for Exchanging Summation and Integration]
        [gpt-4.1]
        若 $\sum_n \int |f_n| \, d\mu < \infty$, 则有 $\sum_n \int f_n \, d\mu = \int \sum_n f_n \, d\mu$.

    \end{thm}
    
    
    
    \begin{dfn}
        [Definition-of-Independence-of-Random-Variables]
        {随机变量的独立性定义}
        [Definition of Independence of Random Variables]
        [gpt-4.1]
        两个随机变量 $X$ 和 $Y$ 是独立的,当且仅当对所有 $C, D \in \mathcal{R}$,有
\[
P(X \in C, Y \in D) = P(X \in C) P(Y \in D)
\]

    \end{dfn}
    
    
    
    \begin{thm}
        [Independence-of-Sigma-Algebras-Generated-by-Independent-Pi-Systems]
        {独立的 $\pi$-系统的生成$\sigma$-代数的独立性}
        [Independence of Sigma-Algebras Generated by Independent Pi-Systems]
        [gpt-4.1]
        Suppose $\mathcal{A}_1, \mathcal{A}_2, \ldots, \mathcal{A}_n$ are independent and each $\mathcal{A}_i$ is a $\pi$-system. Then $\sigma(\mathcal{A}_1), \sigma(\mathcal{A}_2), \ldots, \sigma(\mathcal{A}_n)$ are independent.
    \end{thm}
    
    
    
    \begin{dfn}
        [Definition-of-Random-Variable]
        {随机变量的定义}
        [Definition of Random Variable]
        [gpt-4.1]
        一个定义在 $\Omega$ 上的实值函数 $X$ 被称为随机变量,如果对每个 Borel 集合 $B \subset \mathbf{R}$,都有 $X^{-1}(B) = \{\omega : X(\omega) \in B\} \in {\mathcal{F}}$.
    \end{dfn}
    
    
    
    \begin{thm}
        [Jensens-Inequality]
        {Jensen不等式}
        [Jensen's Inequality]
        [gpt-4.1]
        设 $\varphi$ 是凸函数,即对所有 $\lambda \in (0, 1)$ 和 $x, y \in \mathbf{R}$,有
\[
\lambda \varphi(x) + (1 - \lambda)\varphi(y) \geq \varphi(\lambda x + (1 - \lambda)y)
\]
则若 $E|X| < \infty$ 且 $E|\varphi(X)| < \infty$,成立
\[
E(\varphi(X)) \geq \varphi(EX)
\]
其中 $E$ 表示期望算子.

    \end{thm}
    
    
    
    \begin{thm}
        [Weak-Law-of-Large-Numbers-Standard-Form]
        {弱大数定律(最常见形式)}
        [Weak Law of Large Numbers (Standard Form)]
        [gpt-4.1]
        
设 $X_{1}, X_{2}, \ldots$ 是一列独立同分布随机变量,且 $E|X_{i}| < \infty$.令 $S_{n} = X_{1} + \cdots + X_{n}$,$\mu = E X_{1}$.
则有 $S_{n}/n \to \mu$ 以概率收敛.

    \end{thm}
    
    
    
    \begin{thm}
        [Second-Borel-Cantelli-Lemma]
        {第二个Borel-Cantelli引理}
        [Second Borel-Cantelli Lemma]
        [gpt-4.1]
        
若事件 $A_n$ 相互独立,且有 $\sum P(A_n) = \infty$,则 $P(A_n \text{ i.o.}) = 1$.

    \end{thm}
    
    
    
    \begin{thm}
        [Theorem-on-the-Distribution-of-the-Sum-of-Independent-Random-Variables]
        {独立随机变量和的分布定理}
        [Theorem on the Distribution of the Sum of Independent Random Variables]
        [gpt-4.1]
        
如果 $X$ 和 $Y$ 相互独立,$F(x) = P(X \leq x)$,$G(y) = P(Y \leq y)$,则有
\[
P(X + Y \leq z) = \int F(z - y) dG(y)
\]
右侧的积分称为 $F$ 和 $G$ 的卷积,记作 $F * G(z)$.

    \end{thm}
    
    
    
    \begin{dfn}
        [Definition-of-Independence-for-Infinite-Collections]
        {无限对象独立性的定义}
        [Definition of Independence for Infinite Collections]
        [gpt-4.1]
        一个无限对象(如 $\sigma$-域、随机变量或集合)被称为独立的,当且仅当其中任意有限子集都是独立的.
    \end{dfn}
    
    
    
    \begin{dfn}
        [Definition-of-Supremum-Norm]
        {无穷范数的定义}
        [Definition of Supremum Norm]
        [gpt-4.1]
        设 $\| f \|_{\infty} = \inf \{ M : \mu( \{ x : |f(x)| > M \} ) = 0 \}$.
    \end{dfn}
    
    
    
    \begin{thm}
        [Continuity-of-the-Integral-Function]
        {积分函数的连续性}
        [Continuity of the Integral Function]
        [gpt-4.1]
        设 $f$ 在区间 $[a, b]$ 上可积,定义 $g(x) = \int_{[a, x]} f(y) \, dy$,则 $g(x)$ 在开区间 $(a, b)$ 上是连续的.
    \end{thm}
    
    
    
    \begin{prf}
        [Proof-of-Expectation-Inequality]
        {关于期望值不等式的证明}
        [Proof of Expectation Inequality]
        [gpt-4.1]
        
根据 $i_{A}$ 的定义以及 $\varphi \geq 0$,有
\[
i_{A} 1_{(X \in A)} \leq \varphi(X) 1_{(X \in A)} \leq \varphi(X)
\]
因此,取期望并结合定理 1.6.1 的 (c) 部分,可以得到所需结论.

    \end{prf}
    
    
    
    \begin{xmp}
        [Moment-and-Variance-Calculation-for-Exponentially-Distributed-Random-Variable]
        {指数分布随机变量的矩和方差计算}
        [Moment and Variance Calculation for Exponentially Distributed Random Variable]
        [gpt-4.1]
        若随机变量 $X$ 服从参数为 1 的指数分布,则
\[
E X^{k} = \int_{0}^{\infty} x^{k} e^{-x} dx = k!
\]
所以 $X$ 的均值为 1,方差为 $E X^{2} - (E X)^{2} = 2 - 1^{2} = 1$.
    \end{xmp}
    
    
    
    \begin{xmp}
        [Example-of-Bernstein-Polynomial]
        {Bernstein多项式的例子}
        [Example of Bernstein Polynomial]
        [gpt-4.1]
        
设 $f$ 是 $[0,1]$ 上的连续函数,令
\[
f_n(x) = \sum_{m=0}^n \binom{n}{m} x^m (1-x)^{n-m} f(m/n) \quad \text{where} \quad \binom{n}{m} = \frac{n!}{m!(n-m)!}
\]
则 $f_n(x)$ 是与 $f$ 相关的 $n$ 次 Bernstein 多项式.

    \end{xmp}
    
    
    
    \begin{dfn}
        [Definition-of-Uncorrelated-Family-of-Random-Variables]
        {随机变量族的无相关性定义}
        [Definition of Uncorrelated Family of Random Variables]
        [gpt-4.1]
        
设 $\{X_i\}_{i \in I}$ 是一族随机变量,满足 $E X_i^2 < \infty$.若对任意 $i 
eq j$,都有
\[
E(X_i X_j) = E X_i E X_j
\]
则称该随机变量族是无相关的(uncorrelated).

    \end{dfn}
    
    
    
    \begin{thm}
        [Measurability-Theorem-for-Generated-Sigma-Algebra]
        {生成σ-代数的可测性定理}
        [Measurability Theorem for Generated Sigma-Algebra]
        [gpt-4.1]
        如果对所有 $A \in \mathcal{A}$,都有 $\{ \omega : X(\omega) \in A \} \in \mathcal{F}$,且 $\mathcal{A}$ 生成 $\mathcal{S}$(即 $\mathcal{S}$ 是包含 $\mathcal{A}$ 的最小 $\sigma$-域),那么 $X$ 是可测的.
    \end{thm}
    
    
    
    \begin{dfn}
        [Generalized-Definition-of-Random-Variable]
        {广义随机变量的定义}
        [Generalized Definition of Random Variable]
        [gpt-4.1]
        一个定义域为 $D \in \mathcal{F}$,值域为 $\mathbf{R}^{*} \equiv [ -\infty, \infty ]$ 的函数称为随机变量,如果对于所有 $B \in \mathcal{R}^{*}$,都有 $X^{-1}(B) = \{ \omega : X(\omega) \in B \} \in \mathcal{F}$.
    \end{dfn}
    
    
    
    \begin{xmp}
        [Example-of-Uniform-Distribution-on-01]
        {区间(0,1)上的均匀分布的例子}
        [Example of Uniform Distribution on (0,1)]
        [gpt-4.1]
        在区间 $(0,1)$ 上的均匀分布,其概率密度函数为 $f(x) = 1$ 当 $x \in (0, 1)$,否则为 $0$.
其分布函数为
\[
F(x) =
\begin{cases}
0 & x \leq 0 \\
x & 0 < x < 1 \\
1 & x \geq 1
\end{cases}
\]

    \end{xmp}
    
    
    
    \begin{thm}
        [Application-of-Theorem-1.6.8]
        {定理 1.6.8 的应用}
        [Application of Theorem 1.6.8]
        [gpt-4.1]
        可以应用定理 1.6.8,其中 $h(x) = x$,$g(x) = |x|^{2/\alpha}$,但序列 $X_n$ 并不被某个可积函数支配.
    \end{thm}
    
    
    
    \begin{thm}
        [Exchangeability-of-Expectation-for-Series-of-Non-negative-Random-Variables]
        {非负随机变量级数的期望可交换性}
        [Exchangeability of Expectation for Series of Non-negative Random Variables]
        [gpt-4.1]
        如果 $X_n \geq 0$,则
\[
E\left(\sum_{n=0}^{\infty} X_{n}\right) = \sum_{n=0}^{\infty} E X_{n}
\]
即,非负随机变量级数的期望等于各项期望之和.

    \end{thm}
    
    
    
    \begin{thm}
        [Additivity-Theorem-of-Gamma-Distribution]
        {伽马分布的可加性定理}
        [Additivity Theorem of Gamma Distribution]
        [gpt-4.1]
        若 $X = \mathrm{gamma}(\alpha, \lambda)$ 与 $Y = \mathrm{gamma}(\beta, \lambda)$ 相互独立,则 $X + Y$ 的分布为 $\mathrm{gamma}(\alpha + \beta, \lambda)$.

因此,若 $X_1, \dots, X_n$ 是相互独立的 $\mathrm{exponential}(\lambda)$ 随机变量,则 $X_1 + \cdots + X_n$ 服从 $\mathrm{gamma}(n, \lambda)$ 分布.
    \end{thm}
    
    
    
    \begin{lma}
        [Riemann-Lebesgue-Lemma]
        {Riemann-Lebesgue引理}
        [Riemann-Lebesgue Lemma]
        [gpt-4.1]
        
如果 $g$ 可积,则有
\[
\lim_{n \to \infty} \int g(x) \cos n x\, d x = 0
\]

    \end{lma}
    
    
    
    \begin{dfn}
        [Definition-of-Normal-Distribution]
        {正态分布的定义}
        [Definition of Normal Distribution]
        [gpt-4.1]
        
正态分布的概率密度函数为
\[
(2\pi a)^{-1/2} \exp\left( - (x - \mu)^2 / 2a \right),
\]
其中 $\mu$ 为均值,$a$ 为方差.

    \end{dfn}
    
    
    
    \begin{thm}
        [Expectation-Inequality-for-Convex-Functions]
        {凸函数的期望不等式}
        [Expectation Inequality for Convex Functions]
        [gpt-4.1]
        
设 $\varphi : \mathbf{R}^n \to \mathbf{R}$ 是凸函数,$X_1,\ldots,X_n$ 为随机变量,若 $E|\varphi(X_1, \ldots, X_n)| < \infty$ 且 $E|X_i| < \infty$ 对所有 $i$ 成立,则有
\[
E \varphi(X_1, \ldots, X_n) \geq \varphi(E X_1, \ldots, E X_n).
\]

    \end{thm}
    
    
    
    \begin{dfn}
        [Definition-of-Semialgebra]
        {半代数的定义}
        [Definition of Semialgebra]
        [gpt-4.1]
        
设 $\mathcal{S}$ 是一组集合,当且仅当满足以下两个条件时,称 $\mathcal{S}$ 为半代数:
(i) 对任意 $S, T \in \mathcal{S}$,有 $S \cap T \in \mathcal{S}$,即在交运算下封闭;
(ii) 对任意 $S \in \mathcal{S}$,其补集 $S^c$ 可以表示为有限个 $\mathcal{S}$ 中集合的不交并.

    \end{dfn}
    
    
    
    \begin{dfn}
        [Definition-of-Convergence-in-Probability]
        {概率收敛的定义}
        [Definition of Convergence in Probability]
        [gpt-4.1]
        我们称 $Y_n$ 以概率收敛于 $Y$,如果对所有 $\epsilon > 0$,有 $P(|Y_n - Y| > \epsilon) \to 0$ 当 $n \to \infty$ 时成立.
    \end{dfn}
    
    
    
    \begin{dfn}
        [Definition-of-Density-Function]
        {密度函数的定义}
        [Definition of Density Function]
        [gpt-4.1]
        当分布函数 $F(x) = P(X \leq x)$ 具有如下形式

\[
F(x) = \int_{-\infty}^x f(y) dy
\]

我们称 $X$ 具有密度函数 $f$.
    \end{dfn}
    
    
    
    \begin{dfn}
        [Necessary-and-Sufficient-Condition-for-Density-Function]
        {密度函数的必要与充分条件}
        [Necessary and Sufficient Condition for Density Function]
        [gpt-4.1]
        对于函数 $f(x)$,要使其定义一个分布函数,充要条件是 $f(x) \geq 0$ 且 $\int f(x) dx = 1$.
    \end{dfn}
    
    
    
    \begin{thm}
        [Change-of-Variables-Formula]
        {变量变换公式}
        [Change of Variables Formula]
        [gpt-4.1]
        设 $X$ 是 $(S, \mathcal{S})$ 上的随机元, 分布为 $\mu$, 即 $\mu(A) = P(X \in A)$.如果 $f$ 是从 $(S, \mathcal{S})$ 到 $(\mathbf{R}, \mathcal{R})$ 的可测函数,且 $f \geq 0$ 或 $E|f(X)| < \infty$,则有
\[
E f ( X ) = \int_{S} f ( y ) \mu ( d y )
\]

    \end{thm}
    
    
    
    \begin{thm}
        [Measure-Representation-of-the-Integral-of-Non-negative-Measurable-Function]
        {非负可测函数积分的测度表达式}
        [Measure Representation of the Integral of Non-negative Measurable Function]
        [gpt-4.1]
        
设 $g \ge 0$ 是 $(X, \mathcal{A}, \mu)$ 上的可测函数,则
\[
\int_{X} g \, d\mu = (\mu \times \lambda) (\{ (x, y): 0 \leq y < g(x) \}) = \int_{0}^{\infty} \mu(\{ x : g(x) > y \}) dy
\]

    \end{thm}
    
    
    
    \begin{thm}
        [Continuity-from-Above-for-Measures]
        {测度的上极限连续性}
        [Continuity from Above for Measures]
        [gpt-4.1]
        
若 $A_{i} \downarrow A$,即 $A_{1} \supset A_{2} \supset \ldots$ 且 $\bigcap_{i} A_{i} = A$,并且 $\mu(A_{1}) < \infty$,则有 $\mu(A_{i}) \downarrow \mu(A)$.

    \end{thm}
    
    
    
    \begin{thm}
        [Equivalence-of-Independence-between-Events-and-Indicator-Random-Variables]
        {事件独立性与指示随机变量独立性的等价性}
        [Equivalence of Independence between Events and Indicator Random Variables]
        [gpt-4.1]
        事件 $A$ 和 $B$ 独立,当且仅当它们的指示随机变量 $1_A$ 和 $1_B$ 独立.
    \end{thm}
    
    
    
    \begin{dfn}
        [Definition-of-Independence-of-Two-Sigma-Fields]
        {两个$\sigma$-域的独立性定义}
        [Definition of Independence of Two Sigma-Fields]
        [gpt-4.1]
        两个$\sigma$-域$\mathcal{F}$和$\mathcal{G}$是独立的,如果对所有$A \in \mathcal{F}$和$B \in \mathcal{G}$,事件$A$和$B$都是独立的.
    \end{dfn}
    
    
    
    \begin{dfn}
        [Definition-of-Real-Vector-Space-and-Borel-Sets]
        {实数向量空间与 Borel 集合的定义}
        [Definition of Real Vector Space and Borel Sets]
        [gpt-4.1]
        设 $\mathbf{R}^d$ 表示所有 $d$ 维实数向量 $(x_1, \ldots, x_d)$ 的集合,$\mathcal{R}^d$ 表示 Borel 集,即包含所有开集的最小 $\sigma$-域.当 $d = 1$ 时,省略上标.
    \end{dfn}
    
    
    
    \begin{dfn}
        [Definition-of-π-system]
        {π-系统的定义}
        [Definition of π-system]
        [gpt-4.1]
        称 $\mathcal{A}$ 为一个 $\pi$-系统,如果它对交运算封闭,即若 $A, B \in \mathcal{A}$,则 $A \cap B \in \mathcal{A}$.
    \end{dfn}
    
    
    
    \begin{dfn}
        [Definition-of-Probability-Measure]
        {概率测度的定义}
        [Definition of Probability Measure]
        [gpt-4.1]
        
若 $\mu$ 是定义在集合族 ${\mathcal{F}}$ 上的测度,且满足:(i) 对所有 $A \in {\mathcal{F}}$,有 $\mu (A) \geq \mu (\varnothing) = 0$;(ii) 若 $A_{i} \in {\mathcal{F}}$ 是可数列不交集合,则
\[
\mu \left( \bigcup_{i} A_{i} \right) = \sum_{i} \mu(A_{i})
\]
若 $\mu(\Omega) = 1$,则称 $\mu$ 为概率测度.

    \end{dfn}
    
    
    
    \begin{thm}
        [Theorem-on-Generating-Random-Variables-Using-the-Inverse-Distribution-Function]
        {利用分布函数反函数生成随机变量的定理}
        [Theorem on Generating Random Variables Using the Inverse Distribution Function]
        [gpt-4.1]
        设 $U$ 是在区间 $[0,1]$ 上均匀分布的随机变量,$F$ 是单调递增且右连续的分布函数,则随机变量 $X = F^{-1}(U)$ 的分布函数为 $F$.
    \end{thm}
    
    
    
    \begin{thm}
        [Boundedness-of-Moments-and-Inequality-for-Lower-Moments]
        {矩的有界性与次矩的有界性及不等式}
        [Boundedness of Moments and Inequality for Lower Moments]
        [gpt-4.1]
        
若 $E |X|^{k} < \infty$,则对于 $0 < j < k$,有 $E |X|^{j} < \infty$,并且满足
\[
E |X|^{j} \leq (E |X|^{k})^{j/k}
\]

    \end{thm}
    
    
    
    \begin{axm}
        [Application-of-the-Axiom-of-Choice-and-the-Continuum-Hypothesis]
        {选择公理与连续统假设的应用}
        [Application of the Axiom of Choice and the Continuum Hypothesis]
        [gpt-4.1]
        由选择公理和连续统假设,可以在 $(0,1)$ 上定义一个序关系 $<'$,使得对每个 $y$,集合 $\{ x : x <' y \}$ 是可数的.
    \end{axm}
    
    
    
    \begin{thm}
        [Theorem-on-Integral-and-Error-Estimate]
        {积分与误差估计定理}
        [Theorem on Integral and Error Estimate]
        [gpt-4.1]
        
设 $a \geq 1$,则有积分公式:

\[
\int_0^a \frac{\sin x}{x} dx = \arctan(a) - (\cos a) \int_0^{\infty} \frac{e^{-a y}}{1 + y^2} dy - (\sin a) \int_0^{\infty} \frac{y e^{-a y}}{1 + y^2} dy
\]

并且有误差估计:

\[
\left| \int_0^a \frac{\sin x}{x} dx - \arctan(a) \right| \leq \frac{2}{a}
\]

    \end{thm}
    
    
    
    \begin{thm}
        [Dominated-Convergence-Theorem]
        {主导收敛定理}
        [Dominated Convergence Theorem]
        [gpt-4.1]
        
若 $X_n \to X$ 几乎处处收敛,且对所有 $n$ 有 $|X_n| \leq Y$,其中 $E Y < \infty$,则 $E X_n \to E X$.

    \end{thm}
    
    
    
    \begin{dfn}
        [Definition-of-Convergence-in-Measure]
        {测度收敛的定义}
        [Definition of Convergence in Measure]
        [gpt-4.1]
        称 $f_{n} \to f$ 在测度意义下收敛, 即对于任意 $\epsilon > 0$, 有 $\mu(\{ x : |f_{n}(x) - f(x)| > \epsilon \}) \to 0$ 当 $n \to \infty$.
    \end{dfn}
    
    
    
    \begin{thm}
        [Independence-of-σ-Algebras-Generated-by-Independent-Families]
        {独立族的σ代数独立性}
        [Independence of σ-Algebras Generated by Independent Families]
        [gpt-4.1]
        如果 $\mathcal{A}_1, \mathcal{A}_2, \ldots, \mathcal{A}_n$ 是独立的,则 $\sigma({\mathcal{A}}_1), \sigma({\mathcal{A}}_2), \ldots, \sigma({\mathcal{A}}_n)$ 也是独立的.
    \end{thm}
    
    
    
    \begin{lma}
        [Expectation-Formula-for-Powers-of-Nonnegative-Random-Variables]
        {非负随机变量的幂的期望公式}
        [Expectation Formula for Powers of Nonnegative Random Variables]
        [gpt-4.1]
        若 $Y \geq 0$ 且 $p > 0$,则
\[
E( Y^{p} ) = \int_{0}^{\infty} p y^{p-1} P( Y > y ) d y.
\]

    \end{lma}
    
    
    
    \begin{ppt}
        [Variance-Property-under-Linear-Transformation-of-Variable]
        {变量线性变换的方差性质}
        [Variance Property under Linear Transformation of Variable]
        [gpt-4.1]
        对于任意随机变量 $X$ 及常数 $a, b$,有
\[
\operatorname{var}( a X + b ) = a^{2} \operatorname{var}( X )
\]
即线性变换 $aX + b$ 的方差等于原方差乘以 $a^2$.

    \end{ppt}
    
    
    
    \begin{thm}
        [Mean-and-Variance-of-Exponential-Distribution]
        {指数分布的均值与方差}
        [Mean and Variance of Exponential Distribution]
        [gpt-4.1]
        设 $Y = X / \lambda$,则 $Y$ 的概率密度为 $\lambda e^{-\lambda y}$,定义在 $y \ge 0$,即参数为 $\lambda$ 的指数分布.由定理 1.6.1 (b) 和式 (1.6.4) 可知,$Y$ 的均值为 $1/\lambda$,方差为 $1/\lambda^{2}$.
    \end{thm}
    
    
    
    \begin{thm}
        [Bonferroni-Inequalities]
        {Bonferroni不等式}
        [Bonferroni Inequalities]
        [gpt-4.1]
        
设 $A_1, A_2, \ldots, A_n$ 是事件,$A = \cup_{i=1}^n A_i$,则有以下不等式:

\[
\begin{array}{rl}
& P\left(\bigcup_{i=1}^n A_i\right) \leq \sum_{i=1}^n P(A_i) \\
& P\left(\bigcup_{i=1}^n A_i\right) \geq \sum_{i=1}^n P(A_i) - \sum_{i<j} P(A_i \cap A_j) \\
& P\left(\bigcup_{i=1}^n A_i\right) \leq \sum_{i=1}^n P(A_i) - \sum_{i<j} P(A_i \cap A_j) + \sum_{i<j<k} P(A_i \cap A_j \cap A_k)
\end{array}
\]

    \end{thm}
    
    
    
    \begin{thm}
        [$\pi$-$\lambda$-Theorem]
        {$\pi$-$\lambda$定理}
        [$\pi$-$\lambda$ Theorem]
        [gpt-4.1]
        如果 $\mathcal{P}$ 是一个 $\pi$-系统,$\mathcal{L}$ 是一个包含 $\mathcal{P}$ 的 $\lambda$-系统,则有 $\sigma(\mathcal{P}) \subset \mathcal{L}$.
    \end{thm}
    
    
    
    \begin{dfn}
        [Definition-of-Nice-Space]
        {nice空间的定义}
        [Definition of Nice Space]
        [gpt-4.1]
        $(S, S)$ 被称为 nice,当且仅当存在从 $S$ 到 $\mathbf{R}$ 的一一映射 $\varphi$,使得 $\varphi$ 与 $\varphi^{-1}$ 都是可测的.
    \end{dfn}
    
    
    
    \begin{thm}
        [Theorem-on-the-Convergence-of-the-Mean-of-i.i.d.-Random-Variables]
        {独立同分布随机变量的均值收敛定理}
        [Theorem on the Convergence of the Mean of i.i.d. Random Variables]
        [gpt-4.1]
        设 $X_1, X_2, \dots$ 是独立同分布的随机变量,若 $E X_i^2 < \infty$,则当 $n \to \infty$ 时,$S_n / n$ 以概率收敛于 $\mu = E X_i$,其中 $S_n = X_1 + X_2 + \cdots + X_n$.
    \end{thm}
    
    
    
    \begin{thm}
        [A-Lower-Bound-for-the-Probability-of-a-Nonnegative-Random-Variable-Being-Positive]
        {关于非负随机变量取正概率的下界}
        [A Lower Bound for the Probability of a Nonnegative Random Variable Being Positive]
        [gpt-4.1]
        设 $Y \geq 0$ 且 $E Y^2 < \infty$,则有
\[
P(Y > 0) \geq \frac{(E Y)^2}{E Y^2}
\]
其中 $E$ 表示期望.""

    \end{thm}
    
    
    
    \begin{xmp}
        [Example-Indicator-Function-of-a-Set-as-a-Random-Variable]
        {集合的指示函数作为随机变量的例子}
        [Example: Indicator Function of a Set as a Random Variable]
        [gpt-4.1]
        集合 $A \in {\mathcal{F}}$ 的指示函数定义如下:
\[
1_{A}(\omega) = \left\{
\begin{array}{ll}
1 & \omega \in A \\
0 & \omega 
otin A
\end{array}
\right.
\]
这是随机变量的一个简单但有用的例子.

    \end{xmp}
    
    
    
    \begin{thm}
        [Independence-of-Functions-of-Independent-Random-Variables]
        {独立随机变量函数的独立性}
        [Independence of Functions of Independent Random Variables]
        [gpt-4.1]
        若对 $1 \leq i \leq n$,$1 \leq j \leq m(i)$,$X_{i, j}$ 相互独立,且 $f_{i}: \mathbf{R}^{m(i)} \to \mathbf{R}$ 是可测函数,则 $f_{i}(X_{i, 1}, \ldots, X_{i, m(i)})$ 也是相互独立的.
    \end{thm}
    
    
    
    \begin{thm}
        [Theorem-on-Independence-of-Random-Variables-and-Their-Generated-Sigma-Algebras]
        {独立随机变量与生成σ代数的独立性定理}
        [Theorem on Independence of Random Variables and Their Generated Sigma-Algebras]
        [gpt-4.1]
        
(i) 如果随机变量 $X$ 和 $Y$ 独立,则 $\sigma(X)$ 和 $\sigma(Y)$ 独立.

(ii) 反之,如果 $\mathcal{F}$ 和 $\mathcal{G}$ 独立,$X \in \mathcal{F}$,$Y \in \mathcal{G}$,则 $X$ 和 $Y$ 独立.

    \end{thm}
    
    
    
    \begin{thm}
        [Convergence-Theorem-on-Spaces-of-Finite-Measure]
        {空间有限测度下的收敛定理}
        [Convergence Theorem on Spaces of Finite Measure]
        [gpt-4.1]
        在有限测度空间上,假设 $f_{n} \to f$ 的条件比 $f_{n} \to f$ a.e. 更弱,但下述结论在更一般的条件下更容易证明.
    \end{thm}
    
    
    
    \begin{thm}
        [A-Conclusion-of-the-Monotone-Convergence-Theorem]
        {单调收敛定理的一个结论}
        [A Conclusion of the Monotone Convergence Theorem]
        [gpt-4.1]
        若 $E X_{1}^{-} < \infty$ 且 $X_{n} \uparrow X$,则 $E X_{n} \uparrow E X$.
    \end{thm}
    
    
    
    \begin{dfn}
        [Notation-for-the-Integral-of-a-Function-over-a-Set]
        {集合上函数积分的记号}
        [Notation for the Integral of a Function over a Set]
        [gpt-4.1]
        
我们定义 $f$ 在集合 $E$ 上的积分为:

\[
\int_{E} f d \mu \equiv \int f \cdot 1_{E} d \mu
\]

其中 $1_{E}$ 是集合 $E$ 的示性函数.

    \end{dfn}
    
    
    
    \begin{dfn}
        [Definition-of-the-Family-of-Sets-$\mathcal{L}$]
        {关于集合族 $\mathcal{L}$ 的定义}
        [Definition of the Family of Sets $\mathcal{L}$]
        [gpt-4.1]
        设 $A_2, \ldots, A_n$ 为集合,$A_i \in \mathcal{A}_i$,令 $F = A_2 \cap \cdots \cap A_n$,定义集合族
\[
\mathcal{L} = \{ A : P(A \cap F) = P(A) P(F) \}
\]
其中 $P$ 表示概率.
    \end{dfn}
    
    
    
    \begin{thm}
        [Uniform-Convergence-Theorem]
        {一致收敛定理}
        [Uniform Convergence Theorem]
        [gpt-4.1]
        
Then as $n \to \infty$

\[
\sup_{x \in [0,1]} |f_n(x) - f(x)| \to 0
\]

    \end{thm}
    
    
    
    \begin{prf}
        [Proof-of-Uniform-Convergence-Theorem]
        {一致收敛定理的证明}
        [Proof of Uniform Convergence Theorem]
        [gpt-4.1]
        
First observe that if $S_n$ is the sum of $n$ independent random variables with $P(X_i = 1) = p$ and $P(X_i = 0) = 1 - p$, then $E X_i = p$, $\operatorname{var}(X_i) = p(1-p)$ and

\[
P(S_n = m) = \binom{n}{m} p^m (1-p)^{n-m}
\]

so $E f(S_n / n) = f_n(p)$.

    \end{prf}
    
    
    
    \begin{dfn}
        [Definition-of-Discrete-Probability-Measure]
        {离散概率测度的定义}
        [Definition of Discrete Probability Measure]
        [gpt-4.1]
        若存在一个可数集 $S$ 使得 $P(S^c) = 0$,则称概率测度 $P$(或其对应的分布函数)为离散的.
    \end{dfn}
    
    
    
    \begin{dfn}
        [Definition-of-Exponential-Distribution]
        {指数分布的定义}
        [Definition of Exponential Distribution]
        [gpt-4.1]
        设 $\lambda > 0$,指数分布(率参数为 $\lambda$)的概率密度函数为
\[
f(x) = \lambda e^{-\lambda x}
\]
对所有 $x \geq 0$,否则取值为 0.
    \end{dfn}
    
    
    
    \begin{dfn}
        [Distribution-Function-of-Exponential-Distribution]
        {指数分布的分布函数}
        [Distribution Function of Exponential Distribution]
        [gpt-4.1]
        指数分布(率参数为 $\lambda$)的分布函数为
\[
F(x) =
\begin{cases}
0 & x < 0 \\
1 - e^{-\lambda x} & x \geq 0
\end{cases}
\]
    \end{dfn}
    
    
    
    \begin{thm}
        [Variance-Formula-for-the-Sum-of-Uncorrelated-Random-Variables]
        {不相关随机变量和的方差公式}
        [Variance Formula for the Sum of Uncorrelated Random Variables]
        [gpt-4.1]
        
设 $X_1, \ldots, X_n$ 满足 $E(X_i^2) < \infty$ 且两两不相关,则有
\[
\operatorname{var}(X_1 + \cdots + X_n) = \operatorname{var}(X_1) + \cdots + \operatorname{var}(X_n)
\]
其中 $\operatorname{var}(Y)$ 表示随机变量 $Y$ 的方差.

    \end{thm}
    
    
    
    \begin{lma}
        [Fatous-Lemma]
        {Fatou 引理}
        [Fatou's Lemma]
        [gpt-4.1]
        
设 $X_n \geq 0$ 且 $X_n \to X$ 以概率收敛.
则有
\[
\liminf_{n \to \infty} E X_n \geq E X.
\]

    \end{lma}
    
    
    
    \begin{prf}
        [Proof-of-Independence-of-Collection-$\mathcal{A}-i$]
        {关于集合族 $\mathcal{A}_i$ 独立性的证明}
        [Proof of Independence of Collection $\mathcal{A}_i$]
        [gpt-4.1]
        证明 设 $\mathcal{A}_i$ 是由形如 $\cap_{j} A_{i, j}$ 的集合构成的集合族,其中 $A_{i, j} \in \mathcal{F}_{i, j}$.$\mathcal{A}_i$ 是一个包含 $\Omega$ 且包含 $\cup_{j} \mathcal{F}_{i, j}$ 的 $\pi$-系统,因此由定理 2.1.7 可知 $\sigma(\mathcal{A}_i) = \mathcal{G}_i$ 是独立的.
    \end{prf}
    
    
    
    \begin{dfn}
        [Definition-of-Countably-Generated-$\sigma$-Field]
        {可数生成的 $\sigma$-域的定义}
        [Definition of Countably Generated $\sigma$-Field]
        [gpt-4.1]
        一个 $\sigma$-域 $\mathcal{F}$ 被称为可数生成的,如果存在一个可数集合 $\mathcal{C} \subset \mathcal{F}$,使得 $\sigma(\mathcal{C}) = \mathcal{F}$.
    \end{dfn}
    
    
    
    \begin{thm}
        [One-sided-Chebyshev-Bound]
        {单边Chebyshev界}
        [One-sided Chebyshev Bound]
        [gpt-4.1]
        
设 $a > b > 0$, $0 < p < 1$, 且随机变量 $X$ 满足 $P(X = a) = p$ 和 $P(X = -b) = 1 - p$.
应用定理 1.6.4 于 $\varphi(x) = (x + b)^2$,得:若 $Y$ 是任意满足 $E Y = E X$ 且 $\operatorname{var}(Y) = \operatorname{var}(X)$ 的随机变量,则有
\[
P(Y \geq a) \leq p,
\]
当 $Y = X$ 时等号成立.

    \end{thm}
    
    
    
    \begin{xmp}
        [Example-of-limsup-set-and-divergent-series-in-Borel-measure-space]
        {Borel测度空间中极限上集合与级数发散的例子}
        [Example of limsup set and divergent series in Borel measure space]
        [gpt-4.1]
        设 $\Omega = (0, 1)$, ${\mathcal{F}} =$ Borel 集合, $P =$ Lebesgue 测度.令 $A_{n} = (0, a_{n})$,其中 $a_{n} \to 0$ 当 $n \to \infty$ 时,则 $\limsup A_{n} = \varnothing$,但若 $a_{n} \geq 1/n$,有 $\textstyle\sum a_{n} = \infty$.
    \end{xmp}
    
    
    
    \begin{thm}
        [Theorem-on-the-Distribution-of-the-Sum-of-Normal-Variables]
        {正态分布的和的分布定理}
        [Theorem on the Distribution of the Sum of Normal Variables]
        [gpt-4.1]
        若 $X = \mathrm{normal}(\mu, a)$ 且 $Y = \mathrm{normal}(
u, b)$ 独立,则 $X + Y = \mathrm{normal}(\mu + 
u, a + b)$.
    \end{thm}
    
    
    
    \begin{thm}
        [Probability-Formula-for-the-Sum-of-Independent-Integer-Valued-Random-Variables]
        {独立整数值随机变量和的概率公式}
        [Probability Formula for the Sum of Independent Integer-Valued Random Variables]
        [gpt-4.1]
        
设 $X$ 和 $Y$ 是独立的、取整数值的随机变量,则有
\[
P(X+Y=n) = \sum_{m} P(X=m) P(Y=n-m)
\]

    \end{thm}
    
    
    
    \begin{lma}
        [Linearity-of-Integral-for-Difference-of-Non-Negative-Functions]
        {积分的线性性(非负函数差)}
        [Linearity of Integral for Difference of Non-Negative Functions]
        [gpt-4.1]
        
如果 $f = f_1 - f_2$,其中 $f_1, f_2 \geq 0$ 且 $\int f_i \, d\mu < \infty$,则有
\[
\int f \, d\mu = \int f_1 \, d\mu - \int f_2 \, d\mu
\]

    \end{lma}
    
    
    
    \begin{prf}
        [Proof-of-Lemma-on-Linearity-of-Integral]
        {积分线性性引理的证明}
        [Proof of Lemma on Linearity of Integral]
        [gpt-4.1]
        
$f_1 + f^- = f_2 + f^+$,且这四个函数都 $\geq 0$,因此根据引理 1.4.5 的(iii)有
\[
\int f_1 \, d\mu + \int f^- \, d\mu = \int (f_1 + f^-) \, d\mu = \int (f_2 + f^+) \, d\mu = \int f_2 \, d\mu + \int f^+ \, d\mu
\]
整理即可得到所需结论.

    \end{prf}
    
    
    
    \begin{thm}
        [Application-of-Borel-Cantelli-Lemma]
        {Borel-Cantelli 引理的应用}
        [Application of Borel-Cantelli Lemma]
        [gpt-4.1]
        
设 $T_k$ 为一列随机变量,且存在
\[
P ( | T _ { k } - E T _ { k } | > \delta E T _ { k } ) \leq \frac{1}{\delta^2 k^2}
\]
对于任意 $\delta > 0$,则有
\[
\sum_{k=1}^{\infty} P( | T_k - E T_k | > \delta E T_k ) < \infty
\]
由 Borel-Cantelli 引理可知
\[
P( | T_k - E T_k | > \delta E T_k \text{ i.o.} ) = 0
\]
由于 $\delta$ 任意,故 $T_k / E T_k \to 1$ 几乎必然成立.

    \end{thm}
    
    
    
    \begin{thm}
        [Independence-of-σ-fields-Generated-by-Unions-of-Independent-σ-fields]
        {关于独立 σ-域并集生成的 σ-域的独立性}
        [Independence of σ-fields Generated by Unions of Independent σ-fields]
        [gpt-4.1]
        
设 $\mathcal{F}_{i, j}$,其中 $1 \leq i \leq n, 1 \leq j \leq m(i)$ 是独立的,令 $\mathcal{G}_i = \sigma\left(\cup_{j} \mathcal{F}_{i, j}\right)$,则 $\mathcal{G}_1, \ldots, \mathcal{G}_n$ 是独立的.

    \end{thm}
    
    
    
    \begin{dfn}
        [Definition-of-Gamma-Distribution-Probability-Density-Function]
        {伽玛分布的概率密度函数定义}
        [Definition of Gamma Distribution Probability Density Function]
        [gpt-4.1]
        
伽玛分布的概率密度函数由参数 $\alpha$ 和 $\lambda$ 给出:

\[
f(x) =
\begin{cases}
\lambda^{\alpha} x^{\alpha - 1} e^{-\lambda x} / \Gamma(\alpha) & \text{当 } x \geq 0 \\
0 & \text{当 } x < 0
\end{cases}
\]

其中 $\Gamma(\alpha) = \int_{0}^{\infty} x^{\alpha - 1} e^{-x} dx$.

    \end{dfn}
    
    
    
    \begin{dfn}
        [Definition-of-Truncation-of-Random-Variable]
        {随机变量的截断定义}
        [Definition of Truncation of Random Variable]
        [gpt-4.1]
        将随机变量 $X$ 在水平 $M$ 处截断是指考虑

\[
\bar{X} = X 1_{(|X| \leq M)} = 
\begin{cases}
X & \text{if } |X| \leq M \\
0 & \text{if } |X| > M
\end{cases}
\]

其中,$\bar{X}$ 是 $X$ 的截断版本,当 $|X| \leq M$ 时取 $X$,否则取 $0$.

    \end{dfn}
    
    
    
    \begin{dfn}
        [Definition-of-Asymptotic-Density-of-a-Set]
        {集合的渐近密度的定义}
        [Definition of Asymptotic Density of a Set]
        [gpt-4.1]
        
若 $A \subset \{1, 2, \ldots\}$,称 $A$ 具有渐近密度 $\theta$,如果
\[
\operatorname*{lim}_{n \to \infty} |A \cap \{1, 2, \ldots, n\}| / n = \theta
\]
成立.

    \end{dfn}
    
    
    
    \begin{dfn}
        [Family-of-Sets-with-Existing-Asymptotic-Density]
        {渐近密度存在的集合的族}
        [Family of Sets with Existing Asymptotic Density]
        [gpt-4.1]
        
设 $\mathcal{A}$ 是所有渐近密度存在的集合的集合.

    \end{dfn}
    
    
    
    \begin{thm}
        [Composition-of-Measurable-Functions-is-Measurable]
        {可测函数的复合仍为可测}
        [Composition of Measurable Functions is Measurable]
        [gpt-4.1]
        如果 $X : (\Omega, \mathcal{F}) \to (S, \mathcal{S})$ 和 $f : (S, \mathcal{S}) \to (T, \mathcal{T})$ 都是可测映射,则 $f(X)$ 也是从 $(\Omega, \mathcal{F})$ 到 $(T, \mathcal{T})$ 的可测映射.
    \end{thm}
    
    
    
    \begin{xmp}
        [Example-on-Pairwise-Independence-versus-Mutual-Independence-of-Events]
        {关于事件对独立性与整体独立性的例子}
        [Example on Pairwise Independence versus Mutual Independence of Events]
        [gpt-4.1]
        
设 $X_1, X_2, X_3$ 为相互独立的随机变量,满足
\[
P(X_i = 0) = P(X_i = 1) = 1/2
\]
令 $A_1 = \{ X_2 = X_3 \}$, $A_2 = \{ X_3 = X_1 \}$, $A_3 = \{ X_1 = X_2 \}$.

这些事件是两两独立的,因为若 $i 
eq j$,则
\[
P(A_i \cap A_j) = P(X_1 = X_2 = X_3) = 1/4 = P(A_i) P(A_j)
\]
但它们不是互相独立的,因为
\[
P(A_1 \cap A_2 \cap A_3) = 1/4 
eq 1/8 = P(A_1) P(A_2) P(A_3)
\]

    \end{xmp}
    
    
    
    \begin{thm}
        [Fatous-Lemma]
        {Fatou 引理}
        [Fatou's Lemma]
        [gpt-4.1]
        若 $f_{n} \geq 0$,则有
\[
\liminf_{n \to \infty} \int f_{n} \, d\mu \geq \int \left( \liminf_{n \to \infty} f_{n} \right) d\mu
\]

    \end{thm}
    
    
    
    \begin{dfn}
        [Definition-of-the-Integral-of-a-Non-negative-Function]
        {非负函数的积分的定义}
        [Definition of the Integral of a Non-negative Function]
        [gpt-4.1]
        
若 $f \geq 0$,则定义
\[
\int f d \mu = \operatorname*{sup} \left\{ \int h d \mu : 0 \leq h \leq f,\, h \text{ 是有界函数且 } \mu ( \{ x : h ( x ) > 0 \} ) < \infty \right\}
\]

    \end{dfn}
    
    
    
    \begin{xmp}
        [Example-of-Point-Mass-at-Zero]
        {零点处的离散分布例子}
        [Example of Point Mass at Zero]
        [gpt-4.1]
        
例 1.2.8:点质量分布在 $0$ 处.其分布函数为:$F(x) = 1$ 当 $x \geq 0$,$F(x) = 0$ 当 $x < 0$.

    \end{xmp}
    
    
    
    \begin{ppt}
        [Basic-Properties-of-Distribution-Functions]
        {分布函数的基本性质}
        [Basic Properties of Distribution Functions]
        [gpt-4.1]
        任何分布函数 $F$ 都具有以下性质:

(i) $F$ 是非递减的.

(ii) $\lim_{x \to \infty} F(x) = 1, \quad \lim_{x \to -\infty} F(x) = 0$.

(iii) $F$ 是右连续的,即 $\lim_{y \downarrow x} F(y) = F(x)$.

(iv) 若 $F(x-) = \lim_{y \uparrow x} F(y)$,则 $F(x-) = P(X < x)$.

(v) $P(X = x) = F(x) - F(x-)$.

    \end{ppt}
    
    
    
    \begin{prf}
        [Proof-of-Monotonicity-of-Distribution-Functions]
        {分布函数非递减性的证明}
        [Proof of Monotonicity of Distribution Functions]
        [gpt-4.1]
        证明 (i): 注意如果 $x \leq y$,则 $\{X \leq x\} \subset \{X \leq y\}$,利用定理 1.1 中的 (i) 可得 $P(X \leq x) \leq P(X \leq y)$.
    \end{prf}
    
    
    
    \begin{dfn}
        [Definition-of-Measure]
        {测度的定义}
        [Definition of Measure]
        [gpt-4.1]
        我们令 $\mu ( A ) = \Delta _ { A } F$,因此我们必须假设

(iv) 对任意矩形 $A$,有 $\Delta _ { A } F \ge 0$.
    \end{dfn}
    
    
    
    \begin{thm}
        [Borel-Cantelli-Lemma]
        {Borel-Cantelli 引理}
        [Borel-Cantelli Lemma]
        [gpt-4.1]
        
如果 $\sum_{n=1}^{\infty} P(A_{n}) < \infty$,则
\[
P(A_{n} \text{ i.o.}) = 0,
\]
即几乎必然只有有限个事件 $A_n$ 发生.

    \end{thm}
    
    
    
    \begin{prf}
        [Proof-of-Borel-Cantelli-Lemma]
        {Borel-Cantelli 引理的证明}
        [Proof of Borel-Cantelli Lemma]
        [gpt-4.1]
        
令 $N = \sum_{k} 1_{A_{k}}$ 为发生的事件总数.Fubini 定理推出 $EN = \sum_{k} P(A_{k}) < \infty$,因此几乎必然有 $N < \infty$.

    \end{prf}
    
    
    
    \begin{dfn}
        [Definition-of-Simple-Function-on-Measurable-Space]
        {可测空间上简单函数的定义}
        [Definition of Simple Function on Measurable Space]
        [gpt-4.1]
        
设 $\varphi : \Omega \to \mathbf{R}$,若存在实数 $c_1, \ldots, c_n$ 以及 $A_1, \ldots, A_n \in \mathcal{F}$,使得
\[
\varphi(\omega) = \sum_{m=1}^n c_m 1_{A_m}(\omega)
\]
则称 $\varphi$ 是简单函数.

    \end{dfn}
    
    
    
    \begin{thm}
        [Limit-Theorem-for-i.i.d.-Variables-with-Infinite-Expectation]
        {独立同分布变量无穷大期望的极限定理}
        [Limit Theorem for i.i.d. Variables with Infinite Expectation]
        [gpt-4.1]
        
如果 $X_1, X_2, \dots$ 是独立同分布的随机变量,且 $E | X_i | = \infty$,那么
\[
P ( | X_n | \geq n \ \text{ i.o. } ) = 1.
\]

    \end{thm}
    
    
    
    \begin{thm}
        [Extension-of-Borel-Cantelli-Lemma-for-Pairwise-Independent-Events]
        {对独立事件序列的Borel-Cantelli引理的推广}
        [Extension of Borel-Cantelli Lemma for Pairwise Independent Events]
        [gpt-4.1]
        
设 $A_1, A_2, \ldots$ 是两两独立的事件,且 $\sum_{n=1}^{\infty} P ( A_n ) = \infty$,则当 $n \to \infty$ 时,
\[
\frac{ \sum_{m=1}^{n} 1_{A_m} }{ \sum_{m=1}^{n} P ( A_m ) } \to 1 \quad \text{a.s.}
\]

    \end{thm}
    
    
    
    \begin{crl}
        [Convergence-in-Probability-of-Law-of-Large-Numbers]
        {大数定律的概率收敛推论}
        [Convergence in Probability of Law of Large Numbers]
        [gpt-4.1]
        由 Chebyshev 不等式可得

\[
P ( | S_n - E S_n | > \delta E S_n ) \leq \operatorname{var} ( S_n ) / ( \delta E S_n )^2 \leq 1 / ( \delta^2 E S_n ) \to 0
\]

当 $n \to \infty$ 时(已知 $E S_n \to \infty$),由此可推出 $S_n / E S_n \to 1$ 在概率意义下收敛.

    \end{crl}
    
    
    
    \begin{thm}
        [Glivenko-Cantelli-Theorem]
        {Glivenko-Cantelli 定理}
        [Glivenko-Cantelli Theorem]
        [gpt-4.1]
        
当 $n \to \infty$ 时,
\[
\operatorname{sup}_{x} |F_{n}(x) - F(x)| \to 0 \quad \text{a.s.}
\]

    \end{thm}
    
    
    
    \begin{thm}
        [Properties-of-Complements-of-Independent-Events]
        {独立事件的补事件性质}
        [Properties of Complements of Independent Events]
        [gpt-4.1]
        (i) 若 $A$ 与 $B$ 相互独立,则 $A^c$ 与 $B$、$A$ 与 $B^c$ 以及 $A^c$ 与 $B^c$ 也相互独立.
    \end{thm}
    
    
    
    \begin{lma}
        [Properties-of-Integrals-of-Bounded-Functions-on-Sets-of-Finite-Measure]
        {关于有限测度集上的有界函数积分的性质}
        [Properties of Integrals of Bounded Functions on Sets of Finite Measure]
        [gpt-4.1]
        
设 $E$ 是满足 $\mu(E) < \infty$ 的集合.若 $f$ 和 $g$ 是在 $E^c$ 上为零的有界函数,则有:
(i) 若 $f \geq 0$ 几乎处处,则 $\int f d\mu \geq 0$.
(ii) 对任意 $a \in \mathbf{R}$,有 $\int a f d\mu = a \int f d\mu$.
(iii) 有 $\int (f + g) d\mu = \int f d\mu + \int g d\mu$.
(iv) 若 $g \leq f$ 几乎处处,则 $\int g d\mu \leq \int f d\mu$.
(v) 若 $g = f$ 几乎处处,则 $\int g d\mu = \int f d\mu$.
(vi) 有 $|\int f d\mu| \leq \int |f| d\mu$.

    \end{lma}
    
    
    
    \begin{thm}
        [Fatous-Lemma]
        {Fatou 引理}
        [Fatou's Lemma]
        [gpt-4.1]
        
如果 $X_{n} \geq 0$,则有
\[
\liminf_{n \to \infty} E X_{n} \geq E \left( \liminf_{n \to \infty} X_{n} \right)
\]

    \end{thm}
    
    
    
    \begin{thm}
        [Monotone-Convergence-Theorem]
        {单调收敛定理}
        [Monotone Convergence Theorem]
        [gpt-4.1]
        
如果 $v \leq X_{n} \uparrow X$,则 $E X_{n} \uparrow E X$.

    \end{thm}
    
    
    
    \begin{thm}
        [Convergence-in-Moment-Implies-Convergence-in-Probability]
        {矩的收敛蕴含概率收敛}
        [Convergence in Moment Implies Convergence in Probability]
        [gpt-4.1]
        若 $p > 0$ 且 $E|Z_n|^p \to 0$,则 $Z_n \to 0$ 依概率收敛.
    \end{thm}
    
    
    
    \begin{prf}
        [Proof-of-Convergence-in-Moment-Implies-Convergence-in-Probability]
        {矩的收敛蕴含概率收敛的证明}
        [Proof of Convergence in Moment Implies Convergence in Probability]
        [gpt-4.1]
        由Chebyshev不等式(定理1.6.4),取 $\varphi(x) = x^p$ 且 $X = |Z_n|$,则对于任意 $\epsilon > 0$ 有
\[
P(|Z_n| \ge \epsilon) \le \epsilon^{-p} E|Z_n|^p \to 0.
\]

    \end{prf}
    
    
    
    \begin{thm}
        [Fubinis-Theorem]
        {Fubini 定理}
        [Fubini's Theorem]
        [gpt-4.1]
        
设 $f \geq 0$ 或 $\int |f| d\mu < \infty$,则有
\[
\int_{X} \int_{Y} f(x, y) \mu_{2}(dy) \mu_{1}(dx) = \int_{X \times Y} f d\mu = \int_{Y} \int_{X} f(x, y) \mu_{1}(dx) \mu_{2}(dy)
\]

    \end{thm}
    
    
    
    \begin{thm}
        [Theorem-on-Generation-of-σ-Algebra]
        {生成σ代数的定理}
        [Theorem on Generation of σ-Algebra]
        [gpt-4.1]
        
若 $\mathcal{A}$ 生成 $\boldsymbol{\mathcal{S}}$,则 $X^{-1}( \mathcal{A} ) \equiv \{ \{ X \in A \} : A \in \mathcal{A} \}$ 生成 $\sigma(X) = \{ \{ X \in B \} : B \in \mathcal{S} \}$.

    \end{thm}
    
    
    
    \begin{dfn}
        [Definition-of-Conditions-for-Interchanging-Limit-and-Integral-of-Function-Sequence]
        {函数列极限的积分交换条件的定义}
        [Definition of Conditions for Interchanging Limit and Integral of Function Sequence]
        [gpt-4.1]
        
给定函数列 $\{f_n\}$,我们关注如下积分与极限交换的问题:

\[
\lim_{n \to \infty} \int f_{n} d\mu = \int \left( \lim_{n \to \infty} f_{n} \right) d\mu
\]

为此,首先需要给出相关的定义.

    \end{dfn}
    
    
    
    \begin{dfn}
        [Definition-of-Conditional-Integral]
        {条件积分的定义}
        [Definition of Conditional Integral]
        [gpt-4.1]
        
如果我们仅在 $A \subset \Omega$ 上积分,则记为
\[
E(X; A) = \int_A X \, dP
\]

    \end{dfn}
    
    
    
    \begin{thm}
        [Chebyshevs-Inequality]
        {Chebyshev不等式}
        [Chebyshev's Inequality]
        [gpt-4.1]
        
设 $\varphi : \mathbf{R} \to \mathbf{R}$ 满足 $\varphi \geq 0$,令 $A \in \mathcal{F}$,并设 $i_A = \operatorname{inf}\{\varphi(y) : y \in A\}$.则
\[
i_A P(X \in A) \leq E(\varphi(X); X \in A) \leq E\varphi(X)
\]

    \end{thm}
    
    
    
    \begin{thm}
        [Integral-Formula-for-Probability-Measure]
        {概率测度积分公式}
        [Integral Formula for Probability Measure]
        [gpt-4.1]
        
假设概率测度 $\mu$ 满足 $\mu(A) = \int_A f(x) dx$ 对所有 $A \in {\mathcal{R}}$ 成立,则对于任意 $g$,若 $g \geq 0$ 或 $\int |g(x)| \mu(dx) < \infty$,有
\[
\int g(x) \mu(dx) = \int g(x) f(x) dx
\]

    \end{thm}
    
    
    
    \begin{prf}
        [Proof-on-Independence-and-Measurable-Spaces]
        {关于独立性及测度空间的证明}
        [Proof on Independence and Measurable Spaces]
        [gpt-4.1]
        (i) 如果 $A \in \sigma(X)$,则由 $\sigma(X)$ 的定义可知 $A = \{ X \in C \}$,其中 $C \in \mathcal{R}$.
    
类似地,如果 $B \in \sigma(Y)$,则 $B = \{ Y \in D \}$,其中 $D \in \mathcal{R}$.利用这些事实以及 $X$ 和 $Y$ 的独立性,有:
\[
P(A \cap B) = P(X \in C, Y \in D) = P(X \in C) P(Y \in D) = P(A) P(B)
\]

(ii) 反之,如果 $X \in \mathcal{F}$,$Y \in \mathcal{G}$,且 $C, D \in \mathcal{R}$,由可测性的定义可得 $\{ X \in C \} \in \mathcal{F}$,$\{ Y \in D \} \in \mathcal{G}$.

由于 $\mathcal{F}$ 和 $\mathcal{G}$ 独立,得:
\[
P(X \in C, Y \in D) = P(X \in C) P(Y \in D)
\]

    \end{prf}
    
    
    
    \begin{thm}
        [Extension-of-the-$L^{2}$-Weak-Law-to-Dependent-Sequences]
        {$L^{2}$弱大数定律的依赖序列推广}
        [Extension of the $L^{2}$ Weak Law to Dependent Sequences]
        [gpt-4.1]
        
设随机变量序列 $\{X_n\}$ 满足 $E X_{n} = 0$,并对任意 $m \leq n$ 有 $E X_{n} X_{m} \leq r(n-m)$,其中 $r(k) \to 0$ 当 $k \to \infty$.则有
\[
\frac{X_{1} + \cdots + X_{n}}{n} \xrightarrow{P} 0,
\]
即该序列的均值收敛于零(概率收敛).

    \end{thm}
    
    
    
    \begin{dfn}
        [Definition-of-Algebra-Field]
        {代数(域)的定义}
        [Definition of Algebra (Field)]
        [gpt-4.1]
        一个集合 $\Omega$ 的子集族 $\mathcal{A}$ 称为代数(或域),如果对于任意 $A, B \in \mathcal{A}$,都有 $A^c$ 和 $A \cup B$ 也属于 $\mathcal{A}$.
    \end{dfn}
    
    
    
    \begin{dfn}
        [Definition-of-μs-and-νs-Related-to-Distribution-Function]
        {分布函数相关的μ(s)与ν(s)的定义}
        [Definition of μ(s) and ν(s) Related to Distribution Function]
        [gpt-4.1]
        
设分布函数为 $F(x)$,定义
$\mu(s) = \int_{0}^{s} x\,dF(x)$,
$
u(s) = \mu(s) / [s (1 - F(s))]$.

    \end{dfn}
    
    
    
    \begin{thm}
        [Necessary-and-Sufficient-Condition-for-Weak-Law-of-Large-Numbers-for-Positive-Variables]
        {正变量弱大数律的充要条件}
        [Necessary and Sufficient Condition for Weak Law of Large Numbers for Positive Variables]
        [gpt-4.1]
        
设 $X_{1}, X_{2}, \dots$ 是独立同分布的正随机变量,满足 $P(0 \leq X_{i} < \infty) = 1$ 且 $P(X_{i} > x) > 0$ 对任意 $x$ 成立.存在常数 $a_{n}$ 使 $S_{n}/a_{n} \to 1$ 以概率收敛,当且仅当 $
u(s) \to \infty$ 当 $s \to \infty$.

    \end{thm}
    
    
    
    \begin{xmp}
        [Occupancy-Problem-with-Random-Ball-Assignment]
        {随机分球入盒问题}
        [Occupancy Problem with Random Ball Assignment]
        [gpt-4.1]
        假设将 $r$ 个球随机放入 $n$ 个盒子,即所有 $n^r$ 个球的分配方式概率相等.设 $A_i$ 为第 $i$ 个盒子为空事件,$N_n$ 为空盒子的个数,则有:
\[
P(A_i) = (1 - 1/n)^r \qquad \mathrm{and} \qquad E N_n = n (1 - 1/n)^r
\]
进一步,若 $r/n \to c$,则 $E N_n / n \to e^{-c}$.

    \end{xmp}
    
    
    
    \begin{dfn}
        [Definition-of-Measurable-Map-and-Random-Variable]
        {测度空间之间的可测映射与随机变量的定义}
        [Definition of Measurable Map and Random Variable]
        [gpt-4.1]
        设 $(\Omega, \mathcal{F})$ 和 $(S, \mathcal{S})$ 是两个测度空间.函数 $X : \Omega \to S$ 称为从 $(\Omega, \mathcal{F})$ 到 $(S, \mathcal{S})$ 的可测映射,如果对于任意 $B \in \mathcal{S}$,都有
\[
X^{-1}(B) \equiv \{ \omega : X(\omega) \in B \} \in \mathcal{F}
\]
若 $(S, \mathcal{S}) = (\mathbf{R}^{d}, \mathcal{R}^{d})$ 且 $d > 1$,则 $X$ 称为随机向量;若 $d = 1$,则 $X$ 称为随机变量(简称 r.v.).

    \end{dfn}
    
    
    
    \begin{xmp}
        [Example-of-the-St.-Petersburg-Paradox]
        {圣彼得堡悖论例子}
        [Example of the St. Petersburg Paradox]
        [gpt-4.1]
        
设 $X_{1}, X_{2}, \dots$ 为相互独立的随机变量,且
\[
P(X_{i} = 2^{j}) = 2^{-j} \quad \mathrm{for~} j \geq 1
\]
换句话说,当正面首次出现于第 $j$ 次抛掷时,你赢得 $2^{j}$ 美元.这个悖论在于 $E X_{1} = \infty$,但实际上你并不会为游戏支付无限金额.

    \end{xmp}
    
    
    
    \begin{dfn}
        [Definition-of-Poisson-Distribution]
        {Poisson分布的定义}
        [Definition of Poisson Distribution]
        [gpt-4.1]
        Poisson分布的参数为$\lambda$,其概率质量函数为:$P(Z=k) = e^{-\lambda} \lambda^k / k!$,其中$k=0,1,2,\ldots$.
    \end{dfn}
    
    
    
    \begin{thm}
        [Additivity-Theorem-of-Poisson-Distribution]
        {Poisson分布的可加性定理}
        [Additivity Theorem of Poisson Distribution]
        [gpt-4.1]
        若$X = \operatorname{Poisson}(\lambda)$且$Y = \operatorname{Poisson}(\mu)$相互独立,则$X+Y$服从参数为$\lambda+\mu$的Poisson分布,即$X+Y = \operatorname{Poisson}(\lambda + \mu)$.
    \end{thm}
    
    
    
    \begin{thm}
        [Union-of-Increasing-Sigma-Algebras-is-an-Algebra]
        {σ-代数递增并集是代数}
        [Union of Increasing Sigma-Algebras is an Algebra]
        [gpt-4.1]
        如果 ${\mathcal{F}}_{1} \subset {\mathcal{F}}_{2} \subset \ldots$ 是一组递增的 $\sigma$-代数,则 $\cup_{i} {\mathcal{F}}_{i}$ 是一个代数.
    \end{thm}
    
    
    
    \begin{xmp}
        [Example-that-Union-is-not-a-Sigma-Algebra]
        {并集不是σ-代数的例子}
        [Example that Union is not a Sigma-Algebra]
        [gpt-4.1]
        存在一组递增的 $\sigma$-代数 ${\mathcal{F}}_{1} \subset {\mathcal{F}}_{2} \subset \ldots$,使得它们的并集 $\cup_{i} {\mathcal{F}}_{i}$ 不是一个 $\sigma$-代数.
    \end{xmp}
    
    
    
    \begin{dfn}
        [Definition-of-Lim-Sup-and-Lim-Inf]
        {极限上确界与极限下确界的定义}
        [Definition of Lim Sup and Lim Inf]
        [gpt-4.1]
        
若 $A_{n}$ 是 $\Omega$ 的子集序列, 则定义
\[
\limsup A_{n} = \lim_{m \to \infty} \bigcup_{n = m}^{\infty} A_{n} = \{\omega \text{ that are in infinitely many } A_{n}\}
\]
(该极限存在, 因为关于 $m$ 的序列是递减的),并定义
\[
\liminf A_{n} = \lim_{m \to \infty} \bigcap_{n = m}^{\infty} A_{n} = \{\omega \text{ that are in all but finitely many } A_{n}\}
\]
(该极限存在, 因为关于 $m$ 的序列是递增的).

    \end{dfn}
    
    
    
    \begin{dfn}
        [Definition-of-Lp-Norm]
        {Lp范数的定义}
        [Definition of Lp Norm]
        [gpt-4.1]
        
设 $1 \leq p < \infty$,对于可测函数 $f$,记
\[
\| f \|_{p} = \left( \int |f|^{p} d\mu \right)^{1/p}.
\]

    \end{dfn}
    
    
    
    \begin{ppt}
        [Homogeneity-of-Lp-Norm]
        {Lp范数的齐次性}
        [Homogeneity of Lp Norm]
        [gpt-4.1]
        
对于任意实数 $c$,有
\[
\| c f \|_{p} = |c| \cdot \| f \|_{p}.
\]

    \end{ppt}
    
    
    
    \begin{dfn}
        [Definition-and-Properties-of-Semi-algebra]
        {半代数的定义及其运算性质}
        [Definition and Properties of Semi-algebra]
        [gpt-4.1]
        
$\mathcal{S}$ 是一个半代数.对于集合 $A, C \subseteq X$,$B, D \subseteq Y$,有如下性质:
\[
(A \times B) \cap (C \times D) = (A \cap C) \times (B \cap D)
\]
\[
(A \times B)^{c} = (A^{c} \times B) \cup (A \times B^{c}) \cup (A^{c} \times B^{c})
\]

    \end{dfn}
    
    
    
    \begin{dfn}
        [Generation-of-Sigma-algebra]
        {$\sigma$-代数的生成}
        [Generation of Sigma-algebra]
        [gpt-4.1]
        
设 $\mathcal{F} = \mathcal{A} \times \mathcal{B}$,它是由半代数 $\mathcal{S}$ 生成的 $\sigma$-代数.

    \end{dfn}
    
    
    
    \begin{thm}
        [Bounded-Convergence-Theorem]
        {有界收敛定理}
        [Bounded Convergence Theorem]
        [gpt-4.1]
        设 $E$ 是一个集合,$\mu(E) < \infty$.假设 $f_n$ 在 $E^c$ 上为零,$|f_n(x)| \leq M$,且 $f_n \to f$ 在测度意义下收敛.那么
\[
\int f\, d\mu = \lim_{n \to \infty} \int f_n\, d\mu
\]

    \end{thm}
    
    
    
    \begin{thm}
        [Theorem-on-Interchanging-Integral-and-Summation-for-Non-Negative-Series]
        {非负可数和函数积分与和的交换定理}
        [Theorem on Interchanging Integral and Summation for Non-Negative Series]
        [gpt-4.1]
        如果 $g_{m} \geq 0$, 则
\[
\int \sum_{m = 0}^{\infty} g_{m} \, d\mu = \sum_{m = 0}^{\infty} \int g_{m} \, d\mu
\]

    \end{thm}
    
    
    
    \begin{xmp}
        [Example-of-Infinite-Expectation-but-Normalized-Convergence]
        {期望无穷但归一化后收敛的例子}
        [Example of Infinite Expectation but Normalized Convergence]
        [gpt-4.1]
        
设 $X_{1}, X_{2}, \dots$ 是独立同分布的随机变量,满足 $P(X_{i} > x) = e / (x \log x)$,其中 $x \geq e$.

证明:$E|X_{i}| = \infty$,但存在一列常数 $\mu_{n} \to \infty$ 使得 $S_{n} / n - \mu_{n} \to 0$ 依概率收敛.

    \end{xmp}
    
    
    
    \begin{thm}
        [Weak-Law-of-Large-Numbers-for-Triangular-Arrays]
        {三角阵弱大数律}
        [Weak Law of Large Numbers for Triangular Arrays]
        [gpt-4.1]
        
设对每个 $n$,$X_{n,k}$, $1 \leq k \leq n$ 彼此独立.令 $b_n > 0$ 且 $b_n \to \infty$,定义截断变量 $\bar{X}_{n,k} = X_{n,k} 1_{(|X_{n,k}| \leq b_n)}$.

假设当 $n \to \infty$ 时:
(i) $\sum_{k=1}^n P(|X_{n,k}| > b_n) \to 0$,
(ii) $b_n^{-2} \sum_{k=1}^n E \bar{X}_{n,k}^2 \to 0$.

则三角阵的均值满足弱大数律.

    \end{thm}
    
    
    
    \begin{dfn}
        [Definition-of-Independence-of-Collections-of-Sets]
        {集合族的独立性定义}
        [Definition of Independence of Collections of Sets]
        [gpt-4.1]
        设集合族 $\mathcal{A}_1, \mathcal{A}_2, \ldots, \mathcal{A}_n \subset \mathcal{F}$.当任取 $A_i \in \mathcal{A}_i$ 及 $I \subset \{ 1, \ldots, n \}$ 时,若
\[
P\left(\cap_{i \in I} A_i \right) = \prod_{i \in I} P(A_i),
\]
则称这些集合族是独立的.

当每个集合族仅包含一个集合,即 $\mathcal{A}_i = \{A_i\}$,上述定义退化为集合的独立性定义.

    \end{dfn}
    
    
    
    \begin{xmp}
        [Example-of-Product-Form-Functions-and-Lebesgue-Measure]
        {乘积形式函数的分布测度及Lebesgue测度的例子}
        [Example of Product Form Functions and Lebesgue Measure]
        [gpt-4.1]
        
设 $F(x) = \prod_{i=1}^{d} F_i(x_i)$,其中每个 $F_i$ 满足定理 1.1.4 的条件 (i) 和 (ii),则有
\[
\Delta_{A} F = \prod_{i=1}^{d} \left( F_i(b_i) - F_i(a_i) \right)
\]
当 $F_i(x) = x$ 对所有 $i$ 时,所得到的测度即为 $\mathbb{R}^d$ 上的Lebesgue测度.

    \end{xmp}
    
    
    
    \begin{dfn}
        [Assumptions-on-the-Defining-Function-F]
        {定义函数F的假设条件}
        [Assumptions on the Defining Function F]
        [gpt-4.1]
        设 $F$ 为定义在 $\mathbb{R}^d$ 上的函数,满足如下条件:
(i) $F$ 是单调不减的,即如果 $x \leq y$(即 $x_i \leq y_i$ 对所有 $i$),则 $F(x) \leq F(y)$;
(ii) $F$ 是右连续的,即 $\operatorname* { \lim }_{y \downarrow x} F(y) = F(x)$(其中 $y \downarrow x$ 表示每个 $y_i \downarrow x_i$);
(iii) 若 $x_n \downarrow -\infty$,即每个坐标都趋于 $-\infty$,则 $F(x_n) \downarrow 0$;
若 $x_n \uparrow +\infty$,即每个坐标都趋于 $+\infty$,则 $F(x_n) \uparrow 1$.

    \end{dfn}
    
    
    
    \begin{thm}
        [Probability-of-Sum-of-Independent-Random-Variables-Being-Zero-and-Consequence-for-Continuous-Distributions]
        {独立随机变量和为零的概率及连续分布的推论}
        [Probability of Sum of Independent Random Variables Being Zero and Consequence for Continuous Distributions]
        [gpt-4.1]
        
(i) 设 $X$ 和 $Y$ 独立,具有分布 $\mu$ 和 $
u$,则有
\[
P(X + Y = 0) = \sum_{y} \mu(\{-y\}) 
u(\{y\})
\]

(ii) 推论:若 $X$ 具有连续分布,则 $P(X = Y) = 0$.

    \end{thm}
    
    
    
    \begin{prf}
        [Proof-of-Measurability-under-Set-Transformation]
        {关于集合变换的可测性证明}
        [Proof of Measurability under Set Transformation]
        [gpt-4.1]
        证明 设 $B \in \mathcal{T}$.
$\{ \omega : f(X(\omega)) \in B \} = \{ \omega : X(\omega) \in f^{-1}(B) \} \in \mathcal{F}$, 因为根据假设 $f^{-1}(B) \in \mathcal{S}$.
    \end{prf}
    
    
    
    \begin{ppt}
        [Additivity-of-Variance-for-Uncorrelated-Random-Variables]
        {不相关随机变量方差的可加性}
        [Additivity of Variance for Uncorrelated Random Variables]
        [gpt-4.1]
        对于不相关的随机变量,和的方差等于各自方差之和.
    \end{ppt}
    
    
    
    \begin{ppt}
        [Variance-of-a-Random-Variable-Multiplied-by-a-Constant]
        {随机变量乘常数的方差性质}
        [Variance of a Random Variable Multiplied by a Constant]
        [gpt-4.1]
        对于任意常数 $c$ 和随机变量 $Y$,
\[
\operatorname{var}(cY) = c^2 \operatorname{var}(Y)
\]
    \end{ppt}
    
    
    
    \begin{thm}
        [Limit-Properties-of-Normalized-Sums-with-Infinite-Mean]
        {无限期望下归一化和的极限性质}
        [Limit Properties of Normalized Sums with Infinite Mean]
        [gpt-4.1]
        Feller (1946) 的结果显示,当 $E | X_{1} | = \infty$ 时,$S_{n} / a_{n}$ 不能几乎必然收敛到非零极限.
    \end{thm}
    
    
    
    \begin{thm}
        [Convolution-Formula-for-Normal-Distributions]
        {正态分布卷积公式}
        [Convolution Formula for Normal Distributions]
        [gpt-4.1]
        
设 $Y_1 = \mathrm{normal}(0, a)$,$Y_2 = \mathrm{normal}(0, b)$,则 $Y_1 + Y_2$ 的概率密度函数为

\[
f_{Y_1 + Y_2}(z) = \frac{1}{2\pi \sqrt{a b}} \int e^{-x^2 / 2a} e^{-(z - x)^2 / 2b} d x
\]

进一步化简为

\[
f_{Y_1 + Y_2}(z) = \exp\left( - \frac{z^2}{2(a + b)} \right) \sqrt{2 \pi a b / (a + b)}
\]

这表明两个独立正态分布的和仍为正态分布,其方差为 $a + b$.

    \end{thm}
    
    
    
    \begin{thm}
        [Equivalence-of-Integrals-over-Disjoint-Sets-and-Their-Union-for-Integrable-Functions]
        {可积函数在不交集合上的积分与并集上积分的等价性}
        [Equivalence of Integrals over Disjoint Sets and Their Union for Integrable Functions]
        [gpt-4.1]
        
若 $f$ 是可积函数,$E_{m}$ 是两两不交的集合,且它们的并为 $E$,则有
\[
\sum_{m = 0}^{\infty} \int_{E_{m}} f \, d\mu = \int_{E} f \, d\mu
\]

    \end{thm}
    
    
    
    \begin{dfn}
        [Measure-Defined-by-a-Nonnegative-Integrable-Function]
        {由非负可积函数定义的测度}
        [Measure Defined by a Nonnegative Integrable Function]
        [gpt-4.1]
        
若 $f \geq 0$,则 $
u(E) = \int_{E} f \, d\mu$ 定义了一个测度.

    \end{dfn}
    
    
    
    \begin{dfn}
        [Construction-of-Infinite-Coin-Tossing-Sequence]
        {无限次抛硬币序列的构造}
        [Construction of Infinite Coin Tossing Sequence]
        [gpt-4.1]
        设 $\Omega$ 为单位区间 $(0,1)$,配备有 Borel 集合 $\mathcal{F}$ 和 Lebesgue 测度 $P$.定义
\[
Y_n(\omega) = \begin{cases}
1, & \text{若 } [2^n \omega] \text{为奇数} \\
0, & \text{若 } [2^n \omega] \text{为偶数}
\end{cases}
\]
其中 $[2^n \omega]$ 表示 $2^n \omega$ 的整数部分.
    \end{dfn}
    
    
    
    \begin{thm}
        [Independence-and-Distribution-of-Infinite-Coin-Tossing-Sequence]
        {无限次抛硬币序列的独立性与分布}
        [Independence and Distribution of Infinite Coin Tossing Sequence]
        [gpt-4.1]
        对上述定义的随机变量序列 $Y_1, Y_2, \ldots$,有:
1. $Y_1, Y_2, \ldots$ 相互独立;
2. 对任意 $k$,有 $P(Y_k=0) = P(Y_k=1) = 1/2$.
    \end{thm}
    
    
    
    \begin{xmp}
        [Example-of-Constructing-Independent-Random-Variables]
        {独立随机变量的构造例子}
        [Example of Constructing Independent Random Variables]
        [gpt-4.1]
        
设 $X_{1}, X_{2}, \dots$ 是独立的随机变量,且有
\[
P(X_{n} = n^{-\alpha}) = P(X_{n} = -n^{-\alpha}) = 1/2
\]

    \end{xmp}
    
    
    
    \begin{thm}
        [Criterion-for-Independence-of-Discrete-Random-Variables]
        {离散型随机变量独立性的判据}
        [Criterion for Independence of Discrete Random Variables]
        [gpt-4.1]
        设 $X_{1}, \ldots, X_{n}$ 是取值于可数集 $S_{1}, \ldots, S_{n}$ 的随机变量,若对任意 $x_{i} \in S_{i}$ 都有
\[
P(X_{1} = x_{1}, \ldots, X_{n} = x_{n}) = \prod_{i=1}^{n} P(X_{i} = x_{i}),
\]
则 $X_{1}, \ldots, X_{n}$ 独立.
    \end{thm}
    
    
    
    \begin{dfn}
        [Definition-of-Lower-Semicontinuous-and-Upper-Semicontinuous-Functions]
        {下半连续与上半连续函数的定义}
        [Definition of Lower Semicontinuous and Upper Semicontinuous Functions]
        [gpt-4.1]
        
一个函数 $f$ 被称为下半连续(l.s.c.),当且仅当对于任意 $x$,
\[
\liminf_{y \to x} f(y) \geq f(x)
\]
称 $f$ 为上半连续(u.s.c.),当且仅当 $-f$ 是下半连续函数.

    \end{dfn}
    
    
    
    \begin{thm}
        [Equivalent-Condition-and-Measurability-of-Lower-Semicontinuous-Functions]
        {下半连续函数的等价条件及可测性}
        [Equivalent Condition and Measurability of Lower Semicontinuous Functions]
        [gpt-4.1]
        
函数 $f$ 是下半连续函数,当且仅当对于每个 $a \in \mathbf{R}$,集合 $\{ x : f(x) \leq a \}$ 是闭集.由此可得半连续函数是可测的.

    \end{thm}
    
    
    
    \begin{dfn}
        [Definition-of-Independence-of-σ-fields]
        {σ-域的独立性定义}
        [Definition of Independence of σ-fields]
        [gpt-4.1]
        
$\sigma$-域 $\mathcal{F}_1, \mathcal{F}_2, \ldots, \mathcal{F}_n$ 称为独立的,如果对于任意 $A_i \in \mathcal{F}_i$,$i = 1, \ldots, n$,有
\[
P\left(\cap_{i=1}^n A_i\right) = \prod_{i=1}^n P(A_i)
\]

    \end{dfn}
    
    
    
    \begin{dfn}
        [Definition-of-Independence-of-Random-Variables]
        {随机变量的独立性定义}
        [Definition of Independence of Random Variables]
        [gpt-4.1]
        
随机变量 $X_1, \ldots, X_n$ 称为独立的,如果对于任意 $B_i \in \mathcal{R}$,$i = 1, \ldots, n$,有
\[
P\left(\cap_{i=1}^n \{ X_i \in B_i \}\right) = \prod_{i=1}^n P(X_i \in B_i)
\]

    \end{dfn}
    
    
    
    \begin{dfn}
        [Definition-of-Independence-of-Sets]
        {集合的独立性定义}
        [Definition of Independence of Sets]
        [gpt-4.1]
        
集合 $A_1, \ldots, A_n$ 称为独立的,如果对于任意 $I \subset \{ 1, \ldots, n \}$,有
\[
P\left(\cap_{i \in I} A_i\right) = \prod_{i \in I} P(A_i)
\]

    \end{dfn}
    
    
    
    \begin{xmp}
        [A-Counterexample-Where-the-Weak-Law-of-Large-Numbers-Fails-Cauchy-Distribution]
        {弱大数定律不成立的一个反例:柯西分布}
        [A Counterexample Where the Weak Law of Large Numbers Fails: Cauchy Distribution]
        [gpt-4.1]
        设 $X_{1}, X_{2}, \dots$ 相互独立且服从柯西分布:

\[
P(X_{i} \leq x) = \int_{-\infty}^{x} \frac{dt}{\pi (1 + t^{2})}
\]

当 $x \to \infty$ 时,

\[
P(|X_{1}| > x) = 2 \int_{x}^{\infty} \frac{dt}{\pi (1 + t^{2})} \sim \frac{2}{\pi} \int_{x}^{\infty} t^{-2} dt = \frac{2}{\pi} x^{-1}
\]

由此条件的必要性可知,不存在常数序列 $\mu_{n}$ 使得 $S_{n}/n - \mu_{n} \to 0$.

    \end{xmp}
    
    
    
    \begin{thm}
        [Non-negative-Function-with-Zero-Integral-is-Identically-Zero]
        {积分为零的非负函数恒为零}
        [Non-negative Function with Zero Integral is Identically Zero]
        [gpt-4.1]
        若 $f \geq 0$ 且 $\int f d\mu = 0$,则 $f = 0$ 几乎处处.
    \end{thm}
    
    
    
    \begin{thm}
        [Approximation-Formula-for-Integral-of-Non-negative-Function]
        {非负函数积分的逼近公式}
        [Approximation Formula for Integral of Non-negative Function]
        [gpt-4.1]
        设 $f \geq 0$,并定义 $E_{n, m} = \{ x : m / 2^{n} \leq f(x) < (m+1) / 2^{n} \}$.则当 $n \uparrow \infty$ 时,
\[
\sum_{m=1}^{\infty} \frac{m}{2^{n}} \mu(E_{n, m}) \uparrow \int f d \mu
\]
    \end{thm}
    
    
    
    \begin{thm}
        [Weighted-Arithmetic-Geometric-Mean-Inequality-via-Jensens-Inequality]
        {Jensen不等式推出的加权算术-几何平均不等式}
        [Weighted Arithmetic-Geometric Mean Inequality via Jensen's Inequality]
        [gpt-4.1]
        
如果 $\sum_{m=1}^{n} p(m) = 1$ 且 $p(m), y_{m} > 0$,则有
\[
\sum_{m=1}^{n} p(m) y_{m} \geq \prod_{m=1}^{n} y_{m}^{p(m)}
\]
当 $p(m) = 1/n$ 时,这说明算术平均值大于等于几何平均值.

    \end{thm}
    
    
    
    \begin{thm}
        [Relation-between-Integrals-of-Stieltjes-Measure-Functions-and-Product-Measures]
        {Stieltjes测度函数的积分与积测度的关系}
        [Relation between Integrals of Stieltjes Measure Functions and Product Measures]
        [gpt-4.1]
        
设 $F, G$ 是Stieltjes测度函数,$\mu, 
u$ 分别是对应于它们在 $(\mathbf{R}, \mathcal{R})$ 上的测度,则有:

\[
\int_{(a, b]} \{ F(y) - F(a) \} dG(y) = (\mu \times 
u) (\{ (x, y) : a < x \leq y \leq b \})
\]

\[
\int_{(a, b]} F(y) dG(y) + \int_{(a, b]} G(y) dF(y)
= F(b) G(b) - F(a) G(a) + \sum_{x \in (a, b]} \mu(\{ x \}) 
u(\{ x \})
\]

若 $F = G$ 且连续,则
\[
\int_{(a, b]} 2 F(y) dF(y) = F^2(b) - F^2(a).
\]

    \end{thm}
    
    
    
    \begin{thm}
        [Additivity-of-Expectation-over-Countable-Disjoint-Sets-for-Integrable-Functions]
        {可积函数在可列互不相交集合上的期望可加性}
        [Additivity of Expectation over Countable Disjoint Sets for Integrable Functions]
        [gpt-4.1]
        
若 $X$ 可积,$A_{n}$ 为互不相交集合,其并为 $A$,则
\[
\sum_{n=0}^{\infty} E(X ; A_{n}) = E(X ; A)
\]
即该和绝对收敛,并且其值等于右侧的期望.

    \end{thm}
    
    
    
    \begin{thm}
        [Theorem-of-Unique-Correspondence-between-Stieltjes-Measure-Function-and-Measure]
        {Stieltjes测度函数与测度的唯一对应定理}
        [Theorem of Unique Correspondence between Stieltjes Measure Function and Measure]
        [gpt-4.1]
        每个Stieltjes测度函数$F$都唯一对应一个$(\mathbf{R}, \mathcal{R})$上的测度$\mu$,使得
\[
\mu((a, b]) = F(b) - F(a)
\]

    \end{thm}
    
    
    
    \begin{dfn}
        [Definition-of-Lebesgue-Measure]
        {Lebesgue测度的定义}
        [Definition of Lebesgue Measure]
        [gpt-4.1]
        当$F(x) = x$时,所对应的测度称为Lebesgue测度.

    \end{dfn}
    
    
    
    \begin{thm}
        [Separability-of-Probability-Density-Implies-Independence-of-Random-Variables]
        {概率密度可分性蕴含随机变量独立性}
        [Separability of Probability Density Implies Independence of Random Variables]
        [gpt-4.1]
        
设 $(X_{1}, \ldots, X_{n})$ 的概率密度为 $f(x_{1}, x_{2}, \ldots, x_{n})$,即

\[
P((X_{1}, X_{2}, \ldots, X_{n}) \in A) = \int_{A} f(x) dx \quad \mathrm{for} \ A \in {\mathcal{R}}^{n}
\]

若 $f(x)$ 可表示为 $g_{1}(x_{1}) \cdots g_{n}(x_{n})$,其中 $g_{m} \geq 0$ 且可测,则 $X_{1}, X_{2}, \ldots, X_{n}$ 彼此独立.

    \end{thm}
    
    
    
    \begin{xmp}
        [Example-on-the-Relationship-between-Collection-of-Sets-and-Event-Family]
        {关于集合族与事件族关系的例子}
        [Example on the Relationship between Collection of Sets and Event Family]
        [gpt-4.1]
        
如果 $B_n \in \mathcal{R}$,那么 $\{ X_n \in B_n~\text{i.o.} \} \in \mathcal{T}$.

    \end{xmp}
    
    
    
    \begin{thm}
        [Theorem-on-Limit-Properties-of-Sequences-of-Probabilistic-Events]
        {关于概率事件列的极限性质定理}
        [Theorem on Limit Properties of Sequences of Probabilistic Events]
        [gpt-4.1]
        
若 $P(A_n) \to 0$ 且 $\sum_{n=1}^{\infty} P(A_n^c \cap A_{n+1}) < \infty$,则 $P(A_n \ \text{i.o.}) = 0$.

    \end{thm}
    
    
    
    \begin{dfn}
        [Definition-of-Upper-and-Lower-Envelope-Functions]
        {关于上极限函数和下极限函数的定义}
        [Definition of Upper and Lower Envelope Functions]
        [gpt-4.1]
        设 $f : \mathbf{R}^d \to \mathbf{R}$ 是任意函数,定义
\[
f^{\delta}(x) = \sup\{ f(y) : |y-x| < \delta \}
\]
和
\[
f_{\delta}(x) = \inf\{ f(y) : |y-x| < \delta \}
\]
其中 $|z| = (z_1^2 + \cdots + z_d^2)^{1/2}$.
    \end{dfn}
    
    
    
    \begin{dfn}
        [Definition-of-Upper-and-Lower-Limits]
        {关于上极限和下极限的定义}
        [Definition of Upper and Lower Limits]
        [gpt-4.1]
        定义
\[
f^0 = \lim_{\delta \downarrow 0} f^{\delta}
\]
和
\[
f_0 = \lim_{\delta \downarrow 0} f_{\delta}
\]

    \end{dfn}
    
    
    
    \begin{thm}
        [Measurability-of-the-Set-of-Discontinuity-Points]
        {函数不连续点集合的可测性}
        [Measurability of the Set of Discontinuity Points]
        [gpt-4.1]
        函数 $f$ 的不连续点集合为 $\{ x : f^0(x) 
eq f_0(x) \}$,该集合是可测集.
    \end{thm}
    
    
    
    \begin{prf}
        [Proof-of-a-Property-of-Permutation-Distribution]
        {关于排列分布的性质证明}
        [Proof of a Property of Permutation Distribution]
        [gpt-4.1]
        如果我们令 $\sigma_{m}$ 为 $\{1, \ldots, m\}$ 上的一个随机选取的排列,则 (i) $\pi_{n} \circ \sigma_{m}$ 与 $\pi_{n}$ 具有相同的分布,并且 (ii) 由于 $\sigma_{m}$ 的作用会随机重新排列 $\pi_{n}(1), \ldots, \pi_{n}(m)$,因此所需结论成立.
    \end{prf}
    
    
    
    \begin{dfn}
        [Definition-of-Poisson-Distribution]
        {Poisson分布的定义}
        [Definition of Poisson Distribution]
        [gpt-4.1]
        我们称 $X$ 具有参数 $\lambda$ 的 Poisson 分布,当且仅当

\[
P(X = k) = e^{-\lambda} \lambda^{k} / k! \quad \text{对所有 } k = 0, 1, 2, \ldots
\]

    \end{dfn}
    
    
    
    \begin{ppt}
        [Moment-Properties-of-Poisson-Distribution]
        {Poisson分布的矩性质}
        [Moment Properties of Poisson Distribution]
        [gpt-4.1]
        对于 $k \geq 1$,有
\[
E ( X ( X - 1 ) \cdots ( X - k + 1 ) ) = \lambda^{k}
\]
其中等式成立的理由包括:(i) 当 $j < k$ 时 $j ( j - 1 ) \cdots ( j - k + 1 ) = 0$;(ii) 阶乘部分可以约去;(iii) Poisson 分布的总概率质量为 1.

    \end{ppt}
    
    
    
    \begin{ppt}
        [Expectation-and-Variance-of-Poisson-Distribution]
        {Poisson分布的期望和方差}
        [Expectation and Variance of Poisson Distribution]
        [gpt-4.1]
        由上述公式可以推出 Poisson 随机变量的期望和方差如下:

\[
E X = \lambda
\]

\[
\operatorname{var}(X) = E X^{2} - \left(E X\right)^{2} = E ( X ( X - 1 ) ) + E X - \lambda^{2} = \lambda
\]

    \end{ppt}
    
    
    
    \begin{thm}
        [Weak-Law-of-Large-Numbers]
        {弱大数定律}
        [Weak Law of Large Numbers]
        [gpt-4.1]
        
设 $X_{1}, X_{2}, \dots$ 是同分布独立随机变量序列,满足
\[
x P( | X_{i} | > x ) \to 0 \quad \text{当 } x \to \infty
\]
令 $S_{n} = X_{1} + \cdots + X_{n}$,$\mu_{n} = E( X_{1} 1_{( | X_{1} | \leq n )} )$.则 $S_{n} / n - \mu_{n}$ 以概率收敛于 0.

    \end{thm}
    
    
    
    \begin{xmp}
        [Example-of-Random-Variables-with-Infinite-Expectation-but-Convergent-Mean]
        {期望无穷但均值收敛的随机变量例子}
        [Example of Random Variables with Infinite Expectation but Convergent Mean]
        [gpt-4.1]
        设 $X_1, X_2, \dots$ 是一列独立同分布的随机变量,其中 $P(X_i = (-1)^k k) = C / (k^2 \log k)$,$k \geq 2$,常数 $C$ 使得概率总和为 $1$.证明 $E|X_i| = \infty$,但存在有限常数 $\mu$,使得 $S_n / n \to \mu$ 以概率收敛.
    \end{xmp}
    
    
    
    \begin{dfn}
        [Definition-of-Random-Variable-$Z-n$]
        {随机变量 $Z\_n$ 的定义}
        [Definition of Random Variable $Z_n$]
        [gpt-4.1]
        令 $K \geq 3$ 为素数,$X$ 和 $Y$ 为在 $\{0, 1, \ldots, K-1\}$ 上均匀分布且相互独立的随机变量.对 $0 \leq n < K$,定义随机变量 $Z_{n} = X + n Y \bmod K$.
    \end{dfn}
    
    
    
    \begin{prf}
        [Proof-of-the-closure-properties-of-$\bar{\mathcal{S}}$]
        {关于 $\bar{\mathcal{S}}$ 闭合性的证明}
        [Proof of the closure properties of $\bar{\mathcal{S}}$]
        [gpt-4.1]
        Suppose $A = \bigsqcup_{i} S_i$ and $B = \bigsqcup_{j} T_j$, where $\bigsqcup$ denotes disjoint union and we assume the index sets are finite.
Then $A \cap B = \bigsqcup_{i, j} (S_i \cap T_j) \in \bar{\mathcal{S}}$.
As for complements, if $A = \bigsqcup_{i} S_i$ then $A^c = \bigcap_{i} S_i^c$.
The definition of $\mathcal{S}$ implies $S_i^c \in \bar{\mathcal{S}}$.
We have shown that $\bar{\mathcal{S}}$ is closed under intersection, so it follows by induction that $A^c \in \bar{\mathcal{S}}$.
    \end{prf}
    
    
    
    \begin{dfn}
        [Definition-of-Integrable-Function]
        {可积函数的定义}
        [Definition of Integrable Function]
        [gpt-4.1]
        我们称 $f$ 是可积的,当且仅当 $\int |f| \, d\mu < \infty$.
    \end{dfn}
    
    
    
    \begin{dfn}
        [Definition-of-Positive-and-Negative-Parts-of-a-Function-and-Its-Integral]
        {函数的正负部和积分的定义}
        [Definition of Positive and Negative Parts of a Function and Its Integral]
        [gpt-4.1]
        设
\[
f^+(x) = f(x) \vee 0 \quad \text{and} \quad f^-(x) = (-f(x)) \vee 0
\]
其中 $a \vee b = \max(a, b)$.

显然,
\[
f(x) = f^+(x) - f^-(x) \quad \text{and} \quad |f(x)| = f^+(x) + f^-(x)
\]
我们定义 $f$ 的积分为
\[
\int f \, d\mu = \int f^+ \, d\mu - \int f^- \, d\mu
\]
右侧是良定义的,因为 $f^+$, $f^- \leq |f|$ 且已知引理 1 的 (iv) 成立.
    \end{dfn}
    
    
    
    \begin{thm}
        [Measurability-of-Sums-of-Random-Variables]
        {随机变量加法的可测性}
        [Measurability of Sums of Random Variables]
        [gpt-4.1]
        
如果 $X_{1}, \ldots, X_{n}$ 是随机变量,则 $X_{1} + \cdots + X_{n}$ 也是随机变量.

    \end{thm}
    
    
    
    \begin{prf}
        [Proof-of-Measurability-of-Sums-of-Random-Variables]
        {随机变量加法的可测性证明}
        [Proof of Measurability of Sums of Random Variables]
        [gpt-4.1]
        
根据定理 1.3.5,只需证明 $f(x_{1}, \ldots, x_{n}) = x_{1} + \cdots + x_{n}$ 是可测的.为此,利用例 1.3.2,注意到集合 $\{ x : x_{1} + \cdots + x_{n} < a \}$ 是开集,因此属于 $\mathcal{R}^{n}$.

    \end{prf}
    
    
    
    \begin{thm}
        [Theorem-on-Metric-Preservation-under-Function-Application]
        {函数作用下的度量保持性定理}
        [Theorem on Metric Preservation under Function Application]
        [gpt-4.1]
        
设 $\rho(x, y)$ 是一个度量.假设函数 $h$ 可微,满足 $h(0) = 0$,且对 $x > 0$ 有 $h'(x) > 0$,并且 $h'(x)$ 在 $[0, \infty)$ 上单调递减,则 $h(\rho(x, y))$ 也是一个度量.

    \end{thm}
    
    
    
    \begin{dfn}
        [Definition-of-Convergence-of-Series]
        {数项级数收敛的定义}
        [Definition of Convergence of Series]
        [gpt-4.1]
        我们称 $\sum_{n=1}^{\infty} a_{n}$ 收敛,如果 $\lim_{N \to \infty} \sum_{n=1}^{N} a_{n}$ 存在.
    \end{dfn}
    
    
    
    \begin{dfn}
        [Definition-of-Record-Values]
        {记录值的定义}
        [Definition of Record Values]
        [gpt-4.1]
        设 $X_1, X_2, \dots$ 是一列随机变量,$X_k$ 表示某个个体第 $k$ 次跳高或投掷的距离,定义事件 $A_k = \{ X_k > \sup_{j < k} X_j \}$ 为在时刻 $k$ 出现新纪录的事件.
    \end{dfn}
    
    
    
    \begin{thm}
        [Hölders-Inequality]
        {Hölder不等式}
        [Hölder's Inequality]
        [gpt-4.1]
        
设 $p, q \in [1, \infty]$ 且 $1/p + 1/q = 1$,则
\[
E|XY| \leq \|X\|_p \|Y\|_q
\]
其中 $\|X\|_r = (E|X|^r)^{1/r}$ 对于 $r \in [1, \infty)$;$\|X\|_\infty = \operatorname{inf}\{M : P(|X| > M) = 0\}$.

    \end{thm}
    
    
    
    \begin{dfn}
        [Recursive-Definition-of-Sequences-$a-{m}-b-{m}$]
        {数列 $a\_{m}, b\_{m}$ 的递推定义}
        [Recursive Definition of Sequences $a_{m}, b_{m}$]
        [gpt-4.1]
        令 $a_{0} = 0, b_{0} = 0$, 当 $m \geq 1$ 时,定义 $b_{m} = \sum_{k=1}^{m} x_{k} / a_{k}$.
    \end{dfn}
    
    
    
    \begin{thm}
        [Minkowskis-Inequality]
        {Minkowski不等式}
        [Minkowski's Inequality]
        [gpt-4.1]
        设 $p \in [1, \infty]$,$f, g$ 为适当的函数,$\|f\|_p$ 和 $\|g\|_p$ 有限,则有
\[
\| f + g \|_{p} \leq \| f \|_{p} + \| g \|_{p}
\]
其中当 $p \in (1, \infty)$ 时,可由应用Hölder不等式于 $|f| |f + g|^{p-1}$ 和 $|g| |f + g|^{p-1}$ 得出.该结论在 $p = 1$ 或 $p = \infty$ 时也成立.
    \end{thm}
    
    
    
    \begin{thm}
        [Criterion-for-the-Probability-of-Maximum-Exceeding-Thresholds]
        {关于极大值事件概率的判定}
        [Criterion for the Probability of Maximum Exceeding Thresholds]
        [gpt-4.1]
        
设 $X_1, X_2, \dots$ 独立同分布,分布函数为 $F$,令 $\lambda_n \uparrow \infty$,令 $A_n = \{\max_{1 \leq m \leq n} X_m > \lambda_n\}$.则
\[
P(A_n \ \text{i.o.}) = 0 \text{ 或 } 1
\]
其值取决于级数 $\sum_{n \geq 1} (1 - F(\lambda_n))$ 的敛散性:当 $\sum_{n \geq 1} (1 - F(\lambda_n)) < \infty$ 时为 $0$,当 $= \infty$ 时为 $1$.

    \end{thm}
    
    
    
    \begin{lma}
        [Structure-of-the-Algebra-Generated-by-a-Semialgebra]
        {半代数生成的代数的结构}
        [Structure of the Algebra Generated by a Semialgebra]
        [gpt-4.1]
        
如果 $\mathcal{S}$ 是一个半代数, 则 $\bar{\mathcal{S}} =$ {由 $\mathcal{S}$ 中集合有限个不交并组成的集合} 是一个代数,称为由 $\mathcal{S}$ 生成的代数.

    \end{lma}
    
    
    
    \begin{ppt}
        [Permutability-of-Events-in-the-Tail-σ-field]
        {尾σ域中的事件的可置换性}
        [Permutability of Events in the Tail σ-field]
        [gpt-4.1]
        所有属于尾 $\sigma$-域 $\tau$ 的事件都是可置换的.即若 $A \in \sigma(X_{n+1}, X_{n+2}, \ldots)$,则对任意 $X_{1}, \ldots, X_{n}$ 的置换,事件 $A$ 的发生与否不受影响.
    \end{ppt}
    
    
    
    \begin{thm}
        [Theorem-of-Unique-Extension-of-Measure-on-Semialgebra]
        {半代数上的测度唯一扩张定理}
        [Theorem of Unique Extension of Measure on Semialgebra]
        [gpt-4.1]
        设 $\mathcal{S}$ 是一个半代数,$\mu$ 是定义在 $\mathcal{S}$ 上的测度,且 $\mu(\varnothing) = 0$.
若满足:
(i) 若 $S \in \mathcal{S}$ 是有限个不交的 $S_i \in \mathcal{S}$ 的并,则 $\mu(S) = \sum_i \mu(S_i)$;
(ii) 若 $S_i, S \in \mathcal{S}$ 且 $S = \cup_{i \geq 1} S_i$,则 $\mu(S) \leq \sum_{i \geq 1} \mu(S_i)$.
则 $\mu$ 在由 $\mathcal{S}$ 生成的代数 $\bar{\mathcal{S}}$ 上有唯一扩展 $\bar{\mu}$,且 $\bar{\mu}$ 是一个测度.
如果 $\bar{\mu}$ 是$\sigma$-有限的,则存在唯一扩展 $
u$,它是定义在 $\sigma(\mathcal{S})$ 上的测度.

    \end{thm}
    
    
    
    \begin{dfn}
        [Definition-of-Variables-in-Renewal-Process]
        {续更新过程中的变量定义}
        [Definition of Variables in Renewal Process]
        [gpt-4.1]
        设 $X_{1}, X_{2}, \dots$ 是独立同分布的随机变量,且 $0 < X_{i} < \infty$.定义 $T_{n} = X_{1} + \cdots + X_{n}$,代表第 $n$ 次事件发生的时间.进一步,设 $N_{t} = \sup \{ n : T_{n} \leq t \}$,则 $N_{t}$ 是在时间 $t$ 之前发生的事件次数(例如烧坏的灯泡数).
    \end{dfn}
    
    
    
    \begin{prf}
        [Proof-that-Probability-of-Limsup-Event-is-1]
        {概率极限上确界事件为1的证明}
        [Proof that Probability of Limsup Event is 1]
        [gpt-4.1]
        设 $M < N < \infty$.独立性和 $1 - x \leq e^{-x}$ 蕴含

\[
\begin{array}{r}
P \left( \bigcap_{n=M}^N A_n^c \right) = \prod_{n=M}^N \left( 1 - P ( A_n ) \right) \leq \prod_{n=M}^N \exp ( - P ( A_n ) ) \\
= \exp \left( - \sum_{n=M}^N P ( A_n ) \right) \to 0 \quad \mathrm{as}\ N \to \infty
\end{array}
\]

因此对任意 $M$ 有 $P ( \cup_{n=M}^{\infty} A_n ) = 1$,且由于 $\cup_{n=M}^{\infty} A_n \downarrow \limsup A_n$,从而 $P ( \operatorname*{lim\,sup} A_n ) = 1$.

    \end{prf}
    
    
    
    \begin{thm}
        [0-1-Law-for-Independent-Events]
        {独立事件的 0-1 律}
        [0-1 Law for Independent Events]
        [gpt-4.1]
        如果 $A_{1}, A_{2}, \ldots$ 是独立事件,那么定理 2.5.3 推出 $P(A_{n} \text{ i.o.}) = 0$ 或 $1$.
    \end{thm}
    
    
    
    \begin{thm}
        [Sharpness-of-Chebyshevs-Inequality]
        {切比雪夫不等式的精确性}
        [Sharpness of Chebyshev's Inequality]
        [gpt-4.1]
        (i) 证明定理 1.6.4 是精确的,方法是:如果 $0 < b \leq a$ 已知存在,则存在一个随机变量 $X$ 使得 $E X^2 = b^2$,且 $P(|X| \geq a) = b^2 / a^2$.
(ii) 证明定理 1.6.4 并不精确,即如果 $X$ 满足 $0 < E X^2 < \infty$,则有
\[
\lim_{a \to \infty} \frac{a^2 P(|X| \geq a)}{E X^2} = 0
\]

    \end{thm}
    
    
    
    \begin{thm}
        [Theorem-on-Existence-and-Uniqueness-of-Probability-Measure]
        {关于概率测度存在唯一性的定理}
        [Theorem on Existence and Uniqueness of Probability Measure]
        [gpt-4.1]
        设 $F : \mathbb{R}^d \to [0, 1]$ 满足上述条件 (i)-(iv),则在 $(\mathbb{R}^d, \mathcal{R}^d)$ 上存在唯一的概率测度 $\mu$,使得对所有有限矩形 $A$,有 $\mu(A) = \Delta_A F$.
    \end{thm}
    
    
    
    \begin{xmp}
        [Counterexample-on-Changing-Order-of-Integration]
        {关于交换积分次序的反例}
        [Counterexample on Changing Order of Integration]
        [gpt-4.1]
        
设 $X = (0, 1)$, $Y = (1, \infty)$,均配备 Borel 集合与 Lebesgue 测度.定义函数 $f(x, y) = e^{-xy} - 2e^{-2xy}$.则
\[
\begin{array}{l}
\displaystyle \int_{0}^{1} \int_{1}^{\infty} f(x, y) \, dy \, dx = \int_{0}^{1} x^{-1} (e^{-x} - e^{-2x}) dx > 0 \\
\displaystyle \int_{1}^{\infty} \int_{0}^{1} f(x, y) \, dx \, dy = \int_{1}^{\infty} y^{-1} (e^{-2y} - e^{-y}) dy < 0
\end{array}
\]
该例说明了为什么 $\mu_1$ 和 $\mu_2$ 必须是 $\sigma$-有限的.

    \end{xmp}
    
    
    
    \begin{lma}
        [Lemma-on-Series-Estimate-for-$0-\leq-y-<-1$]
        {关于$0 \leq y < 1$时的级数估计引理}
        [Lemma on Series Estimate for $0 \leq y < 1$]
        [gpt-4.1]
        对于 $0 \leq y < 1$,有
\[
2 y \sum_{k > y} k^{-2} \leq 2 \left( 1 + \sum_{k = 2}^{\infty} k^{-2} \right) \leq 4
\]

    \end{lma}
    
    
    
    \begin{thm}
        [Theorem-on-the-Convergence-of-Number-of-Records-to-Logarithm]
        {记录数与对数的收敛定理}
        [Theorem on the Convergence of Number of Records to Logarithm]
        [gpt-4.1]
        
若 $R_{n} = \sum_{m=1}^n 1_{A_{m}}$ 表示第 $n$ 时刻的记录数,则当 $n \to \infty$ 时,
\[
R_{n} / \log n \to 1 \quad \text{a.s.}
\]
该结果与分布 $F$ 无关(只要 $F$ 是连续的).

    \end{thm}
    
    
    
    \begin{thm}
        [Borel-Subsets-are-Nice-Spaces]
        {Borel 子集是 nice 空间}
        [Borel Subsets are Nice Spaces]
        [gpt-4.1]
        
如果 $S$ 是一个完备可分度量空间 $M$ 的 Borel 子集,$\mathcal{S}$ 是 $S$ 上所有 Borel 子集的集合,则 $(S, \mathcal{S})$ 是 nice 空间.

    \end{thm}
    
    
    
    \begin{ppt}
        [Variance-Formula-for-$N-n$]
        {变量 $N\_n$ 的方差公式}
        [Variance Formula for $N_n$]
        [gpt-4.1]
        
设 $N_n = \sum_{m=1}^n 1_{A_m}$,则有
\[
E N_n^2 = E\left(\sum_{m=1}^n 1_{A_m}\right)^2 = \sum_{1 \leq k, m \leq n} P(A_k \cap A_m)
\]
\[
\operatorname{var}(N_n) = E N_n^2 - (E N_n)^2 = \sum_{1 \leq k, m \leq n} [P(A_k \cap A_m) - P(A_k) P(A_m)]
\]
\[
= n(n-1) \{ (1 - 2/n)^r - (1 - 1/n)^{2r} \} + n \{ (1 - 1/n)^r - (1 - 1/n)^{2r} \}
\]
其中第一项来自 $k 
eq m$,第二项来自 $k = m$.

    \end{ppt}
    
    
    
    \begin{crl}
        [Variance-of-$N-n/n$-Converges-to-Zero]
        {$N\_n/n$ 的方差收敛于零}
        [Variance of $N_n/n$ Converges to Zero]
        [gpt-4.1]
        
由于 $(1 - 2/n)^r \to e^{-2c}$ 且 $(1 - 1/n)^r \to e^{-c}$,由上述公式可知
\[
\operatorname{var}(N_n/n) = \operatorname{var}(N_n)/n^2 \to 0
\]

    \end{crl}
    
    
    
    \begin{thm}
        [Limit-of-Normalized-Counting-Process-with-Expected-Value-of-Random-Variables]
        {随机变量均值与归一化计数过程的极限}
        [Limit of Normalized Counting Process with Expected Value of Random Variables]
        [gpt-4.1]
        
若 $\mathbb{E} X_{1} = \mu \leq \infty$,则当 $t \to \infty$ 时,
\[
N_{t} / t \to 1 / \mu \text{ a.s.}
\]
其中,$1 / \infty = 0$.

    \end{thm}
    
    
    
    \begin{thm}
        [Sufficient-Condition-for-Convergence-of-Independent-Random-Variables]
        {独立随机变量收敛性的充分条件}
        [Sufficient Condition for Convergence of Independent Random Variables]
        [gpt-4.1]
        
设 $X_{1}, X_{2}, \dots$ 是独立的随机变量,且 $E X_{n} = 0$.如果
\[
\sum_{n=1}^{\infty} \operatorname{var}\left(X_{n}\right) < \infty
\]
则几乎必然有 $\sum_{n=1}^{\infty} X_{n}(\omega)$ 收敛.

    \end{thm}
    
    
    
    \begin{lma}
        [Lemma-on-Monotonicity-Equality-and-Absolute-Value-Properties-of-Integrals]
        {积分的单调性、等价性和绝对值性质引理}
        [Lemma on Monotonicity, Equality and Absolute Value Properties of Integrals]
        [gpt-4.1]
        
若 (i) 和 (iii) 成立,则有:
(iv) 若 $\varphi \leq \psi$ 几乎处处成立,则 $\int \varphi d\mu \leq \int \psi d\mu$.
(v) 若 $\varphi = \psi$ 几乎处处成立,则 $\int \varphi d\mu = \int \psi d\mu$.
若进一步 (ii) 在 $a = -1$ 时成立,则有:
(vi) $|\int \varphi d\mu| \leq \int |\varphi| d\mu$.

    \end{lma}
    
    
    
    \begin{prf}
        [Proof-of-the-Lemma-on-Properties-of-Integrals]
        {积分性质引理的证明}
        [Proof of the Lemma on Properties of Integrals]
        [gpt-4.1]
        
由 (iii),有 $\int \psi d\mu = \int \varphi d\mu + \int (\psi - \varphi) d\mu$,且第二个积分根据 (i) 不小于 0,因此 (iv) 成立.

$\varphi = \psi$ 几乎处处成立意味着 $\varphi \leq \psi$ 几乎处处且 $\psi \leq \varphi$ 几乎处处,因此 (v) 可由两次应用 (iv) 得出.

为证明 (vi),注意到 $\varphi \leq |\varphi|$,故由 (iv) 得 $\int \varphi d\mu \leq \int |\varphi| d\mu$.同理,$-\varphi \leq |\varphi|$,故由 (iv) 和 (ii) 得 $-\int \varphi d\mu \leq \int |\varphi| d\mu$.由于 $|y| = \max(y, -y)$,结果成立.

    \end{prf}
    
    
    
    \begin{thm}
        [Normalized-Sum-of-Independent-Poisson-Variables-Convergence-Theorem]
        {独立 Poisson 变量和的归一化收敛定理}
        [Normalized Sum of Independent Poisson Variables Convergence Theorem]
        [gpt-4.1]
        设 $X_n$ 是独立的 Poisson 随机变量,满足 $E X_n = \lambda_n$,定义 $S_n = X_1 + \cdots + X_n$.若 $\sum \lambda_n = \infty$,则有
\[
\frac{S_n}{E S_n} \to 1 \quad \text{几乎处处收敛(a.s.)}
\]
即,$S_n$ 与其期望值之比几乎处处收敛于 1.

    \end{thm}
    
    
    
    \begin{dfn}
        [Definition-of-the-Event-of-Convergent-Random-Variables]
        {收敛随机变量的事件的定义}
        [Definition of the Event of Convergent Random Variables]
        [gpt-4.1]
        
\[
\Omega_{o} \equiv \{ \omega : \lim_{n \to \infty} X_{n} \text{ exists} \} = \{ \omega : \lim_{n \to \infty} | X_{n} - \lim_{n \to \infty} X_{n} | = 0 \}
\]

其中 $\Omega_{o}$ 表示随机变量序列 $\{ X_n \}$ 收敛的样本点事件.

    \end{dfn}
    
    
    
    \begin{prf}
        [Constructive-Proof-for-Approximation-of-Random-Variable-Function]
        {关于随机变量函数逼近的构造性证明}
        [Constructive Proof for Approximation of Random Variable Function]
        [gpt-4.1]
        
为获得上述结果的构造性证明,注意到 $\{\omega : m 2^{-n} \leq Y < (m+1) 2^{-n} \} = \{ X \in B_{m,n} \}$,其中 $B_{m,n} \in \mathcal{R}$,并设 $f_n(x) = m 2^{-n}$ 对于 $x \in B_{m,n}$.证明当 $n \to \infty$ 时,$f_n(x) \to f(x)$ 且 $Y = f(X)$.

    \end{prf}
    
    
    
    \begin{xmp}
        [Calculation-of-Expectation-and-Variance-for-Standard-Normal-Distribution]
        {标准正态分布的期望与方差计算}
        [Calculation of Expectation and Variance for Standard Normal Distribution]
        [gpt-4.1]
        
如果 $X$ 服从标准正态分布,则
\[
\begin{array}{l}
E X = \displaystyle \int x (2 \pi)^{-1/2} \exp(-x^{2}/2) dx = 0 \quad \text{(by symmetry)} \\
\operatorname{var}(X) = E X^{2} = \displaystyle \int x^{2} (2 \pi)^{-1/2} \exp(-x^{2}/2) dx = 1
\end{array}
\]

如果令 $\sigma > 0$,$\mu \in \mathbf{R}$,且 $Y = \sigma X + \mu$,则由 (b) of Theorem 1.6.1 和 (1.6.4) 可得 $E Y = \mu$,$\operatorname{var}(Y) = \sigma^{2}$.

由习题 1.2.5,$Y$ 的密度为
\[
(2\pi \sigma^{2})^{-1/2} \exp\left( -\frac{(y-\mu)^{2}}{2\sigma^{2}} \right)
\]
即均值为 $\mu$,方差为 $\sigma^{2}$ 的正态分布.

    \end{xmp}
    
    
    
    \begin{thm}
        [Strong-Law-of-Large-Numbers]
        {强大数定律}
        [Strong Law of Large Numbers]
        [gpt-4.1]
        
设 $X_{1}, X_{2}, \dots$ 是两两独立同分布的随机变量,且 $E|X_{i}| < \infty$.令 $E X_{i} = \mu$,$S_{n} = X_{1} + \cdots + X_{n}$.则有 $S_{n}/n \to \mu$ 几乎必然成立,当 $n \to \infty$.

    \end{thm}
    
    
    
    \begin{dfn}
        [Definition-of-Binomial-Distribution]
        {二项分布的定义}
        [Definition of Binomial Distribution]
        [gpt-4.1]
        
$X$ 被称为服从参数为 $(n, p)$ 的二项分布,如果
\[
P(X = m) = \binom{n}{m} p^m (1-p)^{n-m}
\]

    \end{dfn}
    
    
    
    \begin{thm}
        [Distribution-of-the-Sum-of-Independent-Binomial-Variables]
        {独立二项分布变量和的分布}
        [Distribution of the Sum of Independent Binomial Variables]
        [gpt-4.1]
        
若 $X = \operatorname{Binomial}(n, p)$ 且 $Y = \operatorname{Binomial}(m, p)$ 彼此独立,则 $X + Y = \operatorname{Binomial}(n+m, p)$.

    \end{thm}
    
    
    
    \begin{thm}
        [Distribution-of-the-Sum-of-Independent-Bernoulli-Variables]
        {独立伯努利变量之和的分布}
        [Distribution of the Sum of Independent Bernoulli Variables]
        [gpt-4.1]
        
$n$ 个独立的 Bernoulli$(p)$ 随机变量之和服从 Binomial$(n, p)$ 分布.

    \end{thm}
    
    
    
    \begin{lma}
        [Two-Properties-of-Sets-and-Measure]
        {关于集合与度量的两个性质}
        [Two Properties of Sets and Measure]
        [gpt-4.1]
        
Suppose only that (i) holds.
(a) If $A, B_{i} \in \bar{\mathcal{S}}$ with $A = \cup_{i=1}^{n} B_{i}$, then $\bar{\mu}(A) = \sum_{i} \bar{\mu}(B_{i})$.
(b) If $A, B_{i} \in \bar{\mathcal{S}}$ with $A \subset \cup_{i=1}^{n} B_{i}$, then $\bar{\mu}(A) \leq \sum_{i} \bar{\mu}(B_{i})$.

    \end{lma}
    
    
    
    \begin{dfn}
        [Definition-of-Notation-for-Identically-Distributed-Variables]
        {相同分布的变量的符号表示定义}
        [Definition of Notation for Identically Distributed Variables]
        [gpt-4.1]
        
当随机变量 $X$ 和 $Y$ 具有相同的分布时,记作
\[
X \overset{d}{=} Y
\]
为了在文本中排版方便,也可记作 $X =_d~ Y$.

    \end{dfn}
    
    
    
    \begin{xmp}
        [Example-of-Discrete-Probability-Spaces]
        {离散概率空间的例子}
        [Example of Discrete Probability Spaces]
        [gpt-4.1]
        设 $\Omega$ 是一个可数集合,即有限或可数无限.令 ${\mathcal{F}}$ 为 $\Omega$ 的所有子集的集合.定义

\[
P(A) = \sum_{\omega \in A} p(\omega),\quad \text{其中 } p(\omega) \geq 0 \text{ 且 } \sum_{\omega \in \Omega} p(\omega) = 1
\]

这就是在该空间上最一般的概率测度.

    \end{xmp}
    
    
    
    \begin{lma}
        [Condition-for-Sections-to-Belong-to-Sub-$\sigma$-Algebra]
        {横截面属于子$\sigma$-代数的条件}
        [Condition for Sections to Belong to Sub-$\sigma$-Algebra]
        [gpt-4.1]
        
如果 $E \in \mathcal{F}$,那么 $E_x \in B$.

    \end{lma}
    
    
    
    \begin{prf}
        [Proof-of-Condition-for-Sections-to-Belong-to-Sub-$\sigma$-Algebra]
        {横截面属于子$\sigma$-代数的条件的证明}
        [Proof of Condition for Sections to Belong to Sub-$\sigma$-Algebra]
        [gpt-4.1]
        
$(E^{c})_{x} = (E_{x})^{c}$ 且 $(\cup_{i} E_{i})_{x} = \cup_{i} (E_{i})_{x}$,因此如果 $\mathcal{E}$ 是所有满足 $E_{x} \in B$ 的集合的族,则 $\mathcal{E}$ 是一个 $\sigma$-代数.由于 $\mathcal{E}$ 包含所有矩形,结论成立.

    \end{prf}
    
    
    
    \begin{dfn}
        [Definition-of-Equal-Distribution-of-Random-Variables]
        {分布相同的随机变量的定义}
        [Definition of Equal Distribution of Random Variables]
        [gpt-4.1]
        如果 $X$ 和 $Y$ 在 $(\mathbf{R}, \mathcal{R})$ 上诱导相同的分布 $\mu$,则称 $X$ 和 $Y$ 在分布上相等.
    \end{dfn}
    
    
    
    \begin{thm}
        [Distribution-Functions-Characterize-Equal-Distribution-of-Random-Variables]
        {分布函数判定随机变量分布相等}
        [Distribution Functions Characterize Equal Distribution of Random Variables]
        [gpt-4.1]
        $X$ 和 $Y$ 在分布上相等当且仅当对所有 $x$ 有 $P(X \leq x) = P(Y \leq x)$.
    \end{thm}
    
    
    
    \begin{dfn}
        [Definition-of-Random-Variable-Distribution-in-an-Unfair-Fair-Game]
        {不公平'公平游戏'中的随机变量分布定义}
        [Definition of Random Variable Distribution in an Unfair 'Fair Game']
        [gpt-4.1]
        
设 $p_{k} = 1 / (2^{k} k(k+1)),\; k = 1, 2, \dots$,$p_{0} = 1 - \sum_{k \geq 1} p_{k}$,定义随机变量序列 $X_1, X_2, \dots$ 独立同分布(i.i.d.),其分布为:
\[
P(X_{n} = -1) = p_{0}
\]
\[
P(X_{n} = 2^{k} - 1) = p_{k} \quad \text{对于 } k \geq 1
\]

    \end{dfn}
    
    
    
    \begin{ppt}
        [Expectation-of-the-Random-Variable-is-Zero]
        {随机变量期望为零}
        [Expectation of the Random Variable is Zero]
        [gpt-4.1]
        
对于上述定义的随机变量 $X_n$,有
\[
E X_{n} = 0
\]

    \end{ppt}
    
    
    
    \begin{dfn}
        [Construction-of-Independent-Random-Variables]
        {独立随机变量的构造}
        [Construction of Independent Random Variables]
        [gpt-4.1]
        设给定有限个分布函数 $F_i, 1 \leq i \leq n$,则可以如下方式构造独立随机变量 $X_1, \ldots, X_n$,使得 $P(X_i \leq x) = F_i(x)$:令概率空间为 $\Omega = \mathbb{R}^n$,$\mathcal{F} = \mathcal{R}^n$,对 $\boldsymbol{\omega} = (\omega_1, \ldots, \omega_n) \in \mathbb{R}^n$,定义 $X_i(\omega_1, \ldots, \omega_n) = \omega_i$,并令 $P$ 为在 $\mathcal{R}^n$ 上的测度,满足
\[
P((a_1, b_1] \times \cdots \times (a_n, b_n]) = (F_1(b_1) - F_1(a_1)) \cdots (F_n(b_n) - F_n(a_n))
\]
若 $\mu_i$ 为分布函数 $F_i$ 对应的测度,则 $P = \mu_1 \times \cdots \times \mu_n$.

    \end{dfn}
    
    
    
    \begin{lma}
        [Lemma-on-Limits-and-Minimum-in-Measure-Spaces]
        {测度空间上关于极限和最小值的引理}
        [Lemma on Limits and Minimum in Measure Spaces]
        [gpt-4.1]
        
设 $E_{n} \uparrow \Omega$ 且 $\mu ( E_{n} ) < \infty$,定义 $a \wedge b = \min(a, b)$.则有
\[
\int_{E_{n}} f \wedge n \, d\mu \uparrow \int f d \mu \quad \text{as } n \uparrow \infty
\]

    \end{lma}
    
    
    
    \begin{thm}
        [Probability-Properties-and-Divergence-of-Independent-Event-Sequences]
        {独立事件序列的概率性质与敛散性}
        [Probability Properties and Divergence of Independent Event Sequences]
        [gpt-4.1]
        设 $\{A_n\}$ 是一列独立事件,且对所有 $n$ 有 $P(A_n) < 1$.若 $P\left(\bigcup A_n\right) = 1$,则 $\sum_n P(A_n) = \infty$,从而 $P(A_n \ \text{i.o.}) = 1$.
    \end{thm}
    
    
    
    \begin{thm}
        [Theorem-on-Convergence-of-Weighted-Mean-of-Uncorrelated-Variables]
        {无关变量加权均值收敛定理}
        [Theorem on Convergence of Weighted Mean of Uncorrelated Variables]
        [gpt-4.1]
        
设 $X_{1}, X_{2}, \dots$ 是无关的随机变量,且 $E X_{i} = \mu_{i}$,并且 $\operatorname{var}(X_{i}) / i \to 0$ 当 $i \to \infty$.令 $S_{n} = X_{1} + \cdots + X_{n}$,$
u_{n} = E S_{n} / n$,则当 $n \to \infty$ 时,$S_{n} / n - 
u_{n} \to 0$ 在 $L^{2}$ 意义下和以概率收敛.

    \end{thm}
    
    
    
    \begin{lma}
        [Generalization-of-Expectation-for-Integrals-of-Non-negative-Functions]
        {关于非负函数积分形式期望的推广}
        [Generalization of Expectation for Integrals of Non-negative Functions]
        [gpt-4.1]
        
若 $H(x) = \int_{(-\infty, x]} h(y) dy$ 且 $h(y) \geq 0$,则有
\[
E\, H(X) = \int_{-\infty}^{\infty} h(y) P(X \geq y)\, dy
\]

    \end{lma}
    
    
    
    \begin{dfn}
        [Special-Notation-for-Integral-Signs]
        {积分符号特殊记法}
        [Special Notation for Integral Signs]
        [gpt-4.1]
        
(a) 当 $( \Omega , { \mathcal F } , \mu ) = ( { \mathbf R } ^ { d } , { \mathcal R } ^ { d } , \lambda )$ 时,记 $\int f ( x ) d x$ 表示 $\int f d \lambda$.

(b) 当 $( \Omega , \mathcal{F} , \mu ) = ( \mathbf{R} , \mathcal{R} , \lambda )$ 且 $E = [ a , b ]$ 时,记 $\int_{a}^{b} f ( x ) d x$ 表示 $\int_{E} f d \lambda$.

(c) 当 $( \Omega , { \mathcal F } , \mu ) = ( \mathbf{R} , { \mathcal R } , \mu )$,且对 $a < b$,有 $\mu ( ( a , b ] ) := G ( b ) - G ( a )$,则记 $\int f ( x ) d G ( x )$ 表示 $\int f d \mu$.

(d) 当 $\Omega$ 是可数集,${ \mathcal F } =$ $\Omega$ 的所有子集,$\mu$ 为计数测度时,记 $\sum_{i \in \Omega} f(i)$ 表示 $\int f d \mu$.

    \end{dfn}
    
    
    
    \begin{dfn}
        [Definition-of-the-Distribution-of-a-Random-Variable]
        {随机变量的分布的定义}
        [Definition of the Distribution of a Random Variable]
        [gpt-4.1]
        若 $X$ 是一个随机变量,则 $X$ 在 $\mathbf{R}$ 上诱导出一个概率测度 $\mu$,称为 $X$ 的分布,通过对 Borel 集 $A$ 定义 $\mu(A) = P(X \in A)$ 给出.
    \end{dfn}
    
    
    
    \begin{dfn}
        [Definition-of-Distribution-Function]
        {分布函数的定义}
        [Definition of Distribution Function]
        [gpt-4.1]
        随机变量 $X$ 的分布通常通过其分布函数 $F(x) = P(X \leq x)$ 给出.
    \end{dfn}
    
    
    
    \begin{xmp}
        [An-Example-of-Semialgebra]
        {关于半代数的一个例子}
        [An Example of Semialgebra]
        [gpt-4.1]
        $\mathcal{S}_d =$ 空集加上所有形如
\[
(a_1, b_1] \times \cdots \times (a_d, b_d] \subset \mathbf{R}^d \quad \mathrm{~其中~} -\infty \leq a_i < b_i \leq \infty
\]
的集合.
    \end{xmp}
    
    
    
    \begin{dfn}
        [Definition-of-Empirical-Distribution-Function]
        {经验分布函数的定义}
        [Definition of Empirical Distribution Function]
        [gpt-4.1]
        
设 $X_{1}, X_{2}, \dots$ 是独立同分布(i.i.d.)的随机变量,分布函数为 $F$.定义经验分布函数 $F_{n}(x)$ 为
\[
F_{n}(x) = n^{-1} \sum_{m=1}^{n} 1_{(X_{m} \leq x)}
\]
其中 $F_{n}(x)$ 表示观测值中小于等于 $x$ 的频率.

    \end{dfn}
    
    
    
    \begin{prf}
        [Proof-of-Relation-between-Countable-Union-of-Sets-and-Measure]
        {集合可列并与测度的关系的证明}
        [Proof of Relation between Countable Union of Sets and Measure]
        [gpt-4.1]
        
观察可知,根据定义,如果 $A = \cup_{i} B_{i}$ 是 $\bar{\mathcal{S}}$ 中集合的有限不交并,且 $B_{i} = \cup_{j} S_{i,j}$,那么
\[
\bar{\mu}(A) = \sum_{i,j} \mu(S_{i,j}) = \sum_{i} \bar{\mu}(B_{i})
\]
为了证明 (b),我们先考虑 $n=1$ 的情形,$B_{1}=B$.
$B = A \cup (B \cap A^c)$,且 $B \cap A^c \in \bar{\mathcal{S}}$,因此
\[
\bar{\mu}(A) \leq \bar{\mu}(A) + \bar{\mu}(B \cap A^c) = \bar{\mu}(B)
\]
对于 $n>1$ 的情况,令 $F_{k} = B_{1}^{c} \cap \cdots \cap B_{k-1}^{c} \cap B_{k}$,并注意
\[
\cup_{i} B_{i} = F_{1} \cup \cdots \cup F_{n} \\
A = A \cap (\cup_{i} B_{i}) = (A \cap F_{1}) \cup \cdots \cup (A \cap F_{n})
\]
因此利用 (a)、(b) 的 $n=1$ 情形以及 (a) 再次得到
\[
\bar{\mu}(A) = \sum_{k=1}^{n} \bar{\mu}(A \cap F_{k}) \leq \sum_{k=1}^{n} \bar{\mu}(F_{k}) = \bar{\mu}\left(\cup_{i} B_{i}\right)
\]
证明了定理 1.

    \end{prf}
    
    
    
    \begin{thm}
        [Theorem-Subsequence-Convergence-Implies-Sequence-Convergence-in-Topological-Space]
        {拓扑空间中子列收敛推出原列收敛的定理}
        [Theorem: Subsequence Convergence Implies Sequence Convergence in Topological Space]
        [gpt-4.1]
        设 $y_{n}$ 是拓扑空间中的元素序列.如果每一个子列 $y_{n(m)}$ 都存在进一步的子列 $y_{n(m_{k})}$ 收敛于 $y$,则 $y_{n} \to y$.
    \end{thm}
    
    
    
    \begin{prf}
        [Proof-of-Subsequence-Convergence-Implies-Sequence-Convergence-in-Topological-Space]
        {关于拓扑空间中子列收敛推出原列收敛的证明}
        [Proof of Subsequence Convergence Implies Sequence Convergence in Topological Space]
        [gpt-4.1]
        如果 $y_{n} 
ot\to y$,则存在包含 $y$ 的开集 $G$ 和一个子列 $y_{n(m)}$,使得对所有 $m$,有 $y_{n(m)} 
otin G$,但显然 $y_{n(m)}$ 的任何子列都不收敛于 $y$.
    \end{prf}
    
    
    
    \begin{dfn}
        [Definition-of-Simple-Function]
        {简单函数的定义}
        [Definition of Simple Function]
        [gpt-4.1]
        $\varphi$ 称为简单函数,如果 $\varphi(\omega) = \sum_{i=1}^n a_i 1_{A_i}(\omega)$,其中 $A_i$ 是两两不交的集合,且 $\mu(A_i) < \infty$.
    \end{dfn}
    
    
    
    \begin{dfn}
        [Definition-of-Integral-of-Simple-Function]
        {简单函数积分的定义}
        [Definition of Integral of Simple Function]
        [gpt-4.1]
        如果 $\varphi$ 是简单函数,则定义
\[
\int \varphi d\mu = \sum_{i=1}^n a_i \mu(A_i)
\]

    \end{dfn}
    
    
    
    \begin{dfn}
        [Definition-of-Almost-Everywhere-Greater-Than-or-Equal]
        {几乎处处大于等于的定义}
        [Definition of Almost Everywhere Greater Than or Equal]
        [gpt-4.1]
        $\varphi \geq \psi$ $\mu$-几乎处处(或 $\varphi \geq \psi$ $\mu$-a.e.)是指 $\mu(\{ \omega : \varphi(\omega) < \psi(\omega) \}) = 0$.
    \end{dfn}
    
    
    
    \begin{prf}
        [Proof-of-Properties-of-Mapping-Collections]
        {关于映射集合的性质的证明}
        [Proof of Properties of Mapping Collections]
        [gpt-4.1]
        将 $\{ X \in B \}$ 作为 $\{ \omega : X(\omega) \in B \}$ 的简写,有

\[
\begin{array}{c}
  \{ X \in \cup_{i} B_{i} \} = \cup_{i} \{ X \in B_{i} \} \\
  \{ X \in B^{c} \} = \{ X \in B \}^{c}
\end{array}
\]

因此,集合类 $\mathcal{B} = \{ B : \{ X \in B \} \in \mathcal{F} \}$ 是一个 $\sigma$-域.
    \end{prf}
    
    
    
    \begin{crl}
        [Corollary-on-Sigma-field-Structure-from-Mapping-Properties]
        {从映射集合性质推得的$\sigma$-域结构}
        [Corollary on Sigma-field Structure from Mapping Properties]
        [gpt-4.1]
        由前面证明中的两个等式可知,如果 $\mathcal{S}$ 是一个 $\sigma$-域,则 $\{ \{ X \in B \} : B \in \mathcal{S} \}$ 也是一个 $\sigma$-域.
    \end{crl}
    
    
    
    \begin{dfn}
        [Sigma-field-Generated-by-Mapping-$X$]
        {由映射$X$生成的$\sigma$-域}
        [Sigma-field Generated by Mapping $X$]
        [gpt-4.1]
        它是在$\Omega$上使$X$成为可测映射的最小$\sigma$-域,被称为由$X$生成的$\sigma$-域,记作$\sigma(X)$.
    \end{dfn}
    
    
    
    \begin{xmp}
        [Example-of-Collection-Systems-on-the-Real-Numbers]
        {实数集上集合系统的例子}
        [Example of Collection Systems on the Real Numbers]
        [gpt-4.1]
        
如果 $(S, \mathcal{S}) = (\mathbf{R}, \mathcal{R})$, 那么定理 1.3.1 中 $\mathcal{A}$ 的可能选择有 $\{ (-\infty, x] : x \in \mathbf{R} \}$ 或 $\{ (-\infty, x) : x \in \mathbf{Q} \}$,其中 $\mathbf{Q}$ 是有理数集.

    \end{xmp}
    
    
    
    \begin{thm}
        [Theorem-on-Normalized-Convergence-of-Random-Variables]
        {随机变量归一化收敛定理}
        [Theorem on Normalized Convergence of Random Variables]
        [gpt-4.1]
        
设 $0 \leq X_{1} \leq X_{2} \leq \cdots$ 是一列随机变量,满足 $E X_{n} \sim a n^{\alpha}$,其中 $a, \alpha > 0$,且 $\operatorname{var}(X_{n}) \leq B n^{\beta}$,其中 $\beta < 2\alpha$.则有
\[
\frac{X_{n}}{n^{\alpha}} \to a
\]
几乎处处收敛.

    \end{thm}
    
    
    
    \begin{xmp}
        [Example-of-Dense-Discontinuities-in-Distribution-Function-of-Discrete-Probability-Measure]
        {离散概率测度分布函数的稠密不连续性例子}
        [Example of Dense Discontinuities in Distribution Function of Discrete Probability Measure]
        [gpt-4.1]
        设 $q_1, q_2, \ldots$ 是有理数的一个枚举.令 $\alpha_i > 0$ 且 $\sum_{i=1}^{\infty} \alpha_i = 1$,并定义
\[
F(x) = \sum_{i=1}^{\infty} \alpha_i 1_{[q_i, \infty)}
\]
其中 $1_{[\theta, \infty)}(x) = 1$ 当且仅当 $x \in [\theta, \infty)$,否则为 $0$.
    \end{xmp}
    
    
    
    \begin{thm}
        [Limit-in-Probability-for-Row-Sums-of-Triangular-Arrays]
        {三角阵随机变量行和的概率极限}
        [Limit in Probability for Row Sums of Triangular Arrays]
        [gpt-4.1]
        设 $S_n$ 是一列随机变量序列,$\mu_n = E S_n$,$\sigma_n^2 = \mathrm{var}(S_n)$.若 $\sigma_n^2 / b_n^2 \to 0$,则
\[
\frac{S_n - \mu_n}{b_n} \to 0 \qquad \text{in probability}
\]

    \end{thm}
    
    
    
    \begin{prf}
        [Proof-of-the-Limit-Theorem-for-Triangular-Array-Row-Sums]
        {三角阵行和极限定理的证明}
        [Proof of the Limit Theorem for Triangular Array Row Sums]
        [gpt-4.1]
        我们的假设意味着
\[
E \left( \frac{S_n - \mu_n}{b_n} \right)^2 = b_n^{-2} \mathrm{var}(S_n) \to 0
\]
因此,所需结论可由引理2推出.
    \end{prf}
    
    
    
    \begin{thm}
        [Theorem-Equivalence-of-Independence-Condition-and-Product-of-Probabilities]
        {独立性条件等价于概率乘积的定理}
        [Theorem: Equivalence of Independence Condition and Product of Probabilities]
        [gpt-4.1]
        
若每个 $\mathcal{A}_i$ 都包含全集 $\Omega$,则以下条件等价:
\[
P\left( \cap_{i=1}^n A_i \right) = \prod_{i=1}^n P(A_i) \quad \text{whenever } A_i \in \mathcal{A}_i
\]
因为可以令 $A_i = \Omega$ 对所有 $i 
otin I$.

    \end{thm}
    
    
    
    \begin{prf}
        [Proof-of-Equivalence-of-Independence-Condition-and-Product-of-Probabilities]
        {独立性条件等价于概率乘积的证明}
        [Proof of Equivalence of Independence Condition and Product of Probabilities]
        [gpt-4.1]
        
如果 $\mathcal{A}_1, \mathcal{A}_2, \ldots, \mathcal{A}_n$ 是独立的,并且 $\bar{\mathcal{A}}_i = \mathcal{A}_i \cup \{\Omega\}$,那么 $\bar{\mathcal{A}}_1, \bar{\mathcal{A}}_2, \ldots, \bar{\mathcal{A}}_n$ 也是独立的.因为若 $A_i \in \bar{\mathcal{A}}_i$ 且 $I = \{ j : A_j = \Omega \}$,则 $\cap_i A_i = \cap_{i \in I} A_i$.

    \end{prf}
    
    
    
    \begin{prf}
        [Proof-of-Probability-Bound-for-$S-n$-and-$\bar{S}-n$]
        {关于 $S\_n$ 与 $\bar{S}_n$ 的概率界的证明}
        [Proof of Probability Bound for $S_n$ and $\bar{S}_n$]
        [gpt-4.1]
        设 $S_{n} = X_{n, 1} + \cdots + X_{n, n}$,令 $a_{n} = \sum_{k=1}^{n} E \bar{X}_{n, k}$,并记 $\bar{S}_{n} = \bar{X}_{n, 1} + \cdots + \bar{X}_{n, n}$.则有:

\[
P\left(\left| \frac{S_{n} - a_{n}}{b_{n}} \right| > \epsilon \right) \leq P(S_{n} 
eq \bar{S}_{n}) + P\left(\left| \frac{\bar{S}_{n} - a_{n}}{b_{n}} \right| > \epsilon \right)
\]

对于第一个项,由下式估计:

\[
P(S_{n} 
eq \bar{S}_{n}) \leq P\left(\cup_{k=1}^{n} \{ \bar{X}_{n, k} 
eq X_{n, k} \} \right) \leq \sum_{k=1}^{n} P( | X_{n, k} | > b_{n} )
\]

对于第二项,根据切比雪夫不等式、$a_{n} = E \bar{S}_{n}$、定理 2.1 和 $\operatorname{var}(X) \leq E X^{2}$,有:

\[
\begin{array}{l}
{\displaystyle
P\left( \left| \frac{\bar{S}_{n} - a_{n}}{b_{n}} \right| > \epsilon \right)
\leq \epsilon^{-2} E \left| \frac{\bar{S}_{n} - a_{n}}{b_{n}} \right|^{2}
= \epsilon^{-2} b_{n}^{-2} \operatorname{var}(\bar{S}_{n}) } \\
{\displaystyle
\quad\quad = (b_{n}\epsilon)^{-2} \sum_{k=1}^{n} \operatorname{var}(\bar{X}_{n, k})
\leq (b_{n}\epsilon)^{-2} \sum_{k=1}^{n} E (\bar{X}_{n, k})^{2}
}
\end{array}
\]

由此证明完成.

    \end{prf}
    
    
    
    \begin{xmp}
        [An-Example-Where-Changing-the-Order-of-Summation-Fails]
        {关于交换求和次序失败的一个例子}
        [An Example Where Changing the Order of Summation Fails]
        [gpt-4.1]
        
设 $X = Y = \{1, 2, \ldots\}$,$A = B =$ 全部子集,$\mu_{1} = \mu_{2} =$ 计数测度.对于 $m \geq 1$,定义 $f(m, m) = 1$,$f(m + 1, m) = -1$,其他情况下 $f(m, n) = 0$.则有

\[
\sum_{m} \sum_{n} f(m, n) = 1 \quad \text{但} \quad \sum_{n} \sum_{m} f(m, n) = 0
\]

    \end{xmp}
    
    
    
    \begin{thm}
        [Linearity-and-Monotonicity-of-Expectation]
        {期望的线性性质与单调性}
        [Linearity and Monotonicity of Expectation]
        [gpt-4.1]
        设 $X, Y \geq 0$ 或 $E|X|, E|Y| < \infty$,则有:

(a) $E(X + Y) = EX + EY$;

(b) $E(aX + b) = aE(X) + b$,其中 $a, b$ 为任意实数;

(c) 如果 $X \geq Y$,则 $EX \geq EY$.

    \end{thm}
    
    
    
    \begin{prf}
        [Proof-of-Monotonicity-Countable-Additivity-and-Limit-Properties-of-Measure]
        {测度单调性、可列次可加性及极限等性质的证明}
        [Proof of Monotonicity, Countable Additivity, and Limit Properties of Measure]
        [gpt-4.1]
        (i) 设 $B - A = B \cap A^{c}$ 是两个集合的差.用 $+$ 表示不交并,则 $B = A + (B - A)$,所以
\[
\mu(B) = \mu(A) + \mu(B - A) \geq \mu(A).
\]

(ii) 令 $A_{n}^{\prime} = A_{n} \cap A$,$B_{1} = A_{1}^{\prime}$,对 $n > 1$,$B_{n} = A_{n}^{\prime} - \bigcup_{m=1}^{n-1} A_{m}^{\prime}$.由于 $B_{n}$ 不交且并为 $A$,利用测度定义(ii)、$B_{m} \subset A_{m}$ 和本定理(i)可得
\[
\mu(A) = \sum_{m=1}^{\infty} \mu(B_{m}) \leq \sum_{m=1}^{\infty} \mu(A_{m})
\]

(iii) 令 $B_{n} = A_{n} - A_{n-1}$.则 $B_{n}$ 不交且有 $\bigcup_{m=1}^{\infty} B_{m} = A$,$\bigcup_{m=1}^{n} B_{m} = A_{n}$,所以
\[
\mu(A) = \sum_{m=1}^{\infty} \mu(B_{m}) = \lim_{n \to \infty} \sum_{m=1}^{n} \mu(B_{m}) = \lim_{n \to \infty} \mu(A_{n})
\]

(iv) $A_{1} - A_{n} \uparrow A_{1} - A$,由(iii)可知 $\mu(A_{1} - A_{n}) \uparrow \mu(A_{1} - A)$.又因 $A_{1} \supset A$,有 $\mu(A_{1} - A) = \mu(A_{1}) - \mu(A)$,由此得到 $\mu(A_{n}) \downarrow \mu(A)$.

    \end{prf}
    
    
    
    \begin{dfn}
        [Definition-of-Metric-for-Convergence-in-Probability]
        {概率收敛度量的定义}
        [Definition of Metric for Convergence in Probability]
        [gpt-4.1]
        
在随机变量的集合上,定义距离函数
\[
d(X, Y) = E \left( \frac{|X - Y|}{1 + |X - Y|} \right)
\]
其中 $E$ 表示期望运算.

    \end{dfn}
    
    
    
    \begin{thm}
        [Properties-and-Equivalence-of-Metric-for-Convergence-in-Probability]
        {概率收敛度量的性质与等价性}
        [Properties and Equivalence of Metric for Convergence in Probability]
        [gpt-4.1]
        
(a) 距离函数 $d(X, Y)$ 满足如下度量空间的性质:
(i) $d(X, Y) = 0$ 当且仅当 $X = Y$ 几乎处处成立;
(ii) $d(X, Y) = d(Y, X)$;
(iii) $d(X, Z) \leq d(X, Y) + d(Y, Z)$.

(b) 对于随机变量序列 $(X_n)$,有 $d(X_n, X) \to 0$ 当 $n \to \infty$ 时当且仅当 $X_n \to X$ 按概率收敛.

    \end{thm}
    
    
    
    \begin{thm}
        [Distribution-Theorem-of-Induced-Permutations]
        {排列诱导的分布定理}
        [Distribution Theorem of Induced Permutations]
        [gpt-4.1]
        设 $\pi_n$ 是由 $(X_1, \ldots, X_n)$ 诱导的排列,$\sigma_n$ 是 $\{1, \ldots, n\}$ 的一个随机排列,且与 $X$ 序列无关.则 $(X_{\sigma(1)}, \ldots, X_{\sigma(n)})$ 诱导的排列为 $\pi_n \circ \sigma_n$,且其分布与 $\pi_n$ 相同.特别地,对任意排列 $\pi$,$\pi \circ \sigma_n$ 在 $n!$ 个可能性上是均匀分布的.

    \end{thm}
    
    
    
    \begin{ppt}
        [Property-of-Conditional-Probability-Calculation]
        {条件概率计算性质}
        [Property of Conditional Probability Calculation]
        [gpt-4.1]
        若 $m < n$ 且 $i_{m+1}, \ldots, i_n$ 是 $\{1, \ldots, n\}$ 的互不相同的元素,则有
\[
P \left( A_m \mid \pi_n(j) = i_j \text{ for } m+1 \leq j \leq n \right) = 1 / m
\]

    \end{ppt}
    
    
    
    \begin{prf}
        [Proof-of-Probability-Property-for-Inverse-Mapping-of-Distribution-Function]
        {随机变量分布函数逆映射的概率性质证明}
        [Proof of Probability Property for Inverse Mapping of Distribution Function]
        [gpt-4.1]
        设 $\Omega = (0, 1)$, $\mathcal{F}$ 为 Borel 集合, $P$ 为 Lebesgue 测度.若 $\omega \in (0, 1)$, 令
\[
X(\omega) = \operatorname*{sup} \{ y : F(y) < \omega \}
\]
需要证明
\[
\{ \omega : X(\omega) \leq x \} = \{ \omega : \omega \leq F(x) \}
\]
由 $P$ 是 Lebesgue 测度可知 $P(\omega : \omega \leq F(x)) = F(x)$.

为验证 $(\star)$,若 $\omega \leq F(x)$,则 $X(\omega) \leq x$,因为 $x 
otin \{ y : F(y) < \omega \}$.反之,若 $\omega > F(x)$,由于 $F$ 右连续,则存在 $\epsilon > 0$ 使得 $F(x + \epsilon) < \omega$,此时 $X(\omega) \geq x + \epsilon > x$.

    \end{prf}
    
    
    
    \begin{dfn}
        [Inverse-Mapping-of-Distribution-Function]
        {分布函数的逆映射}
        [Inverse Mapping of Distribution Function]
        [gpt-4.1]
        虽然 $F$ 可能不是一一对应且非满射,我们仍称 $X$ 为 $F$ 的逆,并记作 $F^{-1}$.
    \end{dfn}
    
    
    
    \begin{dfn}
        [Semialgebra-of-Half-Open-Intervals]
        {半开区间的半代数}
        [Semialgebra of Half-Open Intervals]
        [gpt-4.1]
        设 $\mathcal{S}$ 是所有半开区间 $( a , b ]$ 的半代数,其中 $- \infty \leq a < b \leq \infty$.
    \end{dfn}
    
    
    
    \begin{dfn}
        [Definition-of-Measure-on-Semialgebra-of-Half-Open-Intervals]
        {在半开区间半代数上的测度的定义}
        [Definition of Measure on Semialgebra of Half-Open Intervals]
        [gpt-4.1]
        对半开区间半代数 $\mathcal{S}$ 上的测度 $\mu$ 的定义如下:

\[
F ( \infty ) = \lim_{x \uparrow \infty} F(x), \quad F(-\infty) = \lim_{x \downarrow -\infty} F(x) \quad \text{存在},
\]

且对于任意 $- \infty \leq a < b \leq \infty$,有 $\mu((a, b]) = F(b) - F(a)$,因为 $F(\infty) > -\infty$ 且 $F(-\infty) < \infty$,上述定义是有意义的.
    \end{dfn}
    
    
    
    \begin{thm}
        [Expectation-Variance-and-Probabilistic-Convergence-of-$T-n$]
        {关于 $T\_n$ 的期望与方差及其概率收敛性质}
        [Expectation, Variance, and Probabilistic Convergence of $T_n$]
        [gpt-4.1]
        
设 $T_n$ 满足
\[
E T_n = \sum_{k=1}^n \left( 1 - \frac{k-1}{n} \right)^{-1} = n \sum_{m=1}^n m^{-1} \sim n \log n,
\]
且
\[
\operatorname{var}\left( T_n \right) \le \sum_{k=1}^n \left( 1 - \frac{k-1}{n} \right)^{-2} = n^2 \sum_{m=1}^n m^{-2} \le n^2 \sum_{m=1}^{\infty} m^{-2}.
\]
取 $b_n = n \log n$,根据定理 2.6,有
\[
\frac{T_n - n \sum_{m=1}^n m^{-1}}{n \log n} \to 0 \quad \mathrm{in~probability}
\]
从而 $T_n / (n \log n) \to 1$ 以概率收敛.

    \end{thm}
    
    
    
    \begin{dfn}
        [Definition-of-Pairwise-Independent-Events]
        {两两独立事件的定义}
        [Definition of Pairwise Independent Events]
        [gpt-4.1]
        如果一组事件 $A_1, \ldots, A_n$ 满足对任意 $i 
eq j$,都有 $P(A_i \cap A_j) = P(A_i) P(A_j)$,则称这组事件是两两独立的.
    \end{dfn}
    
    
    
    \begin{ppt}
        [Independent-Events-Are-Necessarily-Pairwise-Independent]
        {独立事件必然两两独立}
        [Independent Events Are Necessarily Pairwise Independent]
        [gpt-4.1]
        如果一组事件是独立的,那么它们必然是两两独立的.
    \end{ppt}
    
    
    
    \begin{thm}
        [Theorem-on-Rate-of-Convergence-via-Random-Series-Proof]
        {随机级数证明收敛速度定理}
        [Theorem on Rate of Convergence via Random Series Proof]
        [gpt-4.1]
        如前所述,随机级数证明的一个优点是它可以对 $S_{n} / n \to \mu$ 的收敛速度给出估计.通过对每个随机变量减去 $\mu$,我们可以并且将假定 $\mu = 0$,而不损失一般性.
    \end{thm}
    
    
    
    \begin{thm}
        [Uniqueness-Theorem-for-Product-Measure-of-Finitely-Many-σ-Finite-Measure-Spaces]
        {有限多个σ-有限测度空间的乘积测度唯一性定理}
        [Uniqueness Theorem for Product Measure of Finitely Many σ-Finite Measure Spaces]
        [gpt-4.1]
        
设 $(\Omega_{i}, \mathcal{F}_{i}, \mu_{i})$,$i = 1, \ldots, n$,为 $n$ 个 $\sigma$-有限测度空间,$\Omega = \Omega_{1} \times \cdots \times \Omega_{n}$,则在由形如 $A_{1} \times \cdots \times A_{n}$,$A_{i} \in \mathcal{F}_{i}$ 的集合生成的 $\sigma$-代数 $\mathcal{F}$ 上,存在唯一的测度 $\mu$,使得
\[
\mu(A_{1} \times \cdots \times A_{n}) = \prod_{m=1}^{n} \mu_{m}(A_{m})
\]
当 $(\Omega_{i}, \mathcal{F}_{i}, \mu_{i}) = (\mathbf{R}, \mathcal{R}, \lambda)$ 对所有 $i$ 成立时,该结果给出了 $n$ 维欧氏空间 $\mathbf{R}^{n}$ 上 Borel 子集上的 Lebesgue 测度.

    \end{thm}
    
    
    
    \begin{ppt}
        [Expectation-and-Variance-Formulas-for-Geometric-Distribution]
        {几何分布的期望与方差公式}
        [Expectation and Variance Formulas for Geometric Distribution]
        [gpt-4.1]
        
设 $N$ 是参数为 $p$ 的几何分布的随机变量,则有:

\[
E N = \sum_{k = 1}^{\infty} k p (1 - p)^{k - 1} = 1 / p
\]

\[
E N (N - 1) = \sum_{k = 1}^{\infty} k (k - 1) p (1 - p)^{k - 1} = \frac{2 (1 - p)}{p^{2}}
\]

\[
\operatorname{var}(N) = E N^{2} - (E N)^{2} = E N (N - 1) + E N - (E N)^{2} = \frac{1 - p}{p^{2}}
\]

    \end{ppt}
    
    
    
    \begin{thm}
        [Limit-Theorem-for-the-Proportion-of-Working-Time-of-Light-Bulbs]
        {灯泡工作时间比例的极限定理}
        [Limit Theorem for the Proportion of Working Time of Light Bulbs]
        [gpt-4.1]
        设第 $i$ 个灯泡的工作时间为 $X_i$,损坏后到被替换的时间为 $Y_i$,其中 $X_i$ 和 $Y_i$ 是正且相互独立的随机变量,$X$ 的分布为 $F$,$Y$ 的分布为 $G$,且两者均有有限均值.令 $R_t$ 为区间 $[0, t]$ 内灯泡处于工作状态的总时间.则几乎必然有
\[
\frac{R_t}{t} \to \frac{E X_i}{E X_i + E Y_i}
\]

    \end{thm}
    
    
    
    \begin{dfn}
        [Definition-of-Finite-Rectangle]
        {有限矩形的定义}
        [Definition of Finite Rectangle]
        [gpt-4.1]
        
我们称 $A = (a_1, b_1] \times \cdots \times (a_d, b_d]$(其中 $-\infty < a_i < b_i < \infty$)为一个有限矩形,以强调不允许 $\infty$ 的出现.

    \end{dfn}
    
    
    
    \begin{dfn}
        [Definition-of-Vertices-of-a-Finite-Rectangle]
        {有限矩形的顶点的定义}
        [Definition of Vertices of a Finite Rectangle]
        [gpt-4.1]
        
对于有限矩形 $A = (a_1, b_1] \times \cdots \times (a_d, b_d]$,定义 $V = \{a_1, b_1\} \times \cdots \times \{a_d, b_d\}$,则 $V$ 即为矩形 $A$ 的顶点集合.

    \end{dfn}
    
    
    
    \begin{dfn}
        [Definition-of-Difference-Operator-of-$F$-on-a-Finite-Rectangle]
        {有限矩形上 $F$ 的差分算子定义}
        [Definition of Difference Operator of $F$ on a Finite Rectangle]
        [gpt-4.1]
        
若 $
u \in V$,则定义
\[
\Delta_{A} F = \sum_{
u \in V}\operatorname{sgn}(
u) F(
u)
\]
其中 $V$ 为有限矩形 $A$ 的顶点集合,$\operatorname{sgn}(
u)$ 表示每个顶点对应的符号.

    \end{dfn}
    
    
    
    \begin{lma}
        [Measurability-and-Integral-Equality-for-Set-Measures]
        {关于集合测度积分的可测性与等式}
        [Measurability and Integral Equality for Set Measures]
        [gpt-4.1]
        若 $E \in \mathcal{F}$, 则 $g(x) \equiv \mu_{2}(E_{x})$ 是 $\mathcal{A}$ 可测函数,并且
\[
\int_{X} g\, d\mu_{1} = \mu(E)
\]

    \end{lma}
    
    
    
    \begin{prf}
        [Proof-of-Measurability-and-Integral-Equality-for-Set-Measures]
        {关于集合测度积分的可测性与等式的证明}
        [Proof of Measurability and Integral Equality for Set Measures]
        [gpt-4.1]
        若结论对 $E_n$ 成立且 $E_n \uparrow E$,由定理 1.3.7 和单调收敛定理可知结论对 $E$ 也成立.由于 $\mu_1$ 和 $\mu_2$ 是 $\sigma$-有限的,因此只需证明当 $E \subset F \times G$ 且 $\mu_1(F) < \infty$, $\mu_2(G) < \infty$ 时结论成立,或者取 $\Omega = F \times G$,可无损地假设 $\mu(\Omega) < \infty$.设 $\mathcal{L}$ 为所有结论成立的集合.现检查 $\mathcal{L}$ 是一个 $\lambda$-系统.性质 (i) 显然成立,(iii) 由证明首句得出.检验 (ii),注意到
\[
\mu_2((A-B)_x) = \mu_2(A_x - B_x) = \mu_2(A_x) - \mu_2(B_x)
\]
对 $x$ 积分即可得第二个结论.由于 $\mathcal{L}$ 包含矩形集(生成 $\mathcal{F}$ 的 $\pi$-系统),由 $\pi$-$\lambda$ 定理可得所需结论.

    \end{prf}
    
    
    
    \begin{thm}
        [Jensens-Inequality]
        {Jensen不等式}
        [Jensen's Inequality]
        [gpt-4.1]
        设 $\varphi$ 是凸函数,即对所有 $\lambda \in (0,1)$ 和 $x, y \in \mathbf{R}$,有
\[
\lambda \varphi(x) + (1-\lambda)\varphi(y) \geq \varphi(\lambda x + (1-\lambda)y)
\]
若 $\mu$ 是概率测度,且 $f$ 和 $\varphi(f)$ 可积,则有
\[
\varphi\left( \int f d\mu \right) \leq \int \varphi(f) d\mu
\]

    \end{thm}
    
    
    
    \begin{prf}
        [Proof-of-Jensens-Inequality]
        {Jensen不等式的证明}
        [Proof of Jensen's Inequality]
        [gpt-4.1]
        令 $c = \int f d\mu$,令 $\ell(x) = a x + b$ 为线性函数,满足 $\ell(c) = \varphi(c)$ 且 $\varphi(x) \geq \ell(x)$.
由于凸性,有
\[
\lim_{h \downarrow 0} \frac{\varphi(c) - \varphi(c-h)}{h} \leq \lim_{h \downarrow 0} \frac{\varphi(c+h) - \varphi(c)}{h}
\]
(上述极限存在,因为数列单调.)取 $a$ 为两极限之间的任意数,令 $\ell(x) = a(x-c) + \varphi(c)$,则 $\ell$ 有所需性质.
建立 $\ell$ 的存在性后,其余部分可由定理 1.4.7 (iv) 得出.

    \end{prf}
    
    
    
    \begin{xmp}
        [Example-of-Uniform-Distribution-on-the-Cantor-Set]
        {康托集上的均匀分布例子}
        [Example of Uniform Distribution on the Cantor Set]
        [gpt-4.1]
        康托集 $C$ 的定义为:从 $[0,1]$ 中去除 $(1/3, 2/3)$,然后对剩余的每个区间去除中间的三分之一,依次进行.
我们定义一个相关的分布函数 $F$,令 $F(x) = 0$ 当 $x \leq 0$,$F(x) = 1$ 当 $x \geq 1$,$F(x) = 1/2$ 当 $x \in [1/3, 2/3]$,$F(x) = 1/4$ 当 $x \in [1/9, 2/9]$,$F(x) = 3/4$ 当 $x \in [7/9, 8/9]$,依此类推.然后利用单调性将 $F$ 扩展到整个 $[0,1]$.

    \end{xmp}
    
    
    
    \begin{dfn}
        [Moment-and-Mean-of-a-Random-Variable]
        {随机变量的矩与均值}
        [Moment and Mean of a Random Variable]
        [gpt-4.1]
        若 $k$ 为正整数,则 $E X^{k}$ 称为随机变量 $X$ 的第 $k$ 阶矩.第 $1$ 阶矩 $E X$ 通常称为均值,记作 $\mu$.
    \end{dfn}
    
    
    
    \begin{dfn}
        [Variance-of-a-Random-Variable]
        {随机变量的方差}
        [Variance of a Random Variable]
        [gpt-4.1]
        若 $E X^{2} < \infty$,则随机变量 $X$ 的方差定义为 $\operatorname{var}( X ) = E ( X - \mu )^{2}$,其中 $\mu = E X$.
    \end{dfn}
    
    
    
    \begin{ppt}
        [Expansion-Formula-for-Variance]
        {方差的展开式}
        [Expansion Formula for Variance]
        [gpt-4.1]
        对于随机变量 $X$,其方差满足
\[
\operatorname{var}( X ) = E ( X - \mu )^{2} = E X^{2} - 2 \mu E X + \mu^{2} = E X^{2} - \mu^{2}
\]
其中 $\mu = E X$.
    \end{ppt}
    
    
    
    \begin{ppt}
        [Variance-Upper-Bound-Property]
        {方差的上界性质}
        [Variance Upper Bound Property]
        [gpt-4.1]
        对于随机变量 $X$,有
\[
\operatorname{var}( X ) \leq E X^{2}
\]
其中 $E X^{2}$ 是 $X^{2}$ 的期望.
    \end{ppt}
    
    
    
    \begin{thm}
        [Basic-Properties-of-Measures]
        {测度的基本性质}
        [Basic Properties of Measures]
        [gpt-4.1]
        设 $\mu$ 是 $(\Omega, \mathcal{F})$ 上的一个测度,则有:
(i) 单调性(monotonicity):如果 $A \subset B$,则 $\mu(A) \leq \mu(B)$.
(ii) 次可加性(subadditivity):如果 $A \subset \bigcup_{m=1}^{\infty} A_{m}$,则 $\mu(A) \leq \sum_{m=1}^{\infty} \mu(A_{m})$.
(iii) 下连续性(continuity from below):如果 $A_{i} \uparrow A$(即 $A_{1} \subset A_{2} \subset \ldots$ 且 $\bigcup_{i} A_{i} = A$),则 $\mu(A_{i}) \uparrow \mu(A)$.

    \end{thm}
    
    
    
    \begin{thm}
        [Upper-Bound-on-Probability-that-Random-Variable-Y-Exceeds-a]
        {关于随机变量Y取大于等于a的概率的上界}
        [Upper Bound on Probability that Random Variable Y Exceeds a]
        [gpt-4.1]
        设随机变量 $Y$ 满足 $E Y = 0$,$\operatorname{var}(Y) = \sigma^2$,且 $a > 0$,则
\[
P(Y \geq a) \leq \frac{\sigma^2}{a^2 + \sigma^2}
\]
且存在一个 $Y$ 使得等号成立.

    \end{thm}
    
    
    
    \begin{prf}
        [Proof-of-Countable-Additivity-for-Product-Set-Measure]
        {积集测度可数可加性证明}
        [Proof of Countable Additivity for Product Set Measure]
        [gpt-4.1]
        
由定理 1.9 可知,如果 $A \times B = \bigsqcup_{i} (A_{i} \times B_{i})$ 是有限或可数个不交集合的并,则有
\[
\mu(A \times B) = \sum_{i} \mu(A_{i} \times B_{i})
\]
对每个 $x \in A$,令 $I(x) = \{ i : x \in A_{i} \}$,于是 $B = \bigsqcup_{i \in I(x)} B_{i}$,因此
\[
1_{A}(x) \mu_{2}(B) = \sum_{i} 1_{A_{i}}(x) \mu_{2}(B_{i})
\]
对 $\mu_{1}$ 做积分并利用练习 1.5.6 得到
\[
\mu_{1}(A) \mu_{2}(B) = \sum_{i} \mu_{1}(A_{i}) \mu_{2}(B_{i})
\]
从而证明了结论.

    \end{prf}
    
    
    
    \begin{xmp}
        [A-Counterexample-to-the-Converse-of-Set-Algebra-Statement]
        {关于集合代数逆命题的反例}
        [A Counterexample to the Converse of Set Algebra Statement]
        [gpt-4.1]
        An example in which the converse is false is:

Example 1.
    \end{xmp}
    
    
    
    \begin{xmp}
        [Example-of-Algebra-Structure-on-the-Set-of-Integers]
        {整数集上的代数结构例子}
        [Example of Algebra Structure on the Set of Integers]
        [gpt-4.1]
        Let $\Omega = \mathbf{Z} =$ the integers.

$\mathcal{A} =$ the collection of $\mathcal{A} \subset \mathbf{Z}$ so that $A$ or $A^c$ is finite is an algebra.
    \end{xmp}
    
    
    
    \begin{thm}
        [Inclusion-Exclusion-Principle]
        {容斥原理}
        [Inclusion-Exclusion Principle]
        [gpt-4.1]
        
设 $A_1, A_2, \ldots, A_n$ 是事件,且 $A = \cup_{i=1}^n A_i$.则有:

\[
P\left(\cup_{i=1}^n A_i\right) = \sum_{i=1}^n P(A_i) - \sum_{i<j} P(A_i \cap A_j)
\quad\quad\quad\quad\quad
+ \sum_{i<j<k} P(A_i \cap A_j \cap A_k) - \dots + (-1)^{n-1} P(\cap_{i=1}^n A_i)
\]

并且,指示函数满足
\[
\boldsymbol{1}_A = 1 - \prod_{i=1}^n (1 - \boldsymbol{1}_{A_i})
\]

    \end{thm}
    
    
    
    \begin{prf}
        [Proof-of-Inclusion-Exclusion-Principle]
        {容斥原理的证明}
        [Proof of Inclusion-Exclusion Principle]
        [gpt-4.1]
        
展开右侧,取期望值得到:

\[
P\left(\cup_{i=1}^n A_i\right) = \sum_{i=1}^n P(A_i) - \sum_{i<j} P(A_i \cap A_j)
\quad\quad\quad\quad\quad
+ \sum_{i<j<k} P(A_i \cap A_j \cap A_k) - \dots + (-1)^{n-1} P(\cap_{i=1}^n A_i)
\]

    \end{prf}
    
    
    
    \begin{prf}
        [Proof-of-Limit-Properties-for-i.i.d.-Random-Variables]
        {关于独立同分布随机变量极限性质的证明}
        [Proof of Limit Properties for i.i.d. Random Variables]
        [gpt-4.1]
        由引理 2.13 得

\[
E | X_1 | = \int_0^{\infty} P ( | X_1 | > x ) dx \leq \sum_{n=0}^{\infty} P ( | X_1 | > n )
\]

由于 $E | X_1 | = \infty$ 且 $X_1, X_2, \dots$ 是独立同分布的, 由第二类 Borel-Cantelli 引理得 $P ( | X_n | \geq n \text{ i.o.} ) = 1$.

为证明第二个结论, 注意到

\[
\frac{ S_n }{ n } - \frac{ S_{n+1} }{ n+1 } = \frac{ S_n }{ n ( n+1 ) } - \frac{ X_{n+1} }{ n+1 }
\]

在 $C \equiv \{ \omega : \lim_{n \to \infty} S_n / n \text{ 存在于 } ( -\infty, \infty ) \}$ 上, $S_n / ( n ( n+1 ) ) \to 0$.

因此, 在 $C \cap \{ \omega : | X_n | \geq n \text{ i.o.} \}$ 上, 有

\[
\left| \frac{ S_n }{ n } - \frac{ S_{n+1} }{ n+1 } \right| > \frac{2}{3} \quad \text{i.o.}
\]

与 $\omega \in C$ 的假设矛盾.

由此可知

\[
\{ \omega : | X_n | \geq n \ \text{i.o.} \} \cap C = \varnothing
\]

且由于 $P ( | X_n | \geq n \ \text{i.o.} ) = 1$, 得 $P ( C ) = 0$.

    \end{prf}
    
    
    
    \begin{thm}
        [Convergence-of-Monte-Carlo-Integration-Estimator]
        {Monte Carlo 积分估计的收敛性}
        [Convergence of Monte Carlo Integration Estimator]
        [gpt-4.1]
        
设 $f$ 是定义在 $[0,1]$ 上的可测函数,且 $\int_{0}^{1} |f(x)| dx < \infty$.令 $U_1, U_2, \dots$ 为在 $[0,1]$ 上独立且均匀分布的随机变量,定义
\[
I_n = n^{-1} (f(U_1) + \cdots + f(U_n))
\]
则 $I_n \to I \equiv \int_{0}^{1} f\, dx$ 以概率收敛.

    \end{thm}
    
    
    
    \begin{thm}
        [Sufficient-Condition-for-Independence-of-Random-Variables]
        {判定随机变量独立性的充分条件}
        [Sufficient Condition for Independence of Random Variables]
        [gpt-4.1]
        
设 $X_1, \ldots, X_n$ 为随机变量,若对所有 $x_1, \ldots, x_n \in (-\infty, \infty]$ 有
\[
P(X_1 \leq x_1, \ldots, X_n \leq x_n) = \prod_{i=1}^{n} P(X_i \leq x_i)
\]
则 $X_1, \ldots, X_n$ 独立.

    \end{thm}
    
    
    
    \begin{prf}
        [Proof-of-Sufficient-Condition-for-Independence-of-Random-Variables]
        {判定随机变量独立性的充分条件的证明}
        [Proof of Sufficient Condition for Independence of Random Variables]
        [gpt-4.1]
        
设 $\mathcal{A}_i =$ 形如 $\{X_i \leq x_i\}$ 的集合.根据设定,对所有 $x_1, \ldots, x_n$,有
\[
P(X_1 \leq x_1, \ldots, X_n \leq x_n) = \prod_{i=1}^{n} P(X_i \leq x_i)
\]
即各随机变量的分布函数之积给出了联合分布函数,因此满足独立性定义.

    \end{prf}
    
    
    
    \begin{thm}
        [Criterion-for-Convergence-of-Independent-Bernoulli-Variables]
        {独立伯努利变量收敛性的判别准则}
        [Criterion for Convergence of Independent Bernoulli Variables]
        [gpt-4.1]
        
设 $X_1, X_2, \dots$ 相互独立,且 $P(X_n = 1) = p_n$, $P(X_n = 0) = 1 - p_n$.则:

(i) $X_n \to 0$ 以概率收敛当且仅当 $p_n \to 0$;

(ii) $X_n \to 0$ 几乎处处收敛当且仅当 $\sum p_n < \infty$.

    \end{thm}
    
    
    
    \begin{prf}
        [Proof-of-Series-Convergence]
        {关于级数收敛性的证明}
        [Proof of Series Convergence]
        [gpt-4.1]
        设 $a_{n} = n^{1/2} (\log n)^{1/2 + \epsilon}$ 对于 $n \geq 2$,且 $a_{1} > 0$.

\[
\sum_{n=1}^{\infty} \operatorname{var}\left( X_{n} / a_{n} \right) = \sigma^{2} \left( \frac{1}{a_{1}^{2}} + \sum_{n=2}^{\infty} \frac{1}{n (\log n)^{1 + 2\epsilon}} \right) < \infty
\]

因此应用定理 2.5.6,可得 $\sum_{n=1}^{\infty} X_{n} / a_{n}$ 收敛,并且由定理 2.5.9 得到所指示的结果.

    \end{prf}
    
    
    
    \begin{dfn}
        [Construction-of-a-Sequence-of-Random-Vectors]
        {随机向量序列的构造}
        [Construction of a Sequence of Random Vectors]
        [gpt-4.1]
        令 $X_0 = (1, 0)$,并通过如下递归定义 $X_n \in \mathbf{R}^2$:$X_{n+1}$ 随机选自以原点为中心、半径为 $|X_n|$ 的圆盘,即 $X_{n+1} / |X_n|$ 在半径为 1 的圆盘上均匀分布,且与 $X_1, \ldots, X_n$ 独立.
    \end{dfn}
    
    
    
    \begin{thm}
        [Limit-of-Normalized-Logarithm-of-Random-Vector-Norms]
        {随机向量模长对数的归一化极限}
        [Limit of Normalized Logarithm of Random Vector Norms]
        [gpt-4.1]
        证明 $n^{-1} \log |X_n| \to c$(几乎处处收敛),并计算常数 $c$.
    \end{thm}
    
    
    
    \begin{thm}
        [Variance-Addition-Formula-for-Uncorrelated-Random-Variables]
        {无关随机变量方差加法公式}
        [Variance Addition Formula for Uncorrelated Random Variables]
        [gpt-4.1]
        
若随机变量 $X_1, X_2, \ldots, X_n$ 两两不相关,则
\[
\operatorname{var}(S_n) = \operatorname{var}(X_1) + \cdots + \operatorname{var}(X_n)
\]
其中 $S_n = X_1 + X_2 + \cdots + X_n$.

    \end{thm}
    
    
    
    \begin{ppt}
        [Variance-Property-of-Binary-Random-Variables]
        {零一变量的方差性质}
        [Variance Property of Binary Random Variables]
        [gpt-4.1]
        
若随机变量 $X_m \in \{0,1\}$,则有
\[
\operatorname{var}(X_m) \leq E(X_m^2) = E(X_m)
\]

    \end{ppt}
    
    
    
    \begin{crl}
        [Variance-Upper-Bound-for-Sums]
        {和的方差上界}
        [Variance Upper Bound for Sums]
        [gpt-4.1]
        
由上述性质可得
\[
\operatorname{var}(S_n) \leq E(S_n)
\]
其中 $S_n = X_1 + X_2 + \cdots + X_n$,且每个 $X_m \in \{0,1\}$.

    \end{crl}
    
    
    
    \begin{crl}
        [Upper-Bound-of-the-Normalized-Limit-Superior-of-Maximum-Length]
        {极大长度的归一化极限上界}
        [Upper Bound of the Normalized Limit Superior of Maximum Length]
        [gpt-4.1]
        
对于任意 $\epsilon > 0$, 有
\[
P(\ell_{n} \geq (1+\epsilon)\log_{2} n) \leq n^{-(1+\epsilon)}
\]
从 Borel-Cantelli 引理可知, 对于 $n \geq N_{\epsilon}$, $\ell_{n} \leq (1+\epsilon)\log_{2} n$ 恒成立.因此,
\[
\limsup_{n \to \infty} L_{n} / \log_{2} n \leq 1
\]

    \end{crl}
    
    
    
    \begin{crl}
        [Lower-Bound-of-the-Normalized-Limit-Inferior-of-Maximum-Length]
        {极大长度的归一化极限下界}
        [Lower Bound of the Normalized Limit Inferior of Maximum Length]
        [gpt-4.1]
        
将前 $n$ 次试验分成长度为 $[(1-\epsilon)\log_{2} n] + 1$ 的不交区块, 在每个区块内所有变量都为 1 的概率为
\[
2^{-[(1-\epsilon)\log_{2} n] - 1} \geq n^{-(1-\epsilon)} / 2
\]
若 $n$ 足够大, 且 $[n / ([(1-\epsilon)\log_{2} n] + 1)] \geq n / \log_{2} n$, 则有
\[
P(L_{n} \leq (1-\epsilon)\log_{2} n) \leq \left(1 - n^{-(1-\epsilon)}/2 \right)^{n / (\log_{2} n)} \leq \exp\left(- n^{\epsilon} / (2 \log_{2} n)\right)
\]

    \end{crl}
    
    
    
    \begin{xmp}
        [Counterexample-on-Measurability-and-Fubinis-Theorem]
        {关于可测性和交换积分次序的反例}
        [Counterexample on Measurability and Fubini's Theorem]
        [gpt-4.1]
        
设 $X = (0, 1)$, $\mathcal{A}$ 为 Borel 集合, $\mu_1$ 为 Lebesgue 测度;$Y = (0, 1)$, $\boldsymbol{B}$ 为所有子集, $\mu_2$ 为计数测度.定义函数 $f(x, y) = 1$ 当 $x = y$ 时, 否则为 $0$.

则有:
\[
\begin{array}{rl}
\displaystyle \int_{Y} f(x, y) \mu_2(dy) = 1 & \mathrm{~对于所有~} x \qquad \displaystyle \int_{X} \int_{Y} f(x, y) \mu_2(dy) \mu_1(dx) = 1 \\
\displaystyle \int_{X} f(x, y) \mu_1(dx) = 0 & \mathrm{~对于所有~} y \qquad \displaystyle \int_{Y} \int_{X} f(x, y) \mu_1(dx) \mu_2(dy) = 0
\end{array}
\]

该例说明了可测性的重要性,或某些集合论公理并非看上去那般无害.

    \end{xmp}
    
    
    
    \begin{dfn}
        [Definition-of-Product-Space-and-Rectangles]
        {积空间与矩形的定义}
        [Definition of Product Space and Rectangles]
        [gpt-4.1]
        设 $(X, \mathcal{A}, \mu_{1})$ 和 $(Y, \mathcal{B}, \mu_{2})$ 是两个 $\sigma$-有限测度空间.令
\[
\Omega = X \times Y = \{ (x, y) : x \in X, y \in Y \} \\
S = \{ A \times B : A \in \mathcal{A}, B \in \mathcal{B} \}
\]
$\mathcal{S}$ 中的集合称为矩形.

    \end{dfn}
    
    
    
    \begin{prf}
        [Proof-of-Linearity-and-Additivity-Properties-of-Integral]
        {积分线性性与加法性质的证明}
        [Proof of Linearity and Additivity Properties of Integral]
        [gpt-4.1]
        证明 (i) :由于我们可以取 $\varphi \equiv 0$,由定义可知 (i) 成立.

证明 (ii) :当 $a > 0$ 时,$a \varphi \leq a f$ 当且仅当 $\varphi \leq f$,所以
\[
\int a f d \mu = \operatorname*{sup}_{\varphi \leq f} \int a \varphi d \mu = \operatorname*{sup}_{\varphi \leq f} a \int \varphi d \mu = a \operatorname*{sup}_{\varphi \leq f} \int \varphi d \mu = a \int f d \mu
\]
当 $a < 0$ 时,$a \varphi \leq a f$ 当且仅当 $\varphi \geq f$,所以
\[
\int a f d \mu = \operatorname*{sup}_{\varphi \geq f} \int a \varphi d \mu = \operatorname*{sup}_{\varphi \geq f} a \int \varphi d \mu = a \operatorname*{inf}_{\varphi \geq f} \int \varphi d \mu = a \int f d \mu
\]

证明 (iii) :若 $\psi _ { 1 } \geq f$ 且 $\psi _ { 2 } \geq g$,则 $\psi _ { 1 } + \psi _ { 2 } \geq f + g$,所以
\[
\operatorname*{inf}_{\psi \geq f + g} \int \psi d \mu \leq \operatorname*{inf}_{\psi_{1} \geq f, \psi_{2} \geq g} \int \psi_{1} + \psi_{2} d \mu
\]
利用简单函数的线性性,推出
\[
\begin{array}{l}
\int f + g d \mu = \operatorname*{inf}_{\psi \geq f + g} \int \psi d \mu \\
\qquad \leq \operatorname*{inf}_{\psi_{1} \geq f, \psi_{2} \geq g} \int \psi_{1} d \mu + \int \psi_{2} d \mu = \int f d \mu + \int g d \mu
\end{array}
\]

证明反向不等式:对 $-f$ 和 $-g$ 应用上述结论以及 (ii),有
\[
- \int f + g d \mu \leq - \int f d \mu - \int g d \mu
\]

(iv)-(vi) 由引理 1 及 (i)-(iii) 推出.

    \end{prf}
    
    
    
    \begin{prf}
        [Proof-of-Lim-Sup-and-Infinite-Probability]
        {关于极限上确界与无穷大概率的证明}
        [Proof of Lim Sup and Infinite Probability]
        [gpt-4.1]
        Proof Since $a_{n} / n$ is increasing, $a_{kn} \geq k a_{n}$ for any integer $k$.

Using this and $a_{n}$ increasing,

\[
\sum_{n=1}^{\infty} P ( | X_{1} | \geq k a_{n} ) \geq \sum_{n=1}^{\infty} P ( | X_{1} | \geq a_{kn} ) \geq \frac{1}{k} \sum_{m=k}^{\infty} P ( | X_{1} | \geq a_{m} )
\]

The last observation shows that if the sum is infinite, $\limsup_{n \to \infty} | X_{n} | / a_{n} = \infty$.

Since $\max \{ | S_{n-1} |, | S_{n} | \} \geq | X_{n} | / 2$, it follows that $\limsup_{n \to \infty} | S_{n} | / a_{n} = \infty$.
    \end{prf}
    
    
    
    \begin{lma}
        [Lemma-on-Product-Probability-Formula]
        {乘积概率公式的引理}
        [Lemma on Product Probability Formula]
        [gpt-4.1]
        
若 $A_1 \in \sigma(\mathcal{A}_1)$ 且 $A_i \in {\mathcal{A}}_i$ 对于 $2 \leq i \leq n$, 则有
\[
P\left(\cap_{i=1}^{n} A_i\right) = P(A_1) P\left(\cap_{i=2}^{n} A_i\right) = \prod_{i=1}^{n} P(A_i)
\]

    \end{lma}
    
    
    
    \begin{dfn}
        [Definition-of-Expected-Value-for-Non-negative-Random-Variables]
        {非负随机变量的期望的定义}
        [Definition of Expected Value for Non-negative Random Variables]
        [gpt-4.1]
        
若 $X \geq 0$ 是概率空间 $(\Omega, \mathcal{F}, P)$ 上的随机变量,则其期望定义为 $EX = \int X \, dP$,该积分总是有意义,但可能等于无穷大.

    \end{dfn}
    
    
    
    \begin{dfn}
        [Definition-of-Expected-Value-for-General-Random-Variables]
        {一般随机变量期望的定义}
        [Definition of Expected Value for General Random Variables]
        [gpt-4.1]
        
对于任意随机变量 $X$,设 $x^+ = \operatorname{max}\{x, 0\}$ 为 $x$ 的正部,$x^- = \operatorname{max}\{-x, 0\}$ 为 $x$ 的负部.当 $EX^+ < \infty$ 或 $EX^- < \infty$ 时,定义 $EX$ 存在,并取 $EX = EX^+ - EX^-$.

    \end{dfn}
    
    
    
    \begin{dfn}
        [Alternative-Name-and-Notation-for-Expected-Value]
        {期望的别名及记号}
        [Alternative Name and Notation for Expected Value]
        [gpt-4.1]
        
$EX$ 通常也称为 $X$ 的均值,并记为 $\mu$.

    \end{dfn}
    
    
    
    \begin{thm}
        [Kroneckers-Lemma]
        {Kronecker 引理}
        [Kronecker's Lemma]
        [gpt-4.1]
        
若 $a_{n} \uparrow \infty$ 且 $\sum_{n=1}^{\infty} x_{n} / a_{n}$ 收敛,则有
\[
a_{n}^{-1} \sum_{m=1}^{n} x_{m} \to 0
\]

    \end{thm}
    
    
    
    \begin{thm}
        [Infimum-Supremum-limsup-and-liminf-of-Random-Variables-are-Random-Variables]
        {随机变量的上下极限仍为随机变量}
        [Infimum, Supremum, limsup and liminf of Random Variables are Random Variables]
        [gpt-4.1]
        如果 $X_{1}, X_{2}, \dots$ 是随机变量,则下列各项也是随机变量:
\[
\inf_{n} X_{n} \qquad \sup_{n} X_{n} \qquad \limsup_{n} X_{n} \qquad \liminf_{n} X_{n}
\]

    \end{thm}
    
    
    
    \begin{prf}
        [Proof-that-Infimum-Supremum-limsup-and-liminf-are-Random-Variables]
        {上下极限为随机变量的证明}
        [Proof that Infimum, Supremum, limsup and liminf are Random Variables]
        [gpt-4.1]
        由于数列的下确界小于 $a$ 当且仅当某一项小于 $a$(如果所有项都不小于 $a$,则下确界也不小于 $a$),有
\[
\{ \inf_{n} X_{n} < a \} = \cup_{n} \{ X_{n} < a \} \in \mathcal{F}
\]
类似地,
\[
\{ \sup_{n} X_{n} > a \} = \cup_{n} \{ X_{n} > a \} \in \mathcal{F}
\]
对于后两者,有
\[
\begin{array}{c}
\liminf_{n \to \infty} X_{n} = \sup_{n} \inf_{m \geq n} X_{m} \\
\limsup_{n \to \infty} X_{n} = \inf_{n} \sup_{m \geq n} X_{m}
\end{array}
\]
以第一种情况为例,$Y_{n} = \inf_{m \geq n} X_{m}$ 对每个 $n$ 都是随机变量,因此 $\sup_{n} Y_{n}$ 也是随机变量.

    \end{prf}
    
    
    
    \begin{thm}
        [Criterion-for-Random-Variables-Converging-to-Zero-via-Characteristic-Functions]
        {随机变量收敛到零的特征函数判据}
        [Criterion for Random Variables Converging to Zero via Characteristic Functions]
        [gpt-4.1]
        设 $Y_n$ 是一列随机变量,其特征函数为 $\varphi_n$,则 $Y_n \Rightarrow 0$ 当且仅当存在 $\delta > 0$ 使得对所有 $|t| \leq \delta$,有 $\varphi_n(t) \to 1$.
    \end{thm}
    
    
    
    \begin{thm}
        [Application-of-the-Central-Limit-Theorem]
        {中心极限定理的应用}
        [Application of the Central Limit Theorem]
        [gpt-4.1]
        
中心极限定理告诉我们 $(S_{n} - n/2) / \sqrt{n/4} \Rightarrow \chi$.

    \end{thm}
    
    
    
    \begin{thm}
        [Inequality-for-$2y-\sum-{k->-y}-k^{-2}-\leq-4$]
        {关于 $2y \sum\_{k > y} k^{-2} \leq 4$ 的不等式}
        [Inequality for $2y \sum_{k > y} k^{-2} \leq 4$]
        [gpt-4.1]
        如果 $y \geq 0$,则 $2y \sum_{k > y} k^{-2} \leq 4$.
    \end{thm}
    
    
    
    \begin{prf}
        [Proof-of-$2y-\sum-{k->-y}-k^{-2}-\leq-4$]
        {关于 $2y \sum\_{k > y} k^{-2} \leq 4$ 的证明}
        [Proof of $2y \sum_{k > y} k^{-2} \leq 4$]
        [gpt-4.1]
        我们首先注意到,如果 $m \geq 2$,则
\[
\sum_{k \geq m} k^{-2} \leq \int_{m-1}^{\infty} x^{-2} dx = (m-1)^{-1}
\]
当 $y \geq 1$ 时,求和从 $k = [y] + 1 \geq 2$ 开始,所以
\[
2y \sum_{k > y} k^{-2} \leq 2y/[y] \leq 4
\]
因为对于 $y \geq 1$,有 $y/[y] \leq 2$(最坏情况为 $y$ 接近 $2$).

    \end{prf}
    
    
    
    \begin{thm}
        [Exponential-Limit-Formula-for-Complex-Sequences]
        {复数列极限的指数极限公式}
        [Exponential Limit Formula for Complex Sequences]
        [gpt-4.1]
        
如果 $c_n \to c \in \mathbf{C}$,则 $(1 + c_n / n)^n \to e^c$.

    \end{thm}
    
    
    
    \begin{ppt}
        [Distribution-Property-of-Permutation-Induced-by-Rearrangement-of-Random-Variables]
        {随机变量重排诱导的排列的分布性质}
        [Distribution Property of Permutation Induced by Rearrangement of Random Variables]
        [gpt-4.1]
        
设 $X_1, \ldots, X_n$ 是一组随机变量,将它们按降序排列得到 $Y_1^n > Y_2^n > \cdots > Y_n^n$,并定义 $\{1, \ldots, n\}$ 的一个随机排列 $\pi_n$,其中 $\pi_n(i) = j$ 当且仅当 $X_i = Y_j^n$,即第 $i$ 个随机变量的秩为 $j$.由于 $(X_1, \ldots, X_n)$ 的分布在交换随机变量顺序时不变,因此排列 $\pi_n$ 在所有 $n!$ 个可能性上均匀分布.

    \end{ppt}
    
    
    
    \begin{lma}
        [Basic-Properties-of-the-Integral-of-Non-negative-Measurable-Functions]
        {非负可测函数积分的基本性质}
        [Basic Properties of the Integral of Non-negative Measurable Functions]
        [gpt-4.1]
        
设 $f, g \ge 0$,则有:

(i) $\int f \, d\mu \geq 0$ 

(ii) 若 $a > 0$,则 $\int a f \, d\mu = a \int f \, d\mu$

(iii) $\int (f + g) \, d\mu = \int f \, d\mu + \int g \, d\mu$

(iv) 若 $0 \leq g \leq f$ 几乎处处(a.e.),则 $\int g \, d\mu \leq \int f \, d\mu$

(v) 若 $0 \leq g = f$ 几乎处处(a.e.),则 $\int g \, d\mu = \int f \, d\mu$

    \end{lma}
    
    
    
    \begin{thm}
        [Dominated-Convergence-Theorem-Probabilistic-Version]
        {主导收敛定理(概率型)}
        [Dominated Convergence Theorem (Probabilistic Version)]
        [gpt-4.1]
        设 $X_n \to X$ 在概率意义下收敛,并且满足以下两个条件之一:
(a) 存在随机变量 $Y$ 使得 $|X_n| \leq Y$ 且 $E Y < \infty$;
(b) 存在连续函数 $g$,当 $|x| \to \infty$ 时 $g(x) > 0$ 且 $|x| / g(x) \to 0$,并且对所有 $n$ 有 $E g(X_n) \leq C < \infty$.
则 $E X_n \to E X$.

    \end{thm}
    
    
    
    \begin{thm}
        [Expectation-Properties-of-Independent-Random-Variables]
        {独立随机变量的期望性质}
        [Expectation Properties of Independent Random Variables]
        [gpt-4.1]
        
设 $X$ 和 $Y$ 相互独立,分布分别为 $\mu$ 和 $
u$.若 $h : \mathbb{R}^{2} \to \mathbb{R}$ 是可测函数,且 $h \geq 0$ 或 $E|h(X, Y)| < \infty$,则有
\[
E h(X, Y) = \iint h(x, y) \mu(dx) 
u(dy)
\]
特别地,若 $h(x, y) = f(x)g(y)$,其中 $f, g : \mathbb{R} \to \mathbb{R}$ 是可测函数,且 $f, g \ge 0$ 或 $E|f(X)|$ 和 $E|g(Y)| < \infty$,则有
\[
E f(X) g(Y) = E f(X) \cdot E g(Y)
\]

    \end{thm}
    
    
    
    \begin{prf}
        [Proof-of-Expectation-Properties-of-Independent-Random-Variables]
        {独立随机变量期望性质的证明}
        [Proof of Expectation Properties of Independent Random Variables]
        [gpt-4.1]
        
利用定理 1.6.9 以及 Fubini 定理(定理 1.7.2),有
\[
E h(X, Y) = \int_{\mathbb{R}^{2}} h \, d(\mu \times 
u) = \iint h(x, y) \mu(dx) 
u(dy)
\]
对于第二个结论,先考虑 $f, g \ge 0$ 的情形.此时,利用第一个结论,以及 $g(y)$ 与 $x$ 无关,再结合定理 1.6.9 两次,有
\[
\begin{array}{l}
{\displaystyle E f(X) g(Y) = \iint f(x) g(y) \mu(dx) 
u(dy) = \int g(y) \int f(x) \mu(dx) 
u(dy)} \\
{\displaystyle ~ = \int E f(X) g(y) 
u(dy) = E f(X) E g(Y)}
\end{array}
\]
对非负 $f$ 和 $g$ 将上述结论应用于 $|f|$ 和 $|g|$,得到 $E|f(X) g(Y)| = E|f(X)| E|g(Y)| < \infty$,并可重复上述推理得到所需结论.

    \end{prf}
    
    
    
    \begin{dfn}
        [Definition-of-Coupon-Collectors-Problem-and-Associated-Random-Variables]
        {Coupon collector 问题及相关随机变量的定义}
        [Definition of Coupon Collector's Problem and Associated Random Variables]
        [gpt-4.1]
        
设 $X_1, X_2, \ldots$ 是独立同分布的随机变量,均匀分布在 $\{1, 2, \ldots, n\}$ 上.令 $\tau_k^n = \operatorname*{inf} \{ m : | \{ X_1, \dots, X_m \} | = k \}$,即第一次收集到 $k$ 种不同物品的时刻.定义 $T_n = \tau_n^n$ 表示集齐全部 $n$ 种物品所需的时间.对 $1 \leq k \leq n$,定义 $X_{n, k} \equiv \tau_k^n - \tau_{k-1}^n$,表示得到与前 $k-1$ 种不同物品的等待时间,则 $X_{n, k}$ 服从参数为 $1 - (k-1)/n$ 的几何分布,且与之前的等待时间 $X_{n, j}$,$1 \leq j < k$ 独立.

    \end{dfn}
    
    
    
    \begin{dfn}
        [Definition-of-Generalized-Binomial-Coefficient]
        {广义二项式系数的定义}
        [Definition of Generalized Binomial Coefficient]
        [gpt-4.1]
        
${\binom { \beta } { n }} = \frac { \beta ( \beta - 1 ) \cdots ( \beta - n + 1 ) } { 1 \cdot 2 \cdots n }$

    \end{dfn}
    
    
    
    \begin{xmp}
        [Example-High-Dimensional-Cube-Approximates-the-Boundary-of-a-Ball]
        {高维立方体近似为球面边界的例子}
        [Example: High-Dimensional Cube Approximates the Boundary of a Ball]
        [gpt-4.1]
        
设 $X_1, X_2, \dots$ 是在区间 $(-1, 1)$ 上独立且均匀分布的随机变量.令 $Y_i = X_i^2$,则 $Y_i$ 也是独立的,因为它们是独立随机变量的函数.易知 $E Y_i = 1/3$ 且 $\mathrm{var}(Y_i) \leq E Y_i^2 \leq 1$,由定理 2.2.3 可得

\[
\frac{X_1^2 + \cdots + X_n^2}{n} \to \frac{1}{3} \quad \mathrm{in~probability~as~} n \to \infty
\]

设 $A_{n, \epsilon} = \{x \in \mathbb{R}^n : (1-\epsilon)\sqrt{n/3} < |x| < (1+\epsilon)\sqrt{n/3}\}$,其中 $|x| = (x_1^2 + \cdots + x_n^2)^{1/2}$.如果用 $|S|$ 表示 $S$ 的勒贝格测度,则上述结论意味着对于任意 $\epsilon > 0$,有 $|A_{n, \epsilon} \cap (-1, 1)^n|/2^n \to 1$.换言之,$(-1, 1)^n$ 绝大多数体积来自于 $A_{n, \epsilon}$,而 $A_{n, \epsilon}$ 几乎是半径为 $\sqrt{n/3}$ 的球的边界.

    \end{xmp}
    
    
    
    \begin{thm}
        [Basic-Properties-of-the-Integral-of-Integrable-Functions]
        {可积函数积分的基本性质}
        [Basic Properties of the Integral of Integrable Functions]
        [gpt-4.1]
        
设 $f$ 和 $g$ 是可积函数,则有:

(i) 若 $f \geq 0$ 几乎处处,则 $\int f \, d\mu \geq 0$.

(ii) 对任意 $a \in \mathbf{R}$,有 $\int a f \, d\mu = a \int f \, d\mu$.

(iii) $\int (f + g) \, d\mu = \int f \, d\mu + \int g \, d\mu$.

(iv) 若 $g \le f$ 几乎处处,则 $\int g \, d\mu \leq \int f \, d\mu$.

(v) 若 $g = f$ 几乎处处,则 $\int g \, d\mu = \int f \, d\mu$.

    \end{thm}
    
    
    
    \begin{thm}
        [$L^2$-Weak-Law-of-Large-Numbers]
        {$L^2$弱大数定律}
        [$L^2$ Weak Law of Large Numbers]
        [gpt-4.1]
        设 $X_1, X_2, \dots$ 是不相关的随机变量,且 $E X_i = \mu$ 且 $\operatorname{var}(X_i) \leq C < \infty$.若 $S_n = X_1 + \cdots + X_n$,则当 $n \to \infty$ 时,$S_n / n \to \mu$ 在 $L^2$ 意义下收敛,也在概率意义下收敛.
    \end{thm}
    
    
    
    \begin{prf}
        [Proof-of-$L^2$-Weak-Law-of-Large-Numbers]
        {$L^2$弱大数定律的证明}
        [Proof of $L^2$ Weak Law of Large Numbers]
        [gpt-4.1]
        证明 $L^2$ 收敛:注意到 $E(S_n / n) = \mu$,所以
\[
E\left(S_n / n - \mu\right)^2 = \operatorname{var}(S_n / n) = \frac{1}{n^2} (\operatorname{var}(X_1) + \cdots + \operatorname{var}(X_n)) \leq \frac{C n}{n^2} \to 0
\]
因此 $S_n / n$ 在 $L^2$ 收敛到 $\mu$.

为了说明概率收敛,应用引理2.
    \end{prf}
    
    
    
    \begin{thm}
        [A-Special-Case-of-the-Monotone-Convergence-Theorem]
        {单调收敛定理的一个特殊情形}
        [A Special Case of the Monotone Convergence Theorem]
        [gpt-4.1]
        若 $g_n \uparrow g$ 且 $\int g_1^{-} \, d\mu < \infty$, 则 $\int g_n \, d\mu \uparrow \int g \, d\mu$.
    \end{thm}
    
    
    
    \begin{thm}
        [Tightness-Theorem-in-the-Moment-Problem]
        {矩问题中的分布列紧性定理}
        [Tightness Theorem in the Moment Problem]
        [gpt-4.1]
        
假设对于每个 $k$,$\textstyle{\int x^{k} dF_{n}(x)}$ 都有极限 $\mu_{k}$.则根据定理3.2.14,分布序列是紧的,并且每个子列极限都具有矩 $\mu_{k}$.

    \end{thm}
    
    
    
    \begin{dfn}
        [Definition-of-Waiting-Time-for-Better-Than-First-Offer]
        {等待优于首次报价的时间的定义}
        [Definition of Waiting Time for Better Than First Offer]
        [gpt-4.1]
        设 $X_{0}, X_{1}, \ldots$ 是一组随机变量,表示你卖车时获得的一系列报价.定义
\[
N = \inf\{n \geq 1 : X_{n} > X_{0}\}
\]
为你第一次收到比首次报价更高报价所需等待的时间(报价次数).
    \end{dfn}
    
    
    
    \begin{ppt}
        [Lower-Bound-for-Probability-of-Waiting-Time-for-Better-Than-First-Offer]
        {等待优于首次报价时间的概率下界}
        [Lower Bound for Probability of Waiting Time for Better Than First Offer]
        [gpt-4.1]
        由对称性可得,对于所有 $n$,
\[
P(N > n) \geq \frac{1}{n+1}
\]
当分布是连续型时,实际上有 $P(N > n) = \frac{1}{n+1}$,但此处分布为一般情形且相等时机会归于先来者.
    \end{ppt}
    
    
    
    \begin{thm}
        [The-Expected-Waiting-Time-for-Better-Than-First-Offer-Is-Infinite]
        {等待优于首次报价时间的期望为无穷大}
        [The Expected Waiting Time for Better Than First Offer Is Infinite]
        [gpt-4.1]
        根据性质,
\[
EN = \sum_{n=0}^{\infty} P(N > n) \geq \sum_{n=0}^{\infty} \frac{1}{n+1} = \infty
\]
因此,直到你收到比第一次报价更高的报价所需的期望等待时间为无穷大.
    \end{thm}
    
    
    
    \begin{thm}
        [Uniqueness-Criterion-for-the-Stieltjes-Moment-Problem]
        {Stieltjes矩问题唯一性判据}
        [Uniqueness Criterion for the Stieltjes Moment Problem]
        [gpt-4.1]
        
若分布集中于 $[0, \infty)$,则当
\[
\lim_{k \to \infty} 
u_k^{1/2k} / 2k < \infty
\]
时,存在唯一的分布在 $[0, \infty)$ 上,其矩为给定的 $
u_k$.

    \end{thm}
    
    
    
    \begin{thm}
        [Uniqueness-Theorem-for-Moment-Growth-Rate]
        {矩母数增长率唯一性定理}
        [Uniqueness Theorem for Moment Growth Rate]
        [gpt-4.1]
        若 $\limsup_{k \to \infty} \mu_{2k}^{1 / 2 k} / 2k = r < \infty$, 则至多存在唯一的分布函数 $F$ 满足对所有正整数 $k$ 有 $\mu_k = \int x^k dF(x)$.
    \end{thm}
    
    
    
    \begin{thm}
        [Normal-Approximation-to-the-Poisson-Distribution]
        {泊松分布的正态近似}
        [Normal Approximation to the Poisson Distribution]
        [gpt-4.1]
        
设 $Z_{\lambda}$ 具有参数为 $\lambda$ 的泊松分布.如果 $X_{1}, X_{2}, \dots$ 是独立的且都服从参数为 1 的泊松分布,则 $S_{n} = X_{1} + \cdots + X_{n}$ 服从参数为 $n$ 的泊松分布.

    \end{thm}
    
    
    
    \begin{thm}
        [Equivalence-of-Convergence-in-Probability-and-Almost-Sure-Convergence-of-Subsequences]
        {收敛于概率与子列几乎处处收敛的等价性}
        [Equivalence of Convergence in Probability and Almost Sure Convergence of Subsequences]
        [gpt-4.1]
        $X_{n} \to X$ 在概率意义下收敛,当且仅当对于每一个子列 $X_{n(m)}$,都存在进一步的子列 $X_{n(m_{k})}$ 使得 $X_{n(m_{k})}$ 几乎处处收敛到 $X$.
    \end{thm}
    
    
    
    \begin{prf}
        [Proof-of-the-Equivalence-between-Convergence-in-Probability-and-Almost-Sure-Convergence-of-Subsequences]
        {收敛于概率与子列几乎处处收敛等价的证明}
        [Proof of the Equivalence between Convergence in Probability and Almost Sure Convergence of Subsequences]
        [gpt-4.1]
        设 $\epsilon_{k}$ 是一个正数列,且 $\epsilon_{k} \downarrow 0$.

对于每个 $k$,存在 $n(m_{k}) > n(m_{k-1})$,使得 $P(|X_{n(m_{k})} - X| > \epsilon_{k}) \leq 2^{-k}$.

由于
\[
\sum_{k=1}^{\infty} P(|X_{n(m_{k})} - X| > \epsilon_{k}) < \infty
\]
由Borel-Cantelli引理可知,$P(|X_{n(m_{k})} - X| > \epsilon_{k} \text{ i. o.}) = 0$,即 $X_{n(m_{k})} \to X$ 几乎处处收敛.
    \end{prf}
    
    
    
    \begin{thm}
        [Expectation-Formula-for-Product-of-Independent-Random-Variables]
        {独立随机变量乘积的期望公式}
        [Expectation Formula for Product of Independent Random Variables]
        [gpt-4.1]
        
如果 $X_{1}, \ldots, X_{n}$ 是独立的,并且满足 (a) $X_{i} \geq 0$ 对所有 $i$,或者 (b) $E|X_{i}| < \infty$ 对所有 $i$,那么
\[
E\left(\prod_{i=1}^{n} X_{i}\right) = \prod_{i=1}^{n} E X_{i}
\]
即左边的期望存在且等于右边的值.

    \end{thm}
    
    
    
    \begin{prf}
        [Proof-of-Expectation-Formula-for-Product-of-Independent-Random-Variables]
        {独立随机变量乘积的期望公式的证明}
        [Proof of Expectation Formula for Product of Independent Random Variables]
        [gpt-4.1]
        
令 $X = X_{1}$,$Y = X_{2} \cdots X_{n}$,根据定理 2.1.10,$X$ 和 $Y$ 独立.取 $f(x) = |x|$ 和 $g(y) = |y|$,有 $E|X_{1} \cdots X_{n}| = E|X_{1}| E|X_{2} \cdots X_{n}|$.由归纳法可得,如果 $1 \leq m \leq n$,
\[
E|X_{m} \cdots X_{n}| = \prod_{i=m}^{n} E|X_{i}|
\]
若 $X_{i} \geq 0$,则 $|X_{i}| = X_{i}$,由 $m = 1$ 的特殊情形得到所需结论.对于一般情形,注意到 $m = 2$ 的特殊情形说明 $E|Y| = E|X_{2} \cdots X_{n}| < \infty$,应用定理 2.1.12,取 $f(x) = x$ 和 $g(y) = y$,得到 $E(X_{1} \cdots X_{n}) = E X_{1} \cdot E(X_{2} \cdots X_{n})$,由归纳法得到所需结论.

    \end{prf}
    
    
    
    \begin{thm}
        [Kolmogorovs-Extension-Theorem]
        {Kolmogorov扩展定理}
        [Kolmogorov's Extension Theorem]
        [gpt-4.1]
        
设我们在 $(\mathbf{R}^n, \mathcal{R}^n)$ 上给定了一组一致的概率测度 $\mu_n$, 即

\[
\mu_{n+1}((a_1, b_1] \times \cdots \times (a_n, b_n] \times \mathbf{R}) = \mu_n((a_1, b_1] \times \cdots \times (a_n, b_n])
\]

那么在 $(\mathbf{R}^{\mathbf{N}}, \mathcal{R}^{\mathbf{N}})$ 上存在唯一的概率测度 $P$, 满足

\[
P(\omega : \omega_i \in (a_i, b_i], 1 \le i \le n) = \mu_n((a_1, b_1] \times \cdots \times (a_n, b_n])
\]

    \end{thm}
    
    
    
    \begin{thm}
        [Relationship-between-Characteristic-Function-Limit-Condition-and-Moments]
        {特征函数极限条件与矩的关系}
        [Relationship between Characteristic Function Limit Condition and Moments]
        [gpt-4.1]
        
若 $\lim_{t \downarrow 0} ( \varphi(t) - 1 ) / t^2 = c > -\infty$,则 $E X = 0$ 且 $E|X|^2 = -2c < \infty$.

    \end{thm}
    
    
    
    \begin{crl}
        [Case-Where-Characteristic-Function-Is-$1-+-ot^2$]
        {特征函数为 $1 + o(t^2)$ 的情形}
        [Case Where Characteristic Function Is $1 + o(t^2)$]
        [gpt-4.1]
        
特别地,如果 $\varphi(t) = 1 + o( t^2 )$,则 $\varphi(t) \equiv 1$.

    \end{crl}
    
    
    
    \begin{dfn}
        [Definition-of-the-Investment-Problem-Model]
        {投资问题模型的定义}
        [Definition of the Investment Problem Model]
        [gpt-4.1]
        
假设每年初你可以用 \$1 购买债券,年末价值为 \$4,或购买股票,其年末价值为随机变量 $V \geq 0$.
如果你每年始终将固定比例 $p$ 的财富投资于债券,则第 $n+1$ 年末的财富为
\[
W_{n+1} = (ap + (1-p)V_n)W_n
\]
其中 $V_1, V_2, \dots$ 是独立同分布随机变量,满足 $E V_n^2 < \infty$ 且 $E(V_n^{-2}) < \infty$.

    \end{dfn}
    
    
    
    \begin{thm}
        [Theorem-on-Convergence-of-Wealth-Growth-Rate]
        {财富增长率收敛定理}
        [Theorem on Convergence of Wealth Growth Rate]
        [gpt-4.1]
        
设 $W_n$ 是根据上述投资策略得到的财富,则
\[
n^{-1} \log W_n \to c(p)
\]
当 $n \to \infty$ 时,几乎必然成立.

    \end{thm}
    
    
    
    \begin{ppt}
        [Concavity-of-the-Growth-Rate]
        {增长率的凹性}
        [Concavity of the Growth Rate]
        [gpt-4.1]
        
$c(p)$ 是关于 $p$ 的凹函数.

    \end{ppt}
    
    
    
    \begin{thm}
        [Theorem-on-Changing-the-Order-of-Integration]
        {积分次序交换定理}
        [Theorem on Changing the Order of Integration]
        [gpt-4.1]
        如果 $\int_{X} \int_{Y} |f(x, y)| \mu_2(dy) \mu_1(dx) < \infty$, 则
\[
\int_{X} \int_{Y} f(x, y) \mu_2(dy) \mu_1(dx) = \int_{X \times Y} f \, d(\mu_1 \times \mu_2) = \int_{Y} \int_{X} f(x, y) \mu_1(dx) \mu_2(dy)
\]

    \end{thm}
    
    
    
    \begin{crl}
        [Corollary-on-Changing-the-Order-of-Integration-for-Counting-Measure]
        {计数测度上的积分次序交换推论}
        [Corollary on Changing the Order of Integration for Counting Measure]
        [gpt-4.1]
        设 $X = \{1, 2, \ldots\}$, $\mathcal{A}$ 为 $X$ 的所有子集, $\mu_1$ 为计数测度, 则上述定理成立.
    \end{crl}
    
    
    
    \begin{thm}
        [Theorem-on-Changing-the-Order-of-Summation-and-Integration]
        {可列加和与积分交换定理}
        [Theorem on Changing the Order of Summation and Integration]
        [gpt-4.1]
        如果 $\sum_{n} \int |f_n| d\mu < \infty$, 则
\[
\sum_{n} \int f_n d\mu = \int \sum_{n} f_n d\mu
\]

    \end{thm}
    
    
    
    \begin{thm}
        [Integral-Inequality-and-$L^1$-and-$L^\infty$-Norms]
        {积分不等式与$L^1$与$L^\infty$范数}
        [Integral Inequality and $L^1$ and $L^\infty$ Norms]
        [gpt-4.1]
        
设可测函数$f, g$,则有
\[
\int |fg| \, d\mu \leq \| f \|_{1} \| g \|_{\infty}
\]
其中$\| f \|_{1}$是$f$的$L^1$范数,$\| g \|_{\infty}$是$g$的$L^\infty$范数.

    \end{thm}
    
    
    
    \begin{thm}
        [Limit-Representation-of-$L^\infty$-Norm-under-Probability-Measure]
        {概率测度下的$L^\infty$范数极限表示}
        [Limit Representation of $L^\infty$ Norm under Probability Measure]
        [gpt-4.1]
        
若$\mu$是概率测度,则有
\[
\| f \|_{\infty} = \lim_{p \to \infty} \| f \|_{p}
\]
其中$\| f \|_{p}$为$p$阶$L^p$范数.

    \end{thm}
    
    
    
    \begin{prf}
        [Proof-of-Truncated-Sum-Convergence]
        {关于截断和的收敛性的证明}
        [Proof of Truncated Sum Convergence]
        [gpt-4.1]
        证明 令 $Y_{k} = X_{k} 1_{(|X_{k}| \leq k^{1/p})}$,$T_{n} = Y_{1} + \cdots + Y_{n}$,则
\[
\sum_{k=1}^{\infty} P(Y_{k} 
eq X_{k}) = \sum_{k=1}^{\infty} P(|X_{k}|^{p} > k) \leq E |X_{k}|^{p} < \infty
\]
由 Borel-Cantelli 引理可知 $P(Y_{k} 
eq X_{k}~\text{i.o.}) = 0$,因此只需证明 $T_{n} / n^{1/p} \to 0$.
利用 $\operatorname{var}(Y_{m}) \leq E(Y_{m}^{2})$,引理 2.2.13(当 $p = 2$),$P(|Y_{m}| > y) \leq P(|X_{1}| > y)$,以及 Fubini 定理(所有项均 $\geq 0$),有
\[
\begin{array}{rl}
\sum_{m=1}^{\infty} \operatorname{var}(Y_{m} / m^{1/p}) & \leq \sum_{m=1}^{\infty} E Y_{m}^{2} / m^{2/p} \\
& \leq \sum_{m=1}^{\infty} \sum_{n=1}^{m} \int_{(n-1)^{1/p}}^{n^{1/p}} \frac{2y}{m^{2/p}} P(|X_{1}| > y) dy \\
& = \sum_{n=1}^{\infty} \int_{(n-1)^{1/p}}^{n^{1/p}} \sum_{m=n}^{\infty} \frac{2y}{m^{2/p}} P(|X_{1}| > y) dy
\end{array}
\]

    \end{prf}
    
    
    
    \begin{thm}
        [Moment-Formula-for-the-Standard-Normal-Distribution]
        {标准正态分布矩的公式}
        [Moment Formula for the Standard Normal Distribution]
        [gpt-4.1]
        标准正态分布的偶数阶矩为
\[
E X^{2n} = \frac{(2n)!}{2^n n!} = (2n-1)(2n-3) \cdots 3 \cdot 1 \equiv (2n-1)!!
\]

    \end{thm}
    
    
    
    \begin{dfn}
        [Tightness-of-a-Family-of-Measures]
        {测度族的紧性}
        [Tightness of a Family of Measures]
        [gpt-4.1]
        若测度族 $\{ \mu_i , i \in I \}$ 满足:对任意 $\varepsilon > 0$,存在 $M > 0$,使得 $\sup_{i} \mu_i ( [ -M, M ]^c ) < \varepsilon$,即 $\sup_{i} \mu_i ( [ -M, M ]^c ) \to 0$ 当 $M \to \infty$,则称该测度族是紧的.
    \end{dfn}
    
    
    
    \begin{prf}
        [Derivation-of-the-Probability-Density-Function-of-the-Sum-of-Random-Variables]
        {随机变量之和的概率密度函数的推导}
        [Derivation of the Probability Density Function of the Sum of Random Variables]
        [gpt-4.1]
        
设 $X \sim \mathrm{Gamma}(\alpha, \lambda)$,$Y \sim \mathrm{Gamma}(\beta, \lambda)$ 独立,则 $X + Y$ 的概率密度函数为
\[
f_{X + Y}(x) = \int_{0}^{x} \frac{\lambda^{\alpha} (x - y)^{\alpha - 1}}{\Gamma(\alpha)} e^{-\lambda(x - y)} \frac{\lambda^{\beta} y^{\beta - 1}}{\Gamma(\beta)} e^{-\lambda y} dy
\]
\[
= \frac{\lambda^{\alpha + \beta} e^{-\lambda x}}{\Gamma(\alpha)\Gamma(\beta)} \int_{0}^{x} (x - y)^{\alpha - 1} y^{\beta - 1} dy
\]
对积分部分变量替换 $y = x u$, $dy = x du$ 得
\[
x^{\alpha + \beta - 1} \int_{0}^{1} (1 - u)^{\alpha - 1} u^{\beta - 1} du = \int_{0}^{x} (x - y)^{\alpha - 1} y^{\beta - 1} dy
\]
故有归一化常数
\[
c_{\alpha, \beta} = \frac{1}{\Gamma(\alpha)\Gamma(\beta)} \int_{0}^{1} (1 - u)^{\alpha - 1} u^{\beta - 1} du
\]
此常数应为 $c_{\alpha, \beta} = 1/\Gamma(\alpha + \beta)$,由 Beta 分布定义可知.

另一种证明方式是对上式两边乘以 $e^{-x}$,对 $x$ 从 $0$ 到 $\infty$ 积分,右侧用 Fubini 定理有
\[
\Gamma(\alpha + \beta) \int_{0}^{1} (1 - u)^{\alpha - 1} u^{\beta - 1} du
= \int_{0}^{\infty} \int_{0}^{x} y^{\beta - 1} e^{-y} (x - y)^{\alpha - 1} e^{-(x - y)} dy dx
= \int_{0}^{\infty} y^{\beta - 1} e^{-y} \int_{y}^{\infty} (x - y)^{\alpha - 1} e^{-(x - y)} dx dy
= \Gamma(\alpha)\Gamma(\beta)
\]
从而得归一化常数结论.

    \end{prf}
    
    
    
    \begin{thm}
        [Covariance-of-Uncorrelated-Variables-is-Zero]
        {不相关变量协方差为零}
        [Covariance of Uncorrelated Variables is Zero]
        [gpt-4.1]
        
若 $X_i$ 和 $X_j$ 不相关,则
\[
E\left[(X_i - \mu_i)(X_j - \mu_j)\right] = E[X_i X_j] - \mu_i E[X_j] - \mu_j E[X_i] + \mu_i \mu_j = E[X_i X_j] - \mu_i \mu_j = 0
\]
即不相关的随机变量之间的协方差为零.

    \end{thm}
    
    
    
    \begin{thm}
        [Distribution-of-Independent-Random-Variables]
        {独立随机变量的分布定理}
        [Distribution of Independent Random Variables]
        [gpt-4.1]
        
假设 $X_{1}, \ldots, X_{n}$ 是独立的随机变量,且 $X_{i}$ 的分布为 $\mu_{i}$,则 $(X_{1}, \ldots, X_{n})$ 的分布为 $\mu_{1} \times \cdots \times \mu_{n}$.

    \end{thm}
    
    
    
    \begin{prf}
        [Proof-of-Distribution-of-Independent-Random-Variables]
        {独立随机变量的分布定理的证明}
        [Proof of Distribution of Independent Random Variables]
        [gpt-4.1]
        
利用以下定义:(i) $A_{1} \times \cdots \times A_{n}$,(ii) 独立性,(iii) $\mu_{i}$,以及 (iv) $\mu_{1} \times \cdots \times \mu_{n}$

\[
\begin{array}{l}
{\displaystyle P\left((X_{1}, \ldots, X_{n}) \in A_{1} \times \cdots \times A_{n}\right) = P\left(X_{1} \in A_{1}, \ldots, X_{n} \in A_{n}\right)} \\
{\displaystyle \quad = \prod_{i=1}^{n} P\left(X_{i} \in A_{i}\right) = \prod_{i=1}^{n} \mu_{i}(A_{i}) = \mu_{1} \times \cdots \times \mu_{n}\left(A_{1} \times \cdots \times A_{n}\right)}
\end{array}
\]

最后一个公式表明,$(X_{1}, \ldots, X_{n})$ 的分布与测度 $\mu_{1} \times \cdots \times \mu_{n}$ 在集合 $A_{1} \times \cdots \times A_{n}$ 上是一致的,而这些集合构成了生成 $\mathcal{R}^{n}$ 的 $\pi$-系统.

    \end{prf}
    
    
    
    \begin{thm}
        [Limit-Theorem-for-Self-normalized-Sums]
        {自归一化和的极限定理}
        [Limit Theorem for Self-normalized Sums]
        [gpt-4.1]
        
设 $X_1, X_2, \dots$ 是独立同分布的随机变量,满足 $E X_i = 0$,$E X_i^2 = \sigma^2 \in (0, \infty)$,则有
\[
\sum_{m=1}^n X_m \Bigg/ \left( \sum_{m=1}^n X_m^2 \right)^{1/2} \Rightarrow \chi
\]

    \end{thm}
    
    
    
    \begin{thm}
        [Theorem-on-Convergence-of-Series-and-Limit-of-Normalized-Sequence]
        {关于级数收敛与归一化序列极限的定理}
        [Theorem on Convergence of Series and Limit of Normalized Sequence]
        [gpt-4.1]
        
(i) 若 $\lim_{n \to \infty} \sum_{m=1}^{n} X_{m}$ 存在,则 $\lim_{n \to \infty} \sum_{m=1}^{n} Y_{m}$ 也存在.

(ii) 令 $T_{n} = \left( \sum_{m \leq n} Y_{m} \right) / c_{n}^{1/2}$,则 $T_{n} \Rightarrow 0$.

    \end{thm}
    
    
    
    \begin{xmp}
        [Example-of-Two-Characteristic-Functions-Coinciding-on-$[-1-1]$]
        {两个特征函数在区间 $[-1, 1]$ 上相同的例子}
        [Example of Two Characteristic Functions Coinciding on $[-1, 1]$]
        [gpt-4.1]
        
定义 $\psi(t)$ 使其在区间 $[-1, 1]$ 上与 $\varphi$ 相同,并以 2 为周期,即 $\psi(t) = \psi(t + 2)$.$\psi$ 的傅里叶级数为
\[
\psi(u) = \frac{1}{2} + \sum_{n = -\infty}^{\infty} \frac{2}{\pi^{2}(2n - 1)^{2}} \exp(i(2n - 1)\pi u)
\]
该表达式是一个离散分布的特征函数,其中
\[
P(X = 0) = 1/2 \quad {\mathrm{and}} \quad P(X = (2n - 1)\pi) = 2\pi^{-2}(2n - 1)^{-2} \quad n \in \mathbf{Z}.
\]

    \end{xmp}
    
    
    
    \begin{thm}
        [Limit-of-Log-Normalized-Extremes]
        {极值的对数正规化极限}
        [Limit of Log-Normalized Extremes]
        [gpt-4.1]
        
设 $X_1, X_2, \dots$ 是独立同分布的随机变量,满足 $P(X_i > x) = e^{-x}$,令 $M_n = \max_{1 \leq m \leq n} X_m$.则有:

(i)
\[
\limsup_{n \to \infty} \frac{X_n}{\log n} = 1 \quad \text{a.s.}
\]

(ii)
\[
\frac{M_n}{\log n} \to 1 \quad \text{a.s.}
\]

    \end{thm}
    
    
    
    \begin{lma}
        [Lemma-on-the-Convolution-of-Distribution-Functions-Difference]
        {关于分布函数的差值卷积的引理}
        [Lemma on the Convolution of Distribution Functions' Difference]
        [gpt-4.1]
        设 $F$ 和 $G$ 是分布函数,且 $G'(x) \leq \lambda < \infty$.
令 $\Delta(x) = F(x) - G(x)$,$\eta = \operatorname{sup} |\Delta(x)|$,$\Delta_{L} = \Delta * H_{L}$,$\eta_{L} = \operatorname{sup} |\Delta_{L}(x)|$.

    \end{lma}
    
    
    
    \begin{lma}
        [Lemma-on-$h-n\epsilon$]
        {关于 $h\_n(\epsilon)$ 的引理}
        [Lemma on $h_n(\epsilon)$]
        [gpt-4.1]
        
设
\[
h _ { n } ( \epsilon ) = \sum _ { m = 1 } ^ { n } E ( X _ { n , m } ^ { 2 } ; | X _ { n , m } | > \epsilon )
\]

    \end{lma}
    
    
    
    \begin{thm}
        [Convergence-of-Expectation-of-Continuous-Functions-of-Binomial-Sample-Mean]
        {二项分布样本均值连续函数的期望收敛定理}
        [Convergence of Expectation of Continuous Functions of Binomial Sample Mean]
        [gpt-4.1]
        
设 $S_n$ 为 $n$ 次独立伯努利试验(参数为 $p$)的和,$f$ 是定义在 $[0,1]$ 上的连续函数,则有
\[
\lim_{n \to \infty} E\, f \left( \frac{S_n}{n} \right) = f(p)
\]
证明如下:  
记 $M = \operatorname* { sup } _ { x \in [ 0 , 1 ] } | f ( x ) |$,任取 $\epsilon > 0$,可取 $\delta > 0$,使得当 $| x - y | < \delta$ 时,$| f ( x ) - f ( y ) | < \epsilon$(因 $f$ 在有界区间上一致连续).  
由 Jensen 不等式及 Chebyshev 不等式可得
\[
| E f ( S _ { n } / n ) - f ( p ) | \leq E | f ( S _ { n } / n ) - f ( p ) | \leq \epsilon + 2 M P ( | S _ { n } / n - p | > \delta )
\]
其中
\[
P ( | S _ { n } / n - p | > \delta ) \leq \frac { \mathrm { var } ( S _ { n } / n ) } { \delta ^ { 2 } } = \frac { p ( 1 - p ) } { n \delta ^ { 2 } } \leq \frac { 1 } { 4 n \delta ^ { 2 } }
\]
令 $n \to \infty$,则 $\operatorname*{sup}_{n \to \infty} | E f ( S _ { n } / n ) - f ( p ) | \le \epsilon$,而 $\epsilon$ 任意,因此极限成立.

    \end{thm}
    
    
    
    \begin{thm}
        [Lindeberg-Feller-Theorem-Generalization-of-Central-Limit-Theorem-for-Triangular-Arrays]
        {Lindeberg-Feller定理(三角阵中心极限定理推广)}
        [Lindeberg-Feller Theorem (Generalization of Central Limit Theorem for Triangular Arrays)]
        [gpt-4.1]
        
对于每个 $n$,设 $X_{n,m}$,$1 \leq m \leq n$,为相互独立的随机变量,且 $E X_{n,m} = 0$.假设:

(i) $\sum_{m=1}^{n} E X_{n,m}^2 \to \sigma^2 > 0$;

(ii) 对任意 $\epsilon > 0$,有 $\lim_{n \to \infty} \sum_{m=1}^{n} E(|X_{n,m}|^2; |X_{n,m}| > \epsilon) = 0$.

则当 $n \to \infty$ 时,$S_n = X_{n,1} + \cdots + X_{n,n} \Rightarrow \sigma \chi$.

    \end{thm}
    
    
    
    \begin{thm}
        [Theorem-on-Continuity-and-Convergence-in-Probability]
        {连续函数与概率收敛的传递性定理}
        [Theorem on Continuity and Convergence in Probability]
        [gpt-4.1]
        如果 $f$ 是连续函数,且 $X_n \to X$ 在概率意义下收敛,则 $f(X_n) \to f(X)$ 也在概率意义下收敛.如果进一步 $f$ 有界,则 $E f(X_n) \to E f(X)$.
    \end{thm}
    
    
    
    \begin{prf}
        [Proof-of-Theorem-on-Continuity-and-Convergence-in-Probability]
        {连续函数与概率收敛的传递性定理的证明}
        [Proof of Theorem on Continuity and Convergence in Probability]
        [gpt-4.1]
        设 $X_{n(m)}$ 是任意子列,则根据定理 2.3.2 存在进一步的子列 $X_{n(m_k)} \to X$ 几乎处处收敛.由于 $f$ 连续,由练习 1.3.3 得 $f(X_{n(m_k)}) \to f(X)$ 几乎处处收敛,且由定理 2.3.2 得 $f(X_n) \to f(X)$ 在概率意义下收敛.如果 $f$ 有界,则由有界收敛定理,$E f(X_{n(m_k)}) \to E f(X)$,再对 $y_n = E f(X_n)$ 应用定理 2.3.3 即得所需结果.
    \end{prf}
    
    
    
    \begin{cxmp}
        [Counterexample-of-Different-Probability-Density-Functions-with-Same-Moments]
        {具有相同矩的不同概率密度函数的反例}
        [Counterexample of Different Probability Density Functions with Same Moments]
        [gpt-4.1]
        设 $\lambda \in ( 0 , 1 )$,且 $-1 \leq a \leq 1$,定义

\[
f_{a, \lambda} ( x ) = c_{\lambda} \exp ( - | x |^{\lambda} ) \{ 1 + a \sin ( \beta | x |^{\lambda} \operatorname{sgn} ( x ) ) \}
\]

其中 $\beta = \tan ( \lambda \pi / 2 )$,$1 / c_{\lambda} = \int \exp ( - | x |^{\lambda} ) d x$.

证明这些是概率密度函数,且对固定的 $\lambda$,它们具有相同的矩,只需证明

\[
\int x^{n} \exp ( - | x |^{\lambda} ) \sin ( \beta | x |^{\lambda} \operatorname{sgn} ( x ) ) d x = 0 \quad \mathrm{for}~ n = 0 , 1 , 2 , \dots
\]

对于偶数 $n$,由于被积函数是奇函数,上述等式显然成立.

    \end{cxmp}
    
    
    
    \begin{thm}
        [Normalization-and-Convergence-to-Limiting-Distribution]
        {归一化和收敛至极限分布}
        [Normalization and Convergence to Limiting Distribution]
        [gpt-4.1]
        
设 $c_n = \sum_{m=1}^{n} \operatorname{var}(Y_m)$,$X_{n, m} = (Y_m - E Y_m) / c_n^{1/2}$,则有 $E X_{n, m} = 0$,$\sum_{m=1}^{n} E X_{n, m}^2 = 1$,且对任意 $\epsilon > 0$,
\[
\sum_{m=1}^{n} E(|X_{n, m}|^2 ; |X_{n, m}| > \epsilon) \to 0
\]
当 $S_n = X_{n, 1} + \cdots + X_{n, n}$ 时,由定理 3.10 可得 $S_n \Rightarrow \chi$.

    \end{thm}
    
    
    
    \begin{dfn}
        [Definition-of-Arithmetic-Distribution]
        {算术型分布的定义}
        [Definition of Arithmetic Distribution]
        [gpt-4.1]
        如果随机变量 $X$ 是一个格型分布,且可取 $b = 0$,即 $P(X \in h\mathbf{Z}) = 1$,则称 $X$ 是算术型分布.
    \end{dfn}
    
    
    
    \begin{thm}
        [Proof-of-Uniform-Convergence-Theorem-for-Distribution-Functions]
        {分布函数一致收敛定理的证明}
        [Proof of Uniform Convergence Theorem for Distribution Functions]
        [gpt-4.1]
        若对每个 $k$,存在分点 $x_{j,k}$,$1 \leq j \leq k-1$,使得对于任意 $n \geq N_k(\omega)$,有
\[
| F_n(x_{j,k}) - F(x_{j,k}) | < k^{-1} \quad \mathrm{and} \quad | F_n(x_{j,k}-) - F(x_{j,k}-) | < k^{-1}
\]
成立.若令 $x_{0,k} = -\infty$,$x_{k,k} = \infty$,上述不等式对 $j=0$ 或 $k$ 亦成立.若 $x \in (x_{j-1,k}, x_{j,k})$,$1 \leq j \leq k$,则根据分布函数的单调性及 $F(x_{j,k}-) - F(x_{j-1,k}) \le k^{-1}$,有
\[
\begin{array}{rl}
& F_n(x) \leq F_n(x_{j,k}-) \leq F(x_{j,k}-) + k^{-1} \leq F(x_{j-1,k}) + 2k^{-1} \leq F(x) + 2k^{-1} \\
& F_n(x) \geq F_n(x_{j-1,k}) \geq F(x_{j-1,k}) - k^{-1} \geq F(x_{j,k}-) - 2k^{-1} \geq F(x) - 2k^{-1}
\end{array}
\]
因此,$\sup_{x} |F_n(x) - F(x)| \leq 2k^{-1}$,从而得证.

    \end{thm}
    
    
    
    \begin{thm}
        [Limit-Distribution-of-Normalized-Sums-with-Bounded-Random-Variables-and-Divergent-Variance]
        {关于有界随机变量和方差发散时归一化和的极限分布}
        [Limit Distribution of Normalized Sums with Bounded Random Variables and Divergent Variance]
        [gpt-4.1]
        
设随机变量 $X_i$ 满足 $|X_i| \le M$,且 $\sum_n \operatorname{var}(X_n) = \infty$,则有
\[
\frac{S_n - E S_n}{\sqrt{\operatorname{var}(S_n)}} \Rightarrow \chi
\]
其中 $S_n$ 为部分和,$\chi$ 表示极限分布(具体分布需根据上下文定义).

    \end{thm}
    
    
    
    \begin{thm}
        [Density-Formula-for-Sum-of-Independent-Random-Variables]
        {独立随机变量和的密度公式}
        [Density Formula for Sum of Independent Random Variables]
        [gpt-4.1]
        
假设 $X$ 具有密度 $f$,$Y$ 具有分布函数 $G$,且 $X$ 与 $Y$ 独立,则 $X+Y$ 的密度为
\[
h(x) = \int f(x - y) dG(y)
\]

当 $Y$ 具有密度 $g$ 时,上述公式可写为
\[
h(x) = \int f(x - y) g(y) dy
\]

    \end{thm}
    
    
    
    \begin{prf}
        [Proof-of-Density-Formula-for-Sum-of-Independent-Random-Variables]
        {独立随机变量和的密度公式的证明}
        [Proof of Density Formula for Sum of Independent Random Variables]
        [gpt-4.1]
        
由定理 2.1.15、密度函数的定义以及 Fubini 定理 (定理 1.7.2),由于所有函数均非负,有
\[
\begin{array}{l}
P(X + Y \leq z) = \displaystyle \int F(z - y) dG(y) = \int \int_{-\infty}^{z} f(x - y) dx dG(y) \\
\qquad = \int_{-\infty}^{z} \int f(x - y) dG(y) dx
\end{array}
\]
最后一行说明 $X + Y$ 的密度为 $h(x) = \int f(x - y) dG(y)$.
第二个公式由第一个公式结合 $dG(y)$ 的含义推出.

    \end{prf}
    
    
    
    \begin{xmp}
        [An-Example-Where-$EXY-=-EX-\cdot-EY$-but-Variables-Are-Not-Independent]
        {期望乘积等于乘积期望但不独立的例子}
        [An Example Where $E(XY) = EX \cdot EY$ but Variables Are Not Independent]
        [gpt-4.1]
        
设随机变量 $X$ 和 $Y$ 的联合分布由下表给出:

\[
\begin{array}{cccccc}
& & & Y & & \\
& & 1 & 0 & -1 & \\
1 & 0 & a & 0 & \\
X & 0 & b & c & b \\
-1 & 0 & a & 0 & \\
\end{array}
\]

其中 $a, b > 0$, $c \geq 0$, 且 $2a + 2b + c = 1$.
由构造可知 $XY \equiv 0$,对称性可得 $EX = 0$ 且 $EY = 0$,因此 $E(XY) = 0 = EX \cdot EY$.
但随机变量并非独立,因为
\[
P(X = 1, Y = 1) = 0 < ab = P(X = 1)P(Y = 1)
\]

    \end{xmp}
    
    
    
    \begin{dfn}
        [Definition-of-Uncorrelated-Random-Variables]
        {不相关随机变量的定义}
        [Definition of Uncorrelated Random Variables]
        [gpt-4.1]
        
若两个随机变量 $X$ 和 $Y$ 满足 $EX^2, EY^2 < \infty$ 且 $EXY = EX \cdot EY$,则称它们是不相关的.有限二阶矩是必要的条件,因为由Cauchy-Schwarz不等式可知 $E|XY| < \infty$.

    \end{dfn}
    
    
    
    \begin{dfn}
        [Definition-of-Infinite-Product-Space]
        {无限乘积空间的定义}
        [Definition of Infinite Product Space]
        [gpt-4.1]
        
无限乘积空间 $\mathbf{R}^{\mathbf{N}}$ 定义为
\[
\mathbf{R}^{\mathbf{N}} = \{ (\omega_1, \omega_2, \ldots) : \omega_i \in \mathbf{R} \} = \{ \text{functions } \omega : \mathbf{N} \to \mathbf{R} \}
\]
其中 $\mathbf{N} = \{ 1, 2, \ldots \}$ 表示自然数集.

    \end{dfn}
    
    
    
    \begin{dfn}
        [Definition-of-Coordinate-Mapping-on-Infinite-Product-Space]
        {无限乘积空间上的坐标映射的定义}
        [Definition of Coordinate Mapping on Infinite Product Space]
        [gpt-4.1]
        
在无限乘积空间 $\mathbf{R}^{\mathbf{N}}$ 上,定义随机变量 $X_i$ 为 $X_i(\omega) = \omega_i$.

    \end{dfn}
    
    
    
    \begin{dfn}
        [Definition-of-Product-σ-field-on-Infinite-Product-Space]
        {无限乘积空间上的乘积σ-代数的定义}
        [Definition of Product σ-field on Infinite Product Space]
        [gpt-4.1]
        
将 $\mathbf{R}^{\mathbf{N}}$ 配备上乘积 $\sigma$-代数 $\mathcal{R}^{\mathbf{N}}$,它由有限维集合生成,即形如 $\{ \omega : \omega_i \in B_i, 1 \le i \le n \}$ 的集合,其中 $B_i \in \mathcal{R}$.

    \end{dfn}
    
    
    
    \begin{thm}
        [Theorem-on-$\exp-|t|^\alpha$-as-a-Characteristic-Function]
        {关于$\exp(-|t|^\alpha)$为特征函数的定理}
        [Theorem on $\exp(-|t|^\alpha)$ as a Characteristic Function]
        [gpt-4.1]
        对于 $0 < \alpha \leq 1$,$\exp(-|t|^\alpha)$ 是一个特征函数.

(当 $\alpha=1$ 时,对应柯西分布.)

    \end{thm}
    
    
    
    \begin{xmp}
        [Example-of-Coin-Flips]
        {抛硬币的例子}
        [Example of Coin Flips]
        [gpt-4.1]
        
设 $X_{1}, X_{2}, \dots$ 是一列独立同分布(i.i.d.)的随机变量,满足 $P(X_{i} = 0) = P(X_{i} = 1) = 1/2$.如果 $X_{i} = 1$ 表示第 $i$ 次抛硬币结果为正面,则 $S_{n} = X_{1} + \cdots + X_{n}$ 表示第 $n$ 次时正面出现的总次数.

    \end{xmp}
    
    
    
    \begin{thm}
        [Dominated-Convergence-Theorem]
        {受控收敛定理}
        [Dominated Convergence Theorem]
        [gpt-4.1]
        若 $f_{n} \to f$ 几乎处处成立,且对所有 $n$ 有 $|f_{n}| \leq g$,其中 $g$ 可积,则有 $\int f_{n} \, d\mu \to \int f \, d\mu$.
    \end{thm}
    
    
    
    \begin{prf}
        [Proof-of-the-Dominated-Convergence-Theorem]
        {受控收敛定理的证明}
        [Proof of the Dominated Convergence Theorem]
        [gpt-4.1]
        $f_{n} + g \geq 0$,因此 Fatou 引理给出
\[
\liminf_{n \to \infty} \int (f_{n} + g) \, d\mu \geq \int (f + g) \, d\mu
\]
两边都减去 $\int g \, d\mu$,得到
\[
\liminf_{n \to \infty} \int f_{n} \, d\mu \geq \int f \, d\mu
\]
对 $-f_{n}$ 应用上一个结果,得
\[
\limsup_{n \to \infty} \int f_{n} \, d\mu \leq \int f \, d\mu
\]
从而证明完成.
    \end{prf}
    
    
    
    \begin{prf}
        [Proof-of-Probability-Inequality]
        {概率不等式的证明}
        [Proof of Probability Inequality]
        [gpt-4.1]
        
证明如下:

设 $A_{k} = \{ |S_{k}| \ge x$ 但 $|S_{j}| < x$ 对所有 $j < k \}$,即我们按照 $|S_{k}|$ 首次超过 $x$ 的时刻将事件分解.由于各 $A_{k}$ 互不相交,且 $(S_{n} - S_{k})^{2} \ge 0$,有

\[
\begin{array}{l}
E S_{n}^{2} \geq \sum_{k=1}^{n} \int_{A_{k}} S_{n}^{2} dP = \sum_{k=1}^{n} \int_{A_{k}} S_{k}^{2} + 2 S_{k} (S_{n} - S_{k}) + (S_{n} - S_{k})^{2} dP \\
\geq \sum_{k=1}^{n} \int_{A_{k}} S_{k}^{2} dP + \sum_{k=1}^{n} \int 2 S_{k} 1_{A_{k}} \cdot (S_{n} - S_{k}) dP
\end{array}
\]

其中 $S_{k} 1_{A_{k}} \in \sigma(X_{1}, \ldots, X_{k})$,$S_{n} - S_{k} \in \sigma(X_{k+1}, \ldots, X_{n})$,由定理 2.1.10 可知它们独立,利用定理 2.1.13 及 $E(S_{n} - S_{k}) = 0$ 可得:

\[
\int 2 S_{k} 1_{A_{k}} \cdot (S_{n} - S_{k}) dP = E(2 S_{k} 1_{A_{k}}) \cdot E(S_{n} - S_{k}) = 0
\]

注意到在 $A_{k}$ 上有 $|S_{k}| \geq x$,且各 $A_{k}$ 互不相交,

\[
E S_{n}^{2} \geq \sum_{k=1}^{n} \int_{A_{k}} S_{k}^{2} dP \geq \sum_{k=1}^{n} x^{2} P(A_{k}) = x^{2} P\left(\max_{1 \leq k \leq n} |S_{k}| \geq x\right)
\]

    \end{prf}
    
    
    
    \begin{thm}
        [Local-Limit-Theorem-for-the-Lattice-Case]
        {格点情形的局部极限定理}
        [Local Limit Theorem for the Lattice Case]
        [gpt-4.1]
        
设 $X_1, X_2, \dots$ 是一列独立同分布随机变量,满足 $E X_i = 0$,$E X_i^2 = \sigma^2 \in (0, \infty)$,且具有公共格点分布,跨度为 $h$.则存在适用于格点分布的局部极限定理.

    \end{thm}
    
    
    
    \begin{thm}
        [Shannons-Theorem-Entropy-and-Asymptotic-Equipartition-Property]
        {香农定理与信息熵及渐近等分性质}
        [Shannon's Theorem, Entropy and Asymptotic Equipartition Property]
        [gpt-4.1]
        
设 $X_1, X_2, \dotsc$ 是取值于 $\{1, \dotsc, r\}$ 的独立随机变量,且 $P(X_i = k) = p(k) > 0$,其中 $1 \leq k \leq r$.令 $\pi_n(\omega) = p(X_1(\omega)) \cdots p(X_n(\omega))$ 表示前 $n$ 次试验所观察到的结果的概率.则由于 $\log \pi_n(\omega)$ 是独立随机变量之和,根据强大数定律,有

\[
- n^{-1} \log \pi_n(\omega) \to H \equiv - \sum_{k=1}^r p(k) \log p(k) \quad \mathrm{a.s.}
\]

其中常数 $H$ 被称为信息源的熵,表示其随机性的度量.

此外,渐近等分性质成立:若 $\epsilon > 0$,则当 $n \to \infty$ 时,
\[
P\left\{ \exp(-n(H+\epsilon)) \leq \pi_n(\omega) \leq \exp(-n(H-\epsilon)) \right\} \to 1
\]

    \end{thm}
    
    
    
    \begin{dfn}
        [Definition-of-Entropy-of-an-Information-Source]
        {信息源的熵的定义}
        [Definition of Entropy of an Information Source]
        [gpt-4.1]
        
信息源的熵 $H$ 定义为
\[
H \equiv - \sum_{k=1}^r p(k) \log p(k)
\]
其中 $p(k)$ 为每个符号的概率.熵度量了信息源的随机性.

    \end{dfn}
    
    
    
    \begin{thm}
        [Local-Limit-Theorem-for-Symmetric-Random-Walk]
        {对称随机游走局部极限定理}
        [Local Limit Theorem for Symmetric Random Walk]
        [gpt-4.1]
        
设 $X_1, X_2, \dots$ 是独立同分布的随机变量,且 $P(X_1 = 1) = P(X_1 = -1) = 1/2$.设 $k_n$ 是一列整数,使得 $2 k_n / (2n)^{1/2} \to x$,则有
\[
P(S_{2n} = 2 k_n) \sim (\pi n)^{-1/2} \exp(- x^2 / 2)
\]
其中 $S_{2n} = X_1 + X_2 + \cdots + X_{2n}$.

    \end{thm}
    
    
    
    \begin{dfn}
        [Definition-of-the-Longest-Run-Length-of-Positive-Entries]
        {最长连续正项长度的定义}
        [Definition of the Longest Run Length of Positive Entries]
        [gpt-4.1]
        
设 $\{X_n\}$ 是满足 $P(X_{n}=1) = P(X_{n}=-1) = 1/2$ 的随机变量序列.定义在时刻 $n$,$\ell_{n} = \max\{m : X_{n-m+1} = \ldots = X_{n} = 1\}$ 表示以 $n$ 结尾的连续 $+1$ 的最长长度,$L_{n} = \max_{1 \leq m \leq n} \ell_{m}$ 表示前 $n$ 项中的最长连续 $+1$ 的长度.

    \end{dfn}
    
    
    
    \begin{thm}
        [Limit-Theorem-for-the-Longest-Run-Length-of-Positive-Entries]
        {最长连续正项长度的极限定理}
        [Limit Theorem for the Longest Run Length of Positive Entries]
        [gpt-4.1]
        
对于满足 $P(X_{n}=1) = P(X_{n}=-1) = 1/2$ 的随机变量序列,有
\[
L_{n} / \log_{2} n \to 1 \quad \text{a.s.}
\]
即随着 $n$ 增大,前 $n$ 项的最长连续 $+1$ 的长度与 $\log_2 n$ 的比值几乎必然收敛于 $1$.

    \end{thm}
    
    
    
    \begin{thm}
        [Poisson-Approximation-and-Fraction-of-Empty-Boxes]
        {泊松分布逼近与空盒率}
        [Poisson Approximation and Fraction of Empty Boxes]
        [gpt-4.1]
        如果 $n \to \infty$ 且 $r / n \to c$,则在随机将 $r$ 个球放入 $n$ 个盒子的情况下,任一给定盒子的球数将接近于均值为 $c$ 的泊松分布.由此可知,空盒子的比例趋近于 $e^{-c}$.
    \end{thm}
    
    
    
    \begin{thm}
        [Berry-Esseen-Theorem---Rate-of-Convergence-Estimate]
        {Berry-Esseen定理-收敛速率估计}
        [Berry-Esseen Theorem - Rate of Convergence Estimate]
        [gpt-4.1]
        
设 $X_{1}, X_{2}, \dots$ 为独立同分布随机变量,满足 $E X_{i} = 0$, $E X_{i}^{2} = \sigma^{2}$, 且 $E|X_{i}|^{3} = \rho < \infty$.若 $F_{n}(x)$ 为 $(X_{1} + \cdots + X_{n}) / (\sigma \sqrt{n})$ 的分布函数,而 $\mathcal{N}(x)$ 为标准正态分布函数,则有

\[
| F_{n}(x) - \mathcal{N}(x) | \leq \frac{3\rho}{\sigma^{3} \sqrt{n}}
\]

该不等式对所有 $n$ 和 $x$ 均成立,但由于 $\rho \geq \sigma^{3}$,仅当 $n \geq 10$ 时有实际意义.

    \end{thm}
    
    
    
    \begin{thm}
        [Preservation-of-Independence-under-Complements-and-Indicator-Variables]
        {独立性在补集和指示变量下的保持}
        [Preservation of Independence under Complements and Indicator Variables]
        [gpt-4.1]
        设 $A_1, A_2, \ldots, A_n$ 是独立的事件.
(i) $A_1^c, A_2, \ldots, A_n$ 也是独立的;
(ii) $1_{A_1}, \ldots, 1_{A_n}$ (即各自的指示变量)也是独立的.
    \end{thm}
    
    
    
    \begin{prf}
        [Proof-of-Preservation-of-Independence-under-Complements-and-Indicator-Variables]
        {独立性在补集和指示变量下的保持的证明}
        [Proof of Preservation of Independence under Complements and Indicator Variables]
        [gpt-4.1]
        (i) 令 $B_1 = A_1^c$,$B_i = A_i$ 对于 $i > 1$.
若 $I \subset \{ 1, \ldots, n \}$ 不含 1,则 $P(\cap_{i \in I} B_i) = \prod_{i \in I} P(B_i)$ 显然成立.
若 $1 \in I$,令 $\mathbf{J} = I - \{ 1 \}$,有 $P(\cap_{i \in I} A_i) = \prod_{i \in I} P(A_i)$ 与 $P(\cap_{i \in J} A_i) = \prod_{i \in J} P(A_i)$,相减得 $P(A_1^c \cap \cap_{i \in J} A_i) = P(A_1^c) \prod_{i \in J} P(A_i)$.
(ii) 对 (i) 迭代,若 $B_i \in \{ A_i, A_i^c \}$,则 $B_1, \ldots, B_n$ 依然独立.
故若 $C_i \in \{ A_i, A_i^c, \Omega \}$,则 $P(\cap_{i=1}^n C_i) = \prod_{i=1}^n P(C_i)$.
若有 $C_i = \varnothing$,等式显然成立.注意到 $1_{A_i} \in \{ \varnothing, A_i, A_i^c, \Omega \}$,从而结论成立.
    \end{prf}
    
    
    
    \begin{thm}
        [A-Case-of-the-Strong-Law-of-Large-Numbers]
        {强大数定律的一个情形}
        [A Case of the Strong Law of Large Numbers]
        [gpt-4.1]
        设 $X_{1}, X_{2}, \ldots$ 是独立同分布的随机变量,$E X_{i} = \mu$ 且 $E X_{i}^{4} < \infty$.若 $S_{n} = X_{1} + \cdots + X_{n}$,则 $S_{n}/n \to \mu$ 几乎处处成立.
    \end{thm}
    
    
    
    \begin{prf}
        [Proof-of-a-Case-of-the-Strong-Law-of-Large-Numbers]
        {强大数定律的一个情形的证明}
        [Proof of a Case of the Strong Law of Large Numbers]
        [gpt-4.1]
        通过令 $X_{i}' = X_{i} - \mu$,可不失一般性地假设 $\mu = 0$.此时,

\[
E S_{n}^{4} = E \left( \sum_{i=1}^{n} X_{i} \right)^{4} = E \sum_{1 \leq i, j, k, \ell \leq n} X_{i} X_{j} X_{k} X_{\ell}
\]

和中形如 $E(X_{i}^{3} X_{j})$、$E(X_{i}^{2} X_{j} X_{k})$ 以及 $E(X_{i} X_{j} X_{k} X_{\ell})$(若 $i, j, k, \ell$ 互异)的项为 0,因为积的期望等于期望的积,在这些情况下至少有一项的期望为 0.唯一不为零的项是 $E X_{i}^{4}$ 和 $E X_{i}^{2} X_{j}^{2} = (E X_{i}^{2})^{2}$.对应项分别有 $n$ 和 $3n(n-1)$ 个(在第二种情况中,两指标有 $n(n-1)/2$ 种选法,指标固定后该项总共有 6 种排列),因此

\[
E S_{n}^{4} = n E X_{1}^{4} + 3(n^{2} - n)(E X_{1}^{2})^{2} \leq C n^{2}
\]

其中 $C < \infty$.由切比雪夫不等式可得

\[
P(|S_{n}| > n \epsilon) \leq E(S_{n}^{4}) / (n \epsilon)^{4} \leq C / (n^{2} \epsilon^{4})
\]

对 $n$ 求和并用 Borel-Cantelli 引理可知 $P(|S_{n}| > n \epsilon \ \text{i.o.}) = 0$.由于 $\epsilon$ 任意,证毕.
    \end{prf}
    
    
    
    \begin{dfn}
        [Definition-of-Total-Variation-Distance]
        {全变差距离的定义}
        [Definition of Total Variation Distance]
        [gpt-4.1]
        
设 $S$ 是一个可数集,$\mu$ 和 $
u$ 是 $S$ 上的两个测度,定义它们的全变差距离为
\[
\| \mu - 
u \| \equiv { \frac { 1 } { 2 } } \sum _ { z } | \mu ( z ) - 
u ( z ) | = \sup _ {A \subset S} | \mu ( A ) - 
u ( A ) |
\]
其中,第一个等式是定义.

    \end{dfn}
    
    
    
    \begin{thm}
        [Steins-Theorem-on-Rate-of-Convergence]
        {Stein收敛速率界定理}
        [Stein's Theorem on Rate of Convergence]
        [gpt-4.1]
        
C. Stein (1987)证明了如下不等式:

\[
\sup_{A} | \mu_n(A) - 
u_n(A) | \leq (\lambda \vee 1)^{-1} \sum_{m = 1}^n p_{n, m}^2
\]

其中 $\mu_n$ 和 $
u_n$ 为概率测度,$\lambda$ 是某参数,$p_{n, m}$ 是相关概率参数.

    \end{thm}
    
    
    
    \begin{thm}
        [Theorem-Relation-Between-Expectation-and-Integration]
        {期望值与积分的关系定理}
        [Theorem: Relation Between Expectation and Integration]
        [gpt-4.1]
        
设 $X$ 是定义在概率空间上的随机变量,$\mu$ 是其分布.对于可积函数 $f$,有
\[
E f ( X ) = \int_{S} f ( y ) \mu ( d y )
\]
具体地,证明分为如下四种情况:

1. 指示函数:若 $f = 1_B$,则
\[
E 1_{B} ( X ) = P ( X \in B ) = \mu ( B ) = \int_{S} 1_{B} ( y ) \mu ( d y )
\]

2. 单函数:$f(x) = \sum_{m=1}^{n} c_m 1_{B_m}$,则
\[
E f ( X ) = \sum_{m=1}^{n} c_m \int_{S} 1_{B_m} ( y ) \mu ( d y ) = \int_{S} f ( y ) \mu ( d y )
\]

3. 非负函数:若 $f \geq 0$,构造单调递增的简单函数列 $f_n$,利用单调收敛定理,有
\[
E f ( X ) = \lim_{n} \int_{S} f_n ( y ) \mu ( d y ) = \int_{S} f ( y ) \mu ( d y )
\]

4. 一般可积函数:将 $f$ 分解为 $f^+ - f^-$,由线性性和前述结果得
\[
E f ( X ) = \int_{S} f ( y ) \mu ( d y )
\]

    \end{thm}
    
    
    
    \begin{thm}
        [Upper-and-Lower-Bounds-for-Gaussian-Tail-Integral]
        {关于高斯尾积分的上界与下界}
        [Upper and Lower Bounds for Gaussian Tail Integral]
        [gpt-4.1]
        对于 $x > 0$,有
\[(x^{-1} - x^{-3}) \exp(-x^2 / 2) \leq \int_{x}^{\infty} \exp(-y^2 / 2) dy \leq x^{-1} \exp(-x^2 / 2)\]
    \end{thm}
    
    
    
    \begin{prf}
        [Proof-of-Bounds-for-Gaussian-Tail-Integral]
        {高斯尾积分界的证明}
        [Proof of Bounds for Gaussian Tail Integral]
        [gpt-4.1]
        改变变量 $y = x + z$ 并利用 $\exp(-z^2 / 2) \le 1$,有
\[
\int_{x}^{\infty} \exp(-y^2 / 2) dy \leq \exp(-x^2 / 2) \int_{0}^{\infty} \exp(-x z) dz = x^{-1} \exp(-x^2 / 2)
\]
对于另一方向,有
\[
\int_{x}^{\infty} (1 - 3 y^{-4}) \exp(-y^2 / 2) dy = (x^{-1} - x^{-3}) \exp(-x^2 / 2)
\]
    \end{prf}
    
    
    
    \begin{dfn}
        [Definition-of-Absolutely-Continuous-and-Singular-Distribution-Functions]
        {绝对连续分布函数与奇异分布函数的定义}
        [Definition of Absolutely Continuous and Singular Distribution Functions]
        [gpt-4.1]
        在 $\mathbf{R}$ 上的分布函数如果有密度,则称为绝对连续的;如果对应的测度是奇异的,则称为奇异的.
    \end{dfn}
    
    
    
    \begin{thm}
        [Conclusion-of-Theorem-3.2.10]
        {定理3.2.10的结论}
        [Conclusion of Theorem 3.2.10]
        [gpt-4.1]
        
若 $k(y)$ 关于测度 $
u$ 有界且连续,则由定理 3.2.10 可得
\[
\int k ( y ) 
u _ { n } ( d y ) \to \int k ( y ) 
u ( d y )
\]
由于 $\alpha_n \to \alpha$,这进一步推出
\[
{\sqrt{n}} P ( S_n \in ( x_n + a , x_n + b ) ) \to ( b - a ) n(x)
\]
这即为定理 3.2.10 的结论.

    \end{thm}
    
    
    
    \begin{xmp}
        [An-Example-of-Strict-Inequality]
        {严格不等式的一个例子}
        [An Example of Strict Inequality]
        [gpt-4.1]
        函数 $f_{n}(x) = n 1_{(0, 1/n]}(x)$ 在 $(0,1)$ 赋予 Borel 集合和 Lebesgue 测度的空间上,说明在定理 1.5.5 中可能出现严格不等式.这表明在有限测度空间上也可能发生这种情况.
    \end{xmp}
    
    
    
    \begin{prf}
        [Proof-of-Example-of-Strict-Inequality]
        {严格不等式例子的证明}
        [Proof of Example of Strict Inequality]
        [gpt-4.1]
        设 $g_{n}(x) = \inf_{m \geq n} f_{m}(x)$.由于 $f_{n}(x) \geq g_{n}(x)$,且当 $n \uparrow \infty$ 时,
\[
g_{n}(x) \uparrow g(x) = \lim_{n \to \infty} \inf_{m \geq n} f_{m}(x)
\]
又因为 $\int f_{n} \, d\mu \geq \int g_{n} \, d\mu$,只需证明
\[
\liminf_{n \to \infty} \int g_{n} \, d\mu \geq \int g \, d\mu
\]
令 $E_{m} \uparrow \Omega$ 为有限测度的集合.由于 $g_{n} \geq 0$,对于固定的 $m$,
\[
(g_{n} \wedge m) \cdot 1_{E_{m}} \to (g \wedge m) \cdot 1_{E_{m}} \quad \mathrm{a.e.}
\]
由有界收敛定理(定理 1.5.3)可得
\[
\liminf_{n \to \infty} \int_{E_m} g_{n} \wedge m \, d\mu \geq \int_{E_m} g \wedge m \, d\mu
\]
对 $m$ 取上确界,并利用定理 1.4.4 即得所需结论.
    \end{prf}
    
    
    
    \begin{thm}
        [Equivalence-between-Distribution-Distance-and-Coupling-Probability]
        {分布距离与耦合概率的等价关系}
        [Equivalence between Distribution Distance and Coupling Probability]
        [gpt-4.1]
        
$\| \mu - 
u \| \leq 2 \delta$ 当且仅当存在分布分别为 $\mu$ 和 $
u$ 的随机变量 $X$ 和 $Y$,使得 $P ( X 
eq Y ) \leq \delta$.

    \end{thm}
    
    
    
    \begin{dfn}
        [Definition-of-the-sequence-$\alpha-n$-and-its-logarithmic-properties]
        {关于数列 $\alpha\_n$ 的定义及其对数性质}
        [Definition of the sequence $\alpha_n$ and its logarithmic properties]
        [gpt-4.1]
        
设
\[
\begin{array}{c}
\alpha_n = n^{1/\log\log n} \\
\log \alpha_n = \log n / \log\log n \\
\log\log \alpha_n = \log\log n - \log\log\log n
\end{array}
\]

    \end{dfn}
    
    
    
    \begin{ppt}
        [Two-properties-of-$\alpha-n$]
        {$\alpha\_n$ 的两个性质}
        [Two properties of $\alpha_n$]
        [gpt-4.1]
        
数列 $\alpha_n$ 具有以下两个性质:

(a) $\left(\sum_{\alpha_n < p \leq n} 1/p\right) / (\log \log n)^{1/2} \to 0$ 当 $n\to\infty$.

(b) 若 $\epsilon > 0$, 则当 $n$ 充分大时有 $\alpha_n \leq n^{\epsilon}$,因此对任意 $r < \infty$,有 $\alpha_n^r / n \to 0$.

    \end{ppt}
    
    
    
    \begin{thm}
        [Poisson-Limit-Theorem-for-Sums-of-Independent-Bernoulli-Variables]
        {独立伯努利和的泊松极限定理}
        [Poisson Limit Theorem for Sums of Independent Bernoulli Variables]
        [gpt-4.1]
        设对于每个 $n$,有独立的随机变量 $X_{n,m}$,$1 \leq m \leq n$,其中 $P(X_{n,m} = 1) = p_{n,m}$,$P(X_{n,m} = 0) = 1 - p_{n,m}$.假设
(i) $\sum_{m=1}^n p_{n,m} \to \lambda$,
(ii) $\max_{1 \leq m \leq n} p_{n,m} \to 0$.
若 $S_n = X_{n,1} + \cdots + X_{n,n}$,则 $S_n \Rightarrow Z$,其中 $Z$ 服从均值为 $\lambda$ 的泊松分布,即
\[
P(Z = k) = e^{-\lambda} \lambda^k / k!
\]

    \end{thm}
    
    
    
    \begin{thm}
        [Limit-Poisson-Distribution-Theorem-for-Number-of-Empty-Boxes]
        {空盒数的极限泊松分布定理}
        [Limit Poisson Distribution Theorem for Number of Empty Boxes]
        [gpt-4.1]
        如果 $n e^{ - r / n } \to \lambda \in [ 0 , \infty )$,则空盒子的数量趋于以 $\lambda$ 为均值的泊松分布.
    \end{thm}
    
    
    
    \begin{prf}
        [Proof-of-Constructing-a-Continuous-Injection-from-$S$-to-$[0-1^{\mathbb{N}}$]
        {从 $S$ 到 $[0, 1)^{\mathbb{N}}$ 的连续单射的构造证明}
        [Proof of Constructing a Continuous Injection from $S$ to $[0, 1)^{\mathbb{N}}$]
        [gpt-4.1]
        我们首先处理特殊情况 $\boldsymbol{S} = [0, 1)^{\mathbf{N}}$,度量为
\[
\rho(x, y) = \sum_{n=1}^{\infty} |x_n - y_n| / 2^n
\]
设 $x = (x^{1}, x^{2}, x^{3}, \ldots)$,将每个分量以二进制展开 $x^{j} = x_{1}^{j} x_{2}^{j} x_{3}^{j} \ldots$(选择无限个 0 结尾的展开).定义
\[
\varphi_{o}(x) = x_{1}^{1} x_{2}^{1} x_{1}^{2} x_{3}^{1} x_{2}^{2} x_{1}^{3} x_{4}^{1} x_{3}^{2} x_{2}^{3} x_{1}^{4} \ldots
\]
对于一般情形,令
\[
d(x, y) = \rho(x, y) / (1 + \rho(x, y))
\]
(更多细节见练习 2.1.3),此时度量满足 $d(x, y) < 1$ 对所有 $x, y$ 成立.设 $q_{1}, q_{2}, \ldots$ 是 $S$ 的可数稠密集.定义
\[
\psi(x) = (d(x, q_{1}), d(x, q_{2}), \ldots)
\]
则 $\psi: S \to [0, 1)^{\mathbb{N}}$ 是连续且一一对应的映射.$\varphi_{o} \circ \psi$ 给出了所需的映射.
    \end{prf}
    
    
    
    \begin{crl}
        [Continuum-Hypothesis-for-Borel-Subsets-of-Complete-Separable-Metric-Spaces]
        {连续可分度量空间的 Borel 子集的连续统假设}
        [Continuum Hypothesis for Borel Subsets of Complete Separable Metric Spaces]
        [gpt-4.1]
        一个有趣的结果是,在完备可分度量空间的 Borel 子集中,连续统假设成立:即所有集合要么是有限的、可数无限的,要么具有实数的势(基数).
    \end{crl}
    
    
    
    \begin{thm}
        [Bound-on-Probability-Distance-Between-Binomial-and-Poisson-Distributions]
        {二项分布与泊松分布的概率距离界}
        [Bound on Probability Distance Between Binomial and Poisson Distributions]
        [gpt-4.1]
        
设 $\mu_n$ 为参数为 $n$、成功概率 $1/n$ 的二项分布,$
u_n$ 为参数为 $1$ 的泊松分布,则有
\[
\sup_{A} | \mu_n(A) - 
u_n(A) | \leq 1/n
\]
其中 $A$ 取遍所有可能的事件集合.该界衡量了两种分布的最大概率距离.

    \end{thm}
    
    
    
    \begin{prf}
        [Proof-by-Repeated-Differentiation-Under-the-Integral]
        {对积分下求导的连续使用证明}
        [Proof by Repeated Differentiation Under the Integral]
        [gpt-4.1]
        This is proved by repeatedly differentiating under the integral and using Theorem A.5.1 to justify this.
    \end{prf}
    
    
    
    \begin{thm}
        [Theorem-of-Independent-Poisson-Processes]
        {独立泊松过程定理}
        [Theorem of Independent Poisson Processes]
        [gpt-4.1]
        
设 $N_{j}(t)$ 表示在 $i \leq N(t)$ 中满足 $Y_{i} = j$ 的 $i$ 的个数,则对所有 $j$,$N_{j}(t)$ 是独立的、速率为 $\lambda P(Y_{i}=j)$ 的泊松过程.

    \end{thm}
    
    
    
    \begin{crl}
        [Corollary-of-Convergence-Transfer]
        {收敛性传递的推论}
        [Corollary of Convergence Transfer]
        [gpt-4.1]
        
如果 $X_n \Rightarrow X$ 且 $Z_n - X_n \Rightarrow 0$,则 $Z_n \Rightarrow X$.

    \end{crl}
    
    
    
    \begin{dfn}
        [Definition-of-Construction-for-Poisson-Process]
        {泊松过程的构造定义}
        [Definition of Construction for Poisson Process]
        [gpt-4.1]
        设 $\xi_{1}, \xi_{2}, \ldots$ 是独立随机变量,满足 $P(\xi_{i} > t) = e^{-\lambda t}$ 对所有 $t \geq 0$.定义 $T_{n} = \xi_{1} + \cdots + \xi_{n}$,$T_{0} = 0$,以及 $N_{t} = \operatorname*{sup}\{n : T_{n} \leq t\}$.
    \end{dfn}
    
    
    
    \begin{thm}
        [Independence-of-Male-and-Female-Counts-in-Thinned-Poisson-Process]
        {泊松过程分流后男女人数独立性定理}
        [Independence of Male and Female Counts in Thinned Poisson Process]
        [gpt-4.1]
        设有一个以速率 $\lambda$ 的泊松过程,将每次到来的顾客通过抛硬币决定性别(如男或女).则每小时到达的男性和女性顾客人数是相互独立的,并且各自分别服从一定参数的泊松过程.
    \end{thm}
    
    
    
    \begin{dfn}
        [Definition-of-Poisson-Process]
        {泊松过程的定义}
        [Definition of Poisson Process]
        [gpt-4.1]
        一族随机变量 $N_{t}, t \geq 0$,满足:

(i) 如果 $0 = t_{0} < t_{1} < \dots < t_{n}$,则 $N(t_{k}) - N(t_{k-1})$,$1 \leq k \leq n$ 彼此独立;  
(ii) $N(t) - N(s)$ 服从 Poisson$(\lambda(t - s))$ 分布,

称为强度为 $\lambda$ 的泊松过程(Poisson process).

    \end{dfn}
    
    
    
    \begin{lma}
        [Linearity-Properties-of-Simple-Function-Integrals]
        {简单函数积分的线性性质}
        [Linearity Properties of Simple Function Integrals]
        [gpt-4.1]
        
设 $\varphi$ 和 $\psi$ 为简单函数.
(i) 若 $\varphi \geq 0$ 几乎处处,则 $\int \varphi d\mu \geq 0$.
(ii) 对任意 $a \in \mathbf{R}$,有 $\int a\varphi d\mu = a \int \varphi d\mu$.
(iii) 有 $\int (\varphi + \psi) d\mu = \int \varphi d\mu + \int \psi d\mu$.

    \end{lma}
    
    
    
    \begin{prf}
        [Proof-of-Linearity-Properties-of-Simple-Function-Integrals]
        {简单函数积分线性性质的证明}
        [Proof of Linearity Properties of Simple Function Integrals]
        [gpt-4.1]
        
(i) 和 (ii) 是定义的直接结果.
(iii) 证明如下:设
\[
\varphi = \sum_{i=1}^m a_i 1_{A_i} \quad \text{and} \quad \psi = \sum_{j=1}^n b_j 1_{B_j}
\]
为了使两函数的支持相同,令 $A_0 = \bigcup_{j} B_j \setminus \bigcup_{i} A_i$,$B_0 = \bigcup_{i} A_i \setminus \bigcup_{j} B_j$,且 $a_0 = b_0 = 0$.

则
\[
\varphi + \psi = \sum_{i=0}^m \sum_{j=0}^n (a_i + b_j) 1_{A_i \cap B_j}
\]
且 $A_i \cap B_j$ 两两不交,因此
\[
\begin{array}{l}
\displaystyle \int (\varphi + \psi) d\mu = \sum_{i=0}^m \sum_{j=0}^n (a_i + b_j) \mu(A_i \cap B_j) \\
\displaystyle \qquad = \sum_{i=0}^m \sum_{j=0}^n a_i \mu(A_i \cap B_j) + \sum_{j=0}^n \sum_{i=0}^m b_j \mu(A_i \cap B_j) \\
\displaystyle \qquad = \sum_{i=0}^m a_i \mu(A_i) + \sum_{j=0}^n b_j \mu(B_j) = \int \varphi d\mu + \int \psi d\mu
\end{array}
\]

在倒数第二步中,使用了 $A_i = \bigsqcup_j (A_i \cap B_j)$ 和 $B_j = \bigsqcup_i (A_i \cap B_j)$,其中 $\bigsqcup$ 表示不交并.

    \end{prf}
    
    
    
    \begin{dfn}
        [Definition-of-Probability-Distribution-Function-for-Normalized-Sum]
        {归一化和的概率分布函数的定义}
        [Definition of Probability Distribution Function for Normalized Sum]
        [gpt-4.1]
        
设
\[
p _ { n } ( x ) = P \left( \frac{S _ { n }}{ \sqrt{ n } } = x \right)
\]
其中 $x \in {\mathcal{L}}_n = \left\{ \frac{ n b + h z }{ \sqrt{ n } } : z \in \mathbf{Z} \right\}$.

    \end{dfn}
    
    
    
    \begin{dfn}
        [Definition-of-Standard-Normal-Distribution-Function]
        {标准正态分布函数的定义}
        [Definition of Standard Normal Distribution Function]
        [gpt-4.1]
        
设
\[
n(x) = (2\pi\sigma^2)^{-1/2} \exp\left(-\frac{x^2}{2\sigma^2}\right)
\]
其中 $x \in (-\infty, \infty)$.

    \end{dfn}
    
    
    
    \begin{xmp}
        [Example-of-Unions-of-Intervals-and-Measures]
        {关于区间并的集合及测度的例子}
        [Example of Unions of Intervals and Measures]
        [gpt-4.1]
        
设 $\Omega = \mathbf{R}$ 且 $\mathcal{S} = \mathcal{S}_1$,则 $\bar{\mathcal{S}}_1 =$ 空集加上所有形如
\[
\bigcup_{i = 1}^{k} (a_i, b_i] \quad \mathrm{~其中~} -\infty \leq a_i < b_i \leq \infty
\]
的集合.给定一个定义在 $\mathcal{S}$ 上的集合函数 $\mu$,可以将其扩展到 $\bar{\mathcal{S}}$,方法为
\[
\mu\left(\bigsqcup_{i = 1}^{n} A_i\right) = \sum_{i = 1}^{n} \mu(A_i)
\]

    \end{xmp}
    
    
    
    \begin{dfn}
        [Definition-of-Measure-on-an-Algebra]
        {代数上的测度的定义}
        [Definition of Measure on an Algebra]
        [gpt-4.1]
        
在代数 $\mathcal{A}$ 上的测度是指一个集合函数 $\mu$,满足:
(i) 对所有 $A \in \mathcal{A}$,有 $\mu(A) \geq \mu(\varnothing) = 0$;
(ii) 若 $A_i \in \mathcal{A}$ 两两不交且其并属于 $\mathcal{A}$,则
\[
\mu \left( \cup _ { i = 1 } ^ { \infty } A _ { i } \right) = \sum _ { i = 1 } ^ { \infty } \mu ( A _ { i } )
\]

    \end{dfn}
    
    
    
    \begin{dfn}
        [Definition-of-σ-finite-Measure]
        {σ-有限测度的定义}
        [Definition of σ-finite Measure]
        [gpt-4.1]
        
若存在一列集合 $A_n \in \mathcal{A}$ 使得 $\mu(A_n) < \infty$ 且 $\cup_n A_n = \Omega$,则称 $\mu$ 是 $\sigma$-有限的.

    \end{dfn}
    
    
    
    \begin{dfn}
        [Definition-of-Little-o-Notation]
        {小阶符号的定义}
        [Definition of Little-o Notation]
        [gpt-4.1]
        
$o(t^n)$ 表示一个量 $g(t)$,其满足 $g(t) / t^n \to 0$ 当 $t \to 0$ 时.

    \end{dfn}
    
    
    
    \begin{lma}
        [Lemma-on-the-Existence-of-the-Limit]
        {极限存在性引理}
        [Lemma on the Existence of the Limit]
        [gpt-4.1]
        由引理 2.7.1 可得 $\operatorname*{lim}_{n \to \infty} \frac{1}{n} \log P ( S_{n} \geq n a ) = \gamma ( a ) \leq 0$ 存在.
    \end{lma}
    
    
    
    \begin{prf}
        [Proof-of-the-Convergence-of-$S-n$]
        {关于变量$S\_n'$收敛性的证明}
        [Proof of the Convergence of $S_n'$]
        [gpt-4.1]
        证明 令 $X_{n,m}' = 1$ 当且仅当 $X_{n,m} = 1$, 否则为 $0$.令 $S_n' = X_{n,1}' + \cdots + X_{n,n}'$.由 (i)-(ii) 和定理 3.6.1 可得 $S_n' \Rightarrow Z$, (iii) 告诉我们 $P(S_n 
eq S_n') \to 0$,因此结论由'共同收敛引理'以及习题 3.2.13 得出.$\Box$.
    \end{prf}
    
    
    
    \begin{thm}
        [Theorem-on-Sufficient-Condition-for-Convergence-in-Distribution]
        {收敛于分布的充分条件定理}
        [Theorem on Sufficient Condition for Convergence in Distribution]
        [gpt-4.1]
        
如果 $X_n$ 的每个子列都有一个进一步的子列收敛于 $X$,则 $X_n \Rightarrow X$.

    \end{thm}
    
    
    
    \begin{xmp}
        [Example-of-the-Cauchy-Distribution]
        {柯西分布的举例}
        [Example of the Cauchy Distribution]
        [gpt-4.1]
        
柯西分布(Cauchy distribution)密度函数为 $1/\pi (1 + x^2)$,特征函数为 $\exp(-|t|)$.

    \end{xmp}
    
    
    
    \begin{dfn}
        [Definition-of-the-Sign-Function]
        {符号函数的定义}
        [Definition of the Sign Function]
        [gpt-4.1]
        定义函数 $\operatorname{sgn} x$,其中 $\operatorname{sgn} x = 1$ 当 $x > 0$,$\operatorname{sgn} x = -1$ 当 $x < 0$,$\operatorname{sgn} x = 0$ 当 $x = 0$.
    \end{dfn}
    
    
    
    \begin{thm}
        [Infinite-Product-Representation-for-Sine-Function]
        {三角函数无穷乘积表示}
        [Infinite Product Representation for Sine Function]
        [gpt-4.1]
        利用恒等式 $\sin t = 2 \sin(t/2) \cos(t/2)$ 反复展开,可得
\[(\sin t)/t = \prod_{m=1}^\infty \cos(t/2^m).\]
    \end{thm}
    
    
    
    \begin{thm}
        [Theorem-on-Integration-to-the-Limit]
        {积分极限定理}
        [Theorem on Integration to the Limit]
        [gpt-4.1]
        
设 $X_{n} \to X$ 几乎处处收敛.令 $g, h$ 为连续函数,满足以下条件:
(i) $g \geq 0$ 且 $g(x) \to \infty$ 当 $|x| \to \infty$;
(ii) $|h(x)| / g(x) \to 0$ 当 $|x| \to \infty$;
(iii) 对所有 $n$ 有 $E g(X_{n}) \leq K < \infty$.
则 $E h(X_{n}) \to E h(X)$.

    \end{thm}
    
    
    
    \begin{dfn}
        [Definition-of-$T-{k}^{\prime}$]
        {关于$T\_{k}^{\prime}$的定义}
        [Definition of $T_{k}^{\prime}$]
        [gpt-4.1]
        设 $T_{1}^{\prime} = T_{N(t)+1} - t$,且对 $k \geq 2$,有 $T_{k}^{\prime} = T_{N(t)+k} - T_{N(t)+k-1}$.
    \end{dfn}
    
    
    
    \begin{prf}
        [Proof-of-the-Distribution-of-Random-Variable-$Y-n$]
        {关于随机变量 $Y\_n$ 分布的证明}
        [Proof of the Distribution of Random Variable $Y_n$]
        [gpt-4.1]
        设 $\Omega = (0,1)$, ${\mathcal{F}} =$ Borel 集合, $P =$ Lebesgue 测度, 并令 $Y_n(x) = \sup \{ y : F_n(y) < x \}$.
    \end{prf}
    
    
    
    \begin{thm}
        [Theorem-on-Normalized-Sums-of-Random-Variables-and-Their-Convergence]
        {关于随机变量归一化和收敛的定理}
        [Theorem on Normalized Sums of Random Variables and Their Convergence]
        [gpt-4.1]
        
设 $X_{1}, X_{2}, \dots$ 是随机变量,满足 $E X_{i} = 0$ 且 $E X_{i}^{2} = \sigma^{2} < \infty$,令 $S_{n} = X_{1} + \cdots + X_{n}$.若 $\epsilon > 0$,则有
\[
S_{n} / \left[ n^{1/2} (\log n)^{1/2 + \epsilon} \right] \to 0 \quad \mathrm{a.s.}
\]

    \end{thm}
    
    
    
    \begin{thm}
        [Law-of-the-Iterated-Logarithm-Theorem]
        {迭代对数律定理}
        [Law of the Iterated Logarithm Theorem]
        [gpt-4.1]
        
设 $X_{1}, X_{2}, \dots$ 是随机变量,满足 $E X_{i} = 0$ 且 $E X_{i}^{2} = \sigma^{2} < \infty$,令 $S_{n} = X_{1} + \cdots + X_{n}$.则有
\[
\limsup_{n \to \infty} S_{n} / \left[ n^{1/2} (\log \log n)^{1/2} \right] = \sigma \sqrt{2} \quad \mathrm{a.s.}
\]

    \end{thm}
    
    
    
    \begin{thm}
        [Properties-of-the-Minimum-of-Exponential-Random-Variables]
        {最小指数分布变量及其性质}
        [Properties of the Minimum of Exponential Random Variables]
        [gpt-4.1]
        
设 $T_i = \mathrm{exponential}(\lambda_i)$,$V = \operatorname*{min}(T_1, \dots, T_n)$,$I$ 为取最小值的 $T_i$ 的随机索引,则有:

\[
\begin{array}{l}
P(V > t) = \exp(-(\lambda_1 + \cdots + \lambda_n)t) \\
P(I = i) = \displaystyle \frac{\lambda_i}{\lambda_1 + \cdots + \lambda_n}
\end{array}
\]

且 $I$ 与 $V = \operatorname*{min}\{T_1, \dots, T_n\}$ 相互独立.

    \end{thm}
    
    
    
    \begin{thm}
        [Conditions-for-Weak-Convergence-of-Product-of-Random-Variables]
        {随机变量乘积弱收敛的条件}
        [Conditions for Weak Convergence of Product of Random Variables]
        [gpt-4.1]
        
设 $X_{n} \Rightarrow X$, $Y_{n} \geq 0$, 且 $Y_{n} \Rightarrow c$,其中 $c > 0$ 是常数,则 $X_{n} Y_{n} \Rightarrow cX$.

    \end{thm}
    
    
    
    \begin{dfn}
        [Definition-of-Polyas-Distribution]
        {Polya分布的定义}
        [Definition of Polya's Distribution]
        [gpt-4.1]
        
Polya分布的密度函数为
\[
f(x) = \frac{1 - \cos x}{\pi x^2}
\]
其特征函数为
\[
\varphi(t) = (1 - |t|)^+
\]

    \end{dfn}
    
    
    
    \begin{thm}
        [Probabilistic-Limit-Form-of-Central-Limit-Theorem]
        {中心极限定理的概率极限形式}
        [Probabilistic Limit Form of Central Limit Theorem]
        [gpt-4.1]
        
定理 3.1.2 若 $2k / {\sqrt{2n}} \to x$,则 $P(S_{2n} = 2k) \sim (\pi n)^{-1/2} e^{-x^2/2}$.

    \end{thm}
    
    
    
    \begin{dfn}
        [Definition-of-the-Process-$St$]
        {过程 $S(t)$ 的定义}
        [Definition of the Process $S(t)$]
        [gpt-4.1]
        设 $Y_i$ 是某一序列的随机变量,$N(t)$ 表示到时刻 $t$ 为止的某一计数过程,则定义
\[
S(t) = Y_1 + \cdots + Y_{N(t)}
\]
其中若 $N(t) = 0$,则 $S(t) = 0$.
    \end{dfn}
    
    
    
    \begin{lma}
        [Lemma-on-the-Bound-between-Convolution-and-Product-Norm-Differences]
        {卷积与乘积范数的差的界的引理}
        [Lemma on the Bound between Convolution and Product Norm Differences]
        [gpt-4.1]
        
若 $\mu_1 * \mu_2$ 表示 $\mu_1$ 和 $\mu_2$ 的卷积,即
\[
\mu_1 * \mu_2 ( x ) = \sum_{y} \mu_1 ( x - y ) \mu_2 ( y )
\]
则有
\[
\| \mu_1 * \mu_2 - 
u_1 * 
u_2 \| \leq \| \mu_1 \times \mu_2 - 
u_1 \times 
u_2 \|
\]
并且
\[
2 \| \mu_1 * \mu_2 - 
u_1 * 
u_2 \| = \sum_{x} \left| \sum_{y} \mu_1( x - y ) \mu_2( y ) - \sum_{y} 
u_1( x - y ) 
u_2( y ) \right| 
\leq \sum_{x} \sum_{y} | \mu_1( x - y ) \mu_2( y ) - 
u_1( x - y ) 
u_2( y ) | 
= 2 \| \mu_1 \times \mu_2 - 
u_1 \times 
u_2 \|
\]
从而得到所需结论.

    \end{lma}
    
    
    
    \begin{xmp}
        [Example-of-Calculating-the-Distribution-Function]
        {分布函数的具体计算示例}
        [Example of Calculating the Distribution Function]
        [gpt-4.1]
        
F_\theta(\{x\}) / F(\{x\}) = e^{\theta x} / \varphi(\theta)

因此,

F_\theta(\{1\}) = \frac{e^{\theta}}{e^{\theta} + e^{-\theta}}, \quad
F_\theta(\{-1\}) = \frac{e^{-\theta}}{e^{\theta} + e^{-\theta}}

    \end{xmp}
    
    
    
    \begin{thm}
        [Expression-for-the-Solution-of-Renewal-Equation]
        {更新方程解的表达式}
        [Expression for the Solution of Renewal Equation]
        [gpt-4.1]
        如果更新方程有形式
\[
R = R(0) + R * F,
\]
则由定理 2.6.9 可知
\[
R(w) = R(0) U(w),
\]
其中 $U(w)$ 是与 $F$ 相关的函数.
    \end{thm}
    
    
    
    \begin{xmp}
        [Poisson-Distribution-of-Call-Numbers-in-M/G/∞-Queue]
        {M/G/∞ 排队系统中通话数量的泊松分布}
        [Poisson Distribution of Call Numbers in M/G/∞ Queue]
        [gpt-4.1]
        设电话呼叫的到达过程服从泊松过程,通话时长服从一般分布 $G$,其均值为 $\mu$,则在 M/G/∞ 排队系统中,系统内通话的数量在长期极限下服从泊松分布,其均值为:
\[
\lambda \int_{r=0}^{\infty} (1-G(r)) dr = \lambda \mu
\]
    \end{xmp}
    
    
    
    \begin{thm}
        [Zero-Convergence-Scaling-Theorem-for-Arbitrary-Sequence-of-Random-Variables]
        {任意随机变量序列的归零缩放定理}
        [Zero Convergence Scaling Theorem for Arbitrary Sequence of Random Variables]
        [gpt-4.1]
        设 $X_n$ 是任意的随机变量序列,则存在一组常数 $c_n \to \infty$,使得 $X_n / c_n \to 0$.
    \end{thm}
    
    
    
    \begin{thm}
        [Lower-Bound-Theorem-for-Probability-with-Moment-Conditions]
        {关于矩的概率下界定理}
        [Lower Bound Theorem for Probability with Moment Conditions]
        [gpt-4.1]
        对于每个 $K < \infty$ 和 $y < 1$,存在常数 $c_{y,K} > 0$,使得若 $E X^{2} = 1$ 且 $E X^{4} \leq K$,则 $P( | X | > y ) \geq c_{y,K}$.
    \end{thm}
    
    
    
    \begin{thm}
        [Walds-Equation]
        {Wald 方程}
        [Wald's Equation]
        [gpt-4.1]
        设 $X_{1}, X_{2}, \ldots$ 为独立同分布的随机变量,且 $E|X_{i}| < \infty$.若 $N$ 是一个停时,且 $E N < \infty$,则有
\[
E S_{N} = E X_{1} \cdot E N
\]
其中 $S_{N} = X_{1} + \cdots + X_{N}$.
    \end{thm}
    
    
    
    \begin{xmp}
        [Characteristic-Function-Expectation-of-Coin-Flip-Random-Variable]
        {抛硬币随机变量的期望特征函数}
        [Characteristic Function Expectation of Coin Flip Random Variable]
        [gpt-4.1]
        
若 $P(X = 1) = P(X = -1) = 1/2$,则
\[
E e^{i t X} = (e^{i t} + e^{-i t}) / 2 = \cos t
\]

    \end{xmp}
    
    
    
    \begin{thm}
        [Limit-Theorem-for-Counting-Process]
        {计数过程的极限定理}
        [Limit Theorem for Counting Process]
        [gpt-4.1]
        
设 $\mu = E \xi_{i} \in (0, \infty]$,其中 $1 / \infty = 0$.则当 $t \to \infty$ 时,$N_{t} / t \to 1 / \mu$.

    \end{thm}
    
    
    
    \begin{dfn}
        [Definition-of-Probability-Space]
        {概率空间的定义}
        [Definition of Probability Space]
        [gpt-4.1]
        概率空间是一个三元组 $(\Omega, \mathcal{F}, P)$,其中 $\Omega$ 是'结果'集合,$\mathcal{F}$ 是'事件'集合,$P: \mathcal{F} \to [0, 1]$ 是赋予事件概率的函数.
    \end{dfn}
    
    
    
    \begin{dfn}
        [Definition-of-σ-field-σ-algebra]
        {σ-域(σ-代数)的定义}
        [Definition of σ-field (σ-algebra)]
        [gpt-4.1]
        $\mathcal{F}$ 是 $\Omega$ 的(非空)子集的集合,称为σ-域(或σ-代数),如果满足:
(i) 若 $A \in \mathcal{F}$,则 $A^c \in \mathcal{F}$;
(ii) 若 $A_i \in \mathcal{F}$ 是可数序列,则 $\cup_i A_i \in \mathcal{F}$.
    \end{dfn}
    
    
    
    \begin{dfn}
        [Definition-of-Measurable-Space]
        {可测空间的定义}
        [Definition of Measurable Space]
        [gpt-4.1]
        没有概率测度 $P$ 时,$(\Omega, \mathcal{F})$ 称为可测空间,即一个可以定义测度的空间.
    \end{dfn}
    
    
    
    \begin{dfn}
        [Definition-of-Measure]
        {测度的定义}
        [Definition of Measure]
        [gpt-4.1]
        测度是一个非负的可数可加集合函数,即 $\mu: \mathcal{F} \to \mathbf{R}$.
    \end{dfn}
    
    
    
    \begin{thm}
        [Hewitt-Savage-0-1-Law]
        {Hewitt-Savage 0-1律}
        [Hewitt-Savage 0-1 Law]
        [gpt-4.1]
        设 $X_{1}, X_{2}, \dots$ 是独立同分布的随机变量,$A \in \mathcal{E}$,则 $P(A) \in \{ 0, 1 \}$.
    \end{thm}
    
    
    
    \begin{thm}
        [Independence-Theorem-for-Poissonization-and-Occupancy-Problem]
        {泊松化与占据问题的独立性定理}
        [Independence Theorem for Poissonization and Occupancy Problem]
        [gpt-4.1]
        如果我们将服从均值为 $r$ 的泊松分布的若干球放入 $n$ 个盒子,并令 $N_{i}$ 表示第 $i$ 个盒子中的球数,则由上一个练习可知 $N_{1}, \ldots, N_{n}$ 相互独立,并且均服从均值为 $r/n$ 的泊松分布.
    \end{thm}
    
    
    
    \begin{lma}
        [Fatous-Lemma]
        {Fatou 引理}
        [Fatou's Lemma]
        [gpt-4.1]
        
设 $g \ge 0$ 是连续函数.如果 $X_n \Rightarrow X_\infty$,则
\[
\liminf_{n \to \infty} E g(X_n) \geq E g(X_\infty)
\]

    \end{lma}
    
    
    
    \begin{dfn}
        [Definition-of-Integral-for-Bounded-Functions-on-Finite-Measure-Sets]
        {有限测度集上有界函数的积分定义}
        [Definition of Integral for Bounded Functions on Finite Measure Sets]
        [gpt-4.1]
        设 $E$ 是测度有限的集合(即 $\mu(E) < \infty$),$f$ 是一个有界函数且在 $E^c$ 上为零.令 $\varphi, \psi$ 为在 $E^c$ 上为零的简单函数,满足 $\varphi \leq f \leq \psi$,则定义
\[
\int f d\mu = \sup_{\varphi \leq f} \int \varphi d\mu = \inf_{\psi \geq f} \int \psi d\mu
\]

    \end{dfn}
    
    
    
    \begin{prf}
        [Proof-of-Well-Definedness-of-the-Integral-for-Bounded-Functions]
        {有界函数积分定义的合理性证明}
        [Proof of Well-Definedness of the Integral for Bounded Functions]
        [gpt-4.1]
        需证明上述定义中的上确界(sup)与下确界(inf)相等.根据引理 1.4.2 (iv) 有
\[
\sup_{\varphi \leq f} \int \varphi d\mu \leq \inf_{\psi \geq f} \int \psi d\mu
\]
对于另一方向的不等式,设 $|f| \leq M$,令
\[
\begin{array}{l}
\displaystyle E_k = \left\{ x \in E : \frac{kM}{n} \geq f(x) > \frac{(k-1)M}{n} \right\} \quad \text{对于}~ -n \leq k \leq n \\
\displaystyle \psi_n(x) = \sum_{k=-n}^n \frac{kM}{n} 1_{E_k} \qquad \varphi_n(x) = \sum_{k=-n}^n \frac{(k-1)M}{n} 1_{E_k}
\end{array}
\]
则有 $\psi_n(x) - \varphi_n(x) = (M/n)1_E$,因此
\[
\int \psi_n(x) - \varphi_n(x) d\mu = \frac{M}{n} \mu(E)
\]
且 $\varphi_n(x) \leq f(x) \leq \psi_n(x)$,由引理 1.4.1 (iii) 得
\[
\begin{array}{c}
\operatorname*{sup}_{\varphi \leq f} \int \varphi d\mu \geq \int \varphi_n d\mu = -\frac{M}{n} \mu(E) + \int \psi_n d\mu \\
\geq -\frac{M}{n} \mu(E) + \operatorname*{inf}_{\psi \geq f} \int \psi d\mu
\end{array}
\]
上式对任意 $n$ 均成立,令 $n \to \infty$ 即得两者相等,证毕.

    \end{prf}
    
    
    
    \begin{thm}
        [Definition-and-Properties-of-the-Poisson-Process]
        {泊松过程的定义及性质}
        [Definition and Properties of the Poisson Process]
        [gpt-4.1]
        设 $N(s, t)$ 表示在时间区间 $(s, t]$ 内到达银行或冰淇淋店的到达次数.若满足以下条件:

(i) 不相交区间内的到达次数相互独立;  
(ii) $N(s, t)$ 的分布仅依赖于 $t-s$;  
(iii) $P(N(0, h) = 1) = \lambda h + o(h)$;  
(iv) $P(N(0, h) \geq 2) = o(h)$;

其中,$o(h)$ 指的是函数 $g_{1}(h)$ 和 $g_{2}(h)$,且 $g_{i}(h)/h \to 0$ 当 $h \to 0$ 时.则 $N(s, t)$ 满足泊松过程的性质,且在实际应用中泊松分布频繁出现的原因由此得到解释.

    \end{thm}
    
    
    
    \begin{thm}
        [Glivenko-Cantelli-Theorem]
        {Glivenko-Cantelli 定理}
        [Glivenko-Cantelli Theorem]
        [gpt-4.1]
        对于几乎所有的 $\omega$,有
\[
F_{n}(y) = n^{-1} \sum_{m=1}^{n} 1_{(X_{m}(\omega) \leq y)} \to F(y) \quad \mathrm{for\ all\ } y
\]

    \end{thm}
    
    
    
    \begin{xmp}
        [Characteristic-Function-of-Poisson-Distribution]
        {泊松分布的特征函数}
        [Characteristic Function of Poisson Distribution]
        [gpt-4.1]
        
设随机变量 $X$ 满足 $P(X = k) = e^{-\lambda} \lambda^{k} / k!$,其中 $k = 0, 1, 2, \ldots$,则其特征函数为
\[
E e^{i t X} = \sum_{k=0}^{\infty} e^{-\lambda} \frac{\lambda^{k} e^{i t k}}{k!} = \exp(\lambda (e^{i t} - 1))
\]

    \end{xmp}
    
    
    
    \begin{thm}
        [Strong-Law-of-Large-Numbers-Holds-When-Expectation-Exists]
        {强大数定律在期望存在时成立}
        [Strong Law of Large Numbers Holds When Expectation Exists]
        [gpt-4.1]
        
设 $X_{1}, X_{2}, \ldots$ 为独立同分布随机变量,满足 $E X_{i}^{+} = \infty$ 且 $E X_{i}^{-} < \infty$.令 $S_{n} = X_{1} + \cdots + X_{n}$,则有 $S_{n} / n \to \infty$ 几乎必然成立 (a.s.).

    \end{thm}
    
    
    
    \begin{lma}
        [Characteristic-Function-of-Convex-Combination]
        {凸组合的特征函数}
        [Characteristic Function of Convex Combination]
        [gpt-4.1]
        
若 $F_1, \ldots, F_n$ 的特征函数分别为 $\varphi_1, \ldots, \varphi_n$,且 $\lambda_i \geq 0$ 且 $\lambda_1 + \cdots + \lambda_n = 1$,则 $\sum_{i=1}^n \lambda_i F_i$ 的特征函数为 $\sum_{i=1}^n \lambda_i \varphi_i$.

    \end{lma}
    
    
    
    \begin{prf}
        [Calculation-of-Joint-Density-for-Poisson-Process-Counts]
        {泊松过程计数的联合密度计算}
        [Calculation of Joint Density for Poisson Process Counts]
        [gpt-4.1]
        
在事件 $\{N(t) = n\}$ 下, 联合密度为
\[
f_T(t_1, \dots, t_n) = \left(\prod_{m=1}^n \lambda e^{-\lambda (t_m - t_{m-1})}\right) e^{-\lambda (t - t_n)} = \lambda^n e^{-\lambda t}
\]
若除以 $P(N(t) = n) = e^{-\lambda t} (\lambda t)^n / n!$, 则结果为 $n! / t^n$, 这即为 $
u$ 的联合分布.

    \end{prf}
    
    
    
    \begin{thm}
        [Approximation-Theorem-for-Distribution-Functions]
        {分布函数逼近定理}
        [Approximation Theorem for Distribution Functions]
        [gpt-4.1]
        如果 $F$ 是任意分布函数,则存在一列分布函数 $F_n$,其形式为 $\sum_{m=1}^n a_{n,m} 1_{(x_{n,m} \leq x)}$,使得 $F_n \Rightarrow F$.

    \end{thm}
    
    
    
    \begin{thm}
        [Expectation-and-Variance-Formulas-for-Sums-of-IID-Variables]
        {独立同分布求和的期望与方差公式}
        [Expectation and Variance Formulas for Sums of IID Variables]
        [gpt-4.1]
        设 $Y_1, Y_2, \ldots$ 是独立同分布的随机变量,$N$ 是与 $\{Y_i\}$ 独立的非负整数值随机变量,定义 $S = Y_1 + \cdots + Y_N$,且当 $N = 0$ 时 $S = 0$.

(i) 若 $E|Y_i| < \infty$, $EN < \infty$,则 $ES = EN \cdot E Y_i$.

(ii) 若 $E Y_i^2 < \infty$, $EN^2 < \infty$,则
\[
\operatorname{var}(S) = EN \cdot \operatorname{var}(Y_i) + \operatorname{var}(N) (E Y_i)^2.
\]

(iii) 若 $N$ 服从 Poisson$(\lambda)$ 分布,则 $\operatorname{var}(S) = \lambda E Y_i^2$.

    \end{thm}
    
    
    
    \begin{thm}
        [Property-of-Minimum-of-Independent-Exponential-Distributions]
        {最小值的指数分布性质}
        [Property of Minimum of Independent Exponential Distributions]
        [gpt-4.1]
        如果 $S =$ exponential$(\lambda)$ 和 $T =$ exponential$(\mu)$ 是独立的,则 $V = \operatorname*{min}\{S, T\}$ 服从 exponential$(\lambda + \mu)$,且 $P(U = S) = \frac{\lambda}{\lambda + \mu}$.
    \end{thm}
    
    
    
    \begin{xmp}
        [An-Example-of-Triangular-Distribution]
        {三角分布的一个例子}
        [An Example of Triangular Distribution]
        [gpt-4.1]
        
三角分布的密度函数为 $1 - |x|$,定义域为 $x \in (-1, 1)$.其特征函数为 $2 (1 - \cos t) / t^{2}$.证明:如果 $X$ 和 $Y$ 是在 $(-1/2, 1/2)$ 上独立均匀分布的随机变量,则 $X + Y$ 服从三角分布.

    \end{xmp}
    
    
    
    \begin{dfn}
        [Symbolic-Definitions-of-Probability-Distributions]
        {概率分布的符号定义}
        [Symbolic Definitions of Probability Distributions]
        [gpt-4.1]
        
设 $\mu_{n, m}$ 表示随机变量 $X_{n, m}$ 的分布;
$\mu_n$ 表示随机变量 $S_n$ 的分布;
$
u_{n, m}$、$
u_n$ 和 $
u$ 均为 Poisson 分布,其均值分别为 $p_{n, m}$、$\lambda_n = \sum_{m \leq n} p_{n, m}$ 和 $\lambda$.

    \end{dfn}
    
    
    
    \begin{thm}
        [Theorem-on-Distribution-Convergence-and-Total-Variation-Distance-Bound]
        {分布收敛与全变差距离界的定理}
        [Theorem on Distribution Convergence and Total Variation Distance Bound]
        [gpt-4.1]
        
若 $\mu_n = \mu_{n, 1} * \cdots * \mu_{n, n}$,$
u_n = 
u_{n, 1} * \cdots * 
u_{n, n}$,且对所有 $m$ 有 $p_{n, m}$,则有
\[
\| \mu_n - 
u_n \| \leq \sum_{m = 1}^n \| \mu_{n, m} - 
u_{n, m} \| \leq 2 \sum_{m = 1}^n p_{n, m}^2
\]
且由全变差距离定义,
\[
\sup_{A} | \mu_n(A) - 
u_n(A) | \leq \sum_{m = 1}^n p_{n, m}^2
\]
若假设 (i) 和 (ii) 成立,则右边趋于 $0$.由于 $
u_n \Rightarrow 
u$ 随 $n \to \infty$,因此结论成立.

    \end{thm}
    
    
    
    \begin{ppt}
        [Closure-of-Support-Under-Addition]
        {支持集对加法的封闭性}
        [Closure of Support Under Addition]
        [gpt-4.1]
        若 $x$ 属于 $F^{m*}$ 的支持集,$y$ 属于 $F^{n*}$ 的支持集,则 $x + y$ 属于 $F^{(m+n)*}$ 的支持集.
    \end{ppt}
    
    
    
    \begin{thm}
        [Construction-Method-under-Finite-Measure]
        {有限测度下的构造方法}
        [Construction Method under Finite Measure]
        [gpt-4.1]
        
若 $\mu(S) < \infty$,则可根据定理 3.7.4 构造如下:令 $X_1, X_2, \dots$ 为 $S$ 上独立同分布的元素,其分布为 $
u(\cdot) = \mu(\cdot)/\mu(S)$,令 $N$ 为均值为 $\mu(S)$ 的独立 Poisson 随机变量,定义 $m(A) = |\{j \leq N : X_j \in A\}|$.

    \end{thm}
    
    
    
    \begin{thm}
        [Relationship-between-Smoothness-of-Characteristic-Function-and-Decay-of-Measure]
        {特征函数光滑性与测度衰减性的关系}
        [Relationship between Smoothness of Characteristic Function and Decay of Measure]
        [gpt-4.1]
        
在定理 3.17 的证明中, 我们得到了如下不等式:
\[
\mu\{ x : |x| > 2/u \} \leq u^{-1} \int_{-u}^u (1 - \varphi(t)) \, dt
\]
这表明, 特征函数在 $0$ 处的光滑性与测度在无穷远处的衰减性相关.

    \end{thm}
    
    
    
    \begin{dfn}
        [Definition-of-Defective-Probability-Distribution]
        {缺陷概率分布的定义}
        [Definition of Defective Probability Distribution]
        [gpt-4.1]
        
当 $\mu < c$ 时,
\[
F(x) = \int_{0}^{x} \frac{1-G(y)}{c} dy
\]
是一个缺陷概率分布,其中 $F(\infty) = \mu / c$.

    \end{dfn}
    
    
    
    \begin{crl}
        [Corollary-Same-Distribution-for-$X$-and-$-X$-under-Real-Function]
        {实值函数下变量及其相反数的分布相同的推论}
        [Corollary: Same Distribution for $X$ and $-X$ under Real Function]
        [gpt-4.1]
        如果 $\varphi$ 是实值的,则 $X$ 和 $-X$ 具有相同的分布.
    \end{crl}
    
    
    
    \begin{thm}
        [Limit-Distribution-of-Random-Variable-$Y-n$]
        {随机变量$Y\_n$的极限分布}
        [Limit Distribution of Random Variable $Y_n$]
        [gpt-4.1]
        
由定理 3.1.2 和引理 3.1.1 可知,
\[
f_{Y_n}(y) \to (2\pi)^{-1/2} \exp(-y^2/2) \quad \mathrm{as} \ n \to \infty
\]
即当 $n \to \infty$ 时,$Y_n$ 的密度函数收敛到标准正态分布的密度函数.

    \end{thm}
    
    
    
    \begin{crl}
        [Weak-Convergence-of-$Y-n$-to-Standard-Normal-Distribution]
        {随机变量$Y\_n$弱收敛于标准正态分布}
        [Weak Convergence of $Y_n$ to Standard Normal Distribution]
        [gpt-4.1]
        
利用 Scheffé 定理,可得 $Y_n$ 弱收敛于标准正态分布.

    \end{crl}
    
    
    
    \begin{thm}
        [Formula-for-the-Measure-of-a-Singleton-Set]
        {测度在单点集上的表示公式}
        [Formula for the Measure of a Singleton Set]
        [gpt-4.1]
        
设 $\varphi(t)$ 是某种与测度 $\mu$ 相关的函数,则
\[
\mu(\{a\}) = \lim_{T \to \infty} \frac{1}{2T} \int_{-T}^{T} e^{-ita} \varphi(t) dt
\]

    \end{thm}
    
    
    
    \begin{lma}
        [Generalization-of-Convergence-of-Finite-Products-to-Exponential-Function]
        {有限乘积收敛于指数函数的推广}
        [Generalization of Convergence of Finite Products to Exponential Function]
        [gpt-4.1]
        如果 $\max_{1 \leq j \leq n} |c_{j,n}| \to 0$,$\sum_{j=1}^n c_{j,n} \to \lambda$,且 $\sup_n \sum_{j=1}^n |c_{j,n}| < \infty$,那么
\[
\prod_{j=1}^n (1 + c_{j,n}) \to e^{\lambda} .
\]

    \end{lma}
    
    
    
    \begin{thm}
        [Limsup-of-Outcomes-in-the-St.-Petersburg-Game]
        {圣彼得堡游戏结果的极限上确界}
        [Limsup of Outcomes in the St. Petersburg Game]
        [gpt-4.1]
        设 $X_n$ 是圣彼得堡游戏第 $n$ 次玩的结果(参见例 2.2.16),则有
\[
\limsup_{n \to \infty} \frac{X_n}{n \log_2 n} = \infty \quad \text{a.s.}
\]
因此,同样的结果对 $S_n$ 也成立.这说明在概率意义下证明的 $S_n/(n \log_2 n) \to 1$ 的收敛在 a.s. 意义下并不成立.

    \end{thm}
    
    
    
    \begin{dfn}
        [Definition-of-Future-σ-Field-and-Tail-σ-Field]
        {未来σ域与尾σ域的定义}
        [Definition of Future σ-Field and Tail σ-Field]
        [gpt-4.1]
        设 $\mathcal{F}_n' = \sigma(X_n, X_{n+1}, \ldots)$,即时间 $n$ 之后的未来,是使所有 $X_m, m \geq n$ 可测的最小 $\sigma$ 域.设 $\mathcal{T} = \cap_n \mathcal{F}_n'$,称为遥远的未来或尾σ域.直观上,$A \in \mathcal{T}$ 当且仅当改变有限个值不会影响事件的发生.
    \end{dfn}
    
    
    
    \begin{thm}
        [Asymptotic-Estimate-for-Large-Deviation-Probability]
        {大偏差概率的渐近估计}
        [Asymptotic Estimate for Large Deviation Probability]
        [gpt-4.1]
        
假设 $x_o = \infty$,$\theta_+ < \infty$,且 $\varphi'(\theta)/\varphi(\theta)$ 当 $\theta \uparrow \theta_+$ 时递增收敛到有限极限 $a_0$.如果 $a_0 \le a < \infty$,则有
\[
n^{-1} \log P(S_n \geq n a) \to -a \theta_+ + \log \varphi(\theta_+)
\]

    \end{thm}
    
    
    
    \begin{thm}
        [Convergence-in-Distribution-and-Convergence-of-Random-Variables]
        {分布收敛与随机变量收敛}
        [Convergence in Distribution and Convergence of Random Variables]
        [gpt-4.1]
        如果 $F_n \Rightarrow F_\infty$,则存在随机变量 $Y_n$,$1 \leq n \leq \infty$,其分布为 $F_n$,使得 $Y_n \to Y_\infty$ 几乎处处收敛.
    \end{thm}
    
    
    
    \begin{thm}
        [Converse-of-Theorem-2.5.12]
        {定理 2.5.12 的逆命题}
        [Converse of Theorem 2.5.12]
        [gpt-4.1]
        若 $p > 0$,如果 $S_n / n^{1/p} \to 0$,则 $E|X_1|^p < \infty$.
    \end{thm}
    
    
    
    \begin{prf}
        [Physical-Proof-of-the-Integral-Formula-for-Normal-Density-with-Complex-Mean]
        {复数均值的正态密度积分公式的物理证明}
        [Physical Proof of the Integral Formula for Normal Density with Complex Mean]
        [gpt-4.1]
        
\[
\int e^{i t x} (2\pi)^{-1/2} e^{-x^{2}/2} dx = e^{-t^{2}/2} \int (2\pi)^{-1/2} e^{-(x - i t)^{2}/2} dx
\]

The integral is 1, since the integrand is the normal density with mean $i t$ and variance 1.

    \end{prf}
    
    
    
    \begin{thm}
        [Differentiability-of-Characteristic-Function-and-Existence-of-Moments]
        {特征函数的可微性与矩的存在性}
        [Differentiability of Characteristic Function and Existence of Moments]
        [gpt-4.1]
        
如果 $E|X|^n < \infty$,则其特征函数在 $0$ 处可微 $n$ 次,且有 $\varphi^{(n)}(0) = E (i X)^n$.

    \end{thm}
    
    
    
    \begin{thm}
        [A-Large-Deviation-Principle-Result-for-the-Poisson-Distribution]
        {大偏差原理关于泊松分布的一个结果}
        [A Large Deviation Principle Result for the Poisson Distribution]
        [gpt-4.1]
        
设 $P(X_i = k) = e^{-1} / k!$,其中 $k = 0, 1, \ldots$,若 $a > 1$,则有
\[
\frac{1}{n} \log P(S_n \geq n a) \to a - 1 - a \log a
\]

    \end{thm}
    
    
    
    \begin{thm}
        [Inequality-for-$\betaX-Y$]
        {关于 $\beta(X, Y)$ 的不等式}
        [Inequality for $\beta(X, Y)$]
        [gpt-4.1]
        
若 $\alpha(X, Y) = a$,则
\[
\frac{a^2}{1 + a} \leq \beta(X, Y) \leq a + \frac{(1 - a)a}{1 + a}
\]

    \end{thm}
    
    
    
    \begin{thm}
        [Conclusion-of-Central-Limit-Theorem]
        {中心极限定理的结论}
        [Conclusion of Central Limit Theorem]
        [gpt-4.1]
        
设 $S_n$ 为一列随机变量之和,$F_n(y)$ 表示其标准化后小于等于 $y$ 的概率,则有
\[
F_{n}(y) = P\left(\frac{S_{n}}{\sqrt{n}} \leq y\right) \to \int_{-\infty}^{y} (2\pi)^{-1/2} e^{-x^{2}/2} dx
\]
当 $n \to \infty$ 时,$F_n(y)$ 收敛到标准正态分布的分布函数.

    \end{thm}
    
    
    
    \begin{lma}
        [Lemma-on-Error-Estimate-for-Taylor-Expansion-of-Characteristic-Function]
        {关于特征函数的泰勒展开误差估计引理}
        [Lemma on Error Estimate for Taylor Expansion of Characteristic Function]
        [gpt-4.1]
        
对随机变量 $X$,对所有 $t \in \mathbb{R}$,有如下不等式成立:

\[
\left| E e^{itX} - \sum_{m=0}^n E \frac{(itX)^m}{m!} \right| \leq E \left| e^{itX} - \sum_{m=0}^n \frac{(itX)^m}{m!} \right| \leq E \min \left( |tX|^{n+1}, 2|tX|^n \right)
\]

其中,第一步取期望,第二步使用 Jensen 不等式,并应用了引理 3.2 于 $x = tX$,在第二步中为了简化界限去掉了分母.

    \end{lma}
    
    
    
    \begin{thm}
        [Theorem-on-Convergence-to-Zero-of-IID-Centered-Sequence-with-Finite-p-th-Moment]
        {独立同分布中心化$p$阶矩可积序列归零定理}
        [Theorem on Convergence to Zero of IID Centered Sequence with Finite p-th Moment]
        [gpt-4.1]
        
设 $X_{1}, X_{2}, \dots$ 是独立同分布的随机变量,满足 $E X_{1} = 0$ 且 $E |X_{1}|^{p} < \infty$,其中 $1 < p < 2$.令 $S_{n} = X_{1} + \cdots + X_{n}$,则有 $S_{n} / n^{1/p} \to 0$ 几乎必然成立.

    \end{thm}
    
    
    
    \begin{thm}
        [Relation-between-Characteristic-Function-Integral-and-Point-Probability-for-Independent-Random-Variables]
        {独立随机变量特征函数积分与点概率的关系}
        [Relation between Characteristic Function Integral and Point Probability for Independent Random Variables]
        [gpt-4.1]
        
设 $X$ 和 $Y$ 是独立的随机变量,分布为 $\mu$,特征函数为 $\varphi$.则有
\[
\lim _ { T \to \infty } {\frac { 1 } { 2 T }} \int _ { - T } ^ { T } | \varphi ( t ) | ^ { 2 } d t = P ( X - Y = 0 ) = \sum _ { x } \mu ( \{ x \} ) ^ { 2 }
\]

    \end{thm}
    
    
    
    \begin{thm}
        [Holders-Inequality]
        {Holder不等式}
        [Holder's Inequality]
        [gpt-4.1]
        
设 $p, q \in (1, \infty)$ 且 $1/p + 1/q = 1$,则
\[
\int |f g| d\mu \leq \|f\|_{p} \|g\|_{q}
\]

    \end{thm}
    
    
    
    \begin{prf}
        [Proof-of-Holders-Inequality]
        {Holder不等式的证明}
        [Proof of Holder's Inequality]
        [gpt-4.1]
        
若 $\| f \|_{p}$ 或 $\| g \|_{q} = 0$,则 $|fg| = 0$,不等式成立.因此只需证明当 $\| f \|_{p}$ 和 $\| g \|_{q} > 0$ 时的情形,或通过同时除以 $\| f \|_{p} \| g \|_{q}$,即当 $\| f \|_{p} = \| g \|_{q} = 1$ 时的情形.
取 $y \ge 0$,令
\[
\begin{array}{l}
\varphi(x) = x^{p}/p + y^{q}/q - x y \quad \text{对于 } x \geq 0 \\
\varphi'(x) = x^{p-1} - y \quad \text{且} \quad \varphi''(x) = (p-1)x^{p-2}
\end{array}
\]
因此 $\varphi$ 在 $x_{o} = y^{1/(p-1)}$ 处取得极小值.又有 $q = p/(p-1)$ 且 $x_{o}^{p} = y^{p/(p-1)} = y^{q}$,故
\[
\varphi(x_{o}) = y^{q} (1/p + 1/q) - y^{1/(p-1)} y = 0
\]
由于 $x_{o}$ 是极小值点,因此 $x y \leq x^{p}/p + y^{q}/q$.取 $x = |f|, y = |g|$,并对两边积分
\[
\int |f g| d\mu \leq \frac{1}{p} + \frac{1}{q} = 1 = \|f\|_{p} \|g\|_{q}
\]

    \end{prf}
    
    
    
    \begin{ppt}
        [Special-Case-of-Holders-Inequality-Cauchy-Schwarz-Inequality]
        {Holder不等式的特殊情形:Cauchy-Schwarz不等式}
        [Special Case of Holder's Inequality: Cauchy-Schwarz Inequality]
        [gpt-4.1]
        
当 $p = q = 2$ 时,称为Cauchy-Schwarz不等式.对于任意 $\theta$,
\[
0 \leq \int (f + \theta g)^{2} d\mu = \int f^{2} d\mu + \theta \left( 2 \int f g d\mu \right) + \theta^{2} \left( \int g^{2} d\mu \right)
\]
因此右边的二次式 $a \theta^{2} + b \theta + c$ 至多有一个实根.回忆二次方程根公式
\[
\frac{ -b \pm \sqrt{b^{2} - 4 a c} }{2a}
\]
有 $b^{2} - 4ac \leq 0$,即为所需结果.

    \end{ppt}
    
    
    
    \begin{thm}
        [Theorem-on-Convergence-under-Continuous-Functions]
        {连续函数作用下的收敛性定理}
        [Theorem on Convergence under Continuous Functions]
        [gpt-4.1]
        若 $Y_n \overset{d}{=} X_n$ 且 $Y_n \to Y_\infty$ 几乎处处收敛.若 $f$ 连续,则 $D_{f \circ g} \subset D_g$,所以 $P(Y_\infty \in D_{f \circ g}) = 0$,从而 $f(g(Y_n)) \to f(g(Y_\infty))$ 几乎处处收敛.若 $f$ 还有界,则根据有界收敛定理,有 $E f(g(Y_n)) \to E f(g(Y_\infty))$.由于上述对所有有界连续函数成立,根据定理 3.2.9 可知 $g(X_n) \Rightarrow g(X_\infty)$.
    \end{thm}
    
    
    
    \begin{thm}
        [Continuous-Mapping-Theorem]
        {连续映射定理}
        [Continuous Mapping Theorem]
        [gpt-4.1]
        设 $g$ 是一个可测函数,$D_{g} = \{x : g$ 在 $x$ 处不连续$\}$.
若 $X_{n} \Rightarrow X_{\infty}$ 且 $P(X_{\infty} \in D_{g}) = 0$,则 $g(X_{n}) \Rightarrow g(X_{\infty})$.
如果进一步 $g$ 有界,则 $E g(X_{n}) \to E g(X_{\infty})$.
    \end{thm}
    
    
    
    \begin{xmp}
        [Poisson-Approximation-for-Double-Ones-in-Two-Dice-Rolls]
        {掷两骰子出现双一的泊松分布近似}
        [Poisson Approximation for Double Ones in Two Dice Rolls]
        [gpt-4.1]
        
假设我们掷两枚骰子共 36 次.'双一'(每枚骰子均为 1)的概率是 $1/36$,因此该事件出现的次数应该近似服从以 1 为均值的泊松分布.将泊松近似与精确概率进行比较可以发现,即使实验次数很少,两者的吻合度依然较好.

    \end{xmp}
    
    
    
    \begin{thm}
        [Theorem-on-Lower-Bound-for-Large-Deviation-Probability]
        {关于大偏差概率下界的定理}
        [Theorem on Lower Bound for Large Deviation Probability]
        [gpt-4.1]
        
设 $E X_{i} = 0$.若 $\epsilon > 0$,则有
\[
\lim_{n \to \infty} \inf \frac{P(S_{n} \geq n a)}{n P(X_{1} \geq n(a + \epsilon))} \geq 1
\]

    \end{thm}
    
    
    
    \begin{thm}
        [The-Inversion-Formula]
        {反演公式}
        [The Inversion Formula]
        [gpt-4.1]
        
设 $\varphi(t) = \int e^{i t x} \mu(dx)$,其中 $\mu$ 是一个概率测度.如果 $a < b$,则有
\[
\lim_{T \to \infty} (2\pi)^{-1} \int_{-T}^{T} \frac{e^{-i t a} - e^{-i t b}}{i t} \varphi(t) dt = \mu(a, b) + \frac{1}{2} \mu(\{a, b\})
\]
其中极限的存在性也是结论的一部分.

    \end{thm}
    
    
    
    \begin{xmp}
        [Example-of-the-Limiting-Distribution-of-Residual-Waiting-Time]
        {剩余等待时间分布的极限分布例子}
        [Example of the Limiting Distribution of Residual Waiting Time]
        [gpt-4.1]
        设 $h(t) := 1 - F(t + x)$,$h$ 是递减的,且当 $\mu = E(\xi_{i}) < \infty$ 时,$h$ 的积分有限.应用引理 2.6.13 和定理 2.6.12 有:

\[
P(T_{N(t)} - t > x) \to \frac{1}{\mu} \int_{0}^{\infty} h(s) ds = \frac{1}{\mu} \int_{x}^{\infty} 1 - F(t)\, dt
\]

因此(当 $\mu < \infty$ 时),剩余等待时间 $T_{N(t)} - t$ 的分布收敛到能够产生平稳更新过程的延迟分布.
    \end{xmp}
    
    
    
    \begin{xmp}
        [A-Typical-Application-Example-of-the-Renewal-Equation]
        {更新方程的典型应用例子}
        [A Typical Application Example of the Renewal Equation]
        [gpt-4.1]
        设 $x > 0$ 为定值,令 $H ( t ) = P ( T _ { N ( t ) } - t > x )$.通过考虑 $T _ { 1 }$ 的取值,得到

\[
H ( t ) = ( 1 - F ( t + x ) ) + \int _ { 0 } ^ { t } H ( t - s ) d F ( s )
\]

    \end{xmp}
    
    
    
    \begin{thm}
        [Blackwells-Renewal-Theorem]
        {Blackwell 更新定理}
        [Blackwell's Renewal Theorem]
        [gpt-4.1]
        如果 $F$ 是非算术型的,则
\[
U([t, t+h]) \to h/\mu \quad \text{当 } t \to \infty.
\]

    \end{thm}
    
    
    
    \begin{dfn}
        [Definition-of-Large-Deviations-Rate-Function]
        {大偏差速率函数的定义}
        [Definition of Large Deviations Rate Function]
        [gpt-4.1]
        
$\gamma(a) = \displaystyle \lim_{n \to \infty} (1/n) \log P(S_n \geq n a) = -a \theta_a + \kappa(\theta_a)$

    \end{dfn}
    
    
    
    \begin{dfn}
        [Cumulant-Generating-Function-and-Its-Derivative-for-Normal-Distribution]
        {正态分布的累积生成函数及其导数}
        [Cumulant Generating Function and Its Derivative for Normal Distribution]
        [gpt-4.1]
        
正态分布下,累积生成函数及其导数为:
\[
\kappa(\theta) = \theta^2 / 2 \qquad \kappa'(\theta) = \theta \qquad \theta_a = a
\]

    \end{dfn}
    
    
    
    \begin{thm}
        [Limit-Distribution-Theorem]
        {极限分布定理}
        [Limit Distribution Theorem]
        [gpt-4.1]
        当 $n \to \infty$ 时,有
\[
P\left(n^2 \operatorname*{min}_{1 \leq m \leq n} (V_m^n - V_{m-1}^n) > x\right) \to e^{-x}
\]
成立.
    \end{thm}
    
    
    
    \begin{thm}
        [Theorem-on-the-Limit-Relationship-Between-$Ut$-and-$t$]
        {关于 $U(t)$ 与 $t$ 的极限关系定理}
        [Theorem on the Limit Relationship Between $U(t)$ and $t$]
        [gpt-4.1]
        当 $t \to \infty$ 时,有 $U(t) / t \to 1 / \mu$.
    \end{thm}
    
    
    
    \begin{thm}
        [De-Moivre-Laplace-Theorem]
        {De Moivre-Laplace 定理}
        [De Moivre-Laplace Theorem]
        [gpt-4.1]
        
设 $a < b$, 当 $m \to \infty$ 时,
\[
P\left(a \leq \frac{S_m}{\sqrt{m}} \leq b\right) \to \int_{a}^{b} (2\pi)^{-1/2} e^{-x^2/2} dx
\]
(要去除对偶整数的限制,可注意 $S_{2n+1} = S_{2n} \pm 1$.)

    \end{thm}
    
    
    
    \begin{xmp}
        [Example-of-Pairwise-Independence-Not-Satisfying-the-Central-Limit-Theorem]
        {成对独立但不服从中心极限定理的例子}
        [Example of Pairwise Independence Not Satisfying the Central Limit Theorem]
        [gpt-4.1]
        
设 $\xi_1, \xi_2, \ldots$ 是独立同分布的随机变量,$P(\xi_i = 1) = P(\xi_i = -1) = 1/2$.对 $n \geq 1$,令
\[
S_{2^n} = \xi_1 (1 + \xi_2) \cdots (1 + \xi_{n+1}) = 
\begin{cases}
\pm 2^n & \text{with prob } 2^{-n-1} \\
0 & \text{with prob } 1 - 2^{-n}
\end{cases}
\]
为实现此构造,令 $X_1 = \xi_1$,$X_2 = \xi_1 \xi_2$,对于 $m = 2^{n-1} + j, 0 < j \leq 2^{n-1}, n \geq 2$,令 $X_m = X_j \xi_{n+1}$.每个 $X_m$ 都是关于一组不同的 $\xi_j$ 的积,因此它们两两独立,但该构造不满足中心极限定理.

    \end{xmp}
    
    
    
    \begin{thm}
        [Construction-of-Independent-Random-Variables-and-Non-convergence-to-Zero]
        {独立随机变量构造与归零收敛性的否定}
        [Construction of Independent Random Variables and Non-convergence to Zero]
        [gpt-4.1]
        设 $\sum \sigma_n^2 / n^2 = \infty$ 且不妨假设对所有 $n$ 都有 $\sigma_n^2 \leq n^2$,则存在一列独立随机变量 $X_n$,满足 $E X_n = 0$ 且 $\operatorname{var}(X_n) \leq \sigma_n^2$,使得 $X_n / n$ 以及 $n^{-1} \sum_{m \leq n} X_m$ 不收敛于 0.
    \end{thm}
    
    
    
    \begin{lma}
        [Converging-Together-Lemma]
        {共同收敛引理}
        [Converging Together Lemma]
        [gpt-4.1]
        如果 $X_n \Rightarrow X$ 且 $Y_n \Rightarrow c$,其中 $c$ 是常数,则 $X_n + Y_n \Rightarrow X + c$.
    \end{lma}
    
    
    
    \begin{thm}
        [Limit-Theorem-Behavior-of-Maxima-with-Infinite-Expectation]
        {极限定理:无穷期望下的极大值行为}
        [Limit Theorem: Behavior of Maxima with Infinite Expectation]
        [gpt-4.1]
        
设 $X_{1}, X_{2}, \ldots$ 是独立同分布(i.i.d.)随机变量,且 $E|X_{1}| = \infty$,令 $S_{n} = X_{1} + \cdots + X_{n}$.设 $a_{n}$ 是一列正数,且 $a_{n}/n$ 单调递增.则
\[
\operatorname*{limsup}_{n \to \infty} \frac{|S_{n}|}{a_{n}} = 0 \text{ 或 } \infty
\]
当且仅当 $\sum_{n} P(|X_{1}| \geq a_{n}) < \infty$ 或 $= \infty$.

    \end{thm}
    
    
    
    \begin{thm}
        [Equivalent-Characterizations-of-Weak-Convergence]
        {弱收敛的等价刻画}
        [Equivalent Characterizations of Weak Convergence]
        [gpt-4.1]
        
以下陈述是等价的:
(i) $X_{n} \Rightarrow X_{\infty}$;
(ii) 对任意开集 $G$,有 $\liminf_{n \to \infty} P(X_{n} \in G) \geq P(X_{\infty} \in G)$;
(iii) 对任意闭集 $K$,有 $\limsup_{n \to \infty} P(X_{n} \in K) \leq P(X_{\infty} \in K)$;
(iv) 对于所有满足 $P(X_{\infty} \in \partial A) = 0$ 的 Borel 集 $A$,有 $\lim_{n \to \infty} P(X_{n} \in A) = P(X_{\infty} \in A)$.

    \end{thm}
    
    
    
    \begin{thm}
        [Relationship-Between-Convergence-in-Probability-and-Weak-Convergence]
        {概率收敛与分布收敛的关系}
        [Relationship Between Convergence in Probability and Weak Convergence]
        [gpt-4.1]
        
若 $X_n \to X$ 依概率收敛,则 $X_n \Rightarrow X$,即 $X_n$ 依分布收敛于 $X$.反过来,若 $X_n \Rightarrow c$,其中 $c$ 是常数,则 $X_n \to c$ 依概率收敛.

    \end{thm}
    
    
    
    \begin{dfn}
        [Definition-of-Functions-$h-n\epsilon$-and-$g\epsilon$]
        {函数 $h\_n(\epsilon)$ 和 $g(\epsilon)$ 的定义}
        [Definition of Functions $h_n(\epsilon)$ and $g(\epsilon)$]
        [gpt-4.1]
        令 $h_{n}(\epsilon) = E \exp(i t \hat{S}_{n}(\epsilon) / n^{1/\alpha})$,以及 $g(\epsilon) = \exp(-\epsilon^{-\alpha}\{1 - \varphi(\epsilon t)\})$.
    \end{dfn}
    
    
    
    \begin{xmp}
        [Example-of-Probability-Distribution-for-Coin-Flips]
        {关于抛硬币的概率分布例子}
        [Example of Probability Distribution for Coin Flips]
        [gpt-4.1]
        考虑随机变量 $X_i$,满足 $P(X_i = 1) = P(X_i = -1) = \frac{1}{2}$.则其矩母函数为
\[
\varphi(\theta) = \frac{e^\theta + e^{-\theta}}{2}
\]
且其导数与函数之比为
\[
\frac{\varphi'(\theta)}{\varphi(\theta)} = \frac{e^\theta - e^{-\theta}}{e^\theta + e^{-\theta}}
\]
.
    \end{xmp}
    
    
    
    \begin{thm}
        [Limit-Theorem-for-Integration]
        {积分极限的收敛定理}
        [Limit Theorem for Integration]
        [gpt-4.1]
        设 $g, h$ 是连续函数,且 $g(x) > 0$,并且当 $|x| \to \infty$ 时有 $|h(x)|/g(x) \to 0$.若 $F_n \Rightarrow F$ 且 $\int g(x) dF_n(x) \leq C < \infty$,则
\[
\int h(x) dF_n(x) \to \int h(x) dF(x)
\]

    \end{thm}
    
    
    
    \begin{thm}
        [Distribution-Theorem-for-Partitioned-Poisson-Process]
        {泊松过程分割的分布定理}
        [Distribution Theorem for Partitioned Poisson Process]
        [gpt-4.1]
        设 $X_1, \dots, X_m$ 分别计数在 $m$ 个类别中的事件数,假设每个事件被分到第 $j$ 类的概率为 $p_j$,并且总事件数服从参数为 $\lambda t$ 的泊松分布,则

\[
P(X_{1}=k_{1}, \dots, X_{m}=k_{m})
= e^{-\lambda t} \frac{(\lambda t)^{k_{1} + \cdots + k_{m}}}{(k_{1} + \cdots + k_{m})!} \frac{(k_{1} + \cdots + k_{m})!}{k_{1}! \cdots k_{m}!} p_{1}^{k_{1}} \cdots p_{m}^{k_{m}}
= \prod_{j=1}^{m} e^{-\lambda p_{j} t} \frac{(\lambda p_{j} t)^{k_{j}}}{k_{j}!}
\]

即 $X_j \sim \operatorname{Poisson}(\lambda p_j t)$,且 $X_1, \dots, X_m$ 彼此独立.

    \end{thm}
    
    
    
    \begin{thm}
        [Kolmogorovs-Maximal-Inequality]
        {Kolmogorov极大不等式}
        [Kolmogorov's Maximal Inequality]
        [gpt-4.1]
        
设 $X_{1}, \ldots, X_{n}$ 相互独立,且 $E X_{i} = 0$,$\operatorname{var}(X_{i}) < \infty$.令 $S_{n} = X_{1} + \cdots + X_{n}$,则有
\[
P\left(\max_{1 \leq k \leq n} |S_{k}| \geq x\right) \leq x^{-2} \operatorname{var}\left(S_{n}\right)
\]

    \end{thm}
    
    
    
    \begin{xmp}
        [Example-of-Weak-Convergence-IID-Random-Variables-and-Partial-Sums]
        {弱收敛的例子:独立同分布随机变量和部分和}
        [Example of Weak Convergence: IID Random Variables and Partial Sums]
        [gpt-4.1]
        
设 $X_{1}, X_{2}, \dots$ 是独立同分布的随机变量,满足 $P(X_{i} = 1) = P(X_{i} = -1) = 1/2$,令 $S_{n} = X_{1} + \cdots + X_{n}$.

    \end{xmp}
    
    
    
    \begin{ppt}
        [Mean-and-Variance-Formulas-for-Parametric-Distribution-Function]
        {参数化分布函数的均值与方差公式}
        [Mean and Variance Formulas for Parametric Distribution Function]
        [gpt-4.1]
        
若 $\theta \in ( \theta_{ - }, \theta_{ + } )$, 则 $F_{ \theta }$ 是一个分布函数,其均值为
\[
\int x dF_{ \theta }( x ) = \frac{ 1 }{ \varphi( \theta ) } \int_{ - \infty }^{ \infty } x e^{ \theta x } dF( x ) = \frac{ \varphi^{ \prime }( \theta ) }{ \varphi( \theta ) }
\]
且有
\[
\frac{ d }{ d \theta } \frac{ \varphi^{ \prime }( \theta ) }{ \varphi( \theta ) } = \frac{ \varphi^{ \prime \prime }( \theta ) }{ \varphi( \theta ) } - \left( \frac{ \varphi^{ \prime }( \theta ) }{ \varphi( \theta ) } \right)^{ 2 } = \int x^{ 2 } dF_{ \theta }( x ) - \left( \int x dF_{ \theta }( x ) \right)^{ 2 } \geq 0
\]
其中最后一个表达式是 $F_{ \theta }$ 的方差,必为非负.

    \end{ppt}
    
    
    
    \begin{thm}
        [Continuity-Theorem]
        {连续性定理}
        [Continuity Theorem]
        [gpt-4.1]
        
设 $\mu_n$,$1 \leq n \leq \infty$ 为概率测度,其特征函数为 $\varphi_n$.
(i) 若 $\mu_n \Rightarrow \mu_\infty$,则对于所有 $t$,有 $\varphi_n(t) \to \varphi_\infty(t)$.
(ii) 若 $\varphi_n(t)$逐点收敛于极限 $\varphi(t)$,且 $\varphi(t)$ 在 $\boldsymbol{\theta}$ 处连续,则相应的分布序列 $\mu_n$ 紧且弱收敛于特征函数为 $\varphi$ 的测度 $\mu$.

    \end{thm}
    
    
    
    \begin{thm}
        [A-Theorem-on-Large-Deviation-Probability-Convergence]
        {大偏差概率收敛的一个定理}
        [A Theorem on Large Deviation Probability Convergence]
        [gpt-4.1]
        
假设 $E X_{i} = 0$ 且对于所有 $\theta > 0$,$E \exp(\theta X_{i}) = \infty$.则对于所有 $a > 0$,
\[
\frac{1}{n} \log P(S_{n} \geq n a) \to 0
\]

    \end{thm}
    
    
    
    \begin{lma}
        [Kochen-Stone-Lemma]
        {Kochen-Stone引理}
        [Kochen-Stone Lemma]
        [gpt-4.1]
        
设 $\sum P(A_k) = \infty$.如果
\[
\lim_{n \to \infty} \sup_{\mathbf{k} \to \infty} \frac{\left( \sum_{k=1}^{n} P(A_k) \right)^2}{\sum_{1 \leq j, k \leq n} P(A_j \cap A_k)} = \alpha > 0
\]
则 $P(A_n\ \text{i.o.}) \geq \alpha$.

    \end{lma}
    
    
    
    \begin{thm}
        [Limit-Theorem-for-Stable-Distributions]
        {稳定分布的极限定理}
        [Limit Theorem for Stable Distributions]
        [gpt-4.1]
        
若 $S_n$ 是一组随机变量的和,$\alpha$ 是稳定分布参数,且
\[
E \exp ( i t S_n / n^{1/\alpha} ) \to \exp ( - C | t |^{\alpha} )
\]
那么根据定理 3.3.17(ii), 右侧表达式是某随机变量 $Y$ 的特征函数,并且
\[
S_n / n^{1/\alpha} \Rightarrow Y
\]
即 $S_n / n^{1/\alpha}$ 收敛到稳定分布的随机变量 $Y$.

    \end{thm}
    
    
    
    \begin{thm}
        [Theorem-on-Characteristic-Function-of-Limit-of-Sums-of-Independent-Random-Variables]
        {独立随机变量和收敛极限的特征函数定理}
        [Theorem on Characteristic Function of Limit of Sums of Independent Random Variables]
        [gpt-4.1]
        
设 $X_1, X_2, \dots$ 是相互独立的随机变量,$S_n = X_1 + \cdots + X_n$,设 $X_j$ 的特征函数为 $\varphi_j$.若 $S_n \to S_\infty$ 几乎处处收敛,则 $S_\infty$ 的特征函数为
\[
\prod_{j=1}^\infty \varphi_j(t).
\]

    \end{thm}
    
    
    
    \begin{dfn}
        [Definition-of-Exponential-Distribution]
        {指数分布的定义}
        [Definition of Exponential Distribution]
        [gpt-4.1]
        
指数分布的概率密度函数为 $e^{-x}$,定义域为 $x \in (0, \infty)$.其特征函数为 $\frac{1}{1 - i t}$.

    \end{dfn}
    
    
    
    \begin{thm}
        [Properties-of-the-Real-Part-and-Modulus-Squared-of-Character-Maps]
        {特征映射的实部及模方的性质}
        [Properties of the Real Part and Modulus Squared of Character Maps]
        [gpt-4.1]
        设 $\varphi$ 是一个特征映射 (character map),则 $\operatorname{Re} \varphi$ 和 $|\varphi|^2$ 也是特征映射.
    \end{thm}
    
    
    
    \begin{lma}
        [Lemma-on-the-Radon-Nikodym-Derivative]
        {Radon-Nikodym导数的引理}
        [Lemma on the Radon-Nikodym Derivative]
        [gpt-4.1]
        
设 $dF / dF_{ \lambda }$ 表示相关测度的Radon-Nikodym导数,则根据定义有 $dF / dF_{ \lambda } = e^{ - \lambda x } \varphi( \lambda )$.

    \end{lma}
    
    
    
    \begin{dfn}
        [Definition-of-Lipschitz-Continuity]
        {Lipschitz连续性的定义}
        [Definition of Lipschitz Continuity]
        [gpt-4.1]
        如果存在常数 $C$,使得对于所有 $x, y$ 都有 $|f(x) - f(y)| \leq C \rho(x, y)$,则称 $f$ 是Lipschitz连续的.
    \end{dfn}
    
    
    
    \begin{dfn}
        [Definition-of-Age-and-Residual-Lifetime-in-Renewal-Process]
        {更新过程中的年龄和剩余寿命定义}
        [Definition of Age and Residual Lifetime in Renewal Process]
        [gpt-4.1]
        
令 $A_{t} = t - T_{N(t) - 1}$ 表示时刻 $t$ 的'年龄',即自上一次更新以来经过的时间.剩余寿命定义为 $B_{t} = T_{N(t)} - t$,即到下一次更新的剩余时间.

    \end{dfn}
    
    
    
    \begin{thm}
        [Limit-Distribution-Formula-of-Age-in-Renewal-Process]
        {年龄分布的极限分布公式}
        [Limit Distribution Formula of Age in Renewal Process]
        [gpt-4.1]
        
若 $x>0$,则年龄分布 $H(t) = P(A_{t} > x)$ 满足更新方程
\[
H(t) = (1 - F(t))\cdot 1_{(x, \infty)}(t) + \int_{0}^{t} H(t - s) dF(s)
\]
并有极限
\[
P(A_{t} > x) \to \frac{1}{\mu} \int_{x}^{\infty} (1 - F(t))\, dt
\]
其中右侧为剩余寿命 $B_{t}$ 的极限分布.

    \end{thm}
    
    
    
    \begin{thm}
        [Kolmogorovs-Three-Series-Theorem]
        {Kolmogorov三列定理}
        [Kolmogorov's Three-Series Theorem]
        [gpt-4.1]
        
设 $X_{1}, X_{2}, \ldots$ 是独立随机变量,$A > 0$,定义 $Y_{i} = X_{i} 1_{(|X_{i}| \leq A)}$.级数 $\sum_{n=1}^{\infty} X_{n}$ 几乎处处收敛的充要条件是:

\[
\text{(i) } \sum_{n=1}^{\infty} P(|X_{n}| > A) < \infty, \quad \text{(ii) } \sum_{n=1}^{\infty} E Y_{n} \text{ 收敛}, \quad \text{以及 (iii) } \sum_{n=1}^{\infty} \operatorname{var}(Y_{n}) < \infty
\]

    \end{thm}
    
    
    
    \begin{thm}
        [Criterion-for-Tightness-of-Probability-Distribution-Sequences]
        {概率分布列紧性的判据}
        [Criterion for Tightness of Probability Distribution Sequences]
        [gpt-4.1]
        
若存在一个 $\varphi \geq 0$,使得 $\varphi(x) \to \infty$ 当 $|x| \to \infty$,并且
\[
C = \sup_{n} \int \varphi(x)\, dF_{n}(x) < \infty
\]
则 $F_{n}$ 是紧的 (tight).

此外,有不等式:
\[
1 - F_{n}(M) + F_{n}(-M) \leq \frac{C}{\inf_{|x| \geq M} \varphi(x)}.
\]

    \end{thm}
    
    
    
    \begin{thm}
        [Renewal-Process-of-Customer-Entry-Times]
        {客户进入服务时刻的更新过程}
        [Renewal Process of Customer Entry Times]
        [gpt-4.1]
        设客户按照强度为1的泊松过程到达,如果服务器被占用,他们离开;如果空闲,则进入服务,服务时间服从分布$F$,均值为$\mu$.则客户进入服务的时刻构成一个均值为$\mu+1$的更新过程.
    \end{thm}
    
    
    
    \begin{thm}
        [Uniqueness-of-Bounded-Solution-to-the-Renewal-Equation]
        {更新方程有界解的唯一性}
        [Uniqueness of Bounded Solution to the Renewal Equation]
        [gpt-4.1]
        
如果 $h$ 是有界函数,则函数
\[
H ( t ) = \int _ { 0 } ^ { t } h ( t - s ) \, d U ( s )
\]
是在有界区间上有界的更新方程的唯一解.

    \end{thm}
    
    
    
    \begin{prf}
        [Proof-of-the-Product-Property-of-Characteristic-Functions-for-Independent-Random-Variables]
        {独立随机变量特征函数的乘积性质证明}
        [Proof of the Product Property of Characteristic Functions for Independent Random Variables]
        [gpt-4.1]
        
\[
E e^{i t (X_{1} + X_{2})} = E (e^{i t X_{1}} e^{i t X_{2}}) = E e^{i t X_{1}} \, E e^{i t X_{2}}
\]

since $e^{i t X_{1}}$ and $e^{i t X_{2}}$ are independent.

    \end{prf}
    
    
    
    \begin{thm}
        [Limit-Theorem-for-Poisson-Distribution-and-Stirlings-Formula]
        {泊松分布和斯特林公式极限定理}
        [Limit Theorem for Poisson Distribution and Stirling's Formula]
        [gpt-4.1]
        
设 $X_1, X_2, \dots$ 是独立同分布的随机变量,且每个 $X_i$ 服从参数为 1 的泊松分布,则 $S_n = X_1 + \cdots + X_n$ 服从参数为 $n$ 的泊松分布,概率质量函数为 $P(S_n = k) = e^{-n} n^{k} / k!$.若 $(k - n)/\sqrt{n} \to x$,则有
\[
\sqrt{2\pi n} P(S_n = k) \to \exp(-x^2/2)
\]

    \end{thm}
    
    
    
    \begin{xmp}
        [Example-of-the-Integral-Properties-of-Function-$ht$]
        {关于函数 $h(t)$ 的积分性质的例子}
        [Example of the Integral Properties of Function $h(t)$]
        [gpt-4.1]
        
$h(t) = \frac{1}{\mu} \int_{[t,\infty)} 1 - F(s)\, ds$.

$h$ 是递减的,$h(0) = 1$,且

\[
\begin{array}{rcl}
\mu \int_{0}^{\infty} h(t)\, dt &=& \int_{0}^{\infty} \int_{t}^{\infty} 1 - F(s)\, ds\, dt \\
&=& \int_{0}^{\infty} \int_{0}^{s} 1 - F(s)\, dt\, ds \\
&=& \int_{0}^{\infty} s (1 - F(s))\, ds = E(\xi_i^2/2)
\end{array}
\]

因此,若 $
u \equiv E(\xi_i^2) < \infty$,则由引理 2 可得结论.

    \end{xmp}
    
    
    
    \begin{thm}
        [Convergence-Theorem-for-Weighted-Sums-of-Independent-Zero-Mean-Variables]
        {独立零均值变量加权和的收敛性定理}
        [Convergence Theorem for Weighted Sums of Independent Zero-Mean Variables]
        [gpt-4.1]
        
设 $X_1, X_2, \dots$ 是独立随机变量,且 $E X_n = 0$, $\operatorname{var}(X_n) = \sigma_n^2$.若 $\sum_n \sigma_n^2 / n^2 < \infty$,则加权和 $\sum_n X_n / n$ 几乎必然收敛,即 a.s. 收敛.因此,$n^{-1} \sum_{m \leq n} X_m \to 0$ 几乎必然成立.

    \end{thm}
    
    
    
    \begin{ppt}
        [Function-$Ht$-Satisfies-the-Renewal-Equation]
        {函数 $H(t)$ 满足更新方程}
        [Function $H(t)$ Satisfies the Renewal Equation]
        [gpt-4.1]
        $H ( t ) = U ( t ) - \frac{t}{\mu}$ 满足如下的更新方程,其右端为 $h ( t ) = \frac { 1 } { \mu } \int _ { t } ^ { \infty } ( 1 - F ( s ) ) d s$.
    \end{ppt}
    
    
    
    \begin{dfn}
        [Definition-of-Distribution-Function-of-a-Random-Vector]
        {随机向量的分布函数定义}
        [Definition of Distribution Function of a Random Vector]
        [gpt-4.1]
        设 $X = ( X_1, \ldots, X_d )$ 是一个随机向量,其分布函数定义为 $F(x) = P( X \leq x )$,其中 $x \in \mathbf{R}^d$,且 $X \leq x$ 表示 $X_i \leq x_i$ 对所有 $i = 1, \ldots, d$ 成立.
    \end{dfn}
    
    
    
    \begin{thm}
        [Lévys-Theorem-Sufficiency-of-Convergence-of-Sums-of-Independent-Random-Variables]
        {Lévy定理:独立随机变量和收敛的充分性}
        [Lévy's Theorem: Sufficiency of Convergence of Sums of Independent Random Variables]
        [gpt-4.1]
        
设 $X_1, X_2, \dots$ 是独立的随机变量,$S_n = X_1 + \cdots + X_n$.如果 $\operatorname{lim}_{n \to \infty} S_n$ 以概率收敛(exists in probability),则它也依分布收敛(exists a.s.).

    \end{thm}
    
    
    
    \begin{ppt}
        [Monotonicity-and-Concavity-Properties-of-the-Transformation-under-General-Distribution]
        {广义分布下的变换函数的单调性与凹性性质}
        [Monotonicity and Concavity Properties of the Transformation under General Distribution]
        [gpt-4.1]
        假设分布 $F$ 不仅仅在点 $\mu$ 上取值(即不是点质量),则 $\varphi^{ \prime }( \theta ) / \varphi( \theta )$ 是严格递增函数,且 $a \theta - \log \varphi( \theta )$ 是凹函数.
    \end{ppt}
    
    
    
    \begin{crl}
        [Existence-and-Uniqueness-of-the-Maximizer]
        {关于极值点的存在性与唯一性}
        [Existence and Uniqueness of the Maximizer]
        [gpt-4.1]
        由于 $\varphi^{ \prime }( 0 ) / \varphi( 0 ) = \mu$,对于每个 $a > \mu$,至多存在一个 $\theta_{ a } \geq 0$ 使得 $a = \varphi^{ \prime }( \theta_{ a } ) / \varphi( \theta_{ a } )$,并且该值 $\theta_{ a }$ 能够最大化 $a \theta - \log \varphi( \theta )$.
    \end{crl}
    
    
    
    \begin{dfn}
        [Definition-of-parameter-$	heta-a$-yielding-mean-$a$]
        {使期望为$a$的分布参数$	heta\_a$的定义}
        [Definition of parameter $	heta_a$ yielding mean $a$]
        [gpt-4.1]
        
为使概率分布 $F_\theta$ 的期望为 $a$,令 $F_\theta(\{1\}) = \frac{e^{\theta}}{e^{\theta} + e^{-\theta}} = \frac{1+a}{2}$,令 $x = e^{\theta}$,解得 $x = \sqrt{\frac{1+a}{1-a}}$,从而 $\theta_a = \log x = \frac{1}{2} \{ \log(1+a) - \log(1-a) \}$.

    \end{dfn}
    
    
    
    \begin{dfn}
        [Definition-of-functions-$\varphi	heta-a$-and-$\gammaa$-under-parameter-$	heta-a$]
        {参数$	heta\_a$下的函数$\varphi(	heta\_a)$和$\gamma(a)$的定义}
        [Definition of functions $\varphi(	heta_a)$ and $\gamma(a)$ under parameter $	heta_a$]
        [gpt-4.1]
        
对于参数 $\theta_a$,定义
\[
\begin{array}{l}
\displaystyle \varphi(\theta_a) = \frac{e^{\theta_a} + e^{-\theta_a}}{2} = \frac{e^{\theta_a}}{1+a} = \frac{1}{\sqrt{(1+a)(1-a)}} \\
\displaystyle \gamma(a) = -a\theta_a + \kappa(\theta_a) = -\frac{(1+a)\log(1+a) + (1-a)\log(1-a)}{2}
\end{array}
\]

    \end{dfn}
    
    
    
    \begin{thm}
        [Expectation-and-Variance-Formulas-for-Sums-of-Random-Variables]
        {随机变量和的期望与方差公式}
        [Expectation and Variance Formulas for Sums of Random Variables]
        [gpt-4.1]
        
设 $N$ 为非负整数随机变量,$Y_i$ 为一组独立同分布的随机变量,$S = X_1 + \cdots + X_N$,其中每个 $X_i$ 与 $Y_i$ 分布相同,则有以下公式:
\[
E S = (E N) \cdot E Y_{i}
\]
\[
E S^2 = (E N) \cdot \operatorname{var}(Y_{i}) + E N^2 \cdot (E Y_{i})^2
\]
\[
\operatorname{var}(S) = (E N) \cdot \operatorname{var}(Y_{i}) + \operatorname{var}(N) \cdot (E Y_{i})^2
\]
其中 $\operatorname{var}(N) = E N^2 - (E N)^2$.

    \end{thm}
    
    
    
    \begin{crl}
        [Variance-Formula-for-Sums-under-Poisson-Distribution]
        {泊松分布情形下和的方差公式}
        [Variance Formula for Sums under Poisson Distribution]
        [gpt-4.1]
        
当 $N$ 服从泊松分布,即 $E N = \lambda$ 且 $\operatorname{var}(N) = \lambda$ 时,有
\[
\operatorname{var}(S) = \lambda \cdot E Y_{i}^2
\]
其中用到了 $\operatorname{var}(Y_{i}) + (E Y_{i})^2 = E Y_{i}^2$.

    \end{crl}
    
    
    
    \begin{dfn}
        [Characteristic-Function-Representation-of-Stable-Distribution]
        {稳定分布的特征函数表达式}
        [Characteristic Function Representation of Stable Distribution]
        [gpt-4.1]
        
稳定分布的特征函数可以写为
\[
\exp \left( i t c - b |t|^{\alpha} \{ 1 + i \kappa \ \mathrm{sgn}(t) \, w_{\alpha}(t) \} \right)
\]
其中 $-1 \leq \kappa \leq 1$($\kappa = 2 \theta - 1$),且
\[
w_{\alpha}(t) = \begin{cases}
\tan(\pi \alpha / 2) & \text{如果 } \alpha 
eq 1 \\
(2 / \pi) \log |t| & \text{如果 } \alpha = 1
\end{cases}
\]

    \end{dfn}
    
    
    
    \begin{thm}
        [Theorem-on-Limit-Behavior]
        {关于极限的定理}
        [Theorem on Limit Behavior]
        [gpt-4.1]
        
在上述假设下, 当 $n \to \infty$ 时,
\[
\sup_{ x \in {\mathcal{L}}_n } \left| \frac{ n^{1/2} }{ h } p_n(x) - n(x) \right| \to 0
\]

    \end{thm}
    
    
    
    \begin{thm}
        [Limit-Theorem-for-Empty-Boxes-under-Poisson-Approximation]
        {Poisson极限下空箱概率收敛定理}
        [Limit Theorem for Empty Boxes under Poisson Approximation]
        [gpt-4.1]
        如果 $n e^{- r / n} \to \lambda$,则当 $n \to \infty$ 时,空箱概率收敛:$p_0(r,n) \to \sum_{k=0}^\infty (-1)^k \frac{\lambda^k}{k!} = e^{-\lambda}$.对于固定的 $m$,$(n - m) e^{- r / (n-m)} \to \lambda$,因此 $p_0(r, n-m) \to e^{-\lambda}$.
    \end{thm}
    
    
    
    \begin{ppt}
        [Inclusion-Exclusion-Formula-for-Empty-Box-Probability]
        {空箱概率的包含-排除公式}
        [Inclusion-Exclusion Formula for Empty Box Probability]
        [gpt-4.1]
        空箱概率满足包含-排除公式:
\[
p_0(r, n) = \sum_{k=0}^n (-1)^k \binom{n}{k} \left(1 - \frac{k}{n}\right)^r
\]

    \end{ppt}
    
    
    
    \begin{ppt}
        [Formula-for-Probability-of-Exactly-$m$-Empty-Boxes]
        {精确$m$个空箱时的概率公式}
        [Formula for Probability of Exactly $m$ Empty Boxes]
        [gpt-4.1]
        精确$m$个空箱时的概率为:
\[
p_m(r, n) = \binom{n}{m} \left(1 - \frac{m}{n}\right)^r p_0(r, n-m)
\]

    \end{ppt}
    
    
    
    \begin{lma}
        [Limit-Estimate-for-Binomial-Coefficient-Term]
        {二项系数与极限的估计}
        [Limit Estimate for Binomial Coefficient Term]
        [gpt-4.1]
        若 $n e^{- r / n} \to \lambda$,则
\[
\binom{n}{m} \left(1 - \frac{m}{n}\right)^{r} \sim \frac{\lambda^m}{m!}
\]

    \end{lma}
    
    
    
    \begin{thm}
        [Asymptotic-Formula-for-Renewal-Process-Theorem]
        {更新过程渐近公式定理}
        [Asymptotic Formula for Renewal Process Theorem]
        [gpt-4.1]
        设 $U(t)$ 表示更新过程在时间 $t$ 时的更新次数,$\mu$ 为均值,$
u$ 为方差,则有
\[
0 \leq U(t) - t/\mu \to 
u/2\mu^2 \quad \mathrm{as}\ t\to \infty
\]

    \end{thm}
    
    
    
    \begin{xmp}
        [Example-of-Simplifying-Characteristic-Function-Using-Trigonometric-Identity]
        {三角恒等式化简特征函数的例子}
        [Example of Simplifying Characteristic Function Using Trigonometric Identity]
        [gpt-4.1]
        
利用例 3.6 和定理 3.3.2, 可得所需的特征函数为

\[
\left( \frac{e^{i t / 2} - e^{-i t / 2}}{i t} \right)^{2} = \left( \frac{2 \sin (t / 2)}{t} \right)^{2}
\]

再利用三角恒等式 $\cos 2 \theta = 1 - 2 \sin^{2} \theta$ (其中 $\theta = t / 2$), 可将答案转化为此处给定的形式.

    \end{xmp}
    
    
    
    \begin{dfn}
        [Density-Function-of-Normal-Distribution]
        {正态分布的密度函数}
        [Density Function of Normal Distribution]
        [gpt-4.1]
        
正态分布(均值为 $\mu$,方差为 $\sigma^2$)的密度函数为
\[
f(x) = (2\pi)^{-1/2} \exp\left( - \frac{(x-\mu)^2}{2\sigma^2} \right)
\]
其中本例中 $\mu=\theta$,$\sigma^2=1$.

    \end{dfn}
    
    
    
    \begin{dfn}
        [Moment-Generating-Function-of-Normal-Distribution]
        {正态分布的矩母函数}
        [Moment Generating Function of Normal Distribution]
        [gpt-4.1]
        
正态分布(均值为 $0$,方差为 $1$)的矩母函数为
\[
\varphi(\theta) = \int e^{\theta x} (2\pi)^{-1/2} \exp(-x^2/2) dx = \exp(\theta^2/2)
\]
对于 $\theta \in (-\infty, \infty)$.

    \end{dfn}
    
    
    
    \begin{dfn}
        [Distribution-Function-of-Normal-Distribution-with-Mean-θ-and-Variance-1]
        {均值为 $	heta$ 方差为 1 的正态分布的分布函数}
        [Distribution Function of Normal Distribution with Mean θ and Variance 1]
        [gpt-4.1]
        
均值为 $\theta$,方差为 1 的正态分布的分布函数为
\[
F_{\theta}(x) = e^{-\theta^2/2} \int_{-\infty}^{x} e^{\theta y} (2\pi)^{-1/2} e^{-y^2/2} dy
\]

    \end{dfn}
    
    
    
    \begin{thm}
        [Relationship-Between-Existence-of-Second-Derivatives-and-Existence-of-Second-Moments]
        {二阶导数存在性与二阶矩存在性的关系}
        [Relationship Between Existence of Second Derivatives and Existence of Second Moments]
        [gpt-4.1]
        
如果 $\limsup_{h \downarrow 0} \frac{ \varphi(h) - 2\varphi(0) + \varphi(-h) }{ h^2 } > -\infty$,则 $E|X|^2 < \infty$.

    \end{thm}
    
    
    
    \begin{thm}
        [Stability-of-the-Product-of-Symmetric-Stable-and-Independent-Stable-Distributions]
        {对称稳定分布与独立稳定分布的乘积的稳定性}
        [Stability of the Product of Symmetric Stable and Independent Stable Distributions]
        [gpt-4.1]
        若 $X$ 是指数为 $\alpha$ 的对称稳定分布随机变量, $Y \geq 0$ 是与 $X$ 独立的指数为 $\beta < 1$ 的稳定分布随机变量, 则 $X Y^{1/\alpha}$ 也是对称稳定分布, 其指数为 $\alpha \beta$.
    \end{thm}
    
    
    
    \begin{xmp}
        [Three-Examples-of-Infinitely-Divisible-Distributions]
        {分布无限可分性的三个例子}
        [Three Examples of Infinitely Divisible Distributions]
        [gpt-4.1]
        
在前三种情形中,分布是无限可分的,因为它是形如 $( * )$ 的和的极限.这些例子对应相关的极限定理.

    \end{xmp}
    
    
    
    \begin{thm}
        [Necessary-and-Sufficient-Condition-for-Convergence-of-Discrete-Random-Variables]
        {离散型随机变量收敛的充要条件}
        [Necessary and Sufficient Condition for Convergence of Discrete Random Variables]
        [gpt-4.1]
        设 $X_n$, $1 \leq n \leq \infty$, 均为整值随机变量.则
\[
X_n \Rightarrow X_\infty
\]
当且仅当对于所有整数 $m$,都有
\[
P(X_n = m) \to P(X_\infty = m)
\]
成立.
    \end{thm}
    
    
    
    \begin{thm}
        [Theorem-on-the-Limit-of-Probability-for-Random-Variables]
        {关于随机变量和概率极限的定理}
        [Theorem on the Limit of Probability for Random Variables]
        [gpt-4.1]
        
定理 2.7.1 表明
\[
\lim_{n \to \infty} \frac{1}{n} \log P(S_n \geq n a) = \gamma(a)
\]
存在,因此我们对上确界的下界估计已经足够.

    \end{thm}
    
    
    
    \begin{lma}
        [Sufficient-Condition-for-Direct-Riemann-Integrability]
        {直接Riemann可积性的充分条件}
        [Sufficient Condition for Direct Riemann Integrability]
        [gpt-4.1]
        
若 $h(x)\geq 0$ 且为递减函数,满足 $h(0)<\infty$ 且 $\int_{0}^{\infty} h(x)\, dx < \infty$,则 $h$ 是直接Riemann可积的函数.

    \end{lma}
    
    
    
    \begin{thm}
        [Normalization-and-Invariance-of-Stable-Distributions]
        {稳定分布的归一化和分布不变性}
        [Normalization and Invariance of Stable Distributions]
        [gpt-4.1]
        如果 $X_{1}, X_{2}, \dots$ 是独立的随机变量,且它们的特征函数为 $\exp(-|t|^{\alpha})$,那么 $(X_{1} + \cdots + X_{n}) / n^{1/\alpha}$ 与 $X_{1}$ 具有相同的分布.
    \end{thm}
    
    
    
    \begin{prf}
        [Proof-of-Comparison-Between-Integral-and-Discrete-Sum-for-Decreasing-Function]
        {关于单调递减函数积分与离散和的比较的证明}
        [Proof of Comparison Between Integral and Discrete Sum for Decreasing Function]
        [gpt-4.1]
        因为 $h$ 是递减函数,$I^{\delta} = \sum_{k=0}^{\infty} \delta\, h(k\delta)$,$I_{\delta} = \sum_{k=0}^{\infty} \delta\, h((k+1)\delta)$.所以
\[
I^{\delta} \geq \int_{0}^{\infty} h(x)\, dx \geq I_{\delta} = I^{\delta} - h(0)\delta
\]
证明了所需结果.

    \end{prf}
    
    
    
    \begin{thm}
        [Limiting-Distribution-Theorem-for-$F-n$]
        {关于 $F\_n$ 的极限分布结论}
        [Limiting Distribution Theorem for $F_n$]
        [gpt-4.1]
        (i) 若 $p < 1/2$, 则 $F_{n} / n^{1/2 - p} \Rightarrow c \chi$, 其中 $\chi$ 为标准正态分布.

(ii) 若 $p = 1/2$, 则 $F_{n} / (\log n)^{1/2} \Rightarrow c \chi$.
    \end{thm}
    
    
    
    \begin{dfn}
        [Definition-of-Upper-and-Lower-Riemann-Sums]
        {上、下Riemann和的定义}
        [Definition of Upper and Lower Riemann Sums]
        [gpt-4.1]
        
设 $h$ 是 $[0,\infty)$ 上的任意函数,定义

\[
\begin{array}{l}
{\displaystyle I^{\delta} = \sum_{k=0}^{\infty} \delta \sup \{ h(x) : x \in [k\delta,\ (k+1)\delta) \}} \\
{\displaystyle I_{\delta} = \sum_{k=0}^{\infty} \delta \inf \{ h(x) : x \in [k\delta,\ (k+1)\delta) \}}
\end{array}
\]

分别为 $h$ 在 $[0,\infty)$ 上积分的上Riemann和与下Riemann和的近似.

    \end{dfn}
    
    
    
    \begin{xmp}
        [Example-of-Properties-of-Exponential-Distribution-with-Parameter]
        {带参数的指数分布的性质举例}
        [Example of Properties of Exponential Distribution with Parameter]
        [gpt-4.1]
        
设 $X$ 服从参数为 $\lambda$ 的指数分布,若 $\theta < \lambda$,则有
\[
\int_{ 0 }^{ \infty } e^{ \theta x } \lambda e^{ - \lambda x } dx = \frac{\lambda}{\lambda - \theta}
\]
记 $\varphi( \theta ) = \frac{\lambda}{\lambda - \theta}$,则 $\varphi^{ \prime }( \theta ) / \varphi( \theta ) = \frac{1}{\lambda - \theta}$,且
\[
F_{ \theta }( x ) = \frac{ \lambda }{ \lambda - \theta } \int_{ 0 }^{ x } e^{ \theta y } \lambda e^{ - \lambda y } dy
\]
是参数为 $\lambda - \theta$ 的指数分布,因此其均值为 $1 / (\lambda - \theta)$.

    \end{xmp}
    
    
    
    \begin{prf}
        [Proof-of-the-Convergence-Properties-of-Empirical-Distribution-Function]
        {经验分布函数收敛性质的证明}
        [Proof of the Convergence Properties of Empirical Distribution Function]
        [gpt-4.1]
        
Fix $x$ and let $Y_{n} = 1_{(X_{n} \leq x)}$. Since the $Y_{n}$ are i.i.d. with $E Y_{n} = P(X_{n} \leq x) = F(x)$, the strong law of large numbers implies that $F_{n}(x) = n^{-1} \sum_{m=1}^{n} Y_{m} \to F(x)$ a.s. In general, if $F_{n}$ is a sequence of nondecreasing functions that converges pointwise to a bounded and continuous limit $F$, then $\operatorname{sup}_{x} |F_{n}(x) - F(x)| \to 0$ a.s.

    \end{prf}
    
    
    
    \begin{thm}
        [Theorem-Convergence-in-Probability-from-Moment-Convergence-to-a-Constant]
        {由矩的收敛推出随机变量收敛于常数的定理}
        [Theorem: Convergence in Probability from Moment Convergence to a Constant]
        [gpt-4.1]
        
设 $Y_n \geq 0$,且存在 $0 < \alpha < \beta$,使得 $E Y_n^{\alpha} \to 1$ 且 $E Y_n^{\beta} \to 1$,则 $Y_n \to 1$ 以概率收敛.

    \end{thm}
    
    
    
    \begin{xmp}
        [Characteristic-Function-of-Uniform-Distribution-on-$a-b$]
        {区间 $(a, b)$ 上的均匀分布的特征函数}
        [Characteristic Function of Uniform Distribution on $(a, b)$]
        [gpt-4.1]
        
设随机变量 $X$ 在区间 $(a, b)$ 上服从均匀分布,其概率密度函数为
\[
f(x) = \frac{1}{b - a}, \quad x \in (a, b)
\]
其特征函数为
\[
\varphi_X(t) = \frac{e^{i t b} - e^{i t a}}{i t (b - a)}
\]

特别地,在 $a = -c$, $b = c$ 的情形下,特征函数为
\[
\varphi_X(t) = \frac{e^{i t c} - e^{-i t c}}{2 c i t} = \frac{\sin c t}{c t}
\]

    \end{xmp}
    
    
    
    \begin{prf}
        [Proof-of-Characteristic-Function-for-Uniform-Distribution]
        {均匀分布特征函数的计算}
        [Proof of Characteristic Function for Uniform Distribution]
        [gpt-4.1]
        
只需注意到积分公式
\[
\int_{a}^{b} e^{\lambda x} dx = \frac{e^{\lambda b} - e^{\lambda a}}{\lambda}
\]
对复数 $\lambda$ 也成立,由此直接计算即可得均匀分布的特征函数.

    \end{prf}
    
    
    
    \begin{prf}
        [Proof-of-the-Lower-Bound-of-$\liminf-S-n/n$]
        {关于 $S\_n/n$ 下确界的证明}
        [Proof of the Lower Bound of $\liminf S_n/n$]
        [gpt-4.1]
        
Proof Let $M > 0$ and $X_{i}^{M} = X_{i} \wedge M$.
The $X_{i}^{M}$ are i.i.d. with $E | X_{i}^{M} | < \infty$, so if $S_{n}^{M} = X_{1}^{M} + \cdots + X_{n}^{M}$, then Theorem 2.4.1 implies $S_{n}^{M} / n \to E X_{i}^{M}$.
Since $X_{i} \geq X_{i}^{M}$, it follows that

\[
\liminf_{n \to \infty} S_{n} / n \geq \lim_{n \to \infty} S_{n}^{M} / n = E X_{i}^{M}
\]

The monotone convergence theorem implies $E (X_{i}^{M})^{+} \uparrow E X_{i}^{+} = \infty$ as $M \uparrow \infty$, so $E X_{i}^{M} = E (X_{i}^{M})^{+} - E (X_{i}^{M})^{-} \uparrow \infty$, and we have $\liminf_{n \to \infty} S_{n} / n \geq \infty$, which implies the desired result.

    \end{prf}
    
    
    
    \begin{thm}
        [Independent-Variables-and-Zero-under-Identical-Distribution]
        {独立变量和分布相同条件下的变量为零}
        [Independent Variables and Zero under Identical Distribution]
        [gpt-4.1]
        
设随机变量 $X$ 和 $Y$ 独立,且 $X + Y$ 与 $X$ 同分布,则 $Y = 0$ 几乎处处成立.

    \end{thm}
    
    
    
    \begin{lma}
        [Integral-Representation-of-the-Difference-of-Two-Distribution-Functions]
        {两个分布函数之差的积分表达式}
        [Integral Representation of the Difference of Two Distribution Functions]
        [gpt-4.1]
        
设 $K_{1}$ 和 $K_{2}$ 是均值为 $\boldsymbol{\theta}$,其特征函数 $\kappa_{i}$ 可积的分布函数,则有
\[
K_{1}(x) - K_{2}(x) = (2\pi)^{-1} \int e^{-i t x} \frac{\kappa_{1}(t) - \kappa_{2}(t)}{i t} dt
\]

    \end{lma}
    
    
    
    \begin{prf}
        [Proof-of-the-Lemma]
        {引理的证明}
        [Proof of the Lemma]
        [gpt-4.1]
        
由于 $\kappa_{i}$ 可积,根据反演公式(定理 3.3.11),密度 $k_{i}(x)$ 有
\[
k_{i}(y) = (2\pi)^{-1} \int e^{-i t y} \kappa_{i}(t) dt
\]
对上述表达式用 $i=2$ 从 $i=1$ 相减,再对 $y$ 从 $a$ 到 $x$ 积分,令 $\Delta K = K_{1} - K_{2}$,得
\[
\begin{aligned}
\Delta K(x) - \Delta K(a) &= (2\pi)^{-1} \int_{a}^{x} \int e^{-i t y} \left\{ \kappa_{1}(t) - \kappa_{2}(t) \right\} dt\, dy \\
&= (2\pi)^{-1} \int \left\{ e^{-i t a} - e^{-i t x} \right\} \frac{\kappa_{1}(t) - \kappa_{2}(t)}{i t} dt
\end{aligned}
\]
由于 $\kappa_{i}$ 在 $t$ 上可积且 $y$ 在有界区间,应用 Fubini 定理是合理的.因 $K_{i}$ 均值为 0,故 $\{ 1 - \kappa_{i}(t) \} / t \to 0$,从而 $(\kappa_{1}(t) - \kappa_{2}(t)) / i t$ 有界且连续.该因子在 $t$ 大时提高了可积性,因此 $(\kappa_{1}(t) - \kappa_{2}(t)) / i t$ 可积.令 $a \to -\infty$ 并用 Riemann-Lebesgue 引理(练习 1.4.4)可证得结论.

    \end{prf}
    
    
    
    \begin{thm}
        [Coordinate-Convergence-of-Random-Vectors]
        {随机向量收敛的坐标分量}
        [Coordinate Convergence of Random Vectors]
        [gpt-4.1]
        设 $X_n$ 是随机向量.若 $X_n \Rightarrow X$,则各坐标分量 $X_{n,i} \Rightarrow X_i$.
    \end{thm}
    
    
    
    \begin{xmp}
        [Cycle-Decomposition-and-Counting-in-Random-Permutations]
        {随机排列的循环分解与计数}
        [Cycle Decomposition and Counting in Random Permutations]
        [gpt-4.1]
        设 $\Omega_n$ 由 $n!$ 个排列组成(即从 $\{ 1, \ldots, n \}$ 到 $\{ 1, \ldots, n \}$ 的一一对应函数),并假定所有排列等可能,构成一个概率空间.本例讨论随机排列 $\pi$ 的循环结构.排列可分解为若干循环:依次考察序列 $1, \pi(1), \pi(\pi(1)), \ldots$ ,最终 $\pi^k(1) = 1$,此时第一个循环完成且长度为 $k$.之后,选取未在前一循环中的最小整数 $i$,考察 $i, \pi(i), \pi(\pi(i)), \ldots$ ,如此反复,直到所有元素都被包含.例如,若
\[
\begin{array}{cccccccccc}
i & 1 & 2 & 3 & 4 & 5 & 6 & 7 & 8 & 9 \\
\pi(i) & 3 & 9 & 6 & 8 & 2 & 1 & 5 & 4 & 7
\end{array}
\]
则循环分解为 $(136)\ (2975)\ (48)$.定义 $X_{n,k} = 1$ 表示在分解中第 $k$ 个数后出现右括号,否则 $X_{n,k} = 0$,$S_n = X_{n,1} + \cdots + X_{n,n}$ 即为循环的个数.在上述例子中,$X_{9,3} = X_{9,7} = X_{9,9} = 1$,其余 $X_{9,m} = 0$.声明:$P(X_{n,j} = 1) = \frac{1}{n-j+1}$.直观上,因为剩余 $n-j+1$ 个未出现的值,只有其中一个能完成该循环.

    \end{xmp}
    
    
    
    \begin{prf}
        [Proof-of-the-Probability-of-Cycle-Termination-in-Random-Permutations]
        {随机排列循环终止概率的证明}
        [Proof of the Probability of Cycle Termination in Random Permutations]
        [gpt-4.1]
        为证明 $P(X_{n,j} = 1) = \frac{1}{n-j+1}$,采用如下生成排列的特殊方式:令 $i_1 = 1$,从 $\{ 1, \ldots, n \}$ 中随机选取 $j_1$ 并令 $\pi(i_1) = j_1$.若 $j_1 
eq 1$,则 $i_2 = j_1$;若 $j_1 = 1$,则 $i_2 = 2$.两种情况下,从 $\{ 1, \ldots, n \} \setminus \{ j_1 \}$ 中随机选取 $j_2$.一般地,若 $i_1, j_1, \ldots, i_{k-1}, j_{k-1}$ 已被选定,且已设置 $\pi(i_\ell) = j_\ell$,$1 \leq \ell < k$,则:
(a) 若 $j_{k-1} \in \{ i_1, \ldots, i_{k-1} \}$(刚完成一个循环),则 $i_k = \inf ( \{ 1, \ldots, n \} \setminus \{ i_1, \ldots, i_{k-1} \} )$;
(b) 若 $j_{k-1} 
otin \{ i_1, \ldots, i_{k-1} \}$,则 $i_k = j_{k-1}$.
无论哪种情况,从 $\{ 1, \ldots, n \} \setminus \{ j_1, \ldots, j_{k-1} \}$ 中随机选取 $j_k$,并令 $\pi(i_k) = j_k$.

    \end{prf}
    
    
    
    \begin{prf}
        [Proof-Regarding-the-Application-of-the-Strong-Law-of-Large-Numbers]
        {关于强大数定律应用的证明}
        [Proof Regarding the Application of the Strong Law of Large Numbers]
        [gpt-4.1]
        证明 由定理 2.4.1 和 2.4.5, $T_{n} / n \to \mu$ a.s..根据 $N_{t}$ 的定义, 有 $T(N_{t}) \leq t < T(N_{t} + 1)$, 所以两边同时除以 $N_{t}$ 得

\[
\frac{T(N_{t})}{N_{t}} \leq \frac{t}{N_{t}} \leq \frac{T(N_{t} + 1)}{N_{t} + 1} \cdot \frac{N_{t} + 1}{N_{t}}
\]

为了取极限, 注意到由于对所有 $n$ 有 $T_{n} < \infty$, 所以当 $t \to \infty$ 时 $N_{t} \uparrow \infty$.强大数定律指出, 对于满足 $P(\Omega_{0}) = 1$ 的 $\omega \in \Omega_{0}$, 有 $T_{n}(\omega) / n \to \mu$, $N_{t}(\omega) \uparrow \infty$, 因此

\[
\frac{T_{N_{t}(\omega)}(\omega)}{N_{t}(\omega)} \to \mu \qquad \frac{N_{t}(\omega) + 1}{N_{t}(\omega)} \to 1
\]

由此可得, 对于 $\omega \in \Omega_{0}$, 有 $t / N_{t}(\omega) \to \mu$ a.s..

    \end{prf}
    
    
    
    \begin{dfn}
        [Definition-of-Distribution-Function-under-Change-of-Measure]
        {改变测度下的分布函数的定义}
        [Definition of Distribution Function under Change of Measure]
        [gpt-4.1]
        
令 $F_\theta(x)$ 定义为
\[
F_{ \theta }( x ) = \frac{ 1 }{ \varphi( \theta ) } \int_{ - \infty }^{ x } e^{ \theta y } dF( y )
\]
当且仅当 $\varphi( \theta ) < \infty$ 时该定义有意义.

    \end{dfn}
    
    
    
    \begin{ppt}
        [Property-of-Laplace-Transform-for-Stable-Law]
        {稳定分布的拉普拉斯变换性质}
        [Property of Laplace Transform for Stable Law]
        [gpt-4.1]
        设 $Y$ 是参数 $\alpha < 1$ 且 $\kappa = 1$ 的稳定分布, 且 $Y \geq 0$.其拉普拉斯变换定义为 $\psi ( \lambda ) = E \exp ( - \lambda Y )$.对于任意整数 $n \geq 1$, 有
\[
\psi ( \lambda ) ^ { n } = \psi ( n ^ { 1 / \alpha } \lambda )
\]
由此可得
\[
E \exp ( - \lambda Y ) = \exp ( - c \lambda ^ { \alpha } )
\]
其中 $c > 0$ 为常数.

    \end{ppt}
    
    
    
    \begin{lma}
        [Lemma-and-Proof-on-the-Lower-Bound-of-Convolution]
        {关于卷积下界的引理及证明}
        [Lemma and Proof on the Lower Bound of Convolution]
        [gpt-4.1]
        \[
\eta_{L} \geq \frac{\eta}{2} - \frac{12\lambda}{\pi L} \quad \text{or} \quad \eta \leq 2\eta_{L} + \frac{24\lambda}{\pi L}
\]

证明:$\Delta$ 在 $\pm \infty$ 时趋于 0,$G$ 连续,$F$ 是 d.f.,因此存在 $x_{0}$ 使得 $\Delta(x_{0}) = \eta$ 或 $\Delta(x_{0}-) = -\eta$.通过考虑 $(-1)$ 倍 r.v. 的 d.f. 在第二种情况下,无损地假设 $\Delta(x_{0}) = \eta$.由于 $G'(x) \leq \lambda$ 且 $F$ 非递减,$\Delta(x_{0}+s) \geq \eta - \lambda s$.令 $\delta = \eta / 2\lambda$,$t = x_{0} + \delta$,则有
\[
\Delta(t-x) \geq
\begin{cases}
(\eta/2) + \lambda x & \text{当 } |x| \leq \delta \\
-\eta & \text{否则}
\end{cases}
\]

为了估计卷积 $\Delta_{L}$,我们有
\[
2 \int_{\delta}^{\infty} h_{L}(x) dx \leq 2 \int_{\delta}^{\infty} \frac{2}{\pi L x^{2}} dx = \frac{4}{\pi L \delta}
\]

分别考察 $(-\delta, \delta)$ 及其补集,并注意到对称性有
\[
\int_{-\delta}^{\delta} x h_{L}(x) dx = 0
\]
因此
\[
\eta_{L} \geq \Delta_{L}(t) \geq \frac{\eta}{2} \left( 1 - \frac{4}{\pi L \delta} \right) - \eta \frac{4}{\pi L \delta} = \frac{\eta}{2} - \frac{6\eta}{\pi L \delta} = \frac{\eta}{2} - \frac{12\lambda}{\pi L}
\]

从而引理得证.

    \end{lma}
    
    
    
    \begin{dfn}
        [Definition-of-Concavity-of-a-Function]
        {函数的凹性定义}
        [Definition of Concavity of a Function]
        [gpt-4.1]
        
设 $\gamma$ 是一个定义在某集合上的函数.如果对于任意有理数 $\lambda \in [0,1]$,以及任意 $a, b$,都有
\[
\gamma(\lambda a + (1 - \lambda) b) \geq \lambda \gamma(a) + (1 - \lambda) \gamma(b)
\]
则称 $\gamma$ 是凹函数.

    \end{dfn}
    
    
    
    \begin{crl}
        [Extension-of-Concavity-to-All-Parameters]
        {凹性在所有参数上的推广}
        [Extension of Concavity to All Parameters]
        [gpt-4.1]
        
利用单调性可得上述关系对所有 $\lambda \in [0, 1]$ 成立,因此 $\gamma$ 是凹函数,且在 $\gamma(a) > -\infty$ 的紧子集上是 Lipschitz 连续的.

    \end{crl}
    
    
    
    \begin{dfn}
        [Definition-of-the-Piecewise-Function-ψ]
        {分段函数ψ的定义}
        [Definition of the Piecewise Function ψ]
        [gpt-4.1]
        
设 $\psi(x)$ 定义如下:当 $|x| \leq 1$ 时,$\psi(x) = x^2$;当 $|x| \geq 1$ 时,$\psi(x) = |x|$.

    \end{dfn}
    
    
    
    \begin{thm}
        [Convergence-Criterion-for-Series-of-Independent-Zero-Mean-Random-Variables]
        {独立零均值随机变量级数收敛性的判别准则}
        [Convergence Criterion for Series of Independent Zero Mean Random Variables]
        [gpt-4.1]
        
若 $X_1, X_2, \dots$ 是独立随机变量,且 $E X_n = 0$,并且有 $\sum_{n=1}^{\infty} E \psi(X_n) < \infty$,则 $\sum_{n=1}^{\infty} X_n$ 收敛.

    \end{thm}
    
    
    
    \begin{thm}
        [Strong-Law-of-Large-Numbers]
        {强大数定律}
        [Strong Law of Large Numbers]
        [gpt-4.1]
        设 $X_{1}, X_{2}, \ldots$ 是一列独立同分布的随机变量,且 $E|X_{i}| < \infty$.令 $E X_{i} = \mu$,$S_{n} = X_{1} + \cdots + X_{n}$.则有 $S_{n} / n \to \mu$ 几乎必然,当 $n \to \infty$.
    \end{thm}
    
    
    
    \begin{thm}
        [Limit-Property-of-Uniform-Distribution-on-High-Dimensional-Sphere]
        {高维球面分布的极限性质}
        [Limit Property of Uniform Distribution on High-Dimensional Sphere]
        [gpt-4.1]
        设 $X_{n} = (X_{n}^{1}, \ldots, X_{n}^{n})$ 在 $\mathbf{R}^{n}$ 中半径为 $\sqrt{n}$ 的球面上均匀分布,则随着 $n \to \infty$,有 $X_{n}^{1} \Rightarrow$ 标准正态分布(standard normal).

    \end{thm}
    
    
    
    \begin{dfn}
        [Construction-of-Uniformly-Distributed-Variables-on-High-Dimensional-Sphere]
        {高维球面上的均匀分布变量的构造}
        [Construction of Uniformly Distributed Variables on High-Dimensional Sphere]
        [gpt-4.1]
        令 $Y_{1}, Y_{2}, \ldots$ 是独立同分布的标准正态变量,定义 $X_{n}^{i} = Y_{i} \left( n / \sum_{m=1}^{n} Y_{m}^{2} \right)^{1/2}$,则 $X_{n} = (X_{n}^{1}, \ldots, X_{n}^{n})$ 在 $\mathbf{R}^{n}$ 的半径为 $\sqrt{n}$ 的球面上均匀分布.

    \end{dfn}
    
    
    
    \begin{thm}
        [Inequality-for-Maximum-of-Partial-Sums-of-Independent-Random-Variables]
        {独立随机变量部分和最大值概率的不等式}
        [Inequality for Maximum of Partial Sums of Independent Random Variables]
        [gpt-4.1]
        
设 $X_1, X_2, \dots$ 相互独立,定义 $S_{m, n} = X_{m+1} + \cdots + X_n$.则有
\[
P\left(\max_{m < j \leq n} |S_{m, j}| > 2a\right) \leq \min_{m < k \leq n} P(|S_{k, n}| \leq a) \leq P(|S_{m, n}| > a)
\]

    \end{thm}
    
    
    
    \begin{prf}
        [Proof-of-the-Inference-$F-{n}^{-1}x-\leq-y$]
        {关于$F\_{n}^{-1}(x) \leq y$的推断证明}
        [Proof of the Inference $F_{n}^{-1}(x) \leq y$]
        [gpt-4.1]
        设 $y > F^{-1}(x)$ 且 $F$ 在 $y$ 处连续.由于 $x \in \Omega_{0}$,有 $F(y) > x$,并且当 $n$ 充分大时,$F_{n}(y) > x$,即 $F_{n}^{-1}(x) \leq y$.由于上述结论对所有满足条件的 $y$ 均成立,故此结论得证.
    \end{prf}
    
    
    
    \begin{dfn}
        [Definition-of-Delayed-Renewal-Process]
        {延迟更新过程的定义}
        [Definition of Delayed Renewal Process]
        [gpt-4.1]
        
若 $T_0 \geq 0$ 独立于 $\xi_1, \xi_2, \ldots$,且其分布为 $G$,则由 $T_k = T_{k-1} + \xi_k$,$k \geq 1$ 所定义的过程为延迟更新过程,其中 $G$ 称为延迟分布.

    \end{dfn}
    
    
    
    \begin{dfn}
        [Counting-Process-and-Expectation-in-Delayed-Renewal-Process]
        {延迟更新过程的计数过程及期望}
        [Counting Process and Expectation in Delayed Renewal Process]
        [gpt-4.1]
        
令 $N_t = \inf\{k : T_k > t\}$,并设 $V(t) = E N_t$,则按照 $T_0$ 的取值分解,有
\[
V(t) = \int_0^t U(t-s)\,dG(s)
\]
其中 $U(r) = 0$ 对于 $r < 0$,且上式可以写作 $V = U * G$,其中 $*$ 表示卷积.

    \end{dfn}
    
    
    
    \begin{xmp}
        [Example-of-Integral-Equation-with-Constant-Input-Function]
        {常数输入函数的积分方程例子}
        [Example of Integral Equation with Constant Input Function]
        [gpt-4.1]
        $h \equiv 1 \colon U(t) = 1 + \int_{0}^{t} U(t - s) d\mathcal{F}(s)$
    \end{xmp}
    
    
    
    \begin{xmp}
        [Example-of-Integral-Equation-with-General-Input-Function]
        {一般输入函数的积分方程例子}
        [Example of Integral Equation with General Input Function]
        [gpt-4.1]
        $h(t) = G(t) \colon V(t) = G(t) + \int_{0}^{t} V(t - s) dF(s)$
    \end{xmp}
    
    
    
    \begin{prf}
        [Proof-of-the-Approximate-Probability-Density-Formula]
        {概率密度近似公式的证明}
        [Proof of the Approximate Probability Density Formula]
        [gpt-4.1]
        设 $Y$ 是一个随机变量,满足 $P ( Y \in a + \theta \mathbf { Z } ) = 1$,定义特征函数 $\psi ( t ) = E \exp ( i t Y )$.

由练习 3.3.2 的结果,得到
\[
P ( Y = x ) = \frac{ 1 }{ 2 \pi / \theta } \int_{ - \pi / \theta }^{ \pi / \theta } e^{ - i t x } \psi ( t ) d t
\]

令 $\theta = h / \sqrt{ n }$,$\psi ( t ) = E \exp ( i t S_n / \sqrt{ n } ) = \varphi^n ( t / \sqrt{ n } )$,两边乘以 $1 / \theta$ 得
\[
\frac{ n^{1/2} }{ h } p_n(x) = \frac{ 1 }{ 2 \pi } \int_{ - \pi \sqrt{ n } / h }^{ \pi \sqrt{ n } / h } e^{ - i t x } \varphi^n \left( \frac{ t }{ \sqrt{ n } } \right) d t
\]

利用反演公式定理 3.3.14,对于 $n(x)$,其特征函数为 $\exp ( - \sigma^2 t^2 / 2 )$,有
\[
n(x) = \frac{ 1 }{ 2 \pi } \int_{ -\infty }^{ \infty } e^{ - i t x } \exp ( - \sigma^2 t^2 / 2 ) d t
\]

两式相减,得到(注意 $\pi > 1$, $|e^{ - i t x }| \leq 1$)
\[
\begin{array}{l}
\displaystyle \left| \frac{ n^{1/2} }{ h } p_n(x) - n(x) \right| \le \int_{ - \pi \sqrt{ n } / h }^{ \pi \sqrt{ n } / h } \left| \varphi^n \left( \frac{ t }{ \sqrt{ n } } \right) - \exp ( - \sigma^2 t^2 / 2 ) \right| d t \\
\displaystyle \qquad + \int_{ \pi \sqrt{ n } / h }^{ \infty } \exp ( - \sigma^2 t^2 / 2 ) d t
\end{array}
\]

右侧与 $x$ 无关,因此只需证明其趋于 0,即可完成定理 3.5.3 的证明.

    \end{prf}
    
    
    
    \begin{lma}
        [Upper-Bound-for-the-Distance-of-Product-Measures]
        {乘积测度距离的上界}
        [Upper Bound for the Distance of Product Measures]
        [gpt-4.1]
        
如果 $\mu_1 \times \mu_2$ 表示在 $\mathbf{Z} \times \mathbf{Z}$ 上的乘积测度,满足 $(\mu_1 \times \mu_2)(x, y) = \mu_1(x) \mu_2(y)$,那么有

\[
\| \mu_1 \times \mu_2 - 
u_1 \times 
u_2 \| \leq \| \mu_1 - 
u_1 \| + \| \mu_2 - 
u_2 \|
\]

且
\[
\| \mu_1 \times \mu_2 - 
u_1 \times 
u_2 \| = \frac{1}{2} \sum_{x, y} | \mu_1(x) \mu_2(y) - 
u_1(x) 
u_2(y) | \leq \frac{1}{2} \sum_{x, y} | \mu_1(x) \mu_2(y) - 
u_1(x) \mu_2(y) | + \frac{1}{2} \sum_{x, y} | 
u_1(x) \mu_2(y) - 
u_1(x) 
u_2(y) | = \frac{1}{2} \sum_{y} \mu_2(y) \sum_{x} | \mu_1(x) - 
u_1(x) | + \frac{1}{2} \sum_{x} 
u_1(x) \sum_{y} | \mu_2(y) - 
u_2(y) | = \| \mu_1 - 
u_1 \| + \| \mu_2 - 
u_2 \|
\]

这给出了所需的结果.

    \end{lma}
    
    
    
    \begin{thm}
        [Conditional-Cauchy-Schwarz-Inequality]
        {条件柯西-施瓦茨不等式}
        [Conditional Cauchy-Schwarz Inequality]
        [gpt-4.1]
        
对于随机变量 $X$ 和 $Y$ 以及子 $\sigma$-代数 $\mathcal{G}$,成立以下不等式:
\[
E ( X Y | \mathcal { G } ) ^ { 2 } \leq E ( X ^ { 2 } | \mathcal { G } ) E ( Y ^ { 2 } | \mathcal { G } )
\]

    \end{thm}
    
    
    
    \begin{xmp}
        [Distribution-of-Sum-of-Independent-Cauchy-Variables]
        {柯西分布的独立变量和的分布}
        [Distribution of Sum of Independent Cauchy Variables]
        [gpt-4.1]
        若 $X_1, X_2, \ldots$ 是相互独立且服从柯西分布的随机变量,则 $(X_1 + \cdots + X_n)/n$ 的分布与 $X_1$ 相同.
    \end{xmp}
    
    
    
    \begin{dfn}
        [Definition-of-Conditional-Distribution]
        {条件分布的定义}
        [Definition of Conditional Distribution]
        [gpt-4.1]
        
设 $X$ 和 $Y$ 具有联合密度 $f(x,y) > 0$.若
\[
\mu(y, A) = \frac{\int_A f(x, y) dx}{\int f(x, y) dx}
\]
则 $\mu(Y(\omega), A)$ 是在给定 $\sigma(Y)$ 下 $X$ 的条件分布函数(r.c.d.).

    \end{dfn}
    
    
    
    \begin{dfn}
        [Definition-of-Probability-Distribution-of-Random-Variable-$X-j$]
        {随机变量 $X\_j$ 的概率分布定义}
        [Definition of Probability Distribution of Random Variable $X_j$]
        [gpt-4.1]
        设 $X_j$ 是一个随机变量,其概率分布如下:
\[
P(X_j = j) = P(X_j = -j) = \frac{1}{2j^{\beta}}, \quad P(X_j = 0) = 1 - j^{-\beta},
\]
其中 $\beta > 0$.
    \end{dfn}
    
    
    
    \begin{thm}
        [Theorem-on-Convergence-Properties-of-$S-n$]
        {关于 $S\_n$ 收敛性质的定理}
        [Theorem on Convergence Properties of $S_n$]
        [gpt-4.1]
        设 $S_n = \sum_{j=1}^n X_j$,则有:
(i) 当 $\beta > 1$ 时,$S_n \to S_\infty$ 几乎处处收敛;
(ii) 当 $\beta < 1$ 时,$S_n / n^{(3-\beta)/2} \Rightarrow c \chi$;
(iii) 当 $\beta = 1$ 时,$S_n / n \Rightarrow \aleph$,其中
\[
E \exp(i t \mathfrak{s}) = \exp\left(-\int_{0}^{1} x^{-1}(1-\cos xt) dx\right)
\]
.
    \end{thm}
    
    
    
    \begin{thm}
        [Inversion-Formula]
        {反演公式}
        [Inversion Formula]
        [gpt-4.1]
        
若 $A = [ a _ { 1 } , b _ { 1 } ] \times \ldots \times [ a _ { d } , b _ { d } ]$ 且 $\mu ( \partial A ) = 0$,则有
\[
\mu ( A ) = \lim_{T \to \infty} ( 2 \pi ) ^ { - d } \int _ { [ - T , T ] ^ { d } } \prod _ { j = 1 } ^ { d } \psi _ { j } ( t _ { j } ) \varphi ( t ) d t
\]
其中 $\psi _ { j } ( s ) = ( \exp ( - i s a _ { j } ) - \exp ( - i s b _ { j } ) ) / i s$.

    \end{thm}
    
    
    
    \begin{thm}
        [Properties-of-Characteristic-Functions]
        {特征函数的性质}
        [Properties of Characteristic Functions]
        [gpt-4.1]
        
所有特征函数 $\varphi(t)$ 具有如下性质:

(a) $\varphi(0) = 1$,

(b) $\varphi(-t) = \overline{\varphi(t)}$,

(c) $|\varphi(t)| = |E e^{i t X}| \leq E|e^{i t X}| = 1$,

(d) $|\varphi(t + h) - \varphi(t)| \leq E|e^{i h X} - 1|$, 因此 $\varphi(t)$ 在 $(-\infty, \infty)$ 上一致连续,

(e) $E e^{i t (a X + b)} = e^{i t b} \varphi(a t)$.

    \end{thm}
    
    
    
    \begin{prf}
        [Proof-of-Properties-of-Characteristic-Functions]
        {特征函数性质的证明}
        [Proof of Properties of Characteristic Functions]
        [gpt-4.1]
        
(a) 显然成立.

(b) 由
\[
\varphi(-t) = E ( \cos(-t X) + i \sin(-t X) ) = E ( \cos(t X) - i \sin(t X) )
\]
可得.

(c) 由练习 1.6.2,可知 $\varphi(x, y) = (x^{2} + y^{2})^{1/2}$ 是凸函数,故成立.

(d) 由
\[
| \varphi(t + h) - \varphi(t) | = | E ( e^{i (t + h) X} - e^{i t X} ) | \leq E | e^{i (t + h) X} - e^{i t X} | = E | e^{i h X} - 1 |
\]
可见,因有界收敛定理可得一致连续性.

(e) 由 $E e^{i t (a X + b)} = e^{i t b} E e^{i (a t) X} = e^{i t b} \varphi(a t)$ 即得.

    \end{prf}
    
    
    
    \begin{lma}
        [Metric-and-Convergence-Properties-of-Measure-Distance]
        {测度距离的度量与收敛性质}
        [Metric and Convergence Properties of Measure Distance]
        [gpt-4.1]
        (i) $d ( \mu , 
u ) = \| \mu - 
u \|$ 定义了在 $\mathbf{Z}$ 上概率测度的一个度量;(ii) $\| \mu_n - \mu \| \to 0$ 当且仅当对每个 $x \in \mathbf{Z}$,有 $\mu_n(x) \to \mu(x)$,而根据习题 3.2.11,这等价于 $\mu_n \Rightarrow \mu$.
    \end{lma}
    
    
    
    \begin{prf}
        [Proof-of-Metric-and-Convergence-Properties-of-Measure-Distance]
        {测度距离度量与收敛性质的证明}
        [Proof of Metric and Convergence Properties of Measure Distance]
        [gpt-4.1]
        (i) 显然,$d ( \mu , 
u ) = d ( 
u , \mu )$,且 $d ( \mu , 
u ) = 0$ 当且仅当 $\mu = 
u$.验证三角不等式可以注意到,实数上的三角不等式意味着
\[
| \mu ( x ) - 
u ( x ) | + | 
u ( x ) - \pi ( x ) | \geq | \mu ( x ) - \pi ( x ) |
\]
然后对 $x$ 求和即可.

(ii) 一方面是显然的.除非对每个 $x$ 有 $\mu_n(x) \to \mu(x)$,否则不能有 $\| \mu_n - \mu \| \to 0$.为证逆命题,若 $\mu_n(x) \to \mu(x)$,则
\[
\sum _ { x } | \mu_n ( x ) - \mu ( x ) | = 2 \sum _ { x } ( \mu ( x ) - \mu_n ( x ) ) ^ { + } \to 0
\]
由主控收敛定理可得.
    \end{prf}
    
    
    
    \begin{xmp}
        [An-Example-of-Distribution-Function-Convergence]
        {分布函数收敛性的一个例子}
        [An Example of Distribution Function Convergence]
        [gpt-4.1]
        
设随机变量 $X$ 的分布为 $F$.则 $X + 1/n$ 的分布为
\[
F_{n}(x) = P(X + 1/n \leq x) = F(x - 1/n)
\]
当 $n \to \infty$ 时,$F_{n}(x) \to F(x-)$,其中 $F(x-) = \operatorname*{lim}_{y \uparrow x} F(y)$,因此收敛只在连续点发生.

    \end{xmp}
    
    
    
    \begin{thm}
        [Limit-Theorem-on-Probability-Distribution-under-Centering-and-Scaling]
        {极限定理关于中心化和缩放的概率分布}
        [Limit Theorem on Probability Distribution under Centering and Scaling]
        [gpt-4.1]
        
在前述假设下,若 $x_{n} / \sqrt{n} \to x$ 且 $a < b$,则有
\[
\sqrt{n} P(S_{n} \in (x_{n} + a, x_{n} + b)) \to (b - a) n(x)
\]
其中 $S_n = X_1 + \cdots + X_n$,$n(x) = (2\pi\sigma^2)^{-1/2} \exp(-x^2/2\sigma^2)$.

    \end{thm}
    
    
    
    \begin{thm}
        [Renewal-Theorem]
        {更新定理}
        [Renewal Theorem]
        [gpt-4.1]
        如果 $F$ 是非算术的,且 $h$ 是直接Riemann可积函数,则当 $t \to \infty$ 时,

\[
H(t) \sim \frac{1}{\mu} \int_{0}^{\infty} h(s) ds
\]

其中 $H(t)$ 是更新方程的解,$\mu$ 是相关分布的均值.
    \end{thm}
    
    
    
    \begin{dfn}
        [Definition-of-Central-Order-Statistic]
        {中心顺序统计量的定义}
        [Definition of Central Order Statistic]
        [gpt-4.1]
        在区间 $(0,1)$ 上随机放置 $(2n+1)$ 个点,即每个点的位置是独立且服从均匀分布的.设 $V_{n+1}$ 为这 $(2n+1)$ 个点中第 $(n+1)$ 大的点,称 $V_{n+1}$ 为中心顺序统计量.
    \end{dfn}
    
    
    
    \begin{thm}
        [Characteristic-Function-of-the-Sum-of-Independent-Random-Variables]
        {独立随机变量之和的特征函数}
        [Characteristic Function of the Sum of Independent Random Variables]
        [gpt-4.1]
        
如果 $X_{1}$ 和 $X_{2}$ 相互独立,且具有特征函数 $\varphi_{1}$ 和 $\varphi_{2}$,则 $X_{1} + X_{2}$ 的特征函数为 $\varphi_{1}(t) \varphi_{2}(t)$.

    \end{thm}
    
    
    
    \begin{thm}
        [Convergence-of-Types-Theorem]
        {类型收敛定理}
        [Convergence of Types Theorem]
        [gpt-4.1]
        设 $W_n \Rightarrow W$,且存在常数 $\alpha_n > 0$、$\beta_n$,使得 $W_n' = \alpha_n W_n + \beta_n \Rightarrow W'$,其中 $W$ 和 $W'$ 非退化,则存在常数 $\alpha$ 和 $\beta$,使得 $\alpha_n \to \alpha$ 且 $\beta_n \to \beta$.
    \end{thm}
    
    
    
    \begin{dfn}
        [Definition-of-Extreme-Value-Distributions]
        {极值分布的定义}
        [Definition of Extreme Value Distributions]
        [gpt-4.1]
        极值分布定义如下:当极大值 $M_n$ 经过适当归一化后,其分布极限为如下形式时,称为极值分布:

(i) 若 $F(x) = 1 - x^{-\alpha}$,$x \ge 1$,$\alpha > 0$,则对任意 $y > 0$,
\[
P(M_n / n^{1/\alpha} \leq y) \to \exp(-y^{-\alpha})
\]

(ii) 若 $F(x) = 1 - |x|^{\beta}$,$-1 \leq x \leq 0$,$\beta > 0$,则对任意 $y < 0$,
\[
P(n^{1/\beta} M_n \leq y) \to \exp(-|y|^{\beta})
\]

(iii) 若 $F(x) = 1 - e^{-x}$,$x \ge 0$,则对所有 $y \in (-\infty, \infty)$,
\[
P(M_n - \log n \leq y) \to \exp(-e^{-y})
\]

上述极限所对应的分布称为极值分布.

    \end{dfn}
    
    
    
    \begin{thm}
        [Convergence-in-Distribution-Implies-Convergence-in-Probability]
        {分布收敛蕴含概率收敛}
        [Convergence in Distribution Implies Convergence in Probability]
        [gpt-4.1]
        设 $X_1, X_2, \dots$ 是相互独立的随机变量.若 $S_n = \sum_{m \leq n} X_m$ 在分布意义下收敛,则它在概率意义下收敛(因此也几乎处处收敛).
    \end{thm}
    
    
    
    \begin{dfn}
        [Definition-of-Conditional-Expectation]
        {条件期望的定义}
        [Definition of Conditional Expectation]
        [gpt-4.1]
        
设 $(\Omega, \mathcal{F}_o, P)$ 是概率空间,$\mathcal{F} \subset \mathcal{F}_o$ 是一个 $\sigma$-域,$X \in \mathcal{F}_o$ 是满足 $E|X| < \infty$ 的随机变量.称随机变量 $Y$ 为 $X$ 关于 $\mathcal{F}$ 的条件期望,记为 $E(X | \mathcal{F})$,如果满足:

(i) $Y \in \mathcal{F}$,即 $Y$ 是 $\mathcal{F}$-可测的;

(ii) 对于所有 $A \in \mathcal{F}$,有 $\int_A X dP = \int_A Y dP$.

任何满足 (i) 和 (ii) 的 $Y$ 都称为 $E(X | \mathcal{F})$ 的一个版本.

    \end{dfn}
    
    
    
    \begin{thm}
        [Probability-Density-Function-of-Arrival-Time]
        {到达时间的概率密度函数表达式}
        [Probability Density Function of Arrival Time]
        [gpt-4.1]
        
设 $T_n$ 表示第 $n$ 次事件发生的时间,若事件到达过程的速率为 $\lambda$,则 $T_n$ 的概率密度函数为
\[
f_{T_{n}}(s) = \frac{\lambda^{n} s^{n-1}}{(n-1)!} e^{-\lambda s} \quad \mathrm{for}\ s \geq 0
\]
即 $T_n$ 的分布密度如上所示.

    \end{thm}
    
    
    
    \begin{thm}
        [Counting-Distribution-of-Poisson-Process]
        {Poisson过程的计数分布}
        [Counting Distribution of Poisson Process]
        [gpt-4.1]
        
设 $N_t$ 表示在时间区间 $[0, t]$ 内事件发生的次数,若该过程为参数为 $\lambda$ 的Poisson过程,则有
\[
P(N_{t} = 0) = e^{-\lambda t}
\]
对于 $n \geq 1$,
\[
P(N_{t} = n) = e^{-\lambda t} \frac{(\lambda t)^{n}}{n!}
\]
即 $N_t$ 服从均值为 $\lambda t$ 的Poisson分布.

    \end{thm}
    
    
    
    \begin{thm}
        [Central-Limit-Theorem-Finite-Variance-Case]
        {中心极限定理(方差有限情形)}
        [Central Limit Theorem (Finite Variance Case)]
        [gpt-4.1]
        
设 $X_1, X_2, \dots$ 是独立同分布的随机变量,$S_n = X_1 + \cdots + X_n$.如果 $E X_i = \mu$ 且 $\operatorname{var}(X_i) = \sigma^2 \in (0, \infty)$,则有
\[
(S_n - n\mu) / (\sigma n^{1/2}) \Rightarrow \chi
\]
其中 $\chi$ 表示标准正态分布(或极限分布).

    \end{thm}
    
    
    
    \begin{dfn}
        [Definition-of-Stopping-Time]
        {停时的定义}
        [Definition of Stopping Time]
        [gpt-4.1]
        
随机变量 $N$ 称为停时(stopping time),如果对所有 $n < \infty$,都有 $\{N = n\} \in \mathcal{F}_n$,即在时刻 $n$ 停止的决策必须可以由在该时刻已知的信息度量.

    \end{dfn}
    
    
    
    \begin{prf}
        [Proof-of-$T-n/n-	o-\mu$]
        {关于 $T\_n/n 	o \mu$ 的证明}
        [Proof of $T_n/n 	o \mu$]
        [gpt-4.1]
        设 $Y_{k} = X_{k} \boldsymbol{1}_{(|X_{k}| \leq k)}$ 且 $T_{n} = Y_{1} + \cdots + Y_{n}$.令 $Z_{k} = Y_{k} - E Y_{k}$,则 $E Z_{k} = 0$.
由于 $\operatorname{var}(Z_{k}) = \operatorname{var}(Y_{k}) \leq E Y_{k}^{2}$,并且
\[
\sum_{k=1}^{\infty} \operatorname{var}(Z_{k}) / k^{2} \leq \sum_{k=1}^{\infty} E Y_{k}^{2} / k^{2} < \infty
\]
应用定理 2.5.6,可知 $\sum_{k=1}^{\infty} Z_{k} / k$ 收敛 a.s.,从而定理 2.5.9 推出
\[
n^{-1} \sum_{k=1}^{n} (Y_{k} - E Y_{k}) \to 0 \quad \mathrm{and~hence} \quad \frac{T_{n}}{n} - n^{-1} \sum_{k=1}^{n} E Y_{k} \to 0 \quad \mathrm{a.s.}
\]
由主控收敛定理,$E Y_{k} \to \mu$ 当 $k \to \infty$,从而 $n^{-1} \sum_{k=1}^{n} E Y_{k} \to \mu$,因此 $T_{n} / n \to \mu$.

    \end{prf}
    
    
    
    \begin{thm}
        [Erdős-Kac-Central-Limit-Theorem]
        {Erdős-Kac中心极限定理}
        [Erdős-Kac Central Limit Theorem]
        [gpt-4.1]
        
As $n \to \infty$

\[
P_n(m \leq n : g(m) - \log \log n \leq x (\log \log n)^{1/2}) \to P(\chi \leq x)
\]

    \end{thm}
    
    
    
    \begin{dfn}
        [Definition-of-Weak-Convergence-of-Distribution-Functions]
        {分布函数弱收敛的定义}
        [Definition of Weak Convergence of Distribution Functions]
        [gpt-4.1]
        如果 $F_n$ 和 $F$ 是 $\mathbf{R}^d$ 上的分布函数,我们称 $F_n$ 弱收敛于 $F$,记作 $F_n \Rightarrow F$,当且仅当对于 $F$ 的所有连续点 $x$,有 $F_n(x) \to F(x)$.
    \end{dfn}
    
    
    
    \begin{dfn}
        [Weak-Convergence-of-Sequence-of-Distribution-Functions]
        {分布函数列的弱收敛}
        [Weak Convergence of Sequence of Distribution Functions]
        [gpt-4.1]
        若一列分布函数 $F_n$ 满足对 $F$ 的所有连续点 $y$,有 $F_n(y) \to F(y)$,则称 $F_n$ 弱收敛于 $F$(记作 $F_n \Rightarrow F$).
    \end{dfn}
    
    
    
    \begin{dfn}
        [Weak-Convergence-of-Sequence-of-Random-Variables]
        {随机变量序列的弱收敛}
        [Weak Convergence of Sequence of Random Variables]
        [gpt-4.1]
        若一列随机变量 $X_n$ 的分布函数 $F_n(x) = P(X_n \leq x)$ 弱收敛,则称 $X_n$ 弱收敛或依分布收敛于极限 $X_\infty$(记作 $X_n \Rightarrow X_\infty$).
    \end{dfn}
    
    
    
    \begin{dfn}
        [Definition-of-the-density-function-of-Polyas-distribution]
        {Polya分布的密度函数的定义}
        [Definition of the density function of Polya's distribution]
        [gpt-4.1]
        
设 $\delta > 0$,则
\[
h_{0}(y) = \frac{1}{\pi} \cdot \frac{1 - \cos \delta y}{\delta y^{2}}
\]
是Polya分布的密度函数,且令 $h_{\theta}(x) = e^{i\theta x} h_{0}(x)$.

    \end{dfn}
    
    
    
    \begin{dfn}
        [Definition-of-Fourier-transform]
        {Fourier变换的定义}
        [Definition of Fourier transform]
        [gpt-4.1]
        
对于函数 $g(y)$,其Fourier变换定义为
\[
\hat{g}(u) = \int e^{iu y} g(y) dy
\]

    \end{dfn}
    
    
    
    \begin{ppt}
        [Property-of-Fourier-transform-of-Polyas-density]
        {Polya分布密度Fourier变换的性质}
        [Property of Fourier transform of Polya's density]
        [gpt-4.1]
        
根据例3.3.15,
\[
\hat{h}_{0}(u) = 
\begin{cases}
1 - |u/\delta| & \text{当 } |u| \leq \delta \\
0 & \text{否则}
\end{cases}
\]
且易得 $\hat{h}_{\theta}(u) = \hat{h}_{0}(u+\theta)$.

    \end{ppt}
    
    
    
    \begin{thm}
        [Completeness-Theorem-for-Random-Variables]
        {随机变量完备性定理}
        [Completeness Theorem for Random Variables]
        [gpt-4.1]
        若在先前习题中定义的度量下,若 $d(X_m, X_n) \to 0$ 当 $m, n \to \infty$,则存在随机变量 $X_\infty$,使得 $X_n \to X_\infty$ 在概率意义下成立.
    \end{thm}
    
    
    
    \begin{dfn}
        [Definition-of-Alternating-Renewal-Process]
        {交替更新过程的定义}
        [Definition of Alternating Renewal Process]
        [gpt-4.1]
        设 $\xi_{1}, \xi_{2}, \dots > 0$ 是服从分布 $F_{1}$ 的独立同分布随机变量,$\eta_{1}, \eta_{2}, \ldots > 0$ 是服从分布 $F_{2}$ 的独立同分布随机变量.令 $T_{0} = 0$,并对 $k \geq 1$,设 $S_{k} = T_{k - 1} + \xi_{k}$,$T_{k} = S_{k} + \eta_{k}$.该过程描述了一台机器工作 $\xi_{k}$ 单位时间后坏掉,随后需花费 $\eta_{k}$ 单位时间修理.
    \end{dfn}
    
    
    
    \begin{thm}
        [Limit-of-Working-Probability-in-Alternating-Renewal-Process]
        {交替更新过程中机器工作概率的极限}
        [Limit of Working Probability in Alternating Renewal Process]
        [gpt-4.1]
        设 $F = F_{1} * F_{2}$,$H(t)$ 为在时刻 $t$ 机器处于工作状态的概率.若 $F$ 是非算术型分布,则当 $t \to \infty$ 时,
\[
H(t) \to \mu_{1} / (\mu_{1} + \mu_{2})
\]
其中 $\mu_{i}$ 是 $F_{i}$ 的均值.
    \end{thm}
    
    
    
    \begin{thm}
        [Criterion-for-Independence-in-Multivariate-Normal-Distribution]
        {多元正态分布变量独立性的判据}
        [Criterion for Independence in Multivariate Normal Distribution]
        [gpt-4.1]
        
设 $(X_1, \ldots, X_d)$ 服从均值向量 $\theta$ 和协方差矩阵 $\Gamma$ 的多元正态分布,则 $X_1, \ldots, X_d$ 独立当且仅当对任意 $i 
eq j$,有 $\Gamma_{ij} = 0$,即协方差矩阵的非对角元均为零.

    \end{thm}
    
    
    
    \begin{dfn}
        [Integral-Equation-Definition-of-Function-$Ut$]
        {函数 $U(t)$ 的积分方程定义}
        [Integral Equation Definition of Function $U(t)$]
        [gpt-4.1]
        函数 $U(t)$ 满足如下积分方程:
\[
U(t) = 1 + \int_0^t U(t-s)\,dF(s)
\]
或,用卷积符号表示为:
\[
U = 1_{[0,\infty)}(t) + U * F
\]
其中 $F$ 为概率分布函数,$*$ 表示卷积运算.
    \end{dfn}
    
    
    
    \begin{dfn}
        [Convolution-Equation-Definition-of-Function-$Vt$]
        {函数 $V(t)$ 的卷积方程定义}
        [Convolution Equation Definition of Function $V(t)$]
        [gpt-4.1]
        设 $G$ 为另一函数,则 $V = G * U$ 满足如下卷积方程:
\[
V = G * U = G + V * F
\]
其中 $G * U$ 表示 $G$ 与 $U$ 的卷积,$F$ 为概率分布函数.
    \end{dfn}
    
    
    
    \begin{xmp}
        [Example-of-Weak-Convergence-Without-Pointwise-Density-Convergence]
        {弱收敛但密度不逐点收敛的例子}
        [Example of Weak Convergence Without Pointwise Density Convergence]
        [gpt-4.1]
        
给出一组具有密度 $f_n$ 的随机变量 $X_n$,使得 $X_n \Rightarrow$ 在 $(0,1)$ 上的均匀分布,但对任意 $x \in [0, 1]$,$f_n(x)$ 不收敛于 $1$.

    \end{xmp}
    
    
    
    \begin{dfn}
        [Definition-of-Maximum-Random-Variable]
        {最大值随机变量的定义}
        [Definition of Maximum Random Variable]
        [gpt-4.1]
        
设 $X_1,X_2,\dots$ 是独立的,分布函数为 $F$ 的随机变量,定义 $M_n = \max_{m \leq n} X_m$.

    \end{dfn}
    
    
    
    \begin{ppt}
        [Property-of-the-Distribution-of-Maximum]
        {最大值的分布性质}
        [Property of the Distribution of Maximum]
        [gpt-4.1]
        
有 $P(M_n \leq x) = F(x)^n$.

    \end{ppt}
    
    
    
    \begin{dfn}
        [Definition-of-Measures-and-Probability-Measures]
        {测度和概率测度的定义}
        [Definition of Measures and Probability Measures]
        [gpt-4.1]
        设
\[
\mu_n(A) = \sqrt{n}P(S_n - x_n \in A), \quad \text{and} \quad \mu(A) = n(x)|A|
\]
其中 $|A|$ 表示集合 $A$ 的勒贝格测度.

定义概率测度为
\[
u_n(B) = \frac{1}{\alpha_n}\int_B h_0(y)\mu_n(dy), \quad \text{and} \quad 
u(B) = \frac{1}{\alpha}\int_B h_0(y)\mu(dy)
\]

    \end{dfn}
    
    
    
    \begin{dfn}
        [Definition-of-Normalization-Factors]
        {规范化因子的定义}
        [Definition of Normalization Factors]
        [gpt-4.1]
        设
\[
\alpha_n = \sqrt{n}E h_0(S_n - x_n) \quad \text{and} \quad \alpha = n(x) \int h_0(y)dy = n(x)
\]

    \end{dfn}
    
    
    
    \begin{thm}
        [Theorem-of-Convergence-of-Probability-Measures]
        {概率测度收敛定理}
        [Theorem of Convergence of Probability Measures]
        [gpt-4.1]
        取 $\theta=0$ 得 $\alpha_n \to \alpha$,由(a)可知
\[
\int e^{i\theta y}
u_n(dy) \to \int e^{i\theta y}
u(dy)
\]
由于这对所有 $\theta$ 都成立,由定理 3.17 可知 $
u_n \Rightarrow 
u$.

    \end{thm}
    
    
    
    \begin{thm}
        [Uniform-Convergence-Theorem-for-Continuous-Limit-Distribution]
        {连续极限分布一致收敛定理}
        [Uniform Convergence Theorem for Continuous Limit Distribution]
        [gpt-4.1]
        如果 $F_n \Rightarrow F$ 且 $F$ 是连续的,则有 $\sup_x |F_n(x) - F(x)| \to 0$.
    \end{thm}
    
    
    
    \begin{thm}
        [Integration-Exchange-Theorem-for-Conditional-Expectation]
        {条件期望积分交换定理}
        [Integration Exchange Theorem for Conditional Expectation]
        [gpt-4.1]
        
对于给定的随机变量 $X$ 和 $Y$,以及 $\sigma$-代数 $\mathcal{F}$,对于任意 $A \in \mathcal{F}$,有
\[
\int_{A} X E(Y|\mathcal{F}) dP = \int_{A} X Y dP
\]

    \end{thm}
    
    
    
    \begin{lma}
        [Lemma-on-Interchanging-Limit-and-Integral-in-Limit-Theorems]
        {关于极限定理中极限与积分交换的引理}
        [Lemma on Interchanging Limit and Integral in Limit Theorems]
        [gpt-4.1]
        
为证明 (3.8.8) 并合理交换极限与积分,现有如下引理(取 $\delta < 2 - \alpha$):

Lemma 3.

    \end{lma}
    
    
    
    \begin{thm}
        [Theorem-on-Distribution-of-Linear-Combinations-of-Independent-Stable-Variables]
        {独立稳定分布变量的线性组合分布定理}
        [Theorem on Distribution of Linear Combinations of Independent Stable Variables]
        [gpt-4.1]
        
设 $Y, Y_{1}, Y_{2}, \dots$ 是独立的,并且都服从指标为 $\alpha$ 的稳定分布.则存在常数 $\alpha_{k}$ 和 $\beta_{k}$,使得 $Y_{1} + \cdots + Y_{k}$ 与 $\alpha_{k} Y + \beta_{k}$ 有相同的分布.具体地,
(i) $\alpha_{k} = k^{1/\alpha}$,
(ii) 当 $\alpha < 1$ 时,$\beta_{k} = 0$.

    \end{thm}
    
    
    
    \begin{cxmp}
        [Counterexample-for-Near-Optimal-Integral-Bound-on-Nonnegative-Random-Variables]
        {非负随机变量上积分界的近似最优反例}
        [Counterexample for Near-Optimal Integral Bound on Nonnegative Random Variables]
        [gpt-4.1]
        
设 $\lambda \in ( 0 , 1 / 2 ), \beta = \tan ( \lambda \pi ), -1 \leq a \leq 1$,定义
\[
f_{a}(x) = c_{\lambda} \exp(-x^{\lambda}) (1 + a \sin(\beta x^{\lambda})) \quad \text{对于}~ x \geq 0
\]
该例说明对于非负随机变量,上述积分界结果已经非常接近最优.

    \end{cxmp}
    
    
    
    \begin{dfn}
        [Definition-of-Cantor-Distribution]
        {康托分布的定义}
        [Definition of Cantor Distribution]
        [gpt-4.1]
        设 $X_1, X_2, \dots$ 是相互独立的随机变量,每个取值 $0$ 或 $1$,概率均为 $1/2$.定义随机变量
\[
X = 2 \sum_{j \geq 1} X_j / 3^j
\]
则 $X$ 服从康托分布.
    \end{dfn}
    
    
    
    \begin{thm}
        [Continuity-and-Formula-for-Higher-Derivatives-of-Characteristic-Functions]
        {特征函数高阶导数的连续性与表达式}
        [Continuity and Formula for Higher Derivatives of Characteristic Functions]
        [gpt-4.1]
        
如果 $\int |x|^n \mu(dx) < \infty$,则其特征函数 $\varphi$ 存在连续的 $n$ 阶导数,并且有
\[
\varphi^{(n)}(t) = \int (ix)^n e^{itx} \mu(dx).
\]

    \end{thm}
    
    
    
    \begin{prf}
        [Proof-of-Finiteness-of-$\log\varphi	heta-+$]
        {关于 $\log(\varphi(	heta\_+))$ 有限性的证明}
        [Proof of Finiteness of $\log(\varphi(	heta_+))$]
        [gpt-4.1]
        由于 $(\log \varphi(\theta))' = \varphi'(\theta)/\varphi(\theta)$,对其从 $0$ 到 $\theta_+$ 积分可得 $\log(\varphi(\theta_+)) < \infty$.
    \end{prf}
    
    
    
    \begin{dfn}
        [Definition-of-Conditional-Variance]
        {条件方差的定义}
        [Definition of Conditional Variance]
        [gpt-4.1]
        
令 $\operatorname{var} ( X | \mathcal F ) = E ( X ^ { 2 } | \mathcal F ) - E ( X | \mathcal F ) ^ { 2 }$.

    \end{dfn}
    
    
    
    \begin{thm}
        [Law-of-Total-Variance]
        {全方差分解公式}
        [Law of Total Variance]
        [gpt-4.1]
        
证明如下公式:
\[
\operatorname { var }( X ) = E ( \operatorname { var }( X | \mathcal { F } ) ) + \operatorname { var }( E ( X | \mathcal { F } ) )
\]

    \end{thm}
    
    
    
    \begin{thm}
        [Equivalent-Conditions-for-the-Convergence-of-Independent-Nonnegative-Random-Variables]
        {独立非负随机变量收敛性的等价条件}
        [Equivalent Conditions for the Convergence of Independent Nonnegative Random Variables]
        [gpt-4.1]
        
设 $X_n \geq 0$ 为独立随机变量 ($n \geq 1$).下列条件等价:
\[
\sum_{n = 1}^{\infty} X_n < \infty
\]
当且仅当
\[
\sum_{n = 1}^{\infty} [ P(X_n > 1) + E(X_n 1_{(X_n \leq 1)}) ] < \infty
\]
以及
\[
\sum_{n = 1}^{\infty} E \left( \frac{X_n}{1 + X_n} \right) < \infty
\]

    \end{thm}
    
    
    
    \begin{thm}
        [Application-of-the-Nonnegative-Martingale-Convergence-Theorem]
        {非负鞅收敛定理的应用}
        [Application of the Nonnegative Martingale Convergence Theorem]
        [gpt-4.1]
        若 $X_{n}$ 是非负鞅,则由定理 4.2.12 可知 $X_{n} \to X$ 在 $
u$-a.s. 意义下成立.
    \end{thm}
    
    
    
    \begin{thm}
        [Unnecessary-Centering-in-Theorem-3.8.2-When-$\alpha-<-1$]
        {当 $\alpha < 1$ 时定理 3.8.2 可去中心化}
        [Unnecessary Centering in Theorem 3.8.2 When $\alpha < 1$]
        [gpt-4.1]
        设定理 3.8.2 需要对某些变量进行中心化.证明:当 $\alpha < 1$ 时,中心化是不必要的,即可以取 $b_n = 0$.
    \end{thm}
    
    
    
    \begin{ppt}
        [Integration-by-Parts-Formula]
        {分部积分公式}
        [Integration by Parts Formula]
        [gpt-4.1]
        
分部积分公式为
\[
\int_0^t K(y)\,dH(y) = H(t)K(t) - H(0)K(0) - \int_0^t H(y)\,dK(y)
\]

    \end{ppt}
    
    
    
    \begin{ppt}
        [Integral-Expression-for-Probability-Distribution]
        {概率分布的积分表达式}
        [Integral Expression for Probability Distribution]
        [gpt-4.1]
        
令 $H(y) = (y-t)/\mu$ 且 $K(y) = 1 - F(y)$, 则有
\[
\frac{1}{\mu} \int_0^t (1-F(y))\,dy = \frac{t}{\mu} - \int_0^t \frac{t-y}{\mu}\,dF(y)
\]
因此有
\[
G(t) = \frac{1}{\mu} \int_0^t (1-F(y))\,dy
\]
并且 $\mu = \int_{[0,\infty)} (1-F(y))\,dy$,所以上述公式定义了一个概率分布.

    \end{ppt}
    
    
    
    \begin{dfn}
        [Definition-of-Random-Variables-$A-{t}$-and-$B-{t}$]
        {随机变量 $A\_{t}$ 和 $B\_{t}$ 的定义}
        [Definition of Random Variables $A_{t}$ and $B_{t}$]
        [gpt-4.1]
        设 $A_{t} = t - T_{N(t) - 1}$,$B_{t} = T_{N(t)} - t$.
    \end{dfn}
    
    
    
    \begin{thm}
        [Limit-of-Joint-Probability-of-$A-{t}$-and-$B-{t}$]
        {关于 $A\_{t}$ 和 $B\_{t}$ 的联合概率极限}
        [Limit of Joint Probability of $A_{t}$ and $B_{t}$]
        [gpt-4.1]
        证明如下极限成立:
\[
P(A_{t} > x, B_{t} > y) \to \frac{1}{\mu} \int_{x + y}^{\infty} (1 - F(t))\, dt
\]

    \end{thm}
    
    
    
    \begin{thm}
        [Limit-Theorem-for-Renewal-Process-with-Directly-Riemann-Integrable-Density]
        {具有直接Riemann可积密度的更新过程的极限定理}
        [Limit Theorem for Renewal Process with Directly Riemann Integrable Density]
        [gpt-4.1]
        设 $F(t)$ 有一个直接Riemann可积的密度函数 $f(t)$,则 $V=U-1_{[0,\infty)}$ 有一个密度 $
u$,满足
\[
u(t) = f(t) + \int_{0}^{t} 
u(t - s) dF(s)
\]
此外,如果 $f$ 是直接Riemann可积的,则 $
u(t) \to 1/\mu$ 当 $t \to \infty$,其中 $\mu$ 是分布 $F$ 的均值.
    \end{thm}
    
    
    
    \begin{thm}
        [Generalization-Supremum-of-Submartingales-is-Still-a-Submartingale]
        {子鞅的上确界仍为子鞅的推广}
        [Generalization: Supremum of Submartingales is Still a Submartingale]
        [gpt-4.1]
        
设 $X_n$ 和 $Y_n$ 是关于滤子 ${\mathcal{F}}_n$ 的子鞅,则 $X_n \lor Y_n$ 也是关于 ${\mathcal{F}}_n$ 的子鞅.

    \end{thm}
    
    
    
    \begin{dfn}
        [Definition-of-Renewal-Process]
        {更新过程的定义}
        [Definition of Renewal Process]
        [gpt-4.1]
        设 $\xi_1, \xi_2, \ldots$ 是独立同分布的正随机变量(即 $P(\xi_i > 0) = 1$),分布为 $F$,定义一列时间序列 $T_0 = 0$,对于 $k \geq 1$ 有 $T_k = T_{k-1} + \xi_k$.$T_k$ 被称为第 $k$ 次更新发生的时间,其中'更新'指过程在 $T_k$ 时刻重新开始,即 $\{T_{k+j} - T_k, j \ge 1\}$ 与 $\{T_j, j \ge 1\}$ 具有相同分布.

    \end{dfn}
    
    
    
    \begin{dfn}
        [Definition-of-Number-of-Renewals]
        {更新次数的定义}
        [Definition of Number of Renewals]
        [gpt-4.1]
        令 $N_t = \inf\{k : T_k > t\}$,则 $N_t$ 表示时间区间 $[0, t]$ 内发生的更新次数(包括初始时刻 0 的更新).$N_t$ 是一个停时,即事件 $\{N_t = k\}$ 关于 $\mathcal{F}_k$ 是可测的.

    \end{dfn}
    
    
    
    \begin{thm}
        [Representation-and-Continuity-of-the-Density-Function-of-a-Distribution]
        {分布的密度函数表示及其连续性}
        [Representation and Continuity of the Density Function of a Distribution]
        [gpt-4.1]
        
设 $\mu$ 为一个分布,$\varphi(t)$ 为其特征函数,则
\[
\mu(a, b) + \frac{1}{2} \mu(\{a, b\}) = \frac{1}{2\pi} \int_{-\infty}^{\infty} \frac{e^{-ita} - e^{-itb}}{it} \varphi(t) dt
\]
且有估计
\[
\left| \frac{e^{-ita} - e^{-itb}}{it} \right| = \left| \int_{a}^{b} e^{-ity} dy \right| \leq |b-a|
\]
因此积分绝对收敛,$\mu$ 没有点质量.进一步,$\mu$ 在区间 $(x, x+h)$ 的取值为
\[
\mu(x, x+h) = \frac{1}{2\pi} \int \frac{e^{-itx} - e^{-it(x+h)}}{it} \varphi(t) dt = \int_{x}^{x+h} \left( \frac{1}{2\pi} \int e^{-ity} \varphi(t) dt \right) dy
\]
由 Fubini 定理,$\mu$ 的密度函数为
\[
f(y) = \frac{1}{2\pi} \int e^{-ity} \varphi(t) dt
\]
由主导收敛定理,$f$ 是连续函数.

    \end{thm}
    
    
    
    \begin{thm}
        [Theorem-on-the-Distribution-of-Counts-in-a-Poisson-Process]
        {Poisson过程计数分布定理}
        [Theorem on the Distribution of Counts in a Poisson Process]
        [gpt-4.1]
        如果条件 (i)–(iv) 满足,则 $N(0, t)$ 服从均值为 $\lambda t$ 的 Poisson 分布.
    \end{thm}
    
    
    
    \begin{prf}
        [Proof-of-the-Theorem-on-the-Distribution-of-Counts-in-a-Poisson-Process]
        {Poisson过程计数分布定理的证明}
        [Proof of the Theorem on the Distribution of Counts in a Poisson Process]
        [gpt-4.1]
        设 $X_{n, m} = N((m-1)t/n, mt/n)$,其中 $1 \leq m \leq n$,然后应用 Theorem 3.
    \end{prf}
    
    
    
    \begin{dfn}
        [Definition-of-Ky-Fan-Metric]
        {Ky Fan度量的定义}
        [Definition of Ky Fan Metric]
        [gpt-4.1]
        Ky Fan度量定义如下:对于随机变量$X$和$Y$,
\[
\alpha(X, Y) = \inf\{ \epsilon \geq 0 : P(|X - Y| > \epsilon) \leq \epsilon \}
\]

    \end{dfn}
    
    
    
    \begin{thm}
        [Relation-between-Ky-Fan-Metric-and-Levy-Distance]
        {Ky Fan度量与Levy距离的关系}
        [Relation between Ky Fan Metric and Levy Distance]
        [gpt-4.1]
        若$\alpha(X, Y) = \alpha$,则对应分布的Levy距离$\rho(F, G) \leq \alpha$.
    \end{thm}
    
    
    
    \begin{thm}
        [Doobs-Decomposition-Theorem]
        {Doob 分解定理}
        [Doob's Decomposition Theorem]
        [gpt-4.1]
        任何次鞅 $X_{n}$, $n \geq 0$, 都可以唯一地表示为 $X_{n} = M_{n} + A_{n}$,其中 $M_{n}$ 是鞅,$A_{n}$ 是可预测的递增序列且 $A_{0} = 0$.
    \end{thm}
    
    
    
    \begin{thm}
        [Inversion-Formula-Theorem-under-Absolutely-Integrable-Condition]
        {积分绝对收敛时的反演公式定理}
        [Inversion Formula Theorem under Absolutely Integrable Condition]
        [gpt-4.1]
        
如果 $\int |\varphi(t)| dt < \infty$,则测度 $\mu$ 有有界连续的密度
\[
f(y) = \frac{1}{2\pi} \int e^{-ity} \varphi(t) dt
\]

    \end{thm}
    
    
    
    \begin{thm}
        [Random-Index-Central-Limit-Theorem]
        {随机索引中心极限定理}
        [Random Index Central Limit Theorem]
        [gpt-4.1]
        
设 $X_1, X_2, \dots$ 是一组独立同分布的随机变量,满足 $E X_i = 0$ 且 $E X_i^2 = \sigma^2 \in (0, \infty)$,并令 $S_n = X_1 + \cdots + X_n$.令 $N_n$ 为一列非负整数值随机变量,$a_n$ 为一列整数,且 $a_n \to \infty$ 并且 $N_n / a_n \to 1$ 依概率收敛.

则有
\[
S_{N_n} / (\sigma \sqrt{a_n}) \Rightarrow \chi
\]

    \end{thm}
    
    
    
    \begin{dfn}
        [Definition-of-the-Levy-Metric]
        {Levy度量的定义}
        [Definition of the Levy Metric]
        [gpt-4.1]
        
设$F$和$G$为分布函数,定义Levy度量$\rho(F, G)$如下:
\[
\rho(F, G) = \inf\{ \epsilon : F(x - \epsilon) - \epsilon \leq G(x) \leq F(x + \epsilon) + \epsilon \text{ for all } x \}
\]

    \end{dfn}
    
    
    
    \begin{thm}
        [Properties-of-the-Levy-Metric-and-Equivalence-to-Weak-Convergence]
        {Levy度量的性质与分布收敛的等价性}
        [Properties of the Levy Metric and Equivalence to Weak Convergence]
        [gpt-4.1]
        
$\rho(F, G)$定义了分布空间上的一个度量,且$\rho(F_n, F) \to 0$当且仅当$F_n \Rightarrow F$.

    \end{thm}
    
    
    
    \begin{thm}
        [Asymptotic-Formula-for-the-Probability-of-First-Return-to-Zero-in-Random-Walk]
        {随机游走首次回归零点的概率渐近公式}
        [Asymptotic Formula for the Probability of First Return to Zero in Random Walk]
        [gpt-4.1]
        
设 $X_1, X_2, \dots$ 是独立同分布的随机变量,满足 $P(X_i = 1) = P(X_i = -1) = 1/2$,令 $S_n = X_1 + \cdots + X_n$,定义 $\tau = \operatorname{inf} \{ n \geq 1 : S_n = 0 \}$,则
\[
P(\tau > 2n) \sim \pi^{-1/2} n^{-1/2} \quad \text{当 } n \to \infty
\]

    \end{thm}
    
    
    
    \begin{thm}
        [Convergence-of-Return-Times-to-Stable-Law]
        {归零时刻和稳定分布的收敛性}
        [Convergence of Return Times to Stable Law]
        [gpt-4.1]
        
令 $\tau_1, \tau_2, \dots$ 独立且同分布于 $\tau$,定义 $T_n = \tau_1 + \cdots + \tau_n$.$T_n$ 的分布与 $S_m$ 第 $n$ 次达到 $0$ 的时刻分布相同.则有 $T_n / n^2$ 收敛于参数为 $\alpha = 1/2$, $\kappa = 1$ 的稳定分布.

    \end{thm}
    
    
    
    \begin{thm}
        [Theorem-on-the-Limiting-Upper-Bound-for-Large-Deviations]
        {大数上界极限的定理}
        [Theorem on the Limiting Upper Bound for Large Deviations]
        [gpt-4.1]
        假设除了 (H1) 和 (H2) 之外,还存在 $\theta_{a} \in (0, \theta_{+})$,使得 $a = \varphi'( \theta_{a} ) / \varphi( \theta_{a} )$.则当 $n \to \infty$ 时,
\[
n^{-1} \log P( S_{n} \geq n a ) \to -a \theta_{a} + \log \varphi( \theta_{a} )
\]

    \end{thm}
    
    
    
    \begin{prf}
        [Proof-of-the-Theorem-on-the-Limiting-Upper-Bound-for-Large-Deviations-Partial]
        {大数上界极限定理的证明(部分)}
        [Proof of the Theorem on the Limiting Upper Bound for Large Deviations (Partial)]
        [gpt-4.1]
        左边的上极限 $\leq$ 右边,依据 (2.7.2).
    \end{prf}
    
    
    
    \begin{thm}
        [Convergence-Criterion-for-Infinite-Products]
        {无穷乘积收敛性判别定理}
        [Convergence Criterion for Infinite Products]
        [gpt-4.1]
        
设 $y_{n} > -1$ 对所有 $n$ 成立,且 $\sum y_{n} < \infty$,则无穷乘积 $\prod_{m=1}^{\infty} (1 + y_{m})$ 收敛,即该乘积存在.

    \end{thm}
    
    
    
    \begin{thm}
        [Criterion-for-Absolute-Continuity-and-Mutual-Singularity-of-Measures]
        {测度绝对连续性与互相奇异性的判别定理}
        [Criterion for Absolute Continuity and Mutual Singularity of Measures]
        [gpt-4.1]
        
$\mu \ll 
u \text{ 或 } \mu \perp 
u$,当且仅当
\[
\prod_{m=1}^{\infty} \int \sqrt{q_m} \, dG_m > 0 \text{ 或 } = 0.
\]

    \end{thm}
    
    
    
    \begin{dfn}
        [Definition-of-the-Exceptional-Set]
        {例外集合的定义}
        [Definition of the Exceptional Set]
        [gpt-4.1]
        
设 $a_x = \sup \{ y : F(y) < x \}$, $b_x = \inf \{ y : F(y) > x \}$, 定义 $\Omega_0 = \{ x : (a_x, b_x) = \emptyset \}$,其中 $(a_x, b_x)$ 是以 $a_x$ 和 $b_x$ 为端点的开区间.

    \end{dfn}
    
    
    
    \begin{ppt}
        [Countability-of-the-Exceptional-Set]
        {例外集合的可数性}
        [Countability of the Exceptional Set]
        [gpt-4.1]
        
$\Omega - \Omega_0$ 是可数集,因为 $(a_x, b_x)$ 互不相交,且每个非空区间都包含一个不同的有理数.

    \end{ppt}
    
    
    
    \begin{thm}
        [Liminf-Property-of-Inverse-Function-Convergence]
        {逆函数收敛性的下极限性质}
        [Liminf Property of Inverse Function Convergence]
        [gpt-4.1]
        
对于 $x \in \Omega_0$,有
\[
\liminf_{n \to \infty} F_n^{-1}(x) \geq F^{-1}(x)
\]

    \end{thm}
    
    
    
    \begin{lma}
        [Lemma-on-Exponential-Limit-of-Powers]
        {极限幂形式的指数极限引理}
        [Lemma on Exponential Limit of Powers]
        [gpt-4.1]
        如果 $c_{j} \to 0$,$a_{j} \to \infty$ 且 $a_{j} c_{j} \to \lambda$,那么 $(1 + c_{j})^{a_{j}} \to e^{\lambda}$.
    \end{lma}
    
    
    
    \begin{prf}
        [Proof-of-Lemma-on-Exponential-Limit-of-Powers]
        {极限幂形式的指数极限引理的证明}
        [Proof of Lemma on Exponential Limit of Powers]
        [gpt-4.1]
        因为当 $x \to 0$ 时,$\log(1 + x)/x \to 1$,所以 $a_{j} \log(1 + c_{j}) \to \lambda$,由此得到所需结论.
    \end{prf}
    
    
    
    \begin{dfn}
        [Definition-of-Asymptotic-Equivalence-at-Infinity]
        {无穷极限下的渐近等价的定义}
        [Definition of Asymptotic Equivalence at Infinity]
        [gpt-4.1]
        $a_{n} \sim b_{n}$ 的意思是当 $n \to \infty$ 时,$a_{n}/b_{n} \to 1$.
    \end{dfn}
    
    
    
    \begin{thm}
        [Stirlings-Formula]
        {斯特林公式}
        [Stirling's Formula]
        [gpt-4.1]
        斯特林公式说明:
\[
n! \sim n^{n} e^{-n} \sqrt{2\pi n} \quad \text{ 当 } n \to \infty
\]

    \end{thm}
    
    
    
    \begin{thm}
        [Equivalent-Definition-of-Weak-Convergence]
        {弱收敛的等价定义}
        [Equivalent Definition of Weak Convergence]
        [gpt-4.1]
        $X_{n} \Rightarrow X_{\infty}$ 当且仅当对任意有界连续函数 $g$,都有 $E g(X_{n}) \to E g(X_{\infty})$.
    \end{thm}
    
    
    
    \begin{prf}
        [Proof-of-the-Equivalent-Definition-of-Weak-Convergence]
        {弱收敛等价定义的证明}
        [Proof of the Equivalent Definition of Weak Convergence]
        [gpt-4.1]
        设 $Y_{n}$ 与 $X_{n}$ 同分布且 $Y_{n}$ 收敛到 $Y_{\infty}$ 几乎处处.由于 $g$ 连续,$g(Y_{n}) \to g(Y_{\infty})$ 几乎处处.由有界收敛定理有
\[
E g(X_{n}) = E g(Y_{n}) \to E g(Y_{\infty}) = E g(X_{\infty})
\]

反过来,令
\[
g_{x, \epsilon}(y) = 
\begin{cases}
1 & y \leq x \\
0 & y \geq x + \epsilon \\
\text{linear} & x < y < x + \epsilon
\end{cases}
\]
由于 $g_{x, \epsilon}$ 连续且有界,
\[
\limsup_{n \to \infty} P(X_{n} \leq x) \leq \limsup_{n \to \infty} E g_{x, \epsilon}(X_{n}) = E g_{x, \epsilon}(X_{\infty}) \leq P(X_{\infty} \leq x + \epsilon)
\]
令 $\epsilon \to 0$ 得 $\limsup_{n \to \infty} P(X_{n} \leq x) \leq P(X_{\infty} \leq x)$.

同理,对于 $\liminf$,有
\[
\liminf_{n \to \infty} P(X_{n} \leq x) \geq \liminf_{n \to \infty} E g_{x - \epsilon, \epsilon}(X_{n}) = E g_{x - \epsilon, \epsilon}(X_{\infty}) \geq P(X_{\infty} \leq x - \epsilon)
\]
令 $\epsilon \to 0$ 得 $\liminf_{n \to \infty} P(X_{n} \leq x) \geq P(X_{\infty} < x) = P(X_{\infty} \leq x)$(当 $x$ 是连续点).由上两式,得到所需结论.
    \end{prf}
    
    
    
    \begin{dfn}
        [Definition-of-the-Sequence-$F-n$]
        {关于序列 $F\_n$ 的定义}
        [Definition of the Sequence $F_n$]
        [gpt-4.1]
        
设
\[
F_n = n^{-p} \sum_{m=1}^n \frac{\mathrm{sgn}(Y_m)}{|Y_m|^p}
\]
其中 $Y_m$ 为一组随机变量,$\mathrm{sgn}(Y_m)$ 表示符号函数.

    \end{dfn}
    
    
    
    \begin{dfn}
        [Definition-and-Distribution-Properties-of-$Z-m$]
        {$Z\_m$ 的定义与分布特性}
        [Definition and Distribution Properties of $Z_m$]
        [gpt-4.1]
        
令 $Z_m = \frac{\mathrm{sgn}(Y_m)}{|Y_m|^p}$,则 $Z_m$ 关于 0 对称,且
\[
P(|Z_m| > x) = P(|Y_m| < x^{-1/p})
\]

    \end{dfn}
    
    
    
    \begin{thm}
        [Conditions-of-Theorem-3.8.2-Hold]
        {定理 3.8.2 的条件成立}
        [Conditions of Theorem 3.8.2 Hold]
        [gpt-4.1]
        
对于上述定义的 $Z_m$,定理 3.8.2 中的条件 (i) 成立,取 $\theta = 1/2$,条件 (ii) 成立,取 $\alpha = 1/p$.

    \end{thm}
    
    
    
    \begin{dfn}
        [Definition-of-Scaling-and-Centering-Constants]
        {缩放与中心化常数的定义}
        [Definition of Scaling and Centering Constants]
        [gpt-4.1]
        
缩放常数 $a_n \sim C n^p$,中心化常数由于对称性为 0.

    \end{dfn}
    
    
    
    \begin{xmp}
        [Simple-Random-Walk-on-$\mathbf{Z}^d$]
        {d维整数格上的简单随机游走}
        [Simple Random Walk on $\mathbf{Z}^d$]
        [gpt-4.1]
        
设 $X_1, X_2, \ldots$ 是一列独立同分布的随机变量,取值为 $d$ 维单位向量 $e_1, \ldots, e_d$ 的正负方向.对于每个 $n$,有
\[
P(X_n = +e_i) = P(X_n = -e_i) = \frac{1}{2d} \quad \mathrm{对于}~i = 1, \ldots, d
\]
此外,$E X_n^i = 0$,若 $i 
eq j$,则 $E X_n^i X_n^j = 0$,因为两个分量不可能同时为非零.因此协方差矩阵为 $\Gamma_{ij} = (1/2d) I$.

    \end{xmp}
    
    
    
    \begin{thm}
        [Linearity-of-Conditional-Expectation]
        {条件期望的线性性质}
        [Linearity of Conditional Expectation]
        [gpt-4.1]
        
设 $X$ 和 $Y$ 为随机变量,$a$ 为常数,$\mathcal{F}$ 为一个子 $\sigma$-代数,则
\[
E(aX + Y | \mathcal{F}) = a E(X | \mathcal{F}) + E(Y | \mathcal{F})
\]
且 $a E(X | \mathcal{F}) + E(Y | \mathcal{F})$ 是 $\mathcal{F}$-可测的.

    \end{thm}
    
    
    
    \begin{thm}
        [Monotonicity-of-Conditional-Expectation]
        {条件期望的单调性}
        [Monotonicity of Conditional Expectation]
        [gpt-4.1]
        
若 $X \leq Y$ 几乎处处成立,则
\[
E(X | \mathcal{F}) \leq E(Y | \mathcal{F})
\]
几乎处处成立.

    \end{thm}
    
    
    
    \begin{thm}
        [Limit-Property-of-Conditional-Expectation]
        {条件期望的极限性质}
        [Limit Property of Conditional Expectation]
        [gpt-4.1]
        
设 $X_n$ 为随机变量序列,$Y_n = X - X_n$,若 $Y_n \downarrow 0$,则
\[
E(Y_n | \mathcal{F}) \downarrow 0
\]
几乎处处成立.

    \end{thm}
    
    
    
    \begin{xmp}
        [Example-of-Fixed-Points-in-Permutations]
        {关于排列不动点的例子}
        [Example of Fixed Points in Permutations]
        [gpt-4.1]
        设 $\pi$ 是 $\{1, 2, \ldots, n\}$ 的一个随机排列,定义 $X_{n, m} = 1$ 当且仅当 $m$ 是排列中的不动点(否则为 0),令 $S_n = X_{n, 1} + \cdots + X_{n, n}$ 表示排列中的不动点数.我们希望计算 $P(S_n = 0)$,即排列没有不动点的概率.
    \end{xmp}
    
    
    
    \begin{dfn}
        [Definition-of-Indicator-Variable-for-Fixed-Points-of-Permutation]
        {排列不动点指示变量的定义}
        [Definition of Indicator Variable for Fixed Points of Permutation]
        [gpt-4.1]
        对每个 $m$,定义事件 $A_{n, m} = \{ X_{n, m} = 1 \}$,即 $m$ 是排列 $\pi$ 的不动点的事件.
    \end{dfn}
    
    
    
    \begin{prf}
        [Proof-of-the-Law-of-Large-Numbers-Using-the-Dominated-Convergence-Theorem-and-Theorem-2.12]
        {利用主导收敛定理与定理2.12证明大数定律}
        [Proof of the Law of Large Numbers Using the Dominated Convergence Theorem and Theorem 2.12]
        [gpt-4.1]
        
两次应用主导收敛定理可得

\[
\begin{array}{rl}
& x P(|X_{1}| > x) \leq E(|X_{1}| 1_{(|X_{1}| > x)}) \to 0 \quad \mathrm{as~} x \to \infty \\
& \mu_{n} = E(X_{1} 1_{(|X_{1}| \leq n)}) \to E(X_{1}) = \mu \quad \mathrm{as~} n \to \infty
\end{array}
\]

利用定理2.12, 若 $\epsilon > 0$,则 $P(|S_{n}/n - \mu_{n}| > \epsilon/2) \to 0$.由于 $\mu_{n} \to \mu$,因此 $P(|S_{n}/n - \mu| > \epsilon) \to 0$.

    \end{prf}
    
    
    
    \begin{prf}
        [Proof-of-the-Limiting-Theorem]
        {极限定理的证明}
        [Proof of the Limiting Theorem]
        [gpt-4.1]
        
若 $r = n \log n + n x$,则
\[
\begin{array}{rcl}
&& P(T_n - n \log n \leq n x) \to \exp(- e^{-x}) \\
&& \\
&& \text{since then } n e^{- r / n} \to e^{-x}.
\end{array}
\]

    \end{prf}
    
    
    
    \begin{dfn}
        [Definition-of-Mean-Interarrival-Time-under-Truncation-for-Exponential-Distribution]
        {截断条件下的指数分布到达时间均值的定义}
        [Definition of Mean Interarrival Time under Truncation for Exponential Distribution]
        [gpt-4.1]
        
设 $\mu$ 为在到达时间小于 $1$ 的条件下的指数分布到达时间的均值,则有
\[
\mu = \int _ { 0 } ^ { 1 } x \lambda e ^ { - \lambda x } d x = \frac { 1 } { \lambda } - \left( 1 + \frac { 1 } { \lambda } \right) e ^ { - \lambda }
\]

    \end{dfn}
    
    
    
    \begin{thm}
        [Formula-for-Expected-Number-of-Renewals-in-Terminating-Renewal-Process]
        {终止更新过程中的期望更新次数公式}
        [Formula for Expected Number of Renewals in Terminating Renewal Process]
        [gpt-4.1]
        
对于终止的更新过程,期望的更新次数满足
\[
E M = \sum _ { n = 0 } ^ { \infty } e ^ { - \lambda } ( 1 - e ^ { - \lambda } ) ^ { n } n \mu = ( e ^ { \lambda } - 1 ) \mu
\]
其中 $\mu$ 为条件均值,$\lambda$ 为指数分布参数.

    \end{thm}
    
    
    
    \begin{thm}
        [Hellys-Selection-Theorem]
        {Helly选择定理}
        [Helly's Selection Theorem]
        [gpt-4.1]
        对于每一个分布函数序列 $F_{n}$,存在一个子序列 $F_{n(k)}$ 和一个右连续、非减的函数 $F$,使得
\[
\lim_{k \to \infty} F_{n(k)} ( y ) = F ( y )
\]
对于所有 $F$ 的连续点 $y$ 成立.
    \end{thm}
    
    
    
    \begin{xmp}
        [Example-Where-the-Limit-of-Distribution-Functions-is-Not-a-Distribution-Function]
        {分布函数序列极限不是分布函数的例子}
        [Example Where the Limit of Distribution Functions is Not a Distribution Function]
        [gpt-4.1]
        设 $a + b + c = 1$,且 $F_{n} ( x ) = a \cdot 1_{( x \geq n )} + b \cdot 1_{( x \geq - n )} + c \cdot G ( x )$,其中 $G$ 为分布函数,则 $F_{n} ( x ) \to F ( x ) = b + c G ( x )$,
\[
\lim_{x \downarrow -\infty} F ( x ) = b \quad \text{且} \quad \lim_{x \uparrow \infty} F ( x ) = b + c = 1 - a
\]
即有质量 $a$ 跑到 $+\infty$,质量 $b$ 跑到 $-\infty$.
    \end{xmp}
    
    
    
    \begin{dfn}
        [Definition-of-Vague-Convergence]
        {模糊收敛的定义}
        [Definition of Vague Convergence]
        [gpt-4.1]
        在定理 3.2.12 中出现的收敛类型有时称为模糊收敛(vague convergence),记作 $\Rightarrow_{
u}$.
    \end{dfn}
    
    
    
    \begin{thm}
        [Levys-Theorem-on-Convergence-to-the-Normal-Distribution]
        {Levy关于收敛到正态分布的定理}
        [Levy's Theorem on Convergence to the Normal Distribution]
        [gpt-4.1]
        设 $X_{1}, X_{2}, \dots$ 是独立同分布随机变量,$S_{n} = X_{1} + \cdots + X_{n}$.存在常数 $a_{n}$ 和 $b_{n} > 0$ 使得 $(S_{n} - a_{n}) / b_{n} \Rightarrow \chi$ 的充要条件是 ...(原文未给出具体条件,内容缺失).
    \end{thm}
    
    
    
    \begin{dfn}
        [Definition-of-Renewal-Measure]
        {更新测度的定义}
        [Definition of Renewal Measure]
        [gpt-4.1]
        设 $U(A) = \sum_{n=0}^{\infty} P(T_{n} \in A)$,则$U$称为更新测度(renewal measure).
    \end{dfn}
    
    
    
    \begin{thm}
        [Inversion-Formula-for-Characteristic-Function-of-Discrete-Random-Variables]
        {离散型概率变量的特征函数反演公式}
        [Inversion Formula for Characteristic Function of Discrete Random Variables]
        [gpt-4.1]
        
若 $P(X \in h \mathbf{Z}) = 1$,其中 $h > 0$,则其特征函数 $\varphi(t)$ 满足 $\varphi(2\pi/h + t) = \varphi(t)$,并且有
\[
P(X = x) = \frac{h}{2\pi} \int_{-\pi/h}^{\pi/h} e^{-i t x} \varphi(t) d t \quad \mathrm{for~} x \in h \mathbf{Z}
\]

    \end{thm}
    
    
    
    \begin{thm}
        [Property-of-Characteristic-Function-for-Shifted-Random-Variables]
        {平移随机变量的特征函数性质}
        [Property of Characteristic Function for Shifted Random Variables]
        [gpt-4.1]
        
若 $X = Y + b$,则有 $E \exp(i t X) = e^{i t b} E \exp(i t Y)$.

    \end{thm}
    
    
    
    \begin{thm}
        [Inversion-Formula-for-Characteristic-Function-of-Discrete-Random-Variables-with-Shift]
        {带偏移的离散概率变量特征函数反演公式}
        [Inversion Formula for Characteristic Function of Discrete Random Variables with Shift]
        [gpt-4.1]
        
如果 $P(X \in b + h \mathbf{Z}) = 1$,则 (ii) 中的反演公式对 $x \in b + h \mathbf{Z}$ 依然成立.

    \end{thm}
    
    
    
    \begin{prf}
        [Inductive-Proof-of-the-Integral-Expression-for-$F^n$]
        {关于 $F^n$ 的积分表达式的归纳证明}
        [Inductive Proof of the Integral Expression for $F^n$]
        [gpt-4.1]
        我们将通过归纳法证明:

当 $n = 1$ 时,结论成立.

对于 $n > 1$,有
\begin{array}{l}
F^{n} = F^{n-1} * F(z) = \int_{-\infty}^{\infty} dF^{n-1}(x) \int_{-\infty}^{z-x} dF(y) \\
= \int dF_{ \lambda }^{ n-1 }(x) \int dF_{ \lambda }(y) \; 1_{(x+y \le z)} e^{ -\lambda(x+y) } \varphi( \lambda )^{ n } \\
= E\left( 1_{( S_{ n-1 }^{ \lambda } + X_{ n }^{ \lambda } \le z )} e^{ -\lambda( S_{ n-1 }^{ \lambda } + X_{ n }^{ \lambda } ) } \varphi( \lambda )^{ n } \right) \\
= \int_{-\infty}^{z} dF_{ \lambda }^{ n }(u) e^{ -\lambda u } \varphi( \lambda )^{ n }
\end{array}
在最后两步中,我们使用了定理 1.
    \end{prf}
    
    
    
    \begin{lma}
        [Lower-Bound-Estimate-for-$P-S-{n}-\geq-n-a-$]
        {关于 $P( S\_{n} \geq n a )$ 的下界估计}
        [Lower Bound Estimate for $P( S_{n} \geq n a )$]
        [gpt-4.1]
        若 $
u > a$,则引理和单调性推出
\[
P( S_{n} \geq n a ) \geq \int_{ n a }^{ n 
u } e^{ -\lambda x } \varphi( \lambda )^{ n } dF_{ \lambda }^{ n }(x) \geq \varphi( \lambda )^{ n } e^{ -\lambda n 
u } ( F_{ \lambda }^{ n }( n 
u ) - F_{ \lambda }^{ n }( n a ) ).
\]
    \end{lma}
    
    
    
    \begin{thm}
        [Contraction-Property-of-Conditional-Expectation-in-$L^p$]
        {条件期望在 $L^p$ 空间中的收缩性}
        [Contraction Property of Conditional Expectation in $L^p$]
        [gpt-4.1]
        
条件期望在 $L^p$ 空间中是收缩映射.即对于 $p \geq 1$,有
\[
E ( | E ( X | { \mathcal { F } } ) | ^ { p } ) \leq E | X | ^ { p }
\]

    \end{thm}
    
    
    
    \begin{prf}
        [Proof-of-Contraction-Property-of-Conditional-Expectation]
        {条件期望收缩性的证明}
        [Proof of Contraction Property of Conditional Expectation]
        [gpt-4.1]
        
由 (4.1.4) 可知 $| E ( X | { \mathcal { F } } ) | ^ { p } \leq E ( | X | ^ { p } | { \mathcal { F } } )$.

两边取期望值得到
\[
E ( | E ( X | { \mathcal { F } } ) | ^ { p } ) \leq E ( E ( | X | ^ { p } | { \mathcal { F } } ) ) = E | X | ^ { p }
\]

最后一个等式利用了定义中的一个恒等式(对 $A = \Omega$ 使用定义中的性质(ii)即可).

    \end{prf}
    
    
    
    \begin{xmp}
        [An-Example-of-Net-Force-from-$n$-Independent-Random-Objects-in-$[-n-n]$]
        {关于 $n$ 个独立随机对象在区间 $[-n, n]$ 上的力的例子}
        [An Example of Net Force from $n$ Independent Random Objects in $[-n, n]$]
        [gpt-4.1]
        
假设 $n$ 个对象 $X_{n,1}, \ldots, X_{n,n}$ 独立且随机地放置在区间 $[-n, n]$ 上.

定义净作用于 $0$ 点的力为
\[
F_n = \sum_{m=1}^n \frac{\operatorname{sgn}(X_{n,m})}{|X_{n,m}|^p}
\]

现将证明:若 $p > 1/2$,则有
\[
\operatorname*{lim}_{n \to \infty} E \exp(it F_n) = \exp(-c |t|^{1/p})
\]

为此,取 $X_{n,m} = n Y_m$,其中 $Y_m$ 独立同分布于 $[-1, 1]$.

    \end{xmp}
    
    
    
    \begin{thm}
        [Properties-of-Conditional-Expectation]
        {条件期望的性质}
        [Properties of Conditional Expectation]
        [gpt-4.1]
        
设 $E |X|, E |Y| < \infty$.
(a) 条件期望是线性的:
\[
E(aX + Y | \mathcal{F}) = a E(X | \mathcal{F}) + E(Y | \mathcal{F})
\]

(b) 若 $X \leq Y$,则
\[
E(X | \mathcal{F}) \leq E(Y | \mathcal{F})
\]

(c) 若 $X_n \geq 0$ 且 $X_n \uparrow X$ 且 $E X < \infty$,则
\[
E(X_n | \mathcal{F}) \uparrow E(X | \mathcal{F})
\]

此外,应用上述结果于 $Y_1 - Y_n$,可得:若 $Y_n \downarrow Y$ 且 $E |Y_1|, E |Y| < \infty$,则 $E(Y_n | \mathcal{F}) \downarrow E(Y | \mathcal{F})$.

    \end{thm}
    
    
    
    \begin{thm}
        [Generalization-of-Central-Limit-Theorem]
        {中心极限定理的推广}
        [Generalization of Central Limit Theorem]
        [gpt-4.1]
        
设 $\alpha _ { n } = \{ \operatorname{var} ( S _ { n } ) \} ^ { 1 / 2 }$.如果存在 $\delta > 0$ 使得
\[
\lim _ { n \to \infty } \alpha _ { n } ^ { - ( 2 + \delta ) } \sum _ { m = 1 } ^ { n } E ( | X _ { m } - E X _ { m } | ^ { 2 + \delta } ) = 0
\]
那么 $( S _ { n } - E S _ { n } ) / \alpha _ { n } \Rightarrow \chi$.

    \end{thm}
    
    
    
    \begin{thm}
        [Levy-Khinchin-Theorem]
        {Levy-Khinchin定理}
        [Levy-Khinchin Theorem]
        [gpt-4.1]
        
$Z$ 具有无限可分分布当且仅当其特征函数满足

\[
\log \varphi(t) = i c t - \frac{\sigma^{2} t^{2}}{2} + \int \left( e^{i t x} - 1 - \frac{i t x}{1 + x^{2}} \right) \mu(dx)
\]

其中 $\mu$ 是满足 $\mu(\{0\}) = 0$ 且 $\int \frac{x^{2}}{1 + x^{2}} \mu(dx) < \infty$ 的测度.

    \end{thm}
    
    
    
    \begin{dfn}
        [Definition-of-Levy-Measure]
        {Levy测度的定义}
        [Definition of Levy Measure]
        [gpt-4.1]
        
$\mu$ 被称为该分布的Levy测度.

    \end{dfn}
    
    
    
    \begin{thm}
        [Convergence-Theorem]
        {收敛定理}
        [Convergence Theorem]
        [gpt-4.1]
        设 $X_n$, $1 \leq n \leq \infty$ 为随机向量,其特征函数为 $\varphi_n$.$X_n \Rightarrow X_\infty$ 的充要条件是 $\varphi_n(t) \to \varphi_\infty(t)$.
    \end{thm}
    
    
    
    \begin{prf}
        [Proof-of-the-Convergence-Theorem]
        {收敛定理的证明}
        [Proof of the Convergence Theorem]
        [gpt-4.1]
        $\exp(i t \cdot x)$ 是有界且连续的,因此如果 $X_n \Rightarrow X_\infty$,则 $\varphi_n(t) \to \varphi_\infty(t)$.为证明反向结论,只需证明序列是紧的.为此,注意到若固定 $\boldsymbol{\theta} \in \mathbf{R}^d$,则对所有 $s \in \mathbf{R}$,$\varphi_n(s\theta) \to \varphi_\infty(s\theta)$,因此根据定理 3.3.17,$\theta \cdot X_n$ 的分布是紧的.对 $d$ 个单位向量 $e_1, \ldots, e_d$ 应用上述结论,得 $X_n$ 的分布是紧的,因此证明完成.
    \end{prf}
    
    
    
    \begin{thm}
        [Characterization-of-Subsequence-Limits-as-Distribution-Functions-of-Probability-Measures]
        {序列极限为概率测度分布函数的充要条件}
        [Characterization of Subsequence Limits as Distribution Functions of Probability Measures]
        [gpt-4.1]
        
每个子列极限都是概率测度的分布函数的充要条件是序列 $F_n$ 紧致,即对任意 $\epsilon > 0$,存在 $M_\epsilon$,使得
\[
\lim_{n \to \infty} \sup \left(1 - F_n(M_\epsilon) + F_n(-M_\epsilon)\right) \leq \epsilon
\]

    \end{thm}
    
    
    
    \begin{prf}
        [Proof-of-Tightness-and-Subsequence-Limits-as-Distribution-Functions]
        {序列紧致性与极限为概率测度分布函数的证明}
        [Proof of Tightness and Subsequence Limits as Distribution Functions]
        [gpt-4.1]
        
假设序列是紧致的,且 $F_{n(k)} \Rightarrow F$.令 $r < -M_\epsilon$ 和 $s > M_\epsilon$ 为 $F$ 的连续点.由于 $F_n(r) \to F(r)$ 和 $F_n(s) \to F(s)$,我们有
\[
\begin{array}{c}
1 - F(s) + F(r) = \displaystyle \lim_{k \to \infty} \left[1 - F_{n(k)}(s) + F_{n(k)}(r)\right] \\
\leq \displaystyle \lim_{n \to \infty} \left[1 - F_n(M_\epsilon) + F_n(-M_\epsilon)\right] \leq \epsilon
\end{array}
\]
由此推出 $\limsup_{x \to \infty} \left[1 - F(x) + F(-x)\right] \leq \epsilon$.由于 $\epsilon$ 是任意的,故 $F$ 是概率测度的分布函数.

    \end{prf}
    
    
    
    \begin{dfn}
        [Definition-of-Coupon-Collectors-Problem]
        {Coupon collector 问题的定义}
        [Definition of Coupon Collector's Problem]
        [gpt-4.1]
        
设 $X_1, X_2, \ldots$ 是一组独立同分布 (i.i.d.) 的随机变量,均匀取值于集合 $\{1, 2, \ldots, n\}$.定义 $T_n = \inf \{ m : \{ X_1, \ldots, X_m \} = \{ 1, 2, \ldots, n \} \}$,即 $T_n$ 表示收集到所有 $n$ 种类型所需的最少抽取次数.

    \end{dfn}
    
    
    
    \begin{prf}
        [Proof-of-the-Converse-for-Tightness-of-Distribution-Functions]
        {分布函数列紧性反设的证明}
        [Proof of the Converse for Tightness of Distribution Functions]
        [gpt-4.1]
        假设 $F_n$ 不是紧的,则存在 $\epsilon > 0$ 和子列 $n(k)\to\infty$,使得对所有 $k$ 都有

\[
1 - F_{n(k)}(k) + F_{n(k)}(-k) \geq \epsilon
\]

通过进一步取子列 $F_{n(k_j)}$,可以假设 $F_{n(k_j)} \Rightarrow F$.令 $r<0<s$ 为 $F$ 的连续点,则

\[
\begin{array}{l}
1 - F(s) + F(r) = \displaystyle \lim_{j \to \infty} \left[1 - F_{n(k_{j})}(s) + F_{n(k_{j})}(r)\right] \\
\geq \displaystyle \lim_{j \to \infty} \inf \left[1 - F_{n(k_{j})}(k_{j}) + F_{n(k_{j})}(-k_{j})\right] \geq \epsilon
\end{array}
\]

令 $s\to\infty, r\to-\infty$,可知 $F$ 不是概率测度的分布函数.

    \end{prf}
    
    
    
    \begin{lma}
        [Upper-Bound-of-the-Infinite-Series-$\sum-{k=1}^{\infty}-\operatorname{var}Y-{k}/k^{2}$]
        {关于无穷级数$\sum\_{k=1}^{\infty} \operatorname{var}(Y\_{k})/k^{2}$的上界}
        [Upper Bound of the Infinite Series $\sum_{k=1}^{\infty} \operatorname{var}(Y_{k})/k^{2}$]
        [gpt-4.1]
        
$\sum_{k=1}^{\infty} \operatorname{var}(Y_{k})/k^{2} \leq 4 E|X_{1}| < \infty.$

    \end{lma}
    
    
    
    \begin{prf}
        [Proof-of-the-Upper-Bound-of-the-Infinite-Series-$\sum-{k=1}^{\infty}-\operatorname{var}Y-{k}/k^{2}$]
        {关于无穷级数$\sum\_{k=1}^{\infty} \operatorname{var}(Y\_{k})/k^{2}$上界的证明}
        [Proof of the Upper Bound of the Infinite Series $\sum_{k=1}^{\infty} \operatorname{var}(Y_{k})/k^{2}$]
        [gpt-4.1]
        
证明 To bound the sum, we observe

\[
\operatorname{var}(Y_{k}) \leq E(Y_{k}^{2}) = \int_{0}^{\infty} 2y\, P(|Y_{k}| > y)\, dy \leq \int_{0}^{k} 2y\, P(|X_{1}| > y)\, dy
\]

so using Fubini's theorem (since everything is $\geq 0$ and the sum is just an integral with respect to counting measure on $\{1, 2, \ldots\}$),

\[
\sum_{k=1}^{\infty} E(Y_{k}^{2})/k^{2} \leq \sum_{k=1}^{\infty} k^{-2} \int_{0}^{\infty} \mathbf{1}_{(y < k)} 2y\, P(|X_{1}| > y)\, dy
= \int_{0}^{\infty} \left\{ \sum_{k=1}^{\infty} k^{-2} \mathbf{1}_{(y < k)} \right\} 2y\, P(|X_{1}| > y)\, dy
\]

Since $E|X_{1}| = \int_{0}^{\infty} P(|X_{1}| > y)\, dy$, we can complete the proof by showing:

    \end{prf}
    
    
    
    \begin{xmp}
        [Example-Limiting-Distribution-of-the-Mean-of-Reciprocals-of-i.i.d.-Random-Variables]
        {关于独立同分布随机变量倒数均值极限分布的例子}
        [Example: Limiting Distribution of the Mean of Reciprocals of i.i.d. Random Variables]
        [gpt-4.1]
        
设 $X_1, X_2, \dots$ 为独立同分布随机变量,其概率密度函数 $f$ 关于 $0$ 对称,且在 $0$ 处连续且取正值.则有
\[
\frac{1}{n} \left( \frac{1}{X_1} + \cdots + \frac{1}{X_n} \right) \Rightarrow \text{Cauchy 分布 } (\alpha = 1, \kappa = 0)
\]
即当 $n \to \infty$ 时,倒数的均值弱收敛到参数为 $\alpha=1, \kappa=0$ 的柯西分布.

具体计算如下:对于任意 $x > 0$,
\[
P(1/X_i > x) = P(0 < X_i < x^{-1}) = \int_0^{x^{-1}} f(y) dy \sim \frac{f(0)}{x}
\]
当 $x \to \infty$ 时,上述概率近似为 $f(0)/x$.类似地,
\[
P(1/X_i < -x) \sim \frac{f(0)}{x}
\]
故满足定理 3.8.2 中 (i) 条件,取 $\theta = 1/2$,以及 (ii) 条件,取 $\alpha=1$.尺度因子 $a_n \sim 2 f(0) n$,由于 $X$ 的分布关于 $0$ 对称,中心化常数为零.

    \end{xmp}
    
    
    
    \begin{lma}
        [Lemma-on-Truncated-Random-Variables-and-Convergence]
        {关于截断随机变量和收敛的引理}
        [Lemma on Truncated Random Variables and Convergence]
        [gpt-4.1]
        
设 $Y_{k} = X_{k} \boldsymbol{1}_{(|X_{k}| \leq k)}$,$T_{n} = Y_{1} + \cdots + Y_{n}$.若 $\sum_{k=1}^{\infty} P(|X_{k}| > k) \leq \int_{0}^{\infty} P(|X_{1}| > t) dt = E|X_{1}| < \infty$,则 $P(X_{k} 
eq Y_{k}\ \mathrm{i.o.}) = 0$,并且 $|S_{n}(\omega) - T_{n}(\omega)| \leq R(\omega) < \infty$ a.s. 对所有 $n$ 成立,从而有 $T_{n}/n \to \mu$ a.s..

    \end{lma}
    
    
    
    \begin{thm}
        [Theorem-on-Limiting-Characteristic-Function-Formula]
        {极限特征函数公式的定理}
        [Theorem on Limiting Characteristic Function Formula]
        [gpt-4.1]
        
公式 (3.8.10) 给出了极限特征函数的表达式:

\[
\begin{array}{l}
{\displaystyle \exp \left( i t c + \int_{0}^{\infty} \left( e^{i t x} - 1 - \frac{i t x}{1 + x^{2}} \right) \theta \alpha x^{-(\alpha + 1)} dx \right.} \\
{\displaystyle \qquad + \left.\int_{-\infty}^{0} \left( e^{i t x} - 1 - \frac{i t x}{1 + x^{2}} \right) (1 - \theta) \alpha |x|^{-(\alpha + 1)} dx \right) }
\end{array}
\]

此公式用于解释极限特征函数的形式.

    \end{thm}
    
    
    
    \begin{dfn}
        [Definition-of-Direct-Riemann-Integrability]
        {直接Riemann可积性的定义}
        [Definition of Direct Riemann Integrability]
        [gpt-4.1]
        
如果 $I^{\delta}$ 和 $I_{\delta}$ 都在 $\delta \to 0$ 时收敛到同一个有限极限 $I$,则称 $h$ 是直接Riemann可积(directly Riemann integrable)的,并且有

\[
\int_{0}^{t} h(t-s)\, dU(s) \to I/\mu
\]

    \end{dfn}
    
    
    
    \begin{thm}
        [Equivalence-of-Infinite-Product-Vanishing-and-Divergence-of-Infinite-Series]
        {无限乘积与无穷级数发散的等价性}
        [Equivalence of Infinite Product Vanishing and Divergence of Infinite Series]
        [gpt-4.1]
        
设 $p_{m} \in [0, 1)$. 利用 Borel-Cantelli 引理可知:
\[
\prod_{m=1}^{\infty} (1 - p_{m}) = 0 \quad \mathrm{当且仅当} \quad \sum_{m=1}^{\infty} p_{m} = \infty.
\]

    \end{thm}
    
    
    
    \begin{dfn}
        [Definition-of-Functions-$\delta-pn$-and-$gn$]
        {函数$\delta\_p(n)$和$g(n)$的定义}
        [Definition of Functions $\delta_p(n)$ and $g(n)$]
        [gpt-4.1]
        令 $\delta_p(n) = 1$ 当且仅当 $n$ 能被 $p$ 整除,否则 $\delta_p(n) = 0$.再定义
\[
g(n) = \sum_{p \leq n} \delta_p(n)
\]
其中$g(n)$表示$n$的素因子个数,此处及以后对$p$的求和均在所有不超过$n$的素数上进行.
    \end{dfn}
    
    
    
    \begin{thm}
        [De-Moivre-Laplace-Theorem]
        {De Moivre-Laplace 定理}
        [De Moivre-Laplace Theorem]
        [gpt-4.1]
        
设 $X_{1}, X_{2}, \dots$ 是一组独立同分布的随机变量,满足 $P(X_{1}=1) = P(X_{1} = -1) = 1/2$,令 $S_{n} = X_{1} + \cdots + X_{n}$.则对于整数 $n$ 和 $k$,
\[
P(S_{2n} = 2k) = \binom{2n}{n+k} 2^{-2n}
\]
其中 $S_{2n} = 2k$ 当且仅当在前 $2n$ 次抛掷中有 $n + k$ 次结果为 $+1$,$n - k$ 次结果为 $-1$.第一个因子表示这种结果的可能性数,第二个因子表示每种结果的概率.

    \end{thm}
    
    
    
    \begin{thm}
        [A-Variant-of-the-Conditional-Variance-Formula]
        {条件方差公式的变式}
        [A Variant of the Conditional Variance Formula]
        [gpt-4.1]
        
若 $\mathcal{F} \subset \mathcal{G}$,则有

\[
E\left( E[ Y \mid \mathcal{G} ] - E[ Y \mid \mathcal{F} ] \right)^{2} = E \left( E[ Y \mid \mathcal{G} ] \right)^{2} - E \left( E[ Y \mid \mathcal{F} ] \right)^{2}
\]

    \end{thm}
    
    
    
    \begin{prf}
        [Proof-of-Existence-of-Limits-for-Martingales]
        {关于鞅极限存在性的证明}
        [Proof of Existence of Limits for Martingales]
        [gpt-4.1]
        由于 $X_{n} - X_{0}$ 是一个鞅,我们可以无损地假设 $X_{0} = 0$.令 $0 < K < \infty$,定义 $N = \inf\{ n : X_{n} \leq -K \}$.$X_{n \wedge N}$ 是一个鞅,并且 $X_{n \wedge N} \geq -K - M$,因此应用定理 4.2.12 到 $X_{n \wedge N} + K + M$,可知 $\lim X_{n}$ 在 $\{ N = \infty \}$ 上存在.令 $K \to \infty$,我们发现极限在 $\{ \liminf X_{n} > -\infty \}$ 上存在.对 $-X_{n}$ 应用最后的结论,可以得到 $\lim X_{n}$ 在 $\{ \limsup X_{n} < \infty \}$ 上存在,证明完毕.
    \end{prf}
    
    
    
    \begin{thm}
        [Two-Nonexistent-Lower-Bounds-for-Probability-Minimization]
        {两个下界不存在的概率极小化结果}
        [Two Nonexistent Lower Bounds for Probability Minimization]
        [gpt-4.1]
        
(i) 若 $\epsilon > 0$, 则
\[
\inf\{P(|X| > \epsilon) : E X = 0,~\operatorname{var}(X) = 1\} = 0.
\]
(ii) 若 $y \ge 1$, $\sigma^2 \in (0, \infty)$, 则
\[
\inf\{P(|X| > y) : E X = 1,~\operatorname{var}(X) = \sigma^2\} = 0.
\]

    \end{thm}
    
    
    
    \begin{prf}
        [Proof-of-Applying-Walds-Equation-to-Stopping-Time]
        {关于 Wald 方程在停时上的应用的证明}
        [Proof of Applying Wald's Equation to Stopping Time]
        [gpt-4.1]
        我们将把 Wald 方程应用于停时 $N_t$.

首先要证明 $E N_t < \infty$.为此,取 $\delta > 0$ 使得 $P(\xi_i > \delta) = \epsilon > 0$,再取 $K$ 使得 $K \delta \ge t$.

由于连续 $K$ 个 $\xi_i$ 都大于 $\delta$ 则会使 $T_n > t$,可得
\[
P(N_t > mK) \leq (1 - \epsilon^K)^m
\]
从而 $E N_t < \infty$.

若 $\mu < \infty$,应用 Wald 方程得
\[
\mu E N_t = E T_{N_t} \geq t
\]
所以 $U(t) \geq t / \mu$.

当 $\mu = \infty$ 时上述不等式显然成立,所以一般成立.

关于上界,如果 $P(\xi_i \le c) = 1$,重复上述论证可得 $\mu E N_t = E S_{N_t} \leq t + c$,该结论适用于有界分布.

令 $\bar{\xi}_i = \xi_i \wedge c$,并以显然方式定义 $\bar{T}_n$ 和 $\bar{N}_t$,则
\[
E N_t \leq E \bar{N}_t \leq (t + c) / E(\bar{\xi}_i)
\]
令 $t \to \infty$,再令 $c \to \infty$ 得
\[
\limsup_{t \to \infty} E N_t / t \leq 1 / \mu
\]
证明完成.

    \end{prf}
    
    
    
    \begin{xmp}
        [An-Example-of-Expectation-of-a-Function-under-Independence]
        {独立随机变量下函数期望的例子}
        [An Example of Expectation of a Function under Independence]
        [gpt-4.1]
        假设 $X$ 和 $Y$ 是独立的.令 $\varphi$ 为一个函数,满足 $E|\varphi(X, Y)| < \infty$,并且定义 $g(x) = E(\varphi(x, Y))$.
    \end{xmp}
    
    
    
    \begin{dfn}
        [Definition-of-Finite-Permutation]
        {有限排列的定义}
        [Definition of Finite Permutation]
        [gpt-4.1]
        一个有限排列是指从 $\mathbf{N} = \{ 1, 2, \ldots \}$ 到 $\mathbf{N}$ 的一个映射 $\pi$,使得仅有有限个 $i$ 满足 $\pi(i) 
eq i$.
    \end{dfn}
    
    
    
    \begin{dfn}
        [Action-of-Finite-Permutation-on-Sequence]
        {有限排列作用下的点变换}
        [Action of Finite Permutation on Sequence]
        [gpt-4.1]
        若 $\pi$ 是 $\mathbf{N}$ 上的有限排列,$\boldsymbol{\omega} \in \mathcal{S}^{\mathbf{N}}$,则定义 $(\pi \omega)_i = \omega_{\pi(i)}$.
    \end{dfn}
    
    
    
    \begin{dfn}
        [Definition-of-Permutable-Event]
        {可排列事件的定义}
        [Definition of Permutable Event]
        [gpt-4.1]
        当对任意有限排列 $\pi$,有 $\pi^{-1} A \equiv \{ \omega : \pi \omega \in A \} = A$,即仅有限个随机变量重新排列不会影响事件 $A$ 的发生时,称事件 $A$ 是可排列的.
    \end{dfn}
    
    
    
    \begin{dfn}
        [Sigma-field-of-Permutable-Events]
        {可排列事件的σ-域}
        [Sigma-field of Permutable Events]
        [gpt-4.1]
        所有可排列事件的集合构成一个 $\sigma$-域,称为可交换 $\sigma$-域,记作 $\mathcal{E}$.
    \end{dfn}
    
    
    
    \begin{prf}
        [Proof-of-the-Characterization-for-|φt|=1-and-Trichotomy]
        {特征函数模为1的充要条件及三分法证明}
        [Proof of the Characterization for |φ(t)|=1 and Trichotomy]
        [gpt-4.1]
        证明 (ii):
只需证明 $|\varphi(t)| = 1$ 当且仅当对某个 $b$ 有 $P(X \in b + (2\pi / t)\mathbf{Z}) = 1$.

首先,如果 $P(X \in b + (2\pi / t)\mathbf{Z}) = 1$,那么
\[
\varphi(t) = E e^{i t X} = e^{i t b} \sum_{n \in \mathbf{Z}} e^{i 2\pi n} P(X = b + (2\pi / t) n) = e^{i t b}
\]

反之,如果 $|\varphi(t)| = 1$,则在不等式 $|E e^{i t X}| \leq E|e^{i t X}|$ 中取等.所以根据练习 1.6.1,$e^{i t X}$ 的分布必定集中在某一点 $e^{i t b}$,即 $P(X \in b + (2\pi / t)\mathbf{Z}) = 1$.

现在证明三分法.假设 (i) 和 (ii) 不成立,即存在序列 $t_n \downarrow 0$ 使得 $|\varphi(t_n)| = 1$.第一段说明存在 $b_n$ 使得 $P(X \in b_n + (2\pi / t_n)\mathbf{Z}) = 1$.不妨设 $b_n \in (-\pi / t_n, \pi / t_n]$.当 $n \to \infty$ 时,$P(X 
otin (-\pi / t_n, \pi / t_n]) \to 0$,所以有 $P(X = b_n) \to 1$.这仅当 $b_n = b$ 对于足够大的 $n$ 成立,并且 $P(X = b) = 1$.

    \end{prf}
    
    
    
    \begin{dfn}
        [Definition-of-Characteristic-Function]
        {特征函数的定义}
        [Definition of Characteristic Function]
        [gpt-4.1]
        如果 $X$ 是一个随机变量,则称其特征函数(characteristic function,记作 ch.f.)为
\[
\varphi(t) = E e^{i t X} = E \cos t X + i E \sin t X
\]
其中,$\varphi(t)$ 表示 $X$ 的特征函数.
    \end{dfn}
    
    
    
    \begin{dfn}
        [Definition-of-Expectation-for-Complex-Valued-Random-Variables]
        {复值随机变量的期望定义}
        [Definition of Expectation for Complex-Valued Random Variables]
        [gpt-4.1]
        如果 $Z$ 是复值随机变量,则其期望定义为
\[
E Z = E(\operatorname{Re} Z) + i E(\operatorname{Im} Z)
\]
其中 $\operatorname{Re}(a + b i) = a$ 表示复数的实部,$\operatorname{Im}(a + b i) = b$ 表示虚部.
    \end{dfn}
    
    
    
    \begin{dfn}
        [Definition-of-Modulus-and-Conjugate-of-Complex-Numbers]
        {复数的模与共轭的定义}
        [Definition of Modulus and Conjugate of Complex Numbers]
        [gpt-4.1]
        复数 $z = a + b i$ 的模为 $|a + b i| = (a^{2} + b^{2})^{1/2}$,其共轭为 $\bar{z} = a - b i$.
    \end{dfn}
    
    
    
    \begin{thm}
        [Monotonicity-of-Conditional-Expectation-and-Its-Corollary]
        {条件期望的单调性及其推论}
        [Monotonicity of Conditional Expectation and Its Corollary]
        [gpt-4.1]
        
设 $j \le k$,则有 $E ( X_{j} ; N = j ) \le E ( X_{k} ; N = j )$,并对 $j$ 求和可得 $E X_{N} \le E X_{k}$ 的第二种证明.

    \end{thm}
    
    
    
    \begin{lma}
        [Application-of-Lemma-3.1.1]
        {引理3.1.1的应用}
        [Application of Lemma 3.1.1]
        [gpt-4.1]
        
利用引理3.1.1,可得若 $2k = x\sqrt{2n}$,即 $k = x\sqrt{n/2}$,则有:
\[
\begin{aligned}
&\left(1 - \frac{k^{2}}{n^{2}}\right)^{-n} = \left(1 - \frac{x^{2}}{2n}\right)^{-n} \to e^{x^{2}/2} \\
&\left(1 + \frac{k}{n}\right)^{-k} = \left(1 + \frac{x}{\sqrt{2n}}\right)^{-x \sqrt{n/2}} \to e^{-x^{2}/2} \\
&\left(1 - \frac{k}{n}\right)^{k} = \left(1 - \frac{x}{\sqrt{2n}}\right)^{x \sqrt{n/2}} \to e^{-x^{2}/2}
\end{aligned}
\]

    \end{lma}
    
    
    
    \begin{thm}
        [Submartingale-Expectation-Inequality-under-Bounded-Stopping-Time]
        {子鞅在有界停时下的期望不等式}
        [Submartingale Expectation Inequality under Bounded Stopping Time]
        [gpt-4.1]
        如果 $X_{n}$ 是一个子鞅,$N$ 是一个停时且满足 $P(N \leq k) = 1$,则有
\[
E X_{0} \leq E X_{N} \leq E X_{k}
\]

    \end{thm}
    
    
    
    \begin{thm}
        [Limit-Distribution-Theorem]
        {极限分布定理}
        [Limit Distribution Theorem]
        [gpt-4.1]
        当 $n \to \infty$ 时,$( S_{n} - b_{n} ) / a_{n} \Rightarrow Y$,其中 $Y$ 具有非退化分布.
    \end{thm}
    
    
    
    \begin{thm}
        [Second-Borel-Cantelli-Lemma-III]
        {第二类Borel-Cantelli引理(III)}
        [Second Borel-Cantelli Lemma (III)]
        [gpt-4.1]
        假设 $B_n$ 适应于 $\mathcal{F}_n$,并令 $p_n = P(B_n | \mathcal{F}_{n-1})$.则
\[
\sum_{m=1}^n 1_{B(m)} \sim \sum_{m=1}^n p_m \quad \text{a.s. on} \quad \left\{ \sum_{m=1}^{\infty} p_m = \infty \right\}
\]

    \end{thm}
    
    
    
    \begin{dfn}
        [Formula-for-Computing-Higher-Moments]
        {高阶矩的计算公式}
        [Formula for Computing Higher Moments]
        [gpt-4.1]
        
\[
T_{n}^{r} = \sum_{k=1}^{r} \sum_{r_{i}} \frac{r!}{r_{1}! \cdots r_{k}!} \frac{1}{k!} \sum_{i_{j}} Z_{n,i_{1}}^{r_{1}} \cdots Z_{n,i_{k}}
\]

其中 $\sum_{r_{i}}$ 表示对所有满足 $r_{1} + \cdots + r_{k} = r$ 的 $k$ 元组正整数求和,$\sum_{i_{j}}$ 表示对所有 $1 \leq i \leq n$ 的不同整数 $k$ 元组求和.

    \end{dfn}
    
    
    
    \begin{dfn}
        [Definition-of-A-n]
        {A\_n 的定义}
        [Definition of A_n]
        [gpt-4.1]
        
\[
A_{n}(r_{1}, \ldots, r_{k}) = \sum_{i_{j}} E Z_{n,i_{1}}^{r_{1}} \cdots E Z_{n,i_{k}}^{r_{k}}
\]

    \end{dfn}
    
    
    
    \begin{ppt}
        [Property-of-A-n-Zero-when-r-j=1]
        {A\_n 的性质(r\_j=1时为零)}
        [Property of A_n (Zero when r_j=1)]
        [gpt-4.1]
        
当某个 $r_j = 1$ 时,$A_n(r_1, ..., r_k) = 0$,因为 $E Z_{n, i_j} = 0$.

    \end{ppt}
    
    
    
    \begin{ppt}
        [Upper-Bound-of-A-n-when-all-r-j=2]
        {全部 r\_j = 2 时的 A\_n 上界}
        [Upper Bound of A_n when all r_j=2]
        [gpt-4.1]
        
当所有 $r_j=2$ 时,
\[
\sum_{i_{j}} E Z_{n,i_{1}}^{2} \cdots E Z_{n,i_{k}}^{2} \leq \left( \sum_{m=1}^{n} E Z_{n,m}^{2} \right)^{k} \sim \sigma^{2k}
\]

    \end{ppt}
    
    
    
    \begin{ppt}
        [Lower-Bound-of-A-n-when-all-r-j=2]
        {全部 r\_j=2 时的 A\_n 下界}
        [Lower Bound of A_n when all r_j=2]
        [gpt-4.1]
        
对于任意 $1 \leq a < b \leq k$,我们可以估算 $i_a = i_b$ 的情形:
\[
\left( \sum_{m=1}^{n} E Z_{n,m}^{2} \right)^{k} - \sum_{i_{j}} E Z_{n,i_{1}}^{2} \cdots E Z_{n,i_{k}}^{2} \leq \binom{k}{2} (2\epsilon_{n})^{2} \left( \sum_{m=1}^{n} E Z_{n,m}^{2} \right)^{k-1} \to 0
\]

    \end{ppt}
    
    
    
    \begin{ppt}
        [Upper-Bound-of-A-n-when-some-r-j>2-and-all-r-i≥2]
        {r\_i≥2 且部分 r\_j>2 时的 A\_n 上界}
        [Upper Bound of A_n when some r_j>2 and all r_i≥2]
        [gpt-4.1]
        
如果所有 $r_i \geq 2$ 且某些 $r_j > 2$,则有
\[
E |Z_{n,i_{j}}|^{r_{j}} \leq (2\epsilon_{n})^{r_{j} - 2} E Z_{n,i_{j}}^{2}
\]
因此,
\[
|A_{n}(r_{1}, \ldots, r_{k})| \leq \sum_{i_{j}} E |Z_{n,i_{1}}|^{r_{1}} \cdots E |Z_{n,i_{k}}|^{r_{k}} \leq (2\epsilon_{n})^{r-2k} A_{n}(2, \ldots, 2) \to 0
\]

    \end{ppt}
    
    
    
    \begin{crl}
        [E-T-n^r-Converges-to-Zero-when-r-is-Odd]
        {r 为奇数时 E T\_n^r 收敛于零}
        [E T_n^r Converges to Zero when r is Odd]
        [gpt-4.1]
        
当 $r$ 为奇数时,必有某个 $r_j = 1$ 或 $r_j \geq 3$,故由 (a) 和 (c) 可知 $E T_n^r \to 0$.

    \end{crl}
    
    
    
    \begin{cxmp}
        [Counterexample-of-Different-Densities-with-the-Same-Moments]
        {具有相同矩的不同密度函数的反例}
        [Counterexample of Different Densities with the Same Moments]
        [gpt-4.1]
        考虑 lognormal 密度函数
\[
f_{0}(x) = (2\pi)^{-1/2} x^{-1} \exp(-(\log x)^{2} / 2) \qquad x \geq 0
\]
以及对于 $-1 \leq a \leq 1$ 取
\[
f_{a}(x) = f_{0}(x) \{1 + a \sin(2\pi \log x)\}
\]
则 $f_a$ 也是概率密度函数,并且其矩与 $f_0$ 相同.证明如下:对任意非负整数 $r$,有
\[
\int_{0}^{\infty} x^{r} f_{0}(x) \sin(2\pi \log x) dx = 0
\]
通过变量替换 $x = \exp(s + r),\ s = \log x - r,\ ds = dx / x$,该积分变为
\[
(2\pi)^{-1/2} \int_{-\infty}^{\infty} \exp(r s + r^{2}) \exp(- (s + r)^{2} / 2) \sin(2\pi (s + r)) ds 
= (2\pi)^{-1/2} \exp(r^{2}/2) \int_{-\infty}^{\infty} \exp(-s^{2} / 2) \sin(2\pi s) ds = 0
\]
因此,$f_a$ 为一族具有相同全部矩但不同的概率密度函数,说明矩不能唯一确定概率分布.

    \end{cxmp}
    
    
    
    \begin{dfn}
        [Definition-of-the-distance-to-set-function]
        {距离到集合的函数的定义}
        [Definition of the distance-to-set function]
        [gpt-4.1]
        设 $\rho(x, K) = \inf\{ \rho(x, y) : y \in K \}$,$\varphi_{j}(r) = (1 - j r)^{+}$,并定义 $f_{j}(x) = \varphi_{j}(\rho(x, K))$.
    \end{dfn}
    
    
    
    \begin{ppt}
        [Continuity-and-convergence-properties-of-$f-j$]
        {$f\_j$ 的连续性与收敛性质}
        [Continuity and convergence properties of $f_j$]
        [gpt-4.1]
        $f_{j}$ 是 Lipschitz 连续的,取值于 $[0,1]$,且当 $j \uparrow \infty$ 时,$f_{j} \downarrow 1_{K}(x)$.
    \end{ppt}
    
    
    
    \begin{thm}
        [Relationship-between-limit-of-probabilities-and-expectation-convergence]
        {概率极限与期望收敛关系}
        [Relationship between limit of probabilities and expectation convergence]
        [gpt-4.1]
        对于上述定义的 $f_j$ 和集合 $K$,有
\[
\lim_{n \to \infty} P(X_{n} \in K) \leq \lim_{n \to \infty} E f_{j}(X_{n}) = E f_{j}(X_{\infty}) \downarrow P(X_{\infty} \in K) \text{ as } j \uparrow \infty
\]

    \end{thm}
    
    
    
    \begin{ppt}
        [Complementarity-of-open-and-closed-sets-and-the-probability-sum-formula]
        {开集与闭集的互补性质及概率加法公式}
        [Complementarity of open and closed sets and the probability sum formula]
        [gpt-4.1]
        集合 $A$ 是开集当且仅当其补集 $A^{c}$ 是闭集;且 $P(A) + P(A^{c}) = 1$.
    \end{ppt}
    
    
    
    \begin{cxmp}
        [Counterexample-to-the-Nonexistence-of-$L^1$-Maximal-Inequality]
        {关于 $L^1$ 最大不等式不存在的反例}
        [Counterexample to the Nonexistence of $L^1$ Maximal Inequality]
        [gpt-4.1]
        
最大不等式在 $L^1$ 空间中并不成立,其反例由例 4 给出.具体地,存在一个测度空间上的集合序列,使得相应的极大算子在 $L^1$ 空间中不满足最大不等式.

    \end{cxmp}
    
    
    
    \begin{thm}
        [An-Instance-of-the-Central-Limit-Theorem]
        {中心极限定理的一个实例}
        [An Instance of the Central Limit Theorem]
        [gpt-4.1]
        
设 $Y _ { 1 } , Y _ { 2 } , \ldots$ 是独立同分布随机变量,满足 $E Y _ { i } = 0$ 且 $E Y _ { i } ^ { 2 } = \sigma ^ { 2 } \in ( 0 , \infty )$,令 $X _ { n , m } = Y _ { m } / n ^ { 1 / 2 }$.则

\[
\sum _ { m = 1 } ^ { n } E X _ { n , m } ^ { 2 } = \sigma ^ { 2 }
\]

且对任意 $\epsilon > 0$,
\[
\sum _ { m = 1 } ^ { n } E ( | X _ { n , m } | ^ { 2 } ; | X _ { n , m } | > \epsilon ) = n E ( | Y _ { 1 } / n ^ { 1 / 2 } | ^ { 2 } ; | Y _ { 1 } / n ^ { 1 / 2 } | > \epsilon )
= E ( | Y _ { 1 } | ^ { 2 } ; | Y _ { 1 } | > \epsilon n ^ { 1 / 2 } ) \to 0
\]

当 $n \to \infty$ 时,上式由控测收敛定理成立,因为 $E Y _ { 1 } ^ { 2 } < \infty$.

    \end{thm}
    
    
    
    \begin{thm}
        [Expectation-Inequality-for-the-Supremum-of-Stochastic-Processes]
        {关于随机过程上界的期望不等式}
        [Expectation Inequality for the Supremum of Stochastic Processes]
        [gpt-4.1]
        
定理 4.5.7
\[
E \left( \sup_n |X_n| \right) \leq 3 E A_{\infty}^{1/2}.
\]

    \end{thm}
    
    
    
    \begin{xmp}
        [Example-of-Exponential-Martingale]
        {指数鞅的例子}
        [Example of Exponential Martingale]
        [gpt-4.1]
        
设 $Y _ { 1 } , Y _ { 2 } , \ldots$ 是一列非负独立同分布随机变量,且 $E Y _ { m } = 1$.若 $\mathcal { F } _ { n } = \sigma ( Y _ { 1 } , \ldots , Y _ { n } )$,则
\[
M _ { n } = \prod _ { m \leq n } Y _ { m }
\]
定义了一个鞅.
证明如下:
\[
E ( M _ { n + 1 } | \mathcal { F } _ { n } ) = M _ { n } E ( Y _ { n + 1 } | \mathcal { F } _ { n } ) = M _ { n } \cdot 1 = M _ { n }
\]
进一步,若 $Y _ { i } = e ^ { \theta \xi _ { i } }$ 且 $\phi ( \theta ) = E e ^ { \theta \xi _ { i } } < \infty$,则 $Y _ { i } = \exp ( \theta \xi _ { i } ) / \phi ( \theta )$ 满足 $E Y _ { i } = 1$,且
\[
M _ { n } = \prod _ { i = 1 } ^ { n } Y _ { i } = \exp ( \theta S _ { n } ) / \phi ( \theta ) ^ { n }
\]
也是一个鞅.

    \end{xmp}
    
    
    
    \begin{thm}
        [Probabilistic-Proof-of-the-Radon-Nikodym-Theorem]
        {Radon-Nikodym定理的概率证明}
        [Probabilistic Proof of the Radon-Nikodym Theorem]
        [gpt-4.1]
        设$\mathcal{F}$是可数生成的(即存在集合序列$A_n$使得$\mathcal{F} = \sigma(A_n : n \geq 1)$),若$\mu$和$
u$是$\sigma$-有限测度且$\mu \ll 
u$,则存在函数$g$使得对于任意$A \in \mathcal{F}$,有
\[
\mu(A) = \int_A g\, d
u.
\]

    \end{thm}
    
    
    
    \begin{thm}
        [Theorem-on-Normalized-Sum-Converging-to-Stable-Distribution]
        {归一化和收敛到稳定分布的定理}
        [Theorem on Normalized Sum Converging to Stable Distribution]
        [gpt-4.1]
        设 $Y$ 满足 $E \exp(i t Y) = \exp(-C|t|^{\alpha})$,则有 $\hat{S}_{n}(\epsilon_{n}) / n^{1/\alpha} \Rightarrow Y$.
    \end{thm}
    
    
    
    \begin{thm}
        [Theorem-on-the-Relevance-of-Uniform-Integrability-to-$L^1$-Convergence]
        {一致可积性与$L^1$收敛的关系定理}
        [Theorem on the Relevance of Uniform Integrability to $L^1$ Convergence]
        [gpt-4.1]
        
一致可积性与$L ^ { 1 }$收敛的关系体现在如下定理:

Theorem 4.(具体内容未给出)

    \end{thm}
    
    
    
    \begin{lma}
        [Lemma-on-Existence-of-Limit-for-Logarithmic-Subadditive-Sequences]
        {对数下鞅极限存在性引理}
        [Lemma on Existence of Limit for Logarithmic Subadditive Sequences]
        [gpt-4.1]
        引理 2.7.1 如果 $\gamma_{m + n} \geq \gamma_{m} + \gamma_{n}$,则当 $n \to \infty$ 时,有 $\gamma_{n} / n \to \operatorname*{sup}_{m} \gamma_{m} / m$.
    \end{lma}
    
    
    
    \begin{prf}
        [Proof-of-Lemma-on-Existence-of-Limit-for-Logarithmic-Subadditive-Sequences]
        {对数下鞅极限存在性引理的证明}
        [Proof of Lemma on Existence of Limit for Logarithmic Subadditive Sequences]
        [gpt-4.1]
        证明 显然有 $\limsup \gamma_{n} / n \leq \operatorname*{sup} \gamma_{m} / m$.为完成证明,只需证明对于任意 $m$,$\liminf \gamma_{n} / n \geq \gamma_{m} / m$.将 $n = k m + \ell$,其中 $0 \leq \ell < m$,并反复利用假设可得 $\gamma_{n} \geq k \gamma_{m} + \gamma_{\ell}$.两边同除以 $n = k m + \ell$ 得

\[
\frac{\gamma ( n )}{n} \geq \left( \frac{k m}{k m + \ell} \right) \frac{\gamma ( m )}{m} + \frac{\gamma ( \ell )}{n}
\]

令 $n \to \infty$ 并注意 $n = k m + \ell$ 且 $0 \leq \ell < m$,即可得所需结论.
    \end{prf}
    
    
    
    \begin{thm}
        [Extinction-Probability-Theorem-for-Branching-Process]
        {分支过程灭绝概率定理}
        [Extinction Probability Theorem for Branching Process]
        [gpt-4.1]
        
设 $Z_{n}$ 是一类分支过程,其子代分布为 $p_{k}$,生成函数为 $\varphi(\theta) = \sum p_{k} \theta^{k}$.若存在 $\rho < 1$ 满足 $\varphi(\rho) = \rho$,则对于任意初始个体 $Z_0 = x$,有
\[
P(Z_{n} = 0~\text{对于某个}~ n \geq 1 \mid Z_{0} = x) = \rho^{x}.
\]
并且,随机变量 $\rho^{Z_{n}}$ 是一个鞅(martingale).

    \end{thm}
    
    
    
    \begin{xmp}
        [Example-of-Cycles-in-a-Random-Permutation-and-Record-Values]
        {随机排列中的循环与记录值的例子}
        [Example of Cycles in a Random Permutation and Record Values]
        [gpt-4.1]
        
设 $Y_1, Y_2, \ldots$ 相互独立,满足 $P(Y_m = 1) = 1/m$, $P(Y_m = 0) = 1 - 1/m$,则 $E Y_m = 1/m$,$\operatorname{var}(Y_m) = 1/m - 1/m^2$.令 $S_n = Y_1 + \cdots + Y_n$,则 $E S_n \sim \log n$ 且 $\operatorname{var}(S_n) \sim \log n$.定义
\[
X_{n, m} = \frac{Y_m - 1/m}{(\log n)^{1/2}}
\]
有 $E X_{n, m} = 0$,$\sum_{m=1}^{n} E X_{n, m}^2 \to 1$,且对任意 $\epsilon > 0$,
\[
\sum_{m=1}^{n} E(|X_{n, m}|^2 ; |X_{n, m}| > \epsilon) \to 0
\]
因为当 $(\log n)^{-1/2} < \epsilon$ 时和为0.应用定理 3.4.10 得
\[
(\log n)^{-1/2} \left( S_n - \sum_{m=1}^{n} \frac{1}{m} \right) \Rightarrow \chi
\]
注意到
\[
\sum_{m=1}^{n-1} \frac{1}{m} \geq \int_{1}^{n} x^{-1} dx = \log n \geq \sum_{m=2}^{n} \frac{1}{m}
\]
说明 $|\log n - \sum_{m=1}^{n} 1/m| \leq 1$,因此结论可以写为
\[
(S_n - \log n) / (\log n)^{1/2} \Rightarrow \chi
\]

    \end{xmp}
    
    
    
    \begin{dfn}
        [Definition-of-Slowly-Varying-Function]
        {慢变函数的定义}
        [Definition of Slowly Varying Function]
        [gpt-4.1]
        
称 $L$ 为慢变函数(slowly varying),如果对于所有 $t > 0$,都有
\[
\operatorname{lim}_{x \to \infty} \frac{L(t x)}{L(x)} = 1.
\]

    \end{dfn}
    
    
    
    \begin{prf}
        [Diagonal-Proof-of-Convergence-for-Sequence-of-Functions]
        {对函数序列收敛性的对角证明}
        [Diagonal Proof of Convergence for Sequence of Functions]
        [gpt-4.1]
        第一步是对角论证.设 $q_{1}, q_{2}, \dots$ 是有理数的一个枚举.由于对每个 $k$,$F_{m} ( q_{k} ) \in [0, 1]$ 对所有 $m$ 成立,存在一个序列 $m_{k}(i) \to \infty$,它是 $m_{k-1}(j)$ 的子序列(令 $m_{0}(j) \equiv j$),使得
\[
F_{m_{k}(i)} ( q_{k} ) \to G ( q_{k} ) \text{ as } i \to \infty
\]
令 $F_{n(k)} = F_{m_{k}(k)}$.通过构造,$F_{n(k)} ( q ) \to G ( q )$ 对所有有理数 $q$ 成立.

    \end{prf}
    
    
    
    \begin{dfn}
        [Right-Continuous-Function-Defined-by-Infimum]
        {由上确界构造的右连续函数}
        [Right-Continuous Function Defined by Infimum]
        [gpt-4.1]
        函数 $G$ 可能不是右连续的,但定义 $F ( x ) = \inf \{ G ( q ) : q \in \mathbf{Q}, q > x \}$ 则 $F$ 是右连续的,因为
\[
\begin{array}{l}
\lim_{x_{n} \downarrow x} F ( x_{n} ) = \inf \{ G ( q ) : q \in \mathbf{Q}, q > x_{n} \text{ for some } n \} \\
= \inf \{ G ( q ) : q \in \mathbf{Q}, q > x \} = F ( x )
\end{array}
\]

    \end{dfn}
    
    
    
    \begin{thm}
        [Upper-Bound-for-Ratio-of-Heavy-Tailed-Probabilities]
        {关于重尾概率分布的比值上界}
        [Upper Bound for Ratio of Heavy-Tailed Probabilities]
        [gpt-4.1]
        对任意 $\delta > 0$,存在常数 $C$,使得对所有 $t \geq t_{0}$ 和 $y \leq 1$ 有
\[
P ( | X_{1} | > y t ) / P ( | X_{1} | > t ) \leq C y^{ - \alpha - \delta }
\]

    \end{thm}
    
    
    
    \begin{dfn}
        [Definition-of-the-Truncation-Function]
        {截断函数的定义}
        [Definition of the Truncation Function]
        [gpt-4.1]
        
设
\[
\varphi_{M}(x) = \left\{
  \begin{array}{ll}
    M & \text{如果 } x \geq M \\
    x & \text{如果 } |x| \leq M \\
    -M & \text{如果 } x \leq -M
  \end{array}
\right.
\]

    \end{dfn}
    
    
    
    \begin{thm}
        [Equivalence-Conditions-for-Convergence-in-Probability-and-Uniform-Integrability]
        {概率收敛与一致可积性等价条件}
        [Equivalence Conditions for Convergence in Probability and Uniform Integrability]
        [gpt-4.1]
        
设 $E | X_n | < \infty$ 对所有 $n$ 都成立.如果 $X_n \to X$ 按概率收敛,则以下条件等价:

(i) $\{ X_n : n \geq 0 \}$ 是一致可积的.

    \end{thm}
    
    
    
    \begin{thm}
        [Sufficient-Condition-for-Normalized-Sum-Convergence]
        {归一化和收敛的充分条件}
        [Sufficient Condition for Normalized Sum Convergence]
        [gpt-4.1]
        如果 $b_{m} \uparrow \infty$ 且 $\sum_{m=1}^{\infty} E \xi_{m}^{2} / b_{m}^{2} < \infty$,则 $X_{n} / b_{n} \to 0$ 几乎必然收敛.

特别地,如果 $E \xi_{n}^{2} \le K < \infty$ 且 $\sum_{m=1}^{\infty} b_{m}^{-2} < \infty$,则 $X_{n} / b_{n} \to 0$ 几乎必然收敛.

    \end{thm}
    
    
    
    \begin{thm}
        [Levys-0-1-Law]
        {Levy 0-1定律}
        [Levy's 0-1 Law]
        [gpt-4.1]
        设 $\mathcal{F}_{n} \uparrow \mathcal{F}_{\infty}$ 且 $A \in {\mathcal{F}}_{\infty}$,则 $E(1_{A}|\mathcal{F}_{n}) \to 1_{A}$ 几乎处处成立.
    \end{thm}
    
    
    
    \begin{thm}
        [Kolmogorovs-0-1-Law]
        {Kolmogorov 0-1 定律}
        [Kolmogorov's 0-1 Law]
        [gpt-4.1]
        如果 $X_1, X_2, \dots$ 是相互独立的随机变量,且 $A \in \mathcal{T}$,则 $P(A) = 0$ 或 $1$.
    \end{thm}
    
    
    
    \begin{prf}
        [Proof-of-Kolmogorovs-0-1-Law]
        {Kolmogorov 0-1 定律的证明}
        [Proof of Kolmogorov's 0-1 Law]
        [gpt-4.1]
        我们将证明 $A$ 与自身独立,即 $P(A \cap A) = P(A) P(A)$,因此 $P(A) = P(A)^2$,由此得 $P(A) = 0$ 或 $1$.

我们分两步证明这个结论:

(a) 若 $A \in \sigma(X_1, \ldots, X_k)$ 且 $B \in \sigma(X_{k+1}, X_{k+2}, \ldots)$,则 $A$ 与 $B$ 独立.

证明 (a):如果 $B \in \sigma(X_{k+1}, \ldots, X_{k+j})$,这可由定理 2.1.9 得出.因为 $\sigma(X_1, \dots, X_k)$ 与 $\cup_{j} \sigma(X_{k+1}, \ldots, X_{k+j})$ 是包含 $\Omega$ 的 $\pi$-系统,所以 (a) 由定理 2.1.7 得出.

(b) 若 $A \in \sigma(X_1, X_2, \ldots)$ 且 $B \in \mathcal{T}$,则 $A$ 与 $B$ 独立.

证明 (b):因为 $\mathcal{T} \subset \sigma(X_{k+1}, X_{k+2}, \ldots)$,若 $A \in \sigma(X_1, \ldots, X_k)$,则由 (a) 得出.$\cup_{k} \sigma(X_1, \ldots, X_k)$ 与 $\mathcal{T}$ 是包含 $\Omega$ 的 $\pi$-系统,所以 (b) 由定理 2.1.7 得出.

由于 $\mathcal{T} \subset \sigma(X_1, X_2, \ldots)$,(b) 蕴含任意 $A \in \mathcal{T}$ 与自身独立,从而定理 2.5.3 得证.
    \end{prf}
    
    
    
    \begin{thm}
        [Criterion-for-Uniform-Integrability]
        {一致可积性的判别条件}
        [Criterion for Uniform Integrability]
        [gpt-4.1]
        设 $\varphi \geq 0$ 是任意满足 $\varphi(x)/x \to \infty$ 当 $x \to \infty$ 的函数,例如 $\varphi(x)=x^p$ 其中 $p>1$ 或 $\varphi(x)=x\log^+x$.如果对所有 $i\in I$,有 $E\varphi(|X_i|)\leq C$,则 $\{X_i:i\in I\}$ 是一致可积的.
    \end{thm}
    
    
    
    \begin{xmp}
        [Example-of-Waiting-Time-for-Rare-Events]
        {稀有事件等待时间的例子}
        [Example of Waiting Time for Rare Events]
        [gpt-4.1]
        设 $X_p$ 表示在一系列独立实验中获得一次成功所需的试验次数,其中每次成功的概率为 $p$.则 $P(X_p \geq n) = (1 - p)^{n-1}$,其中 $n = 1, 2, 3, \ldots$.由引理 3.1.1 可得,当 $p \to 0$ 时,

\[
P(p X_{p} > x) \to e^{-x} \quad \mathrm{for\ all\ } x \geq 0
\]

换句话说,$p X_{p}$ 依分布收敛于指数分布.

    \end{xmp}
    
    
    
    \begin{dfn}
        [Definition-of-Renewal-Process-and-Related-Sequences]
        {更新过程及相关序列的定义}
        [Definition of Renewal Process and Related Sequences]
        [gpt-4.1]
        令 $T_n$ 是一个更新过程(且 $T_0 = 0$),$T_n'$ 是一个独立的平稳更新过程.令 $\eta_1, \eta_2, \ldots$ 和 $\eta_1', \eta_2', \ldots$ 是独立于 $T_n$ 和 $T_n'$ 的独立同分布随机变量,取值为 $0$ 和 $1$,概率各为 $1/2$.令 $
u_n = \eta_1 + \cdots + \eta_n$,$
u_n' = 1 + \eta_1' + \cdots + \eta_n'$,$S_n = T_{
u_n}$,$S_n' = T_{
u_n'}'$.
    \end{dfn}
    
    
    
    \begin{ppt}
        [Distribution-and-Irreducibility-of-Random-Walk-Increments]
        {随机游走增量的分布与遍历性}
        [Distribution and Irreducibility of Random Walk Increments]
        [gpt-4.1]
        $S_n - S_n'$ 的增量以至少 $1/4$ 的概率为 $0$,且其分布的支持是对称的,并包含 $\xi_k$ 的分布支持.因此,如果 $\xi_k$ 的分布是非算术的,则随机游走 $S_n - S_n'$ 是不可约的.
    \end{ppt}
    
    
    
    \begin{thm}
        [Almost-Sure-Hitting-of-Interval-by-Random-Walk]
        {随机游走首次进入区间的几乎必然性}
        [Almost Sure Hitting of Interval by Random Walk]
        [gpt-4.1]
        由于 $S_n - S_n'$ 的增量均值为 $0$,则 $N = \inf\{n : |S_n - S_n'| < \epsilon\}$ 满足 $P(N < \infty) = 1$,即几乎必然存在满足 $|S_N - S_N'| < \epsilon$ 的 $N$.可令 $J = 
u_N$,$K = 
u_N'$.
    \end{thm}
    
    
    
    \begin{thm}
        [Convergence-Estimate-of-Conditional-Expectation]
        {条件期望的收敛性估计}
        [Convergence Estimate of Conditional Expectation]
        [gpt-4.1]
        Jensen不等式给出
\[
| E ( Y _ { n } ~|~ { \mathcal F } _ { n } ) - E ( Y ~|~ { \mathcal F } _ { n } ) | \leq E ( | Y _ { n } - Y | ~|~ { \mathcal F } _ { n } ) \to 0 \quad \text{当}~ n \to \infty
\]

    \end{thm}
    
    
    
    \begin{thm}
        [Maximal-Inequality-for-Martingales-with-Quadratic-Bounds]
        {二次型鞅最大值概率界定理}
        [Maximal Inequality for Martingales with Quadratic Bounds]
        [gpt-4.1]
        
设 $X_n$ 是一个鞅,$X_0 = 0$ 且 $E X_n^2 < \infty$.则有如下概率不等式成立:

\[
P \left( \max_{1 \leq m \leq n} X_{m} \geq \lambda \right) \leq \frac{E X_{n}^{2}}{E X_{n}^{2} + \lambda^{2}}
\]

其中提示可利用 $( X_{n} + c )^{2}$ 是次鞅,并对 $c$ 优化.

    \end{thm}
    
    
    
    \begin{thm}
        [de-Finettis-Theorem]
        {de Finetti 定理}
        [de Finetti's Theorem]
        [gpt-4.1]
        若 $X _ { 1 } , X _ { 2 } , \dots$ 是可交换的,则在条件 $\mathcal { E }$ 下,$X _ { 1 } , X _ { 2 } , \ldots$ 相互独立且同分布.
    \end{thm}
    
    
    
    \begin{thm}
        [Uniform-Integrability-of-Conditional-Expectations-Theorem]
        {条件期望族的统一可积性定理}
        [Uniform Integrability of Conditional Expectations Theorem]
        [gpt-4.1]
        给定概率空间 $( \Omega , \mathcal{F}_o , P )$ 和 $X \in L^{1}$,则集合 $\{ E ( X | {\mathcal{F}} ) : {\mathcal{F}} \text{ 是 } \sigma\text{-域}, {\mathcal{F}} \subset \mathcal{F}_o \}$ 是一致可积的.
    \end{thm}
    
    
    
    \begin{crl}
        [Upper-Bound-for-Probability-of-Deviation-in-Coin-Flips]
        {关于硬币翻转偏差概率的上界}
        [Upper Bound for Probability of Deviation in Coin Flips]
        [gpt-4.1]
        对于硬币翻转,若 $\theta \leq 1$,则 $\varphi(\theta) \leq \exp(\varphi(\theta) - 1) \leq \exp(\beta \theta^{2})$,其中 $\beta = \sum_{n=1}^{\infty} 1/(2n)! \approx 0.586$.由(2.7.2)式可得,对于所有 $a \in [0, 1]$,有
\[
P(S_{n} \geq a n) \leq \exp(- n a^{2} / 4\beta).
\]
通常可进一步简化为 $\beta \leq \sum_{n=1}^{\infty} 2^{-n} = 1$.
    \end{crl}
    
    
    
    \begin{dfn}
        [Definition-of-Poisson-Process]
        {Poisson过程的定义}
        [Definition of Poisson Process]
        [gpt-4.1]
        
极限族随机变量 $N(A)$ 称为在 $(-\infty, \infty)$ 上以均值测度 $\mu$ 的 Poisson 过程,其中 $\mu(A) = \int_A \frac{\alpha}{2 | x |^{\alpha + 1}} dx < \infty$.

    \end{dfn}
    
    
    
    \begin{dfn}
        [Definition-of-Symmetric-Simple-Random-Walk]
        {对称简单随机游走的定义}
        [Definition of Symmetric Simple Random Walk]
        [gpt-4.1]
        设 $S_n$ 是对称简单随机游走,初值 $S_0 = 1$,即 $S_n = S_{n-1} + \xi_n$,其中 $\xi_1, \xi_2, \ldots$ 独立同分布,且 $P ( \xi_i = 1 ) = P ( \xi_i = -1 ) = 1 / 2$.
    \end{dfn}
    
    
    
    \begin{dfn}
        [Definition-of-Hitting-Time-N-and-Associated-Process]
        {随机游走首次达到零的时间及相关过程的定义}
        [Definition of Hitting Time N and Associated Process]
        [gpt-4.1]
        令 $N = \inf \{ n : S_n = 0 \}$,并定义 ${\cal X}_n = {\cal S}_{N \wedge n}$.
    \end{dfn}
    
    
    
    \begin{thm}
        [Doobs-Inequality]
        {Doob不等式}
        [Doob's Inequality]
        [gpt-4.1]
        设 $X_m$ 是一个次鞅,

\[
\bar{X}_n = \max_{0 \leq m \leq n} X_m^{+}
\]

$\lambda > 0$, 且 $A = \{\bar{X}_n \geq \lambda\}$.则有

\[
\lambda P(A) \leq E X_n 1_A \leq E X_n^{+}
\]

    \end{thm}
    
    
    
    \begin{prf}
        [Proof-of-Doobs-Inequality]
        {Doob不等式的证明}
        [Proof of Doob's Inequality]
        [gpt-4.1]
        令 $N = \inf\{m : X_m \geq \lambda \text{ 或 } m = n\}$.由于在 $A$ 上有 $X_N \geq \lambda$,

\[
\lambda P(A) \leq E X_N 1_A \leq E X_n 1_A
\]

第二个不等式由定理 4.4.1 推出 $E X_N \leq E X_n$,并且在 $A^c$ 上有 $X_N = X_n$.第二个不等式是显然的,因此证明完成.

    \end{prf}
    
    
    
    \begin{prf}
        [Proof-of-Theorem-2.5]
        {定理2.5的证明}
        [Proof of Theorem 2.5]
        [gpt-4.1]
        设 $A \in \mathcal{E}$.如Kolmogorov 0-1定律的证明中所述,我们将证明$A$与自身独立,即
\[
P(A) = P(A \cap A) = P(A) P(A)
\]
所以$P(A) \in \{ 0, 1 \}$.

令$A_{n} \in \sigma(X_{1}, \ldots, X_{n})$,使得
\[
P(A_{n} \Delta A) \to 0
\]
其中$A \Delta B = (A - B) \cup (B - A)$为对称差.

$A_{n}$可写作$\{ \omega : (\omega_{1}, \ldots, \omega_{n}) \in B_{n} \}$,其中$B_{n} \in S^{n}$.令
\[
\pi(j) = \begin{cases}
j + n & \text{if } 1 \leq j \leq n \\
j - n & \text{if } n + 1 \leq j \leq 2n \\
j & \text{if } j \geq 2n + 1
\end{cases}
\]
注意$\pi^{2}$为恒等变换(所以不用区分$\pi$或$\pi^{-1}$),且各坐标独立同分布,得
\[
P(\omega : \omega \in A_{n} \Delta A) = P(\omega : \pi \omega \in A_{n} \Delta A)
\]
又$\{ \omega : \pi \omega \in A \} = \{ \omega : \omega \in A \}$,因为$A$可置换,并且
\[
\{ \omega : \pi \omega \in A_{n} \} = \{ \omega : (\omega_{n+1}, \ldots, \omega_{2n}) \in B_{n} \}
\]
记$A_{n}^{\prime}$为上述事件,则
\[
\{ \omega : \pi \omega \in A_{n} \Delta A \} = \{ \omega : \omega \in A_{n}^{\prime} \Delta A \}
\]
结合(b)和(c)得
\[
P(A_{n} \Delta A) = P(A_{n}^{\prime} \Delta A)
\]
容易看出
\[
| P(B) - P(C) | \leq P(B \Delta C)
\]
因此(d)推出$P(A_{n}), P(A_{n}^{\prime}) \to P(A)$.

又$A - C \subset (A - B) \cup (B - C)$,类似地对$C - A$成立,故$A \Delta C \subset (A \Delta B) \cup (B \Delta C)$.

最后一个不等式、(d)及(a)推出
\[
P(A_{n} \Delta A_{n}^{\prime}) \leq P(A_{n} \Delta A) + P(A \Delta A_{n}^{\prime}) \to 0
\]
最后结果推出
\[
\begin{array}{rl}
0 \leq & P(A_{n}) - P(A_{n} \cap A_{n}^{\prime}) \\
\leq & P(A_{n} \cup A_{n}^{\prime}) - P(A_{n} \cap A_{n}^{\prime}) = P(A_{n} \Delta A_{n}^{\prime}) \to 0
\end{array}
\]
所以$P(A_{n} \cap A_{n}^{\prime}) \to P(A)$.

但$A_{n}$和$A_{n}^{\prime}$是独立的,因此
\[
P(A_{n} \cap A_{n}^{\prime}) = P(A_{n}) P(A_{n}^{\prime}) \to P(A)^{2}
\]
这说明$P(A) = P(A)^{2}$,从而证明了定理2.5.

    \end{prf}
    
    
    
    \begin{xmp}
        [Example-of-Linear-Martingale]
        {线性鞅的例子}
        [Example of Linear Martingale]
        [gpt-4.1]
        
例 4.2.1 (线性鞅) 如果 $\mu = E \xi _ { i } = 0$,则 $S _ { n }$,$n \geq 0$,是关于 ${ \mathcal { F } } _ { n }$ 的鞅.

    \end{xmp}
    
    
    
    \begin{prf}
        [Proof-of-Linear-Martingale]
        {线性鞅的证明}
        [Proof of Linear Martingale]
        [gpt-4.1]
        
为证此命题,注意到 $S _ { n } \in \mathcal { F } _ { n }$,$E | S _ { n } | < \infty$,且 $\xi _ { n + 1 }$ 与 ${ \mathcal { F } } _ { n }$ 独立,因此利用条件期望的线性性,(4.1.1)以及例 4.1.4,有

\[
E ( S _ { n + 1 } | { \mathcal { F } } _ { n } ) = E ( S _ { n } | { \mathcal { F } } _ { n } ) + E ( \xi _ { n + 1 } | { \mathcal { F } } _ { n } ) = S _ { n } + E \xi _ { n + 1 } = S _ { n }
\]

    \end{prf}
    
    
    
    \begin{ppt}
        [Supermartingale-Property-of-Linear-Martingale]
        {线性鞅的超鞅性质}
        [Supermartingale Property of Linear Martingale]
        [gpt-4.1]
        
如果 $\mu \leq 0$,则上述计算表明 $E ( S _ { n + 1 } | { \mathcal { F } } _ { n } ) \leq S _ { n }$,即 $S _ { n }$ 是超鞅.

    \end{ppt}
    
    
    
    \begin{thm}
        [Martingale-Property-of-Sums-of-Independent-Zero-Mean-Random-Variables]
        {独立零均值随机变量之和的鞅性质}
        [Martingale Property of Sums of Independent Zero-Mean Random Variables]
        [gpt-4.1]
        
设 $S_n = \xi_1 + \cdots + \xi_n$,其中各 $\xi_m$ 独立,且 $E \xi_m = 0$,$\sigma_m^2 = E \xi_m^2 < \infty$,则 $\{S_n\}$ 构成一个鞅.

    \end{thm}
    
    
    
    \begin{thm}
        [Submartingale-Property-of-Squared-Martingale-Sequence]
        {鞅平方序列的次鞅性质}
        [Submartingale Property of Squared Martingale Sequence]
        [gpt-4.1]
        
如果 $\{S_n\}$ 是鞅,则 $X_n = S_n^2$ 是次鞅(submartingale).

    \end{thm}
    
    
    
    \begin{dfn}
        [Definition-of-Stable-Law]
        {稳定分布的定义}
        [Definition of Stable Law]
        [gpt-4.1]
        一个随机变量 $Y$ 被称为具有稳定分布,如果对于每个整数 $k > 0$,都存在常数 $a_k$ 和 $b_k$,使得若 $Y_1, \dots, Y_k$ 是独立同分布且与 $Y$ 分布相同的随机变量,则 $(Y_1 + \cdots + Y_k - b_k)/a_k \stackrel{d}{=} Y$.
    \end{dfn}
    
    
    
    \begin{dfn}
        [Definition-of-the-Wealth-Process-for-an-Insurance-Company]
        {保险公司财富过程的定义}
        [Definition of the Wealth Process for an Insurance Company]
        [gpt-4.1]
        
设一家保险公司以速率 $c$ 收取保费,并在泊松过程 $N_t$ (速率为1)的到达时刻遭遇理赔.如果初始资金为 $x$,则其在时间 $t$ 的财富为

\[
W_{x}(t) = x + ct - \sum_{i=1}^{N_t} Y_i
\]

其中 $Y_1, Y_2, \ldots$ 独立同分布,分布为 $G$,均值为 $\mu$.

    \end{dfn}
    
    
    
    \begin{dfn}
        [Definition-of-the-Probability-of-Non-Ruin]
        {不破产概率的定义}
        [Definition of the Probability of Non-Ruin]
        [gpt-4.1]
        
记
\[
R(x) = P(W_{x}(t) \geq 0~\mathrm{for~all~} t)
\]
即表示以初始资金 $x$ 永不破产的概率.

    \end{dfn}
    
    
    
    \begin{thm}
        [Renewal-Equation-for-Probability-of-Non-Ruin]
        {不破产概率的更新方程}
        [Renewal Equation for Probability of Non-Ruin]
        [gpt-4.1]
        
不破产概率 $R(w)$ 满足如下更新(renewal)方程:

\[
R(w) = R(0) + \int_{0}^{w} R(w-y) \frac{1-G(y)}{c} dy
\]

其中 $G(y)$ 是理赔额度的分布函数,$c$ 是收款速率.

    \end{thm}
    
    
    
    \begin{thm}
        [Thinning-Theorem-for-Nonhomogeneous-Poisson-Process]
        {非齐次泊松过程的稀疏化定理}
        [Thinning Theorem for Nonhomogeneous Poisson Process]
        [gpt-4.1]
        
设有一个泊松过程,速率为 $\lambda$,我们以概率 $p(s)$ 保留落在时刻 $s$ 的点.那么,结果是一个速率为 $\lambda p(s)$ 的非齐次泊松过程.

    \end{thm}
    
    
    
    \begin{thm}
        [Equivalence-of-Limit-Theorem-and-Stable-Law]
        {极限定理与稳定分布的等价性}
        [Equivalence of Limit Theorem and Stable Law]
        [gpt-4.1]
        $Y$ 是序列 $(X_1 + \cdots + X_k - b_k)/a_k$ 的极限,当且仅当 $Y$ 具有稳定分布.
    \end{thm}
    
    
    
    \begin{thm}
        [Theorem-on-Upper-Bound-of-Maximum-Deviation-Probability]
        {最大偏差概率的上界定理}
        [Theorem on Upper Bound of Maximum Deviation Probability]
        [gpt-4.1]
        
设 $|\xi_{m}| \leq K$,利用定理 4.1 可得

\[
P \left( \max_{1 \leq m \leq n} |S_{m}| \leq x \right) \leq \frac{(x + K)^2}{\operatorname{var}(S_{n})}
\]

    \end{thm}
    
    
    
    \begin{ppt}
        [Property-of-Logarithmic-Inequality]
        {对数不等式的性质}
        [Property of Logarithmic Inequality]
        [gpt-4.1]
        对于任意 $a, b > 0$,有
\[
a \log b \leq a \log a + \frac{b}{e} \leq a \log^{+} a + \frac{b}{e}.
\]

    \end{ppt}
    
    
    
    \begin{thm}
        [Submartingale-Expectation-Inequality-under-Stopping-Time-Truncation]
        {关于停时截断的次鞅期望不等式}
        [Submartingale Expectation Inequality under Stopping Time Truncation]
        [gpt-4.1]
        
设 $X$ 是次鞅,$N$ 是停时,则 $X_{N \wedge n}$ 也是次鞅,因而有
\[
E X_0 = E X_{N \wedge 0} \leq E X_{N \wedge k} \leq E X_N
\]

    \end{thm}
    
    
    
    \begin{thm}
        [Predictable-Process-and-Submartingale-Difference-Expectation-Inequality]
        {可预测过程与次鞅差异的期望不等式}
        [Predictable Process and Submartingale Difference Expectation Inequality]
        [gpt-4.1]
        
令 $K_n = 1_{\{N < n\}} = 1_{\{N \leq n - 1\}}$,则 $K_n$ 是可预测过程,且根据定理 $(K \cdot X)_n = X_n - X_{N \wedge n}$ 是次鞅,因此
\[
E X_k - E X_N = E (K \cdot X)_k \geq E (K \cdot X)_0 = 0
\]

    \end{thm}
    
    
    
    \begin{dfn}
        [Definition-of-Random-Variable-T]
        {随机变量T的定义}
        [Definition of Random Variable T]
        [gpt-4.1]
        
设 $T = \operatorname* { inf } \{ n : S _ { n } = - 1 \}$.

    \end{dfn}
    
    
    
    \begin{crl}
        [S-T-Non-zero-Mean-Implies-Infinite-ET]
        {S\_T不均值为0导致ET的无穷大}
        [S_T Non-zero Mean Implies Infinite ET]
        [gpt-4.1]
        
由于 $S _ { T } = - 1$ 不具有均值 0, 得出 $E T ^ { 1 / 2 } = \infty$.

    \end{crl}
    
    
    
    \begin{prf}
        [Proof-of-Etemadis-Theorem]
        {Etemadi定理的证明}
        [Proof of Etemadi's Theorem]
        [gpt-4.1]
        由于 $X_{n}^{+}, n \geq 1$ 和 $X_{n}^{-}, n \geq 1$ 满足定理的假设且 $X_{n} = X_{n}^{+} - X_{n}^{-}$,所以我们可以无损一般性地假设 $X_{n} \geq 0$.
首先,类似于定理 2.3.9 的证明,先对一个子序列证明结论,然后利用单调性控制中间值.令 $\alpha > 1$ 且 $k(n) = [\alpha^{n}]$.由Chebyshev不等式可得,若 $\epsilon > 0$,
\[
\sum_{n = 1}^{\infty} P \left( | T_{k(n)} - E T_{k(n)} | > \epsilon k(n) \right) \leq \epsilon^{-2} \sum_{n = 1}^{\infty} \mathrm{var}(T_{k(n)}) / k(n)^{2}
= \epsilon^{-2} \sum_{n = 1}^{\infty} k(n)^{-2} \sum_{m = 1}^{k(n)} \mathrm{var}(Y_{m})
= \epsilon^{-2} \sum_{m = 1}^{\infty} \mathrm{var}(Y_{m}) \sum_{n : k(n) \geq m} k(n)^{-2}
\]
其中用到了Fubini定理交换了两个非负项的求和顺序.由于 $k(n) = [\alpha^{n}]$ 且 $[\alpha^{n}] \geq \alpha^{n} / 2$,对几何级数求和且首项 $\leq m^{-2}$,得到:
\[
\sum_{n : \alpha^{n} \geq m} [\alpha^{n}]^{-2} \leq 4 \sum_{n : \alpha^{n} \geq m} \alpha^{-2n} \leq 4 (1 - \alpha^{-2})^{-1} m^{-2}
\]
综上可得
\[
\sum_{n = 1}^{\infty} P \left( | T_{k(n)} - E T_{k(n)} | > \epsilon k(n) \right) \leq 4 (1 - \alpha^{-2})^{-1} \epsilon^{-2} \sum_{m = 1}^{\infty} E (Y_{m}^{2}) m^{-2} < \infty
\]
由引理 2.4.3 可知.由于 $\epsilon$ 任意,有 $(T_{k(n)} - E T_{k(n)}) / k(n) \to 0$ a.s..由控测收敛定理知 $E Y_{k} \to E X_{1}$,故 $E T_{k(n)} / k(n) \to E X_{1}$,即 $T_{k(n)} / k(n) \to E X_{1}$ a.s..
对中间值,若 $k(n) \leq m < k(n + 1)$,
\[
\frac{T_{k(n)}}{k(n + 1)} \leq \frac{T_{m}}{m} \leq \frac{T_{k(n + 1)}}{k(n)}
\]
(此处用到 $Y_{i} \geq 0$),记 $k(n) = [\alpha^{n}]$,有 $k(n + 1) / k(n) \to \alpha$,因此
\[
\frac{1}{\alpha} E X_{1} \leq \liminf_{m \to \infty} T_{m} / m \leq \limsup_{m \to \infty} T_{m} / m \leq \alpha E X_{1}
\]
由于 $\alpha > 1$ 任意,证毕.

    \end{prf}
    
    
    
    \begin{prf}
        [First-Proof-of-the-Poisson-Limit-Theorem]
        {泊松极限定理的第一个证明}
        [First Proof of the Poisson Limit Theorem]
        [gpt-4.1]
        令 $\varphi_{n,m}(t) = E(\exp(i t X_{n,m})) = (1 - p_{n,m}) + p_{n,m} e^{i t}$, $S_n = X_{n,1} + \cdots + X_{n,n}$.则
\[
E \exp(i t S_n) = \prod_{m = 1}^{n} (1 + p_{n,m}(e^{i t} - 1))
\]
令 $0 \leq p \leq 1$.有
\[
|\exp(p (e^{i t} - 1))| = \exp(p \operatorname{Re}(e^{i t} - 1)) \leq 1
\]
以及
\[
|1 + p(e^{i t} - 1)| \leq 1
\]
因为它在连接 $1$ 到 $e^{i t}$ 的线段上.利用引理 3.4.3(取 $\theta = 1$)以及引理 3.4.4(当 $\max_m p_{n,m} \leq 1/2$ 且 $|e^{i t} - 1| \leq 2$ 时成立),有
\[
\begin{array}{cl}
\displaystyle \left.\exp\left( \sum_{m = 1}^{n} p_{n,m}(e^{i t} - 1) \right) - \prod_{m = 1}^{n} \{ 1 + p_{n,m}(e^{i t} - 1) \} \right.\\
 & \displaystyle \leq \sum_{m = 1}^{n} \left.\exp(p_{n,m}(e^{i t} - 1)) - \{ 1 + p_{n,m}(e^{i t} - 1) \} \right.\\
 & \displaystyle \leq \sum_{m = 1}^{n} p_{n,m}^2 |e^{i t} - 1|^2
\end{array}
\]
由于 $|e^{i t} - 1| \leq 2$,最后一项有
\[
\leq 4 (\max_{1 \leq m \leq n} p_{n,m}) \sum_{m = 1}^{n} p_{n,m} \to 0
\]
由假设 (i) 和 (ii) 得到.最后的结论和 $\sum_{m = 1}^{n} p_{n,m} \to \lambda$ 推出
\[
E \exp(i t S_n) \to \exp(\lambda (e^{i t} - 1))
\]
至此证明完成.可参考例 3.3.4 关于泊松分布的特征函数,并应用定理 3.3.17.
    \end{prf}
    
    
    
    \begin{dfn}
        [Definition-of-Uniform-Distribution-$P-n$]
        {均匀分布 $P\_n$ 的定义}
        [Definition of Uniform Distribution $P_n$]
        [gpt-4.1]
        设 $P_{n}$ 表示在集合 $\{1, \ldots, n\}$ 上的均匀分布.
    \end{dfn}
    
    
    
    \begin{dfn}
        [Definition-of-Density]
        {密度的定义}
        [Definition of Density]
        [gpt-4.1]
        如果 $P_{\infty}(A) \equiv \lim P_{n}(A)$ 存在,则称该极限为 ${\textbf{\textsf{A}}} \subset \mathbf{Z}$ 的密度.
    \end{dfn}
    
    
    
    \begin{dfn}
        [Definition-of-Divisible-Set-$A-p$]
        {整除集合 $A\_p$ 的定义}
        [Definition of Divisible Set $A_p$]
        [gpt-4.1]
        设 $A_{p}$ 为所有能被 $p$ 整除的整数构成的集合.
    \end{dfn}
    
    
    
    \begin{ppt}
        [Density-Property-of-Prime-Divisible-Sets]
        {素数整除集合的密度性质}
        [Density Property of Prime Divisible Sets]
        [gpt-4.1]
        若 $p$ 为素数,则 $P_{\infty}(A_{p}) = 1/p$,且若 $q 
eq p$ 也是素数,则
\[
P_{\infty}(A_{p} \cap A_{q}) = 1/(pq) = P_{\infty}(A_{p}) P_{\infty}(A_{q})
\]

    \end{ppt}
    
    
    
    \begin{dfn}
        [Definition-of-Exchangeable-Sequence]
        {可交换序列的定义}
        [Definition of Exchangeable Sequence]
        [gpt-4.1]
        一列随机变量 $X _ { 1 } , X _ { 2 } , \dots$ 称为可交换的,如果对每个正整数 $n$ 及 $\{ 1 , \ldots , n \}$ 上的任意排列 $\pi$,$( X _ { 1 } , \ldots , X _ { n } )$ 与 $( X _ { \pi ( 1 ) } , \ldots , X _ { \pi ( n ) } )$ 具有相同的分布.
    \end{dfn}
    
    
    
    \begin{thm}
        [Corollary-of-Theorem-4.1]
        {定理 4.1 的推论}
        [Corollary of Theorem 4.1]
        [gpt-4.1]
        定理 4.1 蕴含了对任意停时 ${\cal N}$,都有 $E S_{N \wedge n} = 0$.
    \end{thm}
    
    
    
    \begin{thm}
        [Expectation-Result-for-Stopping-Time]
        {关于停时的期望结论}
        [Expectation Result for Stopping Time]
        [gpt-4.1]
        由定理 4.5.7 可知,若 $E N^{1/2} < \infty$,则 $E S_N = 0$.
    \end{thm}
    
    
    
    \begin{dfn}
        [Definition-of-the-Increasing-Process-Associated-with-a-Martingale]
        {与鞅相关的增加过程的定义}
        [Definition of the Increasing Process Associated with a Martingale]
        [gpt-4.1]
        
$A _ { n } = \sum _ { m = 1 } ^ { n } E ( X _ { m } ^ { 2 } | \mathcal { F } _ { m - 1 } ) - X _ { m - 1 } ^ { 2 } = \sum _ { m = 1 } ^ { n } E ( ( X _ { m } - X _ { m - 1 } ) ^ { 2 } | \mathcal { F } _ { m - 1 } )$

$A _ { n }$ 称为与 $X _ { n }$ 相关的增加过程.

    \end{dfn}
    
    
    
    \begin{thm}
        [Limit-Distribution-of-Sum-of-Independent-Random-Variables]
        {独立随机变量收敛的和的极限分布}
        [Limit Distribution of Sum of Independent Random Variables]
        [gpt-4.1]
        设 $X_n$ 和 $Y_n$ 对于 $1 \leq n \leq \infty$ 是独立的随机变量,且 $X_n \Rightarrow X_\infty$,$Y_n \Rightarrow Y_\infty$,则有 $X_n + Y_n \Rightarrow X_\infty + Y_\infty$.
    \end{thm}
    
    
    
    \begin{thm}
        [Orthogonality-of-Martingale-Increments]
        {鞅增量的正交性}
        [Orthogonality of Martingale Increments]
        [gpt-4.1]
        
设 $\{X_n\}$ 是一个鞅,且对所有 $n$ 有 $E X_n^2 < \infty$.若 $m \leq n$,且 $Y \in \mathcal{F}_m$ 且 $E Y^2 < \infty$,则有
\[
E \left( (X_n - X_m) Y \right) = 0
\]
因此若 $\ell < m < n$
\[
E \left( (X_n - X_m)(X_m - X_\ell) \right) = 0
\]

    \end{thm}
    
    
    
    \begin{prf}
        [Proof-of-Orthogonality-of-Martingale-Increments]
        {鞅增量正交性的证明}
        [Proof of Orthogonality of Martingale Increments]
        [gpt-4.1]
        
Cauchy-Schwarz 不等式说明 $E | (X_n - X_m) Y | < \infty$.利用 (4.1.5)、定理4.1.14及鞅的定义,
\[
E \left( (X_n - X_m) Y \right) = E \left[ E \left( (X_n - X_m) Y \mid \mathcal{F}_m \right) \right] = E \left[ Y E \left( X_n - X_m \mid \mathcal{F}_m \right) \right] = 0
\]

    \end{prf}
    
    
    
    \begin{prf}
        [Proof-of-Existence-and-Uniqueness-of-Solution-to-the-Renewal-Equation]
        {更新方程解的存在唯一性证明}
        [Proof of Existence and Uniqueness of Solution to the Renewal Equation]
        [gpt-4.1]
        证明 设 $U _ { n } ( A ) = \sum _ { m = 0 } ^ { n } P ( T _ { m } \in A )$,并定义
\[
H _ { n } ( t ) = \int _ { 0 } ^ { t } h ( t - s ) d U _ { n } ( s ) = \sum _ { m = 0 } ^ { n } \left( h * F ^ { m * } \right) ( t )
\]
其中,$F ^ { m * }$ 是 $T _ { m }$ 的分布,并且我们扩展了 $h$ 的定义,使得 $h ( r ) = 0$ 对所有 $r < 0$ 成立.

由上述表达式可知,
\[
H _ { n + 1 } = h + H _ { n } * F
\]
并且 $U ( t ) < \infty$ 蕴含 $U ( t ) - U _ { n } ( t ) \to 0$.

由于 $h$ 有界,
\[
| H _ { n } ( t ) - H ( t ) | \leq \| h \| _ { \infty } | U ( t ) - U _ { n } ( t ) |
\]
且 $H _ { n } ( t ) \to H ( t )$ 在有界区间上一致收敛.

为了估计卷积,有
\[
| H _ { n } * F ( t ) - H * F ( t ) | \le \sup _ { s \le t } | H _ { n } ( s ) - H ( s ) | \le \| h \| _ { \infty } | U ( t ) - U _ { n } ( t ) |
\]
因为 $U - U _ { n } = \sum _ { m = n + 1 } ^ { \infty } F ^ { m * }$ 在 $t$ 上递增.

令 $n \to \infty$,由 $H _ { n + 1 } = h + H _ { n } * F$ 可见 $H$ 是在有界区间上有界的更新方程的解.

为证明唯一性,若 $H _ { 1 }$ 和 $H _ { 2 }$ 为两个解,则 $K = H _ { 1 } - H _ { 2 }$ 满足 $K = K * F$.

若 $K$ 在有界区间上有界,反复迭代得 $K = K * F ^ { n * } \to 0$ 当 $n \to \infty$,故 $H _ { 1 } = H _ { 2 }$.

    \end{prf}
    
    
    
    \begin{thm}
        [Hewitt-Savage-0-1-Law]
        {Hewitt-Savage 0-1 定律}
        [Hewitt-Savage 0-1 Law]
        [gpt-4.1]
        设 $X_1, X_2, \dots$ 是独立同分布的随机变量,若 $A \in \mathcal{E}$,则 $P(A) \in \{0, 1\}$.
    \end{thm}
    
    
    
    \begin{xmp}
        [Example-of-Bilateral-Exponential-Distribution]
        {双边指数分布的例子}
        [Example of Bilateral Exponential Distribution]
        [gpt-4.1]
        
设 $X$ 的概率密度函数为
\[
f(x) = \frac{1}{2} e^{-|x|}, \qquad x \in (-\infty, \infty)
\]
其特征函数为
\[
\phi(t) = \frac{1}{1 + t^2}
\]

    \end{xmp}
    
    
    
    \begin{prf}
        [Proof-of-Bilateral-Exponential-Distribution-Characteristic-Function]
        {双边指数分布特征函数的证明}
        [Proof of Bilateral Exponential Distribution Characteristic Function]
        [gpt-4.1]
        
该结论可由引理 3 得出.

    \end{prf}
    
    
    
    \begin{thm}
        [Probability-Expression-for-Exchangeable-Binary-Random-Variables]
        {可交换二值随机变量的概率表达式}
        [Probability Expression for Exchangeable Binary Random Variables]
        [gpt-4.1]
        
若 $X_1, X_2, \dots$ 是可交换的,并且取值于 $\{0, 1\}$,则存在 $[0,1]$ 上的概率分布,使得

\[
P(X_1 = 1, \ldots, X_k = 1, X_{k+1} = 0, \ldots, X_n = 0) = \int_0^1 \theta^k (1 - \theta)^{n - k} dF(\theta)
\]

该结果对于关注统计学基础的人非常有用.

    \end{thm}
    
    
    
    \begin{thm}
        [Expectation-Upper-Bound-for-Submartingale]
        {关于次鞅的期望上界}
        [Expectation Upper Bound for Submartingale]
        [gpt-4.1]
        
设 $X_{n}$ 是一个次鞅,$\log^{+} x = \max( \log x , 0 )$,则有

\[
E \bar{X}_{n} \leq ( 1 - e^{-1} )^{-1} \{ 1 + E ( X_{n}^{+} \log^{+} ( X_{n}^{+} ) ) \}
\]

其中 $\bar{X}_{n}$ 表示 $X_{n}$ 的最大值,$X_{n}^{+} = \max(X_{n}, 0)$.

    \end{thm}
    
    
    
    \begin{dfn}
        [Definition-of-Positive-Logarithm-Function]
        {正对数函数的定义}
        [Definition of Positive Logarithm Function]
        [gpt-4.1]
        
$\log^{+} x = \max( \log x , 0 )$

    \end{dfn}
    
    
    
    \begin{thm}
        [Probability-Bound-for-the-Supremum-of-a-Supermartingale]
        {超鞅的上界概率估计}
        [Probability Bound for the Supremum of a Supermartingale]
        [gpt-4.1]
        若 $X_n \geq 0$ 是一个超鞅,则 $P(\sup X_n > \lambda) \leq E X_0 / \lambda$.
    \end{thm}
    
    
    
    \begin{thm}
        [Conditional-Expectation-Equality-for-Limit]
        {极限的条件期望等式}
        [Conditional Expectation Equality for Limit]
        [gpt-4.1]
        如果 $X_{-\infty} = \lim_{n \to -\infty} X_{n}$ 且 ${\mathcal{F}}_{-\infty} = \cap_{n} {\mathcal{F}}_{n}$,那么
\[
X_{-\infty} = E(X_{0} | \mathcal{F}_{-\infty}) .
\]

    \end{thm}
    
    
    
    \begin{thm}
        [Convergence-and-Uniform-Integrability-of-Martingales-and-Supermartingales-at-Stopping-Times]
        {关于可停止时间的超鞅与鞅的收敛与一致积分性}
        [Convergence and Uniform Integrability of Martingales and Supermartingales at Stopping Times]
        [gpt-4.1]
        Theorem 4.8.2 如果 $E|X_{N}| < \infty$ 且 $X_{n} 1_{(N > n)}$ 是一致可积的, 那么 $X_{N \wedge n}$ 是一致可积的, 因此 $E X_{0} \leq E X_{N}$.
    \end{thm}
    
    
    
    \begin{thm}
        [Theorem-on-Transformed-Process-Being-a-Martingale]
        {关于变换后过程为鞅的定理}
        [Theorem on Transformed Process Being a Martingale]
        [gpt-4.1]
        
若 $H_m = f(A_m)^{-1}$ 是有界且可预测的,则定理 4.2.8 蕴含
\[
Y_n \equiv (H \cdot X)_n = \sum_{m=1}^n \frac{X_m - X_{m-1}}{f(A_m)}
\]
是一个鞅.

    \end{thm}
    
    
    
    \begin{dfn}
        [Definition-of-Increasing-Process-of-Transformed-Process]
        {变换后过程的递增过程定义}
        [Definition of Increasing Process of Transformed Process]
        [gpt-4.1]
        
若 $B_n$ 是与 $Y_n$ 相关的递增过程,则
\[
\begin{array}{rl}
& B_{n+1} - B_n = E\left( (Y_{n+1} - Y_n)^2 \mid \mathcal{F}_n \right) \\
& \qquad = E\left( \left. \frac{(X_{n+1} - X_n)^2}{f(A_{n+1})^2} \right| \mathcal{F}_n \right) = \frac{A_{n+1} - A_n}{f(A_{n+1})^2}
\end{array}
\]

    \end{dfn}
    
    
    
    \begin{thm}
        [$L^p$-Convergence-Theorem]
        {$L^p$ 收敛定理}
        [$L^p$ Convergence Theorem]
        [gpt-4.1]
        若 $X_n$ 是鞅(martingale),且 $\sup E |X_n|^p < \infty$,其中 $p > 1$,则 $X_n \to X$ 几乎处处收敛并且在 $L^p$ 意义下收敛.
    \end{thm}
    
    
    
    \begin{prf}
        [Proof-of-the-$L^p$-Convergence-Theorem]
        {$L^p$ 收敛定理的证明}
        [Proof of the $L^p$ Convergence Theorem]
        [gpt-4.1]
        $(E (X_n^+))^p \leq (E |X_n|)^p \leq E |X_n|^p$,由鞅收敛定理 (4.2.11) 可知 $X_n \to X$ 几乎处处收敛.定理 4.4.4 的第二个结论得到

\[
E \left( \sup_{0 \leq m \leq n} |X_m| \right)^p \leq \left( \frac{p}{p-1} \right)^p E |X_n|^p
\]

令 $n \to \infty$,并用单调收敛定理,得到 $\sup |X_n| \in L^p$.由于 $|X_n - X|^p \leq (2 \sup |X_n|)^p$,由支配收敛定理知 $E |X_n - X|^p \to 0$.
    \end{prf}
    
    
    
    \begin{thm}
        [Poisson-Limit-Theorem-for-Sums-of-Independent-Integer-Valued-Random-Variables]
        {独立整数值随机变量和的泊松极限定理}
        [Poisson Limit Theorem for Sums of Independent Integer-Valued Random Variables]
        [gpt-4.1]
        
设 $X_{n,m}$, $1 \leq m \leq n$ 为独立的非负整数值随机变量,满足 $P(X_{n,m} = 1) = p_{n,m}$,$P(X_{n,m} \geq 2) = \epsilon_{n,m}$.若:

(i) $\sum_{m=1}^{n} p_{n,m} \to \lambda \in (0, \infty)$,

(ii) $\max_{1 \leq m \leq n} p_{n,m} \to 0$,

(iii) $\sum_{m=1}^{n} \epsilon_{n,m} \to 0$,

则若 $S_n = X_{n,1} + \cdots + X_{n,n}$,有 $S_n \Rightarrow Z$,其中 $Z$ 服从泊松分布 $\mathrm{Poisson}(\lambda)$.

    \end{thm}
    
    
    
    \begin{dfn}
        [Definition-of-Hyperplane-$H-c^i$]
        {超平面 $H\_c^i$ 的定义}
        [Definition of Hyperplane $H_c^i$]
        [gpt-4.1]
        令 $H_c^i = \{ x : x_i = c \}$ 表示第 $i$ 个坐标等于 $c$ 的超平面.
    \end{dfn}
    
    
    
    \begin{dfn}
        [Definition-of-Set-$D^i$]
        {集合 $D^i$ 的定义}
        [Definition of Set $D^i$]
        [gpt-4.1]
        对于每个 $i$,集合 $D^i = \{ c : P( X \in H_c^i ) > 0 \}$,即使得概率 $P( X \in H_c^i )$ 大于零的 $c$ 的集合.
    \end{dfn}
    
    
    
    \begin{ppt}
        [Set-$D^i$-is-at-most-countable]
        {集合 $D^i$ 至多可数}
        [Set $D^i$ is at most countable]
        [gpt-4.1]
        对于每个 $i$,集合 $H_c^i$ 两两不交,因此 $D^i = \{ c : P( X \in H_c^i ) > 0 \}$ 至多是可数集.
    \end{ppt}
    
    
    
    \begin{ppt}
        [Continuity-Condition-for-Distribution-Function-$F$]
        {分布函数 $F$ 的连续性条件}
        [Continuity Condition for Distribution Function $F$]
        [gpt-4.1]
        如果 $x$ 满足对所有 $i$ 均有 $x_i 
otin D^i$,则分布函数 $F$ 在 $x$ 处连续.
    \end{ppt}
    
    
    
    \begin{thm}
        [Convergence-of-Backwards-Martingale-in-$L^p$-Space]
        {后向鞅在 $L^p$ 空间中的收敛性}
        [Convergence of Backwards Martingale in $L^p$ Space]
        [gpt-4.1]
        
设一个后向鞅 $\{X_n\}$ 满足 $X_0 \in L^p$,则该后向鞅收敛于某极限,并且收敛发生在 $L^p$ 空间中.

    \end{thm}
    
    
    
    \begin{prf}
        [Proof-of-Expectation-Bound-for-Stopped-Random-Variable]
        {关于停时随机变量的期望上界证明}
        [Proof of Expectation Bound for Stopped Random Variable]
        [gpt-4.1]
        证明 令 $a > 0$.由于 $A_{n+1} \in \mathcal{F}_n$,$N = \operatorname*{inf} \{ n : A_{n+1} > a^2 \}$ 是一个停时.应用定理 4.5.1 于 $X_{N \wedge n}$,并注意到 $A_{N \wedge n} \leq a^2$,得到
\[
E \left( \operatorname*{sup}_n | X_{N \wedge n} |^2 \right) \leq 4 a^2
\]
所以 $L^2$ 收敛定理 4.4.6 蕴含 $\lim X_{N \wedge n}$ 存在且有限.由于 $a$ 是任意的,所需结论成立.□

    \end{prf}
    
    
    
    \begin{lma}
        [Lemma-on-$a	heta-\kappa	heta>0$-for-$a>\mu$]
        {对$a>\mu$时$a	heta-\kappa(	heta)>0$的引理}
        [Lemma on $a	heta-\kappa(	heta)>0$ for $a>\mu$]
        [gpt-4.1]
        
如果 $a > \mu$ 并且 $\theta > 0$ 很小,那么 $a \theta - \kappa( \theta ) > 0$.

    \end{lma}
    
    
    
    \begin{prf}
        [Proof-of-Lemma-2.7.2]
        {引理2.7.2的证明}
        [Proof of Lemma 2.7.2]
        [gpt-4.1]
        
$\kappa( 0 ) = \log \varphi( 0 ) = 0$,因此只需证明:(i) $\kappa$ 在 0 处连续,(ii) 在 $( 0 , \theta_{+} )$ 上可微,(iii) 当 $\theta \to 0$ 时,$\kappa^{\prime} ( \theta ) \to \mu$.

于是
\[
a \theta - \kappa ( \theta ) = \int_{0}^{\theta} a - \kappa^{\prime} ( x ) dx > 0
\]
对于足够小的 $\theta$ 成立.

    \end{prf}
    
    
    
    \begin{thm}
        [Theorem-on-Increasing-Function-and-Limit]
        {关于递增函数与极限的定理}
        [Theorem on Increasing Function and Limit]
        [gpt-4.1]
        
设 $f \geq 1$ 是递增函数,且 $\int_{0}^{\infty} f(t)^{-2} dt < \infty$.则在事件 $\{A_{\infty} = \infty\}$ 上,有 $X_n / f(A_n) \to 0$.

    \end{thm}
    
    
    
    \begin{thm}
        [Theorem-Expectation-Non-Increase-for-Nonnegative-Supermartingale-at-Stopping-Time]
        {非负超鞅的停时期望不增定理}
        [Theorem: Expectation Non-Increase for Nonnegative Supermartingale at Stopping Time]
        [gpt-4.1]
        如果 $\{X_{n}\}$ 是非负超鞅,$N \leq \infty$ 是一个停时,则有 $E X_{0} \geq E X_{N}$,其中 $X_{\infty} = \lim X_{n}$,极限存在由定理 4 保证.
    \end{thm}
    
    
    
    \begin{crl}
        [Limit-of-Pairwise-Squared-Differences-for-i.i.d.-Random-Variables]
        {独立同分布随机变量差平方和的极限}
        [Limit of Pairwise Squared Differences for i.i.d. Random Variables]
        [gpt-4.1]
        
设 $X_{1}, X_{2}, \dots$ 是独立同分布(i.i.d.)的随机变量,满足 $E X_{i} = \mu$ 且 $\operatorname{var}(X_{i}) = \sigma^{2} < \infty$,则有
\[
\binom{n}{2}^{-1} \sum_{1 \leq i < j \leq n} (X_{i} - X_{j})^{2} \to 2 \sigma^{2}
\]
当 $n \to \infty$ 时成立.

    \end{crl}
    
    
    
    \begin{thm}
        [Expectation-Inequality-for-Uniformly-Integrable-Submartingales-at-Stopping-Times]
        {一致可积次超鞅的停时期取值的期望不等式}
        [Expectation Inequality for Uniformly Integrable Submartingales at Stopping Times]
        [gpt-4.1]
        
若 $X_{n}$ 是一致可积次超鞅,则对于任意停时 $N \leq \infty$,有
\[
E X_{0} \leq E X_{N} \leq E X_{\infty}
\]
其中 $X_{\infty} = \lim X_{n}$.

    \end{thm}
    
    
    
    \begin{dfn}
        [Definition-of-Asymmetric-Simple-Random-Walk]
        {非对称简单随机游走的定义}
        [Definition of Asymmetric Simple Random Walk]
        [gpt-4.1]
        非对称简单随机游走(Asymmetric simple random walk)指的是这样一种随机过程:对于每一步 $\xi_{i}$,有 $P(\xi_{i} = 1) = p$,$P(\xi_{i} = -1) = q \equiv 1-p$,其中 $p 
eq q$.
    \end{dfn}
    
    
    
    \begin{thm}
        [Walds-Second-Equation]
        {Wald 第二公式}
        [Wald's Second Equation]
        [gpt-4.1]
        
设 $S_n = \xi_1 + \cdots + \xi_n$,其中 $\xi_i$ 独立且 $E \xi_i = 0$,$\operatorname{var}(\xi_i) = \sigma^2$.若 $T$ 是一个满足 $E T < \infty$ 的停时,则有
\[
E S_T^2 = \sigma^2 E T.
\]

    \end{thm}
    
    
    
    \begin{ppt}
        [Property-of-Conditional-Distribution-for-Truncated-Stable-Distribution]
        {截断稳定分布的条件概率分布性质}
        [Property of Conditional Distribution for Truncated Stable Distribution]
        [gpt-4.1]
        
对于稳定分布的随机变量 $X_1$,当 $x \ge \epsilon$ 时,
\[
1 - F_{n}^{\epsilon}(x) = P(X_{1} / n^{1/\alpha} > x \mid |X_{1}| / n^{1/\alpha} > \epsilon) = \frac{x^{-\alpha}}{2 \epsilon^{-\alpha}}
\]

    \end{ppt}
    
    
    
    \begin{ppt}
        [Property-of-Characteristic-Function-for-Truncated-Distribution]
        {截断分布的特征函数性质}
        [Property of Characteristic Function for Truncated Distribution]
        [gpt-4.1]
        
上述分布与 $\epsilon X_{1}$ 的分布相同,因此若 $\varphi(t) = E \exp(i t X_{1})$,则分布 $F_{n}^{\epsilon}$ 的特征函数为 $\varphi(\epsilon t)$.

    \end{ppt}
    
    
    
    \begin{thm}
        [Integral-Representation-Theorem-for-Conditional-Expectation]
        {条件期望的积分表示定理}
        [Integral Representation Theorem for Conditional Expectation]
        [gpt-4.1]
        
设 $\mu(\omega, A)$ 是 $X$ 关于 $\mathcal{F}$ 的 regular conditional distribution (r.c.d.).若 $f : (S, S) \to (\mathbb{R}, \mathcal{R})$ 且 $E|f(X)| < \infty$,则有
\[
E(f(X)|\mathcal{F}) = \int \mu(\omega, dx) f(x) \quad a.s.
\]

    \end{thm}
    
    
    
    \begin{prf}
        [Proof-of-Integral-Representation-for-Conditional-Expectation]
        {条件期望积分表示的证明}
        [Proof of Integral Representation for Conditional Expectation]
        [gpt-4.1]
        
若 $f = 1_{A}$,该结论由定义直接得出.由线性性可扩展至简单函数 $f$,再由单调收敛定理推广至非负函数 $f$.最后,将一般的 $f$ 表示为 $f = f^{+} - f^{-}$,即可得出结论.

    \end{prf}
    
    
    
    \begin{prf}
        [Proof-of-Uniqueness-of-Distribution-via-Characteristic-Function-Consistency]
        {特征函数一致性与分布唯一性证明}
        [Proof of Uniqueness of Distribution via Characteristic Function Consistency]
        [gpt-4.1]
        
设 $F$ 为任意分布,具有矩 $\mu_k$,令 $
u_k = \int |x|^k dF(x)$.Cauchy–Schwarz 不等式给出 $
u_{2k+1}^2 \leq \mu_{2k}\mu_{2k+2}$,因此
\[
\lim_{k\to\infty} 
u_k^{1/k}/k = r < \infty
\]
在引理 3.3.2 取 $x = tX$ 并乘以 $e^{i\theta X}$,有
\[
\left| e^{i\theta X} \left( e^{itX} - \sum_{m=0}^{n-1} \frac{(itX)^m}{m!} \right) \right| \leq \frac{|tX|^n}{n!}
\]
取期望,并用练习 3.3.18 得
\[
\left| \varphi(\theta + t) - \varphi(\theta) - t\varphi'(\theta) \ldots - \frac{t^{n-1}}{(n-1)!}\varphi^{(n-1)}(\theta) \right| \leq \frac{|t|^n}{n!} 
u_n
\]
利用上式、$
u_k \leq (r+\epsilon)^k k^k$($k$ 大时成立)和 $e^k \geq k^k / k!$(将左式展开为幂级数),可知对任意 $\theta$,
\[
\varphi(\theta + t) = \varphi(\theta) + \sum_{m=1}^{\infty} \frac{t^m}{m!} \varphi^{(m)}(\theta) \quad \text{当 } |t| < 1/er
\]
设 $G$ 为另一具有上述矩的分布,$\psi$ 为其特征函数.由 $\varphi(0) = \psi(0) = 1$,结合 $(*)$ 和归纳法,有 $\varphi(t) = \psi(t)$ 当 $|t| \leq k/3r$ 对所有 $k$ 成立,因此两个特征函数一致,分布相等.

    \end{prf}
    
    
    
    \begin{dfn}
        [Definition-of-Backwards-Martingale]
        {反向鞅的定义}
        [Definition of Backwards Martingale]
        [gpt-4.1]
        反向鞅(有些作者称为'逆向鞅')是以负整数为指标的鞅,即:$\{X_n, n \leq 0\}$,适应于递增的 $\sigma$-域序列 $\{\mathcal{F}_n\}$,满足
\[
E(X_{n+1} | \mathcal{F}_n) = X_n \quad \text{对所有 } n \leq -1
\]

    \end{dfn}
    
    
    
    \begin{thm}
        [Existence-Theorem-for-the-Limit-of-Backwards-Martingales]
        {反向鞅的极限存在性定理}
        [Existence Theorem for the Limit of Backwards Martingales]
        [gpt-4.1]
        定理 4.7.1:$X_{-\infty} = \lim_{n \to -\infty} X_n$ 存在,且存在于 $L^1$ 空间中.
    \end{thm}
    
    
    
    \begin{thm}
        [Conditional-Probability-Formula-for-Exchangeable-Binary-Variables]
        {可交换二元变量条件概率公式}
        [Conditional Probability Formula for Exchangeable Binary Variables]
        [gpt-4.1]
        
如果 $X_{1}, X_{2}, \dotsc \in \{0, 1\}$ 是可交换的,则有
\[
P(X_{1} = 1, \ldots, X_{k} = 1 \mid S_{n} = m) = \frac{\binom{n - k}{n - m}}{\binom{n}{m}}
\]
其中 $S_{n} = X_{1} + X_{2} + \cdots + X_{n}$.

    \end{thm}
    
    
    
    \begin{xmp}
        [An-Example-of-Bernard-Friedmans-Urn-Model]
        {Bernard Friedman 抽彩球模型的一个例子}
        [An Example of Bernard Friedman's Urn Model]
        [gpt-4.1]
        
考虑 Polya 抽彩球模型(详见第 5.3 节)的一个变体:每次抽取一球后,向所抽中颜色的球中加入 $a$ 个球,向另一种颜色的球中加入 $b$ 个球,其中 $a \geq 0$ 且 $b > 0$.我们将证明:若初始时有 $g$ 个绿色球和 $r$ 个红色球,且 $g, r > 0$,则绿色球的比例 $g_n \to 1/2$.

    \end{xmp}
    
    
    
    \begin{thm}
        [Theorem-on-Limit-Distribution-of-Markov-Chain]
        {关于马氏链极限分布的定理}
        [Theorem on Limit Distribution of Markov Chain]
        [gpt-4.1]
        
设 $X_{n} \in [0,1]$ 是关于滤波族 ${\mathcal{F}}_{n}$ 的适应过程,且对于 $\alpha,\ \beta > 0$ 满足 $\alpha + \beta = 1$,
有
\[
P(X_{n+1} = \alpha + \beta X_{n} \mid \mathcal{F}_{n}) = X_{n},\quad P(X_{n+1} = \beta X_{n} \mid \mathcal{F}_{n}) = 1 - X_{n}
\]
则
\[
P\left(\lim_{n\to\infty} X_{n} = 0\ \text{或}\ 1\right) = 1
\]
且若 $X_{0} = \theta$,则
\[
P\left(\lim_{n\to\infty} X_{n} = 1\right) = \theta
\]

    \end{thm}
    
    
    
    \begin{prf}
        [Proof-that-$Y-n-=---X-n-\leq-0$-is-a-Submartingale]
        {关于$Y\_n = - X\_n \leq 0$是次鞅的证明}
        [Proof that $Y_n = - X_n \leq 0$ is a Submartingale]
        [gpt-4.1]
        证明 $Y_n = - X_n \leq 0$ 是一个次鞅,且 $E Y_n^+ = 0$.
    \end{prf}
    
    
    
    \begin{prf}
        [Proof-of-Inequality-via-Fatous-Lemma]
        {由Fatou引理推导不等式的证明}
        [Proof of Inequality via Fatou's Lemma]
        [gpt-4.1]
        由于 $E X_0 \geq E X_n$,该不等式可由Fatou引理推出.
    \end{prf}
    
    
    
    \begin{xmp}
        [A-Counterexample-Showing-Assumptions-Do-Not-Guarantee-$L^1$-Convergence]
        {一个反例说明定理假设不保证$L^1$收敛}
        [A Counterexample Showing Assumptions Do Not Guarantee $L^1$ Convergence]
        [gpt-4.1]
        例 4.2.13 第一个反例说明定理 4.2.12 (或 4.2.11) 的假设并不保证在 $L^1$ 下收敛.
    \end{xmp}
    
    
    
    \begin{dfn}
        [Definition-of-a-Sequence-of-Random-Variables]
        {随机变量序列的定义}
        [Definition of a Sequence of Random Variables]
        [gpt-4.1]
        设 $\xi_{1}, \xi_{2}, \ldots$ 是独立同分布的随机变量.
    \end{dfn}
    
    
    
    \begin{dfn}
        [Definition-of-Random-Walk]
        {随机游走的定义}
        [Definition of Random Walk]
        [gpt-4.1]
        设 $S_{n} = S_{0} + \xi_{1} + \cdots + \xi_{n}$,其中 $S_{0}$ 是常数.
    \end{dfn}
    
    
    
    \begin{dfn}
        [Definition-of-Natural-Filtration]
        {自然过滤的定义}
        [Definition of Natural Filtration]
        [gpt-4.1]
        设 $\mathcal{F}_{n} = \sigma(\xi_{1}, \ldots, \xi_{n})$,对所有 $n \geq 1$,并取 ${\mathcal{F}}_{0} = \{\emptyset, \Omega\}$.
    \end{dfn}
    
    
    
    \begin{dfn}
        [Definition-of-Conditional-Distribution-$F-n^{\epsilon}$]
        {条件分布 $F\_n^{\epsilon}$ 的定义}
        [Definition of Conditional Distribution $F_n^{\epsilon}$]
        [gpt-4.1]
        
设 $|I_n(\epsilon)| = m$, 则 $\hat{S}_n(\epsilon)/a_n$ 是 $m$ 个服从分布 $F_n^{\epsilon}$ 的独立随机变量的和,其中分布 $F_n^{\epsilon}$ 满足:

\[
\begin{array}{rl}
1 - F_n^{\epsilon}(x) &= P(X_1 / a_n > x \mid |X_1| / a_n > \epsilon) = \theta x^{-\alpha} / \epsilon^{-\alpha} \\
F_n^{\epsilon}(-x) &= P(X_1 / a_n < -x \mid |X_1| / a_n > \epsilon) = (1 - \theta)|x|^{-\alpha} / \epsilon^{-\alpha}
\end{array}
\]

其中 $x \geq \epsilon$.

    \end{dfn}
    
    
    
    \begin{thm}
        [Generating-Function-Formula-for-the-Time-of-Gamblers-Ruin]
        {赌徒破产时间的生成函数公式}
        [Generating Function Formula for the Time of Gambler's Ruin]
        [gpt-4.1]
        
设 $V_0$ 为赌徒破产的时间,$\phi(\theta)$ 为相关函数,$x$ 为初始财富.若 $0 < s < 1$,求解方程 $\phi(\theta) = 1/s$,则有
\[
E_{x}(s^{V_{0}}) = \left( \frac{1 - \sqrt{1 - 4pqs^{2}}}{2ps} \right)^{x}
\]

    \end{thm}
    
    
    
    \begin{thm}
        [Kolmogorovs-Theorem-Characterization-of-Infinitely-Divisible-Distributions]
        {Kolmogorov定理-无限可分分布的特征函数表示}
        [Kolmogorov's Theorem: Characterization of Infinitely Divisible Distributions]
        [gpt-4.1]
        设随机变量 $Z$ 具有均值为 $\boldsymbol{\theta}$、有限方差的无限可分分布,当且仅当其特征函数满足
\[
\log{ \varphi(t) } = \int ( e^{i t x} - 1 - i t x ) x^{-2} 
u(dx)
\]
其中被积函数在 $x=0$ 处取 $- t^{2} / 2$,$
u$ 称为规范测度,且 $\operatorname{var}(Z) = 
u({\bf R})$.

    \end{thm}
    
    
    
    \begin{dfn}
        [Definition-of-Lipschitz-Continuous-Functions]
        {Lipschitz连续函数的定义}
        [Definition of Lipschitz Continuous Functions]
        [gpt-4.1]
        $f$ 称为Lipschitz连续的,如果存在常数 $K$,使得对所有 $0 \leq s, t < 1$,均有
\[
| f ( t ) - f ( s ) | \leq K | t - s |.
\]
    \end{dfn}
    
    
    
    \begin{lma}
        [Error-Estimate-for-Taylor-Expansion-of-Exponential-Function]
        {三角函数泰勒展开的误差估计}
        [Error Estimate for Taylor Expansion of Exponential Function]
        [gpt-4.1]
        
对于任意实数 $x$、非负整数 $n$,有如下估计:
\[
\left| e^{ix} - \sum_{m=0}^n \frac{(ix)^m}{m!} \right| \leq \min\left( \frac{|x|^{n+1}}{(n+1)!}, \frac{2|x|^n}{n!} \right)
\]
其中,右侧的第一项适用于 $x$ 较小的情况,第二项适用于 $x$ 较大的情况.

    \end{lma}
    
    
    
    \begin{prf}
        [Proof-of-Error-Estimate-for-Taylor-Expansion-of-Exponential-Function]
        {三角函数泰勒展开误差估计的证明}
        [Proof of Error Estimate for Taylor Expansion of Exponential Function]
        [gpt-4.1]
        
通过分部积分法,有
\[
\int_0^x (x-s)^n e^{is} ds = \frac{x^{n+1}}{n+1} + \frac{i}{n+1} \int_0^x (x-s)^{n+1} e^{is} ds
\]
当 $n = 0$ 时,
\[
\int_0^x e^{is} ds = x + i \int_0^x (x-s) e^{is} ds
\]
左边为 $(e^{ix} - 1)/i$,整理得
\[
e^{ix} = 1 + ix + i^2 \int_0^x (x-s) e^{is} ds
\]
对 $n = 1$,递推得
\[
e^{ix} = 1 + ix + \frac{i^2 x^2}{2} + \frac{i^3}{2} \int_0^x (x-s)^2 e^{is} ds
\]
迭代后可得
\[
e^{ix} - \sum_{m=0}^n \frac{(ix)^m}{m!} = \frac{i^{n+1}}{n!} \int_0^x (x-s)^n e^{is} ds
\]
对右侧的误差项估计,因 $|e^{is}| \leq 1$,
\[
\left| \frac{i^{n+1}}{n!} \int_0^x (x-s)^n e^{is} ds \right| \leq \frac{|x|^{n+1}}{(n+1)!}
\]
此估计适用于 $x$ 较小的情况.

对于较大的 $x$,分部积分得
\[
\frac{i}{n} \int_0^x (x-s)^n e^{is} ds = -\frac{x^n}{n} + \int_0^x (x-s)^{n-1} e^{is} ds
\]
注意到 $x^n/n = \int_0^x (x-s)^{n-1} ds$,因此
\[
\frac{i^{n+1}}{n!} \int_0^x (x-s)^n e^{is} ds = \frac{i^n}{(n-1)!} \int_0^x (x-s)^{n-1} (e^{is} - 1) ds
\]
由于 $|e^{is} - 1| \leq 2$,有
\[
\left| \frac{i^{n+1}}{n!} \int_0^x (x-s)^n e^{is} ds \right| \leq \left| \frac{2}{(n-1)!} \int_0^x (x-s)^{n-1} ds \right| \leq \frac{2|x|^n}{n!}
\]
综上,得到所需结论.

    \end{prf}
    
    
    
    \begin{xmp}
        [An-Example-of-Random-Variables-and-Limits-in-a-Family-of-Sets]
        {关于随机变量和极限在集合族中的例子}
        [An Example of Random Variables and Limits in a Family of Sets]
        [gpt-4.1]
        
设 $S_n = X_1 + \cdots + X_n$.容易验证:
$\{ \lim_{n \to \infty} S_n~\text{exists} \} \in \mathcal{T}$,
$\{ \sup_{n} S_n > 0 \} 
otin \mathcal{T}$,
$\{ \limsup_{n \to \infty} S_n / c_n > x \} \in \mathcal{T}$,其中 $c_n \to \infty$.

    \end{xmp}
    
    
    
    \begin{lma}
        [Generalization-of-Limit-Exchange-Lemma]
        {极限交换引理的推广}
        [Generalization of Limit Exchange Lemma]
        [gpt-4.1]
        如果 $h_{n}(\epsilon) \to g(\epsilon)$ 对每个 $\epsilon > 0$ 成立,且 $g(\epsilon) \to g(0)$ 当 $\epsilon \to 0$ 时成立,则存在 $\epsilon_{n} \to 0$ 使得 $h_{n}(\epsilon_{n}) \to g(0)$.
    \end{lma}
    
    
    
    \begin{thm}
        [Strong-Markov-Property]
        {强马尔可夫性质}
        [Strong Markov Property]
        [gpt-4.1]
        
定理 5.2.5 (强马尔可夫性质) 设对每个 $n$,$Y_n : \Omega_{0} \to \mathbb{R}$ 是可测的,且对所有 $n$ 皆有 $| Y_n | \leq M$.

    \end{thm}
    
    
    
    \begin{lma}
        [Limit-Distribution-by-Combining-with-Lemma-3.8.1]
        {用引理 3.8.1 合并公式得到极限分布}
        [Limit Distribution by Combining with Lemma 3.8.1]
        [gpt-4.1]
        结合 (3.8.6) 和 (3.8.9) 并利用引理 3.8.1,可得 $(S_n - b_n) / a_n \Rightarrow Y$,其中 $E e^{i t Y}$ 由 (3.8.10) 给出.
    \end{lma}
    
    
    
    \begin{thm}
        [0-1-Law]
        {0-1律}
        [0-1 Law]
        [gpt-4.1]
        
设 $A \in \mathcal{E}$,则 $A$ 与自身独立,因此 $P(A) = P(A \cap A) = P(A) P(A)$,即 $P(A) \in \{ 0, 1 \}$.

    \end{thm}
    
    
    
    \begin{thm}
        [Theorem-on-Independence-of-Conditional-Expectation]
        {条件期望独立性的定理}
        [Theorem on Independence of Conditional Expectation]
        [gpt-4.1]
        
若 $E X^2 < \infty$ 且 $E(X|\mathcal{G}) \in \mathcal{F}$,同时 $X$ 与 $\mathcal{F}$ 独立,则 $E(X|\mathcal{G}) = E X$.

    \end{thm}
    
    
    
    \begin{lma}
        [Proof-of-Lemma-on-Probability-Ratio]
        {关于概率比值的引理证明}
        [Proof of Lemma on Probability Ratio]
        [gpt-4.1]
        
Applying the last result to the first $n$ with $1 / 2 ^ { n } < y$ and noticing $y \leq 1 / 2 ^ { n - 1 }$ , we have

\[
P ( | X _ { 1 } | > y t ) / P ( | X _ { 1 } | > t ) \leq C 2 ^ { \alpha + \delta } y ^ { - \alpha - \delta }
\]

which proves the lemma.

    \end{lma}
    
    
    
    \begin{thm}
        [Finite-Additivity-and-Inclusion-Exclusion-Principle-Formula]
        {有限可加性与容斥原理公式}
        [Finite Additivity and Inclusion-Exclusion Principle Formula]
        [gpt-4.1]
        
\[
\begin{array}{rl}
& P \left( \bigcup_{m=1}^{n} A_{m} \right) = \displaystyle\sum_{m} P(A_{m}) - \sum_{\ell < m} P(A_{\ell} \cap A_{m}) \\
& \qquad + \displaystyle\sum_{k<\ell<m} P(A_{k} \cap A_{\ell} \cap A_{m}) - \ldots \\
& \qquad = n \cdot \frac{1}{n} - {\binom{n}{2}} \frac{(n-2)!}{n!} + {\binom{n}{3}} \frac{(n-3)!}{n!} - \ldots
\end{array}
\]

其中,当指定有 $k$ 个固定点时的排列数为 $(n-k)!$.

    \end{thm}
    
    
    
    \begin{dfn}
        [Definition-of-Stopping-Time-and-Truncated-Process]
        {停时与截断过程的定义}
        [Definition of Stopping Time and Truncated Process]
        [gpt-4.1]
        令 $\tau = \operatorname*{inf}\{n : S_{n} 
otin (a, b)\}$,$Y_{n} = X_{n \wedge \tau}$.
    \end{dfn}
    
    
    
    \begin{dfn}
        [Definition-of-the-Markov-Property]
        {马尔可夫性质的定义}
        [Definition of the Markov Property]
        [gpt-4.1]
        
\[
P ( X _ { n + 1 } = j | X _ { n } = i , X _ { n - 1 } = i _ { n - 1 } , \dots , X _ { 0 } = i _ { 0 } ) = P ( X _ { n + 1 } = j | X _ { n } = i )
\]

即:已知当前状态后,过去的其他信息对于预测未来状态都是无关的.

    \end{dfn}
    
    
    
    \begin{dfn}
        [Definition-of-Absorbing-State]
        {吸收态的定义}
        [Definition of Absorbing State]
        [gpt-4.1]
        一个状态 $a$ 被称为吸收态,如果 $P _ { a } ( X _ { 1 } = a ) = 1$.
    \end{dfn}
    
    
    
    \begin{thm}
        [Inequality-for-the-Expected-Supremum-of-Squared-Stochastic-Process]
        {关于随机变量序列上界二次期望的不等式}
        [Inequality for the Expected Supremum of Squared Stochastic Process]
        [gpt-4.1]
        
$E \left( \operatorname* { sup } _ { m } | X _ { m } | ^ { 2 } \right) \leq 4 E A _ { \infty }.$

    \end{thm}
    
    
    
    \begin{prf}
        [Proof-of-Inequality-for-the-Expected-Supremum-of-Squared-Stochastic-Process]
        {关于随机变量序列上界二次期望不等式的证明}
        [Proof of Inequality for the Expected Supremum of Squared Stochastic Process]
        [gpt-4.1]
        
应用$L^2$最大值不等式(定理4.4.4)于$X_n$,有
\[
E \left( \operatorname* { sup } _ { 0 \leq m \leq n } | X _ { m } | ^ { 2 } \right) \leq 4 E X _ { n } ^ { 2 } = 4 E A _ { n }
\]
由于 $E X _ { n } ^ { 2 } = E M _ { n } + E A _ { n }$ 且 $E M _ { n } = E M _ { 0 } = E X _ { 0 } ^ { 2 } = 0$,利用单调收敛定理即可得所需结果.

    \end{prf}
    
    
    
    \begin{thm}
        [Exponential-Martingale-Theorem-for-Random-Walk]
        {随机游走的指数鞅定理}
        [Exponential Martingale Theorem for Random Walk]
        [gpt-4.1]
        设 $S_{n} = \xi_{1} + \cdots + \xi_{n}$ 为随机游走,若存在 $\theta_{o} < 0$ 使得 $\varphi(\theta_{o}) = E \exp(\theta_{o} \xi_{1}) = 1$ 且 $\xi_{i}$ 非常数,则在该特殊情形下,指数鞅 $X_{n} = \exp(\theta_{o} S_{n})$ 是一个鞅.
    \end{thm}
    
    
    
    \begin{thm}
        [Asymptotic-Behavior-of-Standard-Normal-Tail-Probability]
        {标准正态分布的尾概率渐近行为}
        [Asymptotic Behavior of Standard Normal Tail Probability]
        [gpt-4.1]
        
对于标准正态分布的随机变量 $X_i$,当 $x \to \infty$ 时,有
\[
P(X_i > x) \sim \frac{1}{\sqrt{2\pi} x} e^{-x^2/2}
\]

    \end{thm}
    
    
    
    \begin{crl}
        [Limit-of-Tail-Ratio-for-Standard-Normal-Distribution]
        {标准正态分布尾比值极限}
        [Limit of Tail Ratio for Standard Normal Distribution]
        [gpt-4.1]
        
利用定理1.2.6的结论,对于任意实数 $\theta$,有
\[
\frac{P(X_i > x + (\theta / x))}{P(X_i > x)} \to e^{-\theta}
\]
当 $x \to \infty$ 时成立.

    \end{crl}
    
    
    
    \begin{thm}
        [Limit-Distribution-of-Maximum]
        {极大值的极限分布}
        [Limit Distribution of Maximum]
        [gpt-4.1]
        
若定义 $b_n$ 满足 $P(X_i > b_n) = 1/n$,则有
\[
P(b_n(M_n - b_n) \leq x) \to \exp(-e^{-x})
\]
其中 $M_n = \max\{X_1, X_2, \dots, X_n\}$,极限为 $n \to \infty$.

    \end{thm}
    
    
    
    \begin{crl}
        [Normalization-of-Standard-Normal-Maximum-Converges]
        {标准正态极大值的归一化收敛}
        [Normalization of Standard Normal Maximum Converges]
        [gpt-4.1]
        
有 $b_n \sim (2 \log n)^{1/2}$,并可得
\[
\frac{M_n}{(2 \log n)^{1/2}} \to 1
\]
在概率收敛意义下,当 $n \to \infty$.

    \end{crl}
    
    
    
    \begin{xmp}
        [Example-of-a-Square-Integrable-Martingale]
        {平方可积鞅的例子}
        [Example of a Square Integrable Martingale]
        [gpt-4.1]
        
设 $\epsilon > 0$, $\xi_1, \xi_2, \ldots$ 相互独立,且 $E \xi_m = 0$,令 $f(t) = (t \log^{1+\epsilon} t)^{1/2} \vee 1$.则 $f$ 满足假设条件,且 $E \xi_m^2 = \sigma_m^2$.在此情况下,$X_n = \xi_1 + \cdots + \xi_n$ 是一个平方可积鞅.

    \end{xmp}
    
    
    
    \begin{thm}
        [Extension-of-Martingale-Convergence-Theorem]
        {鞅归纳收敛定理的推广}
        [Extension of Martingale Convergence Theorem]
        [gpt-4.1]
        
由定理 4.5.3 可知,对于一般的 $A_n = \sigma_1^2 + \cdots + \sigma_n^2$,若 $\sum_{i=1}^{\infty} \sigma_i^2 = \infty$,则 $X_n / f(A_n) \to 0$.

    \end{thm}
    
    
    
    \begin{dfn}
        [Definition-of-Random-Variables-and-Related-Sequences]
        {随机变量及相关序列的定义}
        [Definition of Random Variables and Related Sequences]
        [gpt-4.1]
        
设 $\xi_{1}, \xi_{2}, \ldots$ 是独立同分布的随机变量,$\xi_i \in \{-1, 1\}$,各取值的概率均为 $1/2$.令 $S_{0} = 0$, $S_{n} = \xi_{1} + \cdots + \xi_{n}$,$X_{n} = \max\{S_{m}: 0 \leq m \leq n\}$.

    \end{dfn}
    
    
    
    \begin{lma}
        [Lemma-on-the-Limit-of-Average-for-Infinitely-Many-i.i.d.-Random-Variables]
        {关于无穷多个独立同分布随机变量的平均极限引理}
        [Lemma on the Limit of Average for Infinitely Many i.i.d. Random Variables]
        [gpt-4.1]
        
设 $X_{1}, X_{2}, \dots$ 是独立同分布 (i.i.d.) 的随机变量,令
\[
A_{n}(\varphi) = \frac{1}{(n)_k} \sum_{i} \varphi(X_{i_{1}}, \ldots, X_{i_{k}})
\]
其中求和是对所有满足 $1 \leq i_{1}, \ldots, i_{k} \leq n$ 且互不相同的整数序列,且
\[
(n)_k = n(n-1)\cdots(n-k+1)
\]
为此类序列的个数.如果 $\varphi$ 有界,则有 $A_{n}(\varphi) \to E\varphi(X_{1}, \dots, X_{k})$ 几乎处处成立.

    \end{lma}
    
    
    
    \begin{dfn}
        [Definition-of-First-Hitting-Time]
        {首次到达时刻的定义}
        [Definition of First Hitting Time]
        [gpt-4.1]
        设 $V_{x} = \operatorname*{min} \{n \geq 0 : X_{n} = x\}$,其中 $X_n$ 为 Markov 链,$V_x$ 表示 Markov 链首次到达状态 $x$ 的时刻.
    \end{dfn}
    
    
    
    \begin{thm}
        [Equivalent-Conditions-for-Submartingale-Convergence]
        {子鞅的收敛性等价条件}
        [Equivalent Conditions for Submartingale Convergence]
        [gpt-4.1]
        
对于子鞅,下列条件等价:
(i) 它是一致可积的.
(ii) 它几乎处处收敛,并且在 $L^1$ 中收敛.
(iii) 它在 $L^1$ 中收敛.

    \end{thm}
    
    
    
    \begin{dfn}
        [Definition-of-Time-of-Gamblers-Ruin]
        {赌徒破产时间的定义}
        [Definition of Time of Gambler's Ruin]
        [gpt-4.1]
        设 $\xi_1, \xi_2, \ldots$ 是独立随机变量,满足 $P(\xi_i = 1) = p$,$P(\xi_i = -1) = q = 1 - p$,其中 $p < 1/2$.令 $S_n = S_0 + \xi_1 + \cdots + \xi_n$,并令 $V_0 = \min\{n \geq 0 : S_n = 0\}$.
    \end{dfn}
    
    
    
    \begin{thm}
        [Expectation-of-Time-of-Gamblers-Ruin]
        {赌徒破产时间的期望}
        [Expectation of Time of Gambler's Ruin]
        [gpt-4.1]
        $E_x V_0 = x / (1 - 2p)$.
    \end{thm}
    
    
    
    \begin{thm}
        [The-Markov-Property]
        {马尔科夫性质}
        [The Markov Property]
        [gpt-4.1]
        
设 $Y : \Omega_0 \to \mathbf{R}$ 是有界且可测的函数,则有
\[
E_{\mu}(Y \circ \theta_m \mid \mathcal{F}_m) = E_{X_m} Y
\]
其中左侧的下标 $\mu$ 表示条件期望是关于 $P_\mu$ 取的,右侧等价于函数 $\varphi(x) = E_x Y$ 在 $x = X_m$ 处的取值.

    \end{thm}
    
    
    
    \begin{thm}
        [Theorem-on-Lower-Bound-for-Expected-Stopping-Time-of-Martingale]
        {鞅相关的停时期期望下界定理}
        [Theorem on Lower Bound for Expected Stopping Time of Martingale]
        [gpt-4.1]
        设 $S_n = \xi_1 + \cdots + \xi_n$,其中 $\xi_i$ 相互独立,$E \xi_i = 0$ 且 $\operatorname{var}(\xi_i) = \sigma^2$.令 $T = \min\{n : |S_n| > a\}$,则有
\[
E T \ge \frac{a^2}{\sigma^2}.
\]

    \end{thm}
    
    
    
    \begin{dfn}
        [Definition-of-Martingale]
        {鞅的定义}
        [Definition of Martingale]
        [gpt-4.1]
        $S_n^2 - n \sigma^2$ 是一个鞅(martingale).
    \end{dfn}
    
    
    
    \begin{ppt}
        [Relationship-Between-Return-Probability-and-First-Return-Distribution]
        {回到状态 $a$ 的概率与首次回归分布的关系}
        [Relationship Between Return Probability and First Return Distribution]
        [gpt-4.1]
        设 $a \in S$,$f_n = P_a(T_a = n)$,$u_n = P_a(X_n = a)$.则:
(i) 有 $u_n = \sum_{1 \leq m \leq n} f_m u_{n-m}$;
(ii) 令 $u(s) = \sum_{n \geq 0} u_n s^n$,$f(s) = \sum_{n \geq 1} f_n s^n$,则 $u(s) = 1 / (1 - f(s))$.

    \end{ppt}
    
    
    
    \begin{xmp}
        [Example-of-Markov-Chain-Random-Walk]
        {随机游走的马尔可夫链例子}
        [Example of Markov Chain: Random Walk]
        [gpt-4.1]
        
设 $\xi_1, \xi_2, \ldots \in \mathbb{Z}^d$ 是独立的且服从分布 $\mu$,令 $X_n = X_0 + \xi_1 + \cdots + \xi_n$,其中 $X_0$ 是常数.则 $X_n$ 是一个马尔可夫链,其转移概率为
\[
p(i, j) = \mu(\{j-i\})
\]

    \end{xmp}
    
    
    
    \begin{dfn}
        [Definition-of-Conditional-Expectation-with-respect-to-a-random-variable]
        {条件期望的定义(关于随机变量)}
        [Definition of Conditional Expectation (with respect to a random variable)]
        [gpt-4.1]
        
设 $X$ 和 $Y$ 是随机变量,$\sigma(Y)$ 表示由 $Y$ 生成的 $\sigma$-域.则条件期望 $E(X|Y)$ 定义为
\[
E(X|Y) = E(X|\sigma(Y))
\]
其中 $E(X|\sigma(Y))$ 是关于 $\sigma(Y)$ 的条件期望.

    \end{dfn}
    
    
    
    \begin{thm}
        [Dominated-Convergence-Theorem-for-Conditional-Expectations]
        {条件期望的控域收敛定理}
        [Dominated Convergence Theorem for Conditional Expectations]
        [gpt-4.1]
        
假设 $Y_{n} \to Y$ 几乎处处收敛,且对所有 $n$ 有 $|Y_{n}| \leq Z$,其中 $E Z < \infty$.如果 ${\mathcal{F}}_{n} \uparrow {\mathcal{F}}_{\infty}$,则
\[
E(Y_{n}|{\mathcal{F}}_{n}) \to E(Y|{\mathcal{F}}_{\infty}) \quad \text{几乎处处收敛}.
\]

    \end{thm}
    
    
    
    \begin{prf}
        [Proof-of-Dominated-Convergence-Theorem-for-Conditional-Expectations]
        {条件期望控域收敛定理的证明}
        [Proof of Dominated Convergence Theorem for Conditional Expectations]
        [gpt-4.1]
        
令 $W_{N} = \operatorname{sup}\{ |Y_{n} - Y_{m}| : n, m \geq N \}$.有 $W_{N} \le 2Z$,因此 $E W_{N} < \infty$.利用单调性(4.1.2)并对 $W_{N}$ 应用定理 4.6.8 得证.

    \end{prf}
    
    
    
    \begin{prf}
        [Proof-of-Doobs-Decomposition-for-Submartingales]
        {次马氏分解的证明}
        [Proof of Doob's Decomposition for Submartingales]
        [gpt-4.1]
        我们希望 $X_n = M_n + A_n$, 其中 $E(M_n | \mathcal{F}_{n-1}) = M_{n-1}$, 且 $A_n \in \mathcal{F}_{n-1}$.

所以我们必须有
\[
\begin{array}{r}
E(X_n | \mathcal{F}_{n-1}) = E(M_n | \mathcal{F}_{n-1}) + E(A_n | \mathcal{F}_{n-1}) \\
= M_{n-1} + A_n = X_{n-1} - A_{n-1} + A_n
\end{array}
\]
由此可得
\[
A_n - A_{n-1} = E(X_n | \mathcal{F}_{n-1}) - X_{n-1}
\]

由于 $A_0 = 0$, 有
\[
A_n = \sum_{m=1}^n E(X_m - X_{m-1} | \mathcal{F}_{m-1})
\]

为了验证这个构造的有效性, 注意到 $A_n - A_{n-1} \geq 0$,因为 $X_n$ 是次马氏过程且 $A_n \in \mathcal{F}_{n-1}$.

    \end{prf}
    
    
    
    \begin{prf}
        [Proof-that-the-Martingale-Component-is-a-Martingale]
        {马氏部分的证明}
        [Proof that the Martingale Component is a Martingale]
        [gpt-4.1]
        为证明 $M_n = X_n - A_n$ 是马氏过程, 注意到 $A_n \in \mathcal{F}_{n-1}$ 且有
\[
\begin{array}{rl}
E(M_n | \mathcal{F}_{n-1}) &= E(X_n - A_n | \mathcal{F}_{n-1}) \\
&= E(X_n | \mathcal{F}_{n-1}) - A_n = X_{n-1} - A_{n-1} = M_{n-1}
\end{array}
\]
这就完成了证明.

    \end{prf}
    
    
    
    \begin{thm}
        [Arcsine-Law-for-Time-Above-Zero]
        {关于时间在零之上的弧正弦定律}
        [Arcsine Law for Time Above Zero]
        [gpt-4.1]
        设 $\pi_{2n}$ 表示在 $2n$ 步随机游走中,路径位于 $y \geq 0$ 之上的线段数,$u_m = P(S_m = 0)$,则

\[
P(\pi_{2n} = 2k) = u_{2k} u_{2n-2k}
\]

并且,如果 $0 < a < b < 1$,则有
\[
P\left(a \leq \frac{\pi_{2n}}{2n} \leq b\right) \to \int_{a}^{b} \pi^{-1} (x(1-x))^{-1/2} dx
\]

    \end{thm}
    
    
    
    \begin{xmp}
        [Example-of-$\exp----|-t-|-^-{-\alpha-}-$-as-a-Characteristic-Function]
        {关于函数 $\exp ( - | t | ^ { \alpha } )$ 是特征函数的例子}
        [Example of $\exp ( - | t | ^ { \alpha } )$ as a Characteristic Function]
        [gpt-4.1]
        $\exp ( - | t | ^ { \alpha } )$ 是一个特征函数,其中 $0 < \alpha < 2$.
    \end{xmp}
    
    
    
    \begin{dfn}
        [Definition-of-Stopping-Time]
        {停止时间的定义}
        [Definition of Stopping Time]
        [gpt-4.1]
        令 $T = \operatorname*{inf} \{ n : X _ { n } \geq M \}$.
    \end{dfn}
    
    
    
    \begin{lma}
        [Lemma-on-the-Inequality-Relating-Distribution-Difference-and-Characteristic-Function-Distance]
        {分布差与特征函数距离的不等式引理}
        [Lemma on the Inequality Relating Distribution Difference and Characteristic Function Distance]
        [gpt-4.1]
        引理 3.4.19:该引理给出了两个分布之间的差异与它们的特征函数之间的距离之间的一个不等式关系.
    \end{lma}
    
    
    
    \begin{xmp}
        [Example-of-Polyas-Density]
        {Polya密度的例子}
        [Example of Polya's Density]
        [gpt-4.1]
        Polya 密度
\[
h_{L}(x) = \frac{1 - \cos{Lx}}{\pi L x^{2}}
\]
具有特征函数 $\omega_{L}(\theta) = (1 - |\theta / L|)^{+}$,其中 $|\theta| \le L$.其分布函数记为 $H_{L}$.
    \end{xmp}
    
    
    
    \begin{thm}
        [Theorem-on-Characteristic-Function-Property-e]
        {特征函数的性质定理(e)}
        [Theorem on Characteristic Function Property (e)]
        [gpt-4.1]
        定理 3.3.1 的(e)部分:涉及特征函数的某些性质,对例子3.3.15中的Polya密度的特征函数进行了说明.
    \end{thm}
    
    
    
    \begin{prf}
        [Proof-of-Convergence-Probability-Estimate]
        {收敛性概率估计的证明}
        [Proof of Convergence Probability Estimate]
        [gpt-4.1]
        设 $S_{N} = \sum_{n=1}^{N} X_{n}$.由定理 2.5.5 得

\[
P\left(\max_{M \leq m \leq N} |S_{m} - S_{M}| > \epsilon\right) \leq \epsilon^{-2} \operatorname{var}(S_{N} - S_{M}) = \epsilon^{-2} \sum_{n=M+1}^{N} \operatorname{var}(X_{n})
\]

令 $N \to \infty$,得

\[
P\left(\sup_{m \ge M} |S_{m} - S_{M}| > \epsilon\right) \le \epsilon^{-2} \sum_{n=M+1}^{\infty} \operatorname{var}\left(X_{n}\right) \to 0 \quad \text{当 } M \to \infty
\]

若令 $w_{M} = \sup_{m, n \geq M} |S_{m} - S_{n}|$,则 $w_{M} \downarrow$ 随 $M \uparrow$ 并且

\[
P(w_{M} > 2\epsilon) \leq P(\sup_{m \geq M} |S_{m} - S_{M}| > \epsilon) \to 0
\]

当 $M \to \infty$ 时,故 $w_{M} \downarrow 0$ 几乎处处成立.但 $w_{M}(\omega) \downarrow 0$ 蕴含 $S_{n}(\omega)$ 是 Cauchy 列,因此 $\lim_{n \to \infty} S_{n}(\omega)$ 存在,证明完毕.
    \end{prf}
    
    
    
    \begin{xmp}
        [Example-of-Pedestrian-Delay-in-a-Poisson-Process]
        {泊松过程下行人延迟问题的例子}
        [Example of Pedestrian Delay in a Poisson Process]
        [gpt-4.1]
        
一只鸡想要穿过一条马路,马路上的交通流是速率为 $\lambda$ 的泊松过程.它需要连续一个单位时间内没有车辆到达,才能安全过马路.设 $M = \inf \{ t \geq 0 : \text{在} ( t , t + 1 ] \text{内无车辆到达} \}$,表示它开始过马路前的等待时间.通过考虑第一次到达的时间,可知 $H ( t ) = P ( M \leq t )$ 满足

\[
H ( t ) = e ^ { - \lambda } + \int _ { 0 } ^ { 1 } H ( t - y ) \lambda e ^ { - \lambda y } d y
\]

并结合相关例子和定理,得到

\[
H ( t ) = e ^ { - \lambda } \sum _ { n = 0 } ^ { \infty } F ^ { n * } ( t )
\]

此外,也可以通过下式计算:

\[
P ( M \leq t ) = \sum _ { n = 0 } ^ { \infty } P ( T _ { n } \leq t , T _ { n + 1 } = \infty )
\]

最后一种表达式可用于计算 $M$ 的均值.

    \end{xmp}
    
    
    
    \begin{thm}
        [Covariance-Increment-Formula-for-Two-Square-Integrable-Martingales]
        {两个二次可积鞅的协方差增量公式}
        [Covariance Increment Formula for Two Square-Integrable Martingales]
        [gpt-4.1]
        
设 $X_n$ 和 $Y_n$ 是鞅,满足 $E X_n^2 < \infty$ 且 $E Y_n^2 < \infty$,则有
\[
E X_n Y_n - E X_0 Y_0 = \sum_{m=1}^n E (X_m - X_{m-1})(Y_m - Y_{m-1})
\]

    \end{thm}
    
    
    
    \begin{dfn}
        [Definition-of-Martingale-Increment]
        {鞅增量的定义}
        [Definition of Martingale Increment]
        [gpt-4.1]
        
设 $X_n$, $n \geq 0$,为鞅,定义其增量为 $\xi_n = X_n - X_{n-1}$,其中 $n \geq 1$.

    \end{dfn}
    
    
    
    \begin{thm}
        [Limit-Theorem-for-Square-Integrable-Martingales]
        {平方可积鞅的极限存在定理}
        [Limit Theorem for Square-Integrable Martingales]
        [gpt-4.1]
        
若 $E X_0^2$, $\sum_{m=1}^\infty E \xi_m^2 < \infty$,则 $X_n \to X_\infty$ 几乎处处成立,并且在 $L^2$ 意义下收敛.

    \end{thm}
    
    
    
    \begin{xmp}
        [Example-of-Ehrenfest-Chain]
        {Ehrenfest链的例子}
        [Example of Ehrenfest Chain]
        [gpt-4.1]
        
$S = \{ 0 , 1 , \ldots , r \}$

\[
\begin{array} { l } 
{ p ( k , k + 1 ) = ( r - k ) / r } \\ 
{ p ( k , k - 1 ) = k / r } \\ 
{ p ( i , j ) = 0 \qquad \mathrm{otherwise} } 
\end{array}
\]

即有 $r$ 个球分布在两个盒子中;第一个盒子有 $k$ 个球,第二个盒子有 $r-k$ 个球.每次随机挑选一个球并放到另一个盒子中.Ehrenfest 用此模型来描述通过一个小孔连接的两腔室空气分子的分布情况.

    \end{xmp}
    
    
    
    \begin{lma}
        [Integrability-of-$Y$-Satisfying-Conditions-i-and-ii]
        {满足条件 (i) 和 (ii) 的 $Y$ 的可积性}
        [Integrability of $Y$ Satisfying Conditions (i) and (ii)]
        [gpt-4.1]
        
如果 $Y$ 满足 (i) 和 (ii),则 $Y$ 是可积的.

    \end{lma}
    
    
    
    \begin{prf}
        [Proof-of-Integrability-of-$Y$]
        {$Y$ 的可积性证明}
        [Proof of Integrability of $Y$]
        [gpt-4.1]
        
设 $A = \{ Y > 0 \} \in \mathcal{F}$,利用 (ii) 两次,然后相加得

\[
\begin{aligned}
\int_A Y dP &= \int_A X dP \le \int_A |X| dP \\
\int_{A^c} -Y dP &= \int_{A^c} -X dP \le \int_{A^c} |X| dP
\end{aligned}
\]

因此有 $E|Y| \leq E|X|$.

    \end{prf}
    
    
    
    \begin{thm}
        [Probability-Formula-for-Martingale-Exiting-an-Interval]
        {鞅出区间的概率公式}
        [Probability Formula for Martingale Exiting an Interval]
        [gpt-4.1]
        
设 $\{S_n\}$ 是一个对称随机游走(或鞅),初始位置为 $x$,在区间 $[a, b]$ 上.令 $N$ 为首次出区间 $[a, b]$ 的时间,则有
\[
P_x(S_N = a) = \frac{b - x}{b - a}, \quad P_x(S_N = b) = \frac{x - a}{b - a}.
\]

    \end{thm}
    
    
    
    \begin{thm}
        [Walds-Equation]
        {Wald 方程}
        [Wald's Equation]
        [gpt-4.1]
        如果 $\xi_{1}, \xi_{2}, \ldots$ 是独立同分布(i.i.d.)的随机变量,$E \xi_{i} = \mu$,$S_{n} = \xi_{1} + \cdots + \xi_{n}$,$N$ 是一个满足 $E N < \infty$ 的停时,则 $E S_{N} = \mu E N$.
    \end{thm}
    
    
    
    \begin{thm}
        [Conditional-Independence-under-the-Markov-Property]
        {马尔可夫性质下的条件独立性}
        [Conditional Independence under the Markov Property]
        [gpt-4.1]
        
设 $A \in \sigma ( X _ { 0 } , \ldots , X _ { n } )$ 且 $B \in \sigma ( X _ { n } , X _ { n + 1 } , \ldots )$,则对于任意初始分布 $\mu$,
\[
P _ { \mu } ( A \cap B | X _ { n } ) = P _ { \mu } ( A | X _ { n } ) P _ { \mu } ( B | X _ { n } )
\]
即,已知当前状态,过去与未来条件独立.

    \end{thm}
    
    
    
    \begin{thm}
        [Recurrence-of-Simple-Random-Walk-in-Different-Dimensions]
        {简单随机游走在不同维度的遍历性}
        [Recurrence of Simple Random Walk in Different Dimensions]
        [gpt-4.1]
        简单随机游走在 $d \leq 2$ 时是遍历的(recurrent),在 $d \geq 3$ 时是趋于游离的(transient).
    \end{thm}
    
    
    
    \begin{thm}
        [Stopping-Time-Probability-via-Exponential-Martingale]
        {指数鞅方法求停时概率}
        [Stopping Time Probability via Exponential Martingale]
        [gpt-4.1]
        
设$\xi_{i}$是整数值随机变量,满足$P(\xi_{i} < -1) = 0$,$P(\xi_{i} = -1) > 0$,且$E \xi_{i} > 0$.定义$S_n = \sum_{i=1}^n \xi_i$,令$T = \operatorname*{inf}\{n : S_{n} = a\}$,其中$a < 0$.则利用鞅$X_{n} = \exp(\theta_{o} S_{n})$,有
\[
P(T < \infty) = \exp(-\theta_{o} a).
\]

    \end{thm}
    
    
    
    \begin{dfn}
        [Definition-of-Transition-Probability-in-Branching-Process]
        {分支过程的转移概率定义}
        [Definition of Transition Probability in Branching Process]
        [gpt-4.1]
        
设 $S = \{ 0 , 1 , 2 , \ldots \}$,分支过程的转移概率定义如下:

\[
p ( i , j ) = P \left( \sum _ { m = 1 } ^ { i } \xi _ { m } = j \right)
\]

其中 $\xi _ { 1 } , \xi _ { 2 } , \ldots$ 是独立同分布的非负整数值随机变量.即,在第 $n$ 代(或时刻 $n$),每个个体 $i$ 会独立地产生若干后代,其后代数量服从同一分布.

    \end{dfn}
    
    
    
    \begin{dfn}
        [Definition-of-Closed-Set]
        {闭集的定义}
        [Definition of Closed Set]
        [gpt-4.1]
        $C$ 称为闭集,如果 $x \in C$ 且 $\rho_{xy} > 0$ 蕴含 $y \in C$.
    \end{dfn}
    
    
    
    \begin{dfn}
        [Definition-of-Irreducible-Set]
        {不可约集的定义}
        [Definition of Irreducible Set]
        [gpt-4.1]
        $D$ 称为不可约集,如果 $x, y \in D$ 蕴含 $\rho_{xy} > 0$.
    \end{dfn}
    
    
    
    \begin{dfn}
        [Definition-of-$G-n$-and-$R-n$]
        {定义 $G\_n$ 和 $R\_n$}
        [Definition of $G_n$ and $R_n$]
        [gpt-4.1]
        令 $G_n$ 和 $R_n$ 分别表示第 $n$ 次抽取完成后绿色和红色球的数量.
    \end{dfn}
    
    
    
    \begin{dfn}
        [Definition-of-$B-n$-and-$D-n$]
        {定义 $B\_n$ 和 $D\_n$}
        [Definition of $B_n$ and $D_n$]
        [gpt-4.1]
        令 $B_n$ 表示第 $n$ 个被抽取的球是绿色的事件,$D_n$ 表示前 $n$ 次抽取中被抽取出的绿色球的数量.
    \end{dfn}
    
    
    
    \begin{thm}
        [Limit-Theorem-for-Proportion-of-Green-Balls-Drawn]
        {关于绿色球抽取比例的极限定理}
        [Limit Theorem for Proportion of Green Balls Drawn]
        [gpt-4.1]
        有
\[
\frac{D_n}{\sum_{m=1}^n g_{m-1}} \to 1 \quad \mathrm{a.s.\ on} \quad \sum_{m=1}^{\infty} g_{m-1} = \infty
\]
且当 $g_m \geq \frac{g}{g + r + (a+b)m}$ 时,$\sum_{m=1}^{\infty} g_{m-1} = \infty$ 总成立.
    \end{thm}
    
    
    
    \begin{ppt}
        [Upper-Bound-Estimate-for-Power-Differences]
        {幂差的上界估计}
        [Upper Bound Estimate for Power Differences]
        [gpt-4.1]
        
若 $|\alpha|, |\beta| \leq \gamma$,则
\[
|\alpha^n - \beta^n| \leq \sum_{m=0}^{n-1} |\alpha^{n-m} \beta^m - \alpha^{n-m-1} \beta^{m+1}| \leq n|\alpha - \beta|\gamma^{n-1}
\]

    \end{ppt}
    
    
    
    \begin{ppt}
        [Error-Bound-for-Function-Approximation]
        {函数近似的误差界}
        [Error Bound for Function Approximation]
        [gpt-4.1]
        
若 $\sigma^2 = 1$,则有
\[
|\varphi(t) - 1 + t^2 / 2| \leq \rho |t|^3 / 6
\]

    \end{ppt}
    
    
    
    \begin{ppt}
        [Upper-Bound-for-Modulus-of-Function]
        {函数模的上界}
        [Upper Bound for Modulus of Function]
        [gpt-4.1]
        
当 $t^2 \leq 2$ 时,
\[
|\varphi(t)| \leq 1 - t^2 / 2 + \rho |t|^3 / 6
\]

    \end{ppt}
    
    
    
    \begin{ppt}
        [Exponential-Bound-for-Modulus-of-Function-in-Specific-Interval]
        {特定区间内函数模的指数界}
        [Exponential Bound for Modulus of Function in Specific Interval]
        [gpt-4.1]
        
令 $L = 4\sqrt{n} / 3\rho$.若 $|\theta| \leq L$,则
\[
|\varphi(\theta/\sqrt{n})| \leq 1 - \theta^2 / 2n + \rho |\theta|^3 / 6n^{3/2} \leq 1 - 5\theta^2 / 18n \leq \exp(-5\theta^2 / 18n)
\]

    \end{ppt}
    
    
    
    \begin{ppt}
        [Exponential-Bound-for-Powers]
        {幂的指数界}
        [Exponential Bound for Powers]
        [gpt-4.1]
        
若 $n \geq 10$,则
\[
\gamma^{n-1} \leq \exp(-\theta^2 / 4)
\]
其中 $\gamma = \exp(-5\theta^2 / 18n)$.

    \end{ppt}
    
    
    
    \begin{ppt}
        [Error-Bound-between-Exponential-Function-and-Linear-Approximation]
        {指数函数与线性近似的误差界}
        [Error Bound between Exponential Function and Linear Approximation]
        [gpt-4.1]
        
若 $0 < x < 1$,则
\[
|e^{-x} - (1 - x)| = \left| -\frac{x^2}{2!} + \frac{x^3}{3!} - \ldots \right| \leq \frac{x^2}{2}
\]

    \end{ppt}
    
    
    
    \begin{ppt}
        [Upper-Bound-for-Difference-of-Functions]
        {函数差值的上界}
        [Upper Bound for Difference of Functions]
        [gpt-4.1]
        
对于 $|\theta| \leq L \leq \sqrt{2n}$,
\[
n|1 - \theta^2 / 2n - \exp(-\theta^2 / 2n)| \leq \theta^4 / 8n
\]

    \end{ppt}
    
    
    
    \begin{ppt}
        [Combined-Upper-Bound-for-Function-Difference]
        {函数差值的综合上界}
        [Combined Upper Bound for Function Difference]
        [gpt-4.1]
        
\[
n|\alpha - \beta| \leq \rho |\theta|^3 / 6 n^{1/2} + \theta^4 / 8n
\]
\[
\alpha = \varphi(\theta/{\sqrt{n}}), \quad \beta = \exp(-\theta^2 / 2n)
\]

    \end{ppt}
    
    
    
    \begin{thm}
        [Upper-Bound-for-Difference-between-Power-of-Function-and-Exponential-Function]
        {函数幂与指数函数差值的上界}
        [Upper Bound for Difference between Power of Function and Exponential Function]
        [gpt-4.1]
        
利用上述估计,得
\[
\frac{1}{|\theta|} |\varphi^n(\theta/\sqrt{n}) - \exp(-\theta^2 / 2)| \leq \exp(-\theta^2 / 4) \left\{ \frac{\rho \theta^2}{6 n^{1/2}} + \frac{|\theta|^3}{8 n} \right\}
\]
且
\[
\leq \frac{1}{L} \exp(-\theta^2 / 4) \left\{ \frac{2\theta^2}{9} + \frac{|\theta|^3}{18} \right\}
\]
其中 $\rho / \sqrt{n} = 4 / 3L$, $1/n = 1/\sqrt{n} \cdot 1/\sqrt{n} \leq 4 / 3L \cdot 1 / 3$, 且 $\rho \geq 1$, $n \geq 10$.

    \end{thm}
    
    
    
    \begin{thm}
        [Theorem-on-the-Structure-of-the-Set-of-Recurrent-Values]
        {关于重现值集合的结构定理}
        [Theorem on the Structure of the Set of Recurrent Values]
        [gpt-4.1]
        设 $
u$ 为重现值的集合,则 $
u$ 不是空集就是 $\mathbf{R}^d$ 的一个闭子群.在后一种情况下,$
u = \mathcal{U}$,即为所有可能取值的集合.
    \end{thm}
    
    
    
    \begin{thm}
        [Decomposition-Theorem-for-Markov-Chains]
        {马尔可夫链的分解定理}
        [Decomposition Theorem for Markov Chains]
        [gpt-4.1]
        设 $R = \{ x : \rho_{xx} = 1 \}$ 为马尔可夫链的所有常返状态,则 $R$ 可以写为 $\cup_{i} R_{i}$,其中每个 $R_{i}$ 都是封闭且不可约的集合.
    \end{thm}
    
    
    
    \begin{thm}
        [Reflection-Principle]
        {反射原理}
        [Reflection Principle]
        [gpt-4.1]
        设 $\xi_{ 1 }, \xi_{ 2 }, \ldots$ 是独立同分布的随机变量,其分布关于 $0$ 对称.令 $S_{ n } = \xi_{ 1 } + \cdots + \xi_{ n }$.如果 $a > 0$,则

\[
P \left( \sup_{ m \leq n } S_{ m } \geq a \right) \leq 2 P ( S_{ n } \geq a )
\]

    \end{thm}
    
    
    
    \begin{thm}
        [Monotone-Class-Theorem]
        {单调类定理}
        [Monotone Class Theorem]
        [gpt-4.1]
        
设 $\mathcal{A}$ 是一个包含 $\Omega$ 的 $\pi$-系统,$\mathcal{H}$ 是实值函数的集合,满足:

$(i)$ 若 $A \in \mathcal{A}$,则 $1_A \in \mathcal{H}$.

$(ii)$ 若 $f, g \in \mathcal{H}$,则 $f + g$ 和 $cf \in \mathcal{H}$,其中 $c$ 为任意实数.

$(iii)$ 若 $\{f_n\}$ 是 $\mathcal{H}$ 中的非负函数,且递增收敛到有界函数 $f$,则 $f \in \mathcal{H}$.

则 $\mathcal{H}$ 包含所有关于 $\sigma(\mathcal{A})$ 可测的有界函数.

    \end{thm}
    
    
    
    \begin{thm}
        [Asymptotic-Probability-of-Random-Walk-Returning-to-Zero-and-Domain-of-Attraction-of-Stable-Law]
        {随机游走回到零点概率的渐近性质与稳定律吸引域}
        [Asymptotic Probability of Random Walk Returning to Zero and Domain of Attraction of Stable Law]
        [gpt-4.1]
        
设 $R$ 表示一随机游走首次回到零点的时刻,则有
\[
P(R > 2n) = P(S_{2n} = 0) \sim \pi^{-1/2} n^{-1/2}
\]
并且由于 $P(R > x) / P(|R| > x) = 1$,由定理 3.8.8 可得 $R$ 属于参数 $\alpha = 1/2$ 和 $\kappa = 1$ 的稳定律的吸引域.这意味着,如果 $R_n$ 表示第 $n$ 次回到零点的时刻,则有 $R_n / n^2 \Rightarrow Y$,其中 $Y$ 服从上述稳定律.

    \end{thm}
    
    
    
    \begin{thm}
        [Recurrence-of-Simple-Random-Walk-in-Two-Dimensions]
        {二维简单随机游走的常返性}
        [Recurrence of Simple Random Walk in Two Dimensions]
        [gpt-4.1]
        二维简单随机游走是常返的.
    \end{thm}
    
    
    
    \begin{thm}
        [Nonnegativity-of-Expectation-of-Product-of-Exchangeable-Real-Random-Variables]
        {可交换实随机变量乘积期望非负性}
        [Nonnegativity of Expectation of Product of Exchangeable Real Random Variables]
        [gpt-4.1]
        如果 $X_{1}, X_{2}, \dotsc \in \mathbf{R}$ 是可交换的,且 $E X_{i}^{2} < \infty$,则 $E(X_{1} X_{2}) \geq 0$.
    \end{thm}
    
    
    
    \begin{xmp}
        [Birthday-Problem-Probability-Calculation]
        {生日悖论概率计算}
        [Birthday Problem Probability Calculation]
        [gpt-4.1]
        
设 $X_{1}, X_{2}, \dots$ 是在 $\{1, \ldots, N\}$ 上相互独立且均匀分布的随机变量,令 $T_{N} = \operatorname*{min}\{n : X_{n} = X_{m} \text{ for some } m < n\}$.则有
\[
P(T_{N} > n) = \prod_{m=2}^{n} \left(1 - \frac{m-1}{N}\right)
\]
当 $N = 365$ 时,这表示在一个 $n$ 人的群体中两个人生日不同的概率(假设所有生日等概率).

    \end{xmp}
    
    
    
    \begin{crl}
        [Limiting-Distribution-in-Birthday-Problem]
        {生日悖论极限分布}
        [Limiting Distribution in Birthday Problem]
        [gpt-4.1]
        
利用练习3.1.1,可得
\[
P\left(\frac{T_{N}}{N^{1/2}} > x\right) \to \exp(-x^{2}/2) \quad \mathrm{for\ all\ } x \geq 0
\]

    \end{crl}
    
    
    
    \begin{xmp}
        [Approximate-Probability-Calculation-in-Birthday-Problem]
        {生日悖论概率近似计算}
        [Approximate Probability Calculation in Birthday Problem]
        [gpt-4.1]
        
取 $N = 365$,注意到 $22 / \sqrt{365} = 1.1515$,$(1.1515)^{2}/2 = 0.6630$,则
\[
P(T_{365} > 22) \approx e^{-0.6630} \approx 0.515
\]
该答案比真实概率 $0.524$ 小 $2\%$.

    \end{xmp}
    
    
    
    \begin{thm}
        [Relation-between-Conditional-Expectation-and-Transition-Probability-for-Markov-Chains]
        {关于马尔可夫链条件期望与转移概率的关系}
        [Relation between Conditional Expectation and Transition Probability for Markov Chains]
        [gpt-4.1]
        
若 $X_n$ 的转移概率为 $p$,则对任意有界可测函数 $f$,有
\[
E(f(X_{n+1}) \mid \mathcal{F}_n) = \int p(X_n, dy) f(y)
\]

    \end{thm}
    
    
    
    \begin{dfn}
        [Definition-of-the-Set-$\mathcal{H}$]
        {集合 $\mathcal{H}$ 的定义}
        [Definition of the Set $\mathcal{H}$]
        [gpt-4.1]
        
令 $\mathcal{H}$ 为使上述恒等式成立的所有有界函数的集合.

    \end{dfn}
    
    
    
    \begin{thm}
        [Integral-Expansion-Formula-for-Finite-Expectations-of-Markov-Chains]
        {马尔可夫链有限期望的积分展开公式}
        [Integral Expansion Formula for Finite Expectations of Markov Chains]
        [gpt-4.1]
        
设 $\{X_n\}$ 是一个马尔可夫链,$\mu$ 是 $X_0$ 的分布,$p_n(x, dy)$ 是转移概率核,$f_0, f_1, \ldots, f_n$ 是有界可测函数,则有
\[
E\left( \prod_{m=0}^{n} f_m(X_m) \right) = \int \mu(dx_0) f_0(x_0) \int p_0(x_0, dx_1) f_1(x_1) \cdots \int p_{n-1}(x_{n-1}, dx_n) f_n(x_n)
\]

    \end{thm}
    
    
    
    \begin{thm}
        [Transience-of-Simple-Random-Walk-in-Three-Dimensions]
        {三维简单随机游走的遍历性}
        [Transience of Simple Random Walk in Three Dimensions]
        [gpt-4.1]
        三维空间中的简单随机游走是遍历的(即非重返性).直观地说,这成立,因为经过 $2n$ 步后回到原点的概率为 $\sim c n^{-3/2}$,该概率关于 $n$ 可求和.
    \end{thm}
    
    
    
    \begin{thm}
        [Equivalent-Conditions-for-Convergence-in-Distribution]
        {概率分布收敛等价条件}
        [Equivalent Conditions for Convergence in Distribution]
        [gpt-4.1]
        下述命题等价:

(i) 对所有有界连续函数 $f$,有 $E f (X_{n}) \to E f (X_{\infty})$;
(ii) 对所有有界利普希茨连续函数 $f$,有 $E f (X_{n}) \to E f (X_{\infty})$;
(iii) 对所有闭集 $K$,有 $\limsup_{n \to \infty} P(X_{n} \in K) \leq P(X_{\infty} \in K)$;
(iv) 对所有开集 $G$,有 $\liminf_{n \to \infty} P(X_{n} \in G) \geq P(X_{\infty} \in G)$;
(v) 对所有满足 $P(X_{\infty} \in \partial A) = 0$ 的集合 $A$,有 $\lim_{n \to \infty} P(X_{n} \in A) = P(X_{\infty} \in A)$;
(vi) 记 $D_{f}$ 为函数 $f$ 的不连续点集.对所有有界函数 $f$ 且 $P(X_{\infty} \in D_{f}) = 0$,有 $E f (X_{n}) \to E f (X_{\infty})$.

    \end{thm}
    
    
    
    \begin{xmp}
        [Example-of-Poisson-Distribution-and-Characteristic-Function]
        {关于泊松分布和特征函数的例子}
        [Example of Poisson Distribution and Characteristic Function]
        [gpt-4.1]
        
$E \exp ( i t Z ) = \sum_{n = 0}^{\infty} e^{-\lambda} \frac{\lambda^{n}}{n!} \varphi(t)^{n} = \exp(-\lambda (1 - \varphi(t)))$

    \end{xmp}
    
    
    
    \begin{lma}
        [Density-Function-of-Random-Variable-$V-{n+1}$]
        {随机变量 $V\_{n+1}$ 的概率密度函数}
        [Density Function of Random Variable $V_{n+1}$]
        [gpt-4.1]
        
$V_{n+1}$ 的概率密度函数为
\[
f_{V_{n+1}}(x) = (2n+1) \binom{2n}{n} x^n (1-x)^n
\]

    \end{lma}
    
    
    
    \begin{prf}
        [Proof-of-Density-Function-of-$V-{n+1}$]
        {$V\_{n+1}$ 的概率密度函数的证明}
        [Proof of Density Function of $V_{n+1}$]
        [gpt-4.1]
        
有 $2n+1$ 种方式选择落在 $x$ 的观测值,然后需要为落在 $< x$ 的观测值选择 $n$ 个索引,这可以通过 $\binom{2n}{n}$ 种方式完成.

一旦决定了哪些索引落在 $< x$ 和 $> x$,相应的随机变量满足条件的概率为 $x^n (1-x)^n$,而剩下的那个观测值落在 $x$ 的概率密度为 1.

如果对上述表述有疑问,可以计算概率 $X_1 < x-\epsilon, \ldots, X_n < x-\epsilon, x-\epsilon < X_{n+1} < x+\epsilon, X_{n+2} > x+\epsilon, \ldots, X_{2n+1} > x+\epsilon$,然后令 $\epsilon \to 0$.

    \end{prf}
    
    
    
    \begin{prf}
        [A-Proof-on-Expected-Number-of-Visits]
        {关于期望次数的一个证明}
        [A Proof on Expected Number of Visits]
        [gpt-4.1]
        Proof In view of Theorem 5.3.2, it suffices to prove the first claim. Suppose it is false. Then for all $y \in C$, $\rho_{yy} < 1$ and $E_x N(y) = \rho_{xy} / (1 - \rho_{yy})$, but this is ridiculous since it implies

\[
\infty > \sum_{y \in C} E_x N(y) = \sum_{y \in C} \sum_{n=1}^{\infty} p^{n}(x, y) = \sum_{n=1}^{\infty} \sum_{y \in C} p^{n}(x, y) = \sum_{n=1}^{\infty} 1
\]

The first inequality follows from the fact that $C$ is finite and the last equality from the fact that $C$ is closed.

    \end{prf}
    
    
    
    \begin{thm}
        [Probability-Formula-for-First-Hitting-Boundary]
        {关于首次到达界的概率公式}
        [Probability Formula for First Hitting Boundary]
        [gpt-4.1]
        令 $a = 0$ 且 $b = M$ 代入定理 5.3.10 得:
\[
P_x(T_0 > T_M) = \varphi(x)/\varphi(M)
\]
其中 $\varphi$ 为定理中涉及的相关函数.
    \end{thm}
    
    
    
    \begin{thm}
        [Inequality-for-Uniformly-Integrable-Submartingale-at-Two-Stopping-Times]
        {关于一致可积亚鞅在两个停时下的期望与条件期望不等式}
        [Inequality for Uniformly Integrable Submartingale at Two Stopping Times]
        [gpt-4.1]
        
设 $L \leq M$ 是停时,且 $Y_{M \wedge n}$ 是一致可积亚鞅,则有 $E Y_L \leq E Y_M$,并且
\[
Y_L \le E(Y_M | \mathcal{F}_L)
\]

    \end{thm}
    
    
    
    \begin{thm}
        [Relationship-Between-Recurrence-of-State-and-Expected-Number-of-Visits]
        {状态的遍历性与期望访问次数的关系}
        [Relationship Between Recurrence of State and Expected Number of Visits]
        [gpt-4.1]
        状态 $y$ 是遍历的,当且仅当 $E_{y} N(y) = \infty$.
    \end{thm}
    
    
    
    \begin{thm}
        [Strong-Law-of-Large-Numbers]
        {强大数定律}
        [Strong Law of Large Numbers]
        [gpt-4.1]
        
设 $\xi_1, \xi_2, \ldots$ 是独立同分布的随机变量,且 $E|\xi_i| < \infty$.记 $S_n = \xi_1 + \cdots + \xi_n$,则有
\[
\operatorname* { lim } _ { n \to \infty } \frac{S_n}{n} = E(\xi_1) \quad \mathrm{a.s.}
\]

    \end{thm}
    
    
    
    \begin{thm}
        [Irreducibility-of-$C-x$]
        {关于 $C\_x$ 的不可约性}
        [Irreducibility of $C_x$]
        [gpt-4.1]
        2 implies $C_x = \{ y : \rho_{xy} > 0 \}$ is an irreducible closed set.

(If $y, z \in C_x$, then $\rho_{yz} \geq \rho_{yx} \rho_{xz} > 0$.
If $\rho_{yw} > 0$, then $\rho_{xw} \geq \rho_{xy} \rho_{yw} > 0$, so $w \in C_x$.)
    \end{thm}
    
    
    
    \begin{xmp}
        [Example-of-Quadratic-Martingale]
        {二次鞅的例子}
        [Example of Quadratic Martingale]
        [gpt-4.1]
        假设 $E \xi_{i} = 0$ 且 $E \xi_{i}^{2} = \sigma^{2} \in (0, \infty)$,则 $X_{n} = S_{n} - n \sigma^{2}$ 是一个鞅.
    \end{xmp}
    
    
    
    \begin{xmp}
        [Example-of-Exponential-Martingale]
        {指数鞅的例子}
        [Example of Exponential Martingale]
        [gpt-4.1]
        假设 $\phi(\theta) = E \exp(\theta \xi_{i}) < \infty$,则 $X_{n} = \exp(\theta S_{n}) / \phi(\theta)^{n}$ 是一个鞅.
    \end{xmp}
    
    
    
    \begin{thm}
        [Formulas-for-Symmetric-Simple-Random-Walk]
        {对称简单随机游走的相关公式}
        [Formulas for Symmetric Simple Random Walk]
        [gpt-4.1]
        设 $P(\xi_{i} = 1) = P(\xi_{i} = -1) = 1/2$,$S_{0} = x$,令 $N = \operatorname*{min} \{ n : S_{n} 
otin (a, b) \}$,则有

\[
P_{x}(S_{N} = a) = \frac{b - x}{b - a} \qquad P_{x}(S_{N} = b) = \frac{x - a}{b - a}
\]

(b) $E_{0} N = -ab$,因此 $E_{x} N = (b - x)(x - a)$.
    \end{thm}
    
    
    
    \begin{thm}
        [Criterion-for-Recurrence-and-Limit-Formula]
        {可返性判据与极限公式}
        [Criterion for Recurrence and Limit Formula]
        [gpt-4.1]
        
$\boldsymbol{\theta}$ 是可返的当且仅当 $\varphi(M) \to \infty$ 当 $M \to \infty$,即
\[
\varphi(\infty) \equiv \sum_{m=0}^{\infty} \prod_{j=1}^{m} \frac{q_j}{p_j} = \infty
\]
如果 $\varphi(\infty) < \infty$,则 $P_x(T_0 = \infty) = \varphi(x)/\varphi(\infty)$.

    \end{thm}
    
    
    
    \begin{thm}
        [Tightness-Implies-Existence-of-Weakly-Convergent-Subsequence]
        {紧性蕴含弱收敛子列}
        [Tightness Implies Existence of Weakly Convergent Subsequence]
        [gpt-4.1]
        如果 $\mu _ { n }$ 是紧的,则存在一个弱收敛的子列.
    \end{thm}
    
    
    
    \begin{lma}
        [Lemma-on-Events-with-Infinitely-Small-Probability]
        {关于极小概率事件的引理}
        [Lemma on Events with Infinitely Small Probability]
        [gpt-4.1]
        
如果 $\sum_{n=1}^{\infty} P(\|S_{n}\| < \epsilon) < \infty,$ 那么 $P(\|S_{n}\| < \epsilon \text{ i.o.}) = 0$.
如果 $\sum_{n=1}^{\infty} P(\|S_{n}\| < \epsilon) = \infty,$ 那么 $P(\|S_{n}\| < 2\epsilon \text{ i.o.}) = 1$.

    \end{lma}
    
    
    
    \begin{thm}
        [Asymptotic-Formula-for-the-Distribution-of-Normalized-Random-Variables]
        {关于归一化随机变量分布的渐近公式}
        [Asymptotic Formula for the Distribution of Normalized Random Variables]
        [gpt-4.1]
        
若 $k/n \to x$, 则有
\[
n P(L_{2n} = 2k) \sim \pi^{-1} (x(1-x))^{-1/2}
\]

    \end{thm}
    
    
    
    \begin{thm}
        [Integral-Representation-and-Limit-Behavior-of-Probability-Distribution]
        {概率分布的积分表示及极限行为}
        [Integral Representation and Limit Behavior of Probability Distribution]
        [gpt-4.1]
        
设 $2n a_n =$ 最小的大于等于 $2n a$ 的偶数, $2n b_n =$ 最大的小于等于 $2n b$ 的偶数, 定义 $f_n(x) = n P(L_{2n} = k)$ 当 $2k / 2n \leq x < 2(k+1)/2n$ 时成立,则
\[
P(a \leq L_{2n} / 2n \leq b) = \sum_{k = n a_n}^{n b_n} P(L_{2n} = 2k) = \int_{a_n}^{b_n + 1/n} f_n(x) dx
\]
并且在紧集上一致收敛,
\[
f_n(x) \to f(x) = \pi^{-1} (x(1-x))^{-1/2}
\]
且收敛一致性满足
\[
\sup_{a_n \leq x \leq b_n + 1/n} f_n(x) \to \sup_{a \leq x \leq b} f(x) < \infty
\]

    \end{thm}
    
    
    
    \begin{xmp}
        [Example-of-Renewal-Chain]
        {更新链的例子}
        [Example of Renewal Chain]
        [gpt-4.1]
        $S = \{ 0, 1, 2, \ldots \}$, $f_{k} \geq 0$, 且 $\sum_{k=1}^{\infty} f_{k} = 1$.
    \end{xmp}
    
    
    
    \begin{thm}
        [Existence-and-Uniqueness-of-Probability-Measure]
        {唯一概率测度的存在性}
        [Existence and Uniqueness of Probability Measure]
        [gpt-4.1]
        定理 1.1.11 蕴含对于分布 $F$,存在唯一的概率测度,其分布为 $F$.
    \end{thm}
    
    
    
    \begin{thm}
        [Proof-of-the-Central-Limit-Theorem-via-Characteristic-Functions]
        {中心极限定理的证明(基于特征函数)}
        [Proof of the Central Limit Theorem (via Characteristic Functions)]
        [gpt-4.1]
        设 $F_n$ 是 $S_n - x_n$ 的分布,$h_\theta$ 是测试函数,$\varphi(u)$ 是独立同分布随机变量的特征函数,$\sigma^2$ 是方差.则

\[
E h_\theta (S_n - x_n) = \frac{1}{2\pi} \int \varphi(-u)^n e^{i u x_n} \hat{h}_\theta(u) du
\]

当 $n \to \infty$ 时,记 $x_n / \sqrt{n} \to x$,$\hat{h}_\theta$ 支持在 $[-M, M]$,则

\[
E h_\theta (S_n - x_n) \to n(x) \int h_\theta(y) dy
\]

其中 $n(x) = (2\pi \sigma^2)^{-1/2} \exp\left(-\frac{x^2}{2\sigma^2}\right)$,即极限分布为正态分布.

    \end{thm}
    
    
    
    \begin{dfn}
        [Definition-of-Offspring-Distribution]
        {子代分布的定义}
        [Definition of Offspring Distribution]
        [gpt-4.1]
        $p_{k} = P(\xi_{i}^{n} = k)$ 称为子代分布(offspring distribution).
    \end{dfn}
    
    
    
    \begin{thm}
        [Recurrence-of-2D-Random-Walk-under-CLT-Condition]
        {二维随机游走极限定理与常返性}
        [Recurrence of 2D Random Walk under CLT Condition]
        [gpt-4.1]
        
定理 5.4.9 如果 $S_n$ 是 ${\bf R}^2$ 中的随机游走,且 $S_n / n^{1/2} \Rightarrow$ 一个非退化的正态分布,则 $S_n$ 是常返的.

    \end{thm}
    
    
    
    \begin{dfn}
        [Definition-of-Perverted-Exponential-Probability-Density-Function]
        {畸变指数分布的概率密度函数定义}
        [Definition of Perverted Exponential Probability Density Function]
        [gpt-4.1]
        设 $g(x) = C x^{-3} e^{-x}$ 当 $x \geq 1$ 时,$g(x) = 0$ 否则,其中常数 $C$ 取使得 $g$ 是概率密度函数的值.
    \end{dfn}
    
    
    
    \begin{dfn}
        [Definition-of-Convergence-Domain-for-Laplace-Transform]
        {拉普拉斯变换的收敛域定义}
        [Definition of Convergence Domain for Laplace Transform]
        [gpt-4.1]
        记 $\varphi ( \theta ) = \int e^{\theta x} g(x) dx$,则 $\varphi ( \theta ) < \infty$ 当且仅当 $\theta \leq 1$.
    \end{dfn}
    
    
    
    \begin{ppt}
        [Upper-Bound-for-Derivative-Ratio-of-Laplace-Transform-of-Perverted-Exponential]
        {畸变指数分布的拉普拉斯变换导数比值的上界}
        [Upper Bound for Derivative Ratio of Laplace Transform of Perverted Exponential]
        [gpt-4.1]
        当 $\theta \leq 1$ 时,有
\[
\frac{ \varphi' ( \theta ) }{ \varphi( \theta ) } \leq \frac{ \varphi'(1) }{ \varphi(1) } = \int_{1}^{\infty} C x^{-2} dx \Big/ \int_{1}^{\infty} C x^{-3} dx = 2
\]

    \end{ppt}
    
    
    
    \begin{dfn}
        [Definition-of-Supremum-of-Convergence-for-Laplace-Transform]
        {拉普拉斯变换的收敛上界定义}
        [Definition of Supremum of Convergence for Laplace Transform]
        [gpt-4.1]
        记 $\theta_{+} = \operatorname{sup} \{ \theta : \varphi( \theta ) < \infty \}$.
    \end{dfn}
    
    
    
    \begin{thm}
        [Converse-of-Chung-Fuchs-Theorem]
        {Chung-Fuchs 定理及其逆命题}
        [Converse of Chung-Fuchs Theorem]
        [gpt-4.1]
        
假设 $d=1$.如果弱大数定律以 $S_n / n \to 0$(概率收敛)的形式成立,则 $S_n$ 是常返的(recurrent).

    \end{thm}
    
    
    
    \begin{prf}
        [Proof-of-Chung-Fuchs-Theorem]
        {Chung-Fuchs 定理的证明}
        [Proof of Chung-Fuchs Theorem]
        [gpt-4.1]
        
设 $u_n(x) = P(|S_n| < x)$,其中 $x > 0$.

    \end{prf}
    
    
    
    \begin{xmp}
        [An-Example-of-Simple-Random-Walk]
        {关于简单随机游走的例子}
        [An Example of Simple Random Walk]
        [gpt-4.1]
        设 $X_i$ 表示第 $i$ 步的增量,在 $d$ 维整数格 $\mathbf{Z}^d$ 上的简单随机游走满足

\[
P(X_i = e_j) = P(X_i = -e_j) = \frac{1}{2d}
\]

其中 $e_j$ 为第 $j$ 个单位向量,$j=1,2,\ldots,d$,即每一步等概率地朝任一坐标轴的正负方向移动一个单位.
    \end{xmp}
    
    
    
    \begin{thm}
        [Theorem-on-Convergence-of-Characteristic-Functions]
        {特征函数的收敛定理}
        [Theorem on Convergence of Characteristic Functions]
        [gpt-4.1]
        设 $\psi _ { n } ^ { \epsilon } ( t )$ 表示 $F _ { n } ^ { \epsilon }$ 的特征函数,则定理 3.17 蕴含
\[
\psi _ { n } ^ { \epsilon } ( t ) \to \psi ^ { \epsilon } ( t ) = \int _ { \epsilon } ^ { \infty } e ^ { i t x } \theta \epsilon ^ { \alpha } \alpha x ^ { - ( \alpha + 1 ) } d x + \int _ { - \infty } ^ { - \epsilon } e ^ { i t x } ( 1 - \theta ) \epsilon ^ { \alpha } \alpha | x | ^ { - ( \alpha + 1 ) } d x
\]
当 $n \to \infty$ 时成立.
    \end{thm}
    
    
    
    \begin{thm}
        [Recurrence-Criterion-for-the-Chain-in-M/G/1-Queue]
        {M/G/1 排队系统的链的遍历性判别定理}
        [Recurrence Criterion for the Chain in M/G/1 Queue]
        [gpt-4.1]
        
设 $\mu = \sum k a_{k}$ 是一个服务时间内到达的顾客的平均数.若 $\mu > 1$,该马尔可夫链是遍历的(即所有状态都是遍历的);若 $\mu \leq 1$,该链是常返的.

    \end{thm}
    
    
    
    \begin{thm}
        [Theorem-on-Convergence-and-Recurrence]
        {收敛性与遍历性的定理}
        [Theorem on Convergence and Recurrence]
        [gpt-4.1]
        
The convergence (resp. divergence) of $\sum_{n} P(\|S_n\| < \epsilon)$ for a single value of $\epsilon > 0$ is sufficient for transience (resp. recurrence).

    \end{thm}
    
    
    
    \begin{thm}
        [Stationary-Distribution-Formula-for-a-Recurrence-Chain]
        {可递归链的平稳分布公式}
        [Stationary Distribution Formula for a Recurrence Chain]
        [gpt-4.1]
        
如果 $E \xi < \infty$,则 $\pi(j) = P(\xi > j) / E \xi$ 是该链的一个平稳分布.

    \end{thm}
    
    
    
    \begin{dfn}
        [Definition-of-Notations-Related-to-Stable-Laws]
        {稳定律相关符号的定义}
        [Definition of Notations Related to Stable Laws]
        [gpt-4.1]
        
令
\[
\begin{array}{l}
{\displaystyle I_{n}(\epsilon) = \{ m \leq n : | X_{m} | > \epsilon a_{n} \}} \\
{\displaystyle \hat{S}_{n}(\epsilon) = \sum_{m \in I_{n}(\epsilon)} X_{m}} \\
{\displaystyle \bar{S}_{n}(\epsilon) = S_{n} - \hat{S}_{n}(\epsilon)}
\end{array}
\]

    \end{dfn}
    
    
    
    \begin{thm}
        [Stable-Laws-as-Limits-of-Compound-Poisson-Distributions]
        {稳定律是复合泊松分布的极限}
        [Stable Laws as Limits of Compound Poisson Distributions]
        [gpt-4.1]
        
结合以下两点:
1. 若 $\epsilon_n \to 0$, 则 $\bar{S}_n(\epsilon_n)/a_n \Rightarrow 0$.
2. 固定 $\epsilon$,当 $n \to \infty$ 时,有 $|I_n(\epsilon)| \Rightarrow$ Poisson $(\epsilon^{-\alpha})$,且 $\hat{S}_n(\epsilon)/a_n \Rightarrow$ 一个复合泊松分布:
\[
E \exp ( i t \hat{S}_{n}(\epsilon) / a_{n} ) \to \exp ( - \epsilon^{-\alpha} \{ 1 - \psi^{\epsilon}(t) \} )
\]
由此结合 3.8.2 定理的证明,可得稳定律是复合泊松分布的极限.

    \end{thm}
    
    
    
    \begin{xmp}
        [Example-of-Conditional-Expectation-with-Known-Variable]
        {条件期望的自变量例子}
        [Example of Conditional Expectation with Known Variable]
        [gpt-4.1]
        
如果 $X \in \mathcal{F}$,那么 $E ( X | \mathcal{F} ) = X$;也就是说,如果我们已知 $X$,那么我们的'最佳猜测'就是 $X$ 本身.

    \end{xmp}
    
    
    
    \begin{thm}
        [Criterion-for-Recurrence-of-$S-n$]
        {关于 $S\_n$ 重返性的判别条件}
        [Criterion for Recurrence of $S_n$]
        [gpt-4.1]
        
设 $\delta > 0$.$S_n$ 重返当且仅当
\[
\int_{(-\delta, \delta)^d} \mathrm{Re} \frac{1}{1 - \varphi(y)} dy = \infty
\]

    \end{thm}
    
    
    
    \begin{thm}
        [Expectation-Inequality-for-Submartingales-at-Stopping-Times]
        {关于次超鞅在停止时间上的期望不等式}
        [Expectation Inequality for Submartingales at Stopping Times]
        [gpt-4.1]
        
设 $X_n$ 是次超鞅(submartingale),$M \leq N$ 为停止时间,且 $P(N \leq k) = 1$,则有
\[
E X_M \leq E X_N.
\]

    \end{thm}
    
    
    
    \begin{lma}
        [Lemma-on-the-Hitting-Time-Probability-for-Simple-Random-Walk]
        {关于简单随机游走到达零时刻概率的引理}
        [Lemma on the Hitting Time Probability for Simple Random Walk]
        [gpt-4.1]
        
$P(S_1 
eq 0, \ldots, S_{2n} 
eq 0) = P(S_{2n} = 0)$.

    \end{lma}
    
    
    
    \begin{prf}
        [Proof-of-the-Lemma-on-the-Hitting-Time-Probability-for-Simple-Random-Walk]
        {关于简单随机游走到达零时刻概率的证明}
        [Proof of the Lemma on the Hitting Time Probability for Simple Random Walk]
        [gpt-4.1]
        
$P(S_1 > 0, \ldots, S_{2n} > 0) = \sum_{r=1}^{\infty} P(S_1 > 0, \ldots, S_{2n-1} > 0, S_{2n} = 2r)$.

根据定理4.9.2的证明,我们知道从$(0,0)$到$(2n, 2r)$且在正时刻不为0的路径数(即从$(1,1)$到$(2n, 2r)$且在正时刻不为0的路径数)为
\[
N_{2n-1, 2r-1} - N_{2n-1, 2r+1}
\]

令$p_{n, x} = P(S_n = x)$,则有
\[
P(S_1 > 0, \ldots, S_{2n-1} > 0, S_{2n} = 2r) = \frac{1}{2} (p_{2n-1, 2r-1} - p_{2n-1, 2r+1})
\]

对$r=1$到$\infty$求和得到
\[
P(S_1 > 0, \ldots, S_{2n} > 0) = \frac{1}{2} p_{2n-1, 1} = \frac{1}{2} P(S_{2n} = 0)
\]

由对称性知$P(S_1 < 0, \ldots, S_{2n} < 0) = \frac{1}{2} P(S_{2n} = 0)$,证毕.

    \end{prf}
    
    
    
    \begin{thm}
        [Conditional-Distribution-Theorem-for-Arrival-Times-in-Poisson-Process]
        {泊松过程到达时刻的条件分布定理}
        [Conditional Distribution Theorem for Arrival Times in Poisson Process]
        [gpt-4.1]
        
设 $T_n$ 为速率为 $\lambda$ 的泊松过程中的第 $n$ 次到达时刻.令 $U_1, U_2, \dots, U_n$ 是在区间 $(0, t)$ 上的独立均匀分布随机变量,$V_k^n$ 为 $\{U_1, \ldots, U_n\}$ 中第 $k$ 小的数.如果条件为 $N(t) = n$,则向量 $V = (V_1^n, \ldots, V_n^n)$ 与 $T = (T_1, \dots, T_n)$ 有相同的分布.

    \end{thm}
    
    
    
    \begin{thm}
        [Recurrence-Criterion-for-Markov-Chains]
        {马尔可夫链遍历性的判别定理}
        [Recurrence Criterion for Markov Chains]
        [gpt-4.1]
        
设 $S$ 是不可约的状态空间,$\varphi \geq 0$,且对所有 $x 
otin F$($F$ 为有限集)有 $E_{x} \varphi (X_{1}) \leq \varphi(x)$,并且 $\varphi(x) \to \infty$ 随 $x \to \infty$,即对任意有限 $M$,集合 $\{ x : \varphi(x) \leq M \}$ 是有限的,则该马尔可夫链是遍历的(recurrent).

    \end{thm}
    
    
    
    \begin{dfn}
        [Definition-of-Number-of-Renewals-in-Interval-for-Two-Processes]
        {两个过程在区间内的更新次数的定义}
        [Definition of Number of Renewals in Interval for Two Processes]
        [gpt-4.1]
        
设
\[
N^{\prime}[s, t] = |\{n : T_{n}^{\prime} \in [s, t]\}| \quad \text{和} \quad N^{\prime\prime}[s, t] = |\{n : T_{n}^{\prime\prime} \in [s, t]\}|
\]
分别表示两个过程在区间 $[s, t]$ 内的更新次数.

    \end{dfn}
    
    
    
    \begin{thm}
        [Upper-and-Lower-Bounds-for-Expected-Number-of-Renewals]
        {期望更新次数的上界和下界估计}
        [Upper and Lower Bounds for Expected Number of Renewals]
        [gpt-4.1]
        
若 $\epsilon < h/2$,则有
\[
\begin{array}{rl}
U([t, t + h]) &= E N^{\prime\prime}[t, t + h] \geq E(N^{\prime}[t + \epsilon,\ t + h - \epsilon];\ T_{J} \leq t) \\
& \qquad \quad \geq \displaystyle \frac{h - 2\epsilon}{\mu} - P(T_{J} > t) U(h)
\end{array}
\]
且
\[
\begin{array}{rl}
U([t, t + h]) &\leq E(N^{\prime}[t - \epsilon,\ t + h + \epsilon];\ T_{J} \leq t) + E(N^{\prime\prime}[t, t + h];\ T_{J} > t) \\
&\leq \frac{h + 2\epsilon}{\mu} + P(T_{J} > t) U(h)
\end{array}
\]
其中 $P(T_{J} > t) \to 0$ 且 $\epsilon < h / 2$ 是任意的.

    \end{thm}
    
    
    
    \begin{dfn}
        [Definition-of-Recurrent-Value-for-Random-Walk]
        {随机游走的常返值的定义}
        [Definition of Recurrent Value for Random Walk]
        [gpt-4.1]
        数 $x \in \mathbf{R}^d$ 称为随机游走 $S_n$ 的常返值,如果对任意 $\epsilon > 0$,有 $P(\| S_n - x \| < \epsilon \text{ i.o.}) = 1$.
    \end{dfn}
    
    
    
    \begin{dfn}
        [Definition-of-Possible-Value-for-Random-Walk]
        {随机游走的可能值的定义}
        [Definition of Possible Value for Random Walk]
        [gpt-4.1]
        数 $x$ 称为随机游走的可能值,如果对任意 $\epsilon > 0$,存在 $n$ 使得 $P(\| S_n - x \| < \epsilon) > 0$.
    \end{dfn}
    
    
    
    \begin{dfn}
        [Characteristic-Function-of-Multivariate-Random-Variable]
        {多元随机变量的特征函数}
        [Characteristic Function of Multivariate Random Variable]
        [gpt-4.1]
        
$( X _ { 1 } , \ldots , X _ { d } )$ 的特征函数定义为 $\varphi ( t ) = E \exp ( i t \cdot X )$, 其中 $t \cdot X = t _ { 1 } X _ { 1 } + \cdots + t _ { d } X _ { d }$ 是两个向量的通常点积.

    \end{dfn}
    
    
    
    \begin{thm}
        [Uniqueness-of-Stationary-Measure-for-Irreducible-and-Recurrent-Case]
        {不可约且常返的情况下平稳测度的唯一性}
        [Uniqueness of Stationary Measure for Irreducible and Recurrent Case]
        [gpt-4.1]
        如果 $p$ 是不可约且常返的(即所有状态都是常返的),则平稳测度在常数倍意义下是唯一的.
    \end{thm}
    
    
    
    \begin{dfn}
        [Definition-of-Transition-Probability-on-Product-Space]
        {乘积空间上的转移概率的定义}
        [Definition of Transition Probability on Product Space]
        [gpt-4.1]
        
定义在 $S \times S$ 上的转移概率 $\bar{p}$ 为
\[
\bar{p}((x_{1}, y_{1}), (x_{2}, y_{2})) = p(x_{1}, x_{2}) p(y_{1}, y_{2}),
\]
即每个坐标独立地移动.

    \end{dfn}
    
    
    
    \begin{thm}
        [Theorem-on-Convergence-of-Adapted-Process]
        {关于适应过程的收敛性定理}
        [Theorem on Convergence of Adapted Process]
        [gpt-4.1]
        设 $X_n$ 和 $Y_n$ 是正的、可积的、并适应于 ${\mathcal{F}}_n$.若 $E(X_{n+1} | {\mathcal{F}}_n) \leq X_n + Y_n$,且 $\sum Y_n < \infty$,则 $X_n$ 收敛(几乎处处收敛).
    \end{thm}
    
    
    
    \begin{thm}
        [Equivalence-of-Weak-Convergence-in-$\mathbf{R}^d$-and-General-Metric-Spaces]
        {实空间上的弱收敛等价于度量空间上的弱收敛}
        [Equivalence of Weak Convergence in $\mathbf{R}^d$ and General Metric Spaces]
        [gpt-4.1]
        在 $\mathbf{R}^d$ 上,通过分布收敛 $F_n \Rightarrow F$ 所定义的弱收敛,与在一般度量空间上定义的弱收敛是等价的.
    \end{thm}
    
    
    
    \begin{xmp}
        [A-Concrete-Example-of-Bivariate-Distribution-Function]
        {二维分布函数的具体例子}
        [A Concrete Example of Bivariate Distribution Function]
        [gpt-4.1]
        
考虑如下函数:

\[
F(x, y) =
\begin{cases}
1 & \text{if } x \geq 0, y \geq 1 \\
y & \text{if } x \geq 0, 0 \leq y < 1 \\
0 & \text{otherwise}
\end{cases}
\]

$F$ 是 $(0, Y)$ 的分布函数,其中 $Y$ 在 $(0,1)$ 上服从均匀分布.注意该分布没有原子,但当 $y > 0$ 时,$F$ 在 $(0, y)$ 处不连续.

    \end{xmp}
    
    
    
    \begin{thm}
        [Theorem-on-Recurrence-of-Three-Dimensional-Random-Walks]
        {三维随机游走的遍历性定理}
        [Theorem on Recurrence of Three-Dimensional Random Walks]
        [gpt-4.1]
        No truly three-dimensional random walk is recurrent.
    \end{thm}
    
    
    
    \begin{thm}
        [Theorem-on-the-Existence-of-the-Limit-of-Random-Variables]
        {随机变量极限存在性定理}
        [Theorem on the Existence of the Limit of Random Variables]
        [gpt-4.1]
        
$\operatorname* { lim } _ { n \to \infty } X _ { n }$ 存在且有限于 $\{ A _ { \infty } < \infty \}$ 上.

    \end{thm}
    
    
    
    \begin{thm}
        [Limit-of-Conditional-Expectations-with-Increasing-Sigma-Fields]
        {条件期望的极限与递增σ域}
        [Limit of Conditional Expectations with Increasing Sigma-Fields]
        [gpt-4.1]
        
设 $\mathcal{F}_{n} \uparrow \mathcal{F}_{\infty}$,即 ${\mathcal{F}}_{n}$ 是递增的 $\sigma$-域序列,且 ${\mathcal{F}}_{\infty} = \sigma\left(\cup_{n} {\mathcal{F}}_{n}\right)$.则当 $n \to \infty$ 时,

\[
E(X|\mathcal{F}_{n}) \to E(X|\mathcal{F}_{\infty}) \quad \text{几乎处处和在 } L^{1} \text{ 中收敛}
\]

    \end{thm}
    
    
    
    \begin{prf}
        [Proof-of-the-Limit-of-Conditional-Expectations-Theorem]
        {条件期望极限定理的证明}
        [Proof of the Limit of Conditional Expectations Theorem]
        [gpt-4.1]
        
首先注意到如果 $m > n$,则定理 4.1.13 得到

\[
E(E(X|\mathcal{F}_{m})|\mathcal{F}_{n}) = E(X|\mathcal{F}_{n})
\]

因此 $Y_{n} = E(X|\mathcal{F}_{n})$ 构成一个鞅.定理 4.6.1 说明 $Y_{n}$ 是一致可积的,所以定理 4.6.7 得到 $Y_{n}$ 在几乎处处和 $L^{1}$ 中收敛到极限 $Y_{\infty}$.由 $Y_{n}$ 的定义和引理 4.6.6,有 $E(X|\mathcal{F}_{n}) = Y_{n} = E(Y_{\infty}|\mathcal{F}_{n})$,因此

\[
\int_{A} X dP = \int_{A} Y_{\infty} dP \quad \text{对所有 } A \in {\mathcal{F}}_{n}
\]

由于 $X$ 和 $Y_{\infty}$ 可积,且 $\cup_{n} {\mathcal{F}}_{n}$ 是 $\pi$-系,$\pi$-$\lambda$ 定理说明上述结果对所有 $A \in {\mathcal{F}}_{\infty}$ 成立.由于 $Y_{\infty} \in \mathcal{F}_{\infty}$,因此 $Y_{\infty} = E(X|\mathcal{F}_{\infty})$.

    \end{prf}
    
    
    
    \begin{thm}
        [Scheffés-Theorem]
        {Scheffé定理}
        [Scheffé's Theorem]
        [gpt-4.1]
        
设 $f_n$ 和 $f_\infty$ 为概率密度函数,定义在 $-\infty < x < \infty$ 上,且 $f_n \to f_\infty$ 逐点收敛.当 $n \to \infty$ 时,对于任意Borel集 $B$,有
\[
\left| \int_{B} f_{n}(x) dx - \int_{B} f_{\infty}(x) dx \right| \leq \int |f_{n}(x) - f_{\infty}(x)| dx = 2 \int (f_{\infty}(x) - f_{n}(x))^{+} dx \to 0,
\]
其中等号成立是因为 $f_n \geq 0$ 且 $\int f_n = 1$.由主导收敛定理知,上述极限成立.

    \end{thm}
    
    
    
    \begin{dfn}
        [Definition-of-Total-Variation-Norm-of-Measures]
        {测度的全变差范数的定义}
        [Definition of Total Variation Norm of Measures]
        [gpt-4.1]
        
设 $\mu_n$ 和 $\mu_\infty$ 是概率测度,定义它们的全变差范数为
\[
\| \mu_{n} - \mu_{\infty} \| \equiv \operatorname*{sup}_{B} |\mu_{n}(B) - \mu_{\infty}(B)|,
\]
其中上确界取遍所有 Borel 集 $B$.

    \end{dfn}
    
    
    
    \begin{cxmp}
        [An-Important-Counterexample-13]
        {13 的一个重要反例}
        [An Important Counterexample: 13]
        [gpt-4.1]
        13 是本章剩余内容中需要牢记的一个重要反例.
    \end{cxmp}
    
    
    
    \begin{dfn}
        [Definition-of-Conditional-Expectation]
        {条件期望的定义}
        [Definition of Conditional Expectation]
        [gpt-4.1]
        条件期望(Conditional Expectation)是本章及下一章中非常重要的概念.
    \end{dfn}
    
    
    
    \begin{lma}
        [Upper-Bound-on-the-Distance-Between-Bernoulli-and-Poisson-Distributions]
        {伯努利分布与泊松分布的距离上界}
        [Upper Bound on the Distance Between Bernoulli and Poisson Distributions]
        [gpt-4.1]
        设 $\mu$ 是满足 $\mu(1) = p$ 且 $\mu(0) = 1-p$ 的测度,$
u$ 是参数为 $p$ 的泊松分布,则有
\[
\| \mu - 
u \| \leq p^2.
\]

    \end{lma}
    
    
    
    \begin{thm}
        [Conditional-Expectation-Formula-at-Stopping-Times]
        {关于停时的条件期望公式}
        [Conditional Expectation Formula at Stopping Times]
        [gpt-4.1]
        \[
E_{ \mu } ( Y_{ N } \circ \theta_{ N } | \mathcal{ F }_{ N } ) = E_{ X_{ N } } Y_{ N } \ \text{on} \ \left\{ N < \infty \right\}
\]

其中右端为 $\varphi ( x , n ) = E_{ x } Y_{ n }$,在 $x = X_{ N },\ n = N$ 处取值.

    \end{thm}
    
    
    
    \begin{prf}
        [Proof-of-Conditional-Expectation-Formula-at-Stopping-Times]
        {关于停时的条件期望公式的证明}
        [Proof of Conditional Expectation Formula at Stopping Times]
        [gpt-4.1]
        证明 令 $A \in \mathcal{ F }_{ N }$.
将问题根据 $N$ 的取值拆解:

\[
E_{ \mu } ( Y_{ N } \circ \theta_{ N } ; A \cap \{ N < \infty \} ) = \sum_{ n = 0 }^{ \infty } E_{ \mu } ( Y_{ n } \circ \theta_{ n } ; A \cap \{ N = n \} )
\]

由于 $A \cap \{ N = n \} \in \mathcal{ F }_{ n }$,利用定理 5.2.3 可将右侧转化为

\[
\sum_{ n = 0 }^{ \infty } E_{ \mu } ( E_{ X_{ n } } Y_{ n } ; A \cap \{ N = n \} ) = E_{ \mu } ( E_{ X_{ N } } Y_{ N } ; A \cap \{ N < \infty \} )
\]

因此得证.

    \end{prf}
    
    
    
    \begin{thm}
        [Necessary-and-Sufficient-Condition-for-$S-n$-to-be-Recurrent]
        {$S\_n$ 递归的充要条件}
        [Necessary and Sufficient Condition for $S_n$ to be Recurrent]
        [gpt-4.1]
        
$S_n$ 是递归的当且仅当
\[
\operatorname*{sup}_{r < 1} \int_{(-\delta, \delta)^d} \mathrm{Re} \frac{1}{1 - r \varphi(y)} dy = \infty
\]
其中 $\delta > 0$.

    \end{thm}
    
    
    
    \begin{dfn}
        [Definition-of-Good-Rectangle]
        {好矩形的定义}
        [Definition of Good Rectangle]
        [gpt-4.1]
        称矩形 $A = ( a _ { 1 } , b _ { 1 } ] \times \ldots \times ( a _ { d } , b _ { d } ]$ 是好矩形,如果对所有 $i$,都有 $a _ { i }, b _ { i } 
otin D ^ { i }$.
    \end{dfn}
    
    
    
    \begin{xmp}
        [An-Example-of-Branching-Process]
        {分支过程的一个例子}
        [An Example of Branching Process]
        [gpt-4.1]
        
如果没有子代的概率为正,则 $\rho_{k0} > 0$ 且对于 $k \geq 1$ 有 $\rho_{0k} = 0$,因此根据定理 5.3.3,所有状态 $k \geq 1$ 都是瞬态的.

    \end{xmp}
    
    
    
    \begin{dfn}
        [Definition-of-First-Hitting-Time]
        {首次到达时间的定义}
        [Definition of First Hitting Time]
        [gpt-4.1]
        设 $T_{x} = \operatorname*{min} \{ n : S_{n} = x \}$.
    \end{dfn}
    
    
    
    \begin{thm}
        [Hitting-Probabilities-of-One-Dimensional-Random-Walk]
        {一维随机游走的首次到达概率}
        [Hitting Probabilities of One-Dimensional Random Walk]
        [gpt-4.1]
        取 $a = 0, x = 1$ 和 $b = M$,则有
\[
P_{1}(T_{M} < T_{0}) = \frac{1}{M} \qquad P_{1}(T_{0} < T_{M}) = \frac{M-1}{M}
\]

    \end{thm}
    
    
    
    \begin{crl}
        [Limit-of-Probability-of-First-Hitting-0]
        {首次到达原点的概率极限}
        [Limit of Probability of First Hitting 0]
        [gpt-4.1]
        令 $M \to \infty$,则有 $P_{1}(T_{0} < \infty) = 1$.
    \end{crl}
    
    
    
    \begin{thm}
        [Conditional-Hölder-Inequality]
        {条件Hölder不等式}
        [Conditional Hölder Inequality]
        [gpt-4.1]
        
设 $p, q \in (1, \infty)$ 满足 $1/p + 1/q = 1$,则对于随机变量 $X, Y$ 和次σ-代数 $\mathcal{G}$,有
\[
E(|XY| \mid \mathcal{G}) \leq E(|X|^p \mid \mathcal{G})^{1/p} E(|Y|^q \mid \mathcal{G})^{1/q}
\]

    \end{thm}
    
    
    
    \begin{thm}
        [Chapman-Kolmogorov-Equation]
        {Chapman-Kolmogorov方程}
        [Chapman-Kolmogorov Equation]
        [gpt-4.1]
        
设$\{X_n\}$为马尔可夫过程,则对任意状态$x,z$与非负整数$m,n$,有
\[
P_{x}(X_{m+n}=z) = \sum_{y} P_{x}(X_{m}=y)\, P_{y}(X_{n}=z)
\]
直观上,若要在$m+n$步由$x$到达$z$,则在第$m$步必然到达某个$y$,而马尔可夫性质保证对于给定$y$,前后两段过程相互独立.

    \end{thm}
    
    
    
    \begin{prf}
        [Proof-of-Chapman-Kolmogorov-Equation]
        {Chapman-Kolmogorov方程的证明}
        [Proof of Chapman-Kolmogorov Equation]
        [gpt-4.1]
        
$P_{x}(X_{n+m}=z) = E_{x}(P_{x}(X_{n+m}=z \mid \mathcal{F}_m)) = E_{x}(P_{X_{m}}(X_{n}=z))$,这是由马尔可夫性质与定理5.2.3得到的,因为$1_{(X_{n}=z)} \circ \theta_{m} = 1_{(X_{n+m}=z)}$.

    \end{prf}
    
    
    
    \begin{thm}
        [Cramer-Wold-Device]
        {Cramer-Wold判据}
        [Cramer-Wold Device]
        [gpt-4.1]
        
充分条件:若对所有 $\boldsymbol{\theta} \in \mathbf{R}^d$ 都有 $\theta \cdot X_n \Rightarrow \theta \cdot X_\infty$,则有 $X_n \Rightarrow X_\infty$.

    \end{thm}
    
    
    
    \begin{prf}
        [Proof-of-Cramer-Wold-Device]
        {Cramer-Wold判据的证明}
        [Proof of Cramer-Wold Device]
        [gpt-4.1]
        
该条件意味着 $E \exp(i \theta \cdot X_n) \to E \exp(i \theta \cdot X_\infty)$ 对所有 $\boldsymbol{\theta} \in \mathbf{R}^d$ 成立.

    \end{prf}
    
    
    
    \begin{thm}
        [Stopping-Time-Defined-Piecewise]
        {关于分段定义的停时}
        [Stopping Time Defined Piecewise]
        [gpt-4.1]
        若 $A \in \mathcal{F}_{M}$,则

\[
L = \begin{cases}
M & \text{on } A \\
N & \text{on } A^{c}
\end{cases}
\]

是一个停时(stopping time).
    \end{thm}
    
    
    
    \begin{dfn}
        [Definition-of-First-Entrance-Time]
        {首次进入时刻的定义}
        [Definition of First Entrance Time]
        [gpt-4.1]
        设 $T _ { y } = \operatorname*{inf} \{ n \geq 1 : X _ { n } = y \}$,其中 $T_y$ 表示随机过程 $\{X_n\}$ 首次进入状态 $y$ 的时刻.
    \end{dfn}
    
    
    
    \begin{thm}
        [First-Entrance-Decomposition-of-Transition-Probability]
        {转移概率的首次进入分解公式}
        [First Entrance Decomposition of Transition Probability]
        [gpt-4.1]
        对于马尔可夫链的转移概率,有
\[
p ^ { n } ( x , y ) = \sum _ { m = 1 } ^ { n } P _ { x } ( T _ { y } = m ) p ^ { n - m } ( y , y )
\]
其中 $p^n(x,y)$ 表示从状态 $x$ 经过 $n$ 步到达状态 $y$ 的概率,$P_x(T_y = m)$ 表示从 $x$ 出发,首次在第 $m$ 步到达 $y$ 的概率.
    \end{thm}
    
    
    
    \begin{thm}
        [Poisson-Limit-Theorem-for-Normalized-Random-Variables]
        {归一化随机变量分布极限的泊松定理}
        [Poisson Limit Theorem for Normalized Random Variables]
        [gpt-4.1]
        
若 $0 < a < b$ 且 $a n^{1/\alpha} > 1$,则有
\[
P ( a n^{1/\alpha} < X_1 < b n^{1/\alpha} ) = \frac{1}{2} ( a^{-\alpha} - b^{-\alpha} ) n^{-1}
\]
因此,根据定理 3.6.1,
\[
N_n ( a, b ) \equiv | \{ m \leq n : X_m / n^{1/\alpha} \in ( a, b ) \} | \Rightarrow N ( a, b )
\]
其中 $N(a, b)$ 服从均值为 $( a^{-\alpha} - b^{-\alpha} ) / 2$ 的泊松分布.

    \end{thm}
    
    
    
    \begin{thm}
        [Convergence-Theorem]
        {收敛定理}
        [Convergence Theorem]
        [gpt-4.1]
        
假设 $p$ 是不可约的、无周期的(即所有状态有 $d_{x} = 1$),并且有平稳分布 $\pi$.则当 $n \to \infty$ 时,$p^{n}(x, y) \to \pi(y)$.

    \end{thm}
    
    
    
    \begin{dfn}
        [Definition-of-Density-Function-of-a-Distribution]
        {分布的密度函数的定义}
        [Definition of Density Function of a Distribution]
        [gpt-4.1]
        若分布 $F$ 满足
\[
F ( x _ { 1 } , \dots , x _ { k } ) = \int _ { - \infty } ^ { x _ { 1 } } \dots \int _ { - \infty } ^ { x _ { k } } f ( y ) d y _ { k } \dots d y _ { 1 }
\]
则称 $F$ 有密度 $f$.
    \end{dfn}
    
    
    
    \begin{thm}
        [Theorem-on-the-Probability-of-the-Supremum-of-Random-Variables]
        {关于随机变量极大值概率的定理}
        [Theorem on the Probability of the Supremum of Random Variables]
        [gpt-4.1]
        (i) 若 $P(\alpha < \infty) < 1$,则 $P(\sup S_{n} < \infty) = 1$.
(ii) 若 $P(\alpha < \infty) = 1$,则 $P(\sup S_{n} = \infty) = 1$.
    \end{thm}
    
    
    
    \begin{prf}
        [Proof-of-Equality-in-Measure-Inequality]
        {关于测度不等式等号成立的证明}
        [Proof of Equality in Measure Inequality]
        [gpt-4.1]
        由于 $
u ( x ) \geq 
u ( a ) \mu _ { a } ( x )$,且左右两边相等,故当 $p ^ { n } ( x , a ) > 0$ 时,必有 $
u ( x ) = 
u ( a ) \mu _ { a } ( x )$.由于 $p$ 是不可约的,故对所有 $x \in S$,都有 $
u ( x ) = 
u ( a ) \mu _ { a } ( x )$,证明完成.
    \end{prf}
    
    
    
    \begin{dfn}
        [Definition-and-Intuitive-Explanation-of-Martingale]
        {鞅的定义与直观解释}
        [Definition and Intuitive Explanation of Martingale]
        [gpt-4.1]
        鞅(martingale)$X_n$ 可以被看作是在公平游戏中,第 $n$ 时刻下注者的财富;亚鞅(submartingale)(上鞅)和超鞅(supermartingale)(下鞅)分别是下注于有利(不利)游戏的结果.
    \end{dfn}
    
    
    
    \begin{dfn}
        [Definition-of-Truly-Three-Dimensional-Random-Walk]
        {三维随机游走的真正三维性定义}
        [Definition of Truly Three-Dimensional Random Walk]
        [gpt-4.1]
        
若 $X_1$ 的分布满足对所有 $\theta 
eq 0$ 都有 $P ( X_1 \cdot \theta 
eq 0 ) > 0$,则称 $\mathbf{R}^3$ 中的随机游走是'真正三维的'.

    \end{dfn}
    
    
    
    \begin{thm}
        [Distributional-Equivalence-of-Poisson-Process-Arrival-Times-and-Order-Statistics-of-Uniform-Variables]
        {泊松过程到达时刻与均匀分布顺序统计量的分布一致性}
        [Distributional Equivalence of Poisson Process Arrival Times and Order Statistics of Uniform Variables]
        [gpt-4.1]
        
设 $T_n$ 为速率为 $\lambda$ 的泊松过程中的第 $n$ 次到达时刻.设 $U_1, U_2, \dots, U_n$ 为在 $(0, 1)$ 上独立均匀分布的随机变量,$V_k^n$ 为 $\{U_1, \ldots, U_n\}$ 中第 $k$ 小的数.则向量 $(V_1^n, \ldots, V_n^n)$ 与 $(T_1 / T_{n+1}, \ldots, T_n / T_{n+1})$ 具有相同的分布.

    \end{thm}
    
    
    
    \begin{thm}
        [Theorem-on-Finiteness-of-Measure-in-Recurrent-States]
        {关于递归状态下测度有限性的定理}
        [Theorem on Finiteness of Measure in Recurrent States]
        [gpt-4.1]
        
当 $x$ 是递归状态时,若 $\rho_{xy} > 0$,则 $\rho_{yx} > 0$,由此可得 $\mu_{x}(y) < \infty$.

    \end{thm}
    
    
    
    \begin{ppt}
        [Property-of-Finiteness-of-Measure]
        {测度有限性的性质}
        [Property of Finiteness of Measure]
        [gpt-4.1]
        
若 $\rho_{xy} = 0$,则 $\mu_{x}(y) = 0$.

    \end{ppt}
    
    
    
    \begin{prf}
        [Proof-of-Properties-of-the-Generating-Function-for-Binomial-Distribution]
        {二项分布生成函数的性质证明}
        [Proof of Properties of the Generating Function for Binomial Distribution]
        [gpt-4.1]
        
证明 $\varphi ( 1 ) = 1$.对该式求导,并参考定理A.5的结论,有,对于 $s < 1$
\[
\varphi ^ { \prime } ( s ) = \sum _ { k = 1 } ^ { \infty } k p _ { k } s ^ { k - 1 } \geq 0,
\]

    \end{prf}
    
    
    
    \begin{thm}
        [Martingale-Property-of-the-Function-$\varphiy$]
        {关于函数$\varphi(y)$的鞅性质}
        [Martingale Property of the Function $\varphi(y)$]
        [gpt-4.1]
        
若 $\varphi(y) = \left( \frac{1-p}{p} \right)^{y}$,则 $\varphi(S_{n})$ 是一个鞅(martingale).

    \end{thm}
    
    
    
    \begin{thm}
        [Infinite-Divisibility-of-Sums-of-Independent-Poisson-Random-Variables]
        {独立泊松和的无限可分性}
        [Infinite Divisibility of Sums of Independent Poisson Random Variables]
        [gpt-4.1]
        设 $\xi_1, \xi_2, \ldots$ 为独立同分布的随机变量,$N(\lambda)$ 为均值为 $\lambda$ 的独立泊松随机变量,则 $Z = \xi_1 + \cdots + \xi_{N(\lambda)}$ 具有无限可分分布.
    \end{thm}
    
    
    
    \begin{thm}
        [Equivalent-Conditions-for-Positive-Recurrence-in-Irreducible-Markov-Chains]
        {不可约马尔可夫链的正遍历性等价条件}
        [Equivalent Conditions for Positive Recurrence in Irreducible Markov Chains]
        [gpt-4.1]
        设 $p$ 是不可约的,则以下条件等价:
(i) 某个状态 $x$ 是正遍历的;
(ii) 存在平稳分布;
(iii) 所有状态都是正遍历的.

    \end{thm}
    
    
    
    \begin{thm}
        [Recurrence/Transience-Criterion-for-Birth-Death-Chains]
        {出生死亡链递归/瞬时性判别定理}
        [Recurrence/Transience Criterion for Birth-Death Chains]
        [gpt-4.1]
        
设 $p_{j} = 1/2 + \epsilon_{j}$,其中 $\epsilon_{j} \sim C j^{-\alpha}$ 当 $j \to \infty$,$q_{j} = 1 - p_{j}$.则有:

(i) 若 $\alpha > 1$,则 0 是递归点;
(ii) 若 $\alpha < 1$,则 0 是瞬时点;
(iii) 若 $\alpha = 1$,则当 $C > 1/4$ 时,0 是瞬时点;当 $C < 1/4$ 时,0 是递归点.

    \end{thm}
    
    
    
    \begin{thm}
        [Theorem-on-Transition-Probabilities-for-Finite-Irreducible-Aperiodic-Markov-Chains]
        {有限不可约非周期马尔可夫链的状态间转移概率定理}
        [Theorem on Transition Probabilities for Finite Irreducible Aperiodic Markov Chains]
        [gpt-4.1]
        
若 $S$ 是有限集,$p$ 是不可约且非周期的转移矩阵,则存在整数 $m$,使得对任意 $x, y \in S$,有 $p^m(x, y) > 0$.

    \end{thm}
    
    
    
    \begin{thm}
        [Generalization-of-the-Existence-of-Conditional-Probability-Measures]
        {条件概率测度的存在性推广}
        [Generalization of the Existence of Conditional Probability Measures]
        [gpt-4.1]
        
设 $X$ 和 $Y$ 取值于良好空间 $(S, S)$,并且 $\mathcal{G} = \sigma(\boldsymbol{Y})$.存在函数 $\mu : S \times S \to [0, 1]$,使得:

(i) 对每个 $A$,$\mu(Y(\omega), A)$ 是 $P(X \in A \mid \mathcal{G})$ 的一个版本;
(ii) 对任意 $\omega$,$A \mapsto \mu(Y(\omega), A)$ 是 $(S, S)$ 上的概率测度.

    \end{thm}
    
    
    
    \begin{thm}
        [Convergence-Theorem-for-Martingales-with-Bounded-Increments]
        {有界增量的鞅收敛性定理}
        [Convergence Theorem for Martingales with Bounded Increments]
        [gpt-4.1]
        有界增量的鞅要么收敛,要么在 $+\infty$ 和 $-\infty$ 之间振荡.
    \end{thm}
    
    
    
    \begin{thm}
        [Theorem-on-the-Cycle-Trick]
        {关于循环技巧的定理}
        [Theorem on the Cycle Trick]
        [gpt-4.1]
        
设 $x$ 是一个常返态,令 $T = \inf\{ n \geq 1 : X_{n} = x \}$.则

\[
\mu_{x}(y) = E_{x} \left( \sum_{n=0}^{T-1} 1_{\{ X_{n} = y \}} \right) = \sum_{n=0}^{\infty} P_{x}(X_{n} = y, T > n)
\]

定义了一个平稳测度.

    \end{thm}
    
    
    
    \begin{dfn}
        [Definition-of-Tightness-for-a-Sequence-of-Probability-Measures]
        {概率测度列的紧性定义}
        [Definition of Tightness for a Sequence of Probability Measures]
        [gpt-4.1]
        若概率测度列 $\mu_n$ 满足: 对任意 $\epsilon > 0$,存在 $M$,使得
\[
\liminf_{n \to \infty} \mu_n([ - M , M ]^d) \geq 1 - \epsilon,
\]
则称$\mu_n$是紧的.
    \end{dfn}
    
    
    
    \begin{thm}
        [Three-Possibilities-for-the-Modulus-of-the-Characteristic-Function]
        {特征函数模值的三种可能性}
        [Three Possibilities for the Modulus of the Characteristic Function]
        [gpt-4.1]
        设 $\varphi(t) = E e^{i t X}$,则只有以下三种可能性:

(i) 对所有 $t 
eq 0$,有 $|\varphi(t)| < 1$;

(ii) 存在 $\lambda > 0$,使得 $|\varphi(\lambda)| = 1$ 且当 $0 < t < \lambda$ 时 $|\varphi(t)| < 1$.此时,$X$ 具有以 $2\pi/\lambda$ 为跨度的格分布;

(iii) 对所有 $t$,$|\varphi(t)| = 1$.此时,$X = b$ 几乎处处成立,其中 $b$ 为某常数.

    \end{thm}
    
    
    
    \begin{thm}
        [Inequality-for-$E--H-\cdot-Y--n$]
        {关于 $E ( H \cdot Y )_n$ 的不等式}
        [Inequality for $E ( H \cdot Y )_n$]
        [gpt-4.1]
        
设 $Y_n - Y_0 = ( H \cdot Y )_n + ( K \cdot Y )_n$,其中 $K_m = 1 - H_m$.由定理 4.2.8 可得 $E ( K \cdot Y )_n \geq E ( K \cdot Y )_0 = 0$,因此 $E ( H \cdot Y )_n \leq E ( Y_n - Y_0 )$,从而证明了所需的不等式.

    \end{thm}
    
    
    
    \begin{xmp}
        [Example-of-Quadratic-Martingale]
        {二次型鞅的例子}
        [Example of Quadratic Martingale]
        [gpt-4.1]
        
设 $\mu = E \xi_{i} = 0$ 且 $\sigma^{2} = \operatorname{var}(\xi_{i}) < \infty$,则 $S_{n}^{2} - n \sigma^{2}$ 是一个鞅.

    \end{xmp}
    
    
    
    \begin{dfn}
        [Definition-of-Distribution-Function]
        {分布函数的定义}
        [Definition of Distribution Function]
        [gpt-4.1]
        设 $F(x) = P(X_{i} \leq x)$,则 $F(x)$ 是随机变量 $X_{i}$ 的分布函数.
    \end{dfn}
    
    
    
    \begin{thm}
        [Continuity-and-Differentiability-of-Generalized-Moment-Generating-Function]
        {广义矩母函数的连续性和可微性}
        [Continuity and Differentiability of Generalized Moment Generating Function]
        [gpt-4.1]
        (i) 若 $0 < \theta < \theta_{0} < \theta_{+}$,则
\[
e^{\theta x} \leq 1 + e^{\theta_{0} x}
\]
因此,由控测收敛定理,当 $\theta \to 0$ 时,
\[
\int e^{\theta x} dF \to \int 1 dF = 1
\]

(ii) 若 $|h| < h_{0}$,则
\[
|e^{h x} - 1| = \left| \int_{0}^{h x} e^{y} dy \right| \leq |h x| e^{h_{0} x}
\]
由控测收敛定理可得
\[
\begin{array}{l}
\varphi^{\prime} ( \theta ) = \lim_{h \to 0} \frac{ \varphi ( \theta + h ) - \varphi ( \theta ) }{ h } \\[1em]
\quad = \lim_{h \to 0} \int \frac{ e^{h x} - 1 }{ h } e^{\theta x} dF ( x ) \\[1em]
\quad = \int x e^{\theta x} dF ( x ) \quad \mathrm{for} \ \theta \in ( 0 , \theta_{+} )
\end{array}
\]

    \end{thm}
    
    
    
    \begin{thm}
        [Characterization-of-Multivariate-Normal-Distribution-via-Linear-Combinations]
        {多元正态分布的线性组合判别定理}
        [Characterization of Multivariate Normal Distribution via Linear Combinations]
        [gpt-4.1]
        
$(X_1, \ldots, X_d)$ 具有均值向量 $\theta$ 和协方差矩阵 $\Gamma$ 的多元正态分布,当且仅当对任意线性组合 $c_1 X_1 + \cdots + c_d X_d$,其分布为均值 $c \theta^t$、方差 $c \Gamma c^t$ 的一元正态分布.

    \end{thm}
    
    
    
    \begin{ppt}
        [Property-of-Poisson-Process]
        {泊松过程的性质}
        [Property of Poisson Process]
        [gpt-4.1]
        $V_{n} = J_{1} + \cdots + J_{n}$ 是一个强度为 $\alpha$ 的泊松过程,因此当 $x - y \geq 0$ 时,有 $\psi[0, x - y] = 1 + \alpha(x - y)$.
    \end{ppt}
    
    
    
    \begin{dfn}
        [Definition-of-Symmetric-Simple-Random-Walk-and-Its-Exit-Time]
        {对称简单随机游走及其出界时刻的定义}
        [Definition of Symmetric Simple Random Walk and Its Exit Time]
        [gpt-4.1]
        设 $S_{n}$ 是从 $0$ 开始的对称简单随机游走,$T = \operatorname*{inf}\{n : S_{n} 
otin (-a, a)\}$,其中 $a$ 是一个整数.
    \end{dfn}
    
    
    
    \begin{thm}
        [Limit-Theorem-for-Non-negative-Supermartingales]
        {非负超鞅极限定理}
        [Limit Theorem for Non-negative Supermartingales]
        [gpt-4.1]
        
如果 $X_n \geq 0$ 是一个超鞅,则当 $n \to \infty$ 时,$X_n \to X$ 几乎处处收敛,且 $E X \leq E X_0$.

    \end{thm}
    
    
    
    \begin{dfn}
        [Recursive-Definition-of-GI/G/1-Queue-Model]
        {GI/G/1队列模型的递归定义}
        [Recursive Definition of GI/G/1 Queue Model]
        [gpt-4.1]
        
设 $\xi_{1}, \xi_{2}, \ldots$ 是一列独立同分布(i.i.d.)的随机变量,$W_{n}$ 通过递归关系定义为 $W_{n} = (W_{n-1} + \xi_{n})^{+}$,其中 $x^{+} = \max\{x, 0\}$.

    \end{dfn}
    
    
    
    \begin{thm}
        [Ratio-Limit-Theorem]
        {比值极限定理}
        [Ratio Limit Theorem]
        [gpt-4.1]
        
设 $y$ 是常返点,$m$ 是在 $C_{y} = \{z : \rho_{y z} > 0\}$ 上唯一(仅差常数倍)的平稳测度.令 $N_{n}(z) = |\{m \leq n : X_{m} = z\}|$.将路径在每次回到 $y$ 时分段,则有
\[
\frac{N_{n}(z)}{N_{n}(y)} \to \frac{m(z)}{m(y)}
\]
$P_{x}$-几乎处处成立,适用于所有 $x, z \in C_{y}$.

    \end{thm}
    
    
    
    \begin{ppt}
        [In-One-Dimension-$\varphir-=-1-+-or^2$-Implies-$\varphir-\equiv-1$]
        {一维情况下$\varphi(r) = 1 + o(r^2)$蕴含$\varphi(r) \equiv 1$}
        [In One Dimension, $\varphi(r) = 1 + o(r^2)$ Implies $\varphi(r) \equiv 1$]
        [gpt-4.1]
        在一维情况下,如果 $\varphi(r) = 1 + o(r^2)$,则 $\varphi(r) \equiv 1$.
    \end{ppt}
    
    
    
    \begin{crl}
        [Generalization-of-$\varphir-	heta$-in-Higher-Dimensions-and-Recurrence-of-Three-Dimensional-Random-Walks]
        {高维情况下$\varphi(r 	heta)$的推广与三维随机游走的遍历性}
        [Generalization of $\varphi(r 	heta)$ in Higher Dimensions and Recurrence of Three-Dimensional Random Walks]
        [gpt-4.1]
        通过考虑 $\varphi(r \theta)$,其中 $r$ 为实数,$\theta$ 为固定向量,上述结论可自然推广到 $\mathbf{R}^d$,$d > 1$.这表明,在排除停留于过原点的某一平面上的步行后,三维随机游走不是遍历的.
    \end{crl}
    
    
    
    \begin{dfn}
        [Definition-of-Return-Times-in-Markov-Chains]
        {马尔可夫链返回时刻的定义}
        [Definition of Return Times in Markov Chains]
        [gpt-4.1]
        设 $T_{y}^{0} = 0$,对于 $k \geq 1$,定义
\[
T_{y}^{k} = \operatorname{inf}\{ n > T_{y}^{k-1} : X_{n} = y \}
\]
其中 $T_{y}^{k}$ 表示马尔可夫链 $(X_n)$ 第 $k$ 次返回到状态 $y$ 的时刻.
    \end{dfn}
    
    
    
    \begin{dfn}
        [Definition-of-First-Return-Time-and-Hitting-Probability]
        {首次返回时刻与到达概率的定义}
        [Definition of First Return Time and Hitting Probability]
        [gpt-4.1]
        令 $T_{y} = T_{y}^{1}$,$\rho_{xy} = P_{x}(T_{y} < \infty)$,其中 $T_{y}$ 表示从状态 $x$ 出发首次到达状态 $y$ 的时刻,$\rho_{xy}$ 表示从状态 $x$ 出发最终能够到达状态 $y$ 的概率.
    \end{dfn}
    
    
    
    \begin{dfn}
        [Definition-of-Martingale]
        {鞅的定义}
        [Definition of Martingale]
        [gpt-4.1]
        
定义一个鞅 $X_n$,其中 $X_0 = 0$,且对 $n \geq 1$,有
\[
X_n - X_{n-1} = 1_{B_n} - P(B_n | \mathcal{F}_{n-1})
\]

    \end{dfn}
    
    
    
    \begin{thm}
        [Martingale-Limit-and-Convergence-to-Zero]
        {鞅极限与归零结论}
        [Martingale Limit and Convergence to Zero]
        [gpt-4.1]
        
若 $\{A_\infty < \infty\} \cap \{\sum_m p_m = \infty\}$,则
\[
\frac{X_n}{\sum_{m=1}^n p_m} \to 0
\]

    \end{thm}
    
    
    
    \begin{thm}
        [Martingale-Limit-and-Divergence-Conclusion]
        {鞅极限与发散结论}
        [Martingale Limit and Divergence Conclusion]
        [gpt-4.1]
        
$\{A_\infty = \infty\} = \left\{ \sum_m p_m (1 - p_m) = \infty \right\} \subset \left\{ \sum_m p_m = \infty \right\}$,在 $\{A_\infty = \infty\}$ 上,所需结论可以由定理4.5.3(取 $f(t) = t \vee 1$)推出.

    \end{thm}
    
    
    
    \begin{dfn}
        [Definition-of-Extended-Function]
        {扩展函数的定义}
        [Definition of Extended Function]
        [gpt-4.1]
        
设 $f$ 是定义在集合 $S$ 上的有界可测函数,定义扩展函数 $\bar{f} = 
u f$ 如下:对 $x \in S$,有 $\bar{f}(x) = f(x)$,对 $\alpha$,有 $\bar{f}(\alpha) = \int f\, d\rho$.

    \end{dfn}
    
    
    
    \begin{dfn}
        [Definition-of-Random-Variables-$X-i$-and-Process-$S-n$]
        {关于随机变量 $X\_i$ 和随机过程 $S\_n$ 的定义}
        [Definition of Random Variables $X_i$ and Process $S_n$]
        [gpt-4.1]
        设 $\theta$, $U_{1}$, $U_{2}$, ... 是在区间 $(0, 1)$ 上相互独立且服从均匀分布的随机变量.定义随机变量 $X_{i}$ 如下:
\[
X_{i} =
\begin{cases}
1, & \text{如果 } U_{i} \le \theta \\
-1, & \text{如果 } U_{i} > \theta
\end{cases}
\]
同时定义随机过程 $S_{n} = X_{1} + \cdots + X_{n}$.

    \end{dfn}
    
    
    
    \begin{ppt}
        [$S-n$-is-a-Temporally-Inhomogeneous-Markov-Chain]
        {$S\_n$ 是一个时间不齐次马尔可夫链}
        [$S_n$ is a Temporally Inhomogeneous Markov Chain]
        [gpt-4.1]
        随机过程 $S_n$ 是一个时间不齐次马尔可夫链,即
\[
P(X_{n+1} = 1 \mid X_{1}, \ldots, X_{n}) = P(X_{n+1} = 1 \mid S_n)
\]
这是因为 $S_{n}$ 是估计参数 $\theta$ 的一个充分统计量.

    \end{ppt}
    
    
    
    \begin{dfn}
        [Definition-of-Ladder-Variables]
        {阶梯变量的定义}
        [Definition of Ladder Variables]
        [gpt-4.1]
        设 $\alpha(\omega) = \inf\{ n : \omega_{1} + \cdots + \omega_{n} > 0 \}$,其中 $\inf \varnothing = \infty$,并定义 $\alpha(\Delta) = \infty$.

令 $\alpha_{0} = 0$,并对 $k \geq 1$ 定义
\[
\alpha_{k}(\omega) = \alpha_{k-1}(\omega) + \alpha(\theta^{\alpha_{k-1}} \omega)
\]

    \end{dfn}
    
    
    
    \begin{thm}
        [Theorem-on-Limsup-and-Liminf-of-a-Martingale]
        {关于马氏链极限上界与下界的定理}
        [Theorem on Limsup and Liminf of a Martingale]
        [gpt-4.1]
        
设 $X_{1}, X_{2}, \dots$ 是一个鞅过程,且 $|X_{n+1} - X_{n}| \leq M < \infty$.令
\[
D = \left\{ \limsup X_{n} = +\infty \text{ 且 } \liminf X_{n} = -\infty \right\}
\]
则 $P(C \cup D) = 1$.

    \end{thm}
    
    
    
    \begin{thm}
        [Any-One-sided-Stationary-Sequence-Can-Be-Embedded-in-a-Two-sided-Stationary-Sequence]
        {单边平稳序列可嵌入双边平稳序列}
        [Any One-sided Stationary Sequence Can Be Embedded in a Two-sided Stationary Sequence]
        [gpt-4.1]
        任何单边平稳序列 $\{X_n, n \geq 0\}$ 都可以嵌入到一个双边平稳序列 $\{Y_n : n \in \mathbf{Z}\}$ 中.
    \end{thm}
    
    
    
    \begin{thm}
        [Mean-Square-Error-Decomposition-for-Conditional-Expectation]
        {条件期望的均方误差分解}
        [Mean Square Error Decomposition for Conditional Expectation]
        [gpt-4.1]
        
设 $\mathcal{G} \subset \mathcal{F}$ 且 $E X^2 < \infty$,则有
\[
E ( \{ X - E ( X | \mathcal { F } ) \} ^ { 2 } ) + E ( \{ E ( X | \mathcal { F } ) - E ( X | \mathcal { G } ) \} ^ { 2 } ) = E ( \{ X - E ( X | \mathcal { G } ) \} ^ { 2 } )
\]

    \end{thm}
    
    
    
    \begin{dfn}
        [Definition-of-Absolute-Continuity]
        {绝对连续性的定义}
        [Definition of Absolute Continuity]
        [gpt-4.1]
        设 $
u$ 和 $\mu$ 为测度,若对任意 $A$,$\mu(A) = 0$ 蕴含 $
u(A) = 0$,则称 $
u$ 关于 $\mu$ 绝对连续(记作 $
u \ll \mu$).
    \end{dfn}
    
    
    
    \begin{thm}
        [Radon-Nikodym-Theorem]
        {Radon-Nikodym 定理}
        [Radon-Nikodym Theorem]
        [gpt-4.1]
        设 $\mu$ 和 $
u$ 是定义在 $(\Omega, \mathcal{F})$ 上的 $\sigma$-有限测度.若 $
u \ll \mu$,则存在函数 $f \in \mathcal{F}$,使得对所有 $A \in \mathcal{F}$ 有
\[
\int_A f\, d\mu = 
u(A)
\]
$f$ 通常记作 $d
u/d\mu$,称为 Radon-Nikodym 导数.
    \end{thm}
    
    
    
    \begin{dfn}
        [Definition-of-Invariant-Random-Variable]
        {不变随机变量的定义}
        [Definition of Invariant Random Variable]
        [gpt-4.1]
        一个随机变量 $Z$ 满足 $Z = Z \circ \theta$,因此对所有 $n$ 有 $Z = Z \circ \theta_{n}$,称为不变随机变量.
    \end{dfn}
    
    
    
    \begin{thm}
        [Pollaczek-Khintchine-Formula]
        {Pollaczek-Khintchine公式}
        [Pollaczek-Khintchine Formula]
        [gpt-4.1]
        
\[
E \exp ( - s M ) = \frac { ( 1 - \alpha \cdot E \eta ) s } { s - \alpha + \alpha B ( s ) }
\]

这是著名的 Pollaczek-Khintchine 公式.

    \end{thm}
    
    
    
    \begin{thm}
        [Theorem-on-the-Expression-of-Random-Variable-$M-n$]
        {关于随机变量$M\_n$的表达式的定理}
        [Theorem on the Expression of Random Variable $M_n$]
        [gpt-4.1]
        
设 $B_{n} \in \mathcal{F}_{n}$,则
\[
M_{n} = \sum_{m=1}^{n} \left( 1_{B_{m}} - E(1_{B_{m}} | \mathcal{F}_{m-1}) \right)
\]

    \end{thm}
    
    
    
    \begin{thm}
        [Corollary-of-Theorem-5.8.8]
        {定理5.8.8的推论}
        [Corollary of Theorem 5.8.8]
        [gpt-4.1]
        如果 $(A', B')$ 是另一个满足定义条件的对,那么定理5.8.8蕴含 $P_{\alpha}(\bar{X}_n \in A' \ \mathrm{io}) = 1$,因此递归性或瞬时性不依赖于$(A, B)$的选择.
    \end{thm}
    
    
    
    \begin{thm}
        [Bound-on-the-Convergence-Rate-of-Markov-Chains]
        {马尔可夫链收敛速度的界}
        [Bound on the Convergence Rate of Markov Chains]
        [gpt-4.1]
        
设 $X_n$ 是一个以 $x$ 为初始状态的马尔可夫链,$Y_n$ 是以平稳分布 $\pi$ 为初始分布的马尔可夫链.令 $T$ 为两个链首次相遇的时间,则对于任意 $n$,有

\[
\sum_{y} |p^n(x, y) - \pi(y)| \leq 2 P(T > n)
\]

其中 $p^n(x, y)$ 表示从 $x$ 出发 $n$ 步后到达 $y$ 的概率,$\pi(y)$ 为平稳分布.

    \end{thm}
    
    
    
    \begin{dfn}
        [Transition-Probability-and-Stationary-Measure-of-Random-Walk]
        {随机游走的转移概率与驻留测度}
        [Transition Probability and Stationary Measure of Random Walk]
        [gpt-4.1]
        设 ${\boldsymbol{S}} = \mathbf{Z}^{d}$,定义转移概率为 $p(x, y) = f(y-x)$,其中 $f(z) \geq 0$ 且 $\sum_z f(z) = 1$.在此情况下,$\mu(x) \equiv 1$ 是一个驻留测度,因为
\[
\sum_{x} p(x, y) = \sum_{x} f(y - x) = 1
\]

    \end{dfn}
    
    
    
    \begin{dfn}
        [Definition-of-Doubly-Stochastic-Transition-Probability]
        {双随机转移概率的定义}
        [Definition of Doubly Stochastic Transition Probability]
        [gpt-4.1]
        若转移概率满足 $\sum_{x} p(x, y) = 1$,则称其为双随机(doubly stochastic)转移概率.
    \end{dfn}
    
    
    
    \begin{thm}
        [Necessary-and-Sufficient-Condition-for-Constant-Stationary-Measure]
        {驻留测度为常数的充要条件}
        [Necessary and Sufficient Condition for Constant Stationary Measure]
        [gpt-4.1]
        转移概率满足 $\sum_{x} p(x, y) = 1$ 是驻留测度 $\mu(x) \equiv 1$ 的充要条件.
    \end{thm}
    
    
    
    \begin{xmp}
        [Example-of-Asymmetric-Simple-Random-Walk]
        {非对称简单随机游走的例子}
        [Example of Asymmetric Simple Random Walk]
        [gpt-4.1]
        
设 $S = \mathbf{Z}$,转移概率为
\[
p(x, x+1) = p \qquad p(x, x-1) = q = 1 - p
\]
则 $\mu(x) \equiv 1$ 是一个平稳测度.当 $p 
eq q$ 时,$\mu(x) = (p/q)^{x}$ 也是另一个平稳测度.

    \end{xmp}
    
    
    
    \begin{prf}
        [Proof-of-the-Decomposition-of-Sets-$C-x$]
        {关于集合 $C\_x$ 的分解证明}
        [Proof of the Decomposition of Sets $C_x$]
        [gpt-4.1]
        如果 $x \in R$, 令 $C_x = \{ y : \rho_{xy} > 0 \}$.
由定理 5.3.2, $C_x \subset R$, 并且如果 $y \in C_x$, 那么 $\rho_{yx} > 0$.
由此可以容易得出,要么 $C_x \cap C_y = \varnothing$,要么 $C_x = C_y$.
为了证明最后一条, 假设 $C_x \cap C_y 
eq \emptyset$.
如果 $z \in C_x \cap C_y$, 则 $\rho_{xy} \geq \rho_{xz} \rho_{zy} > 0$,所以如果 $w \in C_y$, 有 $\rho_{xw} \geq \rho_{xy} \rho_{yw} > 0$,于是 $C_x \supset C_y$.
交换 $x$ 和 $y$ 的角色得到 $C_y \supset C_x$,因此证明了我们的断言.
如果我们令 $R_i$ 为所有出现过的 $C_x$ 的集合的列表,则得到了所需的分解.
    \end{prf}
    
    
    
    \begin{dfn}
        [Definition-of-Stopping-Time]
        {停时的定义}
        [Definition of Stopping Time]
        [gpt-4.1]
        $N$ 称为停时 (stopping time),如果对所有 $n$,都有 $\{ N = n \} \in \mathcal{F}_n$.
    \end{dfn}
    
    
    
    \begin{dfn}
        [Definition-of-Sigma-Algebra-at-Stopping-Time]
        {停时上的信息 σ-代数的定义}
        [Definition of Sigma-Algebra at Stopping Time]
        [gpt-4.1]
        设 $N$ 是一个停时,定义

\[
\mathcal{F}_N = \{ A : A \cap \{ N = n \} \in \mathcal{F}_n \text{ for all } n \}
\]

其中 $\mathcal{F}_N$ 表示在时刻 $N$ 已知的信息.
    \end{dfn}
    
    
    
    \begin{dfn}
        [Definition-of-Shift-Transformation-at-Stopping-Time]
        {停时上的移位变换的定义}
        [Definition of Shift Transformation at Stopping Time]
        [gpt-4.1]
        定义停时上的移位变换 $\theta_N$ 如下:

\[
\theta _ { N } \omega = \left\{
\begin{array}{ll}
\theta _ { n } \omega & \text{在 } \{ N = n \} \\
\Delta & \text{在 } \{ N = \infty \}
\end{array}
\right.
\]

其中 $\Delta$ 是添加到 $\Omega_{0}$ 的一个额外点.
    \end{dfn}
    
    
    
    \begin{thm}
        [Central-Limit-Theorem-for-Additive-Functionals]
        {加性泛函的中心极限定理}
        [Central Limit Theorem for Additive Functionals]
        [gpt-4.1]
        
设满足练习 5.6.5 的条件,且 $\sum f(y) \pi(y) = 0$,并且 $E_{x}(V_{k}^{|f|})^{2} < \infty$.则对于任意初始分布 $\mu$,
(i) 有
\[
\frac{1}{\sqrt{n}} \sum_{m=1}^{K_{n}} V_{m}^{f} \Rightarrow c\chi \quad \text{under } P_{\mu}
\]
(ii) 有
\[
\max_{1 \leq m \leq n} V_{m}^{|f|} / \sqrt{n} \to 0
\]
在概率意义下成立,并可推得
\[
\frac{1}{\sqrt{n}} \sum_{m=1}^{n} f(X_{m}) \Rightarrow c\chi \quad \text{under } P_{\mu}
\]

    \end{thm}
    
    
    
    \begin{thm}
        [Uniqueness-of-Irreducible-Recurrent-States-in-Harris-Chains-on-Countable-State-Space]
        {可数状态空间的哈里斯链的不可约递归状态唯一性}
        [Uniqueness of Irreducible Recurrent States in Harris Chains on Countable State Space]
        [gpt-4.1]
        如果 $X_n$ 是可数状态空间上的递归哈里斯链,则 $S$ 只能有一个不可约的递归状态集合,但可以有非空的暂态状态集合.
    \end{thm}
    
    
    
    \begin{thm}
        [Transition-Probability-Property-of-Markov-Chain]
        {马尔可夫链的转移概率性质}
        [Transition Probability Property of Markov Chain]
        [gpt-4.1]
        若 $X_n$ 是关于 $\mathcal{F}_n = \sigma(X_0, X_1, \ldots, X_n)$ 的马尔可夫链,且转移概率为 $p$,则有
\[
P_{\mu}(X_{n+1} \in B \mid \mathcal{F}_n) = p(X_n, B)
\]

    \end{thm}
    
    
    
    \begin{prf}
        [Proof-of-the-Transition-Probability-Property-of-Markov-Chain]
        {马尔可夫链转移概率性质的证明}
        [Proof of the Transition Probability Property of Markov Chain]
        [gpt-4.1]
        令 $A = \{X_0 \in B_0, X_1 \in B_1, \ldots, X_n \in B_n\}$, $B_{n+1} = B$,利用积分的定义、集合 $A$ 的定义以及 $P_\mu$ 的定义,有
\[
\begin{array}{rl}
\int_{A} 1_{(X_{n+1} \in B)} dP_{\mu} &= P_{\mu}(A, X_{n+1} \in B) \\
&= P_{\mu}(X_{0} \in B_{0}, X_{1} \in B_{1}, \ldots, X_{n} \in B_{n}, X_{n+1} \in B) \\
&= \displaystyle \int_{B_{0}} \mu(dx_{0}) \int_{B_{1}} p(x_{0}, dx_{1}) \cdots \int_{B_{n}} p(x_{n-1}, dx_{n}) p(x_{n}, B_{n+1})
\end{array}
\]
我们希望断言最后一个表达式等于
\[
\int_{A} p(X_n, B) dP_\mu
\]
为此,首先注意到
\[
\int_{B_0} \mu(dx_0) \int_{B_1} p(x_0, dx_1) \cdots \int_{B_n} p(x_{n-1}, dx_n) 1_{\mathcal{C}}(x_n) = \int_{A} 1_{\mathcal{C}}(X_n) dP_\mu
\]
线性性说明对于简单函数也成立.

    \end{prf}
    
    
    
    \begin{xmp}
        [Example-of-Limit-Theorem-for-Dice-Rolls]
        {掷骰子的极限定理示例}
        [Example of Limit Theorem for Dice Rolls]
        [gpt-4.1]
        
设 $X_1, X_2, \dots$ 是一组随机变量,每个变量 $X_n$ 的取值为 $e_i$,其中 $i = 1, 2, \ldots, 6$,且 $P(X_n = e_i) = 1/6$.
也就是说,我们在掷骰子,并记录每次出现的点数.

对每个 $i$,有 $E X_{n,i} = 1/6$,且对 $i 
eq j$,有 $E X_{n,i} X_{n,j} = 0$.
因此,协方差矩阵 $\Gamma_{ij}$ 满足:
当 $i = j$ 时,$\Gamma_{ij} = (1/6)(5/6)$;
当 $i 
eq j$ 时,$\Gamma_{ij} = - (1/6)^2$.

在这种情况下,极限分布集中在集合 $\{ x : \sum_i x_i = 0 \}$ 上.

    \end{xmp}
    
    
    
    \begin{thm}
        [Kolmogorovs-Cycle-Condition-Necessary-and-Sufficient]
        {Kolmogorov 循环条件的充要性}
        [Kolmogorov's Cycle Condition: Necessary and Sufficient]
        [gpt-4.1]
        
设 $p$ 是一个不可约的转移概率.存在可逆测度的充要条件是:

(i) 若 $p(x, y) > 0$,则 $p(y, x) > 0$;

(ii) 对于任意环路 $x_0, x_1, \ldots, x_n = x_0$,且 $\prod_{1 \leq i \leq n} p(x_i, x_{i-1}) > 0$,有

\[
\prod_{i=1}^{n} \frac{p(x_{i-1}, x_i)}{p(x_i, x_{i-1})} = 1
\]

    \end{thm}
    
    
    
    \begin{prf}
        [Application-of-Jensens-Inequality-under-Conditional-Expectation]
        {Jensen 不等式在条件期望下的应用}
        [Application of Jensen's Inequality under Conditional Expectation]
        [gpt-4.1]
        
By Jensen's inequality and the assumptions

\[
E ( \varphi ( X_{n+1} ) | \mathcal{F}_n ) \geq \varphi ( E ( X_{n+1} | \mathcal{F}_n ) ) \geq \varphi ( X_n )
\]

    \end{prf}
    
    
    
    \begin{dfn}
        [Definition-of-Filtration]
        {滤过的定义}
        [Definition of Filtration]
        [gpt-4.1]
        
设 $\mathcal{F}_n$, $n \geq 0$ 是一个滤过.

    \end{dfn}
    
    
    
    \begin{dfn}
        [Definition-of-Fixed-Point-ρ]
        {不动点 ρ 的定义}
        [Definition of Fixed Point ρ]
        [gpt-4.1]
        设 $\rho$ 是区间 $[0, 1)$ 上满足 $\phi(\rho) = \rho$ 的最小解.
    \end{dfn}
    
    
    
    \begin{thm}
        [Ballot-Theorem]
        {投票定理}
        [Ballot Theorem]
        [gpt-4.1]
        假设在一次选举中,候选人 $A$ 获得 $\alpha$ 票,候选人 $B$ 获得 $\beta$ 票,其中 $\beta < \alpha$.在计票过程中,$A$ 始终领先于 $B$ 的概率是 $(\alpha - \beta) / (\alpha + \beta)$.
    \end{thm}
    
    
    
    \begin{prf}
        [Proof-of-the-Ballot-Theorem]
        {投票定理的证明}
        [Proof of the Ballot Theorem]
        [gpt-4.1]
        设 $x = \alpha - \beta$, $n = \alpha + \beta$.显然,满足条件的结果数等价于从 $(1,1)$ 到 $(n, x)$ 且途中值始终不为 0 的路径数.反射原理说明,从 $(1,1)$ 到 $(n, x)$ 途中某时为 0 的路径数等于从 $(1,-1)$ 到 $(n, x)$ 的路径数,因此根据 (4.9.1),从 $(1,1)$ 到 $(n, x)$ 且途中值始终不为 0 的路径数为

\[
\begin{array}{rl}
& N_{n-1, x-1} - N_{n-1, x+1} = \binom{n-1}{\alpha-1} - \binom{n-1}{\alpha} \\
& \qquad = \frac{(n-1)!}{(\alpha-1)! (n-\alpha)!} - \frac{(n-1)!}{\alpha! (n-\alpha-1)!} \\
& \qquad = \frac{\alpha - (n-\alpha)}{n} \cdot \frac{n!}{\alpha! (n-\alpha)!} = \frac{\alpha - \beta}{\alpha + \beta} N_{n, x}
\end{array}
\]

由于 $n = \alpha + \beta$,这就证明了所需的结论.
    \end{prf}
    
    
    
    \begin{cxmp}
        [Counterexample-of-Two-State-Markov-Chain]
        {二状态马尔可夫链的反例}
        [Counterexample of Two-State Markov Chain]
        [gpt-4.1]
        考虑状态空间 $S = \{0, 1\}$ 的链,其转移概率为 $p(x, \{1 - x\}) = 1$.在这种情况下,平稳分布为 $\pi(0) = \pi(1) = 1/2$,且 $(X_{0}, X_{1}, \ldots) = (0, 1, 0, 1, \ldots)$ 或 $(1, 0, 1, 0, \ldots)$,每种情形出现的概率为 $1/2$.
    \end{cxmp}
    
    
    
    \begin{lma}
        [Lemma-on-Lower-Bound-of-Cosine-Function]
        {余弦函数的下界引理}
        [Lemma on Lower Bound of Cosine Function]
        [gpt-4.1]
        若 $|x| \le \pi/3$, 则 $1 - \cos x \geq x^2 / 4$.
    \end{lma}
    
    
    
    \begin{prf}
        [Proof-of-Lemma-on-Lower-Bound-of-Cosine-Function]
        {余弦函数的下界引理的证明}
        [Proof of Lemma on Lower Bound of Cosine Function]
        [gpt-4.1]
        只需证明 $x > 0$ 的情形.若 $z \le \pi / 3$,则 $\cos z \ge 1/2$,
\[
\sin y = \int_0^y \cos z \, dz \geq \frac{y}{2}
\]

\[
1 - \cos x = \int_0^x \sin y \, dy \geq \int_0^x \frac{y}{2} \, dy = \frac{x^2}{4}
\]
由此得到所需结果.
    \end{prf}
    
    
    
    \begin{dfn}
        [Definition-of-Sums-of-Large-and-Small-Terms]
        {大项和与小项和的定义}
        [Definition of Sums of Large and Small Terms]
        [gpt-4.1]
        令 $\epsilon > 0$,定义
\[
\begin{array}{l}
\displaystyle I_n ( \epsilon ) = \{ m \leq n : | X_m | > \epsilon n^{1/\alpha} \} \\
\displaystyle \hat{S}_n ( \epsilon ) = \sum_{m \in I_n ( \epsilon )} X_m \qquad \bar{S}_n ( \epsilon ) = S_n - \hat{S}_n ( \epsilon )
\end{array}
\]
其中,$I_n ( \epsilon )$ 为'较大项'的指标集合,即满足 $|X_m| > \epsilon n^{1/\alpha}$ 的所有 $m$,$\hat{S}_n ( \epsilon )$ 为所有较大项的和,$\bar{S}_n ( \epsilon )$ 为其余项的和.

    \end{dfn}
    
    
    
    \begin{dfn}
        [Definition-of-Truncated-Random-Variable]
        {截断后的随机变量的定义}
        [Definition of Truncated Random Variable]
        [gpt-4.1]
        定义
\[
\bar{X}_m ( \epsilon ) = X_m 1_{ ( | X_m | \leq \epsilon n^{1/\alpha} ) }
\]
即 $\bar{X}_m ( \epsilon )$ 为当 $| X_m | \leq \epsilon n^{1/\alpha}$ 时取 $X_m$,否则取 $0$ 的随机变量.

    \end{dfn}
    
    
    
    \begin{xmp}
        [Example-of-the-Holtsmark-Distribution]
        {Holtsmark分布的例子}
        [Example of the Holtsmark Distribution]
        [gpt-4.1]
        设恒星在空间中依照参数为 $t$ 的 Poisson 过程分布,且恒星的质量独立同分布.记 $X_t$ 为密度为 $t$ 时在原点的引力 $x$ 分量.

密度从 $1$ 变为 $t$ 对应长度变化 $1 \to t^{-1/3}$,引力服从平方反比定律,所以
\[
X_{t} \overset{d}{=} t^{3/2} X_{1}
\]

如果将 Poisson 过程稀疏化,相当于掷 $n$ 面骰子,则由定理 3.7.4 得
\[
X_{t} \overset{d}{=} X_{t/n}^{1} + \cdots + X_{t/n}^{n}
\]
其中右侧的随机变量相互独立且与 $X_{t/n}$ 同分布.

又由定理 3.8.8 可知 $X_t$ 服从稳定分布.由缩放性质 (3.8.14) 可得 $\alpha = 3/2$.由于 $X_t \overset{d}{=} -X_t$,所以 $\kappa = 0$.

    \end{xmp}
    
    
    
    \begin{ppt}
        [Mean-Value-Inequality-for-Superharmonic-Functions]
        {超调和函数的平均值不等式}
        [Mean Value Inequality for Superharmonic Functions]
        [gpt-4.1]
        
如果 $f$ 是超调和函数(即 $f$ 具有连续的 $\leq 2$ 阶导数且 $\partial ^{2} f / \partial x_{1}^{2} + \cdots + \partial ^{2} f / \partial x_{d}^{2} \leq 0$),则对任意 $x$ 和 $r$,有
\[
f(x) \geq \frac{1}{|B(x, r)|} \int_{B(x, r)} f(y) d y
\]
其中 $B(x, r) = \{ y : |x - y| \leq r \}$ 是以 $x$ 为中心、半径为 $r$ 的球, $|B(x, r)|$ 是该球的体积.

    \end{ppt}
    
    
    
    \begin{prf}
        [Proof-of-Monotonicity-and-Upper-Bound-of-the-Limit]
        {极限递增性与上界的证明}
        [Proof of Monotonicity and Upper Bound of the Limit]
        [gpt-4.1]
        证明 (b): 显然 $\theta_m = P(Z_m = 0)$ 是递增的.为了用归纳法证明 $\theta_m \leq \rho$,注意到 $\theta_0 = 0 \leq \rho$,并且若对 $m-1$ 成立,则
\[
\theta_m = \varphi(\theta_{m-1}) \leq \varphi(\rho) = \rho.
\]

    \end{prf}
    
    
    
    \begin{dfn}
        [Definition-of-Simple-Random-Walk-and-Stopping-Time]
        {简单随机游走及停时定义}
        [Definition of Simple Random Walk and Stopping Time]
        [gpt-4.1]
        
记 $S_n$ 为从 $S_0 = 1$ 出发的简单随机游走,$N = \inf\{ n : S_n = 0 \}$,$X_n = S_{N \wedge n}$.

    \end{dfn}
    
    
    
    \begin{thm}
        [Hitting-Probability-and-Expected-Maximum-of-Simple-Random-Walk]
        {简单随机游走的 hitting 概率及最大值期望}
        [Hitting Probability and Expected Maximum of Simple Random Walk]
        [gpt-4.1]
        
对于从 $S_0 = 1$ 出发的简单随机游走 $S_n$ 及 $X_n = S_{N \wedge n}$,有
\[
P \left( \max_m X_m \geq M \right) = \frac{1}{M}
\]
因此
\[
E \left( \max_m X_m \right) = \sum_{M=1}^{\infty} P \left( \max_m X_m \geq M \right) = \sum_{M=1}^{\infty} \frac{1}{M} = \infty.
\]
单调收敛定理可推出 $E \left( \max_{m \leq n} X_m \right) \uparrow \infty$ 当 $n \uparrow \infty$.

    \end{thm}
    
    
    
    \begin{dfn}
        [Definition-of-Even-Sum-and-Odd-Sum-Subspaces]
        {偶数和与奇数和子空间的定义}
        [Definition of Even-Sum and Odd-Sum Subspaces]
        [gpt-4.1]
        令 $L_{0} = \{ z \in \mathbf{Z}^{d} : z^{1} + \cdots + z^{d} \text{ 为偶数} \}$ ,$L_{1} = \mathbf{Z}^{d} - L_{0}$.
    \end{dfn}
    
    
    
    \begin{dfn}
        [Definition-of-Stationary-Sequence]
        {平稳序列的定义}
        [Definition of Stationary Sequence]
        [gpt-4.1]
        
设 $\xi_{1}, \xi_{2}, \ldots$ 是一个平稳序列,且 $E | \xi_{k} | < \infty$,令 $X_{m, n} = \xi_{m+1} + \cdots + \xi_{n}$.

    \end{dfn}
    
    
    
    \begin{dfn}
        [Definition-of-$\psi-nA$-in-Random-Walks]
        {随机游走中$\psi\_n(A)$的定义}
        [Definition of $\psi_n(A)$ in Random Walks]
        [gpt-4.1]
        $\psi_{n}(A)$ 是随机游走在第 $n$ 步时达到集合 $A$ 中新的最大值(或梯级高度,见练习 5.4.2)的概率,因此 $\psi(A)$ 是 $A$ 中梯级点的个数,且有 $\psi(\{0\}) = 1$.
    \end{dfn}
    
    
    
    \begin{xmp}
        [An-Example-Where-the-Square-of-a-Submartingale-Is-a-Supermartingale]
        {关于子鞅的平方是超鞅的一个例子}
        [An Example Where the Square of a Submartingale Is a Supermartingale]
        [gpt-4.1]
        给出一个子鞅 $X_n$,使得 $X_n^2$ 是超鞅.例如,令 $X_n = n$,则 $X_n$ 是确定性的递增过程(满足子鞅条件),但 $X_n^2$ 递增且不是超鞅.可以考虑 $X_n$ 为随机变量的情形,使其满足 $X_n$ 为子鞅且 $X_n^2$ 为超鞅.
    \end{xmp}
    
    
    
    \begin{thm}
        [Theorem-on-Normalization-of-Sums-under-Heavy-Tailed-Distributions]
        {重尾分布下部分和归一化的定理}
        [Theorem on Normalization of Sums under Heavy-Tailed Distributions]
        [gpt-4.1]
        设 $X_{1}, X_{2}, \dots$ 是独立同分布的随机变量, 其分布满足:

(i) $\operatorname{lim}_{x \to \infty} \frac{P(X_{1} > x)}{P(|X_{1}| > x)} = \theta \in [0,1]$ 

(ii) $P(|X_{1}| > x) = x^{-\alpha} L(x)$, 其中 $\alpha < 2$, $L(x)$ 是慢变函数.

令 $S_{n} = X_{1} + \cdots + X_{n}$,

\[
a_{n} = \operatorname{inf}\{x : P(|X_{1}| > x) \leq n^{-1}\}, \quad b_{n} = n E(X_{1} 1_{(|X_{1}| \leq a_{n})})
\]

    \end{thm}
    
    
    
    \begin{dfn}
        [Definition-of-Normalization-Parameters-for-Sums-of-Heavy-Tailed-Random-Variables]
        {重尾分布部分和的归一化参数的定义}
        [Definition of Normalization Parameters for Sums of Heavy-Tailed Random Variables]
        [gpt-4.1]
        对于独立同分布的随机变量 $X_{1}, X_{2}, \dots$,定义

\[
a_{n} = \operatorname{inf}\{x : P(|X_{1}| > x) \leq n^{-1}\}
\]

和

\[
b_{n} = n E(X_{1} 1_{(|X_{1}| \leq a_{n})})
\]

    \end{dfn}
    
    
    
    \begin{lma}
        [Transformation-of-Inhomogeneous-Markov-Chain]
        {关于非齐次马尔可夫链的转换}
        [Transformation of Inhomogeneous Markov Chain]
        [gpt-4.1]
        设 $Y_n$ 是一个非齐次马尔可夫链,满足 $p_{2k} = 
u$ 且 $p_{2k+1} = \bar{p}$.则 $\bar{X}_n = Y_{2n}$ 是一个具有转移概率 $\bar{p}$ 的马尔可夫链,且 $X_n = Y_{2n+1}$ 是一个具有转移概率 $p$ 的马尔可夫链.
    \end{lma}
    
    
    
    \begin{lma}
        [Application-of-Lemma-5.4.12-on-Integral-Inequality]
        {关于积分不等式的引理 5.4.12 的应用}
        [Application of Lemma 5.4.12 on Integral Inequality]
        [gpt-4.1]
        
设 $\mu_n$ 是概率测度,$\varphi^n(t)$ 是其特征函数的幂,$S_n$ 是相关随机变量的和,则有如下不等式:

\[
P(\|S_n\| < 1/\delta) \geq \int_{(-1/\delta, 1/\delta)^d} \prod_{i=1}^d (1 - |\delta x_i|) \mu_n(dx) 
= \int \prod_{i=1}^d \frac{1 - \cos(t_i / \delta)}{\pi t_i^2 / \delta} \varphi^n(t) \, dt
\]

    \end{lma}
    
    
    
    \begin{lma}
        [Application-of-Lemma-5.4.13-on-Lower-Bound-of-Integral]
        {关于积分下界的引理 5.4.13 的应用}
        [Application of Lemma 5.4.13 on Lower Bound of Integral]
        [gpt-4.1]
        
对任意 $r \in (0,1)$,有

\[
\sum_{n=0}^{\infty} r^n P(\|S_n\| < 1/\delta) \ge (4\pi\delta)^{-d} \int_{(-\delta, \delta)^d} \operatorname{Re} \frac{1}{1 - r \varphi(t)} \, dt
\]

    \end{lma}
    
    
    
    \begin{thm}
        [Conditional-Expectation-Properties-of-Submartingales-and-Martingales]
        {关于子鞅与鞅的条件期望性质}
        [Conditional Expectation Properties of Submartingales and Martingales]
        [gpt-4.1]
        
(i) 如果 $X_n$ 是子鞅,则对于 $n > m$,有 $E ( X_n | \mathcal{F}_m ) \geq X_m$.
(ii) 如果 $X_n$ 是鞅,则对于 $n > m$,有 $E ( X_n | \mathcal{F}_m ) = X_m$.

    \end{thm}
    
    
    
    \begin{dfn}
        [Recursive-Definition-of-$\mux$]
        {关于$\mu(x)$的递归定义}
        [Recursive Definition of $\mu(x)$]
        [gpt-4.1]
        
固定 $a \in S$,令 $\mu(a) = 1$.若 $x_{0} = a, x_{1}, \ldots, x_{n} = x$ 是一条满足
\[
\prod_{1 \leq i \leq n} p(x_{i}, x_{i-1}) > 0
\]
的路径(不可约性保证这样的路径存在),则定义
\[
\mu(x) = \prod_{i=1}^{n} \frac{p(x_{i-1}, x_{i})}{p(x_{i}, x_{i-1})}
\]
其中,循环条件保证该定义与路径的选择无关.

    \end{dfn}
    
    
    
    \begin{xmp}
        [Example-of-Integral-Calculation-with-Limits-and-Distributions]
        {关于极限与分布的积分计算例子}
        [Example of Integral Calculation with Limits and Distributions]
        [gpt-4.1]
        
令 $n \to \infty$,注意到 $(S_{n} \mid T = n)$ 服从指数分布且 $S_{n} \to -\infty$ 在 $\{T = \infty\}$ 上,我们有
\[
1 = r \int_{0}^{\infty} e^{\theta x} \beta e^{-\beta x} dx = \frac{r \beta}{\beta - \theta}
\]

    \end{xmp}
    
    
    
    \begin{lma}
        [Lemma-of-Parseval-Relation]
        {Parseval 关系引理}
        [Lemma of Parseval Relation]
        [gpt-4.1]
        
设 $\mu$ 和 $
u$ 是 $\mathbf{R}^d$ 上的概率测度,$\varphi$ 和 $\psi$ 分别是它们的特征函数,则有
\[
\int \psi(t) \, \mu(dt) = \int \varphi(x) \, 
u(dx)
\]

    \end{lma}
    
    
    
    \begin{prf}
        [Proof-of-Lemma-of-Parseval-Relation]
        {Parseval 关系引理的证明}
        [Proof of Lemma of Parseval Relation]
        [gpt-4.1]
        
由于 $e^{it \cdot x}$ 有界,Fubini 定理推出
\[
\int \psi(t) \, \mu(dt) = \int \int e^{itx} \, 
u(dx) \, \mu(dt) = \int \int e^{itx} \, \mu(dt) \, 
u(dx) = \int \varphi(x) \, 
u(dx)
\]

    \end{prf}
    
    
    
    \begin{dfn}
        [Definition-of-Recurrence-and-Transience-for-Markov-Chains]
        {马尔可夫链的递归性和暂态性的定义}
        [Definition of Recurrence and Transience for Markov Chains]
        [gpt-4.1]
        
设 $R = \operatorname*{inf}\{ n \geq 1 : \bar{X}_n = \alpha \}$.若 $P_\alpha(R < \infty) = 1$,则称该链是递归的;否则称该链是暂态的.

    \end{dfn}
    
    
    
    \begin{thm}
        [Transience-of-Simple-Random-Walk-in-High-Dimensions]
        {高维简单随机游走的遍历性}
        [Transience of Simple Random Walk in High Dimensions]
        [gpt-4.1]
        当 $d > 3$ 时,简单随机游走是遍历的(transient).
    \end{thm}
    
    
    
    \begin{dfn}
        [Definition-of-Subsequence-and-Stopping-Times-for-3D-Random-Walk]
        {三维随机游走的嵌入子序列和相关停时}
        [Definition of Subsequence and Stopping Times for 3D Random Walk]
        [gpt-4.1]
        设 $T_{n} = (S_{n}^{1}, S_{n}^{2}, S_{n}^{3})$,定义 $N(0) = 0$,并令
\[N(n) = \inf \{ m > N(n-1) : T_{m} 
eq T_{N(n-1)} \}\]
即 $N(n)$ 是第一次使 $T_{m}$ 不等于 $T_{N(n-1)}$ 的时间点,其中 $m > N(n-1)$.
    \end{dfn}
    
    
    
    \begin{thm}
        [No-Exceptional-Set-When-$A-=-[a-b$]
        {区间 $A = [a, b)$ 时无特异集}
        [No Exceptional Set When $A = [a, b)$]
        [gpt-4.1]
        
若 $A = [ a , b )$,则特异集为 $\varnothing$.

    \end{thm}
    
    
    
    \begin{thm}
        [Limit-Property-of-Conditional-Expectation-for-Exchangeable-Sequences]
        {交换序列条件期望的极限性质}
        [Limit Property of Conditional Expectation for Exchangeable Sequences]
        [gpt-4.1]
        
对于任意交换序列,成立
\[
A_n(\varphi) \to E(\varphi(X_1, \dots, X_k) | \mathcal{E})
\]
其中 $\mathcal{E}$ 可以是非平凡的,因此不能保证极限为 $E(\varphi(X_1, \ldots, X_k))$.

    \end{thm}
    
    
    
    \begin{thm}
        [Factorization-of-Conditional-Expectation-of-Product-Functions-for-Exchangeable-Sequences]
        {交换序列乘积函数条件期望的因式分解}
        [Factorization of Conditional Expectation of Product Functions for Exchangeable Sequences]
        [gpt-4.1]
        
对于交换序列 $X_1, X_2, \ldots$,以及有界函数 $f$ 和 $g$,
\[
E(f(X_1, \dots, X_{k-1}) g(X_k) | \mathcal{E}) = E(f(X_1, \dots, X_{k-1}) | \mathcal{E}) E(g(X_k) | \mathcal{E})
\]
并且归纳地,
\[
E\left( \prod_{j=1}^k f_j(X_j) \Bigg| \mathcal{E} \right) = \prod_{j=1}^k E(f_j(X_j) | \mathcal{E})
\]

    \end{thm}
    
    
    
    \begin{lma}
        [Lemma-on-Equivalence-of-Expectations]
        {期望的等价性引理}
        [Lemma on Equivalence of Expectations]
        [gpt-4.1]
        如果 $\boldsymbol{\mu}$ 是 $(S, S)$ 上的概率测度, 则
\[
E_\mu f(X_n) = E_\mu \bar{f}(\bar{X}_n)
\]

    \end{lma}
    
    
    
    \begin{prf}
        [Proof-of-Equivalence-of-Expectations]
        {期望等价性的证明}
        [Proof of Equivalence of Expectations]
        [gpt-4.1]
        注意, 如果 $X_n$ 和 $\bar{X}_n$ 按照 Lemma 5.8.5 的方式构造,并且 $P(\bar{X}_0 \in S) = 1$, 则 $X_0 = \bar{X}_0$, 且 $X_n$ 是通过按照 $
u$ 从 $\bar{X}_n$ 进行转移获得的.
    \end{prf}
    
    
    
    \begin{thm}
        [$L^p$-Maximum-Inequality]
        {$L^p$最大不等式}
        [$L^p$ Maximum Inequality]
        [gpt-4.1]
        若$X_n$是次鞅(submartingale),则对$1 < p < \infty$有
\[
E(\bar{X}_n^p) \leq \left(\frac{p}{p-1}\right)^p E(X_n^{+})^p
\]

因此,若$Y_n$是鞅(martingale),且$Y_n^* = \max_{0 \leq m \leq n} |Y_m|$,则
\[
E|Y_n^*|^p \leq \left(\frac{p}{p-1}\right)^p E(|Y_n|^p)
\]

    \end{thm}
    
    
    
    \begin{prf}
        [Proof-of-$L^p$-Maximum-Inequality]
        {$L^p$最大不等式的证明}
        [Proof of $L^p$ Maximum Inequality]
        [gpt-4.1]
        第二个不等式可通过将第一个不等式应用于$X_n = |Y_n|$得到.
    \end{prf}
    
    
    
    \begin{crl}
        [Distribution-of-the-Sum-of-Independent-Normal-Variables]
        {独立正态分布变量之和的分布}
        [Distribution of the Sum of Independent Normal Variables]
        [gpt-4.1]
        若 $X_{i}$,$i=1,2$ 是独立且分别服从均值 $\boldsymbol{\theta}$、方差 $\sigma_{i}^2$ 的正态分布,则 $X_1 + X_2$ 服从均值 $0$、方差 $\sigma_1^2 + \sigma_2^2$ 的正态分布.
    \end{crl}
    
    
    
    \begin{thm}
        [Statement-of-the-Strong-Law-of-Large-Numbers]
        {强大数定律的表述}
        [Statement of the Strong Law of Large Numbers]
        [gpt-4.1]
        
由于 $\mathcal{T}$ 是平凡的,遍历定理推出
\[
\frac{1}{n} \sum_{m=0}^{n-1} X_m \to E X_0 \quad \mathrm{a.s.}~\mathrm{and~in~} L^1
\]
其中 a.s. 收敛即为强大数定律.

    \end{thm}
    
    
    
    \begin{thm}
        [Supermartingale-Property-under-Bounded-Predictable-Multipliers]
        {可预测有界乘子的超鞅保持性}
        [Supermartingale Property under Bounded Predictable Multipliers]
        [gpt-4.1]
        设 $X_n$, $n \geq 0$, 是一个超鞅(supermartingale).如果 $H_n \geq 0$ 可预测且每个 $H_n$ 有界,则 $(H \cdot X)_n$ 也是一个超鞅.
    \end{thm}
    
    
    
    \begin{prf}
        [Proof-that-Conditional-Expectation-Equals-a-Measurable-Function]
        {条件期望等于可测函数的证明}
        [Proof that Conditional Expectation Equals a Measurable Function]
        [gpt-4.1]
        
证明如下:

已知 $E ( \varphi ( X , Y ) | X ) = g ( X )$,其中 $g(X)$ 是关于 $X$ 的可测函数,即 $g ( X ) \in \sigma ( X )$.

对任意 $A \in \sigma ( X )$,有 $A = \{ X \in C \}$.利用变量变换公式(定理 1.6.9)和 $(X, Y)$ 的分布为乘积测度(定理 2.1.11),根据 $g$ 的定义,再次使用变量变换公式,

\[
\begin{array}{l}
\displaystyle \int_{A} \varphi(X, Y) dP = E\{ \varphi(X, Y) 1_{C}(X) \} \\
\displaystyle \qquad = \int \int \varphi(x, y) 1_{C}(x) 
u(dy) \mu(dx) \\
\displaystyle \qquad = \int 1_{C}(x) g(x) \mu(dx) = \int_{A} g(X) dP
\end{array}
\]

因此,证明了条件期望等于关于 $X$ 的可测函数 $g(X)$.

    \end{prf}
    
    
    
    \begin{dfn}
        [Definition-of-Ergodicity]
        {遍历性的定义}
        [Definition of Ergodicity]
        [gpt-4.1]
        在 $(\Omega, \mathcal{F}, P)$ 上的测度保持变换如果 $\mathcal{T}$ 是平凡的,则称其为遍历的.即,对每个 $A \in \mathcal{T}$,都有 $P(A) \in \{0, 1\}$.
    \end{dfn}
    
    
    
    \begin{dfn}
        [Definition-of-Superharmonic-Function]
        {超调和函数的定义}
        [Definition of Superharmonic Function]
        [gpt-4.1]
        设 $f$ 是一个函数,若对任意 $x$ 有 $f(x) \geq \sum_{y} p(x, y) f(y)$,或者等价地,$f(X_n)$ 是一个超鞅(supermartingale),则称 $f$ 是超调和的(superharmonic).
    \end{dfn}
    
    
    
    \begin{thm}
        [Criterion-for-Constant-Superharmonic-Functions-in-Irreducible-Markov-Chains]
        {不可约马尔可夫链的常值超调和函数判据}
        [Criterion for Constant Superharmonic Functions in Irreducible Markov Chains]
        [gpt-4.1]
        设 $p$ 是不可约的转移概率,则 $p$ 是常返的当且仅当所有非负的超调和函数都是常数函数.
    \end{thm}
    
    
    
    \begin{thm}
        [Theorem-on-Recurrence-and-Transition-Probabilities]
        {关于递归性和转移概率的定理}
        [Theorem on Recurrence and Transition Probabilities]
        [gpt-4.1]
        如果 $x$ 是递归的且 $\rho_{xy} > 0$,则 $y$ 是递归的且 $\rho_{yx} = 1$.
    \end{thm}
    
    
    
    \begin{prf}
        [Proof-of-Theorem-on-Recurrence-and-Transition-Probabilities]
        {关于递归性和转移概率的定理的证明}
        [Proof of Theorem on Recurrence and Transition Probabilities]
        [gpt-4.1]
        我们首先通过证明如果 $\rho_{xy} > 0$ 且 $\rho_{yx} < 1$,则 $\rho_{xx} < 1$ 来证明 $\rho_{yx} = 1$.

令 $K = \operatorname{inf} \{ k : p^{k}(x, y) > 0 \}$.存在一列 $y_1, \ldots, y_{K-1}$,使得
\[
p(x, y_1) p(y_1, y_2) \cdots p(y_{K-1}, y) > 0
\]
由于 $K$ 是最小的,$y_i 
eq x$ 对于 $1 \leq i \leq K-1$.

若 $\rho_{yx} < 1$,则有
\[
P_x(T_x = \infty) \geq p(x, y_1) p(y_1, y_2) \cdots p(y_{K-1}, y) (1 - \rho_{yx}) > 0
\]
矛盾.

所以 $\rho_{yx} = 1$.

为了证明 $y$ 是递归的,注意到 $\rho_{yx} > 0$ 蕴含存在 $L$ 使得 $p^{L}(y, x) > 0$.

现在有:
\[
p^{L+n+K}(y, y) \geq p^{L}(y, x) p^{n}(x, x) p^{K}(x, y)
\]

对 $n$ 求和,得到
\[
\sum_{n=1}^{\infty} p^{L+n+K}(y, y) \geq p^{L}(y, x) p^{K}(x, y) \sum_{n=1}^{\infty} p^{n}(x, x) = \infty
\]
因此由定理 5.3.1 可知 $y$ 是递归的.
    \end{prf}
    
    
    
    \begin{prf}
        [Proof-of-Jensens-Inequality-in-Submartingale-Property]
        {Jensen不等式在鞅性质中的应用证明}
        [Proof of Jensen's Inequality in Submartingale Property]
        [gpt-4.1]
        由Jensen不等式和定义可得:

\[
E ( \varphi ( X_{n+1} ) | \mathcal{F}_n ) \geq \varphi ( E ( X_{n+1} | \mathcal{F}_n ) ) = \varphi ( X_n )
\]

    \end{prf}
    
    
    
    \begin{thm}
        [Submartingale-Property-under-Convex-Function]
        {凸函数作用下的亚鞅仍为亚鞅}
        [Submartingale Property under Convex Function]
        [gpt-4.1]
        设 $X_n$ 是关于 $\mathcal{F}_n$ 的亚鞅,$\varphi$ 是递增凸函数,且对所有 $n$ 有 $E | \varphi ( X_n ) | < \infty$,则 $\varphi ( X_n )$ 关于 $\mathcal{F}_n$ 仍为亚鞅.

    \end{thm}
    
    
    
    \begin{xmp}
        [Example-on-Probability-of-Fixed-Points-in-Permutations]
        {关于排列固定点计数的概率例子}
        [Example on Probability of Fixed Points in Permutations]
        [gpt-4.1]
        
通过考虑排列的固定点位置,我们有
\[
\begin{array}{rl}
& P(S_n = k) = \binom{n}{k} \frac{1}{n(n-1)\cdots(n-k+1)} P(S_{n-k} = 0) \\
& \qquad = \frac{1}{k!} P(S_{n-k} = 0) \to e^{-1}/k!
\end{array}
\]
这是对于排列中有 $k$ 个固定点的概率的计算.

    \end{xmp}
    
    
    
    \begin{thm}
        [Limit-of-Visit-Frequency-for-Recurrent-States]
        {关于可复状态的访问频率极限}
        [Limit of Visit Frequency for Recurrent States]
        [gpt-4.1]
        
设 $y$ 是可复状态.对于任意 $x \in S$,当 $n \to \infty$ 时,有
\[
\frac{N_n(y)}{n} \to \frac{1}{E_y T_y} 1_{\{T_y < \infty\}} \quad P_x\text{-a.s.}
\]
其中 $N_n(y)$ 表示在时间 $n$ 前对 $y$ 的访问次数,$E_y T_y$ 表示从 $y$ 出发返回 $y$ 的期望时间,$1/\infty = 0$.

    \end{thm}
    
    
    
    \begin{thm}
        [Criterion-for-Recurrence-of-${-\bar-{-X-}-}---{-n-}$]
        {关于${ \bar { X } } _ { n }$遍历性的判定定理}
        [Criterion for Recurrence of ${ \bar { X } } _ { n }$]
        [gpt-4.1]
        ${ \bar { X } } _ { n }$ 是 recurrent 的充分必要条件是 $\sum _ { n = 1 } ^ { \infty } \bar { p } ^ { n } ( \alpha , \alpha ) = \infty$.
    \end{thm}
    
    
    
    \begin{xmp}
        [Example-on-Supremum-and-Infimum-of-Limits]
        {关于极限上确界和下确界的例子}
        [Example on Supremum and Infimum of Limits]
        [gpt-4.1]
        
From this we see that

\[
\sup_{m \geq 1} \frac{E \log \alpha_{0,m}}{m} = -X = \inf_{m \geq 1} \frac{E \log \beta_{0,m}}{m}
\]

Example 6.

    \end{xmp}
    
    
    
    \begin{thm}
        [Probability-Equality-for-Ergodic-Sequence-and-Expectation]
        {关于遍历性和期望的概率等式}
        [Probability Equality for Ergodic Sequence and Expectation]
        [gpt-4.1]
        若在 Theorem 6.3.2 的假设基础上,进一步假设 $P(X_{i} > 1) = 0$,$E X_{i} > 0$,且序列 $X_{i}$ 是遍历的,则有 $P(A) = E X_{i}$.
    \end{thm}
    
    
    
    \begin{prf}
        [Proof-of-Limit-of-Measure]
        {关于测度极限的证明}
        [Proof of Limit of Measure]
        [gpt-4.1]
        设 $A _ { k } = [ a + 1 / k , b - 1 / k )$.如果 $b - a > 2 / k$,由遍历定理可知

\[
\frac { 1 } { n } \sum _ { m = 0 } ^ { n - 1 } 1 _ { A _ { k } } ( \varphi ^ { m } \omega ) \to b - a - \frac { 2 } { k }
\]

    \end{prf}
    
    
    
    \begin{thm}
        [Statement-of-the-Limiting-Value]
        {极限值的表述}
        [Statement of the Limiting Value]
        [gpt-4.1]
        
结合前述两个观察以及 $c(a)$ 严格递增的事实,可以得到
\[
\gamma = \operatorname*{inf} \{ a : \log \mu - c(a) > 0 \}
\]
该结果来自 Biggins (1977).

    \end{thm}
    
    
    
    \begin{dfn}
        [Definition-of-Expectation-Function-and-Variable]
        {期望函数与变量的定义}
        [Definition of Expectation Function and Variable]
        [gpt-4.1]
        设 $\varphi(\theta) = E \exp(-\theta t_{i})$, 并定义
\[
Y_{n} = (\mu \varphi(\theta))^{-n} \sum_{i = 1}^{Z_{n}} \exp(-\theta T_{n}(i))
\]

    \end{dfn}
    
    
    
    \begin{dfn}
        [Definition-of-Filtration-$\mathcal{F}-n$]
        {关于滤波族 $\mathcal{F}_n$ 的定义}
        [Definition of Filtration $\mathcal{F}_n$]
        [gpt-4.1]
        设 $\mathcal{F}_n = \sigma(\xi_i^m : i \ge 1, 1 \le m \le n)$,即由满足 $i \ge 1$ 且 $1 \le m \le n$ 的所有 $\xi_i^m$ 所生成的 $\sigma$-代数.
    \end{dfn}
    
    
    
    \begin{dfn}
        [Definition-of-$\mu$]
        {$\mu$ 的定义}
        [Definition of $\mu$]
        [gpt-4.1]
        设 $\mu = E \xi_i^m \in (0, \infty)$,即 $\mu$ 是随机变量 $\xi_i^m$ 的期望,且属于 $(0, \infty)$.
    \end{dfn}
    
    
    
    \begin{thm}
        [$Z-n/\mu^n$-is-a-Martingale-with-respect-to-$\mathcal{F}-n$]
        {$Z\_n/\mu^n$ 关于 $\mathcal{F}_n$ 是鞅}
        [$Z_n/\mu^n$ is a Martingale with respect to $\mathcal{F}_n$]
        [gpt-4.1]
        $Z_n / \mu^n$ 关于滤波族 ${\mathcal{F}}_n$ 是一个鞅.
    \end{thm}
    
    
    
    \begin{dfn}
        [Definition-of-Independence-between-Random-Variable-and-Information]
        {随机变量与信息独立的定义}
        [Definition of Independence between Random Variable and Information]
        [gpt-4.1]
        假设随机变量 $X$ 与 $\mathcal{F}$ 独立,即对于所有 $B \in \mathcal{R}$ 和 $A \in \mathcal{F}$,有
\[
P(\{ X \in B \} \cap A) = P(X \in B) P(A)
\]

    \end{dfn}
    
    
    
    \begin{thm}
        [Property-of-Conditional-Expectation-under-Independence]
        {独立情形下条件期望的性质}
        [Property of Conditional Expectation under Independence]
        [gpt-4.1]
        在上述独立的情况下,$E(X | \mathcal{F}) = EX$;也就是说,如果对 $X$ 没有任何信息,则最好的估计是均值 $EX$.

    \end{thm}
    
    
    
    \begin{lma}
        [Upper-Bound-Lemma-for-Probability-Sum]
        {概率和的上界引理}
        [Upper Bound Lemma for Probability Sum]
        [gpt-4.1]
        
Lemma 5.4.6 设 $m$ 为整数且 $m \geq 2$,则有
\[
\sum_{n=0}^{\infty} P(\|S_{n}\| < m\epsilon) \leq (2m)^{d} \sum_{n=0}^{\infty} P(\|S_{n}\| < \epsilon)
\]

    \end{lma}
    
    
    
    \begin{prf}
        [Proof-of-the-Upper-Bound-Lemma-for-Probability-Sum]
        {概率和上界引理的证明}
        [Proof of the Upper Bound Lemma for Probability Sum]
        [gpt-4.1]
        
证明 我们首先注意到
\[
\sum_{n=0}^{\infty} P(\|S_{n}\| < m\epsilon) \leq \sum_{n=0}^{\infty} \sum_{k} P(S_{n} \in k\epsilon + [0,\epsilon)^{d})
\]
其中内层求和是对 $k \in \{-m, \ldots, m-1\}^{d}$ 进行的.

如果我们令
\[
T_{k} = \inf\{\ell \geq 0 : S_{\ell} \in k\epsilon + [0,\epsilon)^{d}\}
\]
则根据 $T_{k}$ 的取值进行分解并利用 Fubini 定理可得
\[
\begin{array}{r}
\displaystyle\sum_{n=0}^{\infty} P(S_{n} \in k\epsilon + [0,\epsilon)^{d}) = \displaystyle\sum_{n=0}^{\infty} \sum_{\ell=0}^{n} P(S_{n} \in k\epsilon + [0,\epsilon)^{d},\, T_{k} = \ell) \\
\displaystyle\le \sum_{\ell=0}^{\infty} \sum_{n=\ell}^{\infty} P(\|S_{n} - S_{\ell}\| < \epsilon,\, T_{k} = \ell)
\end{array}
\]

由于事件 $\{T_{k} = \ell\}$ 与 $\{\|S_{n} - S_{\ell}\| < \epsilon\}$ 是相互独立的,最后一项等于
\[
= \sum_{m=0}^{\infty} P(T_{k} = m) \sum_{j=0}^{\infty} P(\|S_{j}\| < \epsilon)
\]

    \end{prf}
    
    
    
    \begin{xmp}
        [Example-of-a-Modified-Discrete-O.U.-Process]
        {离散O.U.过程的变体例子}
        [Example of a Modified Discrete O.U. Process]
        [gpt-4.1]
        取例 5.8.2 中的离散 O.U. 过程,并在整数 $x \geq 2$ 处修改转移概率,使得

\[
\begin{array}{l}
p(x, \{x+1\}) = 1 - x^{-2} \\
p(x, A) = x^{-2} |A| \quad \mathrm{for}\ A \subset (0, 1)
\end{array}
\]

其中 $p$ 是 Harris 链的转移概率,但有

\[
P_2(X_n = n+2\ \mathrm{for\ all}\ n) > 0
\]

这个例子说明某些链在理论上虽然不常见于实际应用,但在理论推广时更容易包含它们而不是修改假设以排除它们.
    \end{xmp}
    
    
    
    \begin{thm}
        [Properties-of-the-First-Entry-into-Set-A-for-Markov-Chains]
        {关于Markov链首次进入集合A的性质}
        [Properties of the First Entry into Set A for Markov Chains]
        [gpt-4.1]
        若 $P(X_{n} \in A \text{ 至少出现一次}) = 1$,则在以 $X_{0} \in A$ 为条件的概率下,$t_{n} = T_{n} - T_{n-1}$ 是平稳序列,且有 $E(T_{1} \mid X_{0} \in A) = 1 / P(X_{0} \in A)$.
    \end{thm}
    
    
    
    \begin{dfn}
        [Definitions-of-Variables-in-Wieners-Maximal-Inequality]
        {Wiener最大不等式中的变量定义}
        [Definitions of Variables in Wiener's Maximal Inequality]
        [gpt-4.1]
        
设 $X_{j}(\omega) = X(\varphi^{j} \omega)$,$S_{k}(\omega) = X_{0}(\omega) + \cdots + X_{k-1}(\omega)$,$A_{k}(\omega) = S_{k}(\omega)/k$,$D_{k} = \operatorname*{max}(A_{1}, \ldots, A_{k})$.

    \end{dfn}
    
    
    
    \begin{dfn}
        [Definition-of-Coordinate-Mapping]
        {坐标映射的定义}
        [Definition of Coordinate Mapping]
        [gpt-4.1]
        设序列空间 $(\mathbf{R}^{\mathbf{N}}, \mathcal{R}^{\mathbf{N}})$,对每个 $n$,定义坐标映射 $\xi_n(\omega) = \omega_n$.
    \end{dfn}
    
    
    
    \begin{dfn}
        [Definition-of-Distribution-Functions]
        {分布函数的定义}
        [Definition of Distribution Functions]
        [gpt-4.1]
        对于测度 $\mu$ 和 $
u$,定义分布函数 $F_n(x) = \mu(\xi_n \leq x)$,$G_n(x) = 
u(\xi_n \leq x)$.
    \end{dfn}
    
    
    
    \begin{dfn}
        [Definition-of-Radon-Nikodym-Derivative]
        {Radon-Nikodym 导数的定义}
        [Definition of Radon-Nikodym Derivative]
        [gpt-4.1]
        若 $F_n \ll G_n$,则令 $q_n = \frac{dF_n}{dG_n}$.
    \end{dfn}
    
    
    
    \begin{dfn}
        [Definition-of-Dyadic-Rationals]
        {二进分数的定义}
        [Definition of Dyadic Rationals]
        [gpt-4.1]
        $\mathbf{Q}_2 = \{m 2^{-n} : m, n \geq 0\}$ 称为二进分数.
    \end{dfn}
    
    
    
    \begin{thm}
        [Equivalence-Theorem-of-Positive-Recurrence-Stationary-Distribution-and-Ergodicity]
        {正再生性、平稳分布与遍历性的等价定理}
        [Equivalence Theorem of Positive Recurrence, Stationary Distribution, and Ergodicity]
        [gpt-4.1]
        定理 5. 对于马尔可夫链,以下条件等价:
(i) 存在状态 $x$ 是正再生的;
(ii) 存在平稳分布;
(iii) 对所有状态 $y$,$E _ { y } T _ { y } < \infty$(期望回归时间有限且不可约).

    \end{thm}
    
    
    
    \begin{prf}
        [Proof-of-the-Equivalence-Theorem]
        {等价定理的证明}
        [Proof of the Equivalence Theorem]
        [gpt-4.1]
        (i) 推 (ii):如果 $x$ 是正再生的,则
\[
\pi ( y ) = \sum _ { n = 0 } ^ { \infty } P _ { x } ( X _ { n } = y , T _ { x } > n ) / E _ { x } T _ { x }
\]
定义了一个平稳分布.

(ii) 推 (iii):由 11 可知 $\pi ( y ) = 1 / E _ { y } T _ { y }$,且不可约性保证对所有 $y$ 有 $\pi ( y ) > 0$,因此 $E _ { y } T _ { y } < \infty$.

(iii) 推 (i):显然成立.

    \end{prf}
    
    
    
    \begin{thm}
        [The-Class-of-Invariant-Events-is-a-Sigma-Field-and-the-Equivalence-of-Invariance]
        {不变事件类为σ-域且不变性等价条件}
        [The Class of Invariant Events is a Sigma-Field and the Equivalence of Invariance]
        [gpt-4.1]
        
证明不变事件类 $\mathcal{T}$ 是一个 $\sigma$-域,且 $X \in \mathcal{Z}$ 当且仅当 $X$ 是不变的,即 $X \circ \varphi = X$ 几乎处处成立.

    \end{thm}
    
    
    
    \begin{dfn}
        [Definition-of-Probability-Sequence-$g-n$]
        {关于概率序列 $g\_n$ 的定义}
        [Definition of Probability Sequence $g_n$]
        [gpt-4.1]
        
设 $g_{n} = P(S_{1} 
eq 0,\, \ldots,\, S_{n} 
eq 0)$ 对于 $n \geq 1$,并且 $g_{0} = 1$.

    \end{dfn}
    
    
    
    \begin{thm}
        [Summation-Formula-for-Expected-Value-$E-R-n$]
        {期望 $E R\_n$ 的求和表达式}
        [Summation Formula for Expected Value $E R_n$]
        [gpt-4.1]
        
证明:有
\[
E R_{n} = \sum_{m=1}^{n} g_{m-1}.
\]

    \end{thm}
    
    
    
    \begin{thm}
        [Theorem-on-Non-Differentiability-of-Brownian-Paths]
        {布朗运动路径的不可微性定理}
        [Theorem on Non-Differentiability of Brownian Paths]
        [gpt-4.1]
        以概率1,布朗运动路径在任意一点都不是Lipschitz连续的,因此也不可微.
    \end{thm}
    
    
    
    \begin{thm}
        [Ergodic-Theorem-for-Markov-Chains]
        {马尔可夫链的遍历定理}
        [Ergodic Theorem for Markov Chains]
        [gpt-4.1]
        
设 $X_n$ 是在可数状态空间上的不可约马尔可夫链,具有平稳分布 $\pi$.设 $f$ 是满足
\[
\sum_x |f(x)|\pi(x) < \infty
\]
的函数.则有
\[
\frac{1}{n} \sum_{m=0}^{n-1} f(X_m) \to \sum_x f(x)\pi(x) \quad \mathrm{a.s.}~\text{and in}~L^1
\]

    \end{thm}
    
    
    
    \begin{dfn}
        [Definition-of-Regular-Conditional-Distribution]
        {正则条件分布的定义}
        [Definition of Regular Conditional Distribution]
        [gpt-4.1]
        设 $(\Omega, \mathcal{F}, P)$ 是概率空间,$X: (\Omega, \mathcal{F}) \to (S, S)$ 是可测映射,$\mathcal{G}$ 是 $\sigma$-域,$\mathcal{G} \subset \mathcal{F}$.如果 $\mu: \Omega \times S \to [0,1]$ 满足:

(i) 对任意 $A$,$\omega \mapsto \mu(\omega, A)$ 是 $P(X \in A | \mathcal{G})$ 的一个版本;
(ii) 对于每个 $\omega$,$A \mapsto \mu(\omega, A)$ 是 $(S, S)$ 上的概率测度;

则称 $\mu$ 是 $X$ 关于 $\mathcal{G}$ 的正则条件分布.当 $S = \Omega$ 且 $X$ 为恒等映射时,$\mu$ 称为正则条件概率.

    \end{dfn}
    
    
    
    \begin{thm}
        [Switching-Principle-Supermartingale-under-Stopping-Time-Switching]
        {切换原理:停时切换下的超鞅性}
        [Switching Principle: Supermartingale under Stopping Time Switching]
        [gpt-4.1]
        
假设 $X_{n}^{1}$ 和 $X_{n}^{2}$ 是关于 ${\mathcal{F}}_{n}$ 的超鞅,$N$ 是一个停时,且 $X_{N}^{1} \ge X_{N}^{2}$.则

\[
\begin{array}{r}
Y_{n} = X_{n}^{1} 1_{(N > n)} + X_{n}^{2} 1_{(N \leq n)} \text{ 是一个超鞅.} \\
Z_{n} = X_{n}^{1} 1_{(N \geq n)} + X_{n}^{2} 1_{(N < n)} \text{ 是一个超鞅.}
\end{array}
\]

    \end{thm}
    
    
    
    \begin{dfn}
        [Definition-of-Galton-Watson-Process]
        {Galton-Watson过程的定义}
        [Definition of Galton-Watson Process]
        [gpt-4.1]
        设 $\xi _ { i } ^ { n }$, $i, n \geq 1$,是一组独立同分布的非负整数值随机变量.定义序列 $Z _ { n }, n \geq 0$,其中 $Z _ { 0 } = 1$,且
\[
Z _ { n + 1 } = \begin{cases}
\xi _ { 1 } ^ { n + 1 } + \dots + \xi _ { Z _ { n } } ^ { n + 1 } & \text{若 } Z _ { n } > 0 \\
0 & \text{若 } Z _ { n } = 0
\end{cases}
\]
$Z _ { n }$ 被称为Galton-Watson过程.

    \end{dfn}
    
    
    
    \begin{dfn}
        [Definition-of-Measure-Preserving-Map]
        {测度不变映射的定义}
        [Definition of Measure Preserving Map]
        [gpt-4.1]
        可测映射 $\varphi: \Omega \to \Omega$ 被称为测度不变的,如果对所有 $A \in \mathcal{F}$ 都有 $P(\varphi^{-1}A) = P(A)$.
    \end{dfn}
    
    
    
    \begin{dfn}
        [Definition-of-Iteration-of-Measure-Preserving-Map-and-Stationary-Sequence]
        {测度不变映射的迭代及平稳序列的定义}
        [Definition of Iteration of Measure Preserving Map and Stationary Sequence]
        [gpt-4.1]
        设 $\varphi^n$ 是$\varphi$的第$n$次迭代,递归定义为 $\varphi^n = \varphi(\varphi^{n-1})$,其中 $\varphi^0(\omega) = \omega$.若 $X \in \mathcal{F}$,则 $X_n(\omega) = X(\varphi^n \omega)$ 定义了一个平稳序列.
    \end{dfn}
    
    
    
    \begin{thm}
        [Expected-Return-Time-for-Irreducible-Discrete-Markov-Chains]
        {离散不可约马尔可夫链回归时间的期望}
        [Expected Return Time for Irreducible Discrete Markov Chains]
        [gpt-4.1]
        如果 $X_n$ 是以计数状态空间 $S$ 上的不可约马尔可夫链,且初始分布为其平稳分布 $\pi$,对于 $A = \{x\}$,则有 $E_{x} T_{x} = 1/\pi(x)$.
    \end{thm}
    
    
    
    \begin{thm}
        [Ergodic-Theorem]
        {遍历定理}
        [Ergodic Theorem]
        [gpt-4.1]
        对于任意 $X \in L^1$,

\[
\frac{1}{n} \sum_{m=0}^{n-1} X(\varphi^m \omega) \to E(X | \mathcal{T}) \quad \text{a.s. 并且在 } L^1
\]

该结果由 Birkhoff (1931) 给出,有时被称为逐点或个体遍历定理,因为结论中的收敛是几乎处处收敛.

    \end{thm}
    
    
    
    \begin{thm}
        [Value-Range-of-$\mathcal{H}-\gamma$-for-$\gamma-<-1/2$]
        {关于 $\gamma < 1/2$ 时 $\mathcal{H}_\gamma$ 的取值范围}
        [Value Range of $\mathcal{H}_\gamma$ for $\gamma < 1/2$]
        [gpt-4.1]
        
定理 7.1.5 说明,对于 $\gamma < 1/2$,有 $P(\mathcal{H}_\gamma = [0, \infty)) = 1$.

    \end{thm}
    
    
    
    \begin{thm}
        [Reflection-Principle]
        {反射原理}
        [Reflection Principle]
        [gpt-4.1]
        
定理 4.9.1 (反射原理) 若 $\dot{\boldsymbol{x}}, \boldsymbol{y} > 0$,则从 $(0, x)$ 到 $(n, y)$ 的路径中,某时刻为 $0$ 的路径数等于从 $(0, -x)$ 到 $(n, y)$ 的路径数.

    \end{thm}
    
    
    
    \begin{prf}
        [Proof-of-Reflection-Principle]
        {反射原理的证明}
        [Proof of Reflection Principle]
        [gpt-4.1]
        
设 $(0, s_0), (1, s_1), \ldots, (n, s_n)$ 是从 $(0, x)$ 到 $(n, y)$ 的一条路径.令 $K = \operatorname{inf} \{ k : s_k = 0 \}$.定义 $s_k' = -s_k$ 当 $k \leq K$,$s_k' = s_k$ 当 $K \leq k \leq n$.则 $(k, s_k')$, $0 \leq k \leq n$,是一条从 $(0, -x)$ 到 $(n, y)$ 的路径.反之,若 $(0, t_0), (1, t_1), \ldots, (n, t_n)$ 是从 $(0, -x)$ 到 $(n, y)$ 的路径,则必然会经过 0.令 $K = \operatorname{inf} \{ k : t_k = 0 \}$.定义 $t_k' = -t_k$ 当 $k \leq K$,$t_k' = t_k$ 当 $K \leq k \leq n$.那么 $(k, t_k')$, $0 \leq k \leq n$,是一条从 $(0, -x)$ 到 $(n, y)$,且在时刻 $K$ 处为 0 的路径.上述两个过程建立了一一对应关系,因此两类路径的数量相等.

    \end{prf}
    
    
    
    \begin{dfn}
        [Definition-of-Stopping-Time-and-Function-h]
        {停时和函数h的定义}
        [Definition of Stopping Time and Function h]
        [gpt-4.1]
        令 $V_{C} = \operatorname*{inf} \{ n \geq 0 : X_{n} \in C \}$,并令 $h(x) = P_{x}(V_{A} < V_{B})$.
    \end{dfn}
    
    
    
    \begin{thm}
        [Recursive-Equation-Satisfied-by-hx]
        {h(x)满足的递推方程}
        [Recursive Equation Satisfied by h(x)]
        [gpt-4.1]
        若 $A \cap B = \emptyset$,$S \setminus (A \cup B)$ 有限,且对于所有 $x \in S - (A \cup B)$,$P_{x}(V_{A \cup B} < \infty) > 0$,则对于 $x 
otin A \cup B$,
\[
h(x) = \sum_{y} p(x, y) h(y)
\]
成立.
    \end{thm}
    
    
    
    \begin{thm}
        [hXn∧V-{A∪B}-is-a-Martingale]
        {h(X(n∧V\_{A∪B}))是鞅}
        [h(X(n∧V_{A∪B})) is a Martingale]
        [gpt-4.1]
        如果 $h$ 满足 $(*)$,即 $h(x) = \sum_{y} p(x, y) h(y)$ 对 $x 
otin A \cup B$ 成立,则 $h(X(n \land V_{A\cup B}))$ 是一个鞅.
    \end{thm}
    
    
    
    \begin{thm}
        [Uniqueness-Theorem-for-Boundary-Conditions]
        {满足边界条件的唯一性定理}
        [Uniqueness Theorem for Boundary Conditions]
        [gpt-4.1]
        $h(x) = P_{x}(V_{A} < V_{B})$ 是满足 $(*)$,且在 $A$ 上为 1,$B$ 上为 0 的唯一解.
    \end{thm}
    
    
    
    \begin{dfn}
        [Definition-of-Brownian-Motion]
        {布朗运动的定义}
        [Definition of Brownian Motion]
        [gpt-4.1]
        布朗运动是一类高斯马尔可夫过程,具有平稳独立增量.
    \end{dfn}
    
    
    
    \begin{lma}
        [Lemma-on-the-Probability-$PL-{2n}-=-2k$]
        {关于概率$P(L\_{2n} = 2k)$的引理}
        [Lemma on the Probability $P(L_{2n} = 2k)$]
        [gpt-4.1]
        
设 $u_{2m} = P(S_{2m} = 0)$,则有
\[
P(L_{2n} = 2k) = u_{2k} u_{2n-2k}
\]

    \end{lma}
    
    
    
    \begin{prf}
        [Proof-of-the-Expression-for-$PL-{2n}-=-2k$]
        {概率$P(L\_{2n} = 2k)$表达式的证明}
        [Proof of the Expression for $P(L_{2n} = 2k)$]
        [gpt-4.1]
        
$P(L_{2n} = 2k) = P(S_{2k} = 0, S_{2k+1} 
eq 0, \dots, S_{2n} 
eq 0)$,所以所需结果由引理4.9.4得到.

    \end{prf}
    
    
    
    \begin{thm}
        [Theorem-on-Martingale-Property-of-$Y-n$-and-Limiting-Probability]
        {关于$Y\_n$的鞅性及极限概率的定理}
        [Theorem on Martingale Property of $Y_n$ and Limiting Probability]
        [gpt-4.1]
        
证明 $Y_n$ 是非负鞅,并利用此结论可得,如果 $\exp(-\theta a)/\mu\varphi(\theta)>1$,则 $P(X_{0,n}\leq an)\to 0$.

    \end{thm}
    
    
    
    \begin{dfn}
        [Definition-of-Marginal-Distribution]
        {边缘分布的定义}
        [Definition of Marginal Distribution]
        [gpt-4.1]
        
若 $F$ 是 $( X _ { 1 } , \ldots , X _ { d } )$ 的分布, 则 $F _ { i } ( x ) = P ( X _ { i } \leq x )$ 是其边缘分布.

    \end{dfn}
    
    
    
    \begin{thm}
        [Construction-of-Distribution-Function-with-Given-Marginals]
        {具有给定边缘分布的分布函数的构造}
        [Construction of Distribution Function with Given Marginals]
        [gpt-4.1]
        
设 $F _ { 1 } , \ldots , F _ { d }$ 是 $\mathbf { R }$ 上的分布函数.对于任意 $\alpha \in [ - 1 , 1 ]$,定义
\[
F ( x _ { 1 } , \dots , x _ { d } ) = \left\{ 1 + \alpha \prod _ { i = 1 } ^ { d } ( 1 - F _ { i } ( x _ { i } ) ) \right\} \prod _ { j = 1 } ^ { d } F _ { j } ( x _ { j } )
\]
则 $F$ 是一个具有给定边缘分布 $F _ 1 , \ldots , F _ d$ 的多元分布函数.

    \end{thm}
    
    
    
    \begin{thm}
        [Theorem-on-Finite-Closed-Set-Containing-a-Recurrent-State]
        {有限闭集包含一个常返态的定理}
        [Theorem on Finite Closed Set Containing a Recurrent State]
        [gpt-4.1]
        设 $C$ 是一个有限闭集.则 $C$ 包含一个常返态.如果 $C$ 是不可约的,则 $C$ 中所有状态都是常返态.
    \end{thm}
    
    
    
    \begin{dfn}
        [Definition-of-Scaling-Relation-for-Fourth-Moment-of-Brownian-Motion]
        {布朗运动的四次矩的缩放关系定义}
        [Definition of Scaling Relation for Fourth Moment of Brownian Motion]
        [gpt-4.1]
        
对于布朗运动 $B_t$,根据定义中的部分(b)和缩放关系,有
\[
{\mathcal{E}}_0 (|B_t - B_s|)^4 = {\mathcal{E}}_0 |B_{t-s}|^4 = C (t-s)^2
\]
其中 $C = E_0|B_1|^4 < \infty$.

    \end{dfn}
    
    
    
    \begin{thm}
        [Theorem-on-the-first-hitting-time-of-zero-for-Brownian-motion]
        {关于布朗运动首次过零时刻的定理}
        [Theorem on the first hitting time of zero for Brownian motion]
        [gpt-4.1]
        
若 $\tau = \inf \{ t \geq 0 : B_{t} > 0 \}$,则 $P_{0} ( \tau = 0 ) = 1$.

    \end{thm}
    
    
    
    \begin{xmp}
        [Rotation-of-the-Circle]
        {单位圆上的旋转}
        [Rotation of the Circle]
        [gpt-4.1]
        设 $\Omega = [0, 1)$,$\mathcal{F}$ 是 Borel 子集,$P$ 为 Lebesgue 测度.设 $\theta \in (0, 1)$,对于 $n \geq 0$,定义 $X_{n}(\omega) = (\omega + n\theta) \bmod 1$,其中 $x \bmod 1 = x - [x]$,$[x]$ 表示小于等于 $x$ 的最大整数.为了说明其命名原因,将 $[0, 1)$ 映射到 $\mathbf{C}$,映射为 $x \to \exp(2\pi i x)$.此例是前一例的特例.令 $p(x, \{y\}) = 1$ 当且仅当 $y = (x + \theta) \bmod 1$.
    \end{xmp}
    
    
    
    \begin{thm}
        [Theorem-of-Hölder-Continuity-for-Brownian-Paths]
        {布朗运动路径的 Hölder 连续性定理}
        [Theorem of Hölder Continuity for Brownian Paths]
        [gpt-4.1]
        布朗运动路径对于任意指数 $\gamma < 1/2$ 都是 Hölder 连续的.
    \end{thm}
    
    
    
    \begin{thm}
        [Cycle-Trick-Formula-for-Markov-Chains]
        {马尔可夫链循环技巧公式}
        [Cycle Trick Formula for Markov Chains]
        [gpt-4.1]
        
若 $P ( X _ { n } \in A \text{ 至少一次} ) = 1$ 且 $A \cap B = \emptyset$,则有
\[
E \left( \sum_{1 \leq m \leq T_{1}} 1_{( X_{m} \in B )} \mid X_{0} \in A \right) = \frac{P( X_{0} \in B )}{P( X_{0} \in A )}
\]
当 $A = \{ x \}$ 且 $X_{n}$ 为马尔可夫链时,这就是用于定义平稳测度的'循环技巧'.

    \end{thm}
    
    
    
    \begin{thm}
        [Theorem-on-the-First-Hitting-Time-of-Zero-for-Brownian-Motion]
        {关于布朗运动首次回零时间的定理}
        [Theorem on the First Hitting Time of Zero for Brownian Motion]
        [gpt-4.1]
        设 $T_{0} = \operatorname*{inf}\{ t > 0 : B_{t} = 0 \}$,则 $P_{0}(T_{0} = 0) = 1$.
    \end{thm}
    
    
    
    \begin{dfn}
        [Definition-of-Superharmonic-Function]
        {超调和函数的定义}
        [Definition of Superharmonic Function]
        [gpt-4.1]
        
设 $f$ 是定义在 $\mathbf{R}^d$ 上的函数, 若对任意 $x \in \mathbf{R}^d$ 及任意球 $B(x, r)$ 有
\[
f(x) \geq \frac{1}{|B(x, r)|} \int_{B(x, r)} f(y) dy
\]
则称 $f$ 为超调和函数 (superharmonic function).

    \end{dfn}
    
    
    
    \begin{thm}
        [Supermartingale-Property-of-Superharmonic-Function]
        {超调和函数的超鞅性质}
        [Supermartingale Property of Superharmonic Function]
        [gpt-4.1]
        
设 $f$ 是定义在 $\mathbf{R}^{d}$ 上的超调和函数.令 $\xi_{1}, \xi_{2}, \ldots$ 为独立同分布且均匀分布在 $B(0,1)$ 上的随机变量,定义 $S_{n}$ 递归为 $S_{n} = S_{n-1} + \xi_{n}$, $S_{0} = x$.则序列 $X_{n} = f(S_{n})$ 构成一个超鞅 (supermartingale).

    \end{thm}
    
    
    
    \begin{thm}
        [Constancy-of-Nonnegative-Superharmonic-Functions-in-Low-Dimensions]
        {低维非负超调和函数的常值性}
        [Constancy of Nonnegative Superharmonic Functions in Low Dimensions]
        [gpt-4.1]
        
在 $d \leq 2$ 的情况下,非负的超调和函数必须是常数函数.

    \end{thm}
    
    
    
    \begin{cxmp}
        [Counterexample-Nonconstant-Superharmonic-Function-in-Higher-Dimensions]
        {高维超调和函数非常值的反例}
        [Counterexample: Nonconstant Superharmonic Function in Higher Dimensions]
        [gpt-4.1]
        
在 $d > 2$ 的情形下,函数 $f(x) = |x|^{2-d}$ 是一个非常值的超调和函数,说明在高维下超调和函数不必为常数.

    \end{cxmp}
    
    
    
    \begin{xmp}
        [Example-of-Conditional-Expectation-with-Partition]
        {条件期望的分割情形例子}
        [Example of Conditional Expectation with Partition]
        [gpt-4.1]
        
设 $\Omega_{1}, \Omega_{2}, \ldots$ 是对 $\Omega$ 的有限或无限分割,每个集合互不相交且具有正概率,令 $\mathcal{F} = \sigma(\Omega_{1}, \Omega_{2}, \ldots)$ 为由这些集合生成的 $\sigma$-代数,则有
\[
E(X | \mathcal{F}) = \frac{E(X ; \Omega_{i})}{P(\Omega_{i})} \quad \mathrm{on~} \Omega_{i}
\]
即,已知结果落在 $\Omega_{i}$ 后,对 $X$ 的最优估计为 $X$ 在 $\Omega_{i}$ 上的平均值.特殊情形:若 ${\mathcal{F}} = \{\emptyset, \Omega\}$(平凡 $\sigma$-代数),则 $E(X|{\mathcal{F}}) = E X$.

    \end{xmp}
    
    
    
    \begin{dfn}
        [Definition-of-Conditional-Probability]
        {条件概率的定义}
        [Definition of Conditional Probability]
        [gpt-4.1]
        
设 $A$ 为事件,$\mathcal{G}$ 为子 $\sigma$-代数,则
\[
P(A | \mathcal{G}) = E(1_{A} | \mathcal{G})
\]
对于 $B$ 满足 $P(B) > 0$ 有
\[
P(A | B) = \frac{P(A \cap B)}{P(B)}
\]

    \end{dfn}
    
    
    
    \begin{xmp}
        [An-Example-on-Ergodic-Properties-and-Measure]
        {关于遍历性质和测度的例子}
        [An Example on Ergodic Properties and Measure]
        [gpt-4.1]
        
设 $\omega \in \Omega_{k}$ 且 $P(\Omega_{k}) = 1$.令 $G = \cap \Omega_{k}$,其中交集取所有满足 $b - a > 2 / k$ 的整数 $k$.有 $P(G) = 1$,因此 $G$ 在 $[0,1)$ 上稠密.若 $x \in [0,1)$ 且 $\omega_{k} \in G$ 使得 $|\omega_{k} - x| < 1 / k$,则 $\varphi^{m} \omega_{k} \in A_{k}$ 蕴含 $\varphi^{m} x \in A$,于是

\[
\operatorname*{lim}_{n \to \infty} \frac{1}{n} \sum_{m = 0}^{n - 1} 1_{A}(\varphi^{m} x) \geq b - a - \frac{2}{k}
\]

对所有充分大的 $k$ 都成立.注意 $k$ 是任意的,类似推理对 $A^{c}$ 亦适用,于是有

\[
\frac{1}{n} \sum_{m = 0}^{n - 1} 1_{A}(\varphi^{m} x) \to b - a
\]

    \end{xmp}
    
    
    
    \begin{dfn}
        [Definition-of-Right-Limit-Filtration]
        {右极限过滤子的定义}
        [Definition of Right Limit Filtration]
        [gpt-4.1]
        
$\mathcal{F}_s^{+} = \cap_{t > s} \mathcal{F}_t^{+} = \cap_{t > s} \left( \cap_{u > t} \mathcal{F}_u^{o} \right) = \cap_{u > s} \mathcal{F}_u^{o}$

其中,$\mathcal{F}_s^{+}$ 表示'对未来的最小窥视'(infimal peek into the future).

    \end{dfn}
    
    
    
    \begin{lma}
        [Proof-of-Lemma-7]
        {引理7的证明}
        [Proof of Lemma 7]
        [gpt-4.1]
        
设 $q, r \in \mathbf{Q}_2 \cap [0, 1]$ 且 $0 < r - q < 2^{-N}$.令 $I_i^k = [(i-1)/2^k, i/2^k)$,$m$ 是使得 $q$ 和 $r$ 处于不同区间 $I_i^k$ 的最小 $k$.由于在 $k-1$ 层时它们在同一区间,故 $q \in I_i^m$, $r \in I_{i+1}^m$.

    \end{lma}
    
    
    
    \begin{xmp}
        [Example-of-Rotation-of-the-Circle]
        {圆周旋转的例子}
        [Example of Rotation of the Circle]
        [gpt-4.1]
        
设 $\Omega = [0, 1)$,$\varphi(\omega) = (\omega + \theta) \bmod 1$.假设 $\theta \in (0, 1)$ 是无理数,则根据第7.1节的结果,$\mathcal{Z}$ 是平凡的.令 $X(\omega) = 1_A(\omega)$,其中 $A$ 是 $[0,1)$ 的 Borel 子集,则遍历定理推出
\[
\frac{1}{n} \sum_{m=0}^{n-1} 1_{(\varphi^m \omega \in A)} \to |A| \quad \mathrm{a.s.}
\]
其中 $|A|$ 表示 $A$ 的勒贝格测度.对于 $\omega=0$ 的情形,这一结果通常称为 Weyl 的等分布定理,尽管 Bohl 和 Sierpinski 也应当获得认可.

    \end{xmp}
    
    
    
    \begin{ppt}
        [Property-of-Maximum-Sum-of-Squares-for-Non-negative-Numbers-Summing-to-1]
        {非负数和为1时平方和的最大值性质}
        [Property of Maximum Sum of Squares for Non-negative Numbers Summing to 1]
        [gpt-4.1]
        如果一组非负数 $a_{j,k} \geq 0$ 且 $\sum_{j,k} a_{j,k} = 1$,则有
\[
\sum_{j,k} a_{j,k}^{2} \le \max_{j,k} a_{j,k}.
\]

    \end{ppt}
    
    
    
    \begin{thm}
        [Estimate-of-Maximum-Value-of-Trinomial-Coefficient]
        {三项系数极值的估计}
        [Estimate of Maximum Value of Trinomial Coefficient]
        [gpt-4.1]
        对于所有 $j, k$,有
\[
\max_{j,k} \left( 3^{-n} \frac{n!}{j! \, k! \, (n-j-k)!} \right) \leq C n^{-1}
\]
其中常数 $C$ 与 $n$ 无关.

    \end{thm}
    
    
    
    \begin{ppt}
        [Symmetry-of-Trinomial-Denominator-at-Maximum-Point]
        {极值点处三项分母的对称性}
        [Symmetry of Trinomial Denominator at Maximum Point]
        [gpt-4.1]
        当 $j, k, n-j-k$ 三者都尽量接近 $n/3$ 时,$\frac{n!}{j! \, k! \, (n-j-k)!}$ 取得最大值;若其中任一项小于 $[n/3]$,则增加最小项并减少最大项会增加分母,从而降低整体值.

    \end{ppt}
    
    
    
    \begin{ppt}
        [Application-of-Stirlings-Formula-to-Trinomial-Coefficient]
        {Stirling公式在三项系数中的应用}
        [Application of Stirling's Formula to Trinomial Coefficient]
        [gpt-4.1]
        Stirling公式给出如下近似:
\[
\frac{n!}{j! \, k! \, (n-j-k)!} \sim \frac{n^{n}}{j^{j} \, k^{k} \, (n-j-k)^{n-j-k}} \cdot \sqrt{ \frac{n}{j k (n-j-k)} } \cdot \frac{1}{2\pi}
\]
当 $j, k$ 均在 $n/3$ 附近时,第一项的大小 $\leq C 3^{n}$.

    \end{ppt}
    
    
    
    \begin{thm}
        [Conditional-Variance-Formula]
        {条件方差公式}
        [Conditional Variance Formula]
        [gpt-4.1]
        
如果 $X_n$ 是一个鞅,且对所有 $n$ 有 $E X_n^2 < \infty$,则有
\[
E \left( (X_n - X_m)^2 \mid \mathcal{F}_m \right) = E \left( X_n^2 \mid \mathcal{F}_m \right) - X_m^2
\]

    \end{thm}
    
    
    
    \begin{prf}
        [Proof-of-Conditional-Variance-Formula]
        {条件方差公式的证明}
        [Proof of Conditional Variance Formula]
        [gpt-4.1]
        
利用条件期望的线性性以及定理 4.1.14,有
\[
\begin{aligned}
& E \left( X_n^2 - 2 X_n X_m + X_m^2 \mid \mathcal{F}_m \right) \\
= & E \left( X_n^2 \mid \mathcal{F}_m \right) - 2 X_m E \left( X_n \mid \mathcal{F}_m \right) + X_m^2 \\
= & E \left( X_n^2 \mid \mathcal{F}_m \right) - 2 X_m^2 + X_m^2
\end{aligned}
\]
从而得到所需结果.

    \end{prf}
    
    
    
    \begin{prf}
        [Proof-of-the-Stable-Law-Limit-Theorem]
        {稳定分布极限定理的证明}
        [Proof of the Stable Law Limit Theorem]
        [gpt-4.1]
        如果 $Y$ 具有稳定分布,则我们可以取 $X_1, X_2, \dots$ 为独立同分布,分布同 $Y$.

反过来,令
\[
Z_n = \frac{X_1 + \cdots + X_n - b_n}{a_n}
\]
以及 $S_n^j = X_{(j-1)n+1} + \cdots + X_{jn}$.

通过简单运算可得
\[
\begin{array}{rl}
& Z_{nk} = \frac{S_n^1 + \cdots + S_n^k - b_{nk}}{a_{nk}} \\
& a_{nk} Z_{nk} = (S_n^1 - b_n) + \cdots + (S_n^k - b_n) + (k b_n - b_{nk}) \\
& \frac{a_{nk} Z_{nk}}{a_n} = \frac{S_n^1 - b_n}{a_n} + \cdots + \frac{S_n^k - b_n}{a_n} + \frac{k b_n - b_{nk}}{a_n}
\end{array}
\]

右边前 $k$ 项在 $n \to \infty$ 时趋于 $Y_1 + \cdots + Y_k$,其中 $Y_1, \dots, Y_k$ 独立且分布与 $Y$ 相同,且 $Z_{nk} \Rightarrow Y$.

取 $W_n = Z_{nk}$ 以及
\[
W_n' = \frac{a_{kn}}{a_n} Z_{nk} - \frac{k b_n - b_{nk}}{a_n}
\]
即可得到所需结果.

    \end{prf}
    
    
    
    \begin{lma}
        [Lemma-on-Lower-Bound-of-Integral-via-Compactness]
        {关于积分下界的紧性引理}
        [Lemma on Lower Bound of Integral via Compactness]
        [gpt-4.1]
        
设 $S = \{x : |x| = 1\}$,$\theta \in S$,$\mu$ 是一个测度.若对所有 $\theta \in S$,有 $\int |x \cdot \theta|^{2} \mu(dx) > 0$,则存在 $\rho_{0} > 0$ 和常数 $C > 0$,使得对所有 $0 < \rho < \rho_{0}$,

\[
\inf_{\theta \in S} \int_{|x \cdot \theta| < \pi/3\rho} |x \cdot \theta|^{2} \mu(dx) = C > 0
\]

说明:若不存在这样的 $C > 0$,则可构造序列 $\theta_n \in S$ 使得 $\int_{|x \cdot \theta_{n}| < \pi/3n} |x \cdot \theta_{n}|^{2} \mu(dx) \leq 1/n$,但 $S$ 紧致,存在收敛子列,与 $\int |x \cdot \theta|^{2} \mu(dx) > 0$ 矛盾.

    \end{lma}
    
    
    
    \begin{dfn}
        [Definition-of-Shift-Transformation-for-Brownian-Motion-Paths]
        {布朗运动路径的平移变换定义}
        [Definition of Shift Transformation for Brownian Motion Paths]
        [gpt-4.1]
        
对于 $s \geq 0$,定义平移变换 $\theta_s : C \to C$ 为
\[
(\theta_s \omega)(t) = \omega(s + t) \quad \mathrm{~for~} t \geq 0
\]
即,将路径在时刻 $s$ 之前的部分截去,并将时刻 $s$ 作为新的零时刻.

    \end{dfn}
    
    
    
    \begin{thm}
        [Stationarity-and-Ergodicity-of-Sequence-under-Independent-Blocks-Method]
        {独立分块方法下的序列平稳性与遍历性}
        [Stationarity and Ergodicity of Sequence under Independent Blocks Method]
        [gpt-4.1]
        
设 $X_1, X_2, \dots$ 是一个平稳序列.令 $n<\infty$,并令 $Y_1, Y_2, \ldots$ 是一列,使得 $(Y_{nk+1}, \ldots, Y_{n(k+1)})$, $k\geq 0$ 是独立同分布的,且 $(Y_1, \dots, Y_n) = (X_1, \dots, X_n)$.令 $
u$ 在 $\{1,2,\ldots,n\}$ 上均匀分布,且与 $Y$ 独立,令 $Z_m = Y_{
u + m}$, $m\geq 1$.则序列 $Z$ 是平稳且遍历的.

    \end{thm}
    
    
    
    \begin{thm}
        [Recurrence-Theorem-for-Random-Walks-with-Stationary-Increments]
        {平稳增量随机游走的回归性定理}
        [Recurrence Theorem for Random Walks with Stationary Increments]
        [gpt-4.1]
        
设 $X_1, X_2, \dots$ 是取值于 $\mathbf{Z}$ 的平稳序列,且 $E|X_i| < \infty$.定义 $S_n = X_1 + \cdots + X_n$,令 $A = \{ S_1 
eq 0, S_2 
eq 0, \ldots \}$.

(i) 若 $E(X_1 | \mathcal{T}) = 0$,则 $P(A) = 0$.

(ii) 若 $P(A) = 0$,则 $P( S_n = 0 \text{ i.o.} ) = 1$.

    \end{thm}
    
    
    
    \begin{thm}
        [Theorem-on-Filtration-Inclusion-for-Stopping-Times]
        {停止时间的滤子包含性定理}
        [Theorem on Filtration Inclusion for Stopping Times]
        [gpt-4.1]
        如果 $S \leq T$ 是两个停止时间,则有 $\mathcal{F}_S \subset \mathcal{F}_T$.
    \end{thm}
    
    
    
    \begin{thm}
        [Theorem-Measurability-of-$Bs$-with-respect-to-$\mathcal{F}-S$]
        {可测性定理:$Bs$ 的值关于 $\mathcal{F}_S$ 的可测性}
        [Theorem: Measurability of $Bs$ with respect to $\mathcal{F}_S$]
        [gpt-4.1]
        $Bs \in \mathcal{F}_S$, 即 $Bs$ 的值关于在时刻 $S$ 已知的信息是可测的.
    \end{thm}
    
    
    
    \begin{thm}
        [Expectation-Formula-for-Hitting-Time-of-Brownian-Motion]
        {布朗运动击中时间的期望公式}
        [Expectation Formula for Hitting Time of Brownian Motion]
        [gpt-4.1]
        
设 $T_a$ 为布朗运动首次到达 $a$ 的击中时间,则有
\[
E_{0}(\exp(-\lambda T_{a})) = \exp(-a \sqrt{2\lambda})
\]

    \end{thm}
    
    
    
    \begin{dfn}
        [Definition-of-Infinitely-Divisible-Distribution]
        {无限可分布的定义}
        [Definition of Infinitely Divisible Distribution]
        [gpt-4.1]
        若随机变量 $Z$ 对于任意正整数 $n$,都存在一组独立同分布的随机变量 $Y_{n,1}, \dots, Y_{n,n}$,使得
\[
Z \overset{d}{=} Y_{n,1} + \cdots + Y_{n,n}
\]
则称 $Z$ 具有无限可分布(infinitely divisible distribution).

    \end{dfn}
    
    
    
    \begin{thm}
        [A-Distribution-is-a-Limit-of-Sums-if-and-only-if-it-is-Infinitely-Divisible]
        {极限可表示为和的分布当且仅当为无限可分布}
        [A Distribution is a Limit of Sums if and only if it is Infinitely Divisible]
        [gpt-4.1]
        设 $Z$ 是某一类和的极限,即存在一列独立同分布随机变量 $X_{n,1}, \dots, X_{n,n}$ 使
\[
S_n = X_{n,1} + \cdots + X_{n,n} \Rightarrow Z,
\]
则当且仅当 $Z$ 具有无限可分布.

    \end{thm}
    
    
    
    \begin{prf}
        [Proof-of-Equivalence-of-Limits-of-Sums-and-Infinite-Divisibility]
        {极限为和的分布等价于无限可分的证明}
        [Proof of Equivalence of Limits of Sums and Infinite Divisibility]
        [gpt-4.1]
        只需证明必要性.写作
\[
S_{2n} = ( X_{2n,1} + \cdots + X_{2n,n} ) + ( X_{2n,n+1} + \cdots + X_{2n,2n} ) \equiv Y_n + Y_n'
\]
随机变量 $Y_n$ 与 $Y_n'$ 独立且同分布.如果 $S_n \Rightarrow Z$,则 $Y_n$ 的分布形成紧列,因为
\[
P ( Y_n > y )^2 = P ( Y_n > y ) P ( Y_n' > y ) \leq P ( S_{2n} > 2y )
\]
同理有 $P ( Y_n < -y )^2 \leq P ( S_{2n} < -2y )$.取子列 $n_k$ 使 $Y_{n_k} \Rightarrow Y$(故 $Y_{n_k}' \Rightarrow Y'$),则 $Z \overset{d}{=} Y + Y'$.类似可证明 $Z$ 可被分为 $n > 2$ 个部分,证毕.

    \end{prf}
    
    
    
    \begin{thm}
        [Theorem-on-Sets-of-Stopping-Times-Belonging-to-Filtration]
        {关于停止时间的集合属于信息流的定理}
        [Theorem on Sets of Stopping Times Belonging to Filtration]
        [gpt-4.1]
        设 $S$ 和 $T$ 是停止时间,则集合 $\{ S < T \}$、$\{ S > T \}$ 和 $\{ S = T \}$ 均属于 $\mathcal{F}_{S}$(并且属于 $\mathcal{F}_{T}$).
    \end{thm}
    
    
    
    \begin{prf}
        [Proof-of-Theorem-6]
        {定理6的证明}
        [Proof of Theorem 6]
        [gpt-4.1]
        
定理6的证明分为四个步骤.第一个、第二个和第四个步骤可追溯到Kingman(1968).现有的关于次可加遍历定理的六种证明均在关键的第三步采用不同的方法.本证明采用S. Leventhal(1988)的处理方式,其证明又基于Katznelson和Weiss(1982).第1步:首先需要验证 $E | X_{0,n} | \leq C n$.为此,注意到(i)蕴含 $X_{0,m}^{+} + X_{m,n}^{+} \geq X_{0,n}^{+}$,反复使用该不等式并结合(iii)可得 $E X_{0,n}^{+} \leq n E X_{0,1}^{+} < \infty$.由于 $|x| = 2 x^{+} - x$,由(iv)可得

\[
E | X_{0,n} | \leq 2 E X_{0,n}^{+} - E X_{0,n} \leq C n < \infty
\]

令 $a_{n} = E X_{0,n}$.

    \end{prf}
    
    
    
    \begin{lma}
        [Inequality-Bound-for-$|Xq---Xr|$]
        {关于$|X(q) - X(r)|$的界的不等式}
        [Inequality Bound for $|X(q) - X(r)|$]
        [gpt-4.1]
        
在 ${\mathcal{H}}_N = \cap_{n=N}^{\infty} G_n$ 上,对于 $q, r \in \mathbf{Q}_2 \cap [0, 1]$ 且 $|q - r| < 2^{-N}$,有
\[
|X(q) - X(r)| \leq \frac{3}{1 - 2^{-\gamma}} |q - r|^\gamma
\]

    \end{lma}
    
    
    
    \begin{thm}
        [Theorem-on-Expectation-Calculation-Formula]
        {期望的计算公式定理}
        [Theorem on Expectation Calculation Formula]
        [gpt-4.1]
        
\[
E_{0} B^{2}(T \wedge t) \to E_{0} B_{T}^{2} = a^{2} \frac{b}{b - a} + b^{2} \frac{-a}{b - a} = ab \frac{a - b}{b - a} = -ab
\]

    \end{thm}
    
    
    
    \begin{thm}
        [Conclusion-of-Theorem-4.3.12-for-μ->-1]
        {大于1时的定理4.3.12结论}
        [Conclusion of Theorem 4.3.12 for μ > 1]
        [gpt-4.1]
        当 $\mu > 1$ 时,定理4.3.12 蕴含 $P_{x}(T_{0}<\infty)=\rho^{x}$,其中 $\rho$ 是函数 $\varphi(\theta)=\sum_{k=0}^{\infty} a_{k} \theta^{k}$ 在区间 $(0,1)$ 内唯一的、不动点.
    \end{thm}
    
    
    
    \begin{dfn}
        [Definition-of-the-Initial-Sigma-Algebra]
        {初始σ-代数的定义}
        [Definition of the Initial Sigma-Algebra]
        [gpt-4.1]
        $A \in \mathcal{F}_o$ 当且仅当存在一个时刻序列 $t_1, t_2, \ldots$(其中每个 $t_i \in [0, \infty)$),以及一个集合 $B \in \mathcal{R}^{\{1,2,\ldots\}}$,使得 $A = \{\omega : (\omega(t_1), \omega(t_2), \ldots) \in B\}$.
    \end{dfn}
    
    
    
    \begin{thm}
        [Theorem-on-the-Tail-Sigma-field-of-Irreducible-Recurrent-Period-d-Markov-Chains]
        {不可约、遍历、周期d的Markov链的尾σ域定理}
        [Theorem on the Tail Sigma-field of Irreducible, Recurrent, Period d Markov Chains]
        [gpt-4.1]
        设 $p$ 是不可约的、遍历的Markov链,所有状态的周期为 $d$,$\mathcal{T} = \sigma(\{ X_{0} \in S_{r} \} : 0 \leq r < d)$.则对任意初始分布 $\mu$ 和 $A \in \mathcal{T}$,存在某个 $r$ 使得 $A = \{ X_{0} \in S_{r} \}$,且 $P_{\mu}$-a.s. 成立.

    \end{thm}
    
    
    
    \begin{prf}
        [Proof-of-Theorem-on-the-Tail-Sigma-field-of-Irreducible-Recurrent-Period-d-Markov-Chains]
        {不可约、遍历、周期d的Markov链的尾σ域定理的证明}
        [Proof of Theorem on the Tail Sigma-field of Irreducible, Recurrent, Period d Markov Chains]
        [gpt-4.1]
        我们将证明过程分三步进行.

情形1:假设 $P(X_{0} = x) = 1$.令 $T_{0} = 0$,对 $n \geq 1$,令 $T_{n} = \inf\{ m > T_{n-1} : X_{m} = x \}$ 为第 $n$ 次回到 $x$ 的时刻.令
\[
V_{n} = (X(T_{n-1}), \ldots, X(T_{n}-1))
\]
由练习 5.3.1 可知,向量 $V_{n}$ 互相独立同分布,且尾 $\sigma$ 域包含在 $V_{n}$ 的可交换域中,因此 Hewitt-Savage 0-1 定律(定理 2.5.4,已对取值于一般可测空间的随机变量证明)推出在此情形下 $\tau$ 是平凡的.

    \end{prf}
    
    
    
    \begin{dfn}
        [Definition-of-Events-$F-j$-and-$G-{jk}$]
        {事件 $F\_j$ 和 $G\_{j,k}$ 的定义}
        [Definition of Events $F_j$ and $G_{j,k}$]
        [gpt-4.1]
        设 $F_{j} = \{ S_{i} 
eq 0 \text{ for } 1 \leq i < j,\, S_{j} = 0 \}$,以及
\[
G_{j,k} = \{ S_{j+i} - S_{j} 
eq 0 \text{ for } 1 \leq i < k,\, S_{j+k} - S_{j} = 0 \}.
\]

    \end{dfn}
    
    
    
    \begin{crl}
        [$PA-=-0$-Implies-the-Sum-of-Probabilities-Equals-1]
        {$P(A) = 0$ 蕴含概率和等于1}
        [$P(A) = 0$ Implies the Sum of Probabilities Equals 1]
        [gpt-4.1]
        $P(A) = 0$ 蕴含 $\sum P(F_{k}) = 1$.
    \end{crl}
    
    
    
    \begin{dfn}
        [Definition-of-Absorbing-State]
        {吸收态的定义}
        [Definition of Absorbing State]
        [gpt-4.1]
        状态 $0$ 满足 $p ( 0 , 0 ) = 1$ 且是常返的.称其为吸收态,表示一旦链进入 $0$,则在所有时间都停留在该状态.
    \end{dfn}
    
    
    
    \begin{thm}
        [Equivalence-of-$\mathcal{F}-{s}^{+}$-and-$\mathcal{F}-{s}^{o}$]
        {关于$\mathcal{F}_{s}^{+}$与$\mathcal{F}_{s}^{o}$的等价性}
        [Equivalence of $\mathcal{F}_{s}^{+}$ and $\mathcal{F}_{s}^{o}$]
        [gpt-4.1]
        如果令 $Z \in \mathcal{F}_{s}^{+}$,那么定理 7.2.2 蕴含 $Z = E_x ( Z | \mathcal{F}_{s}^{o} ) \in \mathcal{F}_{s}^{o}$,因此这两个 $\sigma$-域在零测集意义下是相同的.
    \end{thm}
    
    
    
    \begin{thm}
        [Reflection-Principle]
        {反射原理}
        [Reflection Principle]
        [gpt-4.1]
        
设 $a > 0$ 且 $T_{a} = \inf\{t : B_{t} = a\}$,则有
\[
P_{0}(T_{a} < t) = 2P_{0}(B_{t} \geq a)
\]

    \end{thm}
    
    
    
    \begin{thm}
        [Stopping-Time-Property-under-Sup-Inf-Limsup-and-Liminf]
        {极限与上下界操作后的停时性}
        [Stopping Time Property under Sup, Inf, Limsup, and Liminf]
        [gpt-4.1]
        
设 $T_{n}$ 是一列停时(stopping times).则 $\sup_{n} T_{n}$,$\inf_{n} T_{n}$,$\limsup_{n} T_{n}$,$\liminf_{n} T_{n}$ 仍是停时.

    \end{thm}
    
    
    
    \begin{dfn}
        [Definition-of-the-Random-Variable-$Y-m$]
        {随机变量 $Y\_m$ 的定义}
        [Definition of the Random Variable $Y_m$]
        [gpt-4.1]
        设 $Y _ { m } ( \omega ) = 1$ 当且仅当 $m \leq n$ 且 $\omega _ { n - m } \geq a$, 否则 $Y _ { m } ( \omega ) = 0$.
    \end{dfn}
    
    
    
    \begin{ppt}
        [Property-of-$Y-N-\circ-	heta-N$]
        {关于 $Y\_N \circ 	heta\_N$ 的性质}
        [Property of $Y_N \circ 	heta_N$]
        [gpt-4.1]
        $Y _ { N } \circ \theta _ { N } ( \omega ) = 1$ 当且仅当 $\omega _ { n } \geq a$(因此 $N \leq n$),否则为 $0$.
    \end{ppt}
    
    
    
    \begin{thm}
        [Inequality-for-Sums-of-Random-Variables-and-Probabilities]
        {关于随机变量和概率的不等式}
        [Inequality for Sums of Random Variables and Probabilities]
        [gpt-4.1]
        对所有 $n$ 和 $a$,有
\[
P ( S _ { n } > a ) \geq \frac { 1 } { 2 } P ( N \leq n )
\]

    \end{thm}
    
    
    
    \begin{prf}
        [Proof-of-the-Probability-Inequality-for-Sums-of-Random-Variables]
        {随机变量概率不等式的证明}
        [Proof of the Probability Inequality for Sums of Random Variables]
        [gpt-4.1]
        证明如下:

利用强马尔可夫性质,有
\[
E _ { 0 } ( Y _ { N } \circ \theta _ { N } | \mathcal { F } _ { N } ) = E _ { S _ { N } } Y _ { N } \quad \mathrm{on} \ \{ N < \infty \} = \{ N \leq n \}
\]

若 $y > a$,则
\[
E _ { y } Y _ { m } = P _ { y } ( S _ { n - m } \geq a ) \geq P _ { y } ( S _ { n - m } \geq y ) \geq 1 / 2
\]

对 $\{ N \leq n \}$ 进行积分并利用条件期望的定义可得
\[
\frac { 1 } { 2 } P ( N \leq n ) \leq E _ { 0 } ( E _ { 0 } ( Y _ { N } \circ \theta _ { N } | \mathcal{F} _ { N } ) ; N \leq n ) = E _ { 0 } ( Y _ { N } \circ \theta _ { N } ; N \leq n )
\]

由于 $\{ N \leq n \} \in \mathcal { F } _ { N }$,而 $Y _ { N } \circ \theta _ { N } = 1 _ { \{ S _ { n } \geq a \} }$,最后可得
\[
E _ { 0 } ( 1 _ { \{ S _ { n } \geq a \} } ; N \leq n ) = P _ { 0 } ( S _ { n } \geq a )
\]

又 $\{ S _ { n } \geq a \} \subset \{ N \leq n \}$,故证毕.
    \end{prf}
    
    
    
    \begin{lma}
        [Estimate-of-Local-Probability-for-2D-Random-Walk]
        {关于二维随机游走局部性概率的估计}
        [Estimate of Local Probability for 2D Random Walk]
        [gpt-4.1]
        设 $u(n, m) = P(\|S_n\| < m)$,则有
\[
\sum_{n=0}^\infty u(n, 1) \geq (4m^2)^{-1} \sum_{n=0}^\infty u(n, m)
\]
且若 $m / \sqrt{n} \to c$,则
\[
u(n, m) \to \int_{[-c, c]^2} n(x) dx
\]
其中 $n(x)$ 是极限正态分布的密度.
    \end{lma}
    
    
    
    \begin{prf}
        [Proof-of-Estimate-of-Local-Probability-for-2D-Random-Walk]
        {二维随机游走局部性概率的估计证明}
        [Proof of Estimate of Local Probability for 2D Random Walk]
        [gpt-4.1]
        令 $u(n, m) = P(\|S_n\| < m)$.由引理有
\[
\sum_{n=0}^\infty u(n, 1) \geq (4m^2)^{-1} \sum_{n=0}^\infty u(n, m)
\]
若 $m / \sqrt{n} \to c$,则
\[
u(n, m) \to \int_{[-c, c]^2} n(x) dx
\]
其中 $n(x)$ 是极限正态分布的密度.记右侧为 $\rho(c)$,令 $n = [\theta m^2]$,则 $u([\theta m^2], m) \to \rho(\theta^{-1/2})$.
写为
\[
m^{-2} \sum_{n=0}^\infty u(n, m) = \int_0^\infty u([\theta m^2], m) d\theta
\]
令 $m \to \infty$,利用 Fatou 引理,有
\[
\operatorname*{liminf}_{m \to \infty} (4m^2)^{-1} \sum_{n=0}^\infty u(n, m) \geq 4^{-1} \int_0^\infty \rho(\theta^{-1/2}) d\theta
\]
由于正态密度在 0 处是正且连续的,
\[
\rho(c) = \int_{[-c, c]^2} n(x) dx \sim n(0) (2c)^2
\]
当 $c \to 0$ 时,故 $\rho(\theta^{-1/2}) \sim 4 n(0)/\theta$ 当 $\theta \to \infty$,该积分发散.回溯到证明首个不等式推出 $\sum_{n=0}^\infty u(n, 1) = \infty$,从而证毕.
    \end{prf}
    
    
    
    \begin{thm}
        [Theorem-on-Equivalent-Properties-of-Martingales]
        {鞅的等价性质定理}
        [Theorem on Equivalent Properties of Martingales]
        [gpt-4.1]
        对于一个鞅,以下条件是等价的:
(i) 它是一致可积的;
(ii) 它几乎处处收敛且在 $L^{1}$ 中收敛;
(iii) 它在 $L^{1}$ 中收敛;
(iv) 存在一个可积随机变量 $X$,使得 $X_{n} = E(X|\mathcal{F}_{n})$.
    \end{thm}
    
    
    
    \begin{thm}
        [Theorem-on-Probability-Bounds-for-Markov-Chain-Hitting-a-Set]
        {关于马尔可夫链进入集合的概率界定理}
        [Theorem on Probability Bounds for Markov Chain Hitting a Set]
        [gpt-4.1]
        设 $T _ { C } = \operatorname*{inf} \{ n \geq 1 : X _ { n } \in C \}$.假定 $S - C$ 是有限集,并且对每个 $x \in S - C$,有 $P _ { x } ( T _ { C } < \infty ) > 0$.则存在 $N < \infty$ 和 $\epsilon > 0$,使得对所有 $y \in S - C$,都有 $P _ { y } ( T _ { C } > k N ) \leq ( 1 - \epsilon ) ^ { k }$.
    \end{thm}
    
    
    
    \begin{prf}
        [Proof-Using-Theorem-4-and-Fatous-Lemma]
        {利用定理4和Fatou引理的证明}
        [Proof Using Theorem 4 and Fatou's Lemma]
        [gpt-4.1]
        
由定理4和Fatou引理有
\[
E X_{0} \geq \lim_{n \to \infty} \inf E X_{N \wedge n} \geq E X_{N}
\]

    \end{prf}
    
    
    
    \begin{thm}
        [Theorem-on-Intersection-of-$\sigma$-Algebras-of-Stopping-Times]
        {停止时刻的 $\sigma$-代数交集定理}
        [Theorem on Intersection of $\sigma$-Algebras of Stopping Times]
        [gpt-4.1]
        
设 $S$ 和 $T$ 是停止时刻,则有
\[
\mathcal{F}_S \cap \mathcal{F}_T = \mathcal{F}_{S \wedge T} .
\]

    \end{thm}
    
    
    
    \begin{dfn}
        [Definition-of-Benfords-Law]
        {Benford定律的定义}
        [Definition of Benford's Law]
        [gpt-4.1]
        
在集合 $\{1, \ldots, 9\}$ 上的极限分布称为 Benford(1938)定律,尽管它最早由 Newcomb(1881)发现.具体地,若 $\theta = \log_{10} 2$,$1 \leq k \leq 9$,$\boldsymbol{A}_{k} = [\log_{10} k, \log_{10} (k+1))$,则 $2^m$ 的首位数字为 $k$ 当且仅当 $m\theta \bmod 1 \in A_k$,且概率分布为
\[
P(k) = \log_{10} \left( \frac{k+1}{k} \right)
\]

    \end{dfn}
    
    
    
    \begin{thm}
        [Theorem-$\exp	heta-B-{t}---	heta^{2}-t-/-2$-is-a-Martingale]
        {关于$\exp(	heta B\_{t} - (	heta^{2} t / 2))$是鞅的定理}
        [Theorem: $\exp(	heta B_{t} - (	heta^{2} t / 2))$ is a Martingale]
        [gpt-4.1]
        $\exp(\theta B_{t} - (\theta^{2} t / 2))$ 是一个鞅.

    \end{thm}
    
    
    
    \begin{thm}
        [Extension-Theorem-for-Brownian-Motion-Process]
        {布朗运动过程的扩展定理}
        [Extension Theorem for Brownian Motion Process]
        [gpt-4.1]
        To extend $B_t$ to a process defined on $[0, \infty)$, we will show:

Theorem 7.
    \end{thm}
    
    
    
    \begin{ppt}
        [Property_of_$
    u_x$_on_Paths]
        {$
    u_x$ 关于路径的性质}
        [Property of $
    u_x$ on Paths]
        [gpt-4.1]
        Let $T < \infty$ and $x \in \mathbf{R}$.
$
u_x$ assigns probability one to paths $\omega : \mathbf{Q}_2 \to \mathbf{R}$ that are uniformly continuous on $\mathbf{Q}_2 \cap [0, T]$.
    \end{ppt}
    
    
    
    \begin{thm}
        [Limit-of-Sequence-of-Stopping-Times-is-a-Stopping-Time]
        {停止时序列极限仍为停止时}
        [Limit of Sequence of Stopping Times is a Stopping Time]
        [gpt-4.1]
        如果 $T_{n}$ 是一列停止时,且 $T_{n} \uparrow T$,则 $T$ 也是停止时.
    \end{thm}
    
    
    
    \begin{prf}
        [Proof-of-Limit-of-Sequence-of-Stopping-Times-is-a-Stopping-Time]
        {停止时序列极限仍为停止时的证明}
        [Proof of Limit of Sequence of Stopping Times is a Stopping Time]
        [gpt-4.1]
        $\{ T \leq t \} = \cap_{n} \{ T_{n} \leq t \}$.
    \end{prf}
    
    
    
    \begin{dfn}
        [Definition-of-Weak-Convergence-in-Metric-Space]
        {度量空间中弱收敛的定义}
        [Definition of Weak Convergence in Metric Space]
        [gpt-4.1]
        在一般度量空间 $(S, \rho)$ 中,称 $X_{n} \Rightarrow X_{\infty}$,当且仅当对所有有界连续函数 $f$,有 $E f (X_{n}) \to E f (X_{\infty})$.
    \end{dfn}
    
    
    
    \begin{thm}
        [Theorem-on-Lower-and-Upper-Limits-of-a-Sequence]
        {关于数列极限下界和上界的定理}
        [Theorem on Lower and Upper Limits of a Sequence]
        [gpt-4.1]
        
设 $\{a_n\}$ 是一个数列,且对于所有 $m, n$ 有 $a_{m} + a_{n-m} \geq a_{n}$,则有
\[
\frac{a_n}{n} \geq \inf_{m \geq 1} \frac{a_m}{m} \equiv \gamma
\]
对任意 $m$,有
\[
\frac{a_{n}}{n} \leq \frac{k m}{k m + \ell} \cdot \frac{a_{m}}{m} + \frac{a_{\ell}}{n}
\]
其中 $n = k m + \ell$, $0 \leq \ell < m$.当 $n \to \infty$ 时,得
\[
\limsup_{n \to \infty} \frac{a_n}{n} \leq \frac{a_m}{m}
\]
从而 $\liminf_{n \to \infty} \frac{a_n}{n} \geq \gamma$,$\limsup_{n \to \infty} \frac{a_n}{n} \leq \frac{a_m}{m}$ 对任意 $m$ 成立.

    \end{thm}
    
    
    
    \begin{dfn}
        [Natural-filtration-of-prior-information-for-Brownian-motion]
        {布朗运动的先前信息的自然过滤}
        [Natural filtration of prior information for Brownian motion]
        [gpt-4.1]
        
设 $B_r$ 是布朗运动,则
\[
\mathcal{F}_s^o = \sigma(B_r : r \le s)
\]
定义为时刻 $s$ 之前布朗运动的自然过滤.

    \end{dfn}
    
    
    
    \begin{dfn}
        [Right-continuous-filtration-for-Brownian-motion]
        {布朗运动的右连续过滤}
        [Right-continuous filtration for Brownian motion]
        [gpt-4.1]
        
设 $\mathcal{F}_t^o = \sigma(B_r : r \le t)$,则
\[
\mathcal{F}_s^+ = \cap_{t > s} \mathcal{F}_t^o
\]
定义为布朗运动在时刻 $s$ 的右连续过滤.

    \end{dfn}
    
    
    
    \begin{thm}
        [Application-of-the-π-λ-Theorem]
        {π-λ定理的应用}
        [Application of the π-λ Theorem]
        [gpt-4.1]
        
设对所有集合 $A$ 满足
\[
\int_A 1_{(X_{n+1} \in B)} dP_\mu = \int_A p(X_n, B) dP_\mu
\]
的集合族构成一个$\lambda$-系统,而已证明该等式成立的集合族构成一个$\pi$-系统,则根据$\pi$-$\lambda$定理(定理2.1.6),上述等式对于所有 $A \in \mathcal{F}_n$ 都成立.

    \end{thm}
    
    
    
    \begin{thm}
        [Markov-Property-of-Conditional-Probability]
        {条件概率的马尔可夫性质}
        [Markov Property of Conditional Probability]
        [gpt-4.1]
        
对于所有 $B$ 和 $\mathcal{F}_n$,有
\[
P(X_{n+1} \in B \mid \mathcal{F}_n) = p(X_n, B)
\]
成立.

    \end{thm}
    
    
    
    \begin{thm}
        [Proof-of-Theorem-6.4.1]
        {定理 6.4.1 的证明}
        [Proof of Theorem 6.4.1]
        [gpt-4.1]
        对于第二项 $\mathfrak{l} = E ( X_{0,\ell}^{+} / n )$,有 $\mathfrak{l} \to 0$ 当 $n \to \infty$.

对于第一项,注意到 $y^{+} \leq |y|$,根据遍历定理,

\[
E \left| \frac{ X(0, m) + \cdots + X((k-1)m, km) }{ k } - \Gamma_{m} \right| \to 0
\]

因此定理 6.4.1 的证明完成.

    \end{thm}
    
    
    
    \begin{thm}
        [Construction-of-a-New-Stopping-Time-from-a-Given-Stopping-Time-and-Set]
        {由给定停时和集合构造的新停时}
        [Construction of a New Stopping Time from a Given Stopping Time and Set]
        [gpt-4.1]
        设 $S$ 是一个停时,$A \in \mathcal{F}_{S}$,定义随机变量 $R$ 满足:在 $A$ 上 $R = S$,在 $A^{c}$ 上 $R = \infty$.则 $R$ 是一个停时.
    \end{thm}
    
    
    
    \begin{thm}
        [Application-of-Borel-Cantelli-Lemma-to-Continuity-Estimates-of-Stochastic-Processes]
        {Borel-Cantelli引理应用于随机过程的连续性估计}
        [Application of Borel-Cantelli Lemma to Continuity Estimates of Stochastic Processes]
        [gpt-4.1]
        
若 $\sum_{N=1}^{\infty} P(H_{N}^{c}) < \infty$,由Borel-Cantelli引理(Theorem 2.3.1)可得:

\[
|X(q) - X(r)| \leq A |q - r|^{\gamma} \quad \mathrm{for~} q, r \in \mathbf{Q}_{2} \mathrm{~with~} |q - r| < \delta(\omega)
\]

其中 $A, \gamma, \delta(\omega)$ 为相应常数.该结论用于控制随机过程在有理点间的连续性.

    \end{thm}
    
    
    
    \begin{thm}
        [Formula-for-Computing-Expectation-via-Conditional-Expectation]
        {关于条件期望的期望计算公式}
        [Formula for Computing Expectation via Conditional Expectation]
        [gpt-4.1]
        
对于随机变量 $T_{U,V}$,有
\[
E(T_{U,V}) = E\{ E( T_{U,V} | (U,V) ) \} = E( -UV )
\]

    \end{thm}
    
    
    
    \begin{thm}
        [Central-Limit-Theorem]
        {中心极限定理}
        [Central Limit Theorem]
        [gpt-4.1]
        在定理 8.1.2 的假设下,$S_n / \sqrt{n} \Rightarrow \chi$,其中 $\chi$ 服从标准正态分布.
    \end{thm}
    
    
    
    \begin{thm}
        [Probability-Property-of-Interval-Endpoints-as-Limits-of-Local-Maxima-for-Brownian-Motion]
        {区间端点是布朗运动局部极大值点极限的概率性质}
        [Probability Property of Interval Endpoints as Limits of Local Maxima for Brownian Motion]
        [gpt-4.1]
        若 $a < b$,则以概率1,$a$ 是 $B_t$ 在 $(a, b)$ 内局部极大值点的极限.
    \end{thm}
    
    
    
    \begin{thm}
        [Density-and-Countability-of-Local-Maxima-of-Brownian-Motion]
        {布朗运动局部极大值点的稠密性与可数性}
        [Density and Countability of Local Maxima of Brownian Motion]
        [gpt-4.1]
        布朗运动 $B_t$ 的局部极大值点的集合几乎必然是稠密集,但与零点集合不同,它是可数的.
    \end{thm}
    
    
    
    \begin{thm}
        [Theorem-on-Lim-Sup-and-Linear-Recursion]
        {关于极限上确界与线性递推的定理}
        [Theorem on Lim Sup and Linear Recursion]
        [gpt-4.1]
        如果 $\limsup_{n \to \infty} g_n \le x$, 那么 $( { \star } )$ 蕴含 $\limsup_{n \to \infty} D_n / n \le x$,并且(由于 $a > b$)有

\[
\lim_{n \to \infty} \sup_{\substack{g_{n+1} \leq \frac{a x + b ( 1 - x )}{a + b}}} = \frac{b + (a - b)x}{a + b}
\]

右侧是斜率小于 $1$,不动点为 $1/2$ 的线性函数,因此从上界 $x=1$ 开始迭代得出 $\limsup g_n \leq 1 / 2$.互换 red 和 green 的角色可得 $\liminf_{n \to \infty} g_n \geq 1 / 2$,于是结论成立.

    \end{thm}
    
    
    
    \begin{thm}
        [Theorem-on-Hölder-Continuity-of-Brownian-Motion]
        {关于布朗运动Hölder连续性的定理}
        [Theorem on Hölder Continuity of Brownian Motion]
        [gpt-4.1]
        如果 $\gamma > 5/6$,则在区间 $[0,1]$ 的任一点处,$B_t$ 都不是指数为 $\gamma$ 的 Hölder 连续函数.
    \end{thm}
    
    
    
    \begin{thm}
        [Generalized-Theorem-on-Hölder-Continuity-of-Brownian-Motion]
        {布朗运动Hölder连续性的推广定理}
        [Generalized Theorem on Hölder Continuity of Brownian Motion]
        [gpt-4.1]
        对所有 $\gamma > 1/2 + 1/k$,在区间 $[0,1]$ 的任一点处,$B_t$ 都不是指数为 $\gamma$ 的 Hölder 连续函数.
    \end{thm}
    
    
    
    \begin{thm}
        [Theorem-Probability-of-Extinction-or-Infinity-in-Branching-Process-is-1]
        {分支过程灭绝或无穷概率为1的定理}
        [Theorem: Probability of Extinction or Infinity in Branching Process is 1]
        [gpt-4.1]
        设 $Z_{n}$ 是一分支过程,其后代分布为 $p_{k}$(见第5.3节末尾的定义).若 $p_{0} > 0$,则有
\[
P\left( \lim_{n} Z_{n} = 0 \text{ 或 } \infty \right) = 1.
\]

    \end{thm}
    
    
    
    \begin{thm}
        [Markov-Property-Theorem]
        {马尔可夫性定理}
        [Markov Property Theorem]
        [gpt-4.1]
        如果 $s \geq 0$,且 $Y$ 是有界且 $\mathcal{C}$-可测的,则对于所有 $\boldsymbol{x} \in \mathbf{R}^d$ 有
\[
E_x (Y \circ \theta_s \mid \mathcal{F}_s^{+}) = E_{B_s} Y
\]
其中右侧是函数 $\varphi(x) = E_x Y$ 在 $x = B_s$ 处的取值.

    \end{thm}
    
    
    
    \begin{prf}
        [Proof_that_the_Set_$
    u$_is_a_Group]
        {关于集合 $
    u$ 是群的证明}
        [Proof that the Set $
    u$ is a Group]
        [gpt-4.1]
        Suppose $
u 
eq \varnothing$.
It is clear that $\mathcal{V}^c$ is open, so $
u$ is closed.
To prove that $
u$ is a group, we will first show that

$(*)$ if $x \in \mathcal{U}$ and $y \in \mathcal{V}$, then $y - x \in \mathcal{V}$.

This statement has been formulated so that once it is established, the result follows easily.
Let
\[
p_{\delta, m}(z) = P(\| S_n - z \| \geq \delta \text{ for all } n \geq m)
\]
If $y - x 
otin \mathcal{V}$, there is an $\epsilon > 0$ and $m \geq 1$ so that $p_{2\epsilon, m}(y - x) > 0$.
Since $x \in \mathcal{U}$, there is a $k$ so that $P(\| S_k - x \| < \epsilon) > 0$.
Since
\[
P(\| S_n - S_k - (y - x) \| \geq 2\epsilon \text{ for all } n \geq k + m) = p_{2\epsilon, m}(y - x)
\]
and is independent of $\{ \| S_k - x \| < \epsilon \}$, it follows that
\[
p_{\epsilon, m + k}(y) \geq P(\| S_k - x \| < \epsilon) p_{2\epsilon, m}(y - x) > 0
\]
contradicting $y \in \mathcal{V}$, so $y - x \in \mathcal{V}$.

To conclude $
u$ is a group when $
u 
eq \varnothing$, let $q, r \in \mathcal{V}$, and observe: (i) taking $x = y = r$ in $(*)$ shows $0 \in \mathcal{V}$, (ii) taking $x = r$, $y = 0$ shows $-r \in \mathcal{V}$, and (iii) taking $x = -r$, $y = q$ shows $q + r \in \mathcal{V}$.

To prove that $
u = \mathcal{U}$ now, observe that if $u \in \mathcal{U}$ taking $x = u$, $y = 0$ shows $-u \in \mathcal{V}$ and since $
u$ is a group, it follows that $u \in \mathcal{V}$.

    \end{prf}
    
    
    
    \begin{dfn}
        [Definition-of-Transient-and-Recurrent-Random-Walks]
        {游走的遍历性与常返性定义}
        [Definition of Transient and Recurrent Random Walks]
        [gpt-4.1]
        如果 $
u = \varnothing$, 则称该随机游走是遍历(transient)的; 否则称为常返(recurrent)的.
    \end{dfn}
    
    
    
    \begin{thm}
        [Theorem-on-Uniform-Integrability-of-Stopped-Submartingale]
        {子马氏链的停时截断与一致可积性的定理}
        [Theorem on Uniform Integrability of Stopped Submartingale]
        [gpt-4.1]
        
设 $X_{n}$ 是一个子马氏链,且对所有 $n$ 有 $E\left( | X_{n+1} - X_{n} | \mid {\mathcal{F}}_{n} \right) \leq B$.若 $N$ 是一个停时,且 $E N < \infty$,则 $X_{N \wedge n}$ 一致可积,因此 $E X_{N} \ge E X_{0}$.

    \end{thm}
    
    
    
    \begin{dfn}
        [Definition-of-Partial-Derivative-Notation]
        {偏导数符号的定义}
        [Definition of Partial Derivative Notation]
        [gpt-4.1]
        为了简化记号,我们记 $D_{i} f = \partial f / \partial x_{i}$,$D_{ij} f = \partial^{2} f / \partial x_{i} \partial x_{j}$.
    \end{dfn}
    
    
    
    \begin{thm}
        [Theorem-on-Integral-Process-Being-a-Continuous-Martingale]
        {关于积分过程是连续鞅的定理}
        [Theorem on Integral Process Being a Continuous Martingale]
        [gpt-4.1]
        若 $g$ 连续,且 $E \int_0^t |f'(B_s)|^2 ds < \infty$,则
\[
\int_0^t f'(B_s)\, dB_s
\]
是一个连续鞅.
    \end{thm}
    
    
    
    \begin{xmp}
        [Example-of-Function-Substitution-on-Brownian-Motion]
        {关于布朗运动上函数替换的例子}
        [Example of Function Substitution on Brownian Motion]
        [gpt-4.1]
        公式对 $f_{M}$ 成立,但在集合 $\{\sup_{s \leq t} |B_{s}| \leq M\}$ 上,使用 $f_{M}$ 和 $f$ 在公式中没有区别.
    \end{xmp}
    
    
    
    \begin{thm}
        [Joint-Density-Formula-for-Maximum-and-Terminal-Value]
        {最大值与终值的联合密度公式}
        [Joint Density Formula for Maximum and Terminal Value]
        [gpt-4.1]
        
利用 $\{ T_{a} < t \} = \{ M_{t} > a \}$,对 $a$ 求导得到最大值 $M_{t}$ 和终值 $B_{t}$ 的联合密度公式:
\[
f_{( M_{t}, B_{t} )}( a, x ) = \frac{ 2 ( 2a - x ) }{ \sqrt{ 2 \pi t^{3} } } e^{ - ( 2a - x )^{2} / 2t }
\]

    \end{thm}
    
    
    
    \begin{thm}
        [Theorem-Measurable-Function-of-Random-Variables-Is-a-Random-Variable]
        {可测函数作用于随机变量仍为随机变量的定理}
        [Theorem: Measurable Function of Random Variables Is a Random Variable]
        [gpt-4.1]
        如果 $X_{1}, \dots, X_{n}$ 是随机变量,且 $f : (\mathbf{R}^{n}, \mathcal{R}^{n}) \to (\mathbf{R}, \mathcal{R})$ 是可测函数,则 $f(X_{1}, \dots, X_{n})$ 也是随机变量.
    \end{thm}
    
    
    
    \begin{dfn}
        [Definition-of-Hitting-Time-for-Set-A-and-Its-Expected-Value]
        {首次进入集合A的停时和期望停时的定义}
        [Definition of Hitting Time for Set A and Its Expected Value]
        [gpt-4.1]
        设 $V_{A} = \operatorname*{inf} \{n \geq 0 : X_{n} \in A\}$,$g(x) = E_{x} V_{A}$.
    \end{dfn}
    
    
    
    \begin{thm}
        [Equation-Satisfied-by-Expected-Hitting-Time]
        {期望首次进入A的停时满足的方程}
        [Equation Satisfied by Expected Hitting Time]
        [gpt-4.1]
        若 $S - A$ 有限,且对每个 $x \in S - A$,$P_{x}(V_{A} < \infty) > 0$,则
\[
g(x) = 1 + \sum_{y} p(x, y) g(y) \quad \text{对于}~ x 
otin A
\]
其中 $g(x) = E_x V_A$.
    \end{thm}
    
    
    
    \begin{thm}
        [Martingale-Constructed-from-Solution-of-the-Equation]
        {关于g满足方程时构造的鞅}
        [Martingale Constructed from Solution of the Equation]
        [gpt-4.1]
        若 $g$ 满足 $(*)$,则 $g(X(n \land V_A)) + (n \land V_A)$ 是一个鞅.
    \end{thm}
    
    
    
    \begin{thm}
        [Uniqueness-of-Expected-Hitting-Time-as-the-Solution]
        {期望首次进入A的停时的唯一性}
        [Uniqueness of Expected Hitting Time as the Solution]
        [gpt-4.1]
        $g(x) = E_{x} \tau_{A}$ 是在 $A$ 上为0的 $(*)$ 唯一解.
    \end{thm}
    
    
    
    \begin{thm}
        [Recurrence-of-One-Dimensional-Brownian-Motion]
        {一维布朗运动的常返性}
        [Recurrence of One-Dimensional Brownian Motion]
        [gpt-4.1]
        
设 $B_t$ 是一维布朗运动,令 $A = \cap_{n} \{ B_t = 0 \text{ for some } t \geq n \}$.则对于任意初始点 $x$,都有 $P_x(A) = 1$.

也就是说,一维布朗运动是常返的:无论从哪个初始点 $x$ 出发,布朗运动都会'无限多次'返回到0,即存在一列时刻 $t_n \uparrow \infty$ 使得 $B_{t_n} = 0$.

    \end{thm}
    
    
    
    \begin{dfn}
        [Definition-of-Birth-Time-and-Offspring-Time-Lag-in-Generations]
        {世代成员出生时间与后代时间间隔的定义}
        [Definition of Birth Time and Offspring Time Lag in Generations]
        [gpt-4.1]
        设 $p_{0} = 0$,令 $X_{0, m}$ 为第 $m$ 代第一个成员的出生时间,$X_{m, n}$ 为该个体在第 $n$ 代有后代所需的时间间隔.
    \end{dfn}
    
    
    
    \begin{ppt}
        [Random-Selection-Rule-for-Birth-Time-in-Generations]
        {世代成员出生时间的随机选择规则}
        [Random Selection Rule for Birth Time in Generations]
        [gpt-4.1]
        若出现出生时间相同的情况,则从第 $m$ 代中出生于时间 $X_{0, m}$ 的那些个体中随机选取一个.
    \end{ppt}
    
    
    
    \begin{ppt}
        [Relationship-between-Birth-Time-and-Offspring-Time-Lag-across-Generations]
        {世代出生时间与后代时间间隔的关系}
        [Relationship between Birth Time and Offspring Time Lag across Generations]
        [gpt-4.1]
        显然有 $X_{0, n} \leq X_{0, m} + X_{m, n}$.
    \end{ppt}
    
    
    
    \begin{thm}
        [Brownian-Scaling-Relation]
        {布朗运动的缩放关系}
        [Brownian Scaling Relation]
        [gpt-4.1]
        
若 $B_0 = 0$,则对于任意 $t > 0$,
\[
\{B_{st},\, s \geq 0\} \stackrel{d}{=} \{t^{1/2} B_s,\, s \geq 0\}
\]
更精确地,两组随机过程具有相同的有限维分布,即若 $s_1 < \ldots < s_n$,则
\[
(B_{s_1 t}, \ldots, B_{s_n t}) \stackrel{d}{=} (t^{1/2} B_{s_1}, \ldots, t^{1/2} B_{s_n})
\]

    \end{thm}
    
    
    
    \begin{prf}
        [Proof-of-Brownian-Scaling-Relation]
        {布朗运动缩放关系的证明}
        [Proof of Brownian Scaling Relation]
        [gpt-4.1]
        
当 $n = 1$ 时,$t^{1/2}$ 乘以一个均值为 0、方差为 $s$ 的正态分布,得到一个均值为 0、方差为 $st$ 的正态分布.对于 $n > 1$,结果由独立增量性质推出.

    \end{prf}
    
    
    
    \begin{dfn}
        [Definition-of-One-dimensional-Brownian-Motion]
        {一维布朗运动的定义}
        [Definition of One-dimensional Brownian Motion]
        [gpt-4.1]
        一维布朗运动是一个实值过程 $B_t$, $t \geq 0$,满足以下性质:
(a) 如果 $t_0 < t_1 < \ldots < t_n$,则 $B(t_0), B(t_1) - B(t_0), \ldots, B(t_n) - B(t_{n-1})$ 彼此独立;
(b) 若 $s, t \geq 0$,则
\[
P(B(s+t) - B(s) \in A) = \int_A (2\pi t)^{-1/2} \exp(-x^2/2t) dx
\]
(c) 几乎必然地,映射 $t \mapsto B_t$ 是连续的.
    \end{dfn}
    
    
    
    \begin{thm}
        [Theorem-on-Limiting-Event-Probability-for-Cyclic-Classes-in-Markov-Chains]
        {关于马尔可夫链循环类的极限事件概率定理}
        [Theorem on Limiting Event Probability for Cyclic Classes in Markov Chains]
        [gpt-4.1]
        若初始分布集中在某一循环类 $S_0$ 上,设 $A \in \mathcal{T}$,则对于每个 $x$,$P_x(A) \in \{0, 1\}$.若对所有 $x \in S_0$ 有 $P_x(A) = 0$,则 $P_\mu(A) = 0$.若对某些 $y \in S_0$ 有 $P_y(A) = 1$,则对任意 $z \in S_0$,均有 $P_z(A) = 1$,进而 $P_\mu(A) = 1$.
    \end{thm}
    
    
    
    \begin{thm}
        [General-Theorem-on-Limiting-Event-Probability-for-Cyclic-Classes]
        {一般循环类上的极限事件概率定理}
        [General Theorem on Limiting Event Probability for Cyclic Classes]
        [gpt-4.1]
        对于马尔可夫链的每个循环类 $S_r$,对于任意极限事件 $A \in \mathcal{T}$,要么 $\{ X_0 \in S_r \} \subset A$,要么 $\{ X_0 \in S_r \} \cap A = \varnothing$,且 $P(A \mid X_0 = y) \equiv 0$ 或 $1$ 在每个循环类上成立.
    \end{thm}
    
    
    
    \begin{dfn}
        [Definition-of-First-Hitting-Time-of-Zero-for-Brownian-Motion]
        {布朗运动首次归零时间的定义}
        [Definition of First Hitting Time of Zero for Brownian Motion]
        [gpt-4.1]
        
设 $\mathcal{T}_0 = \inf\{ s > 0 : B_s = 0 \}$,即 $\mathcal{T}_0$ 为布朗运动 $B_s$ 首次达到 $0$ 的时间.

    \end{dfn}
    
    
    
    \begin{dfn}
        [Definition-of-First-Hitting-Time-of-Zero-after-$t>1$-for-Brownian-Motion]
        {布朗运动在 $t>1$ 时首次归零时间的定义}
        [Definition of First Hitting Time of Zero after $t>1$ for Brownian Motion]
        [gpt-4.1]
        
设 $R = \inf\{ t > 1 : B_t = 0 \}$,即 $R$ 为布朗运动 $B_t$ 在 $t>1$ 时首次达到 $0$ 的时间.

    \end{dfn}
    
    
    
    \begin{lma}
        [Lemma-on-Lower-Bound-of-Real-Part-Integral]
        {关于实部积分下界的引理引用}
        [Lemma on Lower Bound of Real Part Integral]
        [gpt-4.1]
        
根据 Lemma 5,有
\[
\operatorname{Re}(1 - \varphi(t)) = \int \{1 - \cos(x \cdot t)\} \mu(dx) \geq \int_{|x \cdot t| < \pi/3} \frac{|x \cdot t|^{2}}{4} \mu(dx)
\]

    \end{lma}
    
    
    
    \begin{thm}
        [Theorem-on-the-Probability-of-Sums-Being-Zero-Multiple-Times]
        {关于和过程为零的次数的概率定理}
        [Theorem on the Probability of Sums Being Zero Multiple Times]
        [gpt-4.1]
        已知Stationarity implies $P(G_{j,k}) = P(F_{k})$, and for fixed $j$ the $G_{j,k}$ are disjoint, so $\cup_{k} G_{j,k} = \Omega$ a. It follows that
\[
\sum_{k} P(F_{j} \cap G_{j,k}) = P(F_{j}) \quad \text{and} \quad \sum_{j,k} P(F_{j} \cap G_{j,k}) = 1
\]
在$F_{j} \cap G_{j,k}$上, $S_{j} = 0$ 且 $S_{j+k} = 0$, 因此我们证明了 $P(S_{n} = 0 \text{ 至少两次}) = 1$.重复上述论证可得 $P(S_{n} = 0 \text{ 至少 } k \text{ 次}) = 1$ 对任意 $k$ 都成立, 从而证明结束.
    \end{thm}
    
    
    
    \begin{thm}
        [Conditional-Expectation-Inequality-under-Convex-Function]
        {凸函数条件下的条件期望不等式}
        [Conditional Expectation Inequality under Convex Function]
        [gpt-4.1]
        
设 $\varphi$ 是凸函数,且 $E|X|$, $E|\varphi(X)| < \infty$,则有
\[
\varphi ( E ( X | { \mathcal { F } } ) ) \leq E ( \varphi ( X ) | { \mathcal { F } } )
\]

    \end{thm}
    
    
    
    \begin{prf}
        [Proof-of-Conditional-Expectation-Inequality-under-Convex-Function]
        {凸函数条件下的条件期望不等式的证明}
        [Proof of Conditional Expectation Inequality under Convex Function]
        [gpt-4.1]
        
若 $\varphi$ 是线性的,结论显然成立,故假设 $\varphi$ 非线性.令 $S = \{ (a, b) : a, b \in \mathbf{Q}, a x + b \leq \varphi(x) \text{ 对所有 } x \}$,则 $\varphi(x) = \sup\{a x + b : (a, b) \in S\}$.若 $\varphi(x) \geq a x + b$,则 (4.1.2) 和 (4.1.1) 推出
\[
E ( \varphi ( X ) | { \mathcal { F } } ) \geq a E ( X | { \mathcal { F } } ) + b \quad \mathrm{a.s.}
\]
对 $(a, b) \in S$ 取上确界,得
\[
E ( \varphi ( X ) | { \mathcal { F } } ) \geq \varphi ( E ( X | { \mathcal { F } } ) ) \quad \mathrm{a.s.}
\]
从而证明了所需结果.

    \end{prf}
    
    
    
    \begin{dfn}
        [Definition-of-$F-{\sigma}$-Set]
        {$F\_{\sigma}$集的定义}
        [Definition of $F_{\sigma}$ Set]
        [gpt-4.1]
        设 $A$ 是一个 $F_{\sigma}$ 集,即可表示为可数个闭集的并集.
    \end{dfn}
    
    
    
    \begin{thm}
        [Proof-that-$T-A$-is-a-Stopping-Time]
        {$T\_A$是停时的证明}
        [Proof that $T_A$ is a Stopping Time]
        [gpt-4.1]
        设 $A$ 是一个 $F_{\sigma}$ 集,即可表示为可数个闭集的并集.证明 $T_A = \inf \{ t : B_t \in A \}$ 是一个停时.
    \end{thm}
    
    
    
    \begin{thm}
        [Probability-Density-Formula-for-the-First-Return-to-Zero-of-Brownian-Motion]
        {关于布朗运动首次回到零点的概率密度公式}
        [Probability Density Formula for the First Return to Zero of Brownian Motion]
        [gpt-4.1]
        
设 $R = \operatorname*{inf} \{ t > 1 : B_{t} = 0 \}$,其中 $B_t$ 表示标准布朗运动,则 $R$ 的概率密度为
\[
P_{0} ( R = 1 + t ) = \frac{1}{\pi t^{1/2} ( 1 + t )}
\]

    \end{thm}
    
    
    
    \begin{crl}
        [Hausdorff-Dimension-of-the-Zero-Set-is-at-Most-1/2]
        {零点集的 Hausdorff 维数至多为 1/2}
        [Hausdorff Dimension of the Zero Set is at Most 1/2]
        [gpt-4.1]
        由上述分析可得,若将区间 $[1,2]$ 分为 $n$ 等分,则包含零点的区间的期望数目 $\leq C n^{-1/2}$.由此推出零点集的 Hausdorff 维数至多为 $1/2$.
    \end{crl}
    
    
    
    \begin{thm}
        [Necessary-and-Sufficient-Condition-for-$fB-t$-to-be-a-Martingale]
        {关于 $f(B\_t)$ 是鞅的充要条件}
        [Necessary and Sufficient Condition for $f(B_t)$ to be a Martingale]
        [gpt-4.1]
        设 $f \in C^2$.则 $f(B_t)$ 是鞅当且仅当 $f(x)$ 形如 $a + bx$,其中 $a, b \in \mathbb{R}$.
    \end{thm}
    
    
    
    \begin{thm}
        [Uniqueness-Theorem-for-Stationary-Measures-under-Recurrence]
        {遍历测度唯一性定理}
        [Uniqueness Theorem for Stationary Measures under Recurrence]
        [gpt-4.1]
        
假设 $p$ 是遍历的.如果 $
u$ 是一个 $\sigma$-有限的不变测度,则 $
u = \bar{
u}(\alpha) \mu$,其中 $\mu$ 是在定理 5.8.9 的证明中构造的测度.

    \end{thm}
    
    
    
    \begin{prf}
        [Proof-of-Uniqueness-Theorem-for-Stationary-Measures-under-Recurrence]
        {遍历测度唯一性定理的证明}
        [Proof of Uniqueness Theorem for Stationary Measures under Recurrence]
        [gpt-4.1]
        
由引理 5.8.10,只需证明:如果 $\bar{
u}$ 是 $\bar{p}$ 的不变测度且 $\bar{
u}(\alpha) < \infty$,则 $\bar{
u} = \bar{
u}(\alpha) \bar{\mu}$.重复定理 5.5.9 的证明(取 $a = \alpha$),容易证明 $\bar{
u}(C) \geq \bar{
u}(\alpha) \bar{\mu}(C)$.继续像该证明那样计算.

    \end{prf}
    
    
    
    \begin{prf}
        [Proof-of-the-Probability-that-Brownian-Motion-First-Hits-Zero-at-Time-Zero]
        {关于布朗运动首次达到零时刻概率的证明}
        [Proof of the Probability that Brownian Motion First Hits Zero at Time Zero]
        [gpt-4.1]
        证明 $P_{0} ( \tau \le t ) \ge P_{0} ( B_{t} > 0 ) = 1 / 2$, 因为正态分布关于0是对称的.

令 $t \downarrow 0$,则有
\[
P_{0} ( \tau = 0 ) = \lim_{t \downarrow 0} P_{0} ( \tau \leq t ) \geq 1 / 2
\]

所以由定理 7.3 得 $P_{0} ( \tau = 0 ) = 1$.

    \end{prf}
    
    
    
    \begin{thm}
        [Construction-of-Stationary-Measure-for-Closed-Sets-of-Recurrent-States]
        {构造遍历状态闭集的平稳测度}
        [Construction of Stationary Measure for Closed Sets of Recurrent States]
        [gpt-4.1]
        定理 5.7 允许我们为每一个遍历状态的闭集构造一个平稳测度.
    \end{thm}
    
    
    
    \begin{dfn}
        [Definition-of-Range-of-Random-Walk]
        {随机游走的区间范围定义}
        [Definition of Range of Random Walk]
        [gpt-4.1]
        设 $\xi_{1}, \xi_{2}, \ldots$ 是一个平稳序列,令 $S_{n} = \xi_{1} + \cdots + \xi_{n}$.令
\[
X_{m, n} = | \{ S_{m+1}, \ldots, S_{n} \} |,
\]
其中 $X_{m, n}$ 表示从第 $m+1$ 到第 $n$ 步的和的集合的势(即不同取值的个数).

    \end{dfn}
    
    
    
    \begin{ppt}
        [Basic-Inequality-for-Range-of-Random-Walk]
        {随机游走区间范围的基本不等式}
        [Basic Inequality for Range of Random Walk]
        [gpt-4.1]
        对于上述定义的 $X_{m, n}$,有
\[
X_{0, m} + X_{m, n} \geq X_{0, n}.
\]
并且
\[
0 \leq X_{0, n} \leq n.
\]

    \end{ppt}
    
    
    
    \begin{thm}
        [Donskers-Theorem]
        {Donsker定理}
        [Donsker's Theorem]
        [gpt-4.1]
        
在定理8.1.2的假设下,
\[
S(n\cdot)/\sqrt{n} \Rightarrow B(\cdot),
\]
即,相应的测度在 $C[0, 1]$ 上弱收敛.

    \end{thm}
    
    
    
    \begin{thm}
        [Recurrence-Criterion-for-Symmetric-Stable-Law-Random-Walks]
        {对称稳定律随机游走的遍历性判别}
        [Recurrence Criterion for Symmetric Stable Law Random Walks]
        [gpt-4.1]
        
设随机游走的步长分布的特征函数为 $\varphi(t) = \exp(-|t|^{\alpha})$,则对应的随机游走在 $\alpha < 1$ 时是暂态的(transient),在 $\alpha \geq 1$ 时是遍历的(recurrent).

    \end{thm}
    
    
    
    \begin{thm}
        [Hitting-Time-of-Open-Set-is-a-Stopping-Time]
        {开放集首次进入时间是停时}
        [Hitting Time of Open Set is a Stopping Time]
        [gpt-4.1]
        设 $G$ 是一个开集,$T = \operatorname*{inf} \{ t \geq 0 : B_{t} \in G \}$,则 $T$ 是一个停时.
    \end{thm}
    
    
    
    \begin{prf}
        [Proof-that-Hitting-Time-of-Open-Set-is-a-Stopping-Time]
        {开放集首次进入时间是停时的证明}
        [Proof that Hitting Time of Open Set is a Stopping Time]
        [gpt-4.1]
        因为 $G$ 是开集且 $t \mapsto B_{t}$ 是连续的,$\{ T < t \} = \cup_{q < t} \{ B_{q} \in G \}$,其中并是对所有有理数 $q$ 取的,所以 $\{ T < t \} \in \mathcal{F}_{t}$.这里需要用有理数以得到可数并,从而得到可测集.
    \end{prf}
    
    
    
    \begin{thm}
        [Continuity-and-Martingale-Property-of-Stochastic-Integral-with-Respect-to-Brownian-Motion]
        {关于布朗运动的随机积分的连续性与鞅性质}
        [Continuity and Martingale Property of Stochastic Integral with Respect to Brownian Motion]
        [gpt-4.1]
        
若 $g$ 连续且
\[
E\int_0^t |g(B_s)|^2 ds < \infty,
\]
则
\[
\int_0^t g(B_s)\, dB_s
\]
是一个连续鞅.

    \end{thm}
    
    
    
    \begin{lma}
        [Asymptotic-Behavior-of-$	aun$-in-Poisson-Process-and-Random-Permutation]
        {Poisson过程与随机排列中的$	au(n)$的渐近行为}
        [Asymptotic Behavior of $	au(n)$ in Poisson Process and Random Permutation]
        [gpt-4.1]
        
设 $\tau(n)$ 是 $t$ 的最小值,使得区间 $R_{0, t}$ 中有 $n$ 个点.记 $S_n$ 为区间 $R_{0, \sqrt{n}}$ 中的点数,则
\[
\tau(n) / \sqrt{n} \to 1 \text{ a.s.}
\]
证明:$S_n - S_{n-1}$ 是均值为 $1$ 的独立 Poisson 随机变量,因此强大数定律给出 $S_n / n \to 1$ 几乎必然成立.若 $\epsilon > 0$,则对于大的 $n$,有 $S_{n (1 - \epsilon)} < n < S_{n (1 + \epsilon)}$,因此
\[
\sqrt{(1 - \epsilon)n} \leq \tau(n) \leq \sqrt{(1 + \epsilon)n}
\]
几乎必然成立.

    \end{lma}
    
    
    
    \begin{thm}
        [Limit-Theorem-for-Conditional-Expectation]
        {条件期望极限的收敛性定理}
        [Limit Theorem for Conditional Expectation]
        [gpt-4.1]
        
设 $Y_{n} \to Y_{-\infty}$ 几乎处处 (a.s.),且 $|Y_{n}| \leq Z$ 几乎处处,其中 $E Z < \infty$.若滤子 $\mathcal{F}_{n} \downarrow \mathcal{F}_{-\infty}$,则有
\[
E(Y_{n} | \mathcal{F}_{n}) \to E(Y_{-\infty} | \mathcal{F}_{-\infty}) \quad \text{几乎处处}.
\]

    \end{thm}
    
    
    
    \begin{xmp}
        [An-Example-of-Markov-Chain-for-Finite-Discrete-Random-Variables]
        {关于有限离散随机变量的马尔可夫链例子}
        [An Example of Markov Chain for Finite Discrete Random Variables]
        [gpt-4.1]
        
设 $\xi_{1}, \xi_{2}, \ldots$ 是一列独立同分布(i.i.d.)的随机变量,取值于有限集合 $\{1, 2, \ldots, N\}$,且每个值出现的概率均为 $1/N$.令 $X_{n} = \left|\left\{\xi_{1}, \ldots, \xi_{n}\right\}\right|$ 表示前 $n$ 个随机变量所取值的不同元素个数.证明 $X_{n}$ 是一个马尔可夫链,并计算其转移概率.

    \end{xmp}
    
    
    
    \begin{thm}
        [Theorem-that-$B-t^3---3tB-t$-and-$B-t^3---\int-0^t-3-B-s-ds$-are-martingales]
        {关于$B\_t^3 - 3tB\_t$和$B\_t^3 - \int\_0^t 3 B\_s ds$为鞅的定理}
        [Theorem that $B_t^3 - 3tB_t$ and $B_t^3 - \int_0^t 3 B_s ds$ are martingales]
        [gpt-4.1]
        
$B_t^3 - 3tB_t$ 和 $B_t^3 - \int_0^t 3 B_s ds$ 都是鞅(martingale).

    \end{thm}
    
    
    
    \begin{xmp}
        [Example-of-Ergodicity-for-i.i.d.-Sequences]
        {独立同分布序列的遍历性例子}
        [Example of Ergodicity for i.i.d. Sequences]
        [gpt-4.1]
        
如果 $\boldsymbol{\Omega} = \mathbf{R}^{\{0, 1, \ldots\}}$ 且 $\varphi$ 是移位算子,则不变集 $A$ 满足 $\{\omega : \omega \in A\} = \{\omega : \varphi \omega \in A\} \in \sigma(X_{1}, X_{2}, \ldots)$.

反复作用移位算子得到
\[
A \in \cap_{n=1}^{\infty} \sigma(X_{n}, X_{n+1}, \ldots) = \mathcal{T}, \quad \text{即尾部}\sigma\text{-域}
\]

对于独立同分布序列,Kolmogorov 的 0-1 定律蕴含 $\mathcal{T}$ 是平凡的,因此 $\mathcal{T}$ 是平凡的,序列是遍历的(即,当序列空间 $\Omega = \mathbf{R}^{\{0, 1, 2, \dots\}}$ 上赋予相应的概率测度时,移位算子是遍历的).

    \end{xmp}
    
    
    
    \begin{thm}
        [Extension-of-Kacs-Theorem-for-Stationary-Sequences]
        {Kac定理关于遍历过程的推广}
        [Extension of Kac's Theorem for Stationary Sequences]
        [gpt-4.1]
        
设 $X_{0}, X_{1}, \ldots$ 是取值于 $(S, S)$ 的平稳序列.令 $A \in {\mathcal{S}}$,$T_{0} = 0$,对于 $n \geq 1$,定义 $T_{n} = \operatorname{inf} \{ m > T_{n-1} : X_{m} \in A \}$,即$T_n$为第$n$次返回集合$A$的时刻.则根据Kac(1947b)的结果,$T_n$的性质可由定理6.3.2(ii)的证明方法推广得到.

    \end{thm}
    
    
    
    \begin{dfn}
        [Definition-of-the-Piecewise-Linear-Function-$Su$]
        {分段线性函数 $S(u)$ 的定义}
        [Definition of the Piecewise Linear Function $S(u)$]
        [gpt-4.1]
        
设 $\mathbf{N}$ 为非负整数集,定义函数 $S(u)$ 如下:

\[
S(u) = \left\{
\begin{array}{cl}
S_{k} & \text{若 } u = k \in \mathbf{N} \\
\text{在 } [k, k + 1] \text{ 上为线性} & \text{对任意 } k \in \mathbf{N}
\end{array}
\right.
\]

    \end{dfn}
    
    
    
    \begin{thm}
        [Non-reversible-Probability-Implies-Transience]
        {不可逆概率导致状态的暂态性}
        [Non-reversible Probability Implies Transience]
        [gpt-4.1]
        
若 $\rho_{ij} > 0$ 而 $\rho_{ji} = 0$,则状态 $i$ 必为暂态,否则将与定理 5.3.2 矛盾.

    \end{thm}
    
    
    
    \begin{thm}
        [Non-reversible-Probability-Implies-Transience-Another-Example]
        {不可逆概率导致状态的暂态性(另一例)}
        [Non-reversible Probability Implies Transience (Another Example)]
        [gpt-4.1]
        
若 $\rho_{34} > 0$ 而 $\rho_{43} = 0$,则状态 $3$ 必为暂态.

    \end{thm}
    
    
    
    \begin{thm}
        [Recurrence-of-Irreducible-Closed-Sets]
        {不可约闭集的递归性}
        [Recurrence of Irreducible Closed Sets]
        [gpt-4.1]
        
若集合 $\{1,5\}$ 和 $\{4,6,7\}$ 是不可约闭集,则依据相关定理,这些集内的状态是递归状态.

    \end{thm}
    
    
    
    \begin{dfn}
        [Definition-of-Local-Martingale]
        {局部鞅的定义}
        [Definition of Local Martingale]
        [gpt-4.1]
        设 $M_t$ 是一个随机过程.如果存在一个递增的停时序列 $T_n \uparrow \infty$,使得对所有 $n$,过程 $M(t \wedge T_n)$ 是鞅,则称 $M_t$ 是局部鞅.
    \end{dfn}
    
    
    
    \begin{thm}
        [Theorem-on-Sigma-Algebra-Representation-for-Random-Walk]
        {随机游走的σ代数表示定理}
        [Theorem on Sigma-Algebra Representation for Random Walk]
        [gpt-4.1]
        对于 $d$ 维简单随机游走,

\[
\mathcal{T} = \sigma(\{ X_{0} \in L_{i} \}, i = 0, 1)
\]

即 $\mathcal{T}$ 由初始点 $X_{0}$ 是否属于集合 $L_{0}, L_{1}$ 生成的σ代数所刻画.
    \end{thm}
    
    
    
    \begin{prf}
        [Proof-of-Theorem-on-Sigma-Algebra-Representation-for-Random-Walk]
        {随机游走σ代数表示定理的证明}
        [Proof of Theorem on Sigma-Algebra Representation for Random Walk]
        [gpt-4.1]
        设 $x, y \in L_{i}$,令 $X_{n}, Y_{n}$ 是上述定义的耦合的一次实现,其中 $X_{0} = x$ 且 $Y_{0} = y$.设 $h(x, n)$ 是有界的时空调和函数.鞅性质表明 $h(x, 0) = E_{x} h(X_{n}, n)$.若 $| h | \leq C$,由耦合可得

\[
| h(x, 0) - h(y, 0) | = | E h(X_{n}, n) - E h(Y_{n}, n) | \leq 2C P(X_{n} 
eq Y_{n}) \to 0
\]

因此 $h(x, 0)$ 在 $L_{0}$ 和 $L_{1}$ 上为常数.对 $h'(x, m) = h(x, n+m)$ 应用上述结果,可知 $h(x, n) = a_{n}^{i}$ 在 $L_{i}$ 上为常数.鞅性质给出 $a_{n}^{i} = a_{n+1}^{1-i}$,所需结论由 Theorem 5.7 得出.
    \end{prf}
    
    
    
    \begin{thm}
        [Distribution-Formula-for-Last-Zero-of-Brownian-Motion]
        {布朗运动最后一次过零点的分布公式}
        [Distribution Formula for Last Zero of Brownian Motion]
        [gpt-4.1]
        
设 $L = \operatorname*{sup} \{ t \leq 1 : B_{t} = 0 \}$,其中 $B_t$ 为布朗运动.则有
\[
P_{0} ( L \leq s ) = \frac{2}{\pi} \arcsin ( \sqrt{s} ), \quad 0 < s < 1
\]
其中 $P_0$ 表示以 $B_0 = 0$ 为初始条件的概率.

    \end{thm}
    
    
    
    \begin{prf}
        [Proof-of-Consistency-Between-the-Conditional-Expectation-of-Limiting-Random-Variable-and-the-Initial-Variable]
        {极限随机变量的条件期望与初始变量一致性的证明}
        [Proof of Consistency Between the Conditional Expectation of Limiting Random Variable and the Initial Variable]
        [gpt-4.1]
        显然,$X_{-\infty} \in \mathcal{F}_{-\infty}$.

因为 $X_{n} = E(X_{0} | \mathcal{F}_{n})$,所以若 $A \in {\mathcal{F}}_{-\infty} \subset {\mathcal{F}}_{n}$,则
\[
\int_{A} X_{n} dP = \int_{A} X_{0} dP
\]

由定理 4.7.1 和引理 4.6.5 可知 $E(X_{n}; A) \to E(X_{-\infty}; A)$,因此
\[
\int_{A} X_{-\infty} dP = \int_{A} X_{0} dP
\]
对所有 $A \in {\mathcal{F}}_{-\infty}$ 成立,从而证明了所需结论.

    \end{prf}
    
    
    
    \begin{thm}
        [Recurrence-of-States-with-Stationary-Distribution]
        {存在平稳分布时的状态遍历性}
        [Recurrence of States with Stationary Distribution]
        [gpt-4.1]
        
定理 5.5.10 如果存在一个平稳分布,那么所有满足 $\pi(y) > 0$ 的状态 $y$ 都是遍历的(recurrent).

    \end{thm}
    
    
    
    \begin{dfn}
        [Definition-of-the-function-$pi-j$]
        {关于函数 $p(i, j)$ 的定义}
        [Definition of the function $p(i, j)$]
        [gpt-4.1]
        对于整数 $i, j$,函数 $p(i, j)$ 定义如下:

\[
\begin{array}{rl}
p(0, j) = f_{j+1} & \qquad \text{当 } j \geq 0 \\
p(i, i-1) = 1 & \qquad \text{当 } i \geq 1 \\
p(i, j) = 0 & \qquad \text{其他情况}
\end{array}
\]

    \end{dfn}
    
    
    
    \begin{dfn}
        [Definition-of-Two-dimensional-Brownian-Motion]
        {二维布朗运动的定义}
        [Definition of Two-dimensional Brownian Motion]
        [gpt-4.1]
        令 $B_{t} = ( B_{t}^{1}, B_{t}^{2} )$ 为从 $0$ 出发的二维布朗运动.
    \end{dfn}
    
    
    
    \begin{dfn}
        [Definition-of-Hitting-Time-for-Horizontal-Plane]
        {首次到达水平面的停时定义}
        [Definition of Hitting Time for Horizontal Plane]
        [gpt-4.1]
        令 $T_{a} = \operatorname*{inf} \{ t : B_{t}^{2} = a \}$.
    \end{dfn}
    
    
    
    \begin{thm}
        [Application-of-Itos-Formula-for-Polynomial-Case]
        {Ito公式在多项式情形下的应用}
        [Application of Ito's Formula for Polynomial Case]
        [gpt-4.1]
        
若 $u(t, x)$ 是关于 $t$ 和 $x$ 的多项式,且满足 $\frac{\partial u}{\partial t} + \frac{1}{2} \frac{\partial^{2}u}{\partial x^{2}} = 0$,则 Ito 公式给出

\[
u(t, B_{t}) - u(0, B_{0}) = \int_{0}^{t} \frac{\partial u}{\partial x}(s, B_{s}) d B_{s}
\]

其中 $\frac{\partial u}{\partial x}$ 也是多项式,满足定理7中的可积性条件.

    \end{thm}
    
    
    
    \begin{prf}
        [Proof-of-the-Inequality-between-$X\omega$-$S-{j+1}\omega$-and-$M-{k}\varphi-\omega$]
        {关于 $X(\omega)$ 与 $S\_{j+1}(\omega)$ 及 $M\_{k}(\varphi \omega)$ 的不等式证明}
        [Proof of the Inequality between $X(\omega)$, $S_{j+1}(\omega)$, and $M_{k}(\varphi \omega)$]
        [gpt-4.1]
        如果 $j \le k$,则 $M_{k}(\varphi \omega) \geq S_{j}(\varphi \omega)$,故加上 $X(\omega)$ 得
\[
X(\omega) + M_{k}(\varphi \omega) \geq X(\omega) + S_{j}(\varphi \omega) = S_{j+1}(\omega)
\]
整理后得到
\[
X(\omega) \geq S_{j+1}(\omega) - M_{k}(\varphi \omega) \quad \mathrm{for} \ j = 1, \ldots, k
\]
显然 $X(\omega) \geq S_{1}(\omega) - M_{k}(\varphi \omega)$,因为 $S_{1}(\omega) = X(\omega)$,且 $M_{k}(\varphi \omega) \geq 0$.
由此
\[
\begin{array}{l}
\displaystyle \mathcal{E}(X(\omega); M_{k} > 0) \geq \int_{\{M_{k} > 0\}} \max(S_{1}(\omega), \dots, S_{k}(\omega)) - M_{k}(\varphi \omega) dP \\
\displaystyle = \int_{\{M_{k} > 0\}} M_{k}(\omega) - M_{k}(\varphi \omega) dP
\end{array}
\]
又因 $M_{k}(\omega) = 0$ 且 $M_{k}(\varphi \omega) \geq 0$ 在 $\{M_{k} > 0\}^{c}$ 上成立,所以最后表达式为
\[
\geq \int M_{k}(\omega) - M_{k}(\varphi \omega) dP = 0
\]
因为 $\varphi$ 保测度.

    \end{prf}
    
    
    
    \begin{prf}
        [Proof-that-Mean-Zero-Implies-Recurrence]
        {均值为零蕴含遍历性的证明}
        [Proof that Mean Zero Implies Recurrence]
        [gpt-4.1]
        
如果 $E ( X _ { 1 } | \mathcal{T} ) = 0$,则遍历性定理表明 $S _ { n } / n \to 0$.现在有
\[
\limsup_{n \to \infty} \max_{1 \leq k \leq n} \frac{ | S _ { k } | }{ n } = \limsup_{n \to \infty} \left( \max_{K \leq k \leq n} \frac{ | S _ { k } | }{ n } \right) \leq \max_{k \geq K} \frac{ | S _ { k } | }{ k }
\]
对任意 $K$,右端随着 $K \uparrow \infty$ 而趋于 $0$.
由此容易得到
\[
\lim_{n \to \infty} \frac{ \max_{1 \leq k \leq n} | S _ { k } | }{ n } = 0
\]
由于 $R _ { n } \leq 1 + 2 \max_{1 \leq k \leq n} | S _ { k } |$,可得 $R _ { n } / n \to 0$,由定理 6.3.1 得 $P ( A ) = 0$.

    \end{prf}
    
    
    
    \begin{lma}
        [Reference-to-Lemma-8]
        {引理8的内容引用}
        [Reference to Lemma 8]
        [gpt-4.1]
        利用三角不等式和引理8,可得所需结果.
    \end{lma}
    
    
    
    \begin{thm}
        [Probability-Formula-for-Extremes-and-Stopping-Times-in-a-Stochastic-Process]
        {关于随机过程极值与停时的概率公式}
        [Probability Formula for Extremes and Stopping Times in a Stochastic Process]
        [gpt-4.1]
        设 $1 / 2 < p < 1$.

(c) 如果 $\dot{\mathrm{\Delta}} a < 0$,则 $P(\min_n S_n \leq a) = P(T_a < \infty) = \left\{ (1 - p) / p \right\}^{-a}$.

(d) 如果 $b > 0$,则 $P(T_b < \infty) = 1$ 且 $E T_b = b / (2p - 1)$.

    \end{thm}
    
    
    
    \begin{thm}
        [Itôs-Formula-for-Functions-of-Two-Variables]
        {二维函数的伊藤公式}
        [Itô's Formula for Functions of Two Variables]
        [gpt-4.1]
        
设 $f \in C^2$,即 $f$ 对所有变量有到二阶的连续偏导数.则几乎处处,对于所有 $t \geq 0$,有
\[
\begin{aligned}
f(t, B_t) - f(0, B_0) &= \int_0^t \frac{\partial f}{\partial t}(s, B_s)\, ds + \int_0^t \frac{\partial f}{\partial x}(s, B_s)\, dB_s \\
&\qquad + \frac{1}{2} \int_0^t \frac{\partial^2 f}{\partial x^2}(s, B_s)\, ds
\end{aligned}
\]

    \end{thm}
    
    
    
    \begin{dfn}
        [Definition-of-Last-Zero-Time-Function]
        {最后一次零点的函数定义}
        [Definition of Last Zero Time Function]
        [gpt-4.1]
        令 $\psi(\omega) = \sup\{t \leq 1 : \omega(t) = 0\}$,表示在时间 $1$ 之前路径 $\omega$ 的最后一个零点时刻.
    \end{dfn}
    
    
    
    \begin{thm}
        [Martingale-Central-Limit-Theorem]
        {鞅中心极限定理}
        [Martingale Central Limit Theorem]
        [gpt-4.1]
        设 $X_n, \mathcal{F}_n, n \geq 1$ 是一个鞅差序列,令 $V_k = \sum_{1 \leq n \leq k} E ( X_n^2 | \mathcal{F}_{n-1} )$.
    \end{thm}
    
    
    
    \begin{dfn}
        [Definition-of-Longest-Common-Subsequence]
        {最长公共子序列的定义}
        [Definition of Longest Common Subsequence]
        [gpt-4.1]
        
设有遍历平稳序列 $X_{1}, X_{2}, X_{3}, \ldots$ 和 $Y_{1}, Y_{2}, Y_{3}, \ldots$.定义
\[
L_{m, n} = \max \{ K : X_{i_{k}} = Y_{j_{k}} \text{ for } 1 \leq k \leq K, \text{ where } m < i_{1} < i_{2} < \cdots < i_{K} \leq n \text{ and } m < j_{1} < j_{2} < \cdots < j_{K} \leq n \}
\]
其中 $L_{m, n}$ 表示在区间 $(m, n]$ 内 $X$、$Y$ 两个序列的最长公共子序列的长度.

    \end{dfn}
    
    
    
    \begin{ppt}
        [Subadditivity-of-Longest-Common-Subsequence]
        {最长公共子序列的次可加性}
        [Subadditivity of Longest Common Subsequence]
        [gpt-4.1]
        
有如下不等式:
\[
L_{0, m} + L_{m, n} \leq L_{0, n}
\]
因此 $X_{m, n} = - L_{m, n}$ 是次可加的.

    \end{ppt}
    
    
    
    \begin{ppt}
        [Bounds-of-Longest-Common-Subsequence]
        {最长公共子序列的界限}
        [Bounds of Longest Common Subsequence]
        [gpt-4.1]
        
满足如下界限:
\[
0 \leq L_{0, n} \leq n
\]

    \end{ppt}
    
    
    
    \begin{thm}
        [Limit-Theorem-for-Length-of-Longest-Common-Subsequence]
        {最长公共子序列长度的极限定理}
        [Limit Theorem for Length of Longest Common Subsequence]
        [gpt-4.1]
        
有如下极限成立:
\[
L_{0, n} / n \to \gamma = \sup_{m \geq 1} E ( L_{0, m} / m )
\]

    \end{thm}
    
    
    
    \begin{thm}
        [Strong-Markov-Property-Theorem]
        {强马尔可夫性定理}
        [Strong Markov Property Theorem]
        [gpt-4.1]
        
设 $(s, \omega) \mapsto Y_s(\omega)$ 是有界且 $\mathcal{R} \times \mathcal{C}$ 可测的函数.如果 $S$ 是一个停时,则对所有 $\mathbf{x} \in \mathbf{R}^d$ 有

\[
E_x (Y_S \circ \theta_S \mid \mathcal{F}_S) = E_{B(S)} Y_S \quad \text{在} \ \{ S < \infty \}
\]

其中右侧为函数 $\varphi(x, t) = E_x Y_t$ 在 $x = B(S), t = S$ 处的取值.

    \end{thm}
    
    
    
    \begin{dfn}
        [Definition-of-Number-of-Points-Visited-in-One-Dimensional-Simple-Random-Walk]
        {一维简单随机游走访问点数的定义}
        [Definition of Number of Points Visited in One-Dimensional Simple Random Walk]
        [gpt-4.1]
        
设 $S_n$ 为一维简单随机游走,则
\[
R_n = 1 + \max_{m \leq n} S_m - \min_{m \leq n} S_m
\]
定义为到时间 $n$ 时访问过的点的数量.

    \end{dfn}
    
    
    
    \begin{prf}
        [Proof-of-Convergence-in-Probability-and-the-Dominated-Convergence-Theorem]
        {关于概率收敛与主导收敛定理的证明}
        [Proof of Convergence in Probability and the Dominated Convergence Theorem]
        [gpt-4.1]
        如果 $A_n$ 是一列集合,且 $P(A_n) \to 0$,则主导收敛定理(dominated convergence theorem)蕴含 $E(|X|;A_n) \to 0$.
    \end{prf}
    
    
    
    \begin{crl}
        [Estimate-of-Expectation-When-Probability-Measure-is-Below-a-Threshold]
        {概率测度小于阈值时期望的估计}
        [Estimate of Expectation When Probability Measure is Below a Threshold]
        [gpt-4.1]
        由上述结果可知,若 $\epsilon > 0$,可以取 $\delta > 0$,当 $P(A) \leq \delta$ 时,有 $E(|X|;A) \leq \epsilon$.
    \end{crl}
    
    
    
    \begin{cxmp}
        [Counterexample-for-the-Proposition-Relating-Probability-Measure-and-Expectation]
        {概率测度与期望关联命题的反例证明}
        [Counterexample for the Proposition Relating Probability Measure and Expectation]
        [gpt-4.1]
        (否则,存在集合 $A_n$ 满足 $P(A_n) \leq 1/n$ 且 $E(|X|;A_n) > \epsilon$,这导致矛盾.)
    \end{cxmp}
    
    
    
    \begin{thm}
        [Jensens-Inequality-for-Conditional-Expectation-and-Uniform-Integrability]
        {条件期望的 Jensen 不等式及均匀可积性}
        [Jensen's Inequality for Conditional Expectation and Uniform Integrability]
        [gpt-4.1]
        Jensen 不等式及条件期望的定义可推出
\[
\begin{array}{rl}
& E ( | E ( X | \mathcal{F} ) | ; | E ( X | \mathcal{F} ) | > M ) \leq E ( E ( | X | | \mathcal{F} ) ; E ( | X | | \mathcal{F} ) > M ) \\
& \qquad = E ( | X | ; E ( | X | | \mathcal{F} ) > M )
\end{array}
\]
且 $\{ E ( | X | | \mathcal{F} ) > M \} \in \mathcal{F}$.
    \end{thm}
    
    
    
    \begin{thm}
        [Chebyshevs-Inequality-and-Criterion-for-Uniform-Integrability]
        {Chebyshev 不等式与均匀可积性判据}
        [Chebyshev's Inequality and Criterion for Uniform Integrability]
        [gpt-4.1]
        应用 Chebyshev 不等式并结合 $M$ 的定义,有
\[
P \{ E ( | X | | \mathcal{F} ) > M \} \leq E \{ E ( | X | | \mathcal{F} ) \} / M = E | X | / M \leq \delta
\]
因此,根据 $\delta$ 的选择,
\[
E ( | E ( X | {\mathcal{F}} ) | ; | E ( X | {\mathcal{F}} ) | > M ) \leq \epsilon \quad \text{对所有}~{\mathcal{F}}
\]
由于 $\epsilon$ 是任意的,该集合是均匀可积的(uniformly integrable).
    \end{thm}
    
    
    
    \begin{lma}
        [Expectation-Convergence-for-Bounded-Continuous-Functions]
        {有界连续函数在极限过程中的期望收敛性}
        [Expectation Convergence for Bounded Continuous Functions]
        [gpt-4.1]
        如果 $\varphi$ 是有界且连续的函数,则 $E\varphi(S_{n,(n\cdot)}) \to E\varphi(B(\cdot))$.
    \end{lma}
    
    
    
    \begin{dfn}
        [Definition-of-Martingale-Difference-Array]
        {鞅差阵列的定义}
        [Definition of Martingale Difference Array]
        [gpt-4.1]
        我们称 $X_{n,m}$, $\mathcal{F}_{n,m}$, $1 \leq m \leq n$ 是一个鞅差阵列,如果 $X_{n,m} \in \mathcal{F}_{n,m}$ 且 $E( X_{n,m} \mid \mathcal{F}_{n,m-1} ) = 0$ 对于 $1 \leq m \leq n$ 都成立,其中 $\mathcal{F}_{n,0} = \{ \emptyset, \Omega \}$.
    \end{dfn}
    
    
    
    \begin{prf}
        [Proof-of-the-Limit-of-$R-n/n$]
        {关于$R\_n/n$极限的证明}
        [Proof of the Limit of $R_n/n$]
        [gpt-4.1]
        
证明 假设$X_1, X_2, \dots$在$(\mathbf{R}^d)^{\{0,1,\ldots\}}$上构造,且$X_n(\omega) = \omega_n$,令$\varphi$为移位算子.显然有
\[
R_n \geq \sum_{m=1}^{n} 1_A(\varphi^m \omega)
\]
因为右侧等于$|\{m: 1 \leq m \leq n, S_\ell 
eq S_m \text{ 对所有 } \ell > m\}|$.利用遍历定理得到
\[
\operatorname*{lim\,inf}_{n \to \infty} R_n / n \geq E(1_A|\mathcal{T}) \quad \mathrm{a.s.}
\]
为了证明反向不等式,令$A_k = \{ S_1 
eq 0, S_2 
eq 0, \ldots, S_k 
eq 0 \}$.显然有
\[
R_n \leq k + \sum_{m=1}^{n-k} 1_{A_k}(\varphi^m \omega)
\]
因为右侧的和等于$|\{ m : 1 \leq m \leq n-k, S_\ell 
eq S_m \text{ 对 } m < \ell \leq m+k\}|$.利用遍历定理得到
\[
\operatorname*{lim\,sup}_{n \to \infty} R_n / n \leq E(1_{A_k}|\mathcal{T})
\]
当$k \uparrow \infty$时,$A_k \downarrow A$,所以条件期望的单调收敛定理(定理4.1.9的(c))意味着
\[
E(\boldsymbol{1}_{A_k}|\mathcal{T}) \downarrow E(\boldsymbol{1}_{A}|\mathcal{T}) \quad \mathrm{as\ } k \uparrow \infty
\]
因此证明完成.

    \end{prf}
    
    
    
    \begin{thm}
        [Theorem-on-Convergence-of-Expectations]
        {期望收敛定理}
        [Theorem on Convergence of Expectations]
        [gpt-4.1]
        若 $\partial A_{i} \subset \{ x : f(x) \in \{ \alpha_{i-1}, \alpha_{i} \} \} \cup D_{f}$,则 $P(X_{\infty} \in \partial A_{i}) = 0$,并且由 (v) 得

\[
\sum_{i = 1}^{\ell} \alpha_{i} P(X_{n} \in A_{i}) \to \sum_{i = 1}^{\ell} \alpha_{i} P(X_{\infty} \in A_{i})
\]

又由 $\alpha_{i}$ 的定义可得

\[
0 \leq \sum_{i = 1}^{\ell} \alpha_{i} P(X_{n} \in A_{i}) - E f (X_{n}) \leq \epsilon \quad \text{for } 1 \leq n \leq \infty
\]

由于 $\epsilon$ 可任意选取, 所以 $E f ( X _ { n } ) \to E f ( X _ { \infty } )$.

    \end{thm}
    
    
    
    \begin{lma}
        [Proof-of-Lemma-on-Interval-Subdivision-Distance-Estimate]
        {关于分割区间距离估计的引理证明}
        [Proof of Lemma on Interval Subdivision Distance Estimate]
        [gpt-4.1]
        
结合上面三个不等式以及 $2^{-m} \leq |q - r|$ 和 $1 - 2^{-\gamma} > 0$,完成了引理7的证明.具体内容为:

在 $H_{N}$ 上,存在
\[
\begin{array} { l }
  \displaystyle |X(q) - X((i-1) 2^{-m})| \leq \sum_{h=1}^{k} (2^{-q(h)})^{\gamma} \leq \sum_{h=m}^{\infty} (2^{-\gamma})^{h} = \frac{2^{-\gamma m}}{1 - 2^{-\gamma}} \\
  \displaystyle |X(r) - X(i 2^{-m})| \leq \frac{2^{-\gamma m}}{1 - 2^{-\gamma}}
\end{array}
\]
其中 $N < r(1) < \cdots < r(\ell)$,$N < q(1) < \cdots < q(k)$.

    \end{lma}
    
    
    
    \begin{dfn}
        [Recursive-Definition-of-Total-Assets-of-Insurance-Company]
        {保险公司总资产的递推定义}
        [Recursive Definition of Total Assets of Insurance Company]
        [gpt-4.1]
        
设 $S_{n}$ 表示一家保险公司第 $n$ 年末的总资产.在第 $n$ 年,公司收到总额为 $c>0$ 的保费,并支付索赔 $\zeta_n$,其中 $\zeta_n$ 服从正态分布 $\mathrm{Normal}(\mu, \sigma^2)$,且 $\mu < c$.若记 $\xi_n = c - \zeta_n$,则有递推关系 $S_n = S_{n-1} + \xi_n$.

    \end{dfn}
    
    
    
    \begin{dfn}
        [Definition-of-Ruin-for-Insurance-Company]
        {保险公司破产条件的定义}
        [Definition of Ruin for Insurance Company]
        [gpt-4.1]
        
当保险公司资产降至 $0$ 或以下时,称公司破产.

    \end{dfn}
    
    
    
    \begin{thm}
        [Upper-Bound-on-Probability-of-Ruin-for-Insurance-Company]
        {保险公司破产概率的上界}
        [Upper Bound on Probability of Ruin for Insurance Company]
        [gpt-4.1]
        
若 $S_0>0$ 为非随机常数,则保险公司破产的概率满足

\[
P(\mathrm{ruin}) \leq \exp\left( -\frac{2(c-\mu) S_{0}}{\sigma^{2}} \right)
\]

    \end{thm}
    
    
    
    \begin{dfn}
        [Recursively-Defined-Sequence-of-Stopping-Times]
        {递归定义的停时序列}
        [Recursively Defined Sequence of Stopping Times]
        [gpt-4.1]
        令 $T_0 = 0$,对于 $n \geq 1$,定义
\[
T_n = \inf \{ t \geq T_{n-1} : B_t - B(T_{n-1}) 
otin ( U_n, V_n ) \}
\]
其中 $(U_1, V_1), (U_2, V_2), \ldots$ 是一系列区间,$B_t$ 是独立的布朗运动.
    \end{dfn}
    
    
    
    \begin{thm}
        [Transience-of-Markov-Chain-When-$\mu->-1$]
        {当 $\mu > 1$ 时马尔可夫链的暂态性}
        [Transience of Markov Chain When $\mu > 1$]
        [gpt-4.1]
        
当 $\mu > 1$,若 $\xi_1, \xi_2, \ldots$ 是独立同分布的随机变量,满足 $P(\xi_m = j) = a_{j+1}$ 对所有 $j \geq -1$,并定义 $S_n = \xi_1 + \cdots + \xi_n$,则 $X_0 + S_n$ 与 $X_n$ 在时间 $N = \operatorname*{inf} \{ n : X_0 + S_n = 0 \}$ 之前具有相同的行为.由于 $E\xi_m = \mu - 1 > 0$,故 $S_n \to \infty$ 几乎必然成立,且 $\inf S_n > -\infty$ 几乎必然成立.因此,当 $x$ 很大时,$P_x(N < \infty) < 1$,即该马尔可夫链是暂态的.

    \end{thm}
    
    
    
    \begin{thm}
        [Necessary-and-Sufficient-Condition-for-Equality-of-Random-Variables-via-Conditional-Expectation]
        {条件期望与随机变量相等的充要条件}
        [Necessary and Sufficient Condition for Equality of Random Variables via Conditional Expectation]
        [gpt-4.1]
        
若 $X$ 和 $Y$ 是随机变量,满足 $E ( Y | \mathcal{G} ) = X$ 且 $E Y^2 = E X^2 < \infty$,则 $X = Y$ 几乎处处成立.

    \end{thm}
    
    
    
    \begin{prf}
        [Proof-of-the-Markov-Property]
        {马尔可夫性质的证明}
        [Proof of the Markov Property]
        [gpt-4.1]
        The Markov property implies that

\[
E_{x}(B_{t} | \mathcal{F}_{s}) = E_{B_{s}}(B_{t-s}) = B_{s}
\]

since symmetry implies $E_{y} B_{u} = y$ for all $u \geq 0$.
    \end{prf}
    
    
    
    \begin{lma}
        [Probabilistic-Convergence-of-Truncated-Variable-Second-Moment]
        {关于截断变量二阶矩的概率收敛}
        [Probabilistic Convergence of Truncated Variable Second Moment]
        [gpt-4.1]
        
若 $\epsilon_n \to 0$ 足够慢,则有 $\epsilon_n^{-2} \hat{V}_n ( \epsilon_n ) \to 0$ 按概率收敛,其中
\[
\hat{V}_n(\epsilon) = \sum_{m=1}^{n} E( X_{n, m}^2 1_{( |X_{n, m}| > \epsilon_n )} | \mathcal{F}_{n, m-1} ) .
\]

    \end{lma}
    
    
    
    \begin{xmp}
        [Example-of-the-Ehrenfest-Chain]
        {Ehrenfest链的例子}
        [Example of the Ehrenfest Chain]
        [gpt-4.1]
        $S = \{ 0 , 1 , \ldots , r \}$

\[
p ( k , k + 1 ) = ( r - k ) / r \qquad p ( k , k - 1 ) = k / r
\]

在这种情况下,$\mu ( x ) = 2 ^ { - r } { \binom { r } { x } }$ 是一个平稳分布.

可以通过观察 $\mu$ 对应于抛 $r$ 枚硬币以决定每个球放入哪个盒子,以及链的转移对应于随机选取一枚硬币并翻转,来验证这一点.

或者,你可以验证
\[
\begin{array} { l } 
{ \displaystyle \mu ( k + 1 ) p ( k + 1 , k ) = 2 ^ { - r } \frac { r ! } { ( k + 1 ) ! ( r - k - 1 ) ! } \cdot \frac { k + 1 } { r } = 2 ^ { - r } \frac { ( r - 1 ) ! } { k ! ( r - k - 1 ) ! } } \\ 
{ = 2 ^ { - r } \frac { r ! } { k ! ( r - k ) ! } \cdot \frac { r - k } { r } = \mu ( k ) p ( k , k - 1 ) } 
\end{array}
\]

    \end{xmp}
    
    
    
    \begin{xmp}
        [Example-of-Birth-and-Death-Chains]
        {出生-死亡链的例子}
        [Example of Birth and Death Chains]
        [gpt-4.1]
        $S = \{ 0 , 1 , 2 , \ldots \}$

\[
p ( x , x + 1 ) = p _ { x } \qquad p ( x , x ) = r _ { x } \qquad p ( x , x - 1 ) = q _ { x }
\]
其中 $q _ { 0 } = 0$,且 $p ( i , j ) = 0$ 其他情况下为零.

对于此,存在如下测度:
\[
\mu ( x ) = \prod _ { k = 1 } ^ { x } { \frac { p _ { k - 1 } } { q _ { k } } }
\]
它满足详细平衡条件:
\[
\mu ( x ) p ( x , x + 1 ) = p _ { x } \prod _ { k = 1 } ^ { x } { \frac { p _ { k - 1 } } { q _ { k } } } = \mu ( x + 1 ) p ( x + 1 , x )
\]

    \end{xmp}
    
    
    
    \begin{thm}
        [Properties-of-Reversible-Markov-Chains]
        {可逆Markov链的性质}
        [Properties of Reversible Markov Chains]
        [gpt-4.1]
        设 $\mu$ 是平稳测度,且假设 $X _ { 0 }$ 的'分布'为 $\mu$.则 $Y _ { m } = X _ { n - m }$,$0 \leq m \leq n$ 是一个以 $\mu$ 为初始测度的Markov链,其转移概率为
\[
q ( x , y ) = \mu ( y ) p ( y , x ) / \mu ( x )
\]
其中 $q$ 被称为对偶转移概率.如果 $\mu$ 是可逆测度,则 $q = p$.

    \end{thm}
    
    
    
    \begin{thm}
        [Theorem-of-Measure-Decomposition-and-Its-Proof]
        {测度分解的定理及其证明}
        [Theorem of Measure Decomposition and Its Proof]
        [gpt-4.1]
        设 $\mu$ 和 $
u$ 为定义在同一测度空间上的两个测度,令 $\rho = (\mu+
u)/2$,$\rho_n = (\mu_n+
u_n)/2$ 为 $\rho$ 在 ${\mathcal{F}}_n$ 上的限制;令
\[
Y_n = \frac{d\mu_n}{d\rho_n}, \quad Z_n = \frac{d
u_n}{d\rho_n}
\]
则 $Y_n, Z_n \geq 0$ 且 $Y_n + Z_n = 2$,故 $Y_n, Z_n$ 是有界鞅,存在极限 $Y, Z$,并且
\[
Y = \frac{d\mu}{d\rho}, \qquad Z = \frac{d
u}{d\rho}
\]
且对于所有 $A \in {\mathcal{F}}$,
\[
\mu(A) = \int_A Y\, d\rho
\]
证明过程如下:由引理 4.3.6 的证明,若 $A \in {\mathcal{F}}_m \subset {\mathcal{F}}_n$,则
\[
\mu(A) = \int_A Y_n\, d\rho \to \int_A Y\, d\rho
\]
由有界收敛定理,进一步有
\[
\mu(A) = \int_A Y\, d\rho \quad \text{对于所有 } A \in {\mathcal{G}} = \bigcup_{m} {\mathcal{F}}_{m}
\]
$\mathcal{G}$ 是一个 $\pi$-系统,故由 $\pi$-$\lambda$ 定理,该等式对于所有 $A \in {\mathcal{F}} = \sigma(\mathcal{G})$ 成立,即 $(*)$ 得证.

    \end{thm}
    
    
    
    \begin{prf}
        [Proof-of-Limit-Mean-Convergence]
        {极限均值收敛性的证明}
        [Proof of Limit Mean Convergence]
        [gpt-4.1]
        \[
E \left| \frac { 1 } { n } \sum _ { m = 0 } ^ { n - 1 } X _ { M } ^ { \prime } ( \varphi ^ { m } \omega ) - E ( X _ { M } ^ { \prime } | \mathcal { T } ) \right| \to 0
\]

处理 $X _ { M } ^ { \prime \prime }$ 时, 有

\[
E \left| \frac { 1 } { n } \sum _ { m = 0 } ^ { n - 1 } X _ { M } ^ { \prime \prime } ( \varphi ^ { m } \omega ) \right| \leq \frac { 1 } { n } \sum _ { m = 0 } ^ { n - 1 } E | X _ { M } ^ { \prime \prime } ( \varphi ^ { m } \omega ) | = E | X _ { M } ^ { \prime \prime } |
\]

且 $E | E ( X _ { M } ^ { \prime \prime } | \mathcal { T } ) | \leq E E ( | X _ { M } ^ { \prime \prime } | | \mathcal { T } ) = E | X _ { M } ^ { \prime \prime } |$.

所以

\[
E \left| \frac { 1 } { n } \sum _ { m = 0 } ^ { n - 1 } X _ { M } ^ { \prime \prime } ( \varphi ^ { m } \omega ) - E ( X _ { M } ^ { \prime \prime } | \mathcal { T } ) \right| \leq 2 E | X _ { M } ^ { \prime \prime } |
\]

从而有

\[
\lim _ { n \to \infty } \sup _ { \boldsymbol { \pi } } E \left| \frac { 1 } { n } \sum _ { m = 0 } ^ { n - 1 } \boldsymbol { X } ( \boldsymbol { \varphi } ^ { m } \boldsymbol { \omega } ) - E ( \boldsymbol { X } | \mathcal { T } ) \right| \leq 2 E | \boldsymbol { X } _ { M } ^ { \prime \prime } |
\]

当 $M \to \infty$ 时, 由控测收敛定理 $E | X _ { M } ^ { \prime \prime } | \to 0$, 证毕.

    \end{prf}
    
    
    
    \begin{ppt}
        [Basic-Properties-of-Distribution-Function]
        {分布函数的基本性质}
        [Basic Properties of Distribution Function]
        [gpt-4.1]
        
分布函数 $F$ 具有以下三个基本性质:

(i) $F$ 是非减函数,即若 $x \leq y$, 则 $F(x) \leq F(y)$;
(ii) $\lim_{x \to \infty} F(x) = 1$,且 $\lim_{x \to -\infty} F(x) = 0$;
(iii) $F$ 是右连续函数,即 $\lim_{y \downarrow x} F(y) = F(x)$.

    \end{ppt}
    
    
    
    \begin{thm}
        [Integral-Form-of-Itôs-Formula]
        {伊藤公式的积分形式}
        [Integral Form of Itô's Formula]
        [gpt-4.1]
        
设 $f$ 是二次可微函数,$B_t$ 为布朗运动,则有
\[
f(B_{t}) - f(B_{0}) = \int_{0}^{t} f'(B_{s})\, dB_{s} + \frac{1}{2} \int_{0}^{t} f''(B_{s})\, ds \quad \mathrm{a.s.}
\]

    \end{thm}
    
    
    
    \begin{dfn}
        [Modified-Function-with-Bounded-Derivatives]
        {有界导数的修正函数}
        [Modified Function with Bounded Derivatives]
        [gpt-4.1]
        
令 $M$ 为充分大的正数,定义 $f_{M}$ 满足 $f_{M} = f$ 在区间 $[-M, M]$ 上,且 $|f_{M}'|, |f_{M}''| \le M$.

    \end{dfn}
    
    
    
    \begin{thm}
        [$B-{t}^{2}---t$-is-a-martingale]
        {$B\_{t}^{2} - t$ 是鞅过程}
        [$B_{t}^{2} - t$ is a martingale]
        [gpt-4.1]
        $B_{t}^{2} - t$ 是鞅过程.

证明 写出 $B_{t}^{2} = (B_{s} + B_{t} - B_{s})^{2}$,有
\[
\begin{array}{rl}
& E_{x}(B_{t}^{2} | \mathcal{F}_{s}) = E_{x}(B_{s}^{2} + 2 B_{s}(B_{t} - B_{s}) + (B_{t} - B_{s})^{2} | \mathcal{F}_{s}) \\
& \qquad = B_{s}^{2} + 2 B_{s} E_{x}(B_{t} - B_{s} | \mathcal{F}_{s}) + E_{x}((B_{t} - B_{s})^{2} | \mathcal{F}_{s}) \\
& \qquad = B_{s}^{2} + 0 + (t - s)
\end{array}
\]

因为 $B_{t} - B_{s}$ 与 $\mathcal{F}_{s}$ 独立,且均值为 0,方差为 $t-s$.

    \end{thm}
    
    
    
    \begin{dfn}
        [Definition-of-Coupling-Technique]
        {耦合技术的定义}
        [Definition of Coupling Technique]
        [gpt-4.1]
        耦合(coupling)是一种在同一概率空间上构造两个序列 $X_{n}$ 和 $Y_{n}$ 的技术.通常用于通过证明 $P(X_{n} 
eq Y_{n}) \to 0$,或者更一般地,对于某种度量 $\rho$,有 $\rho(X_{n}, Y_{n}) \to 0$(以概率收敛),来推断 $X_{n}$ 按分布收敛的结论.
    \end{dfn}
    
    
    
    \begin{thm}
        [Theorem-on-Discrete-Approximation-of-Stopping-Times]
        {关于停时的离散逼近的定理}
        [Theorem on Discrete Approximation of Stopping Times]
        [gpt-4.1]
        
设 $S$ 是一个停时,定义 $S_{n} = ( [ 2^{n} S ] + 1 ) / 2^{n}$,其中 $[x]$ 表示不超过 $x$ 的最大整数.即,

\[
S_{n} = (m + 1) 2^{-n} \quad \text{当 } m 2^{-n} \leq S < (m + 1) 2^{-n}
\]

即我们在 $S$ 之后的第一个形如 $k 2^{-n}$ 的时刻停止(即 $>S$).根据描述可知 $S_n$ 也是一个停时.

    \end{thm}
    
    
    
    \begin{prf}
        [Proof-of-Theorem-on-Discrete-Approximation-of-Stopping-Times]
        {离散逼近停时的定理的证明}
        [Proof of Theorem on Discrete Approximation of Stopping Times]
        [gpt-4.1]
        
若 $m 2^{-n} \leq S < (m + 1) 2^{-n}$,则
\[
\{ S_{n} < t \} = \{ S \leq m 2^{-n} \} \in \mathcal{F}_{m 2^{-n}} \subset \mathcal{F}_{t}
\]
因此 $S_n$ 是停时.另外不可能有 $S_n = 0$.

    \end{prf}
    
    
    
    \begin{dfn}
        [Definition-of-Completed-Sigma-Fields-and-Their-Intersection]
        {完备化的σ-域与其交的定义}
        [Definition of Completed Sigma-Fields and Their Intersection]
        [gpt-4.1]
        
令
\[
\begin{array}{rl}
& \mathcal{N}_x = \{ A : A \subset D \text{ with } P_x(D) = 0 \} \\
& \mathcal{F}_s^x = \sigma(\mathcal{F}_s^+ \cup \mathcal{N}_x) \\
& \mathcal{F}_s = \cap_x \mathcal{F}_s^x
\end{array}
\]
其中,$\mathcal{N}_x$ 为 $P_x$ 下的零测集,$\mathcal{F}_s^x$ 为 $P_x$ 下的完备化 $\sigma$-域,$\mathcal{F}_s$ 为所有 $\mathcal{F}_s^x$ 的交.

    \end{dfn}
    
    
    
    \begin{thm}
        [Limit-of-the-Sum-of-Squared-Increments-of-Brownian-Motion]
        {布朗运动增量平方和的极限}
        [Limit of the Sum of Squared Increments of Brownian Motion]
        [gpt-4.1]
        
设 $B_t$ 是布朗运动,$t^n_i$ 是 $[0,t]$ 上的分割点,$i=0,1,\ldots,2^n-1$.则有
\[
\sum_{i=0}^{2^{n}-1} \left(B(t^{n}_{i+1})-B(t^{n}_i)\right)^2 \to t
\]
当 $n\to\infty$ 时,增量平方和收敛到区间长度 $t$.

    \end{thm}
    
    
    
    \begin{prf}
        [Proof-that-the-Difference-of-Conditional-Expectations-is-Zero]
        {条件期望差异为零的证明}
        [Proof that the Difference of Conditional Expectations is Zero]
        [gpt-4.1]
        令 $Y _ { 1 } = E ( X _ { 1 } | \mathcal { F } )$,$Y _ { 2 } = E ( X _ { 2 } | \mathcal { F } )$.取 $A = \{ Y _ { 1 } - Y _ { 2 } \geq \epsilon > 0 \}$,则有
\[
0 = \int _ { A \cap B } ( X _ { 1 } - X _ { 2 } ) d P = \int _ { A \cap B } ( Y _ { 1 } - Y _ { 2 } ) d P \geq \epsilon P ( A )
\]
故 $P ( A ) = 0$,由此结论得证.

    \end{prf}
    
    
    
    \begin{prf}
        [Proof-of-Limiting-Cases-for-Characteristic-Functions]
        {特征函数极限情形的证明}
        [Proof of Limiting Cases for Characteristic Functions]
        [gpt-4.1]
        设 $\varphi_{n}(t) = E \exp(i t W_{n})$

\[
\psi_{n}(t) = E \exp(i t (\alpha_{n} W_{n} + \beta_{n})) = \exp(i t \beta_{n}) \varphi_{n}(\alpha_{n} t)
\]

如果 $\varphi$ 和 $\psi$ 分别是 $W$ 和 $W'$ 的特征函数,则

\[
\varphi_{n}(t) \to \varphi(t) \qquad \psi_{n}(t) \to \exp(i t \beta_{n}) \varphi_{n}(\alpha_{n} t) \psi(t)
\]

取收敛到极限 $\alpha \in [0, \infty]$ 的子列 $\alpha_{n(m)}$.首先注意到 $\alpha = 0$ 不可能.如果发生这种情况,则利用练习3.3.13中证明的均匀收敛性,

\[
|\psi_{n}(t)| = |\varphi_{n}(\alpha_{n} t)| \to 1
\]

$|\psi(t)| \equiv 1$,极限根据定理3.5.2是退化的.令 $t = u / \alpha_{n}$ 并交换 $\varphi$ 和 $\psi$ 的角色可知 $\alpha = \infty$ 也不可能.如果 $\alpha$ 是某个子列的极限,则按(b)中的论证有 $|\psi(t)| = |\varphi(\alpha t)|$.若存在两个子列极限 $\alpha' < \alpha$,对两个极限使用上述等式可得 $|\varphi(u)| = |\varphi(u \alpha' / \alpha)|$.迭代得到 $|\varphi(u)| = |\varphi(u (\alpha'/\alpha)^k)| \to 1$ 随着 $k \to \infty$,这与假设 $W'$ 非退化矛盾,因此 $\alpha_{n} \to \alpha \in [0, \infty)$.现在为了得出 $\beta_{n} \to \beta$,注意到练习3.3.13(ii)意味着 $\varphi_{n} \to \varphi$ 在紧集上一致收敛,因此 $\varphi_{n}(\alpha_{n} t) \to \varphi(\alpha t)$.如果 $\delta$ 足够小使得 $|\varphi(\alpha t)| > 0$ 对所有 $|t| \leq \delta$,则由(a)和再次使用练习3.3.13可得

\[
\exp(i t \beta_{n}) = \frac{\psi_{n}(t)}{\varphi_{n}(\alpha t)} \to \frac{\psi(t)}{\varphi(\alpha t)}
\]

在 $[-\delta, \delta]$ 上一致收敛.$\exp(i t \beta_{n})$ 是在 $\beta_{n}$ 处的点质量的特征函数.现在如定理3.3.17的证明中使用(3.3.1),可知在 $\beta_{n}$ 处的点质量的分布序列是紧的,即 $\beta_{n}$ 有界.如果 $\beta_{n_{m}} \to \beta$,则对 $|t| \leq \delta$ 有 $\exp(i t \beta) = \psi(t) / \varphi(\alpha t)$,因此只有一个子列极限.

    \end{prf}
    
    
    
    \begin{thm}
        [Existence-and-Uniqueness-of-Probability-Measure-for-Brownian-Motion]
        {布朗运动的概率测度存在唯一性}
        [Existence and Uniqueness of Probability Measure for Brownian Motion]
        [gpt-4.1]
        
设 $\Omega_o = \{\text{functions } \omega : [0, \infty) \to \mathbf{R}\}$,$\mathcal{F}_o$ 是由有限维集合 $\{\omega : \omega(t_i) \in A_i \text{ for } 1 \leq i \leq n\}$ 生成的 $\sigma$-域,其中 $A_i \in \mathcal{R}$.对每个 $x \in \mathbf{R}$,在 $(\Omega_o, \mathcal{F}_o)$ 上存在唯一概率测度 $
u_x$,使得 $
u_x\{\omega : \omega(0) = x\} = 1$,且当 $0 < t_1 < \cdots < t_n$ 时,

\[
u_x\{\omega : \omega(t_i) \in A_i\} = \mu_{x, t_1, \ldots, t_n} (A_1 \times \cdots \times A_n)
\]

其中 $\mu_{x, t_1, \ldots, t_n}$ 是布朗运动的有限维分布.这一结论由 Kolmogorov 拓展定理的推广(附录3中的(7.1))推出.

    \end{thm}
    
    
    
    \begin{prf}
        [Proof-of-Set-Construction-under-Continuity-Condition]
        {连续性条件下集合的构造证明}
        [Proof of Set Construction under Continuity Condition]
        [gpt-4.1]
        对于固定的 $\epsilon > 0$,令 $G_{ \delta } = \{ \omega : \text{若 } \| \omega - \omega^{\prime} \| < \delta,\ \text{则 } | \varphi(\omega) - \varphi(\omega^{\prime}) | < \epsilon \}$.
    \end{prf}
    
    
    
    \begin{thm}
        [Riemann-Stieltjes-Integral-Formula]
        {Riemann-Stieltjes积分公式}
        [Riemann-Stieltjes Integral Formula]
        [gpt-4.1]
        
若 $A_{s}$ 在每个有限区间上连续且有界变差,且 $f \in C^{1}$,则有
\[
f( A_{t} ) - f( A_{0} ) = \int_{0}^{t} f^{\prime}( A_{s} ) d A_{s}
\]
其中右边的Riemann-Stieltjes积分定义为
\[
\lim_{n \to \infty} \sum_{i=1}^{k(n)} f^{\prime}( A_{t_{i-1}^{n}} ) [ A( t_{i}^{n} ) - A( t_{i-1}^{n} ) ]
\]

    \end{thm}
    
    
    
    \begin{dfn}
        [Definition-of-Upcrossing-Number]
        {关于上下穿越次数的定义}
        [Definition of Upcrossing Number]
        [gpt-4.1]
        
设 $X_n$, $n \geq 0$, 是一个子鞅.令 $a < b$,$N_0 = -1$,对于 $k \geq 1$,定义
\[
\begin{array}{r}
N_{2k-1} = \operatorname{inf}\{m > N_{2k-2} : X_m \leq a\} \\
N_{2k} = \operatorname{inf}\{m > N_{2k-1} : X_m \geq b\}
\end{array}
\]
其中 $N_j$ 是停时,并定义
\[
U_n = \sup\{k : N_{2k} \leq n\}
\]
即 $U_n$ 是到时刻 $n$ 为止已经完成的从 $a$ 到 $b$ 的上下穿越次数(upcrossings).

    \end{dfn}
    
    
    
    \begin{dfn}
        [Definition-of-Predictable-Sequence]
        {关于可预测序列的定义}
        [Definition of Predictable Sequence]
        [gpt-4.1]
        
定义序列
\[
H_m = \left\{
\begin{array}{ll}
1 & \text{如果存在某个 } k \text{ 使得 } N_{2k-1} < m \leq N_{2k} \\
0 & \text{否则}
\end{array}
\right.
\]
则 $H_m$ 是一个可预测序列,表示在每次完成从 $a$ 到 $b$ 的上下穿越期间激活的赌博系统.

    \end{dfn}
    
    
    
    \begin{thm}
        [Upcrossing-Inequality]
        {上下穿越不等式}
        [Upcrossing Inequality]
        [gpt-4.1]
        
定理 4.2.10(上下穿越不等式) 若 $X_m$, $m \geq 0$, 是一个子鞅,则有
\[
(b - a) E U_n \leq E(X_n - a)^+ - E(X_0 - a)^+
\]

    \end{thm}
    
    
    
    \begin{prf}
        [Proof-of-Upcrossing-Inequality]
        {上下穿越不等式的证明}
        [Proof of Upcrossing Inequality]
        [gpt-4.1]
        
证明 设 $Y_m = a + (X_m - a)^+$.

    \end{prf}
    
    
    
    \begin{prf}
        [Proof-of-Nondifferentiability-of-Brownian-Paths]
        {布朗运动路径不可微性的证明}
        [Proof of Nondifferentiability of Brownian Paths]
        [gpt-4.1]
        
Fix a constant $C < \infty$ and let $A_{n} = \{\omega :$ there is an $s \in [0,1]$ so that $|B_{t} - B_{s}| \leq C|t - s|$ when $|t - s| \leq 3/n \}$.

For $1 \leq k \leq n-2$, let

\[
Y_{k,n} = \max \left\{ \left| B \left( \frac{k + j}{n} \right) - B \left( \frac{k + j - 1}{n} \right) \right| : j = 0, 1, 2 \right\}
\]

The triangle inequality implies $A_{n} \subset B_{n}$.

We pick $k = n-2$ and observe
\[
\begin{array} { l }
  \displaystyle \left| B\left( \frac{n-3}{n} \right) - B\left( \frac{n-2}{n} \right) \right| \leq \left| B\left( \frac{n-3}{n} \right) - B(1) \right| + \left| B(1) - B\left( \frac{n-2}{n} \right) \right| \\
  \leq C \left( \frac{3}{n} + \frac{2}{n} \right)
\end{array}
\]

Using $A_{n} \subset B_{n}$ and the scaling relation (7.1.1) now gives
\[
P(A_{n}) \leq P(B_{n}) \leq n \left[ P\left( |B(1/n)| \leq \frac{5C}{n} \right) \right]^{3} = n \left[ P\left( |B(1)| \leq \frac{5C}{n^{1/2}} \right) \right]^{3}
  \leq n \left\{ \frac{10C}{n^{1/2}} (2\pi)^{-1/2} \right\}^{3}
\]

since $\exp(-x^{2}/2) \leq 1$.

Letting $n \to \infty$ shows $P(A_{n}) \to 0$.

Noticing $A_{n}$ is increasing shows $P(A_{n}) = 0$ for all $n$ and completes the proof.

    \end{prf}
    
    
    
    \begin{thm}
        [Generalization-of-Central-Limit-Theorem]
        {中心极限定理的推广}
        [Generalization of Central Limit Theorem]
        [gpt-4.1]
        
若 (i) $V_{k}/k \to \sigma^{2} > 0$ 以概率收敛,且
\[
n^{-1} \sum_{m \leq n} E ( X_{m}^{2} 1_{( |X_{m}| > \epsilon \sqrt{n} )} ) \to 0
\]
则 $S_{(n \cdot)}/\sqrt{n} \Rightarrow \sigma B(\cdot)$.

    \end{thm}
    
    
    
    \begin{dfn}
        [Definition-of-the-set-$\mathcal{H}$]
        {关于集合$\mathcal{H}$的定义}
        [Definition of the set $\mathcal{H}$]
        [gpt-4.1]
        
设 $A \in \mathcal{F}_m$,定义$\mathcal{H}$为所有有界可测函数$Y$的集合,满足
\[
E _ { \mu } ( Y \circ \theta _ { m } ; A ) = E _ { \mu } ( E _ { X _ { m } } Y ; A )
\]

    \end{dfn}
    
    
    
    \begin{dfn}
        [Definition-of-the-set-$\mathcal{A}$]
        {$\mathcal{A}$集合的定义}
        [Definition of the set $\mathcal{A}$]
        [gpt-4.1]
        
令$\mathcal{A}$为所有形如$\{\omega : \omega_0 \in A_0, \ldots, \omega_k \in A_k\}$的集合的集合.

    \end{dfn}
    
    
    
    \begin{prf}
        [Derivation-for-Two-Point-Distribution-with-Zero-Mean]
        {关于均值为零的两点分布的推导}
        [Derivation for Two-Point Distribution with Zero Mean]
        [gpt-4.1]
        假设随机变量 $X$ 仅取值于 $\{a, b\}$,其中 $a < 0 < b$.由于 $E X = 0$,有
\[
P(X = a) = \frac{b}{b-a} \qquad P(X = b) = \frac{-a}{b-a}
\]
设 $T := T_{a, b} := \inf\{t : B_t 
otin (a, b)\}$,则由定理 7.5.3 可知 $B_T \overset{d}{=} X$,定理 7.5.5 给出
\[
E T = -a b = E B_T^2
\]
对于一般情形,可以将 $F(x) = P(X \leq x)$ 写为均值为零的两点分布的混合.

    \end{prf}
    
    
    
    \begin{thm}
        [Lindeberg-Feller-Theorem-for-Martingales]
        {鞅的Lindeberg-Feller定理}
        [Lindeberg-Feller Theorem for Martingales]
        [gpt-4.1]
        
设 $X_{n, m}, \mathcal{F}_{n, m}, 1 \leq m \leq n$ 是鞅差阵列.若满足以下条件:

(i) $V_{n, [nt]} \to t$ 以概率收敛,对所有 $t \in [0, 1]$;

(ii) 对所有 $\epsilon > 0$,
\[
\sum_{m \leq n} E(X_{n, m}^2 1_{( |X_{n, m}| > \epsilon )} | \mathcal{F}_{n, m-1}) \to 0
\]
以概率收敛,

则 $S_{n, (n \cdot)} \Rightarrow B(\cdot)$.

    \end{thm}
    
    
    
    \begin{thm}
        [Extension-of-Functional-Convergence-under-Continuous-Mapping]
        {连续映射下函数序列收敛的推广}
        [Extension of Functional Convergence under Continuous Mapping]
        [gpt-4.1]
        如果 $\psi : C[0, 1] \to \mathbf{R}$ 满足 $\psi$ 关于 $P_0$ 是连续的,那么有
\[
\psi(S(n\cdot)/\sqrt{n}) \Rightarrow \psi(B(\cdot))
\]

    \end{thm}
    
    
    
    \begin{xmp}
        [Example-of-Convergence-under-Continuous-Mapping]
        {连续映射下收敛的例子}
        [Example of Convergence under Continuous Mapping]
        [gpt-4.1]
        Example 8.
    \end{xmp}
    
    
    
    \begin{thm}
        [Central-Limit-Theorem-in-$\mathbf{R}^d$]
        {多维中心极限定理}
        [Central Limit Theorem in $\mathbf{R}^d$]
        [gpt-4.1]
        
设 $X_1, X_2, \ldots$ 是独立同分布的随机向量,满足 $E X_n = \mu$,且协方差有限,
\[
\Gamma_{ij} = E \left( (X_{n,i} - \mu_i)(X_{n,j} - \mu_j) \right)
\]
若 $S_n = X_1 + \cdots + X_n$,则
\[(S_n - n\mu)/n^{1/2} \Rightarrow \chi\]
其中 $\chi$ 服从均值为 $0$、协方差为 $\Gamma$ 的多元正态分布,即
\[
E \exp(i \theta \cdot \chi) = \exp \left( - \frac{1}{2} \sum_i \sum_j \theta_i \theta_j \Gamma_{ij} \right)
\]

    \end{thm}
    
    
    
    \begin{prf}
        [Proof-of-the-Central-Limit-Theorem-in-$\mathbf{R}^d$]
        {多维中心极限定理的证明}
        [Proof of the Central Limit Theorem in $\mathbf{R}^d$]
        [gpt-4.1]
        
通过考虑 $X_n' = X_n - \mu$,我们可以不失一般性地假设 $\mu = 0$.令 $\theta \in \mathbf{R}^d$,$\theta \cdot X_n$ 是均值为 0 的随机变量,其方差为
\[
E \left( \sum_i \theta_i X_{n,i} \right)^2 = \sum_i \sum_j E \left( \theta_i \theta_j X_{n,i} X_{n,j} \right) = \sum_i \sum_j \theta_i \theta_j \Gamma_{ij}
\]
因此,由一维中心极限定理和定理 3.10.6 可知 $S_n / n^{1/2} \Rightarrow \chi$,其中
\[
E \exp(i \theta \cdot \chi) = \exp \left( - \frac{1}{2} \sum_i \sum_j \theta_i \theta_j \Gamma_{ij} \right)
\]
从而得证.

    \end{prf}
    
    
    
    \begin{thm}
        [Martingale-Property-of-Polynomial-Solutions-to-Parabolic-Equation]
        {多项式解抛物型方程的鞅性质}
        [Martingale Property of Polynomial Solutions to Parabolic Equation]
        [gpt-4.1]
        
如果 $u(t, x)$ 是关于 $t$ 和 $x$ 的多项式,并且满足

\[
\frac{\partial u}{\partial t} + \frac{1}{2} \frac{\partial^{2} u}{\partial x^{2}} = 0
\]

则 $u(t, B_{t})$ 是一个鞅.

    \end{thm}
    
    
    
    \begin{thm}
        [Theorem-4.7.10-and-Related-Proofs]
        {定理 4.7.10 的形式及证明相关内容}
        [Theorem 4.7.10 and Related Proofs]
        [gpt-4.1]
        定理 4.7.10 已被多种不同方式证明.具体可以参考 Feller, Vol. II (1971), 第 228–229 页,其中有与矩问题相关的证明.Diaconis 和 Freedman (1980) 给出了一个很好的证明,其出发点是如下平凡观察:有限可交换序列 $X_m$, $1 \leq m \leq n$ 的分布具有 $p_0 H_{0,n} + \cdots + p_n H_{n,n}$ 的形式,其中 $H_{m,n}$ 表示'从有 $m$ 个 1 和 $n-m$ 个 0 的盒子中不放回地抽取'.若 $m \to \infty$ 且 $m/n \to p$,则 $H_{m,n}$ 收敛到密度为 $p$ 的乘积测度.定理 4.7.10 可由此容易推出,且可以得到收敛速率的界.
    \end{thm}
    
    
    
    \begin{prf}
        [Proof-of-the-Recurrence-Formula-for-Probability-β-{2k2n}]
        {关于概率β_{2k,2n}的递推公式的证明}
        [Proof of the Recurrence Formula for Probability β_{2k,2n}]
        [gpt-4.1]
        设 $\beta_{2k,2n}$ 表示感兴趣的概率.我们将用归纳法证明 $\beta_{2k,2n} = u_{2k} u_{2n-2k}$.

当 $n=1$ 时,显然有
\[
\beta_{0,2} = \beta_{2,2} = 1/2 = u_0 u_2
\]

对于一般的 $n$,首先假设 $k=n$.根据引理4.9.3的证明,我们有
\[
\begin{array}{l}
\displaystyle \frac{1}{2} u_{2n} = P(S_1 > 0, \dots, S_{2n} > 0) \\
\displaystyle \qquad = P(S_1 = 1, S_2 - S_1 \ge 0, \dots, S_{2n} - S_1 \ge 0) \\
\displaystyle \qquad = \frac{1}{2} P(S_1 \ge 0, \dots, S_{2n-1} \ge 0) \\
\displaystyle \qquad = \frac{1}{2} P(S_1 \ge 0, \dots, S_{2n} \ge 0) = \frac{1}{2} \beta_{2n,2n}
\end{array}
\]
倒数第二个等式的理由是,如果 $S_{2n-1} \geq 0$,则 $S_{2n-1} \geq 1$,因此 $S_{2n} \geq 0$.最后的计算证明了 $k=n$ 的情形.由于 $\beta_{0,2n} = \beta_{2n,2n}$,所以 $k=0$ 时结论同样成立.

现设 $1 \leq k \leq n-1$.此时,设 $R$ 是首次回到0的时刻,则 $R=2m$,其中 $0<m<n$.令 $f_{2m} = P(R=2m)$,并根据首次游程是在正半轴还是负半轴分解,有
\[
\beta_{2k,2n} = \frac{1}{2} \sum_{m=1}^k f_{2m} \beta_{2k-2m,2n-2m} + \frac{1}{2} \sum_{m=1}^{n-k} f_{2m} \beta_{2k,2n-2m}
\]
利用归纳假设,得到
\[
\beta_{2k,2n} = \frac{1}{2} u_{2n-2k} \sum_{m=1}^k f_{2m} u_{2k-2m} + \frac{1}{2} u_{2k} \sum_{m=1}^{n-k} f_{2m} u_{2n-2k-2m}
\]
通过考虑首次回到0的时刻,有
\[
u_{2k} = \sum_{m=1}^k f_{2m} u_{2k-2m} \qquad u_{2n-2k} = \sum_{m=1}^{n-k} f_{2m} u_{2n-2k-2m}
\]
于是得证.

    \end{prf}
    
    
    
    \begin{thm}
        [Sigma-Algebra-of-the-Limit-of-Decreasing-Stopping-Times]
        {关于停止时间的极限的σ代数}
        [Sigma-Algebra of the Limit of Decreasing Stopping Times]
        [gpt-4.1]
        若 $T_n \downarrow T$ 为一列递减的停止时间,则有
\[
\mathcal{F}_T = \bigcap \mathcal{F}_{T_n} .
\]

    \end{thm}
    
    
    
    \begin{prf}
        [Proof-of-Sigma-Algebra-for-the-Limit-of-Stopping-Times]
        {关于停止时间极限σ代数的证明}
        [Proof of Sigma-Algebra for the Limit of Stopping Times]
        [gpt-4.1]
        证明 首先由定理 7.3.6 得到 $\mathcal{F}_{T_n} \supset \mathcal{F}_T$ 对所有 $n$ 成立.

为证明反向包含,令 $A \in \bigcap \mathcal{F}_{T_n}$.由于 $A \cap \{ T_n < t \} \in \mathcal{F}_t$ 且 $T_n \downarrow T$,可推出 $A \cap \{ T < t \} \in \mathcal{F}_t$.

    \end{prf}
    
    
    
    \begin{thm}
        [Criterion-for-Continuity-Theorem]
        {连续性判别定理}
        [Criterion for Continuity Theorem]
        [gpt-4.1]
        
设 $E|X_s - X_t|^\beta \leq K|t-s|^{1+\alpha}$,其中 $\alpha, \beta > 0$.如果 $\gamma < \alpha/\beta$,则以概率1存在常数 $C(\omega)$,使得
\[
|X(q) - X(r)| \leq C |q - r|^\gamma \quad \text{对于所有 } q, r \in \mathbf{Q}_2 \cap [0, 1]
\]

    \end{thm}
    
    
    
    \begin{prf}
        [Proof-of-Criterion-for-Continuity-Theorem]
        {连续性判别定理的证明}
        [Proof of Criterion for Continuity Theorem]
        [gpt-4.1]
        
令 $G_n = \{ |X(i/2^n) - X((i-1)/2^n)| \leq 2^{-\gamma n}$ 对所有 $0 < i \leq 2^n \}$.

    \end{prf}
    
    
    
    \begin{dfn}
        [Definition-of-$N-{-n-}$-and-$T-{-1-}$]
        {关于 $N\_{ n }$ 和 $T\_{ 1 }$ 的定义}
        [Definition of $N_{ n }$ and $T_{ 1 }$]
        [gpt-4.1]
        设 $N_{ n } = \inf \{ m : S_{ m } \geq \sqrt{ n } \}$,$T_{ 1 } = \inf \{ t : B_{ t } \geq 1 \}$.
    \end{dfn}
    
    
    
    \begin{thm}
        [Theorem-on-Distributional-Convergence-of-$N-{-n-}$-and-$T-{-1-}$]
        {$N\_{ n }$ 与 $T\_{ 1 }$ 分布收敛的定理}
        [Theorem on Distributional Convergence of $N_{ n }$ and $T_{ 1 }$]
        [gpt-4.1]
        对于 $0 < t < 1$,有
\[
P( N_{ n } \leq n t ) \to P( T_{ 1 } \leq t )
\]

    \end{thm}
    
    
    
    \begin{dfn}
        [Definition-of-Return-Time-and-Return-Probability]
        {重返时间与重返概率的定义}
        [Definition of Return Time and Return Probability]
        [gpt-4.1]
        
令 $T_{y}^{0} = 0$,当 $k \geq 1$ 时,定义
\[
T_{y}^{k} = \operatorname*{inf} \{n > T_{y}^{k-1} : X_{n} = y\}
\]
并定义重返概率
\[
\rho_{xy} = P_{x}(T_{y} < \infty)
\]

    \end{dfn}
    
    
    
    \begin{dfn}
        [Definition-of-Recurrent-and-Transient-States]
        {重返态与游走态的定义}
        [Definition of Recurrent and Transient States]
        [gpt-4.1]
        
若 $\rho_{yy} = 1$,则状态 $y$ 称为重返态(recurrent);若 $\rho_{yy} < 1$,则状态 $y$ 称为游走态(transient).

    \end{dfn}
    
    
    
    \begin{thm}
        [Sufficient-Condition-for-Recurrence]
        {重返态的充分条件}
        [Sufficient Condition for Recurrence]
        [gpt-4.1]
        
若 $y$ 为重返态,则由定理 5.2.6 有 $P_{y}(T_{y}^{k} < \infty) = 1$ 对于所有 $k$ 成立,因此 $P_{y}(X_{n} = y\ \text{i.o.}) = 1$.

    \end{thm}
    
    
    
    \begin{dfn}
        [Definition-of-Zero-Set-and-Related-Notation-for-Brownian-Motion]
        {布朗运动零点与相关记号的定义}
        [Definition of Zero Set and Related Notation for Brownian Motion]
        [gpt-4.1]
        
令 $R_t = \operatorname{inf}\{ u > t : B_u = 0 \}$,${\cal T}_0 = \operatorname{inf}\{ u > 0 : B_u = 0 \}$,$\mathcal{Z}(\omega) = \{ t : B_t(\omega) = 0 \}$.

    \end{dfn}
    
    
    
    \begin{thm}
        [Properties-of-the-Zero-Set-of-Brownian-Motion]
        {布朗运动零点集的性质}
        [Properties of the Zero Set of Brownian Motion]
        [gpt-4.1]
        
(1) $\mathcal{Z}(\omega)$ 是闭集且没有孤立点,因此必不可数;
(2) $\mathcal{Z}(\omega)$ 的勒贝格测度为零,即
\[
E_x(|\mathcal{Z}(\omega) \cap [0, T]|) = \int_0^T P_x(B_t = 0) dt = 0
\]
(3) $\mathcal{Z}(\omega)$ 与康托尔集类似,但其 Hausdorff 维数为 $1/2$,而康托尔集的 Hausdorff 维数为 $\log 2 / \log 3$.

    \end{thm}
    
    
    
    \begin{lma}
        [Estimate-for-Infinite-Series-of-Non-negative-Increasing-Sequence]
        {关于非负递增序列的无穷级数估计}
        [Estimate for Infinite Series of Non-negative Increasing Sequence]
        [gpt-4.1]
        
Lemma 5.4.6 implies

\[
\sum_{n=0}^\infty u_n(1) \geq \frac{1}{2m} \sum_{n=0}^\infty u_n(m) \geq \frac{1}{2m} \sum_{n=0}^{A m} u_n(n / A)
\]

for any $A < \infty$, since $u_n(x) \geq 0$ and is increasing in $x$. By hypothesis $u_n(n / A) \to 1$, so letting $m \to \infty$ and noticing the right-hand side is $A/2$ times the average of the first $A m$ terms

\[
\sum_{n=0}^\infty u_n(1) \geq A/2
\]

Since $A$ is arbitrary, the sum must be $\infty$, and the desired conclusion follows from Theorem 5.4.

    \end{lma}
    
    
    
    \begin{dfn}
        [Definition-of-Exponential-Brownian-Motion]
        {指数布朗运动的定义}
        [Definition of Exponential Brownian Motion]
        [gpt-4.1]
        指数布朗运动 $X_{t}$ 定义为 $X_{t} = u(t, B_{t})$,其中 $u(t, x) = \exp(\mu t + \sigma x)$,$B_t$ 为标准布朗运动,$\mu, \sigma$ 为常数.
    \end{dfn}
    
    
    
    \begin{lma}
        [Lemma-on-Conditional-Probability-and-Adapted-Family]
        {关于条件概率与适应集族的引理}
        [Lemma on Conditional Probability and Adapted Family]
        [gpt-4.1]
        
如果 $A_n$ 对 ${\mathcal{G}}_n$ 是适应的, 那么对于任意非负 $\delta \in \mathcal{G}_0$, 有
\[
P\left( \cup_{m=1}^n A_m \mid \mathcal{G}_0 \right) \leq \delta + P\left( \sum_{m=1}^n P(A_m \mid \mathcal{G}_{m-1}) > \delta \Bigg| \mathcal{G}_0 \right)
\]

    \end{lma}
    
    
    
    \begin{prf}
        [Proof-of-Stationarity-and-Ergodicity-of-Conditional-Variance-Sequence]
        {对条件方差序列平稳遍历性的证明}
        [Proof of Stationarity and Ergodicity of Conditional Variance Sequence]
        [gpt-4.1]
        
$u_{n} \equiv E ( X_{n}^{2} | \mathcal{F}_{n-1} )$ 可以写作 $\varphi ( X_{n-1}, X_{n-2}, \ldots )$,因此定理 7.1.3 推出 $u_{n}$ 是平稳且遍历的,而遍历定理蕴含
\[
n^{-1} \sum_{m=1}^{n} u_{m} \to E u_{0} = E X_{0}^{2} \quad \mathrm{a.s.}
\]
最后的结论表明定理 8.2.8 的(i)成立.

    \end{prf}
    
    
    
    \begin{ppt}
        [Non-independence-of-Stopping-Time-and-Square-of-Brownian-Motion-at-Boundary]
        {边界处布朗运动的停时与平方不独立}
        [Non-independence of Stopping Time and Square of Brownian Motion at Boundary]
        [gpt-4.1]
        
若 $T = \operatorname*{inf} \{ t : B_{t} 
otin ( a, b ) \}$,其中 $a < 0 < b$ 且 $a 
eq -b$,则 $T$ 与 $B_{T}^{2}$ 不是独立的,因此不能像定理7的证明那样计算 $E T^{2}$.

    \end{ppt}
    
    
    
    \begin{ppt}
        [Estimation-of-Moments-of-Stopping-Time-and-Brownian-Motion-Using-Cauchy-Schwarz-Inequality]
        {Cauchy-Schwarz不等式用于估计停时与布朗运动高阶矩的关系}
        [Estimation of Moments of Stopping Time and Brownian Motion Using Cauchy-Schwarz Inequality]
        [gpt-4.1]
        
利用Cauchy-Schwarz不等式,可得到
(i) $E T^{2} \leq 4 E ( B_{T}^{4} )$,
(ii) $E B_{T}^{4} \leq 36 E T^{2}$.

    \end{ppt}
    
    
    
    \begin{thm}
        [Necessary-and-Sufficient-Condition-for-Stationary-Distribution-of-Birth-and-Death-Chains]
        {生灭链存在平稳分布的充要条件}
        [Necessary and Sufficient Condition for Stationary Distribution of Birth and Death Chains]
        [gpt-4.1]
        
生灭链有平稳分布当且仅当

\[
\sum_{x} \prod_{k=1}^{x} \frac{p_{k-1}}{q_{k}} < \infty
\]

    \end{thm}
    
    
    
    \begin{thm}
        [Expectation-Formula-for-Random-Sums]
        {随机求和期望公式}
        [Expectation Formula for Random Sums]
        [gpt-4.1]
        
设 $\{X_m\}$ 是一列随机变量,$N$ 是一个随机变量,$S_N=\sum_{m=1}^{N} X_m$,并且假设每个 $X_m$ 与 $\{N \geq m\}\in \mathcal{F}_{m-1}$ 独立,则

\[
E S_{N} = \sum_{m=1}^{\infty} E X_{m} P(N \geq m)
\]

特别地,如果 $X_m$ 彼此独立且同分布,且 $N$ 独立于所有 $X_m$,则有

\[
E S_{N} = E N \cdot E X_1
\]

    \end{thm}
    
    
    
    \begin{prf}
        [Proof-of-Expectation-Formula-for-Random-Sums]
        {随机求和期望公式的证明}
        [Proof of Expectation Formula for Random Sums]
        [gpt-4.1]
        
首先假设 $X_{i} \geq 0$.则有
\[
E S_{N} = \int S_{N} dP = \sum_{n=1}^{\infty} \int S_{n} 1_{\{ N = n \}} dP = \sum_{n=1}^{\infty} \sum_{m=1}^{n} \int X_{m} 1_{\{ N = n \}} dP
\]

由于 $X_{i} \geq 0$,可以交换求和次序(即使用 Fubini 定理),得到
\[
= \sum_{m=1}^{\infty} \sum_{n=m}^{\infty} \int X_{m} 1_{\{ N = n \}} dP = \sum_{m=1}^{\infty} \int X_{m} 1_{\{ N \geq m \}} dP
\]

由于 $\{ N \ge m \} = \{ N \le m - 1 \}^{c} \in \mathcal{F}_{m-1}$ 并与 $X_{m}$ 独立,故上述表达式
\[
= \sum_{m=1}^{\infty} E X_{m} P(N \geq m)
\]

在 $X_m$ 彼此独立且同分布,且 $N$ 独立于所有 $X_m$ 时,等价于 $E S_{N} = E N E X_{1}$.

对一般情况,若 $E N < \infty$,则
\[
\infty > \sum_{m=1}^{\infty} E | X_{m} | P(N \geq m) = \sum_{m=1}^{\infty} \sum_{n=m}^{\infty} \int | X_{m} | 1_{\{ N = n \}} dP
\]

由上述公式可知双重求和在一种顺序下绝对收敛,因此 Fubini 定理适用:
\[
\sum_{m=1}^{\infty} \sum_{n=m}^{\infty} \int X_{m} 1_{\{ N = n \}} dP = \sum_{n=1}^{\infty} \sum_{m=1}^{n} \int X_{m} 1_{\{ N = n \}} dP
\]

利用 $\{ N \geq m \} \in \mathcal{F}_{m-1}$ 与 $X_{m}$ 的独立性,并重写等式,得到
\[
\sum_{m=1}^{\infty} E X_{m} P(N \geq m) = E S_{N}
\]

由于左边为 $E N E X_{1}$,故证毕.

    \end{prf}
    
    
    
    \begin{xmp}
        [Example-of-Ergodicity-and-State-Space-Decomposition-in-Markov-Chains]
        {马尔可夫链的遍历性和状态空间分解的例子}
        [Example of Ergodicity and State Space Decomposition in Markov Chains]
        [gpt-4.1]
        
假设状态空间 $S$ 可数,且平稳分布满足 $\pi(x) > 0$,对所有 $x \in S$.由定理 5.5.10 和 5.3.5,所有状态都是再生的(recurrent),并且我们可以写 $S = \cup R_{i}$,其中 $R_{i}$ 是不相交的不可约闭集.如果 $X_{0} \in R_{i}$,则以概率一 $X_{n} \in R_{i}$ 对所有 $n \geq 1$ 成立,因此 $\{\omega : X_{0}(\omega) \in R_{i}\} \in \mathcal{T}$.最后的观察表明,如果马尔可夫链不是不可约的,则序列不是遍历的(ergodic).为了证明逆命题,若 $\mathit{A} \in \mathcal{T}$,则 $1_{A} \circ \theta_{n} = 1_{A}$,其中 $\theta_{n}(\omega_{0}, \omega_{1}, \ldots) = (\omega_{n}, \omega_{n+1}, \ldots)$.若记 $\mathcal{F}_{n} = \sigma(X_{0}, \ldots, X_{n})$,由于 $1_{A}$ 的平移不变性和马尔可夫性质,有
\[
E_{\pi}(\mathbf{1}_{A} \mid \mathcal{F}_{n}) = E_{\pi}(\mathbf{1}_{A} \circ \boldsymbol{\theta}_{n} \mid \mathcal{F}_{n}) = h(X_{n})
\]
其中 $h(x) = E_{x} \mathbf{1}_{A}$.Levy 的 0-1 定律意味着左边随着 $n \to \infty$ 收敛到 $1_{A}$.如果 $X_{n}$ 是不可约且再生的,则对于任何 $y \in S$,右边 $h(y)$ 无限次出现.因此要么 $h(x) \equiv 0$,要么 $h(x) \equiv 1$,于是 $P_{\pi}(A) \in \{0, 1\}$.该例子还说明 $\mathcal{T}$ 和 $\tau$ 可能不同.当转移概率 $p$ 不可约时,$\mathcal{T}$ 是平凡的,但如果所有状态周期 $d > 1$,则 $\tau$ 不是.

    \end{xmp}
    
    
    
    \begin{dfn}
        [Definition-of-Stationary-Measure]
        {平稳测度的定义}
        [Definition of Stationary Measure]
        [gpt-4.1]
        设 $\mu$ 是一个测度,若对所有 $y$ 有
\[
\sum_{x} \mu(x) p(x, y) = \mu(y)
\]
则称 $\mu$ 为平稳测度(stationary measure).
    \end{dfn}
    
    
    
    \begin{dfn}
        [Definition-of-Stationary-Distribution]
        {平稳分布的定义}
        [Definition of Stationary Distribution]
        [gpt-4.1]
        若 $\mu$ 是概率测度,则称 $\mu$ 为平稳分布(stationary distribution),它表示马尔可夫链可能达到的一个平衡状态.
    \end{dfn}
    
    
    
    \begin{ppt}
        [Distribution-Invariance-of-Stationary-Measure]
        {平稳测度的分布不变性}
        [Distribution Invariance of Stationary Measure]
        [gpt-4.1]
        若 $X_0$ 的分布为平稳测度 $\mu$,则对所有 $n \geq 1$,有
\[
P_{\mu}(X_n = y) = \mu(y)
\]
即 $X_n$ 的分布始终为 $\mu$.
    \end{ppt}
    
    
    
    \begin{thm}
        [Itôs-Formula-for-Multidimensional-Brownian-Motion]
        {二维布朗运动下的伊藤公式}
        [Itô's Formula for Multidimensional Brownian Motion]
        [gpt-4.1]
        
设 $f \in C^{2}$,则以概率 $1$,对所有 $t \geq 0$ 有
\[
\begin{array}{l}
{
f(t, B_{t}) - f(0, B_{0}) = \displaystyle \int_{0}^{t} \frac{\partial f}{\partial t}(s, B_{s}) ds
+ \sum_{i=1}^{d} \int_{0}^{t} D_{i} f(B_{s}) dB_{s}^{i}
} \\
{
\displaystyle \qquad + \frac{1}{2} \sum_{i=1}^{d} \int_{0}^{t} D_{ii} f(B_{s}) ds
}
\end{array}
\]
其中 $B_t$ 为 $d$ 维布朗运动,$D_i f$ 表示对第 $i$ 个分量的偏导,$D_{ii} f$ 表示对第 $i$ 个分量的二阶偏导.

    \end{thm}
    
    
    
    \begin{prf}
        [Proof-of-Itôs-Formula]
        {伊藤公式的证明}
        [Proof of Itô's Formula]
        [gpt-4.1]
        
令 $x_{0} = t$,并记 $D_{0} f = \partial f / \partial x_{0}$.

    \end{prf}
    
    
    
    \begin{dfn}
        [Definition-of-Notation-for-Random-Variables-and-Set]
        {随机变量和集合的记号定义}
        [Definition of Notation for Random Variables and Set]
        [gpt-4.1]
        设 $S_{n,m} = X_{n,1} + \cdots + X_{n,m}$,$N = \{ 0, 1, 2, \ldots \}$.
    \end{dfn}
    
    
    
    \begin{dfn}
        [Definition-of-Linear-Interpolation]
        {线性插值的定义}
        [Definition of Linear Interpolation]
        [gpt-4.1]
        在本节中,$S_{n,(n \cdot)}$ 表示 $S_{n,m}$ 的线性插值,定义为

\[
S_{n,(u)} = 
\begin{cases}
S_{k} & \text{当 } u = k \in N \\
\text{在 } u \in [ k, k + 1 ] \text{ 时为线性插值} & \text{其中 } k \in N
\end{cases}
\]
.
    \end{dfn}
    
    
    
    \begin{thm}
        [Laplace-Transform-Formula-for-the-Exit-Time-of-Brownian-Motion]
        {关于布朗运动出区时间的拉普拉斯变换公式}
        [Laplace Transform Formula for the Exit Time of Brownian Motion]
        [gpt-4.1]
        
设 $B_t$ 是布朗运动,$T_a$ 是首次达到 $a$ 的时间,$T_b$ 是首次达到 $b$ 的时间,$\sigma = \inf\{ t : B_t 
otin (a, b) \}$,$\lambda > 0$,则有

\[
E_{x} \exp ( -\lambda T_{a} ) = E_{x} ( e^{-\lambda \sigma} ; T_{a} < T_{b} ) + E_{x} ( e^{-\lambda \sigma} ; T_{b} < T_{a} ) E_{b} \exp ( -\lambda T_{a} )
\]

并且

\[
E_{x} ( e^{-\lambda \sigma} ; T_{a} < T_{b} ) = \frac{\sinh ( \sqrt{2\lambda} ( b - x ) )}{\sinh ( \sqrt{2\lambda} ( b - a ) ) }
\]

\[
E_{x} ( e^{-\lambda \sigma} ; T_{b} < T_{a} ) = \frac{\sinh ( \sqrt{2\lambda} ( x - a ) )}{\sinh ( \sqrt{2\lambda} ( b - a ) ) }
\]

    \end{thm}
    
    
    
    \begin{dfn}
        [Definition-of-First-Hitting-Time]
        {首次到达时刻的定义}
        [Definition of First Hitting Time]
        [gpt-4.1]
        设 $\mathcal{T}_0 = \inf\{ s > 0 : B_s = 0 \}$,其中 $B_s$ 为布朗运动,则 $\mathcal{T}_0$ 表示布朗运动首次到达 0 的时刻.
    \end{dfn}
    
    
    
    \begin{dfn}
        [Definition-of-Last-Hitting-Time]
        {最后一次到达时刻的定义}
        [Definition of Last Hitting Time]
        [gpt-4.1]
        设 $L = \sup\{ t \leq 1 : B_t = 0 \}$,其中 $B_t$ 为布朗运动,则 $L$ 表示在区间 $[0,1]$ 上布朗运动最后一次到达 0 的时刻.
    \end{dfn}
    
    
    
    \begin{thm}
        [Probability-Formula-for-First-Hitting-Time-of-Brownian-Motion]
        {关于布朗运动首次到达时刻的概率公式}
        [Probability Formula for First Hitting Time of Brownian Motion]
        [gpt-4.1]
        利用马尔可夫性质在时刻 1 可得
\[
P_x(R > 1 + t) = \int p_1(x, y) P_y(\mathcal{T}_0 > t) dy
\]
其中 $p_1(x, y)$ 为布朗运动从 $x$ 走到 $y$ 在时间 1 的转移密度.
    \end{thm}
    
    
    
    \begin{thm}
        [Distribution-Formula-for-Last-Hitting-Time-of-0-for-Brownian-Motion]
        {关于布朗运动最后一次到达 0 的分布公式}
        [Distribution Formula for Last Hitting Time of 0 for Brownian Motion]
        [gpt-4.1]
        利用马尔可夫性质在时刻 $0 < t < 1$ 可得
\[
P_0(L \leq t) = \int p_t(0, y) P_y(\mathcal{T}_0 > 1 - t) dy
\]
其中 $p_t(0, y)$ 为布朗运动从 0 走到 $y$ 在时间 $t$ 的转移密度.
    \end{thm}
    
    
    
    \begin{thm}
        [Theorem-on-First-Hitting-Probability-of-Brownian-Motion-at-Boundaries]
        {关于布朗运动首次到达界点概率的定理}
        [Theorem on First Hitting Probability of Brownian Motion at Boundaries]
        [gpt-4.1]
        若 $a < x < b$,则 $P_{x}(T_{a} < T_{b}) = \dfrac{b-x}{b-a}$.

其中 $T_a$ 和 $T_b$ 分别表示布朗运动首次达到点 $a$ 和 $b$ 的时刻,$P_x$ 表示起点为 $x$ 时的概率.

    \end{thm}
    
    
    
    \begin{prf}
        [Proof-of-the-Formula-for-First-Hitting-Probability-at-Boundaries]
        {首次到达界点概率公式的证明}
        [Proof of the Formula for First Hitting Probability at Boundaries]
        [gpt-4.1]
        设 $T = T_{a} \land T_{b}$.由定理 7.2.8 可知 $T < \infty$ 几乎必然成立.利用定理 7.5.1 和 7.5.2,有 $x = E_{x} B(T \wedge t)$.令 $t \to \infty$,再利用有界收敛定理,可得

\[
x = a P_{x}(T_{a} < T_{b}) + b (1 - P_{x}(T_{a} < T_{b}))
\]

解出 $P_{x}(T_{a} < T_{b})$ 得到所需结果.

    \end{prf}
    
    
    
    \begin{dfn}
        [Definition-of-Return-Times-to-Zero-for-Random-Walk]
        {随机游走返回零点的时间的定义}
        [Definition of Return Times to Zero for Random Walk]
        [gpt-4.1]
        令 $\tau_0 = 0$,$\tau_n = \inf\{ m > \tau_{n-1} : S_m = 0 \}$ 表示随机游走第 $n$ 次返回 0 的时刻.
    \end{dfn}
    
    
    
    \begin{thm}
        [Multiplicative-Formula-for-Return-Probability-in-Random-Walk]
        {随机游走返回零点概率的乘法公式}
        [Multiplicative Formula for Return Probability in Random Walk]
        [gpt-4.1]
        由强马尔可夫性可得:

\[
P(\tau_n < \infty) = P(\tau_1 < \infty)^n
\]

    \end{thm}
    
    
    
    \begin{thm}
        [Equivalent-Conditions-for-Return-to-Zero-in-Random-Walk]
        {随机游走返回零点的等价条件}
        [Equivalent Conditions for Return to Zero in Random Walk]
        [gpt-4.1]
        对于任意随机游走,以下条件等价:

(i) $P(\tau_1 < \infty) = 1$,

(ii) $P(S_m = 0 \text{ i.o.}) = 1$,

(iii) $\sum_{m=0}^{\infty} P(S_m = 0) = \infty$

    \end{thm}
    
    
    
    \begin{prf}
        [Proof-of-Equivalent-Conditions-for-Return-to-Zero-in-Random-Walk]
        {随机游走返回零点的等价条件的证明}
        [Proof of Equivalent Conditions for Return to Zero in Random Walk]
        [gpt-4.1]
        若 $P(\tau_1 < \infty) = 1$,则对所有 $n$ 都有 $P(\tau_n < \infty) = 1$,并且 $P(S_m = 0 \text{ i.o.}) = 1$.

令
\[
V = \sum_{m=0}^{\infty} 1_{(S_m = 0)} = \sum_{n=0}^{\infty} 1_{(\tau_n < \infty)}
\]
为访问 0 的次数(包括时刻 0 的访问).

取期望并利用 Fubini 定理将期望值移入求和:

\[
E V = \sum_{m=0}^{\infty} P(S_m = 0) = \sum_{n=0}^{\infty} P(\tau_n < \infty) = \sum_{n=0}^{\infty} P(\tau_1 < \infty)^n = \frac{1}{1 - P(\tau_1 < \infty)}
\]

第二个等式说明 (ii) 推出 (iii),结合最后两个等式说明若 (i) 不成立,则 (iii) 也不成立,即 (iii) 推出 (i).

    \end{prf}
    
    
    
    \begin{thm}
        [Weak-Convergence-Theorem-for-Stochastic-Processes]
        {随机过程弱收敛定理}
        [Weak Convergence Theorem for Stochastic Processes]
        [gpt-4.1]
        
$S(n \cdot ) / \sqrt{ n } \implies B(\cdot)$,即相关测度在 $C[0, \infty )$ 上弱收敛.

    \end{thm}
    
    
    
    \begin{prf}
        [Proof-of-Weak-Convergence-Theorem-for-Stochastic-Processes]
        {随机过程弱收敛定理的证明}
        [Proof of Weak Convergence Theorem for Stochastic Processes]
        [gpt-4.1]
        
根据定义,我们只需证明在 $C[0, M]$ 上对所有 $M < \infty$ 都发生弱收敛.当 $1$ 被 $M$ 代替时,定理 8.1.4 的证明方式同样适用.
$\Box$

    \end{prf}
    
    
    
    \begin{thm}
        [Expected-Exit-Time-of-Brownian-Motion-from-an-Interval]
        {布朗运动首次离开区间的期望时间}
        [Expected Exit Time of Brownian Motion from an Interval]
        [gpt-4.1]
        
设 $T = \operatorname{inf} \{ t : B_{t} 
otin (a, b) \}$,其中 $a < 0 < b$.则有
\[
E_{0} T = -ab
\]

    \end{thm}
    
    
    
    \begin{thm}
        [Distribution-Property-of-First-Hitting-Time-to-1-for-Symmetric-Random-Walk]
        {对称随机游走首次到达1的分布性质}
        [Distribution Property of First Hitting Time to 1 for Symmetric Random Walk]
        [gpt-4.1]
        
设 $S_n$ 是对称随机游走,$S_0=0$,令 $T_1 = \operatorname*{min} \{ n : S_n = 1 \}$.则有如下结论:

\[
E s^{T_{1}} = \frac{1 - \sqrt{1 - s^{2}}}{s}
\]

生成函数反演后得到:

\[
P(T_{1} = 2n - 1) = \frac{1}{2n - 1} \cdot \frac{ (2n)! }{ n! n! } 2^{-2n}
\]

    \end{thm}
    
    
    
    \begin{prf}
        [Proof-of-the-Distribution-Property-of-First-Hitting-Time-to-1-for-Symmetric-Random-Walk]
        {对称随机游走首次到达1的分布性质的证明}
        [Proof of the Distribution Property of First Hitting Time to 1 for Symmetric Random Walk]
        [gpt-4.1]
        
我们将使用指数鞅 $X_n = \exp ( \theta S_n ) / \phi ( \theta )^n$,其中 $\theta > 0$.

    \end{prf}
    
    
    
    \begin{dfn}
        [Definition-of-Markov-Chain-for-Consecutive-Heads]
        {关于连续正面次数的马尔可夫链定义}
        [Definition of Markov Chain for Consecutive Heads]
        [gpt-4.1]
        在本问题中,考虑投掷到连续三个正面(HHH)所需的次数 $N(HHH)$,令 $X_{n}$ 表示第 $n$ 次投掷后出现的连续正面次数.$X_{n}$ 构成一个具有如下转移概率的马尔可夫链:

\[
\begin{array}{cccc}
      & \mathbf{0} & \mathbf{1} & 2 & 3 \\
\mathbf{0} & 1/2 & 1/2 & 0 & 0 \\
\mathbf{1} & 1/2 & 0 & 1/2 & 0 \\
2 & 1/2 & 0 & 0 & 1/2 \\
3 & 0 & 0 & 0 & 1 \\
\end{array}
\]

其中,令 $T_{3}$ 表示达到状态 3 的时间.
    \end{dfn}
    
    
    
    \begin{xmp}
        [Example-of-Calculating-the-Probability-of-Brownian-Motion-First-Hitting-a-Fixed-Value]
        {布朗运动首次到达定值的概率计算例子}
        [Example of Calculating the Probability of Brownian Motion First Hitting a Fixed Value]
        [gpt-4.1]
        
给定布朗运动 $B_t$,定义 $T_a$ 为其首次到达定值 $a$ 的时刻.计算 $P(T_a \leq t)$,有以下公式:

\[
P ( T_{a} \leq t ) = 2 \, P_{0} ( B_{t} \geq a ) = 2 \int_{a}^{\infty} (2\pi t)^{-1/2} \exp ( - x^{2} / 2t ) dx
\]

变量变换 $x = ( t^{1/2} a ) / s^{1/2}$ 得

\[
\begin{aligned}
P_{0} ( T_{a} \le t ) &= 2 \int_{t}^{0} (2\pi t)^{-1/2} \exp ( - a^{2} / 2s ) \left( - t^{1/2} a / 2 s^{3/2} \right) ds \\
&= \int_{0}^{t} (2\pi s^{3})^{-1/2} a \exp ( - a^{2} / 2s ) ds
\end{aligned}
\]

利用上述公式可计算布朗运动首次到达定值 $a$ 的概率.

    \end{xmp}
    
    
    
    \begin{dfn}
        [Random-Shift-Induced-by-a-Random-Variable]
        {随机变量诱导的随机位移}
        [Random Shift Induced by a Random Variable]
        [gpt-4.1]
        
给定一个非负随机变量 $S(\omega)$,我们定义随机位移 $\theta_{S}$,它'截去 $\omega$ 在 $S(\omega)$ 之前的部分,并将路径平移,使得时间 $S(\omega)$ 变为时间 0'.具体而言,设

\[
(\theta_{S} \omega)(t) = \begin{cases}
\omega(S(\omega) + t) & \text{在 } \{ S < \infty \} \\
\Delta & \text{在 } \{ S = \infty \}
\end{cases}
\]

其中 $\Delta$ 是我们在 $C$ 中添加的一个附加点.

    \end{dfn}
    
    
    
    \begin{dfn}
        [Information-σ-Algebra-at-a-Random-Time]
        {随机时刻的信息子 σ-代数}
        [Information σ-Algebra at a Random Time]
        [gpt-4.1]
        
给定非负随机变量 $S(\omega)$,我们定义'在时间 $S$ 已知的信息'对应的 σ-代数为

\[
\mathcal{F}_{S} = \{ A : A \cap \{ S \leq t \} \in \mathcal{F}_{t} \text{ 对所有 } t \geq 0 \}
\]

即 $A$ 在 $\{ S \leq t \}$ 上的部分应当关于在时间 $t$ 的信息是可测的.

    \end{dfn}
    
    
    
    \begin{dfn}
        [Space-of-Continuous-Paths-and-the-σ-field-it-Generates]
        {连续路径空间及其生成的σ-域}
        [Space of Continuous Paths and the σ-field it Generates]
        [gpt-4.1]
        设 $C = \{\omega : [0, \infty) \to \mathbf{R} \mid \omega \text{ 连续} \}$,$\mathcal{C}$ 为由坐标映射 $t \mapsto \omega(t)$ 生成的 $\sigma$-域.
    \end{dfn}
    
    
    
    \begin{dfn}
        [Definition-of-the-Probability-Measure-$P-x$]
        {概率测度 $P\_x$ 的定义}
        [Definition of the Probability Measure $P_x$]
        [gpt-4.1]
        设 $\psi$ 为将 $\Omega_q$ 中的任意一致连续点映射到其在 $C$ 中的唯一连续延拓的可测映射.定义概率测度
\[
P_x = 
u_x \circ \psi^{-1}
\]

    \end{dfn}
    
    
    
    \begin{ppt}
        [Finite-Dimensional-Distributions-of-Brownian-Motion]
        {布朗运动的有限维分布性质}
        [Finite Dimensional Distributions of Brownian Motion]
        [gpt-4.1]
        由上述构造保证,$B_t(\omega) = \omega(t)$ 对于 $t \in \mathbf{Q}_2$ 具有正确的有限维分布.
    \end{ppt}
    
    
    
    \begin{ppt}
        [Extension-of-Finite-Dimensional-Distributions-via-Continuity]
        {路径的连续性与有限维分布的推广}
        [Extension of Finite Dimensional Distributions via Continuity]
        [gpt-4.1]
        路径的连续性与简单的极限论证表明,当 $t \in [0, \infty)$ 时,$B_t(\omega) = \omega(t)$ 依然具有正确的有限维分布.
    \end{ppt}
    
    
    
    \begin{ppt}
        [Properties-of-Brownian-Motion-under-Probability-Measure]
        {概率测度下布朗运动的性质}
        [Properties of Brownian Motion under Probability Measure]
        [gpt-4.1]
        在 $P_x$ 下,随机变量 $B_t(\omega) = \omega(t)$ 构成一组随机变量,其中 $B_t$ 为布朗运动,且 $P_x(B_0 = x) = 1$.
    \end{ppt}
    
    
    
    \begin{dfn}
        [Definition-of-Process-Function-$Y-s\omega$]
        {过程函数 $Y\_s(\omega)$ 的定义}
        [Definition of Process Function $Y_s(\omega)$]
        [gpt-4.1]
        
我们仅考虑如下形式的 $Y$:

\[
Y_{s}(\omega) = f_{0}(s) \prod_{m=1}^{n} f_{m}(\omega(t_{m}))
\]

其中 $0 < t_{1} < \ldots < t_{n}$ 且 $f_{0}, \ldots, f_{n}$ 都是有界且连续的函数.

    \end{dfn}
    
    
    
    \begin{thm}
        [Theorem-of-Continuity-for-Integrals]
        {积分连续性的定理}
        [Theorem of Continuity for Integrals]
        [gpt-4.1]
        
如果 $f$ 是有界且连续的,则由控测收敛定理可知

\[
\int d y\, p_{t}(x, y) f(y)
\]

是连续的.

    \end{thm}
    
    
    
    \begin{crl}
        [Boundedness-and-Continuity-of-Expectation-Function-$\varphix-s$]
        {期望函数 $\varphi(x, s)$ 的有界连续性}
        [Boundedness and Continuity of Expectation Function $\varphi(x, s)$]
        [gpt-4.1]
        
由上述结果和归纳法可得

\[
\begin{array}{c}
\varphi(x, s) = E_{x} Y_{s} = f_{0}(s) \displaystyle \int d y_{1}\, p_{t_{1}}(x, y_{1}) f_{1}(y_{1}) \\
\displaystyle \ldots \int d y_{n}\, p_{t_{n} - t_{n-1}}(y_{n-1}, y_{n}) f_{n}(y_{n})
\end{array}
\]

是有界且连续的.

    \end{crl}
    
    
    
    \begin{thm}
        [Backward-Martingale-Convergence-Theorem]
        {反向鞅收敛定理}
        [Backward Martingale Convergence Theorem]
        [gpt-4.1]
        
如果 $\mathcal{F}_{n} \downarrow \mathcal{F}_{-\infty}$ 当 $n \downarrow -\infty$(即 ${\mathcal{F}}_{-\infty} = \cap_{n} {\mathcal{F}}_{n}$),则
\[
E(Y | \mathcal{F}_{n}) \to E(Y | \mathcal{F}_{-\infty}) \quad \text{a.s. and in } L^{1}
\]

    \end{thm}
    
    
    
    \begin{prf}
        [Proof-of-Backward-Martingale-Convergence-Theorem]
        {反向鞅收敛定理的证明}
        [Proof of Backward Martingale Convergence Theorem]
        [gpt-4.1]
        
设 $X_{n} = E(Y | \mathcal{F}_{n})$ 是一个反向鞅,因此由定理 4.7.1 和 4.7.2 可知,当 $n \downarrow -\infty$ 时,$X_{n} \to X_{-\infty}$ 依概率几乎处处(a.s.)且在 $L^{1}$ 意义下收敛,其中
\[
X_{-\infty} = E(X_{0} | \mathcal{F}_{-\infty}) = E(E(Y | \mathcal{F}_{0}) | \mathcal{F}_{-\infty}) = E(Y | \mathcal{F}_{-\infty})
\]

    \end{prf}
    
    
    
    \begin{thm}
        [Recurrence-Criterion-for-1D-Random-Walk]
        {一维随机游走的遍历性判据}
        [Recurrence Criterion for 1D Random Walk]
        [gpt-4.1]
        $S_{n}$ 在 $d = 1$ 时是遍历的(recurrent),当且仅当 $S_{n} / n \to 0$ 以概率收敛.
    \end{thm}
    
    
    
    \begin{thm}
        [Recurrence-Criterion-for-2D-Random-Walk]
        {二维随机游走的遍历性判据}
        [Recurrence Criterion for 2D Random Walk]
        [gpt-4.1]
        $S_{n}$ 在 $d = 2$ 时是遍历的(recurrent),当且仅当 $S_{n} / n^{1/2}$ 收敛于一个非退化的正态分布.
    \end{thm}
    
    
    
    \begin{thm}
        [Transience-of-High-Dimensional-Random-Walk]
        {高维随机游走的非遍历性}
        [Transience of High-Dimensional Random Walk]
        [gpt-4.1]
        $S_{n}$ 在 $d \ge 3$ 且为'真正的三维'时是非遍历的(transient).
    \end{thm}
    
    
    
    \begin{dfn}
        [Abstract-Conditions-on-Filtrations-and-Random-Variables]
        {关于滤子和随机变量的抽象条件}
        [Abstract Conditions on Filtrations and Random Variables]
        [gpt-4.1]
        
设 $\mathcal{F}_n$ 是一列滤子($\sigma$-域),满足以下条件:

(i) $\mathcal{F}_{n}$ 随 $n$ 递增,且 $X_n \in \mathcal{F}_n$;

(ii) $\theta^{-1} \mathcal{F}_m = \mathcal{F}_{m+1}$.

这些抽象条件适用于鞅理论和遍历理论中的相关需求.

    \end{dfn}
    
    
    
    \begin{xmp}
        [Laplace-Transform-of-Hitting-Time-for-Brownian-Motion-with-Drift]
        {布朗运动带漂移首次达到给定水平的拉普拉斯变换}
        [Laplace Transform of Hitting Time for Brownian Motion with Drift]
        [gpt-4.1]
        
设 $X_t = B_t - b t$ 是带漂移 $-b$ 的布朗运动,其中 $B_t$ 是标准布朗运动,$b > 0$.令 $\tau = \inf\{ t : B_t = a + b t \}$,其中 $a > 0$.利用鞅 $\exp ( \theta B_{t} - \theta^{2} t / 2 )$,其中 $\theta = b + (b^2 + 2\lambda)^{1/2}$,可以证明

\[
E_{0} \exp ( -\lambda \tau ) = \exp ( -a \{ b + ( b^{2} + 2\lambda )^{1/2} \} )
\]

令 $\lambda \to 0$ 得到 $P_{0} ( \tau < \infty ) = \exp ( -2 a b )$.

    \end{xmp}
    
    
    
    \begin{thm}
        [Probability-Formula-for-Brownian-Motion-Values-Before-First-Hitting-Time]
        {关于布朗运动在首次到达时刻之前取值区间的概率公式}
        [Probability Formula for Brownian Motion Values Before First Hitting Time]
        [gpt-4.1]
        
如果 $u < 
u \leq a$,则
\[
P_{0}( T_{a} < t, u < B_{t} < 
u ) = P_{0}( 2a - 
u < B_{t} < 2a - u )
\]

    \end{thm}
    
    
    
    \begin{thm}
        [Probability-Density-Formula-for-Brownian-Motion-at-Exact-Location-After-First-Hitting-Time]
        {布朗运动在首次到达时刻后精确位置的概率密度公式}
        [Probability Density Formula for Brownian Motion at Exact Location After First Hitting Time]
        [gpt-4.1]
        
当 $(u, 
u)$ 在公式 (7.4.8) 中收缩至 $x$,对于 $a < x$,有
\[
P_{0}( T_{a} < t, B_{t} = x ) = p_{t}( 0, 2a - x )
\]
即 $\{ T_{a} < t \}$ 上 $B_{t}$ 的(次概率)密度.

    \end{thm}
    
    
    
    \begin{thm}
        [Probability-Formula-for-First-Hitting-Time-of-Boundaries-in-a-Stochastic-Process]
        {随机过程首次达到边界的概率公式}
        [Probability Formula for First Hitting Time of Boundaries in a Stochastic Process]
        [gpt-4.1]
        
若 $T_{z} = \operatorname*{inf} \{ n : S_{n} = z \}$,则对于 $a < x < b$,有
\[
P_{x}( T_{a} < T_{b} ) = \frac{ \varphi(b) - \varphi(x) }{ \varphi(b) - \varphi(a) } \qquad P_{x}( T_{b} < T_{a} ) = \frac{ \varphi(x) - \varphi(a) }{ \varphi(b) - \varphi(a) }
\]
其中 $P_{x}$ 表示从 $x$ 出发的概率,$\varphi$ 为相关函数.

    \end{thm}
    
    
    
    \begin{thm}
        [Central-Limit-Theorem-for-Martingale-Differences-in-Stationary-Sequences]
        {平稳序列的鞅型中心极限定理}
        [Central Limit Theorem for Martingale Differences in Stationary Sequences]
        [gpt-4.1]
        
假设 $X_{n}$, $n \in \mathbf{Z}$, 是一个遍历的平稳二次可积鞅差序列,即 $\sigma^{2} = E X_{n}^{2} < \infty$ 且 $E( X_{n} | \mathcal{F}_{n-1} ) = 0$,其中 $\mathcal{F}_{n} = \sigma ( X_{m}, m \leq n )$.令 $S_{n} = X_{1} + \cdots + X_{n}$.则有
\[
S_{(n \cdot)}/n^{1/2} \Rightarrow \sigma B(\cdot)
\]
其中 $B(\cdot)$ 为标准布朗运动.

    \end{thm}
    
    
    
    \begin{thm}
        [Lim-Sup-Bound-for-Brownian-Motion]
        {关于布朗运动的极限上确界}
        [Lim Sup Bound for Brownian Motion]
        [gpt-4.1]
        设 $B_t$ 是标准布朗运动,$( 1 + t )^{-1/2} \exp ( B_{t}^{2} / ( 2 ( 1 + t ) ) )$ 是一个鞅(martingale),由此可得
\[
\limsup_{t \to \infty} \frac{B_t}{\sqrt{(1 + t)\log(1 + t)}} \leq 1 \quad \text{几乎必然}.
\]

    \end{thm}
    
    
    
    \begin{thm}
        [Central-Limit-Theorem-under-Martingale-Difference-Dependence]
        {鞅差分依赖下的中心极限定理}
        [Central Limit Theorem under Martingale Difference Dependence]
        [gpt-4.1]
        
设 $X_{n}$, $n \in \mathbf{Z}$ 是一个遍历的平稳序列,且 $E X_{n} = 0$.令 $\mathcal{F}_{n}$ 满足 (i), (ii),并满足
\[
\sum_{n \geq 1} \| E ( X_{0} | \mathcal{F}_{-n} ) \|_{2} < \infty
\]
其中 $\| Y \|_{2} = ( E Y^{2} )^{1/2}$.令 $S_{n} = X_{1} + \cdots + X_{n}$.则有
\[
S_{(n \cdot)}/\sqrt{n} \Rightarrow \sigma B(\cdot)
\]
其中
\[
\sigma^{2} = E X_{0}^{2} + 2 \sum_{n=1}^{\infty} E X_{0} X_{n}
\]
且定义中的级数绝对收敛.

    \end{thm}
    
    
    
    \begin{dfn}
        [Definition-of-Occupation-Times-of-Half-Lines]
        {半直线占据时间的定义}
        [Definition of Occupation Times of Half-Lines]
        [gpt-4.1]
        
设
\[
\psi(\omega) = |\{t \in [0, 1] : \omega(t) > a\}|
\]
其中 $\omega$ 是路径,$a$ 为常数,$\psi(\omega)$ 表示路径 $\omega$ 在区间 $[0, 1]$ 上取值大于 $a$ 的时刻的测度.

    \end{dfn}
    
    
    
    \begin{thm}
        [Equivalent-Description-of-Non-Ergodicity]
        {非遍历性的等价描述}
        [Equivalent Description of Non-Ergodicity]
        [gpt-4.1]
        如果 $\varphi$ 不是遍历的,则该空间可以分成两个集合 $A$ 和 $A^{c}$,它们各自具有正测度,且 $\varphi(A) = A$ 且 $\varphi(A^{c}) = A^{c}$.
    \end{thm}
    
    
    
    \begin{lma}
        [Lemma-on-Weak-Convergence-of-Measures-and-Integral-Convergence-for-Function-Sequences]
        {关于弱收敛测度与函数列积分收敛的引理}
        [Lemma on Weak Convergence of Measures and Integral Convergence for Function Sequences]
        [gpt-4.1]
        
如果 (i) 测度 $\mu_{n}$ 在区间 $[0, t]$ 上弱收敛于有限测度 $\mu_{\infty}$,且 (ii) $g_{n}$ 是一列函数满足 $|g_{n}| \le K$,并具有如下性质:每当 $s_{n} \in [0, t] \to s$ 时,有 $g_{n}(s_{n}) \to g(s)$,那么当 $n \to \infty$ 时,
\[
\int g_{n}\, d\mu_{n} \to \int g\, d\mu_{\infty}
\]

    \end{lma}
    
    
    
    \begin{prf}
        [Proof-of-Lemma-on-Weak-Convergence-of-Measures-and-Integral-Convergence]
        {关于弱收敛测度与函数列积分收敛的证明}
        [Proof of Lemma on Weak Convergence of Measures and Integral Convergence]
        [gpt-4.1]
        
通过令 $\mu_{n}'(A) = \mu_{n}(A) / \mu_{n}([0, t])$,可以假设所有的 $\mu_{n}$ 都是概率测度.一个标准构造(见定理 3.2.8)表明存在一列分布为 $\mu_{n}$ 的随机变量 $X_{n}$,使得 $X_{n} \to X_{\infty}$ 当 $n \to \infty$.$g_{n}$ 收敛到 $g$ 意味着 $g_{n}(X_{n}) \to g(X_{\infty})$,所以根据有界收敛定理,结论成立.

    \end{prf}
    
    
    
    \begin{xmp}
        [Example-of-Birth-and-Death-Chains]
        {出生-死亡链的例子}
        [Example of Birth and Death Chains]
        [gpt-4.1]
        
设状态空间为 $\{0, 1, 2, \ldots\}$,转移概率为:
\[
p(i, i+1) = p_i,\quad p(i, i-1) = q_i,\quad p(i, i) = r_i
\]
其中 $q_0 = 0$.

    \end{xmp}
    
    
    
    \begin{dfn}
        [Definition-of-Stopping-Time-$N$]
        {停时 $N$ 的定义}
        [Definition of Stopping Time $N$]
        [gpt-4.1]
        
令 $N = \inf\{ n : X_n = 0 \}$.

    \end{dfn}
    
    
    
    \begin{dfn}
        [Recursive-Definition-of-Function-$\varphi$]
        {函数 $\varphi$ 的递归定义}
        [Recursive Definition of Function $\varphi$]
        [gpt-4.1]
        
首先令 $\varphi(0) = 0$,$\varphi(1) = 1$.
对于 $k \geq 1$,若 $X_n = k$,为了使 $\varphi(X_{N \wedge n})$ 为鞅,需要满足:
\[
\varphi(k) = p_k \varphi(k+1) + r_k \varphi(k) + q_k \varphi(k-1)
\]
利用 $r_k = 1 - (p_k + q_k)$,可改写为:
\[
q_k (\varphi(k) - \varphi(k-1)) = p_k (\varphi(k+1) - \varphi(k))
\]
或
\[
\varphi(k+1) - \varphi(k) = \frac{q_k}{p_k} (\varphi(k) - \varphi(k-1))
\]

    \end{dfn}
    
    
    
    \begin{dfn}
        [Explicit-Expression-of-Function-$\varphi$]
        {函数 $\varphi$ 的显式表达式}
        [Explicit Expression of Function $\varphi$]
        [gpt-4.1]
        
假设 $p_k, q_k > 0$($k \geq 1$),由于 $\varphi(1)-\varphi(0)=1$,递推可得:
\[
\begin{aligned}
& \varphi(m+1)-\varphi(m)=\prod_{j=1}^{m} \frac{q_j}{p_j} \quad \text{对于 } m \geq 1 \\
& \varphi(n)=\sum_{m=0}^{n-1} \prod_{j=1}^{m} \frac{q_j}{p_j} \quad \text{对于 } n \geq 1
\end{aligned}
\]
其中,当 $m=0$ 时,乘积视为 1.

    \end{dfn}
    
    
    
    \begin{dfn}
        [Definition-of-Stopping-Time-$T-c$]
        {停时 $T\_c$ 的定义}
        [Definition of Stopping Time $T_c$]
        [gpt-4.1]
        
令 $T_c = \inf\{ n \geq 1 : X_n = c \}$.

    \end{dfn}
    
    
    
    \begin{thm}
        [Probability-Formula-for-Two-Sided-Hitting-Times]
        {双边停时概率公式}
        [Probability Formula for Two-Sided Hitting Times]
        [gpt-4.1]
        
若 $a < x < b$,则有:
\[
P_x(T_a < T_b) = \frac{\varphi(b) - \varphi(x)}{\varphi(b) - \varphi(a)} \qquad P_x(T_b < T_a) = \frac{\varphi(x) - \varphi(a)}{\varphi(b) - \varphi(a)}
\]

    \end{thm}
    
    
    
    \begin{prf}
        [Proof-of-Probability-Formula-for-Two-Sided-Hitting-Times]
        {双边停时概率公式的证明}
        [Proof of Probability Formula for Two-Sided Hitting Times]
        [gpt-4.1]
        
设 $T = T_a \land T_b$,则 $\varphi(X_{n \wedge T})$ 是有界鞅,由定理 4.8.2 得 $\varphi(x) = E_x \varphi(X_T)$.又 $X_T \in \{a, b\}$.

    \end{prf}
    
    
    
    \begin{thm}
        [Second-Moment-of-the-First-Exit-Time-of-Brownian-Motion-from-an-Interval]
        {布朗运动首次离开区间的时刻二次期望}
        [Second Moment of the First Exit Time of Brownian Motion from an Interval]
        [gpt-4.1]
        
设 $T = \operatorname*{inf} \{ t : B_{t} 
otin (-a, a) \}$,其中 $B_t$ 表示布朗运动.当 $T$ 表示布朗运动首次离开区间 $(-a, a)$ 的时刻时,有 $E T^{2} = \frac{5 a^{4}}{3}$.

    \end{thm}
    
    
    
    \begin{thm}
        [Expectation-Conservation-for-Martingales-at-Bounded-Stopping-Times]
        {有界停止时间的鞅期望守恒定理}
        [Expectation Conservation for Martingales at Bounded Stopping Times]
        [gpt-4.1]
        
设 $X_{t}$ 是适应于右连续滤子且右连续的鞅过程,$T$ 是有界停止时间,则有 $E X_{T} = E X_{0}$.

证明 设 $n$ 为整数,使得 $P(T \leq n-1) = 1$.如强马尔可夫性质的证明中,令 $T_{m} = ([2^{m} T] + 1)/2^{m}$.$Y_{k}^{m} = X(k 2^{-m})$ 是关于滤子 $\mathcal{F}_{k}^{m} = \mathcal{F}(k 2^{-m})$ 的鞅过程,且 $S_{m} = 2^{m} T_{m}$ 是 $(Y_{k}^{m}, \mathcal{F}_{k}^{m})$ 的停止时间,因此由练习 4.4.4 有

\[
X(T_{m}) = Y_{S_{m}}^{m} = E(Y_{n 2^{m}}^{m} | \mathcal{F}_{S_{m}}^{m}) = E(X_{n} | \mathcal{F}(T_{m}))
\]

当 $m \uparrow \infty$ 时,由右连续性 $X(T_{m}) \to X(T)$,而由定理 7.3.7 $\mathcal{F}(T_{m}) \downarrow \mathcal{F}(T)$,因此由定理 4.7.3 得

\[
X(T) = E(X_{n} | \mathcal{F}(T))
\]

对两边取期望得 $E X(T) = E X_{n} = E X_{0}$,因为 $X_{n}$ 是鞅过程.

    \end{thm}
    
    
    
    \begin{dfn}
        [Construction-of-the-Extended-Markov-Chain-for-a-Harris-Chain]
        {Harris链的扩展马尔可夫链的构造}
        [Construction of the Extended Markov Chain for a Harris Chain]
        [gpt-4.1]
        
给定在 $(S, \mathcal{S})$ 上的 Harris 链,我们构造一个在 $({\bar{S}}, {\bar{\mathcal{S}}})$ 上具有转移概率 $\bar{p}$ 的马尔可夫链 $\bar{X}_{n}$,其中 ${\bar{S}} = S \cup \{\alpha\}$,$\bar{\mathcal{S}} = \{ B, B \cup \{\alpha\} : B \in \mathcal{S} \}$.转移概率定义如下:

\[
\begin{array}{rlrl}
& \text{若 } x \in S - A & & \bar{p}(x, C) = p(x, C) \quad \text{对任意 } C \in \mathcal{S} \\
& \text{若 } x \in A & & \bar{p}(x, \{\alpha\}) = \epsilon \\
& & & \bar{p}(x, C) = p(x, C) - \epsilon \rho(C) \quad \text{对任意 } C \in \mathcal{S} \\
& \text{若 } x = \alpha & & \bar{p}(\alpha, D) = \int \rho(dx) \bar{p}(x, D) \quad \text{对任意 } D \in \bar{\mathcal{S}}
\end{array}
\]

其中,$\alpha$ 是人为制造的点,在复发情形下该过程以概率 1 命中该点.

    \end{dfn}
    
    
    
    \begin{dfn}
        [Embedded-Chain-of-a-Harris-Chain-on-Set-A]
        {Harris链在集合A上的嵌入链}
        [Embedded Chain of a Harris Chain on Set A]
        [gpt-4.1]
        
对于 Harris 链,设 $A$ 是满足条件的集合,定义嵌入链 $Y_{k} = X(T_{A}^{k})$,其中 $T_{A}^{k} = \operatorname{inf}\{ n > T_{A}^{k-1} : X_{n} \in A \}$,且 $T_{A}^{0} = 0$.该链在集合 $A$ 上视为 Doeblin 条件成立的马尔可夫链.

    \end{dfn}
    
    
    
    \begin{dfn}
        [Definition-of-the-constant-c]
        {常数c的定义}
        [Definition of the constant c]
        [gpt-4.1]
        设
\[
c = \int_{-\infty}^0 (-u) dF(u) = \int_0^{\infty} 
u dF(
u)
\]

    \end{dfn}
    
    
    
    \begin{thm}
        [Integral-Transformation-Formula-and-Mixture-Representation]
        {积分变换公式与混合表示}
        [Integral Transformation Formula and Mixture Representation]
        [gpt-4.1]
        若 $\varphi$ 有界且 $\varphi(0) = 0$, 则有
\[
\int \varphi(x) dF(x) = c^{-1} \int_0^{\infty} dF(
u) \int_{-\infty}^0 dF(u) (
u - u) \left\{ \frac{
u}{
u - u} \varphi(u) + \frac{-u}{
u - u} \varphi(
u) \right\}
\]
该公式给出了所需的混合表示.

    \end{thm}
    
    
    
    \begin{dfn}
        [Definition-of-Probability-Distribution-for-$U-V$]
        {$(U, V)$的概率分布定义}
        [Definition of Probability Distribution for $(U, V)$]
        [gpt-4.1]
        设 $(U, V) \in \mathbf{R}^2$ 满足
\[
\begin{array}{l}
P\{(U, V) = (0, 0)\} = F(\{0\}) \\
P((U, V) \in \mathcal{A}) = c^{-1} \int \int_{(u, 
u) \in \mathcal{A}} dF(u) dF(
u) (
u - u)
\end{array}
\]

    \end{dfn}
    
    
    
    \begin{thm}
        [Limit-Expressions-for-Partition-Approximations-and-Itos-Formula]
        {关于分段逼近与伊藤公式的极限表达}
        [Limit Expressions for Partition Approximations and Ito's Formula]
        [gpt-4.1]
        
设 $f$ 为二阶可微函数,$B_t$ 为布朗运动,考虑区间 $[0,t]$ 上的分割 $0 = t_{0}^{n} < t_{1}^{n} < \ldots < t_{k(n)}^{n} = t$,令分割的最大长度 $\max_{i} (t_{i}^{n} - t_{i-1}^{n}) \to 0$.则有:

\[
\begin{array}{rl}
& \displaystyle\sum_{i} f'(B_{t_{i}^{n}}) (B_{t_{i+1}^{n}} - B_{t_{i}^{n}}) \to \int_{0}^{t} f'(B_{s}) dB_{s} \\
& \displaystyle\frac{1}{2} \sum_{i} g_{i}^{n}(\omega) (B_{t_{i+1}^{n}} - B_{t_{i}^{n}})^{2} \to \frac{1}{2} \int_{0}^{t} f''(B_{s}) ds
\end{array}
\]

概率意义下,$n \to \infty$ 时成立,其中 $g_{i}^{n}(\omega) = f''(c(B_{t_{i}^{n}}, B_{t_{i+1}^{n}}))$,$c(a, b)$ 为介于 $a$ 和 $b$ 之间的某点.

    \end{thm}
    
    
    
    \begin{ppt}
        [Application-of-Mean-Value-Theorem-in-Calculus]
        {微积分中的均值定理应用}
        [Application of Mean Value Theorem in Calculus]
        [gpt-4.1]
        
对于任意实数 $a$ 和 $b$,存在介于 $a$ 和 $b$ 之间的某点 $c(a, b)$,使得
\[
f(b) - f(a) = (b - a) f'(a) + \frac{1}{2} (b - a)^{2} f''(c(a, b))
\]

    \end{ppt}
    
    
    
    \begin{prf}
        [Proof-of-Conditional-Expectation-Formula]
        {关于条件期望公式的证明}
        [Proof of Conditional Expectation Formula]
        [gpt-4.1]
        我们先在特殊情况下证明该结果,然后利用 $\pi$-$\lambda$ 系统和单调类定理得到一般结果.

设 $A = \{ \omega : \omega _ { 0 } \in A _ { 0 } , \ldots , \omega _ { m } \in A _ { m } \}$,$g _ { 0 } , \ldots , g _ { n }$ 是有界且可测的函数.

利用 (5.2.3) 取 $f _ { k } = 1 _ { A _ { k } }$ 对于 $k < m$,$f _ { m } = 1 _ { A _ { m } } g _ { 0 }$,$f _ { k } = g _ { k - m }$ 对于 $m < k \leq m + n$,得到

\[
\begin{array}{l}
\displaystyle E _ { \mu } \left( \prod _ { k = 0 } ^ { n } g _ { k } ( X _ { m + k } ) ; A \right) = \int _ { A _ { 0 } } \mu ( d x _ { 0 } ) \int _ { A _ { 1 } } p ( x _ { 0 } , d x _ { 1 } ) \cdots \int _ { A _ { m } } p ( x _ { m - 1 } , d x _ { m } ) \\
\displaystyle \phantom { \frac { A _ { 1 } } { A _ { 1 } } } \cdot g _ { 0 } ( x _ { m } ) \int p ( x _ { m } , d x _ { m + 1 } ) g _ { 1 } ( x _ { m + 1 } ) \\
\displaystyle \phantom { \frac { A _ { 1 } } { A _ { 1 } } } \cdots \int p ( x _ { m + n - 1 } , d x _ { m + n } ) g _ { n } ( x _ { m + n } ) \\
\displaystyle \phantom { \frac { A _ { 1 } } { A _ { 1 } } } = E _ { \mu } \left( E _ { X _ { m } } \left( \prod _ { k = 0 } ^ { n } g _ { k } ( X _ { k } ) \right) ; A \right)
\end{array}
\]

满足上述公式的集合是一个 $\lambda$-系统,而我们已证的集合是一个 $\pi$-系统,因此利用 $\pi$-$\lambda$ 定理(定理 2.1.6),可知该恒等式对所有 $A \in \mathcal{F}_m$ 都成立.
    \end{prf}
    
    
    
    \begin{dfn}
        [Definition-of-the-Itô-Integral]
        {伊藤积分的定义}
        [Definition of the Itô Integral]
        [gpt-4.1]
        极限
\[
\int_{0}^{t} f'(B_{s})\, dB_{s}
\]
被定义为逼近的黎曼和的极限.
    \end{dfn}
    
    
    
    \begin{dfn}
        [Equivalent-Definition-of-Brownian-Motion]
        {布朗运动的等价定义}
        [Equivalent Definition of Brownian Motion]
        [gpt-4.1]
        布朗运动的第二个等价定义如下:$B_t, t \geq 0$ 是一个实值过程,满足以下条件:

(a) $B(t)$ 是高斯过程(即其所有有限维分布都是多元正态分布);

(b') $E B_s = 0$ 且 $E B_s B_t = s \wedge t$;

(c') 几乎处处,映射 $t \mapsto B_t$ 是连续的.

    \end{dfn}
    
    
    
    \begin{thm}
        [Arcsine-Law-for-the-Last-Visit-to-Zero]
        {关于最后一次访问零的反正弦律}
        [Arcsine Law for the Last Visit to Zero]
        [gpt-4.1]
        
定理 4.9.5 (Arcsine law for the last visit to 0) 对于 $0 < a < b < 1$,

\[
P(a \leq L_{2n} / 2n \leq b) \to \int_{a}^{b} \pi^{-1} (x(1-x))^{-1/2} dx
\]

为了说明该名称的由来,将积分中变量替换为 $y = x^{1/2}$, $dy = (1/2) x^{-1/2} dx$,则有

\[
\int_{\sqrt{a}}^{\sqrt{b}} \frac{2}{\pi} (1 - y^2)^{-1/2} dy = \frac{2}{\pi} \{ \arcsin(\sqrt{b}) - \arcsin(\sqrt{a}) \}
\]

由于 $L_{2n}$ 是 $2n$ 之前最后一次为零的时间,令人惊讶的是答案关于 $1/2$ 对称.

    \end{thm}
    
    
    
    \begin{crl}
        [Symmetry-of-the-Limit-Distribution-about-$1/2$]
        {极限分布关于 $1/2$ 的对称性}
        [Symmetry of the Limit Distribution about $1/2$]
        [gpt-4.1]
        
极限分布的对称性意味着

\[
P(L_{2n} / 2n \leq 1/2) = 1/2
\]

在赌博术语中,如果两个人每年每天都以掷硬币的方式赌 1 美元,那么以概率 $1/2$,其中一个玩家会从 7 月 1 日领先到年底,这一事件无疑会让另一个玩家抱怨自己的坏运气.

    \end{crl}
    
    
    
    \begin{thm}
        [Itos-Formula-for-One-Dimensional-Brownian-Motion]
        {一维布朗运动的伊藤公式}
        [Ito's Formula for One-Dimensional Brownian Motion]
        [gpt-4.1]
        
设 $f \in C^{2}$,即 $f$ 有两个连续导数.则以概率1,对于所有 $t \geq 0$,有
\[
f( B_{t} ) - f( B_{0} ) = \int_{0}^{t} f^{\prime}( B_{s} ) d B_{s} + \frac{1}{2} \int_{0}^{t} f^{\prime\prime}( B_{s} ) d s
\]

    \end{thm}
    
    
    
    \begin{dfn}
        [Length-of-the-Longest-Increasing-Sequence-in-a-Permutation]
        {排列中最长递增子序列的长度}
        [Length of the Longest Increasing Sequence in a Permutation]
        [gpt-4.1]
        设 $\pi$ 是 $\{1,2,\ldots,n\}$ 的一个排列,定义 $\ell(\pi)$ 为排列 $\pi$ 中最长递增子序列的长度.即,最大的 $k$,使得存在整数 $i_{1} < i_{2} < \cdots < i_{k}$,满足 $\pi(i_{1}) < \pi(i_{2}) < \cdots < \pi(i_{k})$.
    \end{dfn}
    
    
    
    \begin{dfn}
        [Length-of-the-Longest-Increasing-Path-in-a-Square-in-the-Poisson-Process]
        {Poisson过程中正方形内最长递增路径长度}
        [Length of the Longest Increasing Path in a Square in the Poisson Process]
        [gpt-4.1]
        在平面上放置一个速率为1的泊松过程,对于 $s < t \in [0,\infty)$,记 $Y_{s,t}$ 为泊松过程中位于顶点为 $(s,s)$, $(s,t)$, $(t,t)$, $(t,s)$ 的正方形 $R_{s,t}$ 内最长递增路径的长度.即,最大的 $k$,使得存在点 $(x_{i}, y_{i})$ 在泊松过程中,满足 $s < x_{1} < \cdots < x_{k} < t$ 且 $s < y_{1} < \cdots < y_{k} < t$.
    \end{dfn}
    
    
    
    \begin{ppt}
        [Subadditivity-of-Length-of-Increasing-Paths]
        {递增路径长度的次可加性}
        [Subadditivity of Length of Increasing Paths]
        [gpt-4.1]
        对任意 $0 < m < n$,有 $Y_{0,m} + Y_{m,n} \le Y_{0,n}$.
    \end{ppt}
    
    
    
    \begin{dfn}
        [Definition-of-Arrival-Times-in-Renewal-Process]
        {更新过程的到达时间序列定义}
        [Definition of Arrival Times in Renewal Process]
        [gpt-4.1]
        令 $P(\xi_{m} = j) = f_{j}$,设 $T_{0} = i_{0}$,对于 $k \geq 1$,令 $T_{k} = T_{k-1} + \xi_{k}$.其中 $T_{k}$ 表示第 $k$ 次到达时间,$\xi_{k}$ 为第 $k$ 次到达的间隔时间.
    \end{dfn}
    
    
    
    \begin{dfn}
        [Interpretation-of-Arrival-Times-in-Renewal-Process]
        {更新过程中的到达时间的解释}
        [Interpretation of Arrival Times in Renewal Process]
        [gpt-4.1]
        $T_{k}$ 是在更新过程中第 $k$ 次到达的时间,且该过程的第一次到达发生在时间 $i_{0}$.
    \end{dfn}
    
    
    
    \begin{dfn}
        [Definition-of-Indicator-Variable-and-Waiting-Time-in-Renewal-Process]
        {更新过程中的指示变量和等待时间的定义}
        [Definition of Indicator Variable and Waiting Time in Renewal Process]
        [gpt-4.1]
        设
\[
Y_{m} =
\begin{cases}
1 & \text{if } m \in \{ T_{0}, T_{1}, T_{2}, \dots \} \\
0 & \text{otherwise}
\end{cases}
\]
并令 $X_{n} = \inf \{ m - n : m \geq n, \ Y_{m} = 1 \}$.其中 $Y_{m}$ 为时间 $m$ 是否发生更新的指示变量,$X_{n}$ 为从时刻 $n$ 到第一个不早于 $n$ 的更新所需等待的时间.
    \end{dfn}
    
    
    
    \begin{thm}
        [Stopped-Uniformly-Integrable-Submartingale-Remains-Uniformly-Integrable]
        {一致可积次可停鞅的截断依然一致可积}
        [Stopped Uniformly Integrable Submartingale Remains Uniformly Integrable]
        [gpt-4.1]
        
如果 $X_{n}$ 是一致可积的次可停鞅(submartingale),那么对于任意停时 $N$,过程 $X_{N \wedge n}$ 仍然是一致可积的.

    \end{thm}
    
    
    
    \begin{thm}
        [Derivation-of-Positive-Recurrence-Condition-for-Markov-Chains]
        {马尔可夫链正再生条件的推导}
        [Derivation of Positive Recurrence Condition for Markov Chains]
        [gpt-4.1]
        设 $X_n \geq 0$ 是一个马尔可夫链,且对 $x > K$ 有 $E_x X_1 \leq x - \epsilon$,其中 $\epsilon > 0$.令 $Y_n = X_n + n\epsilon$,$\tau = \operatorname*{inf}\{ n : X_n \leq K \}$.则 $Y_{n \wedge \tau}$ 是一个正的超鞅,停时定理可推出 $E_x \tau \leq x / \epsilon$.若进一步对从 $x \leq K$ 出发的链的行为有适当假设,则可得正再生的条件.

    \end{thm}
    
    
    
    \begin{lma}
        [Probability-Inequality-of-Lemma-8.2.7]
        {引理 8.2.7 的概率不等式}
        [Probability Inequality of Lemma 8.2.7]
        [gpt-4.1]
        引理 8.2.7 给出如下概率不等式:

\[
P ( | X_{n,m} | > \epsilon_n \text{ for some } m \leq n ) \leq \delta + P \left( \sum_{m=1}^n P ( | X_{n,m} | > \epsilon_n | \mathcal{F}_{n,m-1} ) > \delta \right)
\]

其中 $A_m = \{ | X_{n,m} | > \epsilon_n \}$, $\mathcal{G}_m = \mathcal{F}_n$, $\epsilon_n$ 为正数,$\delta$ 为任意正数.

    \end{lma}
    
    
    
    \begin{ppt}
        [Conditional-Form-of-Chebyshevs-Inequality]
        {Chebyshev 不等式的条件概率形式}
        [Conditional Form of Chebyshev's Inequality]
        [gpt-4.1]
        Chebyshev 不等式的条件概率形式为:

\[
\sum_{m=1}^n P ( | X_{n,m} | > \epsilon_n | \mathcal{F}_{n,m-1} ) \leq \epsilon_n^{-2} \sum_{m=1}^n E ( \hat{X}_{n,m}^2 | \mathcal{F}_{n,m-1} )
\]

其中 $\epsilon_n$ 为正数,$\hat{X}_{n,m}$ 为相应的随机变量,$\mathcal{F}_{n,m-1}$ 为过滤族.

    \end{ppt}
    
    
    
    \begin{lma}
        [A-Lemma-on-Conditional-Expectation-and-Quadratic-Term-Estimation]
        {关于条件期望和二次项估计的一个引理}
        [A Lemma on Conditional Expectation and Quadratic Term Estimation]
        [gpt-4.1]
        
在 $\{ | X _ { n , m } | \leq \epsilon _ { n }$ 对所有 $1 \leq m \leq n \}$ 上,有
\[
\| S _ { n , ( n \cdot ) } - \tilde { S } _ { n , ( n \cdot ) } \| \leq \sum _ { m = 1 } ^ { n } | E ( \bar { X } _ { n , m } | \mathcal { F } _ { n , m - 1 } ) |
\]
且
\[
\sum _ { m = 1 } ^ { n } | E ( \bar { X } _ { n , m } | \mathcal { F } _ { n , m - 1 } ) | 
= \sum _ { m = 1 } ^ { n } | E ( \hat { X } _ { n , m } | \mathcal { F } _ { n , m - 1 } ) |
\leq \sum _ { m = 1 } ^ { n } E ( | \hat { X } _ { n , m } | | \mathcal { F } _ { n , m - 1 } )
\leq \epsilon _ { n } ^ { - 1 } \sum _ { m = 1 } ^ { n } E ( \hat { X } _ { n , m } ^ { 2 } | \mathcal { F } _ { n , m - 1 } ) \to 0
\]
在概率意义下,由引理 8 得出.

    \end{lma}
    
    
    
    \begin{dfn}
        [Definition-of-Detailed-Balance-Condition]
        {详细平衡条件的定义}
        [Definition of Detailed Balance Condition]
        [gpt-4.1]
        
$\mu$ 满足详细平衡条件(detailed balance condition)如果

\[
\mu ( x ) p ( x , y ) = \mu ( y ) p ( y , x )
\]

即对于所有 $x, y$,从 $x$ 到 $y$ 在一步中转移的质量与从 $y$ 到 $x$ 的转移质量相等.

    \end{dfn}
    
    
    
    \begin{dfn}
        [Definition-of-Reversible-Measure]
        {可逆测度的定义}
        [Definition of Reversible Measure]
        [gpt-4.1]
        
若测度 $\mu$ 满足 $\mu(x)p(x, y) = \mu(y)p(y, x)$,则称 $\mu$ 是一个可逆测度(reversible measure).

    \end{dfn}
    
    
    
    \begin{prf}
        [Proof-of-Independence-of-Brownian-Motion-Increments]
        {关于布朗运动增量独立性的证明}
        [Proof of Independence of Brownian Motion Increments]
        [gpt-4.1]
        Proof Let $\mathcal{A}_1 = \sigma(B_0)$ and $\mathcal{A}_2$ be the events of the form
\[
\{B(t_1) - B(t_0) \in A_1, \ldots, B(t_n) - B(t_{n-1}) \in A_n\}.
\]
The $\mathcal{A}_i$ are $\pi$-systems that are independent, so the desired result follows from the $\pi$-$\lambda$ theorem 2.
    \end{prf}
    
    
    
    \begin{thm}
        [Subadditive-Ergodic-Theorem]
        {次可加遍历定理}
        [Subadditive Ergodic Theorem]
        [gpt-4.1]
        
设 $X_{m, n}$,$0 \leq m < n$ 满足:

(i) $X_{0, m} + X_{m, n} \geq X_{0, n}$

(ii) 对每个 $k$,$\{ X_{nk, (n+1)k}, n \geq 1 \}$ 是平稳序列;

(iii) $\{ X_{m, m+k}, k \geq 1 \}$ 的分布与 $m$ 无关;

(iv) $E X_{0, 1}^{+} < \infty$,且对每个 $n$,$E X_{0, n} \geq \gamma_{0} n$,其中 $\gamma_{0} > -\infty$.

则有:

(a) $\lim_{n \to \infty} E X_{0, n} / n = \inf_{m} E X_{0, m} / m \equiv \gamma$

(b) $X = \lim_{n \to \infty} X_{0, n} / n$ 几乎处处和在 $L^{1}$ 中存在,且 $E X = \gamma$.

(c) 若条件 (ii) 中所有平稳序列均为遍历的,则 $X = \gamma$ 几乎处处成立.

    \end{thm}
    
    
    
    \begin{ppt}
        [Translation-Invariance-of-Brownian-Motion]
        {布朗运动的平移不变性}
        [Translation Invariance of Brownian Motion]
        [gpt-4.1]
        
$\{B_t - B_0, t \geq 0\}$ 独立于 $B_0$,并且与以 $B_0 = 0$ 为起点的布朗运动具有相同的分布.

    \end{ppt}
    
    
    
    \begin{thm}
        [Theorem-on-Martingale-Difference-Array-Convergence-to-Brownian-Motion]
        {鞅差阵列收敛于布朗运动的定理}
        [Theorem on Martingale Difference Array Convergence to Brownian Motion]
        [gpt-4.1]
        
设
\[
V_{n, k} = \sum_{1 \leq m \leq k} E(X_{n, m}^2 | \mathcal{F}_{n, m-1})
\]
假设 $\{ X_{n, m}, \mathcal{F}_{n, m} \}$ 是鞅差阵列.如果

(i) 对每个 $t$,$V_{n, [nt]} \to t$ 依概率收敛;
(ii) 对所有 $m$,$|X_{n, m}| \leq \epsilon_n$,且 $\epsilon_n \to 0$,

则 $S_{n, (n \cdot)} \Rightarrow B(\cdot)$.

    \end{thm}
    
    
    
    \begin{ppt}
        [Property-of-Constancy-of-the-Limit]
        {极限常数性的性质}
        [Property of Constancy of the Limit]
        [gpt-4.1]
        枚举所有边为 $e_1, e_2, \ldots$,可以观察到 $X$ 关于序列 $\tau(e_1), \tau(e_2), \ldots$ 的尾 $\sigma$-域是可测的,从而极限是常数.
    \end{ppt}
    
    
    
    \begin{ppt}
        [Property-of-Weakening-the-Finite-First-Moment-Condition]
        {弱化有限一阶矩条件的性质}
        [Property of Weakening the Finite First Moment Condition]
        [gpt-4.1]
        假设 $\tau$ 的分布为 $F$,且
\[
\int_{0}^{\infty} (1 - F(x))^{2d} dx < \infty
\]
即 $2d$ 个独立同分布变量的最小值具有有限均值,则可以通过从 $0$ 到 $u = (1, 0, \ldots, 0)$ 找到 $2d$ 条不相交路径,推出 $E\tau(0, u) < \infty$,并且可以应用 (6.1).
    \end{ppt}
    
    
    
    \begin{ppt}
        [Property-of-Necessity-of-the-Condition]
        {条件必要性的性质}
        [Property of Necessity of the Condition]
        [gpt-4.1]
        条件 $(*)$ 对于 $X_{0, n} / n$ 收敛到有限极限来说也是必要的.如果 $(*)$ 不成立,设 $Y_n$ 为所有从 $
u$ 出发的边的 $\tau(e)$ 的最小值,则有
\[
\operatorname*{lim}_{n \to \infty} \operatorname*{sup} X_{0, n} / n \geq \operatorname*{lim}_{n \to \infty} \operatorname*{sup} Y_n / n = \infty \quad \mathrm{a.s.}
\]
    \end{ppt}
    
    
    
    \begin{thm}
        [Equivalence-of-Conditional-Expectations-with-Respect-to-Two-Sigma-Algebras]
        {条件期望在两个σ-代数下的等价性}
        [Equivalence of Conditional Expectations with Respect to Two Sigma-Algebras]
        [gpt-4.1]
        
如果 $Z \in \mathcal{C}$ 且有界,则对所有 $s \geq 0$ 和 $\boldsymbol{x} \in \mathbf{R}^{d}$,有
\[
E_{x} ( Z | \mathcal{F}_{s}^{+} ) = E_{x} ( Z | \mathcal{F}_{s}^{o} )
\]

    \end{thm}
    
    
    
    \begin{prf}
        [Proof-of-the-Equivalence-of-Conditional-Expectations]
        {条件期望等价性的证明}
        [Proof of the Equivalence of Conditional Expectations]
        [gpt-4.1]
        
与定理 7.2.1 的证明类似,仅需在
\[
Z = \prod_{m=1}^{n} f_{m} ( B ( t_{m} ) )
\]
且 $f_{m}$ 有界且可测时证明该结论.在此情况下,$Z$ 可写为 $X ( Y \circ \theta_{s} )$,其中 $X \in \mathcal{F}_{s}^{o}$ 且 $Y$ 是 $\mathcal{C}$ 可测的,因此
\[
E_{x} ( Z | \mathcal{F}_{s}^{+} ) = X E_{x} ( Y \circ \theta_{s} | \mathcal{F}_{s}^{+} ) = X E_{B_{s}} Y \in \mathcal{F}_{s}^{o}
\]
证明完毕.

    \end{prf}
    
    
    
    \begin{prf}
        [Proof-of-Bringing-$\exp	heta-B-{s}$-Outside-the-Expectation]
        {关于将 $\exp(	heta B\_{s})$ 提出期望外的证明}
        [Proof of Bringing $\exp(	heta B_{s})$ Outside the Expectation]
        [gpt-4.1]
        证明 将 $\exp(\theta B_{s})$ 提出期望外:

\[
\begin{array}{c}
E_{x}(\exp(\theta B_{t}) | \mathcal{F}_{s}) = \exp(\theta B_{s}) E(\exp(\theta (B_{t} - B_{s})) | \mathcal{F}_{s}) \\
= \exp(\theta B_{s}) \exp(\theta^{2} (t - s)/2)
\end{array}
\]

因为 $B_{t} - B_{s}$ 与 $\mathcal{F}_{s}$ 独立,且服从均值为 $0$、方差为 $t - s$ 的正态分布.

    \end{prf}
    
    
    
    \begin{thm}
        [Laplace-Transform-Formula-for-First-Hitting-Time-of-Brownian-Motion]
        {布朗运动首次到达定值的拉普拉斯变换公式}
        [Laplace Transform Formula for First Hitting Time of Brownian Motion]
        [gpt-4.1]
        设 $T_{a} = \operatorname{inf}\{ t : B_{t} = a \}$,则有
\[
E_{0} \exp(-\lambda T_{a}) = \exp(-a \sqrt{2 \lambda})
\]

    \end{thm}
    
    
    
    \begin{prf}
        [Proof-of-Laplace-Transform-Formula-for-First-Hitting-Time-of-Brownian-Motion]
        {布朗运动首次到达定值的拉普拉斯变换公式的证明}
        [Proof of Laplace Transform Formula for First Hitting Time of Brownian Motion]
        [gpt-4.1]
        证明:定理 7.5.1 和 7.5.6 推出
\[
1 = E_{0} \exp(\theta B(T \wedge t) - \theta^{2} (T_{a} \wedge t) / 2)
\]
令 $\theta = \sqrt{2 \lambda}$,令 $t \to \infty$ 并用有界收敛定理,则
\[
1 = E_{0} \exp(a \sqrt{2 \lambda} - \lambda T_{a})
\]

    \end{prf}
    
    
    
    \begin{thm}
        [Mean-Value-Property-of-Harmonic-Functions]
        {调和函数的均值性质}
        [Mean Value Property of Harmonic Functions]
        [gpt-4.1]
        
若 $h$ 是 $\mathbf{R}^{d}$ 上的调和函数,则对任意 $x \in \mathbf{R}^{d}$ 和任意半径 $r>0$,有
\[
h(x) = \frac{1}{|B(x, r)|} \int_{B(x, r)} h(y) dy
\]
其中 $B(x, r)$ 是以 $x$ 为中心、$r$ 为半径的球,$|B(x, r)|$ 表示其体积.

    \end{thm}
    
    
    
    \begin{thm}
        [Liouville-Theorem-for-Bounded-Harmonic-Functions]
        {有界调和函数常值性}
        [Liouville Theorem for Bounded Harmonic Functions]
        [gpt-4.1]
        
在任意维度中,若 $h$ 是 $\mathbf{R}^{d}$ 上的有界调和函数,则 $h$ 必为常数函数.

    \end{thm}
    
    
    
    \begin{xmp}
        [Example-Constructing-a-Martingale-from-a-Harmonic-Function]
        {由调和函数构造鞅过程的例子}
        [Example: Constructing a Martingale from a Harmonic Function]
        [gpt-4.1]
        
设 $\xi_1, \xi_2, \ldots$ 是在 $B(0,1)$ 上独立同分布的均匀随机变量,$S_0 = x$,$S_n = S_{n-1} + \xi_n$.若 $h$ 是调和函数,则 $X_n = h(S_n)$ 构成一个鞅过程.

    \end{xmp}
    
    
    
    \begin{dfn}
        [Definition-of-Truncated-and-Centered-Variables]
        {截断变量与中心化变量的定义}
        [Definition of Truncated and Centered Variables]
        [gpt-4.1]
        
设
$\hat { X } _ { n , m } = X _ { n , m } 1 _ { ( | X _ { n , m } | > \epsilon _ { n } ) }$,
$\bar { X } _ { n , m } = X _ { n , m } 1 _ { ( | X _ { n , m } | \leq \epsilon _ { n } ) }$,
$\tilde { X } _ { n , m } = \bar { X } _ { n , m } - E ( \bar { X } _ { n , m } | \mathcal { F } _ { n , m - 1 } )$,
其中 $\epsilon_n$ 已按前述方式定义.

    \end{dfn}
    
    
    
    \begin{dfn}
        [Definition-of-Truncation-Parameter]
        {截断参数的定义}
        [Definition of Truncation Parameter]
        [gpt-4.1]
        
令 $\epsilon _ { n } = 1 / m$ 当 $n \in [ N _ { m } , N _ { m + 1 } )$,且 $\epsilon _ { n } = 1$ 当 $n < N _ { 1 }$.

    \end{dfn}
    
    
    
    \begin{lma}
        [Lemma-on-Probability-Upper-Bound]
        {关于概率上界的引理}
        [Lemma on Probability Upper Bound]
        [gpt-4.1]
        
若 $\delta > 0$ 且 $1 / m < \delta$,则对于 $n \in [ N _ { m } , N _ { m + 1 } )$,有
\[
P ( \epsilon _ { n } ^ { - 2 } \hat { V } _ { n } ( \epsilon _ { n } ) > \delta ) \leq P ( m ^ { 2 } \hat { V } _ { n } ( 1 / m ) > 1 / m ) \leq 1 / m
\]

    \end{lma}
    
    
    
    \begin{prf}
        [Proof-of-Lemma-8]
        {引理8的证明}
        [Proof of Lemma 8]
        [gpt-4.1]
        由于 $| \tilde { X } _ { n , m } | \leq 2 \epsilon _ { n }$,我们只需检查定理8中的(ii).为此,我们注意到条件方差公式(定理4.8)意味着
\[
E ( \tilde { X } _ { n , m } ^ { 2 } | \mathcal { F } _ { n , m - 1 } ) = E ( \bar { X } _ { n , m } ^ { 2 } | \mathcal { F } _ { n , m - 1 } ) - E ( \bar { X } _ { n , m } | \mathcal { F } _ { n , m - 1 } ) ^ { 2 }
\]
对第一个项,我们有
\[
E ( \bar { X } _ { n , m } ^ { 2 } | \mathcal { F } _ { n , m - 1 } ) = E ( X _ { n , m } ^ { 2 } | \mathcal { F } _ { n , m - 1 } ) - E ( \hat { X } _ { n , m } ^ { 2 } | \mathcal { F } _ { n , m - 1 } )
\]
对于第二个项,注意到 $E ( X _ { n , m } | \mathcal { F } _ { n , m - 1 } ) = 0$,由此有
\[
E ( \bar { X } _ { n , m } | \mathcal { F } _ { n , m - 1 } ) ^ { 2 } = E ( \hat { X } _ { n , m } | \mathcal { F } _ { n , m - 1 } ) ^ { 2 } \leq E ( \hat { X } _ { n , m } ^ { 2 } | \mathcal { F } _ { n , m - 1 } )
\]
由 Jensen 不等式,因此由(a)和(i)可知
\[
\sum _ { m = 1 } ^ { [ n t ] } E ( { \tilde { X } } _ { n , m } ^ { 2 } | { \mathcal { F } } _ { n , m - 1 } ) \to t \quad \mathrm{for~all~} t \in [ 0 , 1 ]
\]

    \end{prf}
    
    
    
    \begin{thm}
        [Proof-of-Convergence-in-$L^1$]
        {关于 $L^1$ 收敛性的证明}
        [Proof of Convergence in $L^1$]
        [gpt-4.1]
        只需证明在 $L^1$ 下收敛.令 $\Gamma_m = A_m / m$ 为 (6.4.4) 中的极限,记 $E \Gamma_m = E( X_{0,m} / m )$,并设 $\Gamma = \inf \Gamma_m$.由 $|z| = 2 z^+ - z$(分两种情况 $z \geq 0$ 和 $z < 0$),可得
\[
E \left| \frac{X_{0,n}}{n} - \Gamma \right| = 2 E \left( \frac{X_{0,n}}{n} - \Gamma \right)^+ - E \left( \frac{X_{0,n}}{n} - \Gamma \right) \le 2 E \left( \frac{X_{0,n}}{n} - \Gamma \right)^+
\]
因为
\[
E \left( \frac{X_{0,n}}{n} \right) \ge \gamma = \inf E \Gamma_m \ge E \Gamma
\]
利用显然的不等式 $(x + y)^+ \leq x^+ + y^+$ 并注意到 $\Gamma_m \geq \Gamma$,得到
\[
E \left( \frac{X_{0,n}}{n} - \Gamma \right)^+ \leq E \left( \frac{X_{0,n}}{n} - \Gamma_m \right)^+ + E( \Gamma_m - \Gamma )
\]
又因 $E \Gamma_m \to \gamma$ 当 $m \to \infty$ 且由步骤2和3有 $E \Gamma \geq E \overline{ X } \geq E \underline{ X } \geq \gamma$,故 $E \Gamma = \gamma$,从而 $E( \Gamma_m - \Gamma )$ 在 $m$ 大时很小.为估计另一项,注意 (i) 推出
\[
\begin{aligned}
& E \left( \frac{X_{0,n}}{n} - \Gamma_m \right)^+ \leq E \left( \frac{ X(0, m) + \cdots + X( (k-1)m, k m ) }{ k m + \ell } - \Gamma_m \right)^+ \\
& \qquad + E \left( \frac{ X(k m, n) }{ n } \right)^+
\end{aligned}
\]

    \end{thm}
    
    
    
    \begin{thm}
        [Theorem-on-Sum-Distribution-and-Stopping-Times]
        {关于和的分布与停时的定理}
        [Theorem on Sum Distribution and Stopping Times]
        [gpt-4.1]
        设 $X_1, X_2, \ldots$ 是独立同分布(i.i.d.)的随机变量,分布为 $F$,均值为 $\boldsymbol{\theta}$,方差为 $I$,定义 $S_n = X_1 + \cdots + X_n$.则存在一列停时 $T_0 = 0, T_1, T_2, \dots$,使得 $S_n$ 与 $B(T_n)$ 同分布,且 $T_n - T_{n-1}$ 彼此独立同分布.

    \end{thm}
    
    
    
    \begin{thm}
        [Probability-Calculation-for-Martingales-and-Stopping-Times]
        {关于鞅和停时的概率计算}
        [Probability Calculation for Martingales and Stopping Times]
        [gpt-4.1]
        
(a) 由于 $S_n$ 和 $\xi_{n+1}$ 独立,例 4.1.7 给出,在 $\{S_n = m\}$ 上,
\[
\begin{array}{l}
\displaystyle E(\varphi(S_{n+1}) | \mathcal{F}_n) = p \cdot \left( \frac{1 - p}{p} \right)^{m+1} + (1 - p) \left( \frac{1 - p}{p} \right)^{m-1} \\
\displaystyle = \{1 - p + p\} \left( \frac{1 - p}{p} \right)^m = \varphi(S_n)
\end{array}
\]
因此证明了 (a).

(b) 令 $N = T_a \land T_b$,由于 $\varphi(S_{N \wedge n})$ 有界,根据定理 4.8.2,
\[
\varphi(x) = E \varphi(S_N) = P_x(T_a < T_b) \varphi(a) + [1 - P_x(T_a < T_b)] \varphi(b)
\]
整理后可得 $P_x(T_a < T_b)$ 的公式,减去 1 得到 $P_x(T_b < T_a)$ 的公式.

(c) 令 $b \to \infty$ 且有 $\varphi(b) \to 0$,因 $T_a < \infty$ 当且仅当 $T_a < T_b$ 对某个 $b$ 成立,可得 (c) 的结论.

(d) 为证明 (d) 首先注意到 $\varphi(a) \to \infty$ 当 $a \to -\infty$,所以 $P(T_b < \infty) = 1$.其次,令 $X_n = S_n - (p - q)n$,则 $X_n$ 是鞅.由于 $T_b \land n$ 是有界停时,定理 4.4.1 给出
\[
0 = E \left( S_{T_b \wedge n} - (p - q)(T_b \wedge n) \right)
\]
由于 $b \geq S_{T_b \wedge n} \geq \min_m S_m$ 且 (c) 可知 $E(\inf_m S_m) > -\infty$,由主导收敛定理可得 $E S_{T_b \wedge n} \to E S_{T_b}$ 当 $n \to \infty$.由单调收敛定理可得 $E(T_b \land n) \uparrow E T_b$,因此 $b = (p - q) E T_b$.

    \end{thm}
    
    
    
    \begin{lma}
        [Lemma-on-$p^{m}x-x->-0$-When-Greatest-Common-Divisor-is-1]
        {关于当最大公约数为1时$p^{m}(x, x) > 0$的引理}
        [Lemma on $p^{m}(x, x) > 0$ When Greatest Common Divisor is 1]
        [gpt-4.1]
        如果 $d_{x} = 1$,则对于 $m \geq m_{0}$,有 $p^{m}(x, x) > 0$.
    \end{lma}
    
    
    
    \begin{xmp}
        [An-Example-of-Renewal-Chain]
        {关于更新链的一个例子}
        [An Example of Renewal Chain]
        [gpt-4.1]
        考虑更新链(例子 5.5.8),其中 $f_{5} = f_{12} = 1/2$,且 $5, 12 \in I_{0}$.
由于 $m, n \in I_{0}$ 蕴含 $m+n \in I_{0}$,
\[
I_{0} = \{ 5, 10, 12, 15, 17, 20, 22, 24, 25, 27, 29, 30, 32, \ldots \}
\]
例如,5 可以得到 $10 = 5+5$ 和 $17 = 5+12$,10 可以得到 15 和 22,12 可以得到 17 和 24,依此类推.
一旦我们在 $I_{0}$ 中得到五个连续的数,例如 39–43,则其余的数也都可以得到.
    \end{xmp}
    
    
    
    \begin{prf}
        [Proof-that-$I-x$-Contains-Two-Consecutive-Integers]
        {关于$I\_{x}$包含两个连续整数的证明}
        [Proof that $I_x$ Contains Two Consecutive Integers]
        [gpt-4.1]
        我们只需证明 $I_{x}$ 会包含两个连续整数:$k$ 和 $k+1$.
因为这样就会包含 $2k, 2k+1, 2k+2, 3k, 3k+1, 3k+2, 3k+3$,依此类推,直到我们有 $k$ 个连续整数,然后我们有 $(k-1)k, (k-1)k+1, \ldots, (k-1)k+k-1$,其余的数也都可以得到.
为了证明存在两个连续整数,我们利用数论中的一个事实:如果一个集合 $I_{x}$ 的最大公约数为 $1$,则存在整数 $i_{1}, \ldots, i_{m} \in I_{x}$ 及(正或负的)整数系数 $c_{i}$,使得 $c_{1}i_{1} + \cdots + c_{m}i_{m} = 1$.
令 $a_{i} = c_{i}^{+}$,$b_{i} = (c_{i})^{-}$,即 $a_{i}$ 是正系数,$b_{i}$ 是负系数的相反数.
重写上式得到
\[
a_{1}i_{1} + \cdots + a_{m}i_{m} = (b_{1}i_{1} + \cdots + b_{m}i_{m}) + 1
\]
这样我们就在 $I_{x}$ 中找到了一对连续整数.
在数值例子中,$5 \cdot 5 - 2 \cdot 12 = 1$,这给出了连续整数 24 和 25.
    \end{prf}
    
    
    
    \begin{prf}
        [Proof-of-Second-Order-Expansion-and-Vanishing-Terms-in-the-Limit]
        {关于二阶展开及极限项消失的证明}
        [Proof of Second-Order Expansion and Vanishing Terms in the Limit]
        [gpt-4.1]
        对于该结论,最简洁完全严密的证明方式是先证明当 $f$ 是关于 $t$ 和 $x$ 的多项式时结论成立,然后利用如下事实:对任意 $f \in C^{2}$ 及 $M < \infty$,可以找到多项式 $\phi_{n}$,使得 $\phi_{n}$ 及其所有二阶以下偏导数在 $[0, t] \times [-M, M]$ 上一致收敛于 $f$ 及其对应的偏导数.
    \end{prf}
    
    
    
    \begin{lma}
        [Convergence-of-Conditional-Expectation-for-L1-Convergent-Random-Variables]
        {L1收敛的随机变量条件期望的收敛性}
        [Convergence of Conditional Expectation for L1 Convergent Random Variables]
        [gpt-4.1]
        
如果可积随机变量 $X_n \to X$ 在 $L^1$ 中收敛,则有
\[
E ( X _ { n } ; A ) \to E ( X ; A )
\]
并且
\[
| E X _ { m } 1 _ { A } - E X 1 _ { A } | \leq E | X _ { m } 1 _ { A } - X 1 _ { A } | \leq E | X _ { m } - X |
\]

    \end{lma}
    
    
    
    \begin{lma}
        [Representation-of-L1-Convergent-Martingale]
        {L1收敛的鞅的表示}
        [Representation of L1 Convergent Martingale]
        [gpt-4.1]
        
如果鞅 $X_n \to X$ 在 $L^1$ 中收敛,则有 $X_n = E ( X | \mathcal{F}_n )$.

    \end{lma}
    
    
    
    \begin{prf}
        [Proof-of-Representation-of-L1-Convergent-Martingale]
        {L1收敛鞅的表示的证明}
        [Proof of Representation of L1 Convergent Martingale]
        [gpt-4.1]
        
鞅性质意味着若 $m > n$,则 $E ( X_m | \mathcal{F}_n ) = X_n$,因此若 $A \in \mathcal{F}_n$,有 $E ( X_n ; A ) = E ( X_m ; A )$.由引理 4.6.5 可知 $E ( X_m ; A ) \to E ( X ; A )$,所以对于所有 $A \in \mathcal{F}_n$ 有 $E ( X_n ; A ) = E ( X ; A )$.结合条件期望的定义,得 $X_n = E ( X | \mathcal{F}_n )$.

    \end{prf}
    
    
    
    \begin{dfn}
        [Definition-of-Vertex-Set-of-Rectangle]
        {矩形顶点集合的定义}
        [Definition of Vertex Set of Rectangle]
        [gpt-4.1]
        
设 $A = (a_1, b_1] \times \cdots \times (a_d, b_d]$,则
\[
V = \{ a_1, b_1 \} \times \cdots \times \{ a_d, b_d \}
\]
其中 $V$ 是矩形 $A$ 的顶点集合.

    \end{dfn}
    
    
    
    \begin{dfn}
        [Definition-of-Vertex-Sign-Function]
        {顶点符号函数的定义}
        [Definition of Vertex Sign Function]
        [gpt-4.1]
        
若 $
u \in V$,定义
\[
\operatorname{sgn} ( 
u ) = ( -1 )^{ \# \text{ of } a\text{'s in } 
u }
\]
即 $\operatorname{sgn} ( 
u )$ 等于 $-1$ 的 $a$ 的个数次幂.

    \end{dfn}
    
    
    
    \begin{thm}
        [Application-of-Inclusion-Exclusion-Formula-to-Distribution-Functions]
        {包含-排除公式在概率分布函数上的应用}
        [Application of Inclusion-Exclusion Formula to Distribution Functions]
        [gpt-4.1]
        
包含-排除公式说明
\[
P ( X \in A ) = \sum_{ 
u \in V } \operatorname{sgn} ( 
u ) F( 
u )
\]
其中 $V$ 为矩形 $A$ 的顶点集合,$\operatorname{sgn}(
u)$ 为顶点符号函数,$F(
u)$ 为分布函数在顶点处的取值.

    \end{thm}
    
    
    
    \begin{dfn}
        [Definition-of-the-Increment-Operator-on-Rectangle]
        {矩形上的增量算子的定义}
        [Definition of the Increment Operator on Rectangle]
        [gpt-4.1]
        
若用 $\Delta_A F$ 表示右侧的求和式,则
\[
\Delta_A F = \sum_{ 
u \in V } \operatorname{sgn} ( 
u ) F( 
u )
\]
其中 $A$ 为指定的矩形,$V$ 为其顶点集合.

    \end{dfn}
    
    
    
    \begin{ppt}
        [Non-negativity-Condition-for-Probability-Measures]
        {概率测度的非负性条件}
        [Non-negativity Condition for Probability Measures]
        [gpt-4.1]
        
对于所有矩形 $A$,需要满足
\[
\Delta_A F \geq 0
\]
该条件保证赋予每个矩形的测度不小于 $0$.

    \end{ppt}
    
    
    
    \begin{thm}
        [Time-Reversal-of-Brownian-Motion]
        {布朗运动的时间反转}
        [Time Reversal of Brownian Motion]
        [gpt-4.1]
        
如果 $B_{t}$ 是始于 $\boldsymbol{\theta}$ 的布朗运动,则由 $X_{0}=0$ 及 $X_{t}=t B(1/t)$($t > 0$)定义的过程 $X_{t}$ 也是始于 0 的布朗运动.

    \end{thm}
    
    
    
    \begin{prf}
        [Proof-of-Time-Reversal-of-Brownian-Motion]
        {布朗运动时间反转的证明}
        [Proof of Time Reversal of Brownian Motion]
        [gpt-4.1]
        
我们将检验布朗运动的第二个定义.

为此,我们注意到:
(i) 若 $0 < t_1 < \cdots < t_n$,则 $(X(t_1),\ldots,X(t_n))$ 是均值为 0 的多元正态分布.
(ii) $E X_s = 0$,且若 $s<t$,则
\[
E(X_s X_t) = s t E(B(1/s) B(1/t)) = s
\]
(iii) $X$ 在 $t 
eq 0$ 时显然连续.对于 $t=0$,强大数定律蕴含 $B_n/n \to 0$ 当 $n \to \infty$($n$ 取整数).对于非整数点,Kolmogorov 不等式(定理 2.5.5)蕴含
\[
P \left( \sup_{0 < k \le 2^m} | B(n + k 2^{-m}) - B_n | > n^{2/3} \right) \le n^{-4/3} E ( B_{n+1} - B_n )^{2}
\]
令 $m \to \infty$,得到
\[
P \left( \sup_{u \in [n, n+1]} | B_u - B_n | > n^{2/3} \right) \leq n^{-4/3}
\]
由于 $\sum_{n} n^{-4/3} < \infty$,Borel-Cantelli 引理蕴含 $B_u/u \to 0$ 当 $u \to \infty$.

取 $u=1/t$,于是 $X_t \to 0$ 当 $t \to 0$.

    \end{prf}
    
    
    
    \begin{thm}
        [Application-of-Ergodic-Theorem-to-Recurrence-of-Stationary-Sequences]
        {遍历性定理在鞅序列递归性中的应用}
        [Application of Ergodic Theorem to Recurrence of Stationary Sequences]
        [gpt-4.1]
        
设 $X_1, X_2, \dots$ 是取值于 $\mathbf{R}^d$ 的平稳序列,$S_k = X_1 + \cdots + X_k$,$A = \{ S_k 
eq 0 \text{ for all } k \geq 1 \}$,$R_n = |\{ S_1, \ldots, S_n\}|$ 表示时刻 $n$ 已访问过的点的个数.则有:

定理 6.3.1 当 $n \to \infty$ 时,$R_n / n \to E(1_A | \mathcal{I})$ 几乎处处成立(a.s.).

    \end{thm}
    
    
    
    \begin{thm}
        [Theorem-on-$T-a$-Having-Stationary-Independent-Increments]
        {关于$T\_a$具有平稳独立增量的定理}
        [Theorem on $T_a$ Having Stationary Independent Increments]
        [gpt-4.1]
        在概率测度 $P_0$ 下,$\{T_a, a \ge 0\}$ 具有平稳独立增量.
    \end{thm}
    
    
    
    \begin{prf}
        [Proof-of-$T-a$-Having-Stationary-Independent-Increments]
        {关于$T\_a$具有平稳独立增量的证明}
        [Proof of $T_a$ Having Stationary Independent Increments]
        [gpt-4.1]
        首先注意到当 $0 < a < b$ 时,有
\[
T_b \circ \theta_{T_a} = T_b - T_a,
\]
因此若 $f$ 有界且可测,利用强马尔可夫性质、7.3.9 及平移不变性可得
\[
\begin{aligned}
& E_0\left( f(T_b - T_a) \mid \mathcal{F}_{T_a} \right) = E_0\left( f(T_b) \circ \theta_{T_a} \mid \mathcal{F}_{T_a} \right) \\
& \qquad = E_a f(T_b) = E_0 f(T_{b - a})
\end{aligned}
\]

为证明增量独立,令 $a_0 < a_1 < \ldots < a_n$,$f_i$ ($1 \leq i \leq n$) 为有界可测函数,记 $F_i = f_i(T_{a_i} - T_{a_{i-1}})$.对 $\mathcal{F}_{T_{a_{n-1}}}$ 条件化并用上述计算,有
\[
E_0\left( \prod_{i=1}^n F_i \right) = E_0\left( \prod_{i=1}^{n-1} F_i \cdot E_0( F_n \mid \mathcal{F}_{T_{a_{n-1}}} ) \right) = E_0\left( \prod_{i=1}^{n-1} F_i \right) E_0 F_n
\]
归纳得到 $E_0 \prod_{i=1}^n F_i = \prod_{i=1}^n E_0 F_i$,即得所需结论.

    \end{prf}
    
    
    
    \begin{ppt}
        [Scaling-Relation-for-$T-a$]
        {关于$T\_a$的缩放关系}
        [Scaling Relation for $T_a$]
        [gpt-4.1]
        缩放关系 (7.1.1) 蕴含
\[
T_a \overset{d}{=} a^2 T_1
\]
即 $T_a$ 的分布与 $a^2 T_1$ 相同.

    \end{ppt}
    
    
    
    \begin{dfn}
        [Definition-of-First-Passage-Percolation-Model]
        {第一通道渗流模型的定义}
        [Definition of First Passage Percolation Model]
        [gpt-4.1]
        
在 $\mathbf{Z}^d$ 上,将其视为一个图,其中每两个满足 $|x - y| = 1$ 的 $x, y \in \mathbf{Z}^d$ 之间有一条边.对每条边 $e$ 赋予一个独立的非负随机变量 $\tau(e)$,表示沿该边任一方向通过所需的时间.若 $e$ 是连接 $x$ 和 $y$ 的边,则 $\tau(x, y) = \tau(y, x) = \tau(e)$.

若 $x_0 = x, x_1, \ldots, x_n = y$ 是从 $x$ 到 $y$ 的一条路径,即满足 $|x_m - x_{m-1}| = 1$ ($1 \leq m \leq n$) 的序列,则该路径的通过时间定义为 $\tau(x_0, x_1) + \cdots + \tau(x_{n-1}, x_n)$.

定义从 $x$ 到 $y$ 的通过时间 $t(x, y)$ 为所有从 $x$ 到 $y$ 路径的通过时间的下确界(infimum).

    \end{dfn}
    
    
    
    \begin{ppt}
        [Triangle-Inequality-Property-of-Passage-Times]
        {通过时间的三角不等式性质}
        [Triangle Inequality Property of Passage Times]
        [gpt-4.1]
        
对于 $u = (1, 0, \ldots, 0)$,令 $X_{m, n} = t(m u, n u)$,则有 $X_{0, m} + X_{m, n} \geq X_{0, n}$.

    \end{ppt}
    
    
    
    \begin{dfn}
        [Definition-of-the-Maximum-Function]
        {最大值函数的定义}
        [Definition of the Maximum Function]
        [gpt-4.1]
        设 $\psi(\omega) = \max\{\omega(t) : 0 \leq t \leq 1\}$,其中 $\omega \in C[0, 1]$.
    \end{dfn}
    
    
    
    \begin{thm}
        [Limit-Theorem-for-Maximum-Partial-Sums]
        {极限定理关于最大部分和}
        [Limit Theorem for Maximum Partial Sums]
        [gpt-4.1]
        假设 $\psi : C[0, 1] \to \mathbb{R}$ 是连续的,根据定理 8.1.5,
\[
\max_{0 \leq m \leq n} S_m / \sqrt{n} \Rightarrow M_1 \equiv \max_{0 \leq t \leq 1} B_t
\]
其中 $S_m$ 为部分和,$B_t$ 为标准布朗运动.
    \end{thm}
    
    
    
    \begin{ppt}
        [Distribution-Property-of-Brownian-Motion-Maximum]
        {布朗运动最大值的分布性质}
        [Distribution Property of Brownian Motion Maximum]
        [gpt-4.1]
        根据 (7.4.4),右侧 $M_1$ 的分布为
\[
P_0(M_1 \geq a) = P_0(T_a \leq 1) = 2P_0(B_1 \geq a)
\]
其中 $T_a$ 是首次到达 $a$ 的时间,$B_1$ 是布朗运动在 $t=1$ 时的取值.
    \end{ppt}
    
    
    
    \begin{prf}
        [Proof-of-Lemma-8.1.9]
        {引理8.1.9的证明}
        [Proof of Lemma 8.1.9]
        [gpt-4.1]
        设 $B$ 是具有连续路径的过程(因此在 $[0,1]$ 上一致连续),则对任意 $\epsilon > 0$,存在 $\delta > 0$,且 $1/\delta$ 是整数,使得
\[
P( | B_{t} - B_{s} | < \epsilon \text{ for all } 0 \leq s \leq 1, |t - s| < 2\delta ) > 1 - \epsilon
\]

由引理8.1.9的假设可知,若 $n \geq N_{\delta}$,则
\[
P( | \tau_{[nk\delta]}^{n} - k\delta | < \delta \quad \text{for } k = 1, 2, \dots, 1/\delta ) \geq 1 - \epsilon
\]

由于 $m \mapsto \tau_{m}^{n}$ 单调递增,若 $s \in ((k-1)\delta, k\delta)$,则
\[
\begin{array}{ll}
    \tau_{[ns]}^{n} - s \geq \tau_{[n(k-1)\delta]}^{n} - (k-1)\delta \\
    \tau_{[ns]}^{n} - s \leq \tau_{[nk\delta]}^{n} - k\delta
\end{array}
\]
因此当 $n \geq N_{\delta}$ 时,
\[
P \left( \sup_{0 \leq s \leq 1} | \tau_{[ns]}^{n} - s | < 2\delta \right) \geq 1 - \epsilon
\]

当(a)和(b)中的事件发生时,
\[
| S_{n,m} - B_{m/n} | < \epsilon \quad \text{for all } m \leq n
\]

对于 $t = (m + \theta)/n$,其中 $0 < \theta < 1$,有
\[
\begin{array}{c}
    | S_{n, (nt)} - B_{t} | \leq (1-\theta) | S_{n,m} - B_{m/n} | + \theta | S_{n,m+1} - B_{(m+1)/n} | \\
    + (1-\theta) | B_{m/n} - B_{t} | + \theta | B_{(m+1)/n} - B_{t} |
\end{array}
\]

利用(c)对前两项和(a)对后两项,有 $n \geq N_{\delta}$ 且 $1/n < 2\delta$ 时,$\| S_{n, (n \cdot)} - B(\cdot) \| < 2\epsilon$,概率 $\geq 1 - 2\epsilon$.由于 $\epsilon$ 可任意取,故引理8.1.9得证.

    \end{prf}
    
    
    
    \begin{ppt}
        [Closure-of-Operations-for-Stopping-Times]
        {停时的运算闭性}
        [Closure of Operations for Stopping Times]
        [gpt-4.1]
        如果 $S$ 和 $T$ 是停时(stopping times),那么 $S \wedge T = \min\{S, T\}$、$S \vee T = \max\{S, T\}$ 以及 $S + T$ 也是停时.

特别地,如果 $t \geq 0$,则 $S \wedge t$、$S \vee t$ 以及 $S + t$ 也是停时.
    \end{ppt}
    
    
    
    \begin{dfn}
        [Definition-of-Partition-and-Related-Functions]
        {分割和相关函数的定义}
        [Definition of Partition and Related Functions]
        [gpt-4.1]
        设
\[
I_{m}^{1} \equiv \sum_{i} f'(B_{t_{i}^{m}}) (B_{t_{i+1}^{m}} - B_{t_{i}^{m}})
\]
其中 $t_{i}^{m}$ 表示对区间 $[0, t]$ 的 $2^m$ 等分点.为将区间长度进一步细分为 $t 2^{-n}$,定义
\[
i(n, m, j) = [2^{m} j / 2^{n}] / 2^{m}
\]
和
\[
H_{j}^{m} = f'(B_{i(n, m, j)})
\]
以及
\[
K_{j}^{n} = f'(B_{t_{j}^{n}})
\]

    \end{dfn}
    
    
    
    \begin{ppt}
        [Vanishing-Property-of-Expectation]
        {期望消失性质}
        [Vanishing Property of Expectation]
        [gpt-4.1]
        对第二个求和项,若对 $\mathcal{F}_{t_{k}^{n}}$ 取条件期望,则有
\[
(H_{j}^{m} - K_{j}^{n}) (H_{k}^{m} - K_{k}^{n}) (B_{t_{j+1}^{n}} - B_{t_{j}^{n}}) E(B_{t_{k+1}^{n}} - B_{t_{k}^{n}} | \mathcal{F}_{t_{k}^{n}}) = 0
\]

    \end{ppt}
    
    
    
    \begin{ppt}
        [Expectation-of-Squared-Difference-Expression]
        {均方期望表达式}
        [Expectation of Squared Difference Expression]
        [gpt-4.1]
        对第一个求和项,有
\[
E[(I_{m}^{1} - I_{n}^{1})^{2}] = \sum_{j} E[(H_{j}^{m} - K_{j}^{n})^{2}] \cdot 2^{-n} t
\]

    \end{ppt}
    
    
    
    \begin{ppt}
        [Upper-Bound-Estimate-of-Difference]
        {差值上界估计}
        [Upper Bound Estimate of Difference]
        [gpt-4.1]
        若 $|i(n, m, j) / 2^{m} - j / 2^{n}| \leq 1 / 2^{m}$ 且 $|f''(x)| \le K$,则
\[
\sup_{j \leq 2^{n}} |H_{j}^{m} - K_{j}^{n}| \leq K \sup \{ |B_{s} - B_{r}| : 0 \leq r \leq s \leq t , |s - r| \leq 2^{-m} \}
\]

    \end{ppt}
    
    
    
    \begin{dfn}
        [Definition-of-Transition-Probability]
        {转移概率的定义}
        [Definition of Transition Probability]
        [gpt-4.1]
        设 $(S, \mathcal{S})$ 是一个可测空间.若函数 $p : S \times \mathcal{S} \to [0,1]$ 满足:
(i) 对每个 $x \in S$,$A \mapsto p(x, A)$ 是 $(S, \mathcal{S})$ 上的概率测度;
(ii) 对每个 $A \in \mathcal{S}$,$x \mapsto p(x, A)$ 是可测函数,
则称 $p$ 是一个转移概率.
    \end{dfn}
    
    
    
    \begin{dfn}
        [Definition-of-Markov-Chain]
        {马尔可夫链的定义}
        [Definition of Markov Chain]
        [gpt-4.1]
        若随机过程 $X_{n}$ 满足对所有 $B \in \mathcal{S}$,
\[
P(X_{n+1} \in B \mid \mathcal{F}_{n}) = p(X_{n}, B)
\]
其中 $p$ 是转移概率,则称 $X_{n}$ 是关于 $\mathcal{F}_{n}$ 的马尔可夫链.
    \end{dfn}
    
    
    
    \begin{ppt}
        [Consistent-Construction-of-Finite-Dimensional-Distributions-for-Markov-Chains]
        {马尔可夫链有限维分布的一致性构造}
        [Consistent Construction of Finite Dimensional Distributions for Markov Chains]
        [gpt-4.1]
        给定转移概率 $p$ 和初始分布 $\mu$,可以定义一组一致的有限维分布:
\[
P(X_{j} \in B_{j}, 0 \le j \le n) = \int_{B_{0}} \mu(dx_{0}) \int_{B_{1}} p(x_{0}, dx_{1}) \cdots \int_{B_{n}} p(x_{n-1}, dx_{n})
\]

    \end{ppt}
    
    
    
    \begin{thm}
        [Application-of-Kolmogorov-Extension-Theorem-to-Markov-Chains]
        {Kolmogorov扩展定理在马尔可夫链中的应用}
        [Application of Kolmogorov Extension Theorem to Markov Chains]
        [gpt-4.1]
        如果 $(S, \mathcal{S})$ 适当,Kolmogorov扩展定理(定理2.1.21)允许我们在序列空间
\[
(\Omega_{0}, \mathcal{F}_{\infty}) = (S^{\{0, 1, \ldots\}}, \mathcal{S}^{\{0, 1, \ldots\}})
\]
上构造概率测度 $P_{\mu}$,使得坐标映射 $X_{n}(\omega) = \omega_{n}$ 具有预期的分布.
    \end{thm}
    
    
    
    \begin{ppt}
        [Relation-Between-Initial-Distribution-and-State-Probability-Measures]
        {初始分布与状态的概率测度关系}
        [Relation Between Initial Distribution and State Probability Measures]
        [gpt-4.1]
        对每个初始分布 $\mu$,概率测度满足
\[
P_{\mu}(A) = \int \mu(dx) P_{x}(A).
\]

    \end{ppt}
    
    
    
    \begin{dfn}
        [Definition-of-Shift-Operator-on-Sequence-Space]
        {序列空间上的移位算子定义}
        [Definition of Shift Operator on Sequence Space]
        [gpt-4.1]
        在序列空间 $(S^{\{0, 1, \ldots\}}, \mathcal{S}^{\{0, 1, \ldots\}})$ 上,移位算子定义为
\[
\theta_{n}(\omega_{0}, \omega_{1}, \ldots) = (\omega_{n}, \omega_{n+1}, \ldots)
\]

    \end{dfn}
    
    
    
    \begin{dfn}
        [Definition-of-the-Maximum-Row-Sum-Norm-of-a-Matrix]
        {矩阵范数的最大行和范数定义}
        [Definition of the Maximum Row Sum Norm of a Matrix]
        [gpt-4.1]
        
令
\[
\|A\| = \max_{i} \sum_{j} |A(i,j)| = \max \{ \| xA \|_{1} : \| x \|_{1} = 1 \} ,
\]
其中 $(xA)_{j} = \sum_{i} x_{i} A(i,j)$ 且 $\| x \|_{1} = | x_{1} | + \cdots + | x_{k} |$.

    \end{dfn}
    
    
    
    \begin{ppt}
        [Inequality-for-the-Product-of-Matrix-Norms]
        {矩阵范数的乘积不等式}
        [Inequality for the Product of Matrix Norms]
        [gpt-4.1]
        
显然有 $\|AB\| \leq \|A\| \cdot \|B\|$.

    \end{ppt}
    
    
    
    \begin{dfn}
        [Definition-of-Subproduct-Norm-and-Logarithmic-Variable]
        {子乘积范数与对数变量的定义}
        [Definition of Subproduct Norm and Logarithmic Variable]
        [gpt-4.1]
        
令
\[
\beta_{m,n} = \| A_{m+1} \cdots A_{n} \|
\]
并且 $Y_{m,n} = \log \beta_{m,n}$,则 $Y_{m,n}$ 是次可加的.

    \end{dfn}
    
    
    
    \begin{thm}
        [Conclusion-from-the-Subadditive-Ergodic-Theorem]
        {次可加遍历定理的应用结论}
        [Conclusion from the Subadditive Ergodic Theorem]
        [gpt-4.1]
        
很容易利用次可加遍历定理得出
\[
\frac{1}{n} \log \| A_{m+1} \cdots A_{n} \| \to -X \quad \mathrm{a.s.}
\]
其中 $X$ 是 $X_{0,n}/n$ 的极限.

    \end{thm}
    
    
    
    \begin{dfn}
        [Space-of-Continuous-Functions-and-σ-field-Generated-by-Finite-Dimensional-Sets]
        {连续函数空间及其有限维生成的σ-域}
        [Space of Continuous Functions and σ-field Generated by Finite Dimensional Sets]
        [gpt-4.1]
        
设 $C[0, \infty ) = \{ \text{continuous } \omega : [ 0, \infty ) \to \mathbf{R} \}$,即所有从 $[0, \infty )$ 到 $\mathbf{R}$ 的连续函数的集合.$\mathcal{C}[0, \infty )$ 定义为由有限维集合生成的 $\sigma$-域.

    \end{dfn}
    
    
    
    \begin{dfn}
        [Probability-Measure-on-Truncated-Path-Space]
        {截断路径得到的概率测度}
        [Probability Measure on Truncated Path Space]
        [gpt-4.1]
        
对于在 $C[0, \infty )$ 上的概率测度 $\mu$,对应的测度 $\pi_{ M }\mu$ 定义在 $C[0, M] = \{ \text{continuous } \omega : [0, M] \to \mathbf{R} \}$ 上(其中 $\mathcal{C}[0, M]$ 为有限维集合生成的 $\sigma$-域),其通过在时间 $M$ 截断路径获得.具体地,定义 $(\psi_{ M }\omega)(t) = \omega(t)$,对所有 $t \in [0, M]$,并令 $\pi_{ M } \mu = \mu \circ \psi_{ M }^{ -1 }$.

    \end{dfn}
    
    
    
    \begin{dfn}
        [Weak-Convergence-of-Probability-Measures]
        {概率测度的弱收敛}
        [Weak Convergence of Probability Measures]
        [gpt-4.1]
        
称概率测度序列 $\mu_{ n }$ 在 $C[0, \infty )$ 上弱收敛于 $\mu$,若对所有 $M$,$\pi_{ M } \mu_{ n }$ 在 $C[0, M]$ 上弱收敛于 $\pi_{ M } \mu$,其中后者的弱收敛为对 $M = 1$ 的定义的平凡推广.

    \end{dfn}
    
    
    
    \begin{dfn}
        [Symmetry-and-Nonnegative-Definiteness-of-Covariance-Matrix]
        {协方差矩阵的对称性与非负定性}
        [Symmetry and Nonnegative Definiteness of Covariance Matrix]
        [gpt-4.1]
        
设随机向量 $X = (X_1, X_2, \ldots, X_d)$ 满足 $E X_i = 0$, 且 $E X_i X_j = \Gamma_{ij}$,则有
\[
\sum_{i} \sum_{j} \theta_i \theta_j \Gamma_{ij} = E\left( \sum_{i} \theta_i X_i \right)^2 \geq 0
\]
因此协方差矩阵 $\Gamma$ 是对称且非负定的.

    \end{dfn}
    
    
    
    \begin{dfn}
        [Nondegenerate-Normal-Distribution]
        {非退化正态分布}
        [Nondegenerate Normal Distribution]
        [gpt-4.1]
        
如果协方差矩阵 $\Gamma$ 的秩为 $d$,则称该正态分布为非退化的.在这种情况下,其密度函数为
\[
(2\pi)^{-d/2} (\operatorname{det} \Gamma)^{-1/2} \exp\left( - \sum_{i,j} y_i \Gamma_{ij}^{-1} y_j / 2 \right )
\]

    \end{dfn}
    
    
    
    \begin{dfn}
        [Definition-of-Stopping-Time]
        {停止时刻的定义}
        [Definition of Stopping Time]
        [gpt-4.1]
        称取值于 $[0, \infty]$ 的随机变量 $S$ 为一个停止时刻(stopping time),如果对所有 $t \geq 0$,都有 $\{ S < t \} \in \mathcal{F}_t$.
    \end{dfn}
    
    
    
    \begin{xmp}
        [Application-of-Limit-Theorem-to-Products-of-Random-Matrices]
        {随机矩阵积的极限定理应用}
        [Application of Limit Theorem to Products of Random Matrices]
        [gpt-4.1]
        
假设 $A_{1}, A_{2}, \ldots$ 是一列具有正元素的 $k \times k$ 矩阵的平稳序列,定义
\[
\alpha_{m,n}(i,j) = ( A_{m+1} \cdots A_{n} )(i,j)
\]
即矩阵积第 $i$ 行第 $j$ 列的元素.
容易验证:
\[
\alpha_{0,m}(1,1) \alpha_{m,n}(1,1) \leq \alpha_{0,n}(1,1)
\]
令 $X_{m,n} = - \log \alpha_{m,n}(1,1)$,则有 $X_{0,m} + X_{m,n} \geq X_{0,n}$.
进一步有:
\[
\prod_{m=1}^{n} A_{m}(1,1) \leq \alpha_{0,n}(1,1) \leq k^{n-1} \prod_{m=1}^{n} \left( \sup_{i,j} A_{m}(i,j) \right)
\]
对数变换得:
\[
- \sum_{m=1}^{n} \log A_{m}(1,1) \geq X_{0,n} \geq - ( n \log k ) - \sum_{m=1}^{n} \log \left( \sup_{i,j} A_{m}(i,j) \right)
\]
若 $E \log A_{m}(1,1) > -\infty$,则 $E X_{0,1}^{+} < \infty$;若
\[
E \log \left( \sup_{i,j} A_{m}(i,j) \right) < \infty
\]
则 $E X_{0,n} \leq \gamma_{0} n$.
又有
\[
P\left( \log\left( \sup_{i,j} A_{m}(i,j) \right) \geq x \right) \leq \sum_{i,j} P \left( \log A_{m}(i,j) \geq x \right)
\]
因此只需假设
\[
E | \log A_{m}(i,j) | < \infty \quad \mathrm{对所有}~ i,j
\]
当上述条件成立时,应用定理 6.4.1 得到 $X_{0,n}/n \to X$ 几乎处处成立.
利用矩阵元素严格为正,可进一步得到
\[
\frac{1}{n} \log \alpha_{0,n}(i,j) \to -X \quad \mathrm{几乎处处,对所有}~ i,j
\]
该结果最早由 Furstenberg 和 Kesten (1960) 证明.

    \end{xmp}
    
    
    
    \begin{thm}
        [Limit-Consistency-Theorem-for-Conditional-Expectation]
        {条件期望的极限一致性定理}
        [Limit Consistency Theorem for Conditional Expectation]
        [gpt-4.1]
        
若 ${\mathcal{F}}_{n} \uparrow \mathcal{F}_{\infty}$ 且 $Y_{n} \to Y$ 在 $L^{1}$ 中收敛,则有
\[
E(Y_{n} \mid \mathcal{F}_{n}) \to E(Y \mid \mathcal{F}_{\infty})
\]
在 $L^{1}$ 中收敛.

    \end{thm}
    
    
    
    \begin{thm}
        [Recursive-Formula-for-Markov-Chain-Visiting-Probability]
        {关于马氏链访问概率的递推公式}
        [Recursive Formula for Markov Chain Visiting Probability]
        [gpt-4.1]
        设 $T_y^k$ 表示马氏链从状态 $x$ 到状态 $y$ 第 $k$ 次访问的时刻,$\rho_{xy}$ 表示从 $x$ 到 $y$ 的一次访问概率,$\rho_{yy}$ 表示从 $y$ 返回自身的一次访问概率,则
\[
P_{x}(T_y^k < \infty) = \rho_{xy} \rho_{yy}^{k-1}
\]
即从 $x$ 到 $y$ 恰好访问 $k$ 次的概率为首次到达概率乘以返回 $y$ 的概率的 $(k-1)$ 次幂.
    \end{thm}
    
    
    
    \begin{prf}
        [Proof-of-Recursive-Formula-for-Markov-Chain-Visiting-Probability]
        {关于马氏链访问概率递推公式的证明}
        [Proof of Recursive Formula for Markov Chain Visiting Probability]
        [gpt-4.1]
        当 $k=1$ 时,结论显然成立.现设 $k \geq 2$.
设随机变量 $Y(\omega) = 1$ 当且仅当 $\omega_n = y$ 对某个 $n \geq 1$,否则为 $0$.
设 $N = T_y^{k-1}$,则有 $Y \circ \theta_N = 1$ 当且仅当 $T_y^k < \infty$.
由强马氏性(定理 5.2.5)有
\[
E_x(Y \circ \theta_N | \mathcal{F}_N) = E_{X_N} Y \quad \text{在}~ \{N < \infty\}
\]
而在 $\{N < \infty\}$ 上,$X_N = y$,因此右侧为 $P_y(T_y < \infty) = \rho_{yy}$,由此得到
\[
\begin{array}{rl}
& P_x(T_y^k < \infty) = E_x(Y \circ \theta_N; N < \infty) \\
& \qquad = E_x(E_x(Y \circ \theta_N | \mathcal{F}_N); N < \infty) \\
& \qquad = E_x(\rho_{yy}; N < \infty) = \rho_{yy} P_x(T_y^{k-1} < \infty)
\end{array}
\]
由归纳法可得结论.
    \end{prf}
    
    
    
    \begin{dfn}
        [Definition-of-Finite-Dimensional-Distributions-of-Brownian-Motion]
        {布朗运动有限维分布的定义}
        [Definition of Finite-Dimensional Distributions of Brownian Motion]
        [gpt-4.1]
        设 $x \in \mathbf{R}$,对于每组 $0 < t _ { 1 } < \cdots < t _ { n }$,在 $\mathbf{R}^n$ 上定义如下测度:

\[
\mu _ { x , t _ { 1 } , \ldots , t _ { n } } ( A _ { 1 } \times \cdots \times A _ { n } ) = \int _ { A _ { 1 } } d x _ { 1 } \cdots \int _ { A _ { n } } d x _ { n } \prod _ { m = 1 } ^ { n } p _ { t _ { m } - t _ { m - 1 } } ( x _ { m - 1 } , x _ { m } )
\]

其中 $A _ { i } \in \mathcal{R}$,$x _ { 0 } = x$,$t _ { 0 } = 0$,且

\[
p _ { t } ( a , b ) = ( 2 \pi t ) ^ { - 1 / 2 } \exp ( - ( b - a ) ^ { 2 } / 2 t )
\]

该测度族 $\mu$ 是关于布朗运动有限维分布的定义.

    \end{dfn}
    
    
    
    \begin{ppt}
        [Consistency-Property-of-Finite-Dimensional-Distributions-of-Brownian-Motion]
        {布朗运动有限维分布的一致性性质}
        [Consistency Property of Finite-Dimensional Distributions of Brownian Motion]
        [gpt-4.1]
        对于固定的 $x$,测度族 $\mu$ 是一致的有限维分布族,即若 $\{ s _ { 1 }, \dotsc, s _ { n - 1 } \} \subset \{ t _ { 1 }, \dotsc, t _ { n } \}$ 且 $t _ { j } 
otin \left\{ s _ { 1 }, \ldots, s _ { n - 1 } \right\}$,则有

\[
\mathscr{L}_{x, s_1, \ldots, s_{n-1}} (A_1 \times \cdots \times A_{n-1}) = \mu_{x, t_1, \ldots, t_n} (A_1 \times \cdots \times A_{j-1} \times \mathbf{R} \times A_j \times \cdots \times A_{n-1})
\]

特别地,当 $j = n$ 时该结论显然成立.

    \end{ppt}
    
    
    
    \begin{lma}
        [Lemma-on-Convolution-of-Normal-Distributions]
        {正态分布卷积的引理}
        [Lemma on Convolution of Normal Distributions]
        [gpt-4.1]
        对于任意 $t_{j-1} < t_j < t_{j+1}$,有

\[
\int p_{t_j - t_{j-1}} (x, y) p_{t_{j+1} - t_j} (y, z) d y = p_{t_{j+1} - t_{j-1}} (x, z)
\]

其中 $p_t(a, b) = (2\pi t)^{-1/2} \exp( - (b - a)^2 / 2t )$.这表明均值为 $0$、方差分别为 $t_j - t_{j-1}$ 和 $t_{j+1} - t_j$ 的独立正态分布之和,仍为均值 $0$、方差 $t_{j+1} - t_{j-1}$ 的正态分布.

    \end{lma}
    
    
    
    \begin{dfn}
        [Definition-of-Age-dependent-Branching-Process]
        {年龄相关分枝过程的定义}
        [Definition of Age-dependent Branching Process]
        [gpt-4.1]
        年龄相关分枝过程(Age-dependent branching process)是一种分枝过程的变体,其中每个个体在产生后存活一段服从分布 $F$ 的时间,然后以概率 $p_k$ 产生 $k$ 个后代.该过程的描述为:过程从第0代的一个个体开始,该个体在时间0出生,当此个体死亡时,其后代各自独立地开始原过程的拷贝.
    \end{dfn}
    
    
    
    \begin{thm}
        [Embedding-Theorem-for-Martingale-Central-Limit-Theorem]
        {关于鞅的中心极限定理的嵌入定理}
        [Embedding Theorem for Martingale Central Limit Theorem]
        [gpt-4.1]
        
如果 $S_{ n }$ 是一个平方可积鞅,且 $S_{ 0 } = 0$,$B_{ t }$ 是布朗运动,则存在一列对布朗运动的停时 $0 = T_{ 0 } \leq T_{ 1 } \leq T_{ 2 } \ldots$,使得
\[
( S_{ 0 }, S_{ 1 }, \ldots, S_{ k } ) \stackrel{ d }{ = } ( B( T_{ 0 } ), B( T_{ 1 } ), \ldots, B( T_{ k } ) ) \quad \text{对所有 } k \geq 0
\]

    \end{thm}
    
    
    
    \begin{thm}
        [Multiplicative-Property-of-Laplace-Transform]
        {拉普拉斯变换的乘法性质}
        [Multiplicative Property of Laplace Transform]
        [gpt-4.1]
        
对于 $a, x, y \geq 0$,有
\[
\varphi_{x}(\lambda)\varphi_{y}(\lambda) = \varphi_{x+y}(\lambda)
\]
其中 $\varphi_{a}(\lambda) = E_{0} \exp(-\lambda T_{a})$.

    \end{thm}
    
    
    
    \begin{thm}
        [Exponential-Expression-for-Laplace-Transform]
        {拉普拉斯变换的指数表达式}
        [Exponential Expression for Laplace Transform]
        [gpt-4.1]
        
对于 $a \geq 0$,有
\[
\varphi_{a}(\lambda) = \exp(-a c(\lambda))
\]
其中 $c(\lambda) = -\log \varphi_{1}(\lambda)$.

    \end{thm}
    
    
    
    \begin{prf}
        [Proof-of-Exponential-Expression-for-Laplace-Transform]
        {拉普拉斯变换指数表达式的证明}
        [Proof of Exponential Expression for Laplace Transform]
        [gpt-4.1]
        
令 $c(\lambda) = -\log \varphi_{1}(\lambda)$,则对于 $a = 1$,有
\[
\varphi_{1}(\lambda) = \exp(-c(\lambda))
\]
利用乘法性质,对 $x = y = 2^{-m}$,通过归纳法可得对于 $a = 2^{-m}$,$m \geq 1$ 成立.进一步,令 $x = k 2^{-m}$,$y = 2^{-m}$,可得 $a = (k+1)2^{-m}$,$k \geq 1$ 的结果.最终,对于 $a \in [0, \infty)$,利用 $\varphi_{a}(\lambda)$ 的单调性即可推广.

    \end{prf}
    
    
    
    \begin{thm}
        [Explicit-Expression-for-Function-$c\lambda$]
        {函数 $c(\lambda)$ 的具体表达式}
        [Explicit Expression for Function $c(\lambda)$]
        [gpt-4.1]
        
有
\[
E \exp(-T_{a}) = E \exp(-a^{2} T_{1})
\]
因此 $a c(1) = c(a^{2})$,即 $c(\lambda) = c(1)\sqrt{\lambda}$.

    \end{thm}
    
    
    
    \begin{dfn}
        [Basic-Definition-of-the-Wright–Fisher-Model]
        {Wright–Fisher模型的基本定义}
        [Basic Definition of the Wright–Fisher Model]
        [gpt-4.1]
        
Wright–Fisher模型考虑一个由 $N/2$ 个二倍体个体(每个个体有两份染色体)或 $N$ 个单倍体个体(每个个体有一份染色体)组成的固定种群,总共有 $N$ 个基因,它们可以是两种类型之一:$A$ 或 $a$.
在该模型的最简单版本中,第 $n+1$ 代的种群通过对第 $n$ 代的种群有放回地抽取得到.

    \end{dfn}
    
    
    
    \begin{dfn}
        [Transition-Probability-in-the-Wright–Fisher-Model]
        {Wright–Fisher模型中的转移概率}
        [Transition Probability in the Wright–Fisher Model]
        [gpt-4.1]
        
设 $X_n$ 表示第 $n$ 代中 $A$ 等位基因的数量,则 $X_n$ 构成一个马尔可夫链,其转移概率为
\[
p(i, j) = \binom{N}{j} \left( \frac{i}{N} \right)^{j} \left(1 - \frac{i}{N} \right)^{N - j}
\]
其中右侧是 $N$ 次独立试验中成功概率为 $i/N$ 的二项分布.

    \end{dfn}
    
    
    
    \begin{dfn}
        [Absorbing-States-in-the-Wright–Fisher-Model]
        {Wright–Fisher模型的吸收态}
        [Absorbing States in the Wright–Fisher Model]
        [gpt-4.1]
        
在该模型中,$x = 0$ 和 $x = N$(对应于全部为 $a$ 或全部为 $A$ 的种群)是吸收态,即 $p(x, x) = 1$.

    \end{dfn}
    
    
    
    \begin{dfn}
        [Wright–Fisher-Model-with-Mutation]
        {引入突变后的Wright–Fisher模型}
        [Wright–Fisher Model with Mutation]
        [gpt-4.1]
        
为使该模型更有趣,引入突变:即被抽中的 $A$ 基因在下一代以概率 $u$ 变为 $a$,而被抽中的 $a$ 基因在下一代以概率 $
u$ 变为 $A$.
此时,每次抽取产生 $A$ 的概率为
\[
\rho_{i} = \frac{i}{N}(1-u) + \frac{N-i}{N}
u
\]
但转移概率仍为二项式形式:
\[
p(i, j) = \binom{N}{j} (\rho_i)^j (1 - \rho_i)^{N-j}
\]
如果 $u$ 和 $
u$ 都为正,则 $0$ 和 $N$ 不再是吸收态,系统可以收敛到一个平衡分布($t \to \infty$ 时).

    \end{dfn}
    
    
    
    \begin{thm}
        [Convergence-Theorem-in-the-Periodic-Case]
        {周期情况下的收敛定理}
        [Convergence Theorem in the Periodic Case]
        [gpt-4.1]
        假设 $p$ 是不可约的,具有平稳分布 $\pi$,且所有状态的周期为 $d$.设 $x \in S$,$S_{0}, S_{1}, \ldots, S_{d-1}$ 是状态空间的循环分解,且 $x \in S_{0}$.若 $y \in S_{r}$,则有
\[
\lim_{m \to \infty} p^{md + r}(x, y) = \pi(y) d
\]

    \end{thm}
    
    
    
    \begin{prf}
        [Proof-of-Convergence-Theorem-in-the-Periodic-Case]
        {周期情况下收敛定理的证明}
        [Proof of Convergence Theorem in the Periodic Case]
        [gpt-4.1]
        若 $y \in S_{0}$,则利用引理 5.7.1 (iii) 并对 $p^{d}$ 应用定理 5.6.6 可得
\[
\lim_{m \to \infty} p^{md}(x, y) \text{ 存在}
\]
为了确定极限值,注意到 (5.6.1) 蕴含
\[
\frac{1}{n} \sum_{m=1}^{n} p^{m}(x, y) \to \pi(y)
\]
且引理 5.7.1 (ii) 蕴含 $p^{m}(x, y) = 0$ 除非 $d \mid m$,因此第一个式子的极限必须为 $\pi(y) d$.若 $y \in S_{r}$ 且 $1 \leq r < d$,则
\[
p^{md + r}(x, y) = \sum_{z \in S_{r}} p^{r}(x, z) p^{md}(z, y)
\]
由于 $y, z \in S_{r}$,根据证明中的第一种情况可知 $p^{md}(z, y) \to \pi(y) d$ 当 $m \to \infty$.又有 $p^{md}(z, y) \leq 1$,且 $\sum_{z} p^{r}(x, z) = 1$,所以由控测收敛定理可得结论.□

    \end{prf}
    
    
    
    \begin{dfn}
        [Definition-of-Tail-Sigma-Field]
        {尾σ-域的定义}
        [Definition of Tail Sigma-Field]
        [gpt-4.1]
        设 $\mathcal{F}_{n}^{\prime} = \sigma(X_{n+1}, X_{n+2}, \ldots)$,$\mathcal{T} = \cap_{n} \mathcal{F}_{n}^{\prime}$ 被称为尾 $\sigma$-域.

    \end{dfn}
    
    
    
    \begin{dfn}
        [Definition-of-Offspring-Distribution-Generating-Function]
        {子代分布生成函数的定义}
        [Definition of Offspring Distribution Generating Function]
        [gpt-4.1]
        
对于 $s \in [0, 1]$,定义
\[
\varphi ( s ) = \sum_{k \geq 0} p_k s^k
\]
其中 $p_k = P(\xi_i^m = k)$.

    \end{dfn}
    
    
    
    \begin{thm}
        [Extinction-Probability-Theorem-for-Supercritical-Branching-Process]
        {超临界分支过程灭绝概率定理}
        [Extinction Probability Theorem for Supercritical Branching Process]
        [gpt-4.1]
        
设 $\mu > 1$.若 $Z_0 = 1$,则 $P(Z_n = 0 \text{ for some } n) = \rho$,其中 $\rho$ 是方程 $\varphi(\rho) = \rho$ 在区间 $[0, 1)$ 上唯一的解.

    \end{thm}
    
    
    
    \begin{prf}
        [Proof-of-Convergence-of-Normalized-Stochastic-Process-to-Brownian-Motion]
        {归一化随机过程收敛至布朗运动的证明}
        [Proof of Convergence of Normalized Stochastic Process to Brownian Motion]
        [gpt-4.1]
        如果令 $W_n(t) = B(n t) / \sqrt{n} =_d B_t$(由布朗运动的缩放性质),则

\[
S_n / \sqrt{n} \equiv B(T_n) / \sqrt{n} = W_n( T_n / n )
\]

弱大数定律表明 $T_n / n \to 1$(概率收敛).

    \end{prf}
    
    
    
    \begin{crl}
        [Conclusion-on-Convergence-of-Normalized-Sum-to-Brownian-Motion]
        {归一化和收敛于布朗运动的结论}
        [Conclusion on Convergence of Normalized Sum to Brownian Motion]
        [gpt-4.1]
        由上述内容可知 $S_n / \sqrt{n} \Rightarrow B_1$.

    \end{crl}
    
    
    
    \begin{prf}
        [Proof-Details-of-Pointwise-Convergence-Probability-Estimates-for-Normalized-Process]
        {归一化过程逐点收敛概率估计的证明细节}
        [Proof Details of Pointwise Convergence Probability Estimates for Normalized Process]
        [gpt-4.1]
        为补充细节,令 $\epsilon > 0$,选择 $\delta$ 使得

\[
P( | B_t - B_1 | > \epsilon~\mathrm{for~some~} t \in ( 1 - \delta , 1 + \delta ) ) < \epsilon / 2
\]

再取足够大的 $N$,使得对所有 $n \geq N$,有 $P( | T_n / n - 1 | > \delta ) < \epsilon / 2$.

    \end{prf}
    
    
    
    \begin{ppt}
        [Probability-Bound-for-Convergence-of-Normalized-Stochastic-Process-to-Brownian-Motion]
        {归一化随机过程收敛于布朗运动的概率界}
        [Probability Bound for Convergence of Normalized Stochastic Process to Brownian Motion]
        [gpt-4.1]
        以上两个估计说明对于 $n \geq N$,

\[
P( | W_n( T_n / n ) - W_n( 1 ) | > \epsilon ) < \epsilon
\]

由于 $\epsilon$ 任意,因此 $W_n( T_n / n ) - W_n( 1 ) \to 0$(概率收敛).

    \end{ppt}
    
    
    
    \begin{lma}
        [Application-of-the-Converging-Together-Lemma]
        {同时收敛引理的应用}
        [Application of the Converging Together Lemma]
        [gpt-4.1]
        应用同时收敛引理(Lemma 3.2.13),令 $X_n = W_n( 1 )$,$Z_n = W_n( T_n / n )$,即可得到所需结论.

    \end{lma}
    
    
    
    \begin{dfn}
        [Definition-of-Filtration-and-Adapted-Sequence]
        {滤子与适应性序列的定义}
        [Definition of Filtration and Adapted Sequence]
        [gpt-4.1]
        设 ${\mathcal{F}}_n$ 是一个滤子,即一个递增的 $\sigma$-域序列.若序列 $X_n$ 满足 $X_n \in {\mathcal{F}}_n$ 对所有 $n$ 都成立,则称 $X_n$ 适应于滤子 ${\mathcal{F}}_n$.
    \end{dfn}
    
    
    
    \begin{dfn}
        [Definition-of-Martingales-Supermartingales-and-Submartingales]
        {鞅、超鞅和次鞅的定义}
        [Definition of Martingales, Supermartingales, and Submartingales]
        [gpt-4.1]
        设 $X_n$ 是一个满足以下条件的序列:

(i) $E|X_n|<\infty$;
(ii) $X_n$ 适应于滤子 ${\mathcal{F}}_n$;
(iii) $E(X_{n+1}|{\mathcal{F}}_n)=X_n$ 对所有 $n$ 都成立.

则称 $X$ 是(关于滤子 ${\mathcal{F}}_n$ 的)鞅.

若条件 (iii) 中的等号 $=$ 替换为 $\leq$,则称 $X$ 为超鞅;若替换为 $\geq$,则称 $X$ 为次鞅.
    \end{dfn}
    
    
    
    \begin{dfn}
        [Definition-of-Random-Variable-and-Shift-Transformation]
        {随机变量及平移变换的定义}
        [Definition of Random Variable and Shift Transformation]
        [gpt-4.1]
        设 $X_{n}$, $n \in \mathbf{Z}$ 定义在序列空间 $( \mathbf{R}^{\mathbf{Z}}, \mathcal{R}^{\mathbf{Z}}, P )$ 上, 其中 $X_{n} ( \omega ) = \omega_{n}$.设 $( \theta^{n} \omega )(m) = \omega ( m + n )$,如果 $Y$ 是一个随机变量,则 $( \theta^{n} Y ) ( \omega ) = Y ( \theta^{n} \omega )$.
    \end{dfn}
    
    
    
    \begin{dfn}
        [Definition-of-Subspace-$H-n$-and-Orthogonal-Complement-$K-n$]
        {子空间 $H\_n$ 和正交补 $K\_n$ 的定义}
        [Definition of Subspace $H_n$ and Orthogonal Complement $K_n$]
        [gpt-4.1]
        设
\[
\begin{array}{rl}
& H_{n} = \{ Y \in \mathcal{F}_{n} \text{ 且 } E Y^{2} < \infty \} \\
& K_{n} = \{ Y \in H_{n} \text{ 且 } E ( YZ ) = 0 \text{ 对所有 } Z \in H_{n-1} \}
\end{array}
\]
几何上,$H_{0} \supset H_{-1} \supset H_{-2} \ldots$ 是 $L^{2}$ 中的一系列子空间,$K_{n}$ 是 $H_{n-1}$ 在 $H_{n}$ 中的正交补.
    \end{dfn}
    
    
    
    \begin{ppt}
        [Action-of-Shift-Transformation-on-Subspaces]
        {平移变换对子空间的作用}
        [Action of Shift Transformation on Subspaces]
        [gpt-4.1]
        由例子 $Y = f ( X_{-j}, \ldots, X_{k} )$ 推广,具有 $\theta^{n} Y = f ( X_{n-j}, \ldots, X_{n+k} )$,容易看出,如果 $Y \in H_{k}$,则 $\theta^{n} Y \in H_{k+n}$,因此如果 $Y \in K_{j}$,则 $\theta^{n} Y \in K_{n+j}$.
    \end{ppt}
    
    
    
    \begin{prf}
        [Inductive-Proof-of-Conditional-Probability-Inequality]
        {关于条件概率不等式的归纳证明}
        [Inductive Proof of Conditional Probability Inequality]
        [gpt-4.1]
        我们采用归纳法证明如下条件概率不等式.对于 $n=1$ 时,有
\[
P ( A_1 | \mathcal{G}_0 ) \leq \delta + P ( P ( A_1 | \mathcal{G}_0 ) > \delta | \mathcal{G}_0 )
\]
在 $\Omega_{-} \equiv \{ P ( A_1 | \mathcal{G}_0 ) \leq \delta \}$ 上显然成立,在 $\Omega_{+} \equiv \{ P ( A_1 | \mathcal{G}_0 ) > \delta \} \in \mathcal{G}_0$ 上也成立,因为在 $\Omega_{+}$ 上有
\[
P ( P ( A_1 | \mathcal{G}_0 ) > \delta | \mathcal{G}_0 ) = 1 \geq P ( A_1 | \mathcal{G}_0 )
\]
对于 $n$ 个集合的情况,类似地不等式在 $\Omega_{+}$ 上也是显然成立.

令 $B_m = A_m \cap \Omega_{-}$.由于 $\Omega_{-} \in \mathcal{G}_0 \subset \mathcal{G}_{m-1}$,有 $P ( B_m | \mathcal{G}_{m-1} ) = P ( A_m | \mathcal{G}_{m-1} )$ 在 $\Omega_{-}$ 上成立.

对 $n-1$ 个集合应用归纳假设,令 $\gamma = \delta - P ( B_1 | \mathcal{G}_0 ) \geq 0$,则
\[
P \left( \bigcup_{m=2}^n B_m \bigg| \mathcal{G}_1 \right) \leq \gamma + P \left( \sum_{m=2}^n P ( B_m | \mathcal{G}_{m-1} ) > \gamma \bigg| \mathcal{G}_1 \right)
\]
对 $\mathcal{G}_0$ 取条件期望,并注意到 $\gamma \in \mathcal{G}_0$,得到
\[
P \left( \bigcup_{m=2}^n B_m \bigg| \mathcal{G}_0 \right) \leq \gamma + P \left( \sum_{m=1}^n P ( B_m | \mathcal{G}_{m-1} ) > \delta \bigg| \mathcal{G}_0 \right)
\]
又因 $\bigcup_{2 \leq m \leq n} B_m = \left( \bigcup_{2 \leq m \leq n} A_m \right) \cap \Omega_{-}$,并且
\[
\sum_{1 \leq m \leq n} P ( B_m | \mathcal{G}_{m-1} ) = \sum_{1 \leq m \leq n} P ( A_m | \mathcal{G}_{m-1} )
\]
在 $\Omega_{-}$ 上成立.

所以,在 $\Omega_{-}$ 上,
\[
P \left( \bigcup_{m=2}^n A_m \bigg| \mathcal{G}_0 \right) \leq \delta - P ( A_1 | \mathcal{G}_0 ) + P \left( \sum_{m=1}^n P ( A_m | \mathcal{G}_{m-1} ) > \delta \bigg| \mathcal{G}_0 \right)
\]
最终结果由下式推出
\[
P \left( \bigcup_{m=1}^n A_m \bigg| \mathcal{G}_0 \right) \leq P ( A_1 | \mathcal{G}_0 ) + P \left( \bigcup_{m=2}^n A_m \bigg| \mathcal{G}_0 \right)
\]
为证明此不等式,令 $C = \bigcup_{2 \leq m \leq n} A_m$,注意到 $1_{A_1 \cup C} \leq 1_{A_1} + 1_C$,并利用条件期望的单调性即可.

    \end{prf}
    
    
    
    \begin{thm}
        [Non-ergodicity-of-Rational-Rotations-and-Ergodicity-of-Irrational-Rotations]
        {有理数旋转的非遍历性和无理数旋转的遍历性}
        [Non-ergodicity of Rational Rotations and Ergodicity of Irrational Rotations]
        [gpt-4.1]
        
若 $\theta = m / n$,其中 $m < n$ 为正整数,则圆周上的旋转不是遍历的.若 $B$ 是 $[0, 1 / n)$ 的一个 Borel 子集,令
\[
A = \bigcup_{k=0}^{n-1} (B + k / n)
\]
则 $A$ 关于旋转变换是不变的.反之,若 $\theta$ 是无理数,则 $\varphi$ 是遍历的.

    \end{thm}
    
    
    
    \begin{thm}
        [Uniqueness-of-Fourier-Series-Expansion-for-Integrable-Functions]
        {可积函数的傅里叶级数展开唯一性}
        [Uniqueness of Fourier Series Expansion for Integrable Functions]
        [gpt-4.1]
        
若 $f$ 是 $[0, 1)$ 上的可测函数且 $\int f^{2}(x) dx < \infty$,则 $f$ 可以表示为
\[
f(x) = \sum_{k} c_{k} e^{2\pi i k x}
\]
其中级数在 $L^{2}[0, 1)$ 意义下有
\[
\sum_{k=-K}^{K} c_{k} e^{2\pi i k x} \to f(x) \quad \text{当} \ K \to \infty
\]
且系数 $c_{k}$ 唯一,由下式给出
\[
c_{k} = \int f(x) e^{-2\pi i k x} dx.
\]

    \end{thm}
    
    
    
    \begin{thm}
        [Constancy-of-Invariant-Functions-under-Irrational-Rotation]
        {旋转变换不变函数的常值性}
        [Constancy of Invariant Functions under Irrational Rotation]
        [gpt-4.1]
        
若 $\theta$ 为无理数,若 $f(\varphi(x)) = f(x)$,则 $c_{k}(e^{2\pi i k \theta} - 1) = 0$.这意味着 $c_{k} = 0$ 对于 $k 
eq 0$,所以 $f$ 是常值函数.

    \end{thm}
    
    
    
    \begin{crl}
        [Corollary-Ergodicity-Applied-to-Indicator-Functions]
        {遍历性应用到特征函数的推论}
        [Corollary: Ergodicity Applied to Indicator Functions]
        [gpt-4.1]
        
将上一个结果应用于 $f = 1_{A}$,其中 $A \in \mathcal{T}$,可得 $A = \emptyset$ 或 $A = [0, 1)$ 几乎处处成立.

    \end{crl}
    
    
    
    \begin{thm}
        [Theorem-on-Existence-of-Limit-under-Quadratic-Form-Condition]
        {二次型条件下极限存在性定理}
        [Theorem on Existence of Limit under Quadratic Form Condition]
        [gpt-4.1]
        如果 $\sum_{m=1}^{\infty} E( ( S_{m} - S_{m-1} )^{2} \mid \mathcal{F}_{m-1} ) < \infty$,则 $\lim_{n \to \infty} S_{n}$ 存在且有限.
    \end{thm}
    
    
    
    \begin{prf}
        [Proof-of-Theorem-on-Existence-of-Limit-under-Quadratic-Form-Condition]
        {二次型条件下极限存在性定理的证明}
        [Proof of Theorem on Existence of Limit under Quadratic Form Condition]
        [gpt-4.1]
        设 $B_{t}$ 为由布朗运动生成的滤波族,$t_{m} = T_{m} - T_{m-1}$.通过构造有
\[
E( ( S_{m} - S_{m-1} )^{2} \mid \mathcal{F}_{m-1} ) = E( t_{m} \mid \mathcal{B}( T_{m-1} ) )
\]
令 $M = \inf \{ n : \sum_{m=1}^{n+1} E( t_{m} \mid \mathcal{B}( T_{m-1} ) ) > A \}$.$M$ 是停时,并且根据定义,有 $\sum_{m=1}^{M} E( t_{m} \mid \mathcal{B}( T_{m-1} ) ) \leq A$,且 $\{ M \geq m \} \in \mathcal{F}_{m-1}$.
\[
E \sum_{m=1}^{M} t_{m} = \sum_{m=1}^{\infty} E( t_{m}; M \geq m ) = \sum_{m=1}^{\infty} E( E( t_{m} \mid \mathcal{B}( T_{m-1} ); M \geq m ) ) = E \sum_{m=1}^{M} E( t_{m} \mid \mathcal{B}( T_{m-1} ) ) \leq A
\]
因此 $\sum_{m=1}^{M} t_{m} < \infty$.当 $A \to \infty$ 时,$P( M < \infty ) \downarrow 0$,故 $T_{\infty} = \sum_{m=1}^{\infty} t_{m} < \infty$,且 $B( T_{n} ) \to B( T_{\infty} )$ 当 $n \to \infty$.
    \end{prf}
    
    
    
    \begin{dfn}
        [Definition-of-Stopping-Time]
        {停时的定义}
        [Definition of Stopping Time]
        [gpt-4.1]
        设 $M = \inf \{ n : \sum_{m=1}^{n+1} E( t_{m} \mid \mathcal{B}( T_{m-1} ) ) > A \}$,则 $M$ 是一个停时,满足 $\{ M \geq m \} \in \mathcal{F}_{m-1}$.
    \end{dfn}
    
    
    
    \begin{prf}
        [Proof-that-the-Heat-Kernel-Satisfies-the-Heat-Equation]
        {关于热核满足热方程的证明}
        [Proof that the Heat Kernel Satisfies the Heat Equation]
        [gpt-4.1]
        证明如下:令 $p_{t}(x, y) = (2 \pi)^{-1/2} t^{-1/2} \exp(-(y - x)^{2} / 2t)$.
首先检验 $p_{t}$ 满足热方程:$\partial p_{t} / \partial t = (1/2) \partial^{2} p_{t} / \partial y^{2}$.

\[
\begin{array}{l}
\displaystyle \frac{\partial p}{\partial t} = -\frac{1}{2} t^{-1} p_{t} + \frac{(y - x)^{2}}{2 t^{2}} p_{t} \\
\displaystyle \frac{\partial p}{\partial y} = -\frac{y - x}{t} p_{t} \\
\displaystyle \frac{\partial^{2} p}{\partial y^{2}} = -\frac{1}{t} p_{t} + \frac{(y - x)^{2}}{t^{2}} p_{t}
\end{array}
\]

交换 $\partial / \partial t$ 与积分符号,并利用热方程,

\[
\begin{array}{l}
\displaystyle \frac{\partial}{\partial t} E_{x} u(t, B_{t}) = \int \frac{\partial}{\partial t} (p_{t}(x, y) u(t, y)) dy \\
\displaystyle = \int \frac{1}{2} \frac{\partial^{2}}{\partial y^{2}} p_{t}(x, y) u(t, y) + p_{t}(x, y) \frac{\partial}{\partial t} u(t, y) dy
\end{array}
\]

对前式两次分部积分,

\[
= \int p_{t}(x, y) \left( \frac{\partial}{\partial t} + \frac{1}{2} \frac{\partial^{2}}{\partial y^{2}} \right) u(t, y) dy = 0
\]

由于 $u(t, y)$ 是多项式,积分收敛无疑,并且分部积分时无边界项贡献.

因此,$t \to E_{x} u(t, B_{t})$ 是常数.

    \end{prf}
    
    
    
    \begin{thm}
        [Martingale-Property-of-Polynomial-Solutions]
        {广义多项式解的鞅性质}
        [Martingale Property of Polynomial Solutions]
        [gpt-4.1]
        若 $u(t, y)$ 是关于 $y$ 的多项式且满足 $\partial u / \partial t = (1/2) \partial^{2} u / \partial y^{2}$,则对于布朗运动 $B_t$,有
\[
E(u(t, B_{t}) | \mathcal{F}_{s}) = u(s, B_{s})
\]
即 $u(t, B_t)$ 是 $\mathcal{F}_t$ 下的鞅.

    \end{thm}
    
    
    
    \begin{prf}
        [Proof-of-the-Martingale-Property-for-Polynomial-Solutions]
        {广义多项式鞅性质的证明}
        [Proof of the Martingale Property for Polynomial Solutions]
        [gpt-4.1]
        证明如下:

设 $
u(r, x) = u(s + r, x)$,则 $\partial 
u / \partial r = (1/2) \partial^{2} 
u / \partial x^{2}$,由马尔可夫性,

\[
\begin{array}{rl}
& E(u(t, B_{t}) | \mathcal{F}_{s}) = E(
u(t - s, B_{t - s}) \circ \theta_{s} | \mathcal{F}_{s}) \\
& \qquad = E_{B(s)} 
u(t - s, B_{t - s}) = 
u(0, B_{s}) = u(s, B_{s})
\end{array}
\]

其中倒数第二步利用了先前证明的期望值的常数性.

    \end{prf}
    
    
    
    \begin{dfn}
        [Definition-of-Exponential-Type-Characteristic-Function]
        {指数型特征函数的定义}
        [Definition of Exponential Type Characteristic Function]
        [gpt-4.1]
        令 $\alpha < 2$, 定义 $\varphi(t) = \exp( - | t |^{\alpha} )$, 其中 $| t | = ( t_1^2 + t_2^2 )^{1/2}$.
    \end{dfn}
    
    
    
    \begin{ppt}
        [Properties-of-2D-Random-Vector-with-Exponential-Type-Characteristic-Function]
        {具有指数型特征函数的二维随机向量的性质}
        [Properties of 2D Random Vector with Exponential Type Characteristic Function]
        [gpt-4.1]
        设 $\varphi$ 是随机向量 $( X_1, X_2 )$ 的特征函数,且 $\varphi(t) = \exp( - | t |^{\alpha} )$,则该随机向量具有以下两个性质:

(i) $( X_1, X_2 )$ 的分布关于旋转不变;
(ii) $X_1$ 和 $X_2$ 服从指数 $\alpha$ 的对称稳定分布.
    \end{ppt}
    
    
    
    \begin{thm}
        [Transience-Theorem-for-Random-Walks-with-Stable-Distribution-Characteristic-Function]
        {稳定分布特征函数的随机游走遍历性定理}
        [Transience Theorem for Random Walks with Stable Distribution Characteristic Function]
        [gpt-4.1]
        设随机游走的特征函数为 $\exp( - | t |^{\alpha} )$,其中 $\alpha < 2$,则该随机游走是瞬态的.
    \end{thm}
    
    
    
    \begin{ppt}
        [Moment-Finiteness-Property-for-Stable-Distributions]
        {稳定分布的矩期有限性性质}
        [Moment Finiteness Property for Stable Distributions]
        [gpt-4.1]
        当 $p < \alpha$ 时,有 $E | X_1 |^p < \infty$.
    \end{ppt}
    
    
    
    \begin{prf}
        [Proof-that-i-implies-$V-{n-n}-	o-1$-in-Probability]
        {(i)蕴含 $V\_{n, n} 	o 1$ 概率收敛的证明}
        [Proof that (i) implies $V_{n, n} 	o 1$ in Probability]
        [gpt-4.1]
        (i) implies $V_{n, n} \to 1$ in probability.

By stopping each sequence at the first time $V_{n, k} > 2$ and setting the later $X_{n, m} = 0$, we can suppose without loss of generality that $V_{n, n} \leq 2 + \epsilon_n^2$ for all $n$.

By Theorem 8.2.1, we can find stopping times $T_{n, 1}, \ldots, T_{n, n}$ so that $(S_{n, 1}, \ldots, S_{n, n}) =_d (B(T_{n, 1}), \ldots, B(T_{n, n}))$.

By Lemma 8.1.9, it suffices to show that $T_{n, [nt]} \to t$ in probability for each $t \in [0, 1]$.

To do this we let $t_{n, m} = T_{n, m} - T_{n, m-1}$ (with $T_{n, 0} = 0$) and observe that by the remark after the proof of Theorem 8.2.1, $E(t_{n, m} | \mathcal{F}_{n, m-1}) = E(X_{n, m}^2 | \mathcal{F}_{n, m-1})$.

The last observation and hypothesis (i) imply
\[
\sum_{m=1}^{[nt]} E(t_{n, m} | \mathcal{F}_{n, m-1}) \to t \quad \mathrm{~in~probability~}
\]

To get from this to $T_{n, [nt]} \to t$ in probability, we observe
\[
E\left( \sum_{m=1}^{[nt]} (t_{n, m} - E(t_{n, m} | \mathcal{F}_{n, m-1})) \right)^2 = E \sum_{m=1}^{[nt]} \{ t_{n, m} - E(t_{n, m} | \mathcal{F}_{n, m-1}) \}^2
\]
by the orthogonality of martingale increments, Theorem 4.4.7.

Now
\[
\begin{array}{rl}
& E\left( \{ t_{n, m} - E(t_{n, m} | \mathcal{F}_{n, m-1}) \}^2 \mid \mathcal{F}_{n, m-1} \right) \leq E(t_{n, m}^2 | \mathcal{F}_{n, m-1}) \\
& \qquad \leq C E(X_{n, m}^4 | \mathcal{F}_{n, m-1}) \leq C \epsilon_n^2 E(X_{n, m}^2 | \mathcal{F}_{n, m-1})
\end{array}
\]
by Exercise 7.5.4 and assumption (ii).

Summing over $m$, taking expected values, and recalling we have assumed $V_{n, n} \leq 2 + \epsilon_n^2$, it follows that
\[
E\left( \sum_{m=1}^{[nt]} (t_{n, m} - E(t_{n, m} | \mathcal{F}_{n, m-1})) \right)^2 \leq C \epsilon_n^2 E V_{n, n} \to 0
\]

Unscrambling the definitions, we have shown $E( T_{n, [nt]} - V_{n, [nt]} )^2 \to 0$, so Chebyshev's inequality implies $P( | T_{n, [nt]} - V_{n, [nt]} | > \epsilon ) \to 0$, and using (i) now completes the proof.

    \end{prf}
    
    
    
    \begin{thm}
        [Equivalent-Expression-of-$\alpha-{+}E$]
        {关于 $\alpha\_{+}(E)$ 的等价表达式}
        [Equivalent Expression of $\alpha_{+}(E)$]
        [gpt-4.1]
        证明 $\alpha_{+}(E) = \operatorname*{sup}\{\alpha(F) : F \subset E\}$.
    \end{thm}
    
    
    
    \begin{thm}
        [Twice_Differentiability_of_$
    u$_under_Hölder_Continuity_of_$c$]
        {关于 Hölder 连续性条件下 $
    u$ 的二阶可微性}
        [Twice Differentiability of $
    u$ under Hölder Continuity of $c$]
        [gpt-4.1]
        如果除了 (A1)–(A3) 之外,$c$ 是 Hölder 连续的,则 $
u \in C^{2}$,因此满足 (a).
    \end{thm}
    
    
    
    \begin{thm}
        [Birkhoffs-Ergodic-Theorem]
        {Birkhoff遍历定理}
        [Birkhoff's Ergodic Theorem]
        [gpt-4.1]
        Birkhoff遍历定理:设$\varphi$是$(\Omega, \mathcal{F}, P)$上的测度保持变换.
    \end{thm}
    
    
    
    \begin{thm}
        [Theorem-on-Continuity-of-Expectation]
        {关于期望连续性的定理}
        [Theorem on Continuity of Expectation]
        [gpt-4.1]
        
定理 9.2.4 蕴含 $(t, x) \mapsto E_{x} f(B_{t})$ 在 $\{ 0 \} \times \mathbf{R}^{d}$ 的每个点处是连续的,因此得到所需结果.

    \end{thm}
    
    
    
    \begin{lma}
        [Lemma-on-Existence-of-Negative-Subset-in-Measurable-Set]
        {关于可测集的负子集存在性引理}
        [Lemma on Existence of Negative Subset in Measurable Set]
        [gpt-4.1]
        
设 $E$ 是可测集且 $\alpha(E) < 0$,则存在负集 $F \subset E$,使得 $\alpha(F) < 0$.

    \end{lma}
    
    
    
    \begin{dfn}
        [Definition-of-Brownian-Bridge]
        {布朗桥的定义}
        [Definition of Brownian Bridge]
        [gpt-4.1]
        
将过程 $\{ B_{t} - t B_{1}, 0 \le t \le 1 \}$ 记作 $B_{t}^{0}$,称为布朗桥(Brownian bridge).等同于 $\{ B_{t}, 0 \le t \le 1 \mid B_{1} = 0 \}$.

    \end{dfn}
    
    
    
    \begin{dfn}
        [Definition-of-Extended-Measure-on-Measurable-Sets]
        {可测集上的外延测度的定义}
        [Definition of Extended Measure on Measurable Sets]
        [gpt-4.1]
        
我们以显然的方式定义 $\bar{\mu}$:若 $E = A \cup B$,其中 $A \in {\mathcal{F}}$ 且 $B \subset N$,其中 $\mu(N) = 0$,则令 ${\bar{\mu}}(E) = \mu(A)$.

    \end{dfn}
    
    
    
    \begin{thm}
        [Representation-Formula-for-the-Expectation-of-a-Bounded-Continuous-Function]
        {关于连续有界函数的期望的表示公式}
        [Representation Formula for the Expectation of a Bounded Continuous Function]
        [gpt-4.1]
        若 $f$ 是有界且连续的函数,则有

\[
E_x f(B_\tau) = u(x) = \int_{\partial D} k_y(x) f(y) \pi(dy)
\]

其中 $\pi$ 是在 $\partial D$ 上归一化为概率测度的表面积测度.
    \end{thm}
    
    
    
    \begin{thm}
        [Criterion-for-Recurrence-of-Markov-Chain]
        {马尔可夫链常返性的判别准则}
        [Criterion for Recurrence of Markov Chain]
        [gpt-4.1]
        
马尔可夫链是常返的当且仅当

\[
\sum _ { m = 0 } ^ { \infty } \prod _ { j = 1 } ^ { m } { \frac { q _ { j } } { p _ { j } } } = \infty
\]

    \end{thm}
    
    
    
    \begin{thm}
        [On-Cyclic-Decomposition-and-the-Generated-Sigma-Algebra]
        {关于循环分解与生成的σ代数}
        [On Cyclic Decomposition and the Generated Sigma-Algebra]
        [gpt-4.1]
        如果 $S_{0}, \ldots, S_{d-1}$ 是 $S$ 的循环分解,则 $\mathcal{T} = \sigma(\{X_{0} \in S_{r}\} : 0 \leq r < d)$.
    \end{thm}
    
    
    
    \begin{thm}
        [Recurrence-Criterion-for-the-M/G/1-Queue-Chain]
        {M/G/1排队论的链的常返性判据}
        [Recurrence Criterion for the M/G/1 Queue Chain]
        [gpt-4.1]
        
设 $\mu = \sum k a_{k}$ 为一个服务时间内到达的顾客的平均数.在例子5.3.7中,我们证明了该链常返当且仅当 $\mu \leq 1$.

    \end{thm}
    
    
    
    \begin{dfn}
        [Definition-of-Absolute-Continuity-of-Measures]
        {测度的绝对连续性定义}
        [Definition of Absolute Continuity of Measures]
        [gpt-4.1]
        我们称测度 $
u$ 关于测度 $\mu$ 是绝对连续的(记作 $
u \ll \mu$),如果对于任意集合 $A$,当 $\mu(A) = 0$ 时有 $
u(A) = 0$.
    \end{dfn}
    
    
    
    \begin{lma}
        [Lemma-on-the-Expectation-of-Stopping-Time-of-Brownian-Motion]
        {关于布朗运动停时的期望的引理}
        [Lemma on the Expectation of Stopping Time of Brownian Motion]
        [gpt-4.1]
        
若 $c^* = \sup_{x} |c(x)|$,则根据引理 9.8.3,可以取 $r_0$ 充分小,使得对 $T_r = \inf\{ t : |B_t - B_0| > r \}$,当 $r \leq r_0$ 时,均有 $E_x \exp( c^* T_r ) \leq 2$ 对所有 $x$ 成立.

    \end{lma}
    
    
    
    \begin{thm}
        [Construction-and-Properties-of-Nonmeasurable-Sets]
        {关于不可测集的构造与性质}
        [Construction and Properties of Nonmeasurable Sets]
        [gpt-4.1]
        
(1) 设 $B$ 是由定理 A.2.4 构造的不可测集.令 $B_q = q + B$,证明:如果 $D_q \subset B_q$ 是可测集,则 $\lambda(D_q) = 0$.
(2) 利用 (1),得出结论:如果 $A \subset \mathbf{R}$ 且 $\lambda(A) > 0$,则存在不可测集 $S \subset A$.

    \end{thm}
    
    
    
    \begin{lma}
        [Lemma-A]
        {引理A}
        [Lemma A]
        [gpt-4.1]
        1 we proved:

Lemma A.
    \end{lma}
    
    
    
    \begin{xmp}
        [Example-of-Markov-Chain-with-Stationary-Distribution]
        {关于马尔可夫链平稳分布的例子}
        [Example of Markov Chain with Stationary Distribution]
        [gpt-4.1]
        
设 $X_{n}$ 是具有转移概率 $p(x, A)$ 和平稳分布 $\pi$ 的马尔可夫链,即 $\pi(A) = \int \pi(dx) p(x, A)$.如果 $X_{0}$ 的分布为 $\pi$,则 $X_{0}, X_{1}, \ldots$ 构成一个平稳序列.

    \end{xmp}
    
    
    
    \begin{dfn}
        [Definition-of-the-Dirichlet-Problem-for-the-Schrodinger-Equation]
        {Schrodinger方程的Dirichlet问题定义}
        [Definition of the Dirichlet Problem for the Schrodinger Equation]
        [gpt-4.1]
        设$c$为有界连续函数,$f$为定义在$\partial G$上的有界连续函数,Schrodinger方程的Dirichlet问题如下:

(a) $u \in C^{2}$ 且 $\frac{1}{2} \Delta u + c u = 0$ 在$G$内成立;

(b) 对于$\partial G$上的每一个点,$u$是连续的且$u = f$.
    \end{dfn}
    
    
    
    \begin{dfn}
        [Definition-of-Cyclic-Decomposition-of-State-Space]
        {状态空间的循环分解定义}
        [Definition of Cyclic Decomposition of State Space]
        [gpt-4.1]
        状态空间 $S_{0}, S_{1}, \ldots, S_{d-1}$ 的一个分划,满足引理 5.(ii),称为状态空间的循环分解.
    \end{dfn}
    
    
    
    \begin{dfn}
        [Definition-of-Modified-Greens-Function]
        {修正的格林函数的定义}
        [Definition of Modified Green's Function]
        [gpt-4.1]
        在 $d \leq 2$ 的情况下,定义修正的格林函数如下:

\[
G(x, y) = \int_{0}^{\infty} (p_{t}(x, y) - a_{t}) dt
\]

其中 $a_{t}$ 是我们为使积分收敛(至少当 $x 
eq y$ 时)而选择的常数.
    \end{dfn}
    
    
    
    \begin{thm}
        [Transience-of-Brownian-Motion-in-Dimension-Three-and-Higher]
        {布朗运动在三维及以上空间的遍历性}
        [Transience of Brownian Motion in Dimension Three and Higher]
        [gpt-4.1]
        
若 $d \ge 3$, 对于布朗运动,有
\[
P_{x}(S_{r} < \infty) = (r / |x|)^{d-2} < 1 \qquad \text{当 } |x| > r
\]
因此,布朗运动是遍历的,即它不会无限次地回到任何有界集合.

    \end{thm}
    
    
    
    \begin{thm}
        [Validity-Conditions-of-Theorem-9.7.5-in-Different-Dimensions]
        {Theorem 9.7.5 在不同维数下的成立条件}
        [Validity Conditions of Theorem 9.7.5 in Different Dimensions]
        [gpt-4.1]
        定理 9.7.5 在 $d = 2$ 时成立;在 $d = 1$ 时,只需假设 $g$ 连续即可成立.
    \end{thm}
    
    
    
    \begin{thm}
        [Expression-for-Solution-to-the-Dirichlet-Problem]
        {满足Dirichlet问题的解的表达式}
        [Expression for Solution to the Dirichlet Problem]
        [gpt-4.1]
        
设 $f$ 是有界且连续的函数,定义
\[
u(x, y) = \int d\theta\, h_\theta(x, y) f(\theta, 0)
\]
则 $u$ 在 $H$ 中满足Dirichlet问题,并且其边界值为 $f$.

    \end{thm}
    
    
    
    \begin{thm}
        [Donskers-Theorem]
        {Donsker定理}
        [Donsker's Theorem]
        [gpt-4.1]
        Donsker(1951)定理将在本节中被证明.
    \end{thm}
    
    
    
    \begin{lma}
        [The-class-of-measurable-sets-is-a-sigma-field-and-the-restriction-of-outer-measure-is-a-measure]
        {可测集的集合是σ-域且外测度的限制是测度}
        [The class of measurable sets is a sigma-field and the restriction of outer measure is a measure]
        [gpt-4.1]
        
引理 A.1: 集合 $\mathcal{A}^*$(可测集的类)是一个 $\sigma$-域,并且 $\mu^*$ 在 $\mathcal{A}^*$ 上的限制是一个测度.

    \end{lma}
    
    
    
    \begin{lma}
        [Lemma-on-Limiting-Probability-at-Regular-Points]
        {正则点极限概率引理}
        [Lemma on Limiting Probability at Regular Points]
        [gpt-4.1]
        如果 $y$ 是 $G$ 的正则点且 $x_{n} \to y$,则对所有 $\delta > 0$,有
\[
\liminf_{n \to \infty} P_{x_{n}}(\tau < \infty, B_{\tau} \in D(y, \delta)) = 1
\]
其中 $D(y, \delta) = \{ x : |x - y| < \delta \}$.
    \end{lma}
    
    
    
    \begin{xmp}
        [An-Example-Showing-Failure-When-Measure-is-Not-σ-Finite]
        {关于测度非σ-有限情况下定理失效的例子}
        [An Example Showing Failure When Measure is Not σ-Finite]
        [gpt-4.1]
        
例子 A.4.9:定理 A.4.8 可能在测度 $\mu$ 不是 $\sigma$-有限时失效.

    \end{xmp}
    
    
    
    \begin{thm}
        [Statement-on-Failure-of-Theorem-Under-Non-σ-Finite-Measure]
        {关于定理在非σ-有限测度下可能失效的说明}
        [Statement on Failure of Theorem Under Non-σ-Finite Measure]
        [gpt-4.1]
        
定理 A.4.8 可能在测度 $\mu$ 不是 $\sigma$-有限时失效.

    \end{thm}
    
    
    
    \begin{xmp}
        [An-Example-of-Nonnegative-Solution-to-Second-Order-Differential-Equation]
        {关于二阶常微分方程的非负解的例子}
        [An Example of Nonnegative Solution to Second Order Differential Equation]
        [gpt-4.1]
        
$u(x) = \frac{\cos(x\sqrt{2\gamma})}{\cos(a\sqrt{2\gamma})}$ 是方程
\[
\frac{1}{2} u'' + \gamma u = 0,\qquad u(-a) = u(a) = 1
\]
的一个非负解.

    \end{xmp}
    
    
    
    \begin{thm}
        [Theorem-on-Local-Martingale]
        {关于局部鞅的定理}
        [Theorem on Local Martingale]
        [gpt-4.1]
        
定理 9.8.1 表明 $M_t = u(B_t) e^{\gamma t}$ 是区间 $[0, \tau)$ 上的局部鞅.

    \end{thm}
    
    
    
    \begin{thm}
        [Skorokhods-Representation-Theorem]
        {Skorokhod 表示定理}
        [Skorokhod's Representation Theorem]
        [gpt-4.1]
        若随机变量 $X$ 满足 $E X = 0$ 且 $E X^2 < \infty$,则存在关于布朗运动的一个停时 $T$,使得 $B_T \overset{d}{=} X$ 且 $E T = E X^2$.
    \end{thm}
    
    
    
    \begin{lma}
        [Lemma-on-$M-t-=-uB\gammat$-Being-a-Martingale]
        {关于 $M\_t = u(B(\gamma(t)))$ 是鞅的引理}
        [Lemma on $M_t = u(B(\gamma(t)))$ Being a Martingale]
        [gpt-4.1]
        $M_{t} = u(B(\gamma(t)))$ 是一个鞅.
    \end{lma}
    
    
    
    \begin{dfn}
        [Definition-of-Random-Variables-$\psi-{n}A$-and-$\psiA$]
        {随机变量 $\psi\_{n}(A)$ 和 $\psi(A)$ 的定义}
        [Definition of Random Variables $\psi_{n}(A)$ and $\psi(A)$]
        [gpt-4.1]
        
令 $\psi_{n}(A)$ 为下列两个概率表达式的共同值:
\[
P\left(\max_{0 \leq k < n} \bar{S}_{k} < \bar{S}_{n} \in A\right) = P\left(\min_{1 \leq k \leq n} \bar{S}_{k} > 0, \bar{S}_{n} \in A\right)
\]
并定义 $\psi(A) = \sum_{n \geq 0} \psi_{n}(A)$.

    \end{dfn}
    
    
    
    \begin{thm}
        [Carathéodorys-Extension-Theorem]
        {Carathéodory扩展定理}
        [Carathéodory's Extension Theorem]
        [gpt-4.1]
        
设$\boldsymbol{\mathcal{S}}$是一个半代数,$\mu$是定义在$\boldsymbol{\mathcal{S}}$上的集合函数,满足$\mu(\varnothing) = 0$.

    \end{thm}
    
    
    
    \begin{dfn}
        [Definition-of-the-Discrete-Ornstein-Uhlenbeck-Process]
        {离散Ornstein-Uhlenbeck过程的定义}
        [Definition of the Discrete Ornstein-Uhlenbeck Process]
        [gpt-4.1]
        
设 $\xi_1, \xi_2, \ldots$ 是独立同分布的标准正态随机变量,定义序列 $\{V_n\}$ 满足递推关系
\[
V_n = \theta V_{n-1} + \xi_n,
\]
则称 $\{V_n\}$ 为离散Ornstein-Uhlenbeck过程.

    \end{dfn}
    
    
    
    \begin{thm}
        [Transformation-from-Brownian-Motion-to-Brownian-Bridge]
        {布朗运动变换为布朗桥}
        [Transformation from Brownian Motion to Brownian Bridge]
        [gpt-4.1]
        设 $B_t$ 是以 0 为起点的布朗运动.则过程 $X_t = (1-t) B\left(\frac{t}{1-t}\right)$ 是一个布朗桥.
    \end{thm}
    
    
    
    \begin{thm}
        [Conclusion-of-Theorem-4.4.1]
        {定理4.4.1的结论}
        [Conclusion of Theorem 4.4.1]
        [gpt-4.1]
        定理 4.4.1 蕴含 $1 = E \exp(\theta S_{T \wedge n})$.
    \end{thm}
    
    
    
    \begin{thm}
        [Uniqueness-of-Stationary-Measure-for-Irreducible-Markov-Chains-with-Stationary-Distribution]
        {不可约且有平稳分布的马尔可夫链的平稳测度唯一性}
        [Uniqueness of Stationary Measure for Irreducible Markov Chains with Stationary Distribution]
        [gpt-4.1]
        
若 $p$ 是不可约的马尔可夫链并且存在平稳分布 $\pi$,则任何其他平稳测度都是 $\pi$ 的倍数.

    \end{thm}
    
    
    
    \begin{thm}
        [Limit-Theorem-for-One-dimensional-Brownian-Motion]
        {一维布朗运动增长极限定理}
        [Limit Theorem for One-dimensional Brownian Motion]
        [gpt-4.1]
        
设 $B_{t}$ 是以 $\boldsymbol{\theta}$ 为起点的一维布朗运动,则以概率 1,
\[
\lim_{t \to \infty} \sup_{0 \leq t} \frac{B_{t}}{\sqrt{t}} = \infty \qquad \lim_{t \to \infty} \inf_{0 \leq t} \frac{B_{t}}{\sqrt{t}} = -\infty
\]

    \end{thm}
    
    
    
    \begin{ppt}
        [Mean-Value-Property-of-Harmonic-Functions]
        {调和函数的平均值性质}
        [Mean Value Property of Harmonic Functions]
        [gpt-4.1]
        如果 $
u$ 是调和函数,则有
\[
u(x) = E_{x} 
u(B(\tau_{B})) = \int_{\partial D(x, r)} \bar{
u}(y) \pi(dy)
\]
其中 $\pi$ 是在边界 $\partial D(x, r)$ 上归一化的概率测度.这表明调和函数在球面上的平均值等于球心处的函数值.
    \end{ppt}
    
    
    
    \begin{thm}
        [Expectation-Formula-for-First-Exit-Time-of-Brownian-Motion-from-Interval]
        {关于布朗运动首次退出区间的期望公式}
        [Expectation Formula for First Exit Time of Brownian Motion from Interval]
        [gpt-4.1]
        
设 $\tau = \operatorname*{inf} \{ t : B _ { t } 
otin ( - a , a ) \}$.若 $0 < \gamma < \pi ^ { 2 } / 8 a ^ { 2 }$,则
\[
E _ { x } e ^ { \gamma \tau } = \frac { \cos ( x \sqrt { 2 \gamma } ) } { \cos ( a \sqrt { 2 \gamma } ) }
\]
若 $\gamma \geq \pi ^ { 2 } / 8 a ^ { 2 }$,则 $E _ { x } e ^ { \gamma \tau } \equiv \infty$.

    \end{thm}
    
    
    
    \begin{dfn}
        [Definition-of-Signed-Measure]
        {有符号测度的定义}
        [Definition of Signed Measure]
        [gpt-4.1]
        设 $(\Omega, \mathcal{F})$ 为一个可测空间.$\alpha$ 被称为 $(\Omega, \mathcal{F})$ 上的有符号测度,若满足以下条件:
(i) $\alpha$ 取值在 $(-\infty, \infty]$;
(ii) $\alpha(\emptyset) = 0$;
(iii) 若 $E = \bigsqcup_i E_i$ 是不交并,则有
\[
\alpha(E) = \sum_i \alpha(E_i)
\]
其中,若 $\alpha(E) < \infty$,则级数绝对收敛且等于 $\alpha(E)$;若 $\alpha(E) = \infty$,则有 $\sum_i \alpha(E_i)^- < \infty$ 且 $\sum_i \alpha(E_i)^+ = \infty$.

    \end{dfn}
    
    
    
    \begin{thm}
        [Theorem-on-the-Expression-of-the-Greens-Function-in-Two-Dimensions]
        {二维情况下的格林函数表达式定理}
        [Theorem on the Expression of the Green's Function in Two Dimensions]
        [gpt-4.1]
        
在 $d = 2$ 时,若 $0 < |y| < 1$,则
\[
G_D(x, y) = \frac{-1}{\pi} \left( \ln|x - y| - \ln|\,|x|y| - y|y|^{-1}| \right)
\]

    \end{thm}
    
    
    
    \begin{thm}
        [Theorem-on-Mutual-Inverses-of-Radon-Nikodym-Derivatives]
        {Radon-Nikodym 导数互为倒数的定理}
        [Theorem on Mutual Inverses of Radon-Nikodym Derivatives]
        [gpt-4.1]
        
如果 $
u \ll \mu$ 且 $\mu \ll 
u$,则有
\[
\frac{d\mu}{d
u} = \left(\frac{d
u}{d\mu}\right)^{-1}.
\]

    \end{thm}
    
    
    
    \begin{lma}
        [Expression-for-the-Function-$\varphix$]
        {关于函数$\varphi(x)$的表达式}
        [Expression for the Function $\varphi(x)$]
        [gpt-4.1]
        由引理 9.1.3 可得
\[
\varphi(x) = \varphi(r) P_{x}(S_{r} < S_{R}) + \varphi(R) (1 - P_{x}(S_{r} < S_{R}))
\]
其中 $\varphi(r)$ 表示 $\varphi(x)$ 在 $\{ x : |x| = r \}$ 上的取值.

    \end{lma}
    
    
    
    \begin{ppt}
        [Solution-for-Probability-$P-{x}S-{r}-<-S-{R}$]
        {概率$P\_{x}(S\_{r} < S\_{R})$的解}
        [Solution for Probability $P_{x}(S_{r} < S_{R})$]
        [gpt-4.1]
        解得
\[
P_{x}(S_{r} < S_{R}) = \frac{\varphi(R) - \varphi(x)}{\varphi(R) - \varphi(r)}
\]

    \end{ppt}
    
    
    
    \begin{thm}
        [Lebesgue-Decomposition-Theorem]
        {Lebesgue分解定理}
        [Lebesgue Decomposition Theorem]
        [gpt-4.1]
        
设 $\mu, 
u$ 是 $\sigma$-有限测度,则 $
u$ 可以表示为 $
u_{r} + 
u_{s}$,其中 $
u_{s}$ 关于 $\mu$ 是互为奇异的,且
\[
u_{r}(E) = \int_{E} g \, d\mu
\]

    \end{thm}
    
    
    
    \begin{thm}
        [Theorem_on_$
    u$_Satisfying_Condition_b]
        {关于 $
    u$ 满足条件 (b) 的定理}
        [Theorem on $
    u$ Satisfying Condition (b)]
        [gpt-4.1]
        $
u$ satisfies (b).
    \end{thm}
    
    
    
    \begin{prf}
        [Proof_of_Theorem_on_$
    u$_Satisfying_Condition_b]
        {关于 $
    u$ 满足条件 (b) 的证明}
        [Proof of Theorem on $
    u$ Satisfying Condition (b)]
        [gpt-4.1]
        若 $| c | \leq M$,则 $e^{ - M t } \leq \exp ( c _ { t } ) \leq e^{ M t }$,所以 $\exp ( c _ { t } ) \to 1$ 当 $t \to 0$.
    \end{prf}
    
    
    
    \begin{thm}
        [Theorem-of-Differentiating-Under-the-Integral]
        {积分下求导定理}
        [Theorem of Differentiating Under the Integral]
        [gpt-4.1]
        设 $(S, S, \mu)$ 是一个测度空间.设 $f$ 是在 $\mathbf{R} \times S$ 上定义的复值函数.令 $\delta > 0$,假设对 $x \in (y - \delta, y + \delta)$,有
\[
u(x) = \int_{S} f(x, s)\, \mu(ds) \quad \text{且} \quad \int_{S} |f(x, s)|\, \mu(ds) < \infty
\]
(ii) 对于固定的 $s$,$\frac{\partial f}{\partial x}(x, s)$ 存在且是 $x$ 的连续函数,

(iii) $
u(x) = \int_{S} \frac{\partial f}{\partial x}(x, s)\, \mu(ds)$ 在 $x = y$ 处连续,
\[
\int_{S} \int_{-\delta}^{\delta} \left| \frac{\partial f}{\partial x}(y + \theta, s) \right| d\theta\, \mu(ds) < \infty
\]
则 $u'(y) = 
u(y)$.

    \end{thm}
    
    
    
    \begin{thm}
        [Theorem-on-Smoothness-of-Solution-under-Hölder-Continuity]
        {Hölder连续性条件下的解的光滑性定理}
        [Theorem on Smoothness of Solution under Hölder Continuity]
        [gpt-4.1]
        如果 $f$ 和 $c$ 是 Hölder 连续的,则 $
u \in C^{1,2}$,因此满足(a).
    \end{thm}
    
    
    
    \begin{xmp}
        [Example-of-a-Nonmeasurable-Set]
        {不可测集的例子}
        [Example of a Nonmeasurable Set]
        [gpt-4.1]
        我们现在给出一个在 $\mathbf { R }$ 上不可测的集合 $B$ 作为例子.
    \end{xmp}
    
    
    
    \begin{thm}
        [Derivative-Formula-for-Infinite-Series]
        {无穷级数求导公式}
        [Derivative Formula for Infinite Series]
        [gpt-4.1]
        
若 $p \in (0, 1)$,则
\[
\left( \sum_{n=1}^{\infty} (1 - p)^{n} \right)^{\prime} = - \sum_{n=1}^{\infty} n (1 - p)^{n-1}
\]

    \end{thm}
    
    
    
    \begin{prf}
        [Proof-of-the-Derivative-Formula-for-Infinite-Series]
        {无穷级数求导公式的证明}
        [Proof of the Derivative Formula for Infinite Series]
        [gpt-4.1]
        
设 $\sum_{n=1}^{\infty} |(1 - x)^{n}| < \infty$,令 $f_{n}(x) = (1 - x)^{n}$,并令 $y = p$,则 $f_{n}^{\prime}(x) = n (1 - x)^{n-1}$.选择 $\delta$,使得 $[y - \delta, y + \delta] \subset (0, 1)$,且 $x$ 在 $[y - \delta, y + \delta]$ 上连续.

    \end{prf}
    
    
    
    \begin{lma}
        [Lemma-on-Approximation-of-Sets-and-Measures]
        {关于集合近似和测度的引理}
        [Lemma on Approximation of Sets and Measures]
        [gpt-4.1]
        
设 $E$ 是任意集合,满足 $\mu ^ { * } ( E ) < \infty$.

(i) 对任意 $\epsilon > 0$,存在 $A \in \mathcal{A} _ { \sigma }$,使得 $A \supset E$ 且 $\mu ^ { * } ( A ) \leq \mu ^ { * } ( E ) + \epsilon$.

(ii) 对任意 $\epsilon > 0$,存在 $B \in \mathcal{A}$,使得 $\mu ( B \Delta E ) \leq 2 \epsilon$.

(iii) 存在 $C \in \mathcal{A} _ { \sigma \delta }$,使得 $C \supset E$ 且 $\mu ^ { * } ( C ) = \mu ^ { * } ( E )$.

    \end{lma}
    
    
    
    \begin{thm}
        [Questions-Related-to-the-Proof-of-the-Central-Limit-Theorem]
        {中心极限定理的证明相关问题}
        [Questions Related to the Proof of the Central Limit Theorem]
        [gpt-4.1]
        
2 with $\mathcal{F}_{n} = \sigma(X_{m}: m \leq n+1)$ to show that there are constants $\mu$ and $\sigma$ so that $(S_{n} - n\mu)/(\sigma n^{1/2}) \Rightarrow \chi$.

    \end{thm}
    
    
    
    \begin{xmp}
        [Example-of-the-Random-Variable-$Y-0$-Constructed-in-the-Proof-of-Theorem-8.3]
        {关于定理8.3证明中构造的随机变量$Y\_0$的例子}
        [Example of the Random Variable $Y_0$ Constructed in the Proof of Theorem 8.3]
        [gpt-4.1]
        
What is the random variable $Y_{0}$ constructed in the proof of Theorem 8.3 in this case?

    \end{xmp}
    
    
    
    \begin{dfn}
        [Definition-of-Symmetric-Difference]
        {对称差的定义}
        [Definition of Symmetric Difference]
        [gpt-4.1]
        定义对称差为 $A \Delta B = ( A - B ) \cup ( B - A )$.
    \end{dfn}
    
    
    
    \begin{xmp}
        [Example-Defining-a-New-Measure-via-a-Measurable-Function]
        {关于测度和可积函数定义新的测度的例子}
        [Example: Defining a New Measure via a Measurable Function]
        [gpt-4.1]
        
设 $\mu$ 是一个测度,$f$ 是一个函数且 $\int f^- d\mu < \infty$,定义 $\alpha(A) = \int_A f d\mu$.

    \end{xmp}
    
    
    
    \begin{thm}
        [Theorem-on-Local-Martingale]
        {关于局部鞅的定理}
        [Theorem on Local Martingale]
        [gpt-4.1]
        
设 $\tau = \inf \{ t > 0 : B_t 
otin G \}$.如果 $u$ 满足条件 (a),则
\[
M_t = u(B_t) + \int_0^t g(B_s) ds
\]
在区间 $[0, \tau)$ 上是一个局部鞅.

    \end{thm}
    
    
    
    \begin{dfn}
        [Definition-of-the-$\alpha$-mixing-coefficient-between-two-$\sigma$-fields]
        {两个$\sigma$-域的$\alpha$关联系数的定义}
        [Definition of the $\alpha$-mixing coefficient between two $\sigma$-fields]
        [gpt-4.1]
        
$\alpha ( \mathcal{G}, \mathcal{H} ) = \sup \{ | P ( A \cap B ) - P ( A ) P ( B ) | : A \in \mathcal{G}, B \in \mathcal{H} \}$

当 $\alpha = 0$ 时,$\mathcal{G}$ 和 $\mathcal{H}$ 独立,因此 $\alpha$ 衡量两个 $\sigma$-域之间的依赖性.

    \end{dfn}
    
    
    
    \begin{thm}
        [Transitivity-of-Absolute-Continuity-and-Singularity]
        {绝对连续性与互不相交性的传递}
        [Transitivity of Absolute Continuity and Singularity]
        [gpt-4.1]
        如果 $\mu_{1} \ll \mu_{2}$ 并且 $\mu_{2} \perp 
u$,那么 $\mu_{1} \perp 
u$.
    \end{thm}
    
    
    
    \begin{thm}
        [Ergodic-Theorem]
        {遍历定理}
        [Ergodic Theorem]
        [gpt-4.1]
        
关于这些序列的基本事实称为遍历定理:若 $E | f ( X _ { 0 } ) | < \infty$,则有
\[
\lim_{n \to \infty} \frac{1}{n} \sum_{m=0}^{n-1} f(X_m) \quad \text{存在}.
\]
如果 $X_n$ 是遍历的(即对马尔可夫链不可约性的推广),则该极限为 $E f ( X _ { 0 } )$.

    \end{thm}
    
    
    
    \begin{thm}
        [Theorem-on-Uniqueness-of-Solution]
        {关于解的唯一性定理}
        [Theorem on Uniqueness of Solution]
        [gpt-4.1]
        
如果 (a) 存在有界解,则其必为
\[
u(x) \equiv E_x \left( \int_0^{\tau} g(B_t) dt \right)
\]
注意,如果 $g \equiv 1$,则解为 $
u(x) = E_x \tau$.

    \end{thm}
    
    
    
    \begin{ppt}
        [Property-of-the-Sum-of-Absolutely-Continuous-Measures]
        {绝对连续测度的和的性质}
        [Property of the Sum of Absolutely Continuous Measures]
        [gpt-4.1]
        
如果 $
u_{1}, 
u_{2} \ll \mu$,则 $
u_{1} + 
u_{2} \ll \mu$,并且
\[
\frac{d(
u_{1} + 
u_{2})}{d\mu} = \frac{d
u_{1}}{d\mu} + \frac{d
u_{2}}{d\mu}.
\]

    \end{ppt}
    
    
    
    \begin{thm}
        [Theorem-9.5.10-Remains-True-When-Cone-is-Replaced-by-Flat-Cone]
        {定理9.5.10在'锥'替换为'平锥'时依然成立}
        [Theorem 9.5.10 Remains True When 'Cone' is Replaced by 'Flat Cone']
        [gpt-4.1]
        证明定理9.5.10在'锥'被'平锥'替换时依然成立.
    \end{thm}
    
    
    
    \begin{dfn}
        [Definition-of-Regular-Point]
        {正则点的定义}
        [Definition of Regular Point]
        [gpt-4.1]
        称 $y \in \partial G$ 为正则点, 如果 $P_y(\tau = 0) = 1$.
    \end{dfn}
    
    
    
    \begin{thm}
        [Theorem-on-Limit-at-Regular-Point]
        {正则点处极限的定理}
        [Theorem on Limit at Regular Point]
        [gpt-4.1]
        设 $G$ 是任意开集.假设 $f$ 有界且连续,$y$ 是 $\partial G$ 的正则点.如果 $x_n \in G$ 且 $x_n \to y$,则 $
u(x_n) \to f(y)$.
    \end{thm}
    
    
    
    \begin{ppt}
        [Chain-Rule-for-Radon-Nikodym-Derivatives]
        {Radon-Nikodym导数的链式法则}
        [Chain Rule for Radon-Nikodym Derivatives]
        [gpt-4.1]
        
若 $\pi \ll 
u \ll \mu$, 则有
\[
\frac{d\pi}{d\mu} = \frac{d\pi}{d
u} \cdot \frac{d
u}{d\mu}.
\]

    \end{ppt}
    
    
    
    \begin{thm}
        [Characterization-of-Measurable-Sets-under-σ-Finiteness]
        {可测集的特征描述定理(σ-有限情况)}
        [Characterization of Measurable Sets under σ-Finiteness]
        [gpt-4.1]
        设 $\mu$ 在 $\Omega$ 上是 $\sigma$-有限的.当且仅当存在 $A \in \mathcal{A}_{\sigma\delta}$ 与集合 $N$ 满足 $\mu^*(N) = 0$,使得 $B = A - N (= A \cap N^c)$ 时,$B \in \mathcal{A}^*$.
    \end{thm}
    
    
    
    \begin{dfn}
        [Definition-of-Subsequence-Construction-and-Limits]
        {子列的构造与极限的定义}
        [Definition of Subsequence Construction and Limits]
        [gpt-4.1]
        设 $m(k, j)$ 是 $m(k - 1, j)$ 的子列,使得当 $j \to \infty$ 时,
\[
\omega_{m(k, j), k} \to \text{极限 } \theta_k
\]
令 $\omega'_i = \omega_{m(i, i), i}$.则 $\omega'_i$ 是所有子列的子列,且对任意 $k$,有 $\omega'_{i, k} \to \theta_k$.
    \end{dfn}
    
    
    
    \begin{thm}
        [Conclusion-under-Theorem-8.5.2]
        {定理8.5.2的条件下的结论}
        [Conclusion under Theorem 8.5.2]
        [gpt-4.1]
        如果定理8.5.2成立,则 $E X_i = 0$ 且 $E X_i^2 = 1$.
    \end{thm}
    
    
    
    \begin{thm}
        [Radon-Nikodym-Theorem]
        {Radon-Nikodym 定理}
        [Radon-Nikodym Theorem]
        [gpt-4.1]
        若 $\mu, 
u$ 是 $\sigma$-有限测度,且 $
u$ 对 $\mu$ 绝对连续,则存在 $g \ge 0$,使得对任意 $E$ 有 $
u(E) = \int_{E} g \, d\mu$.
    \end{thm}
    
    
    
    \begin{dfn}
        [Definition-of-Outer-Measure]
        {外测度的定义}
        [Definition of Outer Measure]
        [gpt-4.1]
        设 $\mu^*$ 为外测度,则对于任意集合 $E$,存在集合族 $\{A_i\}$,使得 $A \equiv \cup_i A_i \supset E$ 且 $\sum_i \mu(A_i) \leq \mu^*(E) + \epsilon$.
    \end{dfn}
    
    
    
    \begin{dfn}
        [Recursive-Definition-of-Return-Times-to-State-$\alpha$]
        {回到状态 $\alpha$ 的时间的递归定义}
        [Recursive Definition of Return Times to State $\alpha$]
        [gpt-4.1]
        设 $R_1 = R$,对于 $k \geq 2$,定义 $R_k = \operatorname*{inf}\{ n > R_{k-1} : \bar{X}_n = \alpha \}$,即 $R_k$ 是第 $k$ 次回到状态 $\alpha$ 的时刻.
    \end{dfn}
    
    
    
    \begin{thm}
        [Unique-Extension-Theorem-for-Measures]
        {测度唯一扩展定理}
        [Unique Extension Theorem for Measures]
        [gpt-4.1]
        若 $\mu$ 是定义在集合族 $\boldsymbol{\mathcal{S}}$ 上的测度,则 $\mu$ 在由 $\boldsymbol{\mathcal{S}}$ 生成的代数 $\bar{\mathcal{S}}$ 上有唯一的扩展 $\bar{\mu}$,且 $\bar{\mu}$ 是 $\bar{\mathcal{S}}$ 上的测度.

若该扩展是 $\sigma$-有限的,则存在唯一的扩展 $
u$,其为 $\sigma(\boldsymbol{\mathcal{S}})$ 上的测度.
    \end{thm}
    
    
    
    \begin{dfn}
        [Definition-of-Completeness-of-Measure-Space]
        {测度空间的完备性定义}
        [Definition of Completeness of Measure Space]
        [gpt-4.1]
        如果测度空间 $(\Omega, \mathcal{F}, \mu)$ 满足 $\mathcal{F}$ 包含所有测度为 0 的集合的子集,则称其为完备的.
    \end{dfn}
    
    
    
    \begin{dfn}
        [Definition-of-Positive-Set]
        {正集合的定义}
        [Definition of Positive Set]
        [gpt-4.1]
        集合 $A$ 被称为正集合,如果对于任意可测的 $B \subset A$ 都有 $\alpha(B) \geq 0$.
    \end{dfn}
    
    
    
    \begin{dfn}
        [Definition-of-Negative-Set]
        {负集合的定义}
        [Definition of Negative Set]
        [gpt-4.1]
        集合 $A$ 被称为负集合,如果对于任意可测的 $B \subset A$ 都有 $\alpha(B) \leq 0$.
    \end{dfn}
    
    
    
    \begin{dfn}
        [Definition-of-Aperiodicity-for-Harris-Chains]
        {周期性 Harris 链的无周期性定义}
        [Definition of Aperiodicity for Harris Chains]
        [gpt-4.1]
        我们称一个经常回归的 Harris 链 $X_n$ 是无周期的,如果 $\operatorname{gcd}\{ n \geq 1 : p^n(\alpha, \alpha) > 0 \} = 1$.
    \end{dfn}
    
    
    
    \begin{thm}
        [Convergence-Theorem-for-Aperiodic-Harris-Chains]
        {无周期 Harris 链的收敛定理}
        [Convergence Theorem for Aperiodic Harris Chains]
        [gpt-4.1]
        设 $X_n$ 是一个无周期的经常回归 Harris 链,具有平稳分布 $\pi$.如果 $P_x(R < \infty) = 1$,则当 $n \to \infty$ 时,
\[
\| p^n(x, \cdot) - \pi(\cdot) \| \to 0
\]
这里 $\| \cdot \|$ 表示两个测度之间的全变差距离.
    \end{thm}
    
    
    
    \begin{thm}
        [Stopping-Time-Property-of-Limit-of-Stopping-Times]
        {停时序列极限的停时性}
        [Stopping Time Property of Limit of Stopping Times]
        [gpt-4.1]
        如果 $T_{n}$ 是一列停时,且 $T_{n} \downarrow T$,则 $T$ 也是停时.
    \end{thm}
    
    
    
    \begin{prf}
        [Proof-of-Stopping-Time-Property-of-Limit-of-Stopping-Times]
        {停时序列极限的停时性证明}
        [Proof of Stopping Time Property of Limit of Stopping Times]
        [gpt-4.1]
        $\{ T < t \} = \cup_{n} \{ T_{n} < t \}$.
    \end{prf}
    
    
    
    \begin{thm}
        [Recurrence-of-Two-Dimensional-Brownian-Motion]
        {二维布朗运动的遍历性}
        [Recurrence of Two-Dimensional Brownian Motion]
        [gpt-4.1]
        
二维布朗运动具有遍历性,即对于任意开集 $G$,任意初始位置 $x$ 和任意 $r>0$,都有
\[
P_{x}(S_{r} < \infty) = 1
\]
并且
\[
P_{x}(B_{t} \in G~i.o.) \equiv 1
\]
即二维布朗运动几乎必然无穷次访问任意开集.

    \end{thm}
    
    
    
    \begin{dfn}
        [Definition-of-Ball-Domain-Exit-Time-and-Poisson-Kernel]
        {球域、退出时间与 Poisson 核的定义}
        [Definition of Ball Domain, Exit Time, and Poisson Kernel]
        [gpt-4.1]
        
设 $D = \{ x : |x| < 1 \}$,令 $\tau_D = \operatorname{inf}\{ t : B_t 
otin D \}$,并定义 Poisson 核为
\[
k_y(x) = \frac{1 - |x|^2}{|x - y|^d}
\]

    \end{dfn}
    
    
    
    \begin{thm}
        [Theorem-on-Recurrence-and-Measure]
        {关于可复性和测度的定理}
        [Theorem on Recurrence and Measure]
        [gpt-4.1]
        设 $\lambda(C) = \sum_{n=1}^\infty 2^{-n} \bar{p}^n(\alpha, C)$.在可复性(recurrent case)下,若 $\bar{\lambda}(C) > 0$,则 $P_\alpha(\bar{X}_n \in C\ i.o.) = 1$.
    \end{thm}
    
    
    
    \begin{thm}
        [Theorem-on-Absolutely-Continuous-Measures-and-Integration]
        {关于绝对连续测度和积分的定理}
        [Theorem on Absolutely Continuous Measures and Integration]
        [gpt-4.1]
        如果 $
u \ll \mu$ 并且 $f \geq 0$,则有
\[
\int f \, d
u = \int f \frac{d
u}{d\mu} \, d\mu.
\]

    \end{thm}
    
    
    
    \begin{dfn}
        [Definition-of-Greens-Function-on-an-Interval]
        {区间上的格林函数的定义}
        [Definition of Green's Function on an Interval]
        [gpt-4.1]
        若固定 $y$,则 $x \to G_{D}(x, y)$ 由以下条件确定:在区间 $[0, y]$ 和 $[y, 1]$ 上分别为线性函数,且满足 $G_{D}(0, y) = G_{D}(1, y) = 0$,以及 $(1/2) G_{D}^{\prime\prime}(y, y) = -\delta_{y}$,其中 $\delta_{y}$ 为在 $y$ 处的点质量.

    \end{dfn}
    
    
    
    \begin{ppt}
        [Jump-Property-of-the-First-Derivative-of-Greens-Function]
        {格林函数一阶导数的跳跃性质}
        [Jump Property of the First Derivative of Green's Function]
        [gpt-4.1]
        $G_{D}^{\prime}$ 在 $y$ 处不连续,且有 $G_{D}^{\prime}(y-, y) - G_{D}^{\prime}(y+, y) = 2$.

    \end{ppt}
    
    
    
    \begin{thm}
        [Invariance-of-Measure-under-Rotation-of-Rectangle-in-Dimension-Two]
        {二维空间中旋转矩形的测度不变性}
        [Invariance of Measure under Rotation of Rectangle in Dimension Two]
        [gpt-4.1]
        
若 $B$ 是矩形 $A$ 的旋转,则 $\lambda^{*}(B) = \lambda(A)$.

    \end{thm}
    
    
    
    \begin{thm}
        [Equivalence-of-Measure-for-Congruent-Figures]
        {全等图形测度的等价性}
        [Equivalence of Measure for Congruent Figures]
        [gpt-4.1]
        
若 $C$ 与 $D$ 全等,则 $\lambda^{*}(C) = \lambda^{*}(D)$.

    \end{thm}
    
    
    
    \begin{xmp}
        [Example-of-Random-Walk-on-a-Tree]
        {树上的随机游走的例子}
        [Example of Random Walk on a Tree]
        [gpt-4.1]
        
我们将系统视为在一个具有三个生成元 $a, b, c$ 的群上的随机游走,这些生成元满足 $a^{2} = b^{2} = c^{2} = e$,其中 $e$ 是单位元.

    \end{xmp}
    
    
    
    \begin{thm}
        [Martingale-Theorem-for-Functions-of-Multidimensional-Brownian-Motion]
        {多维布朗运动函数的马氏过程定理}
        [Martingale Theorem for Functions of Multidimensional Brownian Motion]
        [gpt-4.1]
        
设 $
u \in C^{2}$,即所有一阶和二阶偏导数都存在且连续,且对任意 $t$ 有 $E\int_{0}^{t} |
abla 
u(B_{s})|^{2} ds < \infty$.则
\[
u(B_{t}) - \int_{0}^{t} \frac{1}{2} \Delta 
u(B_{s}) ds
\]
是一个连续的鞅过程(martingale).

    \end{thm}
    
    
    
    \begin{dfn}
        [Definition-of-Generalized-Green-Function-$G-Hx-y$]
        {广义 Green 函数 $G\_H(x, y)$ 的定义}
        [Definition of Generalized Green Function $G_H(x, y)$]
        [gpt-4.1]
        设 $G_H(x, y) = G(x, y) - G(x, \bar{y})$.
    \end{dfn}
    
    
    
    \begin{dfn}
        [Definition-of-Flat-Cone]
        {平坦锥的定义}
        [Definition of Flat Cone]
        [gpt-4.1]
        定义一个平坦锥 $\bar{V}(y, 
u, a)$ 为 $V(y, 
u, a)$ 与一个包含直线 $\{y + \theta 
u : \theta \in \mathbf{R}\}$ 的 $d-1$ 维超平面的交集.
    \end{dfn}
    
    
    
    \begin{thm}
        [Density-Function-of-Brownian-Motion-Conditioned-on-Not-Exiting-Interval]
        {布朗运动在区间内未越界时的概率密度函数公式}
        [Density Function of Brownian Motion Conditioned on Not Exiting Interval]
        [gpt-4.1]
        $B_t$ 在 $\{ T_{a} \land T_{b} > t \}$ 上的密度函数为
\[
\begin{array} { l l r }
{ P _ { x } ( T _ { a } \wedge T _ { b } > t , B _ { t } = y ) = \sum _ { n = - \infty } ^ { \infty } P _ { x } ( B _ { t } = y + 2 n ( b - a ) ) } \\
& { ~ - P _ { x } ( B _ { t } = 2 b - y + 2 n ( b - a ) ) }
\end{array}
\]

    \end{thm}
    
    
    
    \begin{dfn}
        [Definition-of-Reflection-Mapping-at-a-Point]
        {关于点的反射映射的定义}
        [Definition of Reflection Mapping at a Point]
        [gpt-4.1]
        令 $\rho_{a}(y) = 2a - y$ 表示通过点 $a$ 的反射映射.

    \end{dfn}
    
    
    
    \begin{thm}
        [Period-is-a-Class-Property]
        {周期是类属性}
        [Period is a Class Property]
        [gpt-4.1]
        
如果某个状态的周期为 $d$,那么与其属于同一个交流类的所有状态的周期也为 $d$.因此,在上述例子中,所有状态的周期均为 1.

    \end{thm}
    
    
    
    \begin{thm}
        [Jordan-Decomposition-Theorem]
        {约尔当分解定理}
        [Jordan Decomposition Theorem]
        [gpt-4.1]
        设 $\alpha$ 是一个符号测度.则存在互为互异测度的 $\alpha_{+}$ 和 $\alpha_{-}$,使得 $\alpha = \alpha_{+} - \alpha_{-}$.而且,这样的对是唯一的.
    \end{thm}
    
    
    
    \begin{crl}
        [Corollary-on-the-Non-measurability-of-Set-B]
        {不可测集 B 的推论}
        [Corollary on the Non-measurability of Set B]
        [gpt-4.1]
        由集合 $B$ 不可测性可推出:如果 $B$ 可测,则对于所有 $q \in \mathbf{Q} \cap [0,1)$,集合 $q + ^{\prime} B$ 构成[0,1)内一族可测且两两不交的集合,且其测度均为 $\alpha$,满足
\[
\cup_{q \in \mathbf{Q}\cap[0,1)}(q + ^{\prime} B) = [0,1)
\]
若 $\alpha > 0$,则 $\lambda([0,1)) = \infty$;若 $\alpha = 0$,则 $\lambda([0,1)) = 0$.这两种结论均与 $\lambda([0,1)) = 1$ 相矛盾,因此 $B 
otin \bar{\mathcal{R}}$,即 $B$ 不可测.

    \end{crl}
    
    
    
    \begin{dfn}
        [Definition-of-Stationary-Sequence]
        {平稳序列的定义}
        [Definition of Stationary Sequence]
        [gpt-4.1]
        $X _ { n } , n \geq 0$ , 称为平稳序列,如果对于每个 $k \geq 1$,它与移位后的序列 $X _ { n + k } , n \geq 0$ 有相同的分布.
    \end{dfn}
    
    
    
    \begin{thm}
        [Bounded-Harmonic-Functions-Must-Be-Constant]
        {有界调和函数必为常数}
        [Bounded Harmonic Functions Must Be Constant]
        [gpt-4.1]
        设 $\mathbf{f} \in \mathbf{C}^{2}$ 是有界且调和的函数,则 $\mathbf{f}$ 必为常数函数(适用于任意维度).
    \end{thm}
    
    
    
    \begin{dfn}
        [Definition-of-the-Ornstein-Uhlenbeck-Process]
        {Ornstein-Uhlenbeck过程的定义}
        [Definition of the Ornstein-Uhlenbeck Process]
        [gpt-4.1]
        Ornstein-Uhlenbeck过程是一个扩散过程 $\{ V_{t}, t \in [0, \infty) \}$,用于描述悬浮在液体中的粒子的速度.
    \end{dfn}
    
    
    
    \begin{ppt}
        [Decomposition-Formula-for-Markov-Chain-Transition-Probabilities]
        {关于马氏链转移概率的分解公式}
        [Decomposition Formula for Markov Chain Transition Probabilities]
        [gpt-4.1]
        
设 $z 
eq y$,令 $\bar{p}_{n}(x, z) = P_{x}(X_{n} = z, T_{y} > n)$,则有
\[
\sum_{m=1}^{n} p^{m}(x, z) = \sum_{m=1}^{n} \bar{p}_{m}(x, z) + \sum_{j=1}^{n-1} p^{j}(x, y) \sum_{k=1}^{n-j} \bar{p}_{k}(y, z)
\]

    \end{ppt}
    
    
    
    \begin{thm}
        [Limit-Ratio-of-Sums-of-Markov-Chain-Transition-Probabilities]
        {马氏链转移概率和的极限比值}
        [Limit Ratio of Sums of Markov Chain Transition Probabilities]
        [gpt-4.1]
        
对于马氏链,有
\[
\frac{\sum_{m=1}^{n} p^{m}(x, z)}{\sum_{m=1}^{n} p^{m}(x, y)} \to \frac{m(z)}{m(y)}
\]

    \end{thm}
    
    
    
    \begin{dfn}
        [Definition-of-Positive-Recurrent-State]
        {正遍历态的定义}
        [Definition of Positive Recurrent State]
        [gpt-4.1]
        如果一个状态 $x$ 满足 $E_{x} T_{x} < \infty$,则称其为正遍历态(positive recurrent).
    \end{dfn}
    
    
    
    \begin{dfn}
        [Definition-of-Null-Recurrent-State]
        {零遍历态的定义}
        [Definition of Null Recurrent State]
        [gpt-4.1]
        一个遍历态(recurrent state)若满足 $E_{x} T_{x} = \infty$,则称其为零遍历态(null recurrent).
    \end{dfn}
    
    
    
    \begin{dfn}
        [Definition-of-Period-of-a-State]
        {状态周期的定义}
        [Definition of Period of a State]
        [gpt-4.1]
        设 $x$ 是一个遍历状态,令 $I_x = \{ n \geq 1 : p^n(x, x) > 0 \}$,$d_x$ 是 $I_x$ 的最大公约数,则 $d_x$ 称为状态 $x$ 的周期.
    \end{dfn}
    
    
    
    \begin{ppt}
        [Recurrence-Equations-Satisfied-by-the-Stationary-Distribution]
        {平稳分布满足的递推方程}
        [Recurrence Equations Satisfied by the Stationary Distribution]
        [gpt-4.1]
        设 $\pi$ 为满足 $\pi p = \pi$ 的平稳分布,则有如下递推方程:

\[
\begin{array}{rl}
  & \pi(0) = \pi(0) (a_{0} + a_{1}) + \pi(1) a_{0} \\
  & \pi(1) = \pi(0) a_{2} + \pi(1) a_{1} + \pi(2) a_{0} \\
  & \pi(2) = \pi(0) a_{3} + \pi(1) a_{2} + \pi(2) a_{1} + \pi(3) a_{0}
\end{array}
\]

一般地,对 $j \geq 1$,有
\[
\pi(j) = \sum_{i = 0}^{j + 1} \pi(i) a_{j + 1 - i}
\]

这些方程呈'三角形'结构,已知 $\pi(0)$ 可以递推求解 $\pi(1), \pi(2), \ldots$.

    \end{ppt}
    
    
    
    \begin{thm}
        [Uniqueness-Theorem-of-Measures-Agreeing-on-a-$\pi$-system]
        {测度在$\pi$-系统上的一致性唯一性定理}
        [Uniqueness Theorem of Measures Agreeing on a $\pi$-system]
        [gpt-4.1]
        
设 $\mathcal{P}$ 是一个$\pi$-系统.如果 $
u_1$ 和 $
u_2$ 是定义在$\sigma$-域 $\mathcal{F}_1$ 和 $\mathcal{F}_2$上的测度,且它们在$\mathcal{P}$上一致,并且存在序列 $A_n \in \mathcal{P}$ 使得 $A_n \uparrow \Omega$ 且 $
u_i(A_n) < \infty$,那么 $
u_1$ 和 $
u_2$ 在 $\sigma(\mathcal{P})$ 上一致.

    \end{thm}
    
    
    
    \begin{thm}
        [Strassens-Invariance-Principle]
        {Strassen不变原理}
        [Strassen's Invariance Principle]
        [gpt-4.1]
        
设 $X_{1}, X_{2}, \ldots$ 是独立同分布的随机变量,满足 $E X_{i} = 0$ 且 $E X_{i}^{2} = 1$,令 $S_{n} = X_{1} + \cdots + X_{n}$,$S(n\cdot)$ 是通常的线性插值.则序列
\[
Z_{n}(\cdot) = (2n \log \log n)^{-1/2} S(n\cdot), \quad n \geq 3
\]
的极限集(即收敛子列的极限的集合)为
\[
\left\{ f : f(x) = \int_{0}^{x} g(y) dy,\ \int_{0}^{1} g(y)^{2} dy \leq 1 \right\}.
\]

    \end{thm}
    
    
    
    \begin{ppt}
        [Countable-Additivity-and-Subadditivity-of-Outer-Measure]
        {外测度的可列可加性与次可加性}
        [Countable Additivity and Subadditivity of Outer Measure]
        [gpt-4.1]
        
(a) 若 $A, B _ { i } \in \bar { \mathcal { S } }$ 且 $A = \bigcup_{i = 1}^{n} B_{i}$,则
\[
\bar{\mu}(A) = \sum_{i} \bar{\mu}(B_{i})
\]
(b) 若 $A, B _ { i } \in \bar { \mathcal { S } }$ 且 $A \subset \bigcup_{i = 1}^{n} B_{i}$,则
\[
\bar{\mu}(A) \leq \sum_{i} \bar{\mu}(B_{i})
\]

    \end{ppt}
    
    
    
    \begin{dfn}
        [Definition-of-M-dependent-Sequence]
        {M-依赖序列的定义}
        [Definition of M-dependent Sequence]
        [gpt-4.1]
        设 $X_{n}$, $n \in \mathbf{Z}$, 是一个平稳序列,满足 $EX_{n} = 0$, $EX_{n}^{2} < \infty$,并且 $\{X_{j}, j \le 0\}$ 与 $\{X_{k}, k > M\}$ 相互独立,则称该序列为 $M$-依赖序列.
    \end{dfn}
    
    
    
    \begin{crl}
        [Corollary-Brownian-Motion-in-Three-Dimensions-Does-Not-Hit-a-Straight-Line]
        {三维布朗运动不经过直线的推论}
        [Corollary: Brownian Motion in Three Dimensions Does Not Hit a Straight Line]
        [gpt-4.1]
        布朗运动在 $d = 3$ 维空间中不会经过一条直线.
    \end{crl}
    
    
    
    \begin{thm}
        [Law-of-the-Iterated-Logarithm-for-Brownian-Motion]
        {布朗运动的迭代对数律}
        [Law of the Iterated Logarithm for Brownian Motion]
        [gpt-4.1]
        
设 $B_t$ 为布朗运动,则有
\[
\limsup_{t \to \infty} \frac{B_t}{\left(2t \log\log t\right)^{1/2}} = 1 \quad a.s.
\]
其中LIL表示'迭代对数律',名称源于分母中的$\log\log t$.

    \end{thm}
    
    
    
    \begin{thm}
        [Cesaro-Convergence-Theorem]
        {Cesaro 收敛性定理}
        [Cesaro Convergence Theorem]
        [gpt-4.1]
        
(5.6.1) 给出:序列 $p^n(x, y)$ 总是以 Cesaro 的方式收敛.更具体地,有
\[
\frac{1}{n} \sum_{m=1}^n p^m(x, y) \to \frac{\rho_{xy}}{E_y T_y}
\]
其中 $\rho_{xy}$ 是适当的常数,$E_y T_y$ 是从 $y$ 返回到 $y$ 的期望时间.该结果对 $y$ 的递归和暂态情况都成立:当 $y$ 为递归时,极限为正;当 $y$ 为暂态时,$E_y T_y = \infty$,极限为 0,因此 $\displaystyle\sum_{m} p^m(x, y) < \infty$.

    \end{thm}
    
    
    
    \begin{dfn}
        [Definition-of-Hölder-Continuous-Function]
        {Hölder连续函数的定义}
        [Definition of Hölder Continuous Function]
        [gpt-4.1]
        
当我们称一个函数是 Hölder 连续时,指的是存在某个 $\alpha > 0$ 和常数 $C$,使得对所有 $x, y$,有
\[
|g(x) - g(y)| \leq C |x - y|^\alpha.
\]

    \end{dfn}
    
    
    
    \begin{thm}
        [Hölder_Continuity_Implies_$
    u_\in_C^{12}$]
        {Hölder连续性蕴含$
    u$属于$C^{1,2}$}
        [Hölder Continuity Implies $
    u \in C^{1,2}$]
        [gpt-4.1]
        
定理 9.3.5 如果 $g$ 是 Hölder 连续的,则 $
u \in C^{1,2}$.

    \end{thm}
    
    
    
    \begin{thm}
        [Sufficient-Conditions-for-$C^2$-Regularity-of-the-Function-$w$]
        {关于函数 $w$ 的二阶可微性的充分条件}
        [Sufficient Conditions for $C^2$ Regularity of the Function $w$]
        [gpt-4.1]
        在 $d \ge 2$ 时,如果 $g$ 是 Hölder 连续的,则 $w$ 是 $C^2$ 的.
在 $d = 1$ 时,只需假设 $g$ 是连续的,$w$ 即为 $C^2$.
    \end{thm}
    
    
    
    \begin{xmp}
        [Example-of-Countable-State-Space]
        {可数状态空间的例子}
        [Example of Countable State Space]
        [gpt-4.1]
        
如果 $S$ 是可数的,并且存在一个点 $a$,使得对所有 $x$ 都有 $\rho_{x a} > 0$(该条件比不可约性稍弱),那么我们可以取 $A = \{ a \}$,$B = \{ b \}$,其中 $b$ 是任意满足 $p(a, b) > 0$ 的状态,$\mu = \delta_b$ 是在 $b$ 处的点质量,且 $q(a, b) = p(a, b)$.

    \end{xmp}
    
    
    
    \begin{thm}
        [Carathéodorys-Extension-Theorem]
        {Carathéodory扩张定理}
        [Carathéodory's Extension Theorem]
        [gpt-4.1]
        设$\mu$是代数$\mathcal{A}$上的$\sigma$-有限测度.那么$\mu$在$\sigma(\mathcal{A})$(包含$\mathcal{A}$的最小$\sigma$-代数)上有且仅有一个扩张.
    \end{thm}
    
    
    
    \begin{thm}
        [Bounded-Harmonic-Functions-Are-Constant]
        {有界调和函数为常数}
        [Bounded Harmonic Functions Are Constant]
        [gpt-4.1]
        
设 $h$ 是定义在 $\mathbf{R}^d$ 上的有界调和函数,则 $h$ 必为常数.

    \end{thm}
    
    
    
    \begin{thm}
        [Theorem-of-Completion-of-Measure-Space]
        {测度空间的完备化定理}
        [Theorem of Completion of Measure Space]
        [gpt-4.1]
        如果 $(\Omega, \mathcal{F}, \mu)$ 是一个测度空间,则存在一个完备测度空间 $(\Omega, \bar{\mathcal{F}}, \bar{\mu})$,称为 $(\Omega, \mathcal{F}, \mu)$ 的完备化,使得:
(i) 当且仅当 $E = A \cup B$,其中 $A \in \mathcal{F}$ 且 $B \subset N \in \mathcal{F}$ 且 $\mu(N) = 0$ 时,$E \in \bar{\mathcal{F}}$;
(ii) $\bar{\mu}$ 在 $\mathcal{F}$ 上与 $\mu$ 一致.

    \end{thm}
    
    
    
    \begin{lma}
        [Lemma-on-Recurrence-and-Transition-Probability]
        {递归性与转移概率的引理}
        [Lemma on Recurrence and Transition Probability]
        [gpt-4.1]
        如果 $x$ 是递归的且 $\rho_{xy} > 0$,那么 $y$ 也是递归的并且 $\rho_{yx} = 1$.
    \end{lma}
    
    
    
    \begin{thm}
        [Hölder_Continuity_of_the_Function_$
    u$]
        {函数 $
    u$ 的 Hölder 连续性}
        [Hölder Continuity of the Function $
    u$]
        [gpt-4.1]
        
设 $c$ 和 $f$ 有界,且 $f$ 是 Hölder 连续的,$c(x)$ 也是 Hölder 连续的,则 $
u(r, x)$ 在 $x$ 局部是 Hölder 连续的.

    \end{thm}
    
    
    
    \begin{thm}
        [Theorem-on-the-Nonmeasurability-of-Set-B]
        {不可测集B的不可测性定理}
        [Theorem on the Nonmeasurability of Set B]
        [gpt-4.1]
        设 $B$ 是一个不可测集,则 $B 
otin \bar { \mathcal { R } }$.
    \end{thm}
    
    
    
    \begin{dfn}
        [Definition-of-the-Distance-Parameter-αₙ-for-a-Transition-Matrix]
        {转移矩阵的距离参数αₙ的定义}
        [Definition of the Distance Parameter αₙ for a Transition Matrix]
        [gpt-4.1]
        
对于任意转移矩阵 $p$,定义
\[
\alpha_{n} = \sup_{i, j} \frac{1}{2} \sum_{k} |p^{n}(i, k) - p^{n}(j, k)|
\]
其中 $\sup_{i, j}$ 表示对所有状态 $i$ 和 $j$ 取上确界,$\frac{1}{2} \sum_{k} |p^{n}(i, k) - p^{n}(j, k)|$ 表示第 $n$ 步时从状态 $i$ 和 $j$ 到所有状态 $k$ 的分布之间的总变差距离的一半.

    \end{dfn}
    
    
    
    \begin{thm}
        [Uniqueness-and-Form-of-Bounded-Solution]
        {关于有界解的唯一性及形式}
        [Uniqueness and Form of Bounded Solution]
        [gpt-4.1]
        如果存在在 $[0, T] \times \mathbf{R}^{d}$ 上有界的解,对于任意 $T < \infty$,则该解必为

\[
u(t, x) \equiv E_{x} \left( \int_{0}^{t} g(B_{s})\, ds \right)
\]

    \end{thm}
    
    
    
    \begin{prf}
        [Proof-of-the-Uniqueness-Theorem-for-Bounded-Solution]
        {有界解唯一性定理的证明}
        [Proof of the Uniqueness Theorem for Bounded Solution]
        [gpt-4.1]
        在 $g$ 和 $u$ 满足假设的条件下,$M_{s}, 0 \le s < t$ 按照定理9中的定义.
    \end{prf}
    
    
    
    \begin{thm}
        [Necessary-and-Sufficient-Condition-for-Positivity-of-Set-A]
        {集合A的正性充要条件}
        [Necessary and Sufficient Condition for Positivity of Set A]
        [gpt-4.1]
        $A$ 是正的当且仅当 $\mu(A \cap \{x : f(x) < 0\}) = 0$.
    \end{thm}
    
    
    
    \begin{thm}
        [Theorem-on-Measure-Decomposition-for-Countable-Disjoint-Unions]
        {关于可列不交并的测度分解定理}
        [Theorem on Measure Decomposition for Countable Disjoint Unions]
        [gpt-4.1]
        
若 $E_{i} = A_{i} \cup B_{i}$ 彼此不交,则 $\cup_{i} E_{i}$ 可以分解为 $\cup_{i} A_{i} \cup ( \cup_{i} B_{i} )$,且 $A_{i} \subset E_{i}$ 彼此不交,因此有
\[
\bar{\mu} ( \cup_{i} E_{i} ) = \mu ( \cup_{i} A_{i} ) = \sum_{i} \mu ( A_{i} ) = \sum_{i} \bar{\mu} ( E_{i} )
\]

    \end{thm}
    
    
    
    \begin{thm}
        [Direct-Proof-of-Ergodicity-of-Rotation-of-the-Circle]
        {圆上旋转的遍历性直接证明}
        [Direct Proof of Ergodicity of Rotation of the Circle]
        [gpt-4.1]
        (i) 若 $\theta$ 是无理数,则序列 $x_n = n\theta \bmod 1$ 在区间 $[0,1)$ 中稠密.
(ii) 若 $A$ 是测度为正的 Borel 集合,则对任意 $\delta > 0$,存在区间 $\boldsymbol{J} = [a, b)$ 使得 $|A \cap J| > (1 - \delta)|J|$.
(iii) 结合 (i) 和 (ii) 可得 $P(A) = 1$,即圆上无理数旋转是遍历的.

    \end{thm}
    
    
    
    \begin{thm}
        [Proof-of-Theorem-A.1.3]
        {定理A.1.3的证明}
        [Proof of Theorem A.1.3]
        [gpt-4.1]
        定理A.1.3的证明定义了一个扩展到 $\mathcal{A}^* \supset \sigma(\mathcal{A})$.
    \end{thm}
    
    
    
    \begin{thm}
        [Kolmogorovs-Extension-Theorem]
        {Kolmogorov扩展定理}
        [Kolmogorov's Extension Theorem]
        [gpt-4.1]
        
设我们给定在$(\mathbf{R}^{n}, \mathcal{R}^{n})$上的概率测度$\mu_{n}$,它们是一致的,即
\[
\mu_{n+1}((a_{1}, b_{1}] \times \ldots \times (a_{n}, b_{n}] \times \mathbf{R}) = \mu_{n}((a_{1}, b_{1}] \times \ldots \times (a_{n}, b_{n}])
\]
那么存在唯一的概率测度$P$在$(\mathbf{R}^{\mathbf{N}}, \mathcal{R}^{\mathbf{N}})$上,使得
\[
P(\omega : \omega_{i} \in (a_{i}, b_{i}], 1 \le i \le n) = \mu_{n}((a_{1}, b_{1}] \times \ldots \times (a_{n}, b_{n}])
\]

    \end{thm}
    
    
    
    \begin{thm}
        [Theorem-on-the-Expression-of-Bounded-Solution]
        {有界解的表达式定理}
        [Theorem on the Expression of Bounded Solution]
        [gpt-4.1]
        
如果存在一个有界解 $
u(x)$,则它必须满足
\[
u(x) \equiv E_{x} ( f(B_{\tau}) \exp ( c_{\tau} ) )
\]

    \end{thm}
    
    
    
    \begin{prf}
        [Proof-of-the-Theorem-on-the-Expression-of-Bounded-Solution]
        {有界解表达式定理的证明}
        [Proof of the Theorem on the Expression of Bounded Solution]
        [gpt-4.1]
        
证明 定理 9.8.1 蕴含 $M_{s} = u(B_{s}) \exp ( c_{s} )$ 是区间 $[0, \tau)$ 上的局部鞅.由于 $f$、$c$ 和 $u$ 有界,令 $s \uparrow \tau \wedge t$ 并利用有界收敛定理可得结论.

    \end{prf}
    
    
    
    \begin{lma}
        [Lemma-on-Measurability-and-Equivalence-of-$\mu^*$]
        {关于$\mu^*$的可测性和等价性}
        [Lemma on Measurability and Equivalence of $\mu^*$]
        [gpt-4.1]
        如果 $A \in \mathcal{A}$,则 $\mu^*(A) = \mu(A)$,且 $A$ 是可测的.
    \end{lma}
    
    
    
    \begin{dfn}
        [Recursive-Definition-of-$X-n$-and-$Y-n$]
        {关于 $X\_n$ 和 $Y\_n$ 的递推定义}
        [Recursive Definition of $X_n$ and $Y_n$]
        [gpt-4.1]
        令 $X_{n} = X_{n-1} + Z_{n}$ 对于 $n \geq 1$.

若 $X_{n-1} = Y_{n-1}$,则取 $X_{n} = Y_{n}$.

否则,令
\[
Y_{n} = \left\{
  \begin{array}{ll}
    Y_{n-1} + Z_{n} & \mathrm{if}\ |Z_{n}| > m \\
    \\
    Y_{n-1} + W_{n} & \mathrm{if}\ |Z_{n}| \leq m
  \end{array}
\right.
\]

    \end{dfn}
    
    
    
    \begin{dfn}
        [Definition-of-First-Exit-Time]
        {首次出界时间的定义}
        [Definition of First Exit Time]
        [gpt-4.1]
        令 $\tau = \operatorname*{inf} \{ t > 0 : B_{t} 
otin G \}$.
    \end{dfn}
    
    
    
    \begin{thm}
        [Preservation-of-Stationarity-and-Ergodicity-under-Function-Transformation]
        {函数作用下的平稳性与遍历性的保持}
        [Preservation of Stationarity and Ergodicity under Function Transformation]
        [gpt-4.1]
        若 $X_0, X_1, \ldots$ 是一个平稳序列,$g : \mathbb{R}^{\{0,1,\ldots\}} \to \mathbb{R}$ 是可测的,则 $Y_k = g(X_k, X_{k+1}, \ldots)$ 也是一个平稳序列.如果 $X_n$ 是遍历的,则 $Y_n$ 也是遍历的.
    \end{thm}
    
    
    
    \begin{thm}
        [Equality-Between-Integral-of-Bounded-Measurable-Functions-and-Greens-Function]
        {关于有界可测函数积分与格林函数的等式}
        [Equality Between Integral of Bounded Measurable Functions and Green's Function]
        [gpt-4.1]
        
如果 $g$ 是有界且可测的函数,则有
\[
E_x \int_0^{\tau} g(B_t) dt = \int G_D(x, y) g(y) dy
\]
其中
\[
G_D(x, y) = G(x, y) - \int \frac{1 - |x|^2}{|x - z|^d} G(z, y) \pi(dz)
\]

    \end{thm}
    
    
    
    \begin{thm}
        [Quadratic-Formula-for-Itô-Integral]
        {伊藤积分的平方公式}
        [Quadratic Formula for Itô Integral]
        [gpt-4.1]
        
设 $B_t$ 是标准布朗运动,$B_0 = 0$.则有
\[
\int_{0}^{t} 2 B_{s}\, dB_{s} = B_{t}^{2} - t
\]
该公式与经典微积分公式 $\int_{0}^{a} 2x\, dx = a^{2}$ 形成对比,体现了伊藤积分的不同性质.

    \end{thm}
    
    
    
    \begin{prf}
        [Direct-Proof-of-the-Quadratic-Formula-for-Itô-Integral]
        {伊藤积分平方公式的直接证明}
        [Direct Proof of the Quadratic Formula for Itô Integral]
        [gpt-4.1]
        
利用先前引入的分割,可以写为
\[
B_{t}^{2} = \left( \sum_{i=0}^{2^{n} - 1} (B(t_{i+1}^{n}) - B(t_{i}^{n})) \right)^{2}
\]
从而可以展开和计算 $\int_{0}^{t} 2 B_{s}\, dB_{s}$,从而得到公式 $\int_{0}^{t} 2 B_{s}\, dB_{s} = B_{t}^{2} - t$.

    \end{prf}
    
    
    
    \begin{xmp}
        [Example-of-Markov-Chains-with-Continuous-Densities]
        {具有连续密度的马尔可夫链的例子}
        [Example of Markov Chains with Continuous Densities]
        [gpt-4.1]
        
设 $X_n \in \mathbf{R}^d$ 是一个马尔可夫链,其转移概率满足 $p(x, dy) = p(x, y) dy$,其中 $(x, y) \to p(x, y)$ 是连续的.选取 $(x_0, y_0)$ 使得 $p(x_0, y_0) > 0$.令 $A$ 和 $B$ 分别为围绕 $x_0$ 和 $y_0$ 的开集,且足够小,使得 $p(x, y) \geq \epsilon > 0$ 在 $A \times B$ 上成立.若令 $\rho(C) = |B \cap C| / |B|$,其中 $|B|$ 是 $B$ 的勒贝格测度,则 (ii) 成立.如果 (i) 成立,则 $X_n$ 是 Harris 链.

    \end{xmp}
    
    
    
    \begin{thm}
        [Theorem-on-Brownian-Path-Escaping-Balls]
        {布朗运动路径逃离球体的定理}
        [Theorem on Brownian Path Escaping Balls]
        [gpt-4.1]
        
定理 9.1.4 当 $t \to \infty$ 时,$|B_{t}| \to \infty$.

证明 令 $A_{n} = \{ |B_{t}| > n^{1-\epsilon}~\mathrm{对于所有}~t \geq S_{n} \}$.强马尔可夫性说明

\[
P_{x}(A_{n}^{c}) = E_{x}(P_{B(S_{n})}(S_{n^{1-\epsilon}} < \infty)) = (n^{1-\epsilon}/n)^{d-2} \to 0
\]

当 $n \to \infty$ 时.

现在,$\limsup A_{n} = \cap_{N=1}^{\infty} \cup_{n=N}^{\infty} A_{n}$ 满足

\[
P(\limsup A_{n}) \geq \limsup P(A_{n}) = 1
\]

因此,无穷多次地,布朗运动路径在时间 $S_{n}$ 之后永不回到 $\{ x : |x| \leq n^{1-\epsilon} \}$,这意味着所需结论成立.

    \end{thm}
    
    
    
    \begin{thm}
        [Brownian-Bridge-is-a-Markov-Process]
        {布朗桥是马尔可夫过程的判定}
        [Brownian Bridge is a Markov Process]
        [gpt-4.1]
        
设 $s_1 < \ldots < s_n < s < t < 1$,则
\[
P(B(t) = y \mid B(s) = x, B(s_n) = x_n, \ldots, B(s_1) = x_1) = P(B(t) = y \mid B(s) = x)
\]
说明布朗桥过程 $B(t)$ 满足马尔可夫性质,即给定当前状态,其未来状态的分布仅依赖于当前状态,而与过去无关.

    \end{thm}
    
    
    
    \begin{prf}
        [Proof-of-Theorem-7.5.1]
        {定理7.5.1的证明}
        [Proof of Theorem 7.5.1]
        [gpt-4.1]
        由定理7.5.1有
\[
E ( B ( T \wedge t )^{4} - 6 ( T \wedge t ) B ( T \wedge t )^{2} ) = -3 E ( T \wedge t )^{2}.
\]

又由定理7.5.5知 $E T = a^{2} < \infty$.令 $t \to \infty$,对左侧用主控收敛定理,对右侧用单调收敛定理,有
\[
a^{4} - 6 a^{2} E T = -3 E ( T^{2} )
\]

代入 $E T = a^{2}$ 即得所需结果.

    \end{prf}
    
    
    
    \begin{dfn}
        [Definition-of-Aperiodic-Markov-Chain]
        {无周期马尔可夫链的定义}
        [Definition of Aperiodic Markov Chain]
        [gpt-4.1]
        如果一个链是不可约的且 $d_{x} = 1$,则称其为无周期(aperiodic)的.
    \end{dfn}
    
    
    
    \begin{ppt}
        [Simple-Criterion-for-Aperiodic-State]
        {判定无周期状态的简便方法}
        [Simple Criterion for Aperiodic State]
        [gpt-4.1]
        最简单的判定方法是找到一个状态使得 $p(x, x) > 0$.
    \end{ppt}
    
    
    
    \begin{ppt}
        [Aperiodicity-of-M/G/1-Queue]
        {M/G/1排队模型的无周期性}
        [Aperiodicity of M/G/1 Queue]
        [gpt-4.1]
        $M/G/1$ 排队模型有 $a_{k} > 0$ 对所有 $k \geq 0$,因此具有无周期性.
    \end{ppt}
    
    
    
    \begin{ppt}
        [Aperiodicity-Criterion-for-Renewal-Chain]
        {更新链的无周期判定}
        [Aperiodicity Criterion for Renewal Chain]
        [gpt-4.1]
        若 $\gcd \{ k : f_{k} > 0 \} = 1$,则更新链是无周期的.
    \end{ppt}
    
    
    
    \begin{dfn}
        [Definition-of-Outer-Measure]
        {外测度的定义}
        [Definition of Outer Measure]
        [gpt-4.1]
        对于 $E \subset \Omega$,我们定义
\[
\mu^*(E) = \inf \sum_{i} \mu(A_{i}),
\]
其中下确界取遍所有满足 $E \subset \cup_{i} A_{i}$ 的 $\mathcal{A}$ 中的序列 $\{A_i\}$.
    \end{dfn}
    
    
    
    \begin{dfn}
        [Definition-of-Measurable-Set]
        {可测集的定义}
        [Definition of Measurable Set]
        [gpt-4.1]
        称 $E$ 是可测的,如果对于所有 $F \subset \Omega$,
\[
\mu^*(F) = \mu^*(F \cap E) + \mu^*(F \cap E^{c})
\]
成立.
    \end{dfn}
    
    
    
    \begin{thm}
        [Property_of_$C^{12}$_Function_$
    u$]
        {关于 $C^{1,2}$ 函数 $
    u$ 的性质}
        [Property of $C^{1,2}$ Function $
    u$]
        [gpt-4.1]
        
如果 $
u \in C^{1,2}$,则它满足性质 (a).

    \end{thm}
    
    
    
    \begin{prf}
        [Proof_of_Property_of_$C^{12}$_Function_$
    u$]
        {关于 $C^{1,2}$ 函数 $
    u$ 的性质的证明}
        [Proof of Property of $C^{1,2}$ Function $
    u$]
        [gpt-4.1]
        
如果我们考虑运行时空布朗运动 $(t - s, B_s)$ 来构造解,那么我们有
\[
u(t + h, x) = E_x 
u(t, B_h)
\]
从两边减去 $
u(t, x)$ 并利用泰勒展开
\[
\begin{array}{rl}
& 
u(t + h, x) - 
u(t, x) = E_x \Bigg( \displaystyle \sum_{i = 1}^{d} \frac{\partial 
u}{\partial x_i}(t, x) (B_i(h) - x_i) \\
& \qquad + \displaystyle \frac{1}{2} \sum_{1 \leq i, j \leq d} \frac{\partial^2 
u}{\partial x_i \partial x_j}(t, x) (B_i(h) - x_i) (B_j(h) - x_j) \Bigg) + o(h)
\end{array}
\]

    \end{prf}
    
    
    
    \begin{dfn}
        [Definition-of-Functions-Related-to-Continued-Fractions]
        {连分数相关函数的定义}
        [Definition of Functions Related to Continued Fractions]
        [gpt-4.1]
        设 $\varphi(x) = 1/x - [1/x]$,其中 $x \in (0,1)$,$A(x) = [1/x]$,其中 $[1/x]$ 表示不超过 $1/x$ 的最大整数.
    \end{dfn}
    
    
    
    \begin{dfn}
        [Definition-of-Continued-Fraction-Expansion]
        {连分数展开的定义}
        [Definition of Continued Fraction Expansion]
        [gpt-4.1]
        设 $a_n = A(\varphi^n x)$,$n = 0, 1, 2, \ldots$,则 $x$ 的连分数表示为
\[
x = 1 / (a_0 + 1 / (a_1 + 1 / (a_2 + 1 / \ldots)))
\]

    \end{dfn}
    
    
    
    \begin{thm}
        [Expression-for-Bounded-Solution-of-Dirichlet-Problem]
        {有界解的狄利克雷问题解的表达式}
        [Expression for Bounded Solution of Dirichlet Problem]
        [gpt-4.1]
        
如果狄利克雷问题存在一个有界解,则它必定为

\[
u(x) \equiv E_{x} f(B_{\tau})
\]

    \end{thm}
    
    
    
    \begin{prf}
        [Proof-of-Theorem-9.5.2]
        {定理9.5.2的证明}
        [Proof of Theorem 9.5.2]
        [gpt-4.1]
        
由于 $u$ 有界,$M_{t}$ 是有界鞅,因此 $\lim_{t \to \infty} M_{t}$ 存在.
边界条件意味着 $M_{\infty} = f(B_{\tau})$.
由于鞅是一致可积的,$u(x) = E_{x} M_{\infty} = E_{x} f(B_{\tau})$.

    \end{prf}
    
    
    
    \begin{lma}
        [Maximal-Ergodic-Lemma]
        {极大遍历性引理}
        [Maximal Ergodic Lemma]
        [gpt-4.1]
        设 $X_j(\omega) = X(\varphi^j \omega)$,$S_k(\omega) = X_0(\omega) + \cdots + X_{k-1}(\omega)$,$M_k(\omega) = \max (0, S_1(\omega), \ldots, S_k(\omega))$.
则 $E(X ; M_k > 0) \geq 0$.
    \end{lma}
    
    
    
    \begin{dfn}
        [Definition-of-Additivity-and-Subadditivity-for-Set-Functions]
        {集合函数的可加性和次可加性定义}
        [Definition of Additivity and Subadditivity for Set Functions]
        [gpt-4.1]
        
设 $\boldsymbol{\mathcal{S}}$ 是一个集合族,$\mu$ 是定义在 $\boldsymbol{\mathcal{S}}$ 上的集合函数.若满足以下两个条件:
(i) 如果 $S \in \boldsymbol{\mathcal{S}}$ 是有限多个互不相交集合 $S_i \in \boldsymbol{\mathcal{S}}$ 的并,则
\[
\mu(S) = \sum_{i} \mu(S_i)
\]
(ii) 如果 $S_i, S \in \boldsymbol{\mathcal{S}}$ 且 $S = \bigcup_{i \geq 1} S_i$,则
\[
\mu(S) \leq \sum_{i} \mu(S_i)
\]
则称 $\mu$ 具有有限可加性和可数次可加性(或次可加性).

    \end{dfn}
    
    
    
    \begin{thm}
        [Theorem-of-Countable-Additivity]
        {可数可加性的定理}
        [Theorem of Countable Additivity]
        [gpt-4.1]
        
若 $E = \bigcup _ { i } E _ { i }$, 其中 $E _ { 1 }, E _ { 2 }, \dots$ 是互不相交且可测的集合,则有
\[
\mu ^ { * } ( E ) = \sum _ { i = 1 } ^ { \infty } \mu ^ { * } ( E _ { i } )
\]

    \end{thm}
    
    
    
    \begin{prf}
        [Proof-of-Countable-Additivity]
        {可数可加性的证明}
        [Proof of Countable Additivity]
        [gpt-4.1]
        
设 $F _ { n } = E _ { 1 } \cup \cdots \cup E _ { n }$.由单调性和性质 (c),有
\[
\mu ^ { * } ( E ) \geq \mu ^ { * } ( F _ { n } ) = \sum _ { i = 1 } ^ { n } \mu ^ { * } ( E _ { i } )
\]
令 $n \to \infty$,并利用次可加性,即得所需结论.

    \end{prf}
    
    
    
    \begin{dfn}
        [Definition-of-Congruence-of-Subsets-in-Euclidean-Space]
        {欧几里得空间中子集的全等性定义}
        [Definition of Congruence of Subsets in Euclidean Space]
        [gpt-4.1]
        在欧几里得几何中,若两个 $\mathbf{R}^{d}$ 的子集可以通过平移和旋转将一个子集映射到另一个子集,则称这两个子集是全等的.
    \end{dfn}
    
    
    
    \begin{ppt}
        [Congruent-Measurable-Sets-Have-Equal-Lebesgue-Measure]
        {全等可测集勒贝格测度相等}
        [Congruent Measurable Sets Have Equal Lebesgue Measure]
        [gpt-4.1]
        两个全等的可测集具有相同的勒贝格测度.
    \end{ppt}
    
    
    
    \begin{thm}
        [Theorem-on-the-Representation-of-Bounded-Solutions]
        {有界解的表示形式定理}
        [Theorem on the Representation of Bounded Solutions]
        [gpt-4.1]
        
设 $G$ 是 $d \geq 3$ 的连通开集,所有边界点都是正则点,且在 $G$ 内有 $P_x(\tau < \infty) < 1$,其中 $\tau = \operatorname{inf}\{ t : B_t 
otin G \}$.若边界条件 $f \equiv 0$,则所有有界解都具有形式 $u(x) = c P_x(\tau = \infty)$,其中 $c$ 为常数.

    \end{thm}
    
    
    
    \begin{thm}
        [Conditions-for-Differentiability-of-a-Summation-Function]
        {求和函数可微性的条件}
        [Conditions for Differentiability of a Summation Function]
        [gpt-4.1]
        
设 $\delta > 0$.假设对于 $x \in (y - \delta, y + \delta)$,满足下列条件:
(i) $u(x) = \sum_{n=1}^{\infty} f_{n}(x)$ 且 $\sum_{n=1}^{\infty} |f_{n}(x)| < \infty$;
(ii) 对于每个 $n$,$f_{n}^{\prime}(x)$ 存在且是 $x$ 的连续函数;
(iii) $\sum_{n=1}^{\infty} \sup_{\theta \in ( -\delta, \delta )} |f_{n}^{\prime}(y + \theta)| < \infty$.

则 $u^{\prime}(x) = 
u(x)$.

    \end{thm}
    
    
    
    \begin{dfn}
        [Definition-of-Extended-Measure]
        {扩展测度的定义}
        [Definition of Extended Measure]
        [gpt-4.1]
        我们在 $\bar{\mathcal{S}}$ 上定义 $\bar{\mu}$,当 $A = \bigcup_{i} S_{i}$ 时,
\[
\bar{\mu}(A) = \sum_{i} \mu(S_{i})
\]
其中每个 $S_i \in \mathcal{S}$.
    \end{dfn}
    
    
    
    \begin{thm}
        [Convergence-of-Normal-Vectors-at-Regular-Boundary-Points]
        {边界正则点处法向量收敛性定理}
        [Convergence of Normal Vectors at Regular Boundary Points]
        [gpt-4.1]
        
设 $G$ 和 $g$ 有界.令 $y$ 为 $\partial G$ 的一个正则点.如果 $x_n \in G$ 且 $x_n \to y$,则 $
u(x_n) \to 0$.

    \end{thm}
    
    
    
    \begin{dfn}
        [Basic-Setup-of-Probability-Space]
        {概率空间的基本设置}
        [Basic Setup of Probability Space]
        [gpt-4.1]
        我们基本的设置包括

\[
\begin{array}{rl}
(\Omega, \mathcal{F}, P) & \text{一个概率空间} \\
\varphi & \text{一个保持 } P \text{ 的映射} \\
X_{n}(\omega) = X(\varphi^{n} \omega) & \text{其中 } X \text{ 是随机变量}
\end{array}
\]

    \end{dfn}
    
    
    
    \begin{dfn}
        [Definition-of-Invariant-Set]
        {不变集的定义}
        [Definition of Invariant Set]
        [gpt-4.1]
        若 $A \in \mathcal{F}$ 满足 $\varphi^{-1}A = A$,则称 $A$ 是不变集.
    \end{dfn}
    
    
    
    \begin{thm}
        [Theorem-on-$ut---s-B-s$-Being-a-Local-Martingale]
        {关于$u(t - s, B\_s)$是局部鞅的定理}
        [Theorem on $u(t - s, B_s)$ Being a Local Martingale]
        [gpt-4.1]
        如果 $u$ 满足条件 (a),则 $M_s = u(t - s, B_s)$ 在区间 $[0, t]$ 上是一个局部鞅.
    \end{thm}
    
    
    
    \begin{prf}
        [Proof-of-Theorem-on-$ut---s-B-s$-Being-a-Local-Martingale]
        {关于$u(t - s, B\_s)$是局部鞅的定理的证明}
        [Proof of Theorem on $u(t - s, B_s)$ Being a Local Martingale]
        [gpt-4.1]
        应用伊藤公式,利用 (7.6.9) 得到

\[
\begin{array}{l}
\displaystyle u(t - s, B_s) - u(t, B_0) = \int_{0}^{s} -u_t(t - r, B_r) dr \\
\displaystyle \qquad + \int_{0}^{s} 
abla u(t - r, B_r) \cdot dB_r + \frac{1}{2} \int_{0}^{s} \Delta u(t - r, B_r) dr
\end{array}
\]

由于 $-u_t + \frac{1}{2} \Delta u = 0$,且右侧第二项由定理 7.6.4 可知是局部鞅,因此结论成立.这里需要停时以确保可积性条件成立.
    \end{prf}
    
    
    
    \begin{ppt}
        [Property-of-Conditional-Expectation-and-Cauchy-Schwarz-Inequality]
        {条件期望与柯西-施瓦茨不等式的性质}
        [Property of Conditional Expectation and Cauchy-Schwarz Inequality]
        [gpt-4.1]
        若 $Y \in \mathcal{F}$ 且 $\| Y \|_{2} = 1$,则有
\[
E X Y = E ( E ( X Y | \mathcal{F} ) ) = E ( Y E ( X | \mathcal{F} ) ) \leq \| E ( X | \mathcal{F} ) \|_{2} \| Y \|_{2}
\]
当 $Y = E ( X | \mathcal{F} ) / \| E ( X | \mathcal{F} ) \|_{2}$ 时等号成立.

    \end{ppt}
    
    
    
    \begin{thm}
        [Banach-Tarski-Theorem]
        {Banach-Tarski定理}
        [Banach-Tarski Theorem]
        [gpt-4.1]
        
Banach 和 Tarski(1924)利用选择公理证明,在 $\mathbf{R}^{3}$ 中,可以将球 $\{ x : |x| \leq 1 \}$ 分割成有限个集合 $A_{1}, \ldots, A_{n}$,并找到全等的集合 $B_{1}, \ldots, B_{n}$,使得这些集合的并集是两个互不相交的半径为 1 的球!由于全等集合具有相同的Lebesgue测度,至少有一个集合 $A_{i}$ 必须是不可测的.

    \end{thm}
    
    
    
    \begin{thm}
        [Martingale-Property-and-$L^2$-Boundedness-of-Transformed-Brownian-Bridge]
        {Brownian桥变换的鞅性及$L^2$有界性}
        [Martingale Property and $L^2$-Boundedness of Transformed Brownian Bridge]
        [gpt-4.1]
        
设 $B_t^0$ 是布朗桥(Brownian bridge),定义过程 $X_t = B_t^0 / (1-t)$.则$X_t$是一个鞅,但不是$L^2$有界的.

    \end{thm}
    
    
    
    \begin{lma}
        [Lemma-on-Lower-Bound-of-Probability-under-Limit]
        {极限下概率下界的引理}
        [Lemma on Lower Bound of Probability under Limit]
        [gpt-4.1]
        
设 $y$ 是 $G$ 的正则点且 $t > 0$,则 $P_{y}(\tau \leq t) = 1$.由引理 9.5.8 可知,如果 $x_{n} \to y$,那么
\[
\lim_{n \to \infty} \inf P_{x_{n}}(\tau \leq t) \geq 1 \quad \text{对任意 } t > 0
\]

    \end{lma}
    
    
    
    \begin{thm}
        [Smoothness_of_$
    u$_and_Satisfaction_of_Condition_a]
        {函数 $
    u$ 的光滑性及满足条件 (a)}
        [Smoothness of $
    u$ and Satisfaction of Condition (a)]
        [gpt-4.1]
        如果
\[
u(t, x) = E_{x} f(B_{t}) = \int (2\pi t)^{-d/2} e^{-|x-y|^{2}/2t} f(y)\, dy,
\]
则 $
u \in C^{1,2}$,因此满足条件 (a).

    \end{thm}
    
    
    
    \begin{prf}
        [Proof_of_Smoothness_of_$
    u$]
        {关于 $
    u$ 光滑性的证明}
        [Proof of Smoothness of $
    u$]
        [gpt-4.1]
        我们的解为
\[
u(t, x) = E_{x} f(B_{t}) = \int (2\pi t)^{-d/2} e^{-|x-y|^{2}/2t} f(y)\, dy
\]
利用微积分和分析中的一些基本事实可以证明所需结论.

    \end{prf}
    
    
    
    \begin{thm}
        [Limit-Distribution-Theorem-for-$X-n$]
        {关于 $X\_n$ 的极限分布定理}
        [Limit Distribution Theorem for $X_n$]
        [gpt-4.1]
        
设 $\xi_1, \xi_2, \ldots$ 是独立同分布的随机变量,满足 $P(\xi_m = k) = a_{k+1}$,其中 $k \geq -1$,并设 $S_n = x + \xi_1 + \cdots + \xi_n$,$x \geq 0$.定义
\[
X_n = S_n + \left( \operatorname*{min}_{m \leq n} S_m \right)^-
\]
若 $\mu = \sum k a_k < 1$,则当 $n \to \infty$ 时,
\[
\frac{1}{n} \left| \{ m \leq n : X_{m-1} = 0 , \xi_m = -1 \} \right| \to (1 - \mu) \quad \mathrm{a.s.}
\]
并且 $\pi(0) = (1-\mu)/a_0$.

    \end{thm}
    
    
    
    \begin{thm}
        [Constancy-of-Continuous-Superharmonic-Functions-in-Dimensions-1-and-2]
        {一维与二维中连续上超调和函数的常值性}
        [Constancy of Continuous Superharmonic Functions in Dimensions 1 and 2]
        [gpt-4.1]
        设 $f \in C^2$ 是一个连续的上超调和函数,取值于 $[0, \infty]$.则在维数 $1$ 和 $2$ 的情况下,$f$ 必为常数函数.
    \end{thm}
    
    
    
    \begin{xmp}
        [Example-of-Radial-Superharmonic-Function-in-Higher-Dimensions]
        {高维中径向超调和函数的例子}
        [Example of Radial Superharmonic Function in Higher Dimensions]
        [gpt-4.1]
        在维数 $d \geq 3$ 时,存在形如 $f(x) = g(|x|)$ 的上超调和函数.
    \end{xmp}
    
    
    
    \begin{thm}
        [A-Theorem-on-the-Partial-Differential-Equation-of-Harmonic-Function-$h-\theta$]
        {关于调和函数 $h\_\theta$ 的偏微分方程的一个定理}
        [A Theorem on the Partial Differential Equation of Harmonic Function $h_\theta$]
        [gpt-4.1]
        
\[
\begin{array}{r}
{\displaystyle \sum_{i=1}^{d-1} D_{x_{i}x_{i}} h_{\theta} + D_{yy} h_{\theta} = \frac{ \{ (-d)(d-1) + (-d) \cdot 3 \} y }{ (|x-\theta|^{2} + y^{2})^{(d+2)/2} } } \\
{ + \frac{ d(d+2) y (|x-\theta|^{2} + y^{2}) }{ (|x-\theta|^{2} + y^{2})^{(d+4)/2} } = 0 }
\end{array}
\]

    \end{thm}
    
    
    
    \begin{xmp}
        [An-Example-on-the-Derivative-of-the-Moment-Generating-Function]
        {关于矩母函数导数的一个例子}
        [An Example on the Derivative of the Moment Generating Function]
        [gpt-4.1]
        
如果 $\phi(\theta) = E e^{\theta Z} < \infty$ 对于 $\theta \in [ -\epsilon, \epsilon ]$ 成立,则有 $\phi^{\prime}(0) = E Z$.

    \end{xmp}
    
    
    
    \begin{prf}
        [Proof-that-the-Derivative-of-the-MGF-Equals-the-Expectation]
        {矩母函数导数等于期望的证明}
        [Proof that the Derivative of the MGF Equals the Expectation]
        [gpt-4.1]
        
这里 $\theta$ 起到 $x$ 的作用,我们取 $\mu$ 为 $Z$ 的分布.

    \end{prf}
    
    
    
    \begin{dfn}
        [Definition-of-Infinite-Dimensional-Real-Space-and-Its-Product-σ-Algebra]
        {无限维实空间及其乘积σ-代数的定义}
        [Definition of Infinite Dimensional Real Space and Its Product σ-Algebra]
        [gpt-4.1]
        
设 $\mathbf{N} = \{ 1, 2, \ldots \}$,定义
\[
\mathbf{R}^{\mathbf{N}} = \{ (\omega_{1}, \omega_{2}, \ldots) : \omega_{i} \in \mathbf{R} \}
\]
并赋予 $\mathbf{R}^{\mathbf{N}}$ 以乘积 $\sigma$-代数 $\mathcal{R}^{\mathbf{N}}$,该代数由有限维矩形生成,即所有形如 $\{ \omega : \omega_{i} \in (a_{i}, b_{i}] \text{ for } i = 1, \ldots, n \}$ 的集合,其中 $-\infty \leq a_{i} < b_{i} \leq \infty$.

    \end{dfn}
    
    
    
    \begin{thm}
        [Integral-Representation-of-the-Stochastic-Process-$ut-s-B-s$]
        {关于随机过程 $u(t-s, B\_s)$ 的积分表示}
        [Integral Representation of the Stochastic Process $u(t-s, B_s)$]
        [gpt-4.1]
        
设 $u$ 是满足相应条件的函数,$B_t$ 是布朗运动,则有
\[
u(t-s, B_{s}) - u(t, B_{0}) = \int_{0}^{s} \left(-u_{t} + \frac{1}{2} \Delta u\right)(t-r, B_{r})\, dr 
+ \int_{0}^{s} 
abla u(t-r, B_{r}) \cdot dB_{r}
\]
其中 $-u_{t} + \frac{1}{2} \Delta u = -g$,且右侧第二项根据定理7是局部鞅.

    \end{thm}
    
    
    
    \begin{thm}
        [Application-of-Lindeberg-Feller-Theorem-to-Normalized-Limit-Distribution]
        {Lindeberg-Feller定理应用于归一化极限分布}
        [Application of Lindeberg-Feller Theorem to Normalized Limit Distribution]
        [gpt-4.1]
        设 $\xi_{i}$ 是取值为 $1$ 和 $-1$ 且概率相等的随机变量,令 $c_{k} = (1 + k)^{-p}$,其中 $1/2 < p < 1$.则有归一化极限分布
\[
S_{n} / n^{3/2 - p} \Rightarrow \sigma \chi
\]
其中 $S_n$ 是相关和,$\sigma$ 为参数,$\chi$ 为极限分布.
    \end{thm}
    
    
    
    \begin{thm}
        [Expression-of-Greens-Function-in-the-Ball-Domain]
        {Green函数在球域内的表达式}
        [Expression of Green's Function in the Ball Domain]
        [gpt-4.1]
        
在 $d \geq 3$ 的情形下,若 $0 < |y| < 1$,则有
\[
G_D(x, y) = G(x, y) - |y|^{2-d} G(x, y / |y|^2)
\]

    \end{thm}
    
    
    
    \begin{prf}
        [Proof-of-Expression-of-Greens-Function]
        {Green函数表达式的证明}
        [Proof of Expression of Green's Function]
        [gpt-4.1]
        
根据 (9.7.1),只需证明:(i) 第二项在 $D$ 上是调和函数;(ii) 在边界上等于 $G(x, y)$.

    \end{prf}
    
    
    
    \begin{thm}
        [Blumenthals-0-1-Law]
        {Blumenthal 0-1 定律}
        [Blumenthal's 0-1 Law]
        [gpt-4.1]
        
若 $A \in \mathcal{F}_{0}^{+}$,则对所有 $\boldsymbol{x} \in \mathbf{R}^{d}$,
\[
P_{x}(A) \in \{ 0 , 1 \}.
\]

    \end{thm}
    
    
    
    \begin{prf}
        [Proof-of-Blumenthals-0-1-Law]
        {Blumenthal 0-1 定律的证明}
        [Proof of Blumenthal's 0-1 Law]
        [gpt-4.1]
        
利用 $A \in \mathcal{F}_{0}^{+}$,定理 7.2.2,以及 $\mathcal{F}_{0}^{o} = \sigma(B_{0})$ 在 $P_{x}$ 下是平凡的,得到
\[
1_{A} = E_{x}(1_{A} | \mathcal{F}_{0}^{+}) = E_{x}(1_{A} | \mathcal{F}_{0}^{o}) = P_{x}(A) \quad P_{x}\text{ a.s.}
\]
这表明指示函数 $1_{A}$ 几乎处处等于数 $P_{x}(A)$,结论得证.

    \end{prf}
    
    
    
    \begin{thm}
        [Uniqueness-Theorem-of-Solution]
        {解的唯一性定理}
        [Uniqueness Theorem of Solution]
        [gpt-4.1]
        如果在 $[0, T] \times \mathbf{R}^{d}$ 上对于任意 $T < \infty$ 存在有界解, 则该解必为
\[
u ( t , x ) \equiv E _ { x } \{ f ( B _ { t } ) \exp ( c _ { t } ) \}
\]
在我们的关于 $c$ 和 $u$ 的假设下, $M_s$ ($0 \leq s < t$) 是有界鞅, 且 $M_t \equiv \lim_{s \to t} M_s = f ( B_t ) \exp ( c_t )$.由于 $M_s$ 一致可积, 推得
\[
u ( t , x ) = E_x M_0 = E_x M_t = 
u ( t , x )
\]
从而证明了所需结果.

    \end{thm}
    
    
    
    \begin{lma}
        [Norm-Identity-on-the-Unit-Sphere-Lemma]
        {单位球面上范数等式引理}
        [Norm Identity on the Unit Sphere Lemma]
        [gpt-4.1]
        
若 $|x| = 1$, 则有 $|\,|x|y| - y|y|^{-1}| = |x - y|$.

    \end{lma}
    
    
    
    \begin{thm}
        [Expression-of-$G-Hx-y$-on-an-Interval]
        {区间上 $G\_H(x, y)$ 的表达式}
        [Expression of $G_H(x, y)$ on an Interval]
        [gpt-4.1]
        对于所有 $x, y > 0$,有
\[
G_H(x, y) = 2(x \wedge y)
\]
并且对于 $y \geq x$,$y G_H(x, y)$ 是常数 $= 2x$,即所有点 $y > x$ 都具有相同的期望占据时间.
    \end{thm}
    
    
    
    \begin{prf}
        [Proof-of-Formula-for-$G-Hx-y$]
        {$G\_H(x, y)$ 的公式化证明}
        [Proof of Formula for $G_H(x, y)$]
        [gpt-4.1]
        令 $\cal G(x, y) = -|x - y|$,因此
\[
G_H(x, y) = -|x - y| + |x + y|
\]
分情况讨论,对于 $0 < y < x$,
\[
G_H(x, y) = - (x - y) + (x + y) = 2y
\]
这就给出了所需的公式.
    \end{prf}
    
    
    
    \begin{thm}
        [Theorem_on_$
    u$_Satisfying_Condition_b]
        {关于 $
    u$ 满足条件 (b) 的定理}
        [Theorem on $
    u$ Satisfying Condition (b)]
        [gpt-4.1]
        
定理 9.2.4 $
u$ 满足 (b).

    \end{thm}
    
    
    
    \begin{prf}
        [Proof-of-Theorem-9.2.4]
        {定理 9.2.4 的证明}
        [Proof of Theorem 9.2.4]
        [gpt-4.1]
        
$(B_{t} - B_{0}) \stackrel{d}{=} t^{1/2} \chi$, 其中 $\chi$ 服从均值为 0、方差为 1 的正态分布,因此若 $t_{n} \to 0$ 且 $x_{n} \to x$,由有界收敛定理可知

\[
u(t_{n}, x_{n}) = E f(x_{n} + t_{n}^{1/2} \chi) \to f(x)
\]

这就证明了所需的结果.

    \end{prf}
    
    
    
    \begin{thm}
        [Property-of-$C^{12}$-Functions]
        {关于 $C^{1,2}$ 函数的性质}
        [Property of $C^{1,2}$ Functions]
        [gpt-4.1]
        如果 $
u \in C^{1,2}$,那么它在 $(0, \infty) \times \mathbf{R}^{d}$ 内满足 (a).
    \end{thm}
    
    
    
    \begin{prf}
        [Proof-of-Property-of-$C^{12}$-Functions]
        {关于 $C^{1,2}$ 函数的性质的证明}
        [Proof of Property of $C^{1,2}$ Functions]
        [gpt-4.1]
        如果我们考虑通过运行时空布朗运动 $(t-s, B_{s})$ 来构造解,那么我们可以得到上述结论.
    \end{prf}
    
    
    
    \begin{prf}
        [Proof-of-Measure-Inequality]
        {关于测度不等式的证明}
        [Proof of Measure Inequality]
        [gpt-4.1]
        
设有以下不等式:
\[
\bar{
u}(\alpha) = \int \bar{
u}(dx)\, \bar{p}^n(x, \alpha) \geq \bar{
u}(\alpha) \int \bar{\mu}(dx)\, \bar{p}^n(x, \alpha) = \bar{
u}(\alpha)\, \bar{\mu}(\alpha) = \bar{
u}(\alpha)
\]

令 $S_n = \{ x : p^n(x, \alpha) > 0 \}$.由假设 $\cup_n S_n = S$.

若存在某个集合 $D$ 使得 $\bar{
u}(D) > \bar{
u}(\alpha)\, \bar{\mu}(D)$,则有 $\bar{
u}(D \cap S_n) > \bar{
u}(\alpha)\, \bar{\mu}(D \cap S_n)$,进而推出 $\bar{
u}(\alpha) > \bar{
u}(\alpha)$,矛盾.

    \end{prf}
    
    
    
    \begin{thm}
        [Hahn-Decomposition-Theorem]
        {Hahn分解定理}
        [Hahn Decomposition Theorem]
        [gpt-4.1]
        设 $\alpha$ 是一个符号测度(signed measure).则存在正集 $A$ 和负集 $B$,使得 $\Omega = A \cup B$ 且 $A \cap B = \emptyset$.
    \end{thm}
    
    
    
    \begin{prf}
        [Proof-of-Hahn-Decomposition-Theorem]
        {Hahn分解定理的证明}
        [Proof of Hahn Decomposition Theorem]
        [gpt-4.1]
        令 $c = \inf\{ \alpha(B) : B \text{ 为负集} \} \leq 0$.令 $B_i$ 为使得 $\alpha(B_i) \downarrow c$ 的负集.令 $B = \cup_i B_i$.由引理A.4.3,$B$是负集,因此根据$c$的定义,有$\alpha(B) \geq c$.为证明$\alpha(B) \leq c$,注意到$\alpha(B) = \alpha(B_i) + \alpha(B - B_i) \leq \alpha(B_i)$,因为$B$是负集,令$i \to \infty$.上述两个不等式说明$\alpha(B) = c$,且由符号测度的定义,$c > -\infty$.令 $A := B^c$.为证明$A$是正集,假设$A$包含一个使$\alpha(E) < 0$的集合,则由引理A.4.4,$A$包含一个负集$F$,且$\alpha(F) < 0$,但此时$B \cup F$为一个负集,且$\alpha(B \cup F) = \alpha(B) + \alpha(F) < c$,矛盾.
    \end{prf}
    
    
    
    \begin{lma}
        [Lemma-on-Probability-of-Decreasing-Sets-Tending-to-Zero]
        {极小集合概率趋于零的引理}
        [Lemma on Probability of Decreasing Sets Tending to Zero]
        [gpt-4.1]
        如果 $B_n \in \mathcal{A}$ 且 $B_n \downarrow \emptyset$,则 $P(B_n) \downarrow 0$.
    \end{lma}
    
    
    
    \begin{dfn}
        [Definition-of-Stationary-Sequence]
        {平稳序列的定义}
        [Definition of Stationary Sequence]
        [gpt-4.1]
        $X _ { 0 } , X _ { 1 } , \ldots$ 称为平稳序列,如果对任意 $k$,移位序列 $\{ X _ { k + n } , n \geq 0 \}$ 与原序列具有相同的分布,即对任意 $m$,$( X _ { 0 } , \ldots , X _ { m } )$ 和 $( X _ { k } , \ldots , X _ { k + m } )$ 具有相同的分布.
    \end{dfn}
    
    
    
    \begin{lma}
        [Bound-on-Covariance-under-Conditional-Independence]
        {条件独立情形下协方差的界}
        [Bound on Covariance under Conditional Independence]
        [gpt-4.1]
        设 $p, q, r \in (1, \infty]$ 且 $1/p + 1/q + 1/r = 1$,并且 $X \in \mathcal{G}, Y \in \mathcal{H}$ 满足 $E|X|^{p},\ E|Y|^{q} < \infty$.则有
\[
| E X Y - E X E Y | \leq 8 \| X \|_{p} \| Y \|_{q} ( \alpha ( \mathcal{G}, \mathcal{H} ) )^{1/r}
\]
其中,对 $x > 0$,$x^{0} = 1$,且 $0^{0} = 0$.
    \end{lma}
    
    
    
    \begin{dfn}
        [Definition-of-Three-Variables-for-a-Sequence-of-Random-Variables]
        {关于随机变量序列的三个变量的定义}
        [Definition of Three Variables for a Sequence of Random Variables]
        [gpt-4.1]
        
设 $(X_j)$ 为一组随机变量,$\mathcal{F}_{-1}$、$\mathcal{F}_0$ 为相关的滤子,定义如下三个变量:
\[
\begin{array}{l}
{\displaystyle Z_0 = \sum_{j=0}^\infty E(X_j \mid \mathcal{F}_{-1})} \\[2ex]
{\displaystyle \theta Z_0 = \sum_{j=0}^\infty E(X_{j+1} \mid \mathcal{F}_0)} \\[2ex]
{\displaystyle Y_0 = \sum_{j=0}^\infty \left\{ E(X_j \mid \mathcal{F}_0) - E(X_j \mid \mathcal{F}_{-1}) \right\} }
\end{array}
\]

    \end{dfn}
    
    
    
    \begin{thm}
        [Proof-of-Theorem-8.3]
        {定理 8.3 的证明}
        [Proof of Theorem 8.3]
        [gpt-4.1]
        
假设满足如下条件:
\[
\sum_{j=0}^\infty \| E(X_j \mid \mathcal{F}_{-1}) \|_2 < \infty
\]
则
\[
Y_0 + Z_0 - \theta Z_0 = E(X_0 \mid \mathcal{F}_0) = X_0
\]
其中 $Z_0$ 和 $Y_0$ 的级数在 $L^2$ 中收敛.

    \end{thm}
    
    
    
    \begin{ppt}
        [Brownian-Motion-in-Higher-Dimensions-Does-Not-Hit-Zero-at-Positive-Time]
        {高维布朗运动不会在正时刻到达原点}
        [Brownian Motion in Higher Dimensions Does Not Hit Zero at Positive Time]
        [gpt-4.1]
        
设 $d \geq 2$,则对于所有初始点 $x$,$d$ 维布朗运动满足 $P_x(S_0 < \infty) = 0$,其中 $S_0 = \inf\{ t > 0 : B_t = 0 \}$ 表示第一次在正时刻到达原点的时间.因此,即使布朗运动起点在原点,也不会在正时刻到达原点.

    \end{ppt}
    
    
    
    \begin{dfn}
        [Time-of-First-Hitting-Zero]
        {首次到达原点的时间}
        [Time of First Hitting Zero]
        [gpt-4.1]
        
$S_0 = \inf\{ t > 0 : B_t = 0 \}$,即 $S_0$ 表示布朗运动 $B_t$ 首次在正时刻到达原点的时间.

    \end{dfn}
    
    
    
    \begin{dfn}
        [Definition-of-GI/G/1-Queue]
        {GI/G/1排队系统的定义}
        [Definition of GI/G/1 Queue]
        [gpt-4.1]
        GI/G/1排队系统是指到达时间之间间隔为一般分布(GI),服务时间为一般分布(G),且仅有一个服务台(1)的排队模型.
    \end{dfn}
    
    
    
    \begin{dfn}
        [Definition-of-Recursive-Waiting-Time]
        {递归等待时间定义}
        [Definition of Recursive Waiting Time]
        [gpt-4.1]
        设 $\xi_1, \xi_2, \ldots$ 是一组随机变量,定义递归等待时间为 $W_n = (W_{n-1} + \xi_n)^+$,其中 $W_0=0$ 或其他初值,且 $S_n = \xi_1 + \cdots + \xi_n$.
    \end{dfn}
    
    
    
    \begin{dfn}
        [Definition-of-Net-Input]
        {净输入量的定义}
        [Definition of Net Input]
        [gpt-4.1]
        对于每一个 $\xi_n$,有 $\xi_n = \eta_{n-1} - \zeta_n$,其中 $\eta_{n-1}$ 表示第 $n-1$ 个顾客的服务时间,$\zeta_n$ 表示第 $n$ 个顾客的到达间隔时间.
    \end{dfn}
    
    
    
    \begin{thm}
        [Corollary-of-Donskers-Theorem-Limiting-Distribution-of-$D-n$]
        {Donsker定理的推论:$D\_n$的极限分布}
        [Corollary of Donsker's Theorem: Limiting Distribution of $D_n$]
        [gpt-4.1]
        
$D_{n} \Rightarrow \max_{0 \leq t \leq 1} | B_{t} - t B_{1} |$, 其中 $B_{t}$ 是以 $\mathbf{0}$ 为起点的布朗运动.

    \end{thm}
    
    
    
    \begin{lma}
        [Probability-Expression-for-the-Difference-of-Independent-Exponential-Variables]
        {独立指数分布变量差的概率表达式}
        [Probability Expression for the Difference of Independent Exponential Variables]
        [gpt-4.1]
        设 $X, Y \geq 0$ 独立,且 $P(X > x) = e^{-\lambda x}$,则
\[
P(X - Y > x) = a e^{-\lambda x},
\]
其中 $a = P(X - Y > 0)$.
    \end{lma}
    
    
    
    \begin{prf}
        [Proof-of-Probability-Expression-for-the-Difference-of-Independent-Exponential-Variables]
        {独立指数分布变量差的概率表达式的证明}
        [Proof of Probability Expression for the Difference of Independent Exponential Variables]
        [gpt-4.1]
        设 $F$ 为 $Y$ 的分布, 则
\[
\begin{array}{l}
P(X - Y > x) = \displaystyle \int_0^{\infty} dF(y)\, e^{-\lambda (x + y)} \\
= e^{-\lambda x} \displaystyle \int_0^{\infty} dF(y)\, e^{-\lambda y} = e^{-\lambda x} P(X - Y > 0)
\end{array}
\]
其中最后一个等式是令 $x = 0$ 得到的.
    \end{prf}
    
    
    
    \begin{dfn}
        [Definition-of-Mutually-Singular-Measures]
        {互为奇异测度的定义}
        [Definition of Mutually Singular Measures]
        [gpt-4.1]
        若存在集合 $A$ 使得 $\mu_{1}(A) = 0$ 且 $\mu_{2}(A^{c}) = 0$,则称两个测度 $\mu_{1}$ 和 $\mu_{2}$ 互为奇异(mutually singular).在此情形下,也称 $\mu_{1}$ 关于 $\mu_{2}$ 是奇异的,记作 $\mu_{1} \perp \mu_{2}$.
    \end{dfn}
    
    
    
    \begin{ppt}
        [Monotonicity-of-Outer-Measure]
        {外测度的单调性}
        [Monotonicity of Outer Measure]
        [gpt-4.1]
        如果 $E \subset F$,则 $\mu^*(E) \leq \mu^*(F)$.
    \end{ppt}
    
    
    
    \begin{ppt}
        [Countable-Subadditivity-of-Outer-Measure]
        {外测度的可数次可加性}
        [Countable Subadditivity of Outer Measure]
        [gpt-4.1]
        如果 $F \subset \cup_{i} F_{i}$,其中为可数并,则 $\mu^*(F) \leq \sum_{i} \mu^*(F_{i})$.
    \end{ppt}
    
    
    
    \begin{dfn}
        [Definition-of-Outer-Measure]
        {外测度的定义}
        [Definition of Outer Measure]
        [gpt-4.1]
        任何集合函数 $\mu^*$,满足 $\mu^*(\varnothing) = 0$ 且具有上述单调性和可数次可加性性质,称为外测度.
    \end{dfn}
    
    
    
    \begin{ppt}
        [Property-of-Null-Set-Intersection-in-Hahn-Decomposition]
        {Hahn 分解交集为零测集的性质}
        [Property of Null Set Intersection in Hahn Decomposition]
        [gpt-4.1]
        
设 $\Omega = A_{1} \cup B_{1} = A_{2} \cup B_{2}$ 是 $\Omega$ 上的两个 Hahn 分解,则 $A_{2} \cap B_{1}$ 是正集也是负集,因此 $A_{2} \cap B_{1}$ 是零测集:它的所有子集的测度均为 0.同理,$A_{1} \cap B_{2}$ 也是零测集.

    \end{ppt}
    
    
    
    \begin{thm}
        [Uniqueness-and-Representation-of-Bounded-Solution]
        {有界解的唯一性及表达式}
        [Uniqueness and Representation of Bounded Solution]
        [gpt-4.1]
        
如果存在一个有界解, 那么它必满足
\[
u(t, x) \equiv E_x f(B_t)
\]
其中 $\equiv$ 表示该等式定义了 $
u$.

    \end{thm}
    
    
    
    \begin{prf}
        [Proof-of-Representation-for-Bounded-Solution]
        {有界解表达式的证明}
        [Proof of Representation for Bounded Solution]
        [gpt-4.1]
        
假设 $u$ 是有界解, 则 $M_s, 0 \leq s < t$, 是有界鞅.
鞅收敛定理说明
\[
M_t \equiv \lim_{s \uparrow t} M_s \text{ 存在 a.s.}
\]
如果 $u$ 满足 (b), 该极限必为 $f(B_t)$.
由于 $M_s$ 是一致可积的, 得到 $u(t, x) = E_x M_0 = E_x M_t = 
u(t, x)$.

    \end{prf}
    
    
    
    \begin{thm}
        [Solovays-Theorem-on-the-Existence-of-Nonmeasurable-Sets]
        {Solovay 关于不可测集存在性的定理}
        [Solovay's Theorem on the Existence of Nonmeasurable Sets]
        [gpt-4.1]
        
Solovay (1970) 证明了以下命题:在不使用选择公理的情况下,无法在 Zermelo-Frankel 集合论 (ZF) 中证明存在非 Lebesgue 可测集.

    \end{thm}
    
    
    
    \begin{dfn}
        [Definition-of-Transition-Density-for-Brownian-Motion]
        {布朗运动的转移密度的定义}
        [Definition of Transition Density for Brownian Motion]
        [gpt-4.1]
        $p_t(x, y) = (2\pi t)^{-d/2} e^{-|x-y|^2/2t}$ 是布朗运动的转移密度.
    \end{dfn}
    
    
    
    \begin{dfn}
        [Definition-of-Gamma-Function]
        {Gamma函数的定义}
        [Definition of Gamma Function]
        [gpt-4.1]
        $\Gamma(\alpha) = \int_{0}^{\infty} s^{\alpha-1} e^{-s}\,ds$ 是通常的Gamma函数.
    \end{dfn}
    
    
    
    \begin{dfn}
        [Definition-of-Greens-Function]
        {Green函数的定义}
        [Definition of Green's Function]
        [gpt-4.1]
        $G(x, y) = \int_{0}^{\infty} p_t(x, y)\,dt$
    \end{dfn}
    
    
    
    \begin{thm}
        [Relation-Between-Greens-Formula-and-Brownian-Motion-Integral]
        {Green公式与布朗运动积分的关系}
        [Relation Between Green's Formula and Brownian Motion Integral]
        [gpt-4.1]
        在 $d \geq 3$ 时,若 $f$ 为非负函数,则对 $x 
eq y$ 有 $G(x, y) < \infty$,并且
\[
E_x \int_{0}^{\infty} f(B_t)\,dt = \int G(x, y) f(y)\,dy
\]
    \end{thm}
    
    
    
    \begin{thm}
        [Formula-for-the-Probability-that-Brownian-Motion-Hits-Zero-in-an-Interval]
        {布朗运动在区间内取零概率的公式}
        [Formula for the Probability that Brownian Motion Hits Zero in an Interval]
        [gpt-4.1]
        设 $A_{s, t}$ 表示布朗运动在 $[s, t]$ 区间内至少取到一次零的事件,则有
\[
P_{0}( A_{s, t} ) = \frac{2}{\pi} \arccos \left( \sqrt{\frac{s}{t}} \right).
\]

    \end{thm}
    
    
    
    \begin{dfn}
        [Definition-of-Poissons-Equation]
        {Poisson方程的定义}
        [Definition of Poisson's Equation]
        [gpt-4.1]
        
在有界开集 $G$ 中,若 $u \in C^2$ 并且在 $G$ 内满足
\[{\scriptstyle \frac{1}{2}} \Delta u = -g\]
其中 $g$ 为有界且连续的函数,并且在 $\partial G$ 上每一点 $u$ 连续且 $u = 0$,则称 $u$ 满足Poisson方程.

    \end{dfn}
    
    
    
    \begin{dfn}
        [Definition-of-Lim-Sup]
        {上极限的定义}
        [Definition of Lim Sup]
        [gpt-4.1]
        设 $\bar{X} = \limsup S_{n}/n$,其中 $S_n$ 为一列数列.
    \end{dfn}
    
    
    
    \begin{dfn}
        [Definition-of-Set-D]
        {集合 D 的定义}
        [Definition of Set D]
        [gpt-4.1]
        设 $\epsilon > 0$,定义 $D = \{\omega : \bar{X}(\omega) > \epsilon\}$.
    \end{dfn}
    
    
    
    \begin{thm}
        [Probability-of-Set-D-is-Zero]
        {集合 D 的概率为零}
        [Probability of Set D is Zero]
        [gpt-4.1]
        我们的目标是证明 $P(D) = 0$.
    \end{thm}
    
    
    
    \begin{ppt}
        [Invariance-of-Lim-Sup-under-Transformation]
        {上极限在变换下的不变性}
        [Invariance of Lim Sup under Transformation]
        [gpt-4.1]
        $\bar{X}(\varphi \omega) = \bar{X}(\omega)$,因此 $D \in \mathcal{Z}$.
    \end{ppt}
    
    
    
    \begin{lma}
        [Lemma-on-Expectation-of-Product-and-Partitioned-Variables]
        {关于乘积期望与分段变量的引理}
        [Lemma on Expectation of Product and Partitioned Variables]
        [gpt-4.1]
        
设 $C = \alpha^{-1/q} \lVert Y \rVert_{q}$, $Y_{1} = Y 1_{( | Y | \leq C )}$, $Y_{2} = Y - Y_{1}$,则有
\[
| E X Y - E X E Y | \leq | E X Y_{1} - E X E Y_{1} | + | E X Y_{2} - E X E Y_{2} | 
\leq 6 C \| X \|_{p} \alpha^{1 - 1/p} + 2 \| X \|_{p} \| Y_{2} \|_{\theta}
\]
其中 $\theta = ( 1 - 1/p )^{-1}$.

    \end{lma}
    
    
    
    \begin{thm}
        [Laplace-Transform-of-First-Exit-Time-of-Brownian-Motion-from-Interval]
        {关于布朗运动首出区间时间的拉普拉斯变换}
        [Laplace Transform of First Exit Time of Brownian Motion from Interval]
        [gpt-4.1]
        
设 $T = \operatorname*{inf} \{ t : B_{t} 
otin (-a, a) \}$,其中 $B_t$ 是起点为 $0$ 的布朗运动.则有
\[
E_{0} \exp ( -\lambda T ) = 1 / \cosh ( a \sqrt{2\lambda} ).
\]

    \end{thm}
    
    
    
    \begin{prf}
        [Proof-of-σ-algebra-Property-of-Extended-σ-algebra]
        {扩展σ-代数的σ-代数性证明}
        [Proof of σ-algebra Property of Extended σ-algebra]
        [gpt-4.1]
        第一步是验证 $\bar{\mathcal{F}}$ 是一个 $\sigma$-代数.
如果 $E_i = A_i \cup B_i$,其中 $A_i \in \mathcal{F}$,$B_i \subset N_i$ 且 $\mu(N_i) = 0$,则 $\cup_i A_i \in \mathcal{F}$,并且次可加性(subadditivity)说明 $\mu(\cup_i N_i) \leq \sum_i \mu(N_i) = 0$,所以 $\cup_i E_i \in \bar{\mathcal{F}}$.
对于补集,若 $E = A \cup B$ 且 $B \subset N$,则 $B^c \supset N^c$,所以...
    \end{prf}
    
    
    
    \begin{lma}
        [Properties-of-Measurable-Subsets-and-Positive-Sets]
        {关于可测集与正集的性质}
        [Properties of Measurable Subsets and Positive Sets]
        [gpt-4.1]
        
(i) 每个正集的可测子集也是正集.
(ii) 如果集合 $A_n$ 都是正集,则 $A = \bigcup_n A_n$ 也是正集.

    \end{lma}
    
    
    
    \begin{prf}
        [Proof-of-Properties-of-Measurable-Subsets-and-Positive-Sets]
        {关于可测集与正集性质的证明}
        [Proof of Properties of Measurable Subsets and Positive Sets]
        [gpt-4.1]
        
(i) 显然成立.

(ii) 证明如下:
观察到
\[
B_n = A_n \cap \left( \bigcap_{m=1}^{n-1} A_m^c \right) \subset A_n
\]
这些 $B_n$ 是正集,互不相交,并且 $\bigcup_n B_n = \bigcup_n A_n$.
令 $E \subset A$ 为可测集,令 $E_n = E \cap B_n$,由于 $B_n$ 是正集,所以 $\alpha(E_n) \geq 0$,于是
\[
\alpha(E) = \sum_n \alpha(E_n) \ge 0.
\]
证毕.

    \end{prf}
    
    
    
    \begin{crl}
        [Conclusions-for-Positive-Sets-Also-Hold-for-Negative-Sets]
        {正集结论对负集同样成立}
        [Conclusions for Positive Sets Also Hold for Negative Sets]
        [gpt-4.1]
        
引理 A.4.3 中的结论在将正集替换为负集时仍然成立.

    \end{crl}
    
    
    
    \begin{dfn}
        [Definition-of-Stopping-Time-in-Half-Space]
        {半空间上的停时定义}
        [Definition of Stopping Time in Half-Space]
        [gpt-4.1]
        
设 $H = \{ (x, y) : x \in \mathbf{R}^{d - 1}, y > 0 \}$,令 $\tau = \inf \{ t : B_{t} 
otin H \}$,其中 $B_t$ 表示布朗运动.

    \end{dfn}
    
    
    
    \begin{dfn}
        [Definition-of-Harmonic-Function-$h-{	heta}x-y$-in-Half-Space]
        {半空间上的调和函数 $h\_{	heta}(x, y)$ 的定义}
        [Definition of Harmonic Function $h_{	heta}(x, y)$ in Half-Space]
        [gpt-4.1]
        
对于 $\boldsymbol{\theta} \in \mathbf{R}^{d - 1}$,定义
\[
h_{\theta}(x, y) = \frac{C_{d} y}{(|x - \theta|^{2} + y^{2})^{d / 2}}
\]
其中常数 $C_{d}$ 被选取使得 $\int d\theta\, h_{\theta}(0, 1) = 1$.

    \end{dfn}
    
    
    
    \begin{thm}
        [$\pi$-$\lambda$-Theorem]
        {$\pi$-$\lambda$定理}
        [$\pi$-$\lambda$ Theorem]
        [gpt-4.1]
        
如果 $\mathcal{P}$ 是一个$\pi$-系统,$\mathcal{L}$ 是一个包含 $\mathcal{P}$ 的$\lambda$-系统,则有 $\sigma(\mathcal{P}) \subset \mathcal{L}$.

    \end{thm}
    
    
    
    \begin{prf}
        [Proof-of-the-$\pi$-$\lambda$-Theorem]
        {$\pi$-$\lambda$定理的证明}
        [Proof of the $\pi$-$\lambda$ Theorem]
        [gpt-4.1]
        
我们将证明:

(a) 若 $\ell(\mathcal{P})$ 是包含 $\mathcal{P}$ 的最小$\lambda$-系统,则 $\ell(\mathcal{P})$ 是一个$\sigma$-域.

由(a)可得所需结论.理由如下:由于$\sigma(\mathcal{P})$是包含$\mathcal{P}$的最小$\sigma$-域,而$\ell(\mathcal{P})$是包含$\mathcal{P}$的最小$\lambda$-系统,故有
\[
\sigma(\mathcal{P}) \subset \ell(\mathcal{P}) \subset \mathcal{L}
\]

证明(a)时注意到,若一个$\lambda$-系统对交运算封闭,则它是一个$\sigma$-域,因为
\[
\begin{array}{rl}
& \text{若 } A \in \mathcal{L} \text{ 则 } A^{c} = \Omega - A \in \mathcal{L} \\
& A \cup B = (A^{c} \cap B^{c})^{c} \\
& \bigcup_{i=1}^{n} A_{i} \uparrow \bigcup_{i=1}^{\infty} A_{i} \text{ 当 } n \uparrow \infty
\end{array}
\]

    \end{prf}
    
    
    
    \begin{dfn}
        [Definition-of-Transition-Probability]
        {转移概率的定义}
        [Definition of Transition Probability]
        [gpt-4.1]
        
定义一个转移概率 $
u$,其满足如下条件:

\[
u(x, \{x\}) = 1 \quad \mathrm{if} \quad x \in S \qquad 
u(\alpha, C) = \rho(C)
\]

即,$
u$ 在 $S$ 上保持质量不变,而在 $\alpha$ 处将质量返回到 $S$,并按照 $\rho$ 分布.

    \end{dfn}
    
    
    
    \begin{lma}
        [Equations-for-Transition-Probability]
        {关于转移概率的等式}
        [Equations for Transition Probability]
        [gpt-4.1]
        
引理 5.8.4:$
u \bar{p} = \bar{p}$ 且 $\bar{p} 
u = p$.

    \end{lma}
    
    
    
    \begin{dfn}
        [Definition-of-Brownian-Motion-Hitting-Time]
        {布朗运动 hitting time 的定义}
        [Definition of Brownian Motion Hitting Time]
        [gpt-4.1]
        设 $B_t^2$ 是二维布朗运动的第二个分量,定义 $T_y = \operatorname{inf}\{ t : B_t^2 = y \}$,即 $T_y$ 是 $B_t^2$ 首次达到 $y$ 的时刻.
    \end{dfn}
    
    
    
    \begin{ppt}
        [Independent-Increments-Property-at-Brownian-Hitting-Time]
        {布朗运动 hitting time 处的第一分量的独立增量性质}
        [Independent Increments Property at Brownian Hitting Time]
        [gpt-4.1]
        强马尔可夫性(strong Markov property)蕴含 $B^1(T_y)$ 具有独立增量.
    \end{ppt}
    
    
    
    \begin{ppt}
        [Distribution-Property-under-Brownian-Scaling-at-Hitting-Time]
        {布朗缩放下 hitting time 分布的性质}
        [Distribution Property under Brownian Scaling at Hitting Time]
        [gpt-4.1]
        布朗缩放(Brownian scaling)蕴含 $B^1(T_y) \stackrel{d}{=} y B^1(T_1)$,即 $B^1(T_y)$ 的分布与 $y$ 倍的 $B^1(T_1)$ 的分布相同.
    \end{ppt}
    
    
    
    \begin{crl}
        [Distribution-at-Hitting-Time-is-Symmetric-Stable-Law]
        {hitting time 处第一分量的分布为对称稳定律}
        [Distribution at Hitting Time is Symmetric Stable Law]
        [gpt-4.1]
        由于 $B^1(T_y)$ 的分布与 $-B^1(T_y)$ 相同,其分布必为参数 $\alpha = 1$ 的对称稳定律(symmetric stable law).
    \end{crl}
    
    
    
    \begin{prf}
        [Proof-on-Stationary-Measure-and-Measure-Inequality]
        {关于平稳测度与测度不等式的证明}
        [Proof on Stationary Measure and Measure Inequality]
        [gpt-4.1]
        
设 $
u$ 是一个平稳测度,$a \in S$.有
\[
u(z) = \sum_{y} 
u(y) p(y, z) = 
u(a) p(a, z) + \sum_{y 
e a} 
u(y) p(y, z)
\]
利用上式将右侧的 $
u(y)$ 替换,有
\[
\begin{array}{l}

u(z) = 
u(a) p(a, z) + \displaystyle \sum_{y 
eq a} 
u(a) p(a, y) p(y, z) \\
\qquad + \displaystyle \sum_{x 
eq a} \sum_{y 
eq a} 
u(x) p(x, y) p(y, z) \\
\qquad = 
u(a) P_{a}(X_{1} = z) + 
u(a) P_{a}(X_{1} 
eq a, X_{2} = z) \\
\qquad + P_{
u}(X_{0} 
eq a, X_{1} 
eq a, X_{2} = z)
\end{array}
\]
依次递推,有
\[
\begin{array}{c}

u(z) = 
u(a) \displaystyle \sum_{m=1}^{n} P_{a}(X_{k} 
eq a, 1 \leq k < m, X_{m} = z) \\
+ P_{
u}(X_{j} 
eq a, 0 \leq j < n, X_{n} = z)
\end{array}
\]
最后一项 $\geq 0$.

    \end{prf}
    
    
    
    \begin{thm}
        [Relationship-between-Stationary-Distribution-and-Mean-Return-Time-for-Irreducible-Markov-Chains]
        {不可约马尔可夫链的驻留分布与平均返返时间关系}
        [Relationship between Stationary Distribution and Mean Return Time for Irreducible Markov Chains]
        [gpt-4.1]
        
定理 5.5.11 若$p$是不可约的并且有驻留分布$\pi$,则
\[
\pi(x) = \frac{1}{E_x T_x}
\]
其中$E_x T_x$是从状态$x$出发首次回到$x$的期望时间.

    \end{thm}
    
    
    
    \begin{thm}
        [Kolmogorov-Extension-Theorem]
        {Kolmogorov扩展定理}
        [Kolmogorov Extension Theorem]
        [gpt-4.1]
        设 $Y_{0}, Y_{1}, \ldots$ 是取值于良好空间的平稳序列,则Kolmogorov扩展定理(定理A.3.1)允许我们在序列空间 $(S^{\{0, 1, \ldots\}}, S^{\{0, 1, \ldots\}})$ 上构造一个测度 $P$,使得序列 $X_{n}(\omega) = \omega_{n}$ 与 $\{Y_{n}, n \geq 0\}$ 有相同的分布.
    \end{thm}
    
    
    
    \begin{dfn}
        [Definition-of-Shift-Operator]
        {移位算子的定义}
        [Definition of Shift Operator]
        [gpt-4.1]
        设$\varphi$为移位算子,即$\varphi(\omega_{0}, \omega_{1}, \ldots) = (\omega_{1}, \omega_{2}, \ldots)$,且$X(\omega) = \omega_{0}$,则$\varphi$是测度保持的,且$X_{n}(\omega) = X(\varphi^{n} \omega)$.
    \end{dfn}
    
    
    
    \begin{lma}
        [Lemma-on-Distribution-Equivalence-of-Random-Variable-Sets]
        {随机变量集合分布等价的引理}
        [Lemma on Distribution Equivalence of Random Variable Sets]
        [gpt-4.1]
        
$\{ U_k^n : 1 \leq k \leq n \} \stackrel{d}{=} \{ Z_k / Z_{n+1} : 1 \leq k \leq n \}$

    \end{lma}
    
    
    
    \begin{thm}
        [Theorem-on-$M-t$-Being-a-Local-Martingale]
        {关于$M\_t$为局部鞅的定理}
        [Theorem on $M_t$ Being a Local Martingale]
        [gpt-4.1]
        
设 $\tau = \operatorname{inf}\{ t > 0 : B_{t} 
otin G \}$.如果 $u$ 满足条件 (a),则

\[
M_{t} = u(B_{t}) \exp\left( \int_{0}^{t} c(B_{s}) ds \right)
\]

在区间 $[0, \tau)$ 上是一个局部鞅.

    \end{thm}
    
    
    
    \begin{prf}
        [Proof-that-$M-t$-is-a-Local-Martingale]
        {$M\_t$为局部鞅的证明}
        [Proof that $M_t$ is a Local Martingale]
        [gpt-4.1]
        
令 $c_{t} = \int_{0}^{t} c(B_{s}) ds$.应用 Itô 公式 (见 9.4.1) 可得

\[
u(B_{t}) \exp(c_{t}) - u(B_{0}) = \int_{0}^{t} \exp(c_{s}) 
abla u(B_{s}) \cdot dB_{s} + \int_{0}^{t} u(B_{s}) \exp(c_{s}) c(B_{s}) ds
+ \frac{1}{2} \int_{0}^{t} \Delta u(B_{s}) \exp(c_{s}) ds
\]

对 $t < \tau$ 成立.这证明了结论,因为 $\frac{1}{2} \Delta u + c u = 0$,且根据定理 7.6.4,右侧第一项在 $[0, \tau)$ 上是一个局部鞅.

    \end{prf}
    
    
    
    \begin{thm}
        [Probabilistic-Representation-of-Elliptic-Equations-Satisfying-Strong-Markov-Property]
        {解满足强马尔可夫性质的椭圆方程的概率表示}
        [Probabilistic Representation of Elliptic Equations Satisfying Strong Markov Property]
        [gpt-4.1]
        
设 $x \in D$,$B(x, r) \subset D$,假定 $
u$ 满足强马尔可夫性质,则

\[
u(x) = E_x \int_0^{\tau_B} g(B_s) ds + E_x 
u(B(\tau_B))
\]

且利用泰勒定理,有

\[
E_x 
u(B(\tau_B)) - 
u(x) = \frac{1}{2} \Delta 
u(x) E_x \tau_B + o(r^2)
\]

又有

\[
E_x \int_0^{\tau_B} g(B_s) ds = [g(x) + o(1)] E_x \tau_B
\]

综上,当 $r \to 0$ 时,$
u$ 满足

\[
\frac{1}{2} \Delta 
u = -g
\]

即 $
u$ 的拉普拉斯算子与 $g$ 的关系为上述形式.

    \end{thm}
    
    
    
    \begin{thm}
        [Theorem_on_$
    u$_Satisfying_Condition_b]
        {关于 $
    u$ 满足条件 (b) 的定理}
        [Theorem on $
    u$ Satisfying Condition (b)]
        [gpt-4.1]
        若 $|g| \leq M$,则当 $t \to 0$ 时,
\[
|
u(t, x)| \leq E_x \int_0^t |g(B_s)| ds \leq Mt \to 0
\]
因此,$
u$ 满足条件 (b).

    \end{thm}
    
    
    
    \begin{prf}
        [Proof_that_$
    u$_Satisfies_Condition_b]
        {证明 $
    u$ 满足条件 (b)}
        [Proof that $
    u$ Satisfies Condition (b)]
        [gpt-4.1]
        如果 $|g| \leq M$,则当 $t \to 0$ 时
\[
|
u(t, x)| \leq E_x \int_0^t |g(B_s)| ds \leq Mt \to 0
\]
从而得证.

    \end{prf}
    
    
    
    \begin{dfn}
        [Jordan-Decomposition-of-Finite-Signed-Measure]
        {有限签名测度的Jordan分解}
        [Jordan Decomposition of Finite Signed Measure]
        [gpt-4.1]
        设 $\alpha$ 是 $(\mathbf{R}, \mathcal{R})$ 上的有限签名测度(即不取值为 $\infty$ 或 $-\infty$),则其 Jordan 分解为 $\alpha = \alpha_{+} - \alpha_{-}$.
    \end{dfn}
    
    
    
    \begin{dfn}
        [Distribution-Function-of-Finite-Signed-Measure]
        {有限签名测度的分布函数}
        [Distribution Function of Finite Signed Measure]
        [gpt-4.1]
        设 $A(x) = \alpha((-\infty, x])$, $F(x) = \alpha_{+}((-\infty, x])$, $G(x) = \alpha_{-}((-\infty, x])$,则 $A(x) = F(x) - G(x)$,即有限签名测度的分布函数可以表示为两个有界递增函数之差.
    \end{dfn}
    
    
    
    \begin{dfn}
        [Total-Variation-of-Signed-Measure]
        {签名测度的全变差}
        [Total Variation of Signed Measure]
        [gpt-4.1]
        设 $|\alpha| = \alpha^{+} + \alpha^{-}$,则 $|\alpha|$ 称为 $\alpha$ 的全变差(total variation),其中在本例中 $|\alpha|((a, b])$ 是分析教材中定义的 $A$ 在区间 $(a, b]$ 上的全变差.
    \end{dfn}
    
    
    
    \begin{prf}
        [Proof-of-the-Equation-for-Measure-and-Transition-Probability-Product]
        {关于测度与转移概率的乘积等式的证明}
        [Proof of the Equation for Measure and Transition Probability Product]
        [gpt-4.1]
        证明 在给出证明之前,我们提醒读者,测度在左侧与转移概率相乘,即,在第一种情况下我们要证明 $\mu 
u \bar{p} = \mu \bar{p}$.
如果我们首先按照 $
u$ 进行一次转移,然后按照 $\bar{p}$ 进行一次转移,这相当于只按照 $\bar{p}$ 进行一次转移,因为只有在 $\alpha$ 处的质量受到 $
u$ 的影响,并且

\[
\bar{p}(\alpha, D) = \int \rho(dx)\, \bar{p}(x, D)
\]

第二个等式也很容易从定义中推出.换句话说,如果 $\bar{p}$ 先作用,然后是 $
u$,那么 $
u$ 会将 $\alpha$ 处的质量返回到其原来的位置.

    \end{prf}
    
    
    
    \begin{thm}
        [Theorem-on-Rate-of-Escape-for-Brownian-Motion]
        {Brown运动趋于无穷的速度定理}
        [Theorem on Rate of Escape for Brownian Motion]
        [gpt-4.1]
        假设 $g(t)$ 是正且递减的函数.则
\[
P_{0}(|B_{t}| \le g(t) \sqrt{t}~i.o.~\text{as}~t \uparrow \infty) = 1~\text{or}~0
\]
当且仅当
\[
\int^{\infty} g(t)^{d-2} / t\, dt = \infty~\text{or}~< \infty
\]
其中下限的缺失表示我们只关心积分在 $\infty$ 附近的行为.
    \end{thm}
    
    
    
    \begin{thm}
        [Cone-Condition-Theorem-Cone-Contained-in-Complement-Implies-Regularity]
        {锥条件定理:锥体包含于补集蕴含正则性}
        [Cone Condition Theorem: Cone Contained in Complement Implies Regularity]
        [gpt-4.1]
        如果存在一个以 $y$ 为顶点的锥体 $V$ 和一个 $r > 0$,使得 $V \cap D(y, r) \subset G^{c}$,则 $y$ 是正则点.
    \end{thm}
    
    
    
    \begin{dfn}
        [Definition-of-Cone-with-Vertex-at-$y$]
        {顶点为 $y$ 的锥体的定义}
        [Definition of Cone with Vertex at $y$]
        [gpt-4.1]
        以 $y$ 为顶点,方向为 $
u$,开口为 $a$ 的锥体定义如下:

\[
V(y, 
u, a) = \{ x : x = y + \theta(
u + z),~ \text{其中 } \theta \in (0, \infty),~ z \perp 
u,~ |z| < a \}
\]

    \end{dfn}
    
    
    
    \begin{ppt}
        [Measurability-of-Complement-Union-and-Intersection-of-Measurable-Sets]
        {可测集的补集与并交的可测性}
        [Measurability of Complement, Union, and Intersection of Measurable Sets]
        [gpt-4.1]
        
(a) 如果 $E$ 是可测集,则 $E^{c}$ 也是可测集.
(b) 如果 $E_{1}$ 和 $E_{2}$ 都是可测集,则 $E_{1} \cup E_{2}$ 和 $E_{1} \cap E_{2}$ 也是可测集.

    \end{ppt}
    
    
    
    \begin{dfn}
        [Definition-of-the-Maximum-Correlation-Coefficient]
        {相关系数的最大值定义}
        [Definition of the Maximum Correlation Coefficient]
        [gpt-4.1]
        
设 $\mathcal{G}$ 和 $\mathcal{H}$ 是两个 $\sigma$-代数,则
\[
\rho(\mathcal{G}, \mathcal{H}) = \sup \{ \operatorname{corr}(X, Y) : X \in \mathcal{G}, Y \in \mathcal{H} \}
\]
其中
\[
\operatorname{corr}(X, Y) = \frac{E X Y - E X E Y}{\| X - E X \|_{2} \| Y - E Y \|_{2}}
\]

    \end{dfn}
    
    
    
    \begin{thm}
        [Bound-on-Correlation-Coefficient-for-Sigma-Algebras-Generated-by-Gaussian-Variables]
        {高斯变量生成的 $\sigma$-代数之间相关系数的界}
        [Bound on Correlation Coefficient for Sigma-Algebras Generated by Gaussian Variables]
        [gpt-4.1]
        
Kolmogorov 和 Rozanov (1960) 证明:当 $\mathcal{G}$ 和 $\mathcal{H}$ 由高斯随机变量生成时,有
\[
\rho(\mathcal{G}, \mathcal{H}) \leq 2 \pi \alpha(\mathcal{G}, \mathcal{H})
\]

    \end{thm}
    
    
    
    \begin{thm}
        [Equivalence-of-Empirical-Distribution-Function-Extremes-and-Order-Statistics]
        {经验分布函数极值与有序统计量的等价}
        [Equivalence of Empirical Distribution Function Extremes and Order Statistics]
        [gpt-4.1]
        
\[
\begin{array}{l}
  \displaystyle \operatorname*{sup}_{0 < y < 1} \hat{G}_n(y) - y = \displaystyle \operatorname*{sup}_{1 \leq m \leq n} \frac{m}{n} - U_m^n \\
  \displaystyle \operatorname*{inf}_{0 < y < 1} \hat{G}_n(y) - y = \displaystyle \operatorname*{inf}_{1 \leq m \leq n} \frac{m-1}{n} - U_m^n
\end{array}
\]

其中 $\hat{G}_n$ 为经验分布函数,$U_1^n < U_2^n < \ldots < U_n^n$ 是样本有序统计量.上确界发生在 $\hat{G}_n$ 跳跃处,下确界发生在跳跃前.

    \end{thm}
    
    
    
    \begin{lma}
        [Lemma-on-Convergence-of-Stopping-Time-Probability]
        {关于停时概率收敛的引理}
        [Lemma on Convergence of Stopping Time Probability]
        [gpt-4.1]
        引理 9.5.9: 设 $\epsilon > 0$,则有 $P_{x_n} (\tau > \epsilon) \to 0$.
    \end{lma}
    
    
    
    \begin{ppt}
        [Boundedness-of-Expected-Stopping-Time-and-Measure-on-Bounded-Domain]
        {有界域上停时期望与测度的有界性}
        [Boundedness of Expected Stopping Time and Measure on Bounded Domain]
        [gpt-4.1]
        若 $G$ 有界,则 $C = \sup_x E_x \tau < \infty$,因此有 $\|
u\|_{\infty} \leq C \|g\|_{\infty} < \infty$.
    \end{ppt}
    
    
    
    \begin{prf}
        [Proof-of-Decomposition-When-Signed-Measure-is-Negative]
        {有符号测度为负时的分解构造证明}
        [Proof of Decomposition When Signed Measure is Negative]
        [gpt-4.1]
        
If $E$ is negative, this is true.

If not, let $n_{1}$ be the smallest positive integer, so that there is an $E_{1} \subset E$ with $\alpha(E_{1}) \geq 1 / n_{1}$.

Let $k \geq 2$.

If $F_{k} = E - (E_{1} \cup \dots \cup E_{k-1})$ is negative, we are done.

If not, we continue the construction letting $n_{k}$ be the smallest positive integer, so that there is an $E_{k} \subset F_{k}$ with $\alpha(E_{k}) \geq 1 / n_{k}$.

If the construction does not stop for any $k < \infty$, let

\[
F = \cap_{k} F_{k} = E - (\cup_{k} E_{k})
\]

Since $0 > \alpha(E) > -\infty$ and $\alpha(E_{k}) \geq 0$, it follows from the definition of signed measure that

\[
\alpha(E) = \alpha(F) + \sum_{k=1}^{\infty} \alpha(E_{k})
\]

$\alpha(F) \leq \alpha(E) < 0$, and the sum is finite.

From the last observation and the construction, it follows that $F$ can have no subset $G$ with $\alpha(G) > 0$, for then $\alpha(G) \geq 1 / N$ for some $N$ and we would have a contradiction.

    \end{prf}
    
    
    
    \begin{thm}
        [Form-of-Bounded-Solutions-to-the-Dirichlet-Problem]
        {Dirichlet 问题有界解的形式}
        [Form of Bounded Solutions to the Dirichlet Problem]
        [gpt-4.1]
        
设 $B_t$ 是 $d$ 维布朗运动,$\tau = \operatorname{inf}\{ t : B_t 
otin G \}$,其中 $G$ 是连通开集,所有边界点都是正则点,且 $P_x(\tau < \infty) < 1$ 在 $G$ 内成立.则所有在 $\partial G$ 上取值为 $0$ 的 Dirichlet 问题的有界解都具有如下形式:
\[
u(x) = c P_x(\tau = \infty)
\]
其中 $c$ 为常数.

    \end{thm}
    
    
    
    \begin{dfn}
        [Exit-Time-of-Brownian-Motion-from-Domain]
        {布朗运动首次离开域的时间}
        [Exit Time of Brownian Motion from Domain]
        [gpt-4.1]
        
设 $B_t$ 是 $d$ 维布朗运动,定义
\[
\tau = \operatorname{inf}\{ t : B_t 
otin G \}
\]
为布朗运动首次离开域 $G$ 的时间.

    \end{dfn}
    
    
    
    \begin{thm}
        [Constancy-of-Nonnegative-Superharmonic-Functions-in-Dimensions-1-and-2]
        {一维和二维非负超调和函数的常值性}
        [Constancy of Nonnegative Superharmonic Functions in Dimensions 1 and 2]
        [gpt-4.1]
        设 $f \in C^{2}$,且 $f$ 非负且为超调和函数.则在 $1$ 维和 $2$ 维情况下,$f$ 必为常数函数.
    \end{thm}
    
    
    
    \begin{xmp}
        [Example-of-Radial-Superharmonic-Function-in-Higher-Dimensions]
        {高维径向超调和函数的例子}
        [Example of Radial Superharmonic Function in Higher Dimensions]
        [gpt-4.1]
        在 $d \ge 3$ 的情况下,$f(x) = g(|x|)$ 可以作为一个超调和函数的例子.
    \end{xmp}
    
    
    
    \begin{thm}
        [Finiteness-of-Expected-Stopping-Time-for-Bounded-Domain]
        {有界函数时停时间的期望有限性}
        [Finiteness of Expected Stopping Time for Bounded Domain]
        [gpt-4.1]
        
若 $G$ 有界,则对所有 $x \in G$,有 $E_x \tau < \infty$.

    \end{thm}
    
    
    
    \begin{thm}
        [Martingale-Property-and-Integral-Representation-for-Bounded-Functions]
        {有界可测函数下的鞅性质与积分表示}
        [Martingale Property and Integral Representation for Bounded Functions]
        [gpt-4.1]
        
若 $u$ 和 $g$ 有界,则对 $t < \tau$ 有
\[
|M_t| \leq \|u\|_{\infty} + \tau \|g\|_{\infty}
\]
由于 $M_t$ 被可积随机变量支配,因此有
\[
M_{\tau} \equiv \lim_{t \uparrow \tau} M_t = \int_0^{\tau} g(B_t) dt
\]
且 $u(x) = E_x(M_{\tau})$.

    \end{thm}
    
    
    
    \begin{thm}
        [Property-Under-$C^2$-Condition]
        {二阶连续可微条件下的性质}
        [Property Under $C^2$ Condition]
        [gpt-4.1]
        
若 $
u \in C^2$,则其满足(a).

    \end{thm}
    
    
    
    \begin{xmp}
        [Example-of-the-Boundary-Value-Problem-for-$rac{1}{2}-u-+-\gamma-u-=-0$-on-$-a-a$]
        {关于方程 $rac{1}{2} u'' + \gamma u = 0$ 在区间 $(-a, a)$ 上的边界值问题的例子}
        [Example of the Boundary Value Problem for $rac{1}{2} u'' + \gamma u = 0$ on $(-a, a)$]
        [gpt-4.1]
        
设 $d = 1$,$G = (-a, a)$,$c \equiv \gamma$,$f \equiv 1$,我们要考虑的方程为
\[
\frac{1}{2} u'' + \gamma u = 0 \qquad u(a) = u(-a) = 1
\]
其通解为 $A \cos b x + B \sin b x$,其中 $b = \sqrt{2\gamma}$.

为了满足边界条件,必须有
\[
\begin{cases}
1 = A \cos b a + B \sin b a \\
1 = A \cos(-b a) + B \sin(-b a) = A \cos b a - B \sin b a
\end{cases}
\]
将两式相加再相减得
\[
2 = 2A \cos b a \qquad 0 = 2B \sin b a
\]
由此可知 $B = 0$ 恒成立,是否能解出 $A$ 取决于 $\cos b a$.

若 $\cos b a = 0$,则无解.

若 $\cos b a 
eq 0$,则 $u(x) = \cos b x / \cos b a$ 是一个解.

    \end{xmp}
    
    
    
    \begin{lma}
        [The-Laplacian-of-the-harmonic-function-$h-{	heta}$-vanishes]
        {调和函数 $h\_{	heta}$ 的拉普拉斯算子为零}
        [The Laplacian of the harmonic function $h_{	heta}$ vanishes]
        [gpt-4.1]
        在区域 $H$ 内,$\Delta h_{\theta} = 0$,也就是 $h_{\theta}$ 是调和函数.

证明:(a) 忽略 $C_{d}$,对 $h_{\theta}$ 求导得到
\[
\begin{array}{l}
D_{x_{i}} h_{\theta} = -\frac{d}{2} \cdot \frac{2(x_{i} - \theta_{i}) y}{(|x - \theta|^{2} + y^{2})^{(d + 2)/2}} \\[1em]
D_{x_{i} x_{i}} h_{\theta} = -d \frac{y}{(|x - \theta|^{2} + y^{2})^{(d + 2)/2}} + \frac{d(d + 2)(x_{i} - \theta_{i})^{2} y}{(|x - \theta|^{2} + y^{2})^{(d + 4)/2}} \\[1em]
D_{y} h_{\theta} = \frac{1}{(|x - \theta|^{2} + y^{2})^{d/2}} - d \cdot \frac{y^{2}}{(|x - \theta|^{2} + y^{2})^{(d + 2)/2}} \\[1em]
D_{yy} h_{\theta} = -d \cdot \frac{3y}{(|x - \theta|^{2} + y^{2})^{(d + 2)/2}} + \frac{d(d + 2) y^{3}}{(|x - \theta|^{2} + y^{2})^{(d + 4)/2}}
\end{array}
\]

    \end{lma}
    
    
    
    \begin{prf}
        [Proof-of-Extension-of-Measure-Consistency]
        {关于测度一致性的推广证明}
        [Proof of Extension of Measure Consistency]
        [gpt-4.1]
        设 $A \in \mathcal{P}$,且 $
u_1(A) = 
u_2(A) < \infty$.

令
\[
\mathcal{L} = \{ B \in \sigma(\mathcal{P}) : 
u_1(A \cap B) = 
u_2(A \cap B) \}
\]

首先证明 $\mathcal{L}$ 是一个 $\lambda$-系统:

1. 因为 $A \in \mathcal{P}$,且 $
u_1(A) = 
u_2(A)$,故 $\Omega \in \mathcal{L}$.

2. 若 $B, C \in \mathcal{L}$ 且 $C \subset B$,则
\[
\begin{array}{rl}
& 
u_1(A \cap (B - C)) = 
u_1(A \cap B) - 
u_1(A \cap C) \\
& \qquad = 
u_2(A \cap B) - 
u_2(A \cap C) = 
u_2(A \cap (B - C))
\end{array}
\]
其中利用了 $
u_i(A) < \infty$ 的有限性以保证减法成立.

3. 若 $B_n \in \mathcal{L}$ 且 $B_n \uparrow B$,由定理 1.1.1 的 (iii) 得
\[
u_1(A \cap B) = \lim_{n \to \infty} 
u_1(A \cap B_n) = \lim_{n \to \infty} 
u_2(A \cap B_n) = 
u_2(A \cap B)
\]

由于假设 $\mathcal{P}$ 在交运算下封闭,$\pi$-$\lambda$ 定理推出 $\mathcal{L} \supset \sigma(\mathcal{P})$,即对于任意 $A \in \mathcal{P}$ 且 $
u_1(A) = 
u_2(A) < \infty$,和任意 $B \in \sigma(\mathcal{P})$,都有 $
u_1(A \cap B) = 
u_2(A \cap B)$.

令 $A_n \in \mathcal{P}$ 且 $A_n \uparrow \Omega$,$
u_1(A_n) = 
u_2(A_n) < \infty$,利用上面的结论和定理 1.1.1 (iii),可以得到所需结论.

    \end{prf}
    
    
    
    \begin{lma}
        [Lemma-on-Invariance-of-Lebesgue-Measure-under-Translation-of-Sets]
        {集合平移下勒贝格测度不变性引理}
        [Lemma on Invariance of Lebesgue Measure under Translation of Sets]
        [gpt-4.1]
        
若 $E \subset [ 0 , 1 )$ 属于 $\bar { \mathcal { R } }$,$x \in ( 0 , 1 )$,且 $x + ^ { \prime } E = \{ ( x + y ) \bmod 1 : y \in E \}$,则 $\lambda ( E ) = \lambda ( x + ^ { \prime } E )$.

    \end{lma}
    
    
    
    \begin{prf}
        [Proof-of-Uniqueness]
        {关于唯一性的证明}
        [Proof of Uniqueness]
        [gpt-4.1]
        证明如下:

首先,由
\[
u ( x ) = E _ { x } ( f ( B _ { \tau } ) \exp ( c _ { \tau } ) ; \tau \leq t ) + E _ { x } ( u ( B _ { t } ) \exp ( c _ { t } ) ; \tau > t )
\]
由于 $f$ 有界且 $w(x) = E_x \exp(c_{\tau}) < \infty$,根据控收敛定理,$t \to \infty$ 时第一个项收敛到 $E_x (f(B_\tau) \exp(c_\tau))$.

对于第二项,注意到 $\{\tau > t\} \in \mathcal{F}_t$,由条件期望定义和马尔可夫性质,
\[
E_x(u(B_t)\exp(c_\tau); \tau > t) = E_x(E_x(u(B_t)\exp(c_\tau)|\mathcal{F}_t); \tau > t) = E_x(u(B_t)\exp(c_t)w(B_t); \tau > t)
\]
对于任意 $y \in G$,
\[
w(y) \geq \exp(-c^*) P_y(\tau \leq 1) \geq \epsilon > 0
\]
第一个不等式是显然的,后两个由 (A1) 易得.

用 $\epsilon$ 代替 $w(B_t)$,有
\[
E_x(|u(B_t)|\exp(c_t); \tau > t) \leq \epsilon^{-1} E_x(|u(B_t)|\exp(c_\tau); \tau > t) \leq \epsilon^{-1} \|u\|_\infty E_x(\exp(c_\tau); \tau > t) \to 0
\]
当 $t \to \infty$ 时,由控收敛定理,因 $w(x)=E_x\exp(c_\tau)<\infty$ 且 $P_x(\tau<\infty)=1$.

回到证明中的第一个等式,我们已证 $u(x) = 
u(x)$,证明完成.

    \end{prf}
    
    
    
    \begin{thm}
        [Well-definedness-of-the-Extended-Measure]
        {扩展测度的良定义性}
        [Well-definedness of the Extended Measure]
        [gpt-4.1]
        
设 $E = A_1 \cup B_1 = A_2 \cup B_2$ 是集合 $E$ 的两种分解,其中 $A_1, A_2 \in \mathcal{F}$,$B_1, B_2 \subset N_1, N_2$.令 $A_0 = A_1 \cap A_2$, $B_0 = B_1 \cup B_2$,则 $E = A_0 \cup B_0$ 也是一种分解,且 $A_0 \in \mathcal{F}$,$B_0 \subset N_1 \cup N_2$.对于 $i=1$ 或 $2$,有
\[
\mu(A_0) \le \mu(A_i) \le \mu(A_0) + \mu(N_1 \cup N_2) = \mu(A_0)
\]
由此可知,扩展测度 $\bar{\mu}$ 的取值与分解无关,即 $\bar{\mu}$ 是良定义的.

    \end{thm}
    
    
    
    \begin{dfn}
        [Definitions-of-$X^*\omega$-$S-n^*\omega$-$M-n^*\omega$-$F-n$-and-$F$]
        {关于 $X^*(\omega)$、$S\_n^*(\omega)$、$M\_n^*(\omega)$、$F\_n$ 及 $F$ 的定义}
        [Definitions of $X^*(\omega)$, $S_n^*(\omega)$, $M_n^*(\omega)$, $F_n$ and $F$]
        [gpt-4.1]
        
定义如下:
\[
\begin{array}{rlr}
X^{*}(\omega) = (X(\omega) - \epsilon) 1_{D}(\omega) \qquad & S_{n}^{*}(\omega) = X^{*}(\omega) + \cdots + X^{*}(\varphi^{n-1} \omega) \\
& & M_{n}^{*}(\omega) = \max(0, S_{1}^{*}(\omega), \dots, S_{n}^{*}(\omega)) \qquad F_{n} = \{M_{n}^{*} > 0\} \\
& & F = \cup_{n} F_{n} = \left\{ \sup_{k \geq 1} S_{k}^{*}/k > 0 \right\}
\end{array}
\]

    \end{dfn}
    
    
    
    \begin{dfn}
        [Definition-of-the-set-$D$]
        {关于集合 $D$ 的定义}
        [Definition of the set $D$]
        [gpt-4.1]
        
$D = \{\limsup S_{k}/k > \epsilon\}$

    \end{dfn}
    
    
    
    \begin{ppt}
        [Relationship-between-$F$-and-$D$]
        {$F$ 与 $D$ 的关系}
        [Relationship between $F$ and $D$]
        [gpt-4.1]
        
由于 $X^{*}(\omega) = (X(\omega) - \epsilon) 1_{D}(\omega)$ 且 $D = \{\limsup S_{k}/k > \epsilon\}$,可以得到
\[
F = \left\{ \sup_{k \geq 1} S_{k}/k > \epsilon \right\} \cap D = D
\]

    \end{ppt}
    
    
    
    \begin{dfn}
        [Definition-of-Equivalence-Relation-on-the-Interval]
        {区间上等价关系的定义}
        [Definition of Equivalence Relation on the Interval]
        [gpt-4.1]
        称 $x , y \in [ 0 , 1 )$ 是等价的,记作 $x \sim y$,当且仅当 $x - y$ 是有理数.
    \end{dfn}
    
    
    
    \begin{axm}
        [Application-of-the-Axiom-of-Choice]
        {选择公理的运用}
        [Application of the Axiom of Choice]
        [gpt-4.1]
        根据选择公理,存在一个集合 $B$,它包含每个等价类中恰好一个元素.
    \end{axm}
    
    
    
    \begin{dfn}
        [Arrival-Times-and-Interarrival-Times-in-Renewal-Process]
        {更新过程中的到达时间和间隔}
        [Arrival Times and Interarrival Times in Renewal Process]
        [gpt-4.1]
        设顾客在一个排队系统中按照更新过程到达,其到达时刻为 $0 = T_0 < T_1 < T_2 < \cdots$,其中 $\zeta_n = T_n - T_{n-1}$ 表示第 $n$ 次到达与上一次到达之间的时间间隔,$n \geq 1$.
    \end{dfn}
    
    
    
    \begin{dfn}
        [Customer-Service-Time-and-Service-Demand-Difference]
        {顾客服务时间与服务需求差值}
        [Customer Service Time and Service Demand Difference]
        [gpt-4.1]
        设 $\eta_n$ 表示第 $n$ 个顾客所需的服务时间,令 $\xi_n = \eta_{n-1} - \zeta_n$,其中 $n \geq 1$.
    \end{dfn}
    
    
    
    \begin{dfn}
        [Customer-Waiting-Time]
        {顾客等待时间}
        [Customer Waiting Time]
        [gpt-4.1]
        $W_n$ 表示第 $n$ 个顾客进入服务前需要等待的时间.
    \end{dfn}
    
    
    
    \begin{dfn}
        [Refill-and-Consumption-Process-in-Reservoir-Model]
        {储水库模型的补给与消耗过程}
        [Refill and Consumption Process in Reservoir Model]
        [gpt-4.1]
        假设降雨发生在更新过程的时刻 $\{T_n : n \geq 1\}$,第 $n$ 次降雨带来水量 $\eta_n$,水以常数速率 $c$ 消耗.令 $\zeta_n = T_n - T_{n-1}$,$\xi_n = \eta_{n-1} - c\zeta_n$,则 $W_n$ 表示第 $n$ 次降雨前水库中的水量.
    \end{dfn}
    
    
    
    \begin{dfn}
        [Definition-of-Future-Sigma-Field-and-Tail-Sigma-Field]
        {未来σ-域与尾σ-域的定义}
        [Definition of Future Sigma-Field and Tail Sigma-Field]
        [gpt-4.1]
        
设
\[
\begin{array}{r}
\mathcal{F}_{t}^{\prime} = \sigma( B_{s} : s \geq t ) = \text{在时刻} \ t \ \text{的未来} \\
\mathcal{T} = \cap_{t \geq 0} \mathcal{F}_{t}^{\prime} = \text{尾} \ \sigma\text{-域.}
\end{array}
\]

    \end{dfn}
    
    
    
    \begin{thm}
        [0-1-Law-for-Tail-Sigma-Field-Events]
        {尾σ-域事件的0-1律}
        [0-1 Law for Tail Sigma-Field Events]
        [gpt-4.1]
        
定理 7.2.7 若 $A \in \mathcal{T}$,则 $P_{x}(A) \equiv 0$ 或 $P_{x}(A) \equiv 1$.

    \end{thm}
    
    
    
    \begin{prf}
        [Proof-of-0-1-Law-for-Tail-Sigma-Field]
        {尾σ-域0-1律的证明}
        [Proof of 0-1 Law for Tail Sigma-Field]
        [gpt-4.1]
        
由于 $B$ 的尾 $\sigma$-域与 $X$ 的germ $\sigma$-域相同,故 $P_{0}(A) \in \{0, 1\}$.为了得到更强结论,注意到 $A \in \mathcal{F}_{1}^{\prime}$,因此 $1_{A}$ 可写为 $1_{D} \circ \theta_{1}$.应用马尔可夫性质得到
\[
\begin{array}{l}
P_{x}(A) = E_{x}(1_{D} \circ \theta_{1}) = E_{x}( E_{x}(1_{D} \circ \theta_{1} \mid \mathcal{F}_{1}) ) = E_{x}( E_{B_{1}} 1_{D}) \\
\qquad = \int (2\pi)^{-1/2} \exp( - (y-x)^{2}/2 ) P_{y}(D) \, dy
\end{array}
\]
取 $x = 0$,若 $P_{0}(A) = 0$,则 Lebesgue 测度下几乎处处 $P_{y}(D) = 0$,再次利用上式可知对所有 $x$ 有 $P_{x}(A) = 0$.对 $P_{0}(A) = 1$ 的情形,注意 $A^{c} \in \mathcal{T}$ 且 $P_{0}(A^{c}) = 0$,由前述结果得对所有 $x$ 有 $P_{x}(A^{c}) = 0$.

    \end{prf}
    
    
    
    \begin{thm}
        [Theorem-on-Local-Martingale]
        {关于局部鞅的定理}
        [Theorem on Local Martingale]
        [gpt-4.1]
        设 $c _ { s } = \int _ { 0 } ^ { s } c ( B _ { r } ) d r$.若 $u$ 满足条件 (a),则

\[
M _ { s } = u ( t - s , B _ { s } ) \exp ( c _ { s } )
\]

在区间 $[ 0 , t ]$ 上是一个局部鞅.

    \end{thm}
    
    
    
    \begin{dfn}
        [Product-Measure-Generated-by-Distribution-Functions]
        {分布函数生成的乘积测度}
        [Product Measure Generated by Distribution Functions]
        [gpt-4.1]
        
设 $F_{1}, F_{2}, \ldots$ 为分布函数,定义在 $\mathbf{R}^{n}$ 上的测度 $\mu_{n}$ 使得

\[
\mu_{n}((a_{1}, b_{1}] \times \ldots \times (a_{n}, b_{n}]) = \prod_{m=1}^{n}(F_{m}(b_{m}) - F_{m}(a_{m}))
\]

其中 $(a_{m}, b_{m}]$ 为实数区间.此测度称为由分布函数 $F_1, F_2, \ldots, F_n$ 生成的乘积测度.

    \end{dfn}
    
    
    
    \begin{ppt}
        [Independence-and-Distribution-of-Random-Variables-Generated-by-Distribution-Functions]
        {分布函数生成的随机变量的独立性与分布}
        [Independence and Distribution of Random Variables Generated by Distribution Functions]
        [gpt-4.1]
        
令 $X_{n}(\omega) = \omega_{n}$,则 $X_{n}$ 彼此独立,且 $X_{n}$ 的分布为 $F_{n}$.

    \end{ppt}
    
    
    
    \begin{thm}
        [Property-of-$C^2$-Functions-in-Domain-$G$]
        {关于二阶连续函数在域上的性质}
        [Property of $C^2$ Functions in Domain $G$]
        [gpt-4.1]
        如果 $
u \in C ^ { 2 }$,则它在 $G$ 中满足 (a).

证明:设 $x \in D$ 且 $B(x, r) \subset D$.令 $\sigma$ 为从 $B(x, r)$ 退出的时间.强马尔可夫性表明
\[
u(x) = E_x(
u(B_\sigma)\exp(c_\sigma))
\]
利用泰勒定理,
\[
\begin{array} { l }
  \displaystyle \tau 
u ( B _ { \sigma } ) \exp ( c _ { \sigma } ) - 
u ( x ) = E _ { x } \left[ \sum _ { i = 1 } ^ { d } \frac { \partial 
u } { \partial x _ { i } } ( x ) ( B _ { i } ( \tau _ { D } ) - x _ { i } ) \right] \\
  \displaystyle + \frac { 1 } { 2 } E _ { x } \left[ \sum _ { 1 \leq i , j \leq d } \frac { \partial ^ { 2 } 
u } { \partial x _ { i } \partial x _ { j } } ( x ) ( B _ { i } ( \tau _ { D } ) - x _ { i } ) ( B _ { j } ( \tau _ { D } ) - x _ { j } ) \right] + o ( r _ { D } )
\end{array}
\]
由于当 $i 
eq j$ 时,$B_i(t)$、$B_i^2(t) - t$ 和 $B_i(t)B_j(t)$ 是鞅,有
\[
E_x(
u(B(\tau_D))) - 
u(x) = \frac{1}{2} \Delta 
u(x) E_x(\tau_B) + o(r^2)
\]
另一方面,
\[
E_x\left[ \int_0^{\tau_B} g(B_s) ds \right] = [g(x) + o(1)] E_x(\tau_B)
\]
所以令 $r \to 0$,得 $(1/2)\Delta 
u = -g$.
    \end{thm}
    
    
    
    \begin{thm}
        [Expectation-Formula-for-Occupation-Times-of-Brownian-Motion-in-the-Half-Space]
        {关于半空间布朗运动占据时间的期望公式}
        [Expectation Formula for Occupation Times of Brownian Motion in the Half Space]
        [gpt-4.1]
        
若 $x \in H$,$f \geq 0$ 且有紧支撑,且 $\{x : f(x) > 0\} \subset H$,则
\[
E_x\left(\int_0^{\tau} f(B_t) dt\right) = \int G(x, y) f(y) dy - \int G(x, \bar{y}) f(y) dy
\]
其中 $\tau = \inf\{t : B_t 
otin H\}$,$\bar{y}$ 为 $y$ 关于平面 $\{y \in \mathbf{R}^d : y_d = 0\}$ 的反射.

    \end{thm}
    
    
    
    \begin{prf}
        [Proof-of-Expectation-Formula-for-Occupation-Times-of-Brownian-Motion-in-the-Half-Space]
        {半空间布朗运动占据时间期望公式的证明}
        [Proof of Expectation Formula for Occupation Times of Brownian Motion in the Half Space]
        [gpt-4.1]
        
利用 Fubini 定理(由于 $f \geq 0$ 有理),再结合反射原理和 $\{x : f(x) > 0\} \subset H$,有
\[
\begin{array}{rl}
E_x \displaystyle\int_0^{\tau} f(B_t) dt &= \displaystyle\int_0^{\infty} E_x(f(B_t); \tau > t) dt \\
&= \displaystyle\int_0^{\infty} \int_H (p_t(x, y) - p_t(x, \bar{y})) f(y) dy dt \\
&= \displaystyle\int_0^{\infty} \int_H (p_t(x, y) - a_t) f(y) dy dt \\
&\quad - \int_0^{\infty} \int_H (p_t(x, \bar{y}) - a_t) f(y) dy dt \\
&= \int G(x, y) f(y) dy - \int G(x, \bar{y}) f(y) dy
\end{array}
\]

    \end{prf}
    
    
    
    \begin{thm}
        [Finite-Additivity-of-Outer-Measure-on-Disjoint-Measurable-Sets]
        {外测度在有限可列不交可测集合上的可加性}
        [Finite Additivity of Outer Measure on Disjoint Measurable Sets]
        [gpt-4.1]
        
设 $G \subset \Omega$,$E_1, \ldots, E_n$ 是两两不交的可测集,则
\[
\mu^*\left( G \cap \bigcup_{i=1}^n E_i \right) = \sum_{i=1}^n \mu^*(G \cap E_i)
\]

    \end{thm}
    
    
    
    \begin{prf}
        [Proof-of-Finite-Additivity-of-Outer-Measure]
        {外测度有限可加性的证明}
        [Proof of Finite Additivity of Outer Measure]
        [gpt-4.1]
        
设 $F_m = \bigcup_{i \leq m} E_i$.$E_n$ 是可测集,$F_n \supset E_n$,$F_{n-1} \cap E_n = \varnothing$,所以
\[
\begin{array}{c}
\mu^*(G \cap F_n) = \mu^*(G \cap F_n \cap E_n) + \mu^*(G \cap F_n \cap E_n^c) \\
= \mu^*(G \cap E_n) + \mu^*(G \cap F_{n-1})
\end{array}
\]
由此通过归纳法即可得到所需结论.

    \end{prf}
    
    
    
    \begin{lma}
        [Estimation-of-Gaussian-Tail-Integral]
        {高斯尾积分的估计}
        [Estimation of Gaussian Tail Integral]
        [gpt-4.1]
        
对于 $x > 0$,

\[
(x^{-1} - x^{-3}) \exp(-x^2/2) \leq \int_x^\infty \exp(-y^2/2) dy \leq x^{-1} \exp(-x^2/2)
\]

因此,

\[
\int_x^\infty \exp(-y^2/2) dy \sim \frac{1}{x} \exp(-x^2/2) \quad \text{当 } x \to \infty
\]

    \end{lma}
    
    
    
    \begin{ppt}
        [Relationship-between-Brownian-Motion-Maximum-and-First-Passage-Time-Probability]
        {布朗运动极大值概率与边界首次通过时间的关系}
        [Relationship between Brownian Motion Maximum and First Passage Time Probability]
        [gpt-4.1]
        
\[
P_0 \left( \max_{0 \leq s \leq 1} B_s > a \right) = P_0(T_a \leq 1) = 2 P_0(B_1 \geq a)
\]

    \end{ppt}
    
    
    
    \begin{ppt}
        [Asymptotic-Estimate-for-Large-Deviations-of-Brownian-Motion]
        {布朗运动大偏差概率的渐近估计}
        [Asymptotic Estimate for Large Deviations of Brownian Motion]
        [gpt-4.1]
        
由(8.5.3)和布朗缩放性,有

\[
P_0\left(B_t > (t f(t))^{1/2}\right) \sim \kappa f(t)^{-1/2} \exp(-f(t)/2)
\]

其中 $\kappa = (2\pi)^{-1/2}$ 是常数.

    \end{ppt}
    
    
    
    \begin{prf}
        [Proof-of-Translation-Invariance-of-Set-Measure]
        {集合测度平移不变性的证明}
        [Proof of Translation Invariance of Set Measure]
        [gpt-4.1]
        设 $A = E \cap [ 0 , 1 - x )$ 和 $B = E \cap [ 1 - x , 1 )$.

令 $A ^ { \prime } = x + A = \{ x + y : y \in A \}$ 和 $B ^ { \prime } = x - 1 + B$.

$A, B \in \bar { \mathcal { R } }$,由平移不变性可知 $A ^ { \prime }, B ^ { \prime } \in \bar { \mathcal { R } }$ 且 $\lambda ( A ) = \lambda ( A ^ { \prime } )$,$\lambda ( B ) = \lambda ( B ^ { \prime } )$.

由于 $A ^ { \prime } \subset [ x , 1 )$ 且 $B ^ { \prime } \subset [ 0 , x )$ 互不相交,

\[
\lambda ( E ) = \lambda ( A ) + \lambda ( B ) = \lambda ( A ^ { \prime } ) + \lambda ( B ^ { \prime } ) = \lambda ( x + ^ { \prime } E )

\]

由引理 A 得证.

    \end{prf}
    
    
    
    \begin{dfn}
        [Definition-of-Functions-Related-to-Brownian-Motion]
        {布朗运动相关函数的定义}
        [Definition of Functions Related to Brownian Motion]
        [gpt-4.1]
        
引入两个与布朗运动相关的新函数:

\[
\varphi(x) = 
\begin{cases}
\log |x| & d = 2 \\
|x|^{2-d} & d \geq 3
\end{cases}
\]

其中,$\varphi(x)$ 在各自情形下满足 $\Delta \varphi = 0$.

    \end{dfn}
    
    
    
    \begin{lma}
        [Lemma-on-Expectation-Formula-for-Brownian-Motion]
        {布朗运动期望公式的引理}
        [Lemma on Expectation Formula for Brownian Motion]
        [gpt-4.1]
        
$\varphi(x) = E_{x} \varphi(B_{\tau})$.

    \end{lma}
    
    
    
    \begin{thm}
        [Theorem-on-Expectation-Conservation-for-Brownian-Motion-Functions]
        {布朗运动函数期望守恒定理}
        [Theorem on Expectation Conservation for Brownian Motion Functions]
        [gpt-4.1]
        
若 $\psi(x) = g(|x|)$ 是 $C^{2}$ 且具有紧支撑,并且在 $r < |x| < R$ 时满足 $\psi(x) = \varphi(x)$,则有

\[
\psi(x) = E_{x} \psi(B_{t \wedge \tau})
\]

令 $t \to \infty$ 即得所需结果.

    \end{thm}
    
    
    
    \begin{lma}
        [Lemma-on-Intersection-Closure-of-Lambda-system]
        {λ-系统的交封闭性引理}
        [Lemma on Intersection Closure of Lambda-system]
        [gpt-4.1]
        设 $\ell(\mathcal{P})$ 是由 $\pi$-系统 $\mathcal{P}$ 生成的最小 $\lambda$-系统,则 $\ell(\mathcal{P})$ 关于交运算封闭.即任意 $A, B \in \ell(\mathcal{P})$,有 $A \cap B \in \ell(\mathcal{P})$.
    \end{lma}
    
    
    
    \begin{dfn}
        [Definition-of-Family-$\mathcal{G}-A$]
        {集合族 $\mathcal{G}_A$ 的定义}
        [Definition of Family $\mathcal{G}_A$]
        [gpt-4.1]
        对任意 $A$,定义集合族 $\mathcal{G}_A = \{ B : A \cap B \in \ell(\mathcal{P}) \}$.
    \end{dfn}
    
    
    
    \begin{lma}
        [Lemma-that-$\mathcal{G}-A$-is-a-Lambda-system]
        {$\mathcal{G}_A$ 是 $\lambda$-系统的引理}
        [Lemma that $\mathcal{G}_A$ is a Lambda-system]
        [gpt-4.1]
        若 $A \in \ell(\mathcal{P})$,则 $\mathcal{G}_A$ 是一个 $\lambda$-系统.
    \end{lma}
    
    
    
    \begin{dfn}
        [Definition-of-Irreducible-Markov-Chain-and-Related-Function]
        {不可约马尔可夫链及相关函数的定义}
        [Definition of Irreducible Markov Chain and Related Function]
        [gpt-4.1]
        设 $\zeta_{n}$, $n \in \mathbf{Z}$ 是在可数状态空间 $S$ 上的不可约马尔可夫链,每个 $\zeta_{n}$ 都具有平稳分布 $\pi$.令 $X_{n} = f(\zeta_{n})$,其中 $\sum f(x)\pi(x) = 0$ 且 $\sum f(x)^{2} \pi(x) < \infty$.
    \end{dfn}
    
    
    
    \begin{ppt}
        [Expression-and-Bound-of-Conditional-Expectation]
        {条件期望的表达式和估计}
        [Expression and Bound of Conditional Expectation]
        [gpt-4.1]
        令 $\mathcal{F}_{-n} = \sigma(\zeta_{m}, m \leq -n)$,则
\[
E(X_{0} \mid \mathcal{F}_{-n}) = \sum_{y} p^{n}(\zeta_{-n}, y) f(y)
\]
其中 $p^{n}(x, y)$ 是 $n$ 步转移概率.

进一步,
\[
\|E(X_{0} \mid \mathcal{F}_{-n})\|_{2}^{2} = \sum_{x} \pi(x) \left( \sum_{y} p^{n}(x, y) f(y) \right)^{2}
\]

当 $f$ 有界时,利用 $\sum f(x)\pi(x) = 0$ 可得如下估计:
\[
\|E(X_{0} \mid \mathcal{F}_{-n})\|_{2}^{2} \leq \|f\|_{\infty}^{2} \sum_{x} \pi(x) \|p^{n}(x, \cdot) - \pi(\cdot)\|^{2}
\]

其中 $\| f \|_{\infty} = \sup |f(x)|$, $\| \cdot \|$ 为全变差范数.
    \end{ppt}
    
    
    
    \begin{thm}
        [Convergence-Rate-Estimate-for-Irreducible-Markov-Chains]
        {不可约马尔可夫链的收敛速率估计}
        [Convergence Rate Estimate for Irreducible Markov Chains]
        [gpt-4.1]
        若链是无周期的,则有
\[
\sup_{x} \| p^{n}(x, \cdot) - \pi(\cdot) \| \leq C e^{-\epsilon n}
\]
其中 $C$ 和 $\epsilon$ 为常数.
    \end{thm}
    
    
    
    \begin{cxmp}
        [Counterexample-for-Absence-of-Central-Limit-Theorem]
        {无中心极限定理的反例}
        [Counterexample for Absence of Central Limit Theorem]
        [gpt-4.1]
        若令 $\mathcal{X}_{n} = (\xi_{n}, \xi_{n+1})$,$f(H, T) = 1$,$f(T, H) = -1$,$f(H, H) = f(T, T) = 0$,则 $\sum_{m=1}^{n} f(X_{m}) \in \{-1, 0, 1\}$,因此不存在中心极限定理.
    \end{cxmp}
    
    
    
    \begin{dfn}
        [Definition-of-Triangular-Array-of-Random-Variables-and-Their-Sums]
        {随机变量三角数组及其和的定义}
        [Definition of Triangular Array of Random Variables and Their Sums]
        [gpt-4.1]
        设 $X_{n,m}$, $1 \leq m \leq n$, 是一个三角数组的随机变量,定义 $S_{n,m} = X_{n,1} + \cdots + X_{n,m}$,并且假设 $S_{n,m} = B(\tau_{m}^{n})$.
    \end{dfn}
    
    
    
    \begin{dfn}
        [Definition-of-Piecewise-Linear-Interpolated-Random-Process]
        {分段线性插值随机过程的定义}
        [Definition of Piecewise Linear Interpolated Random Process]
        [gpt-4.1]
        定义
\[
S_{n,(u)} = \left\{
    \begin{array}{cl}
        S_{n,m} & \text{if } u = m \in \{0, 1, \ldots, n\} \\
        \text{linear for } u \in [m-1, m] & \text{when } m \in \{1, \ldots, n\}
    \end{array}
\right.
\]
即 $S_{n,(u)}$ 在整数点取 $S_{n,m}$,在区间 $[m-1, m]$ 内为线性插值.
    \end{dfn}
    
    
    
    \begin{lma}
        [Condition-for-Piecewise-Linear-Interpolated-Process-to-Converge-to-Brownian-Motion]
        {分段线性插值过程收敛于布朗运动的条件}
        [Condition for Piecewise Linear Interpolated Process to Converge to Brownian Motion]
        [gpt-4.1]
        如果对每个 $s \in [0, 1]$,都有 $\tau_{[ns]}^{n} \to s$(以概率收敛),则
\[
\| S_{n, (n \cdot)} - B(\cdot) \| \to 0 \quad \text{以概率收敛}
\]

    \end{lma}
    
    
    
    \begin{dfn}
        [Definition-of-Random-Walk]
        {随机游走的定义}
        [Definition of Random Walk]
        [gpt-4.1]
        设 $\xi_{1}, \xi_{2}, \ldots$ 是一列独立同分布(i.i.d.)的随机变量,满足 $P(\xi_{n} = x) = 1/3$,其中 $x = a, b, c$.定义 $X_{n} = X_{n-1} \xi_{n}$,其中 $X_{0} = e$.
    \end{dfn}
    
    
    
    \begin{dfn}
        [Definition-of-Word-Length]
        {字的长度的定义}
        [Definition of Word Length]
        [gpt-4.1]
        设 $X_{n}$ 是经过最大化约化后的字,定义其长度为 $L_{n}$,即 $L_{n}$ 是约化后字 $X_{n}$ 的长度;若 $X_{n} = e$,则 $L_{n} = 0$.
    \end{dfn}
    
    
    
    \begin{dfn}
        [Definition-of-Reduction-Rule]
        {约化规则的定义}
        [Definition of Reduction Rule]
        [gpt-4.1]
        在构造过程中,若 $X_{n-1}$ 的最后一个字母与 $\xi_{n}$ 相同,则将其擦除;否则将新字母 $\xi_{n}$ 添加到末尾.
    \end{dfn}
    
    
    
    \begin{ppt}
        [Markov-Property-of-the-Length-Process]
        {长度过程的马尔可夫性}
        [Markov Property of the Length Process]
        [gpt-4.1]
        $L_{n}$ 构成一个马尔可夫链,其转移概率满足 $p(0, 1) = 1$.
    \end{ppt}
    
    
    
    \begin{dfn}
        [Definition-of-the-Minimum-Position-in-Random-Walk]
        {最小随机游走位置的定义}
        [Definition of the Minimum Position in Random Walk]
        [gpt-4.1]
        设 $m_n = \operatorname*{min}(S_0, S_1, \ldots, S_n)$,其中 $S_n$ 是之前定义的随机游走.
    \end{dfn}
    
    
    
    \begin{dfn}
        [Definition-of-Reversed-Random-Walk-Variable]
        {反转随机游走变量的定义}
        [Definition of Reversed Random Walk Variable]
        [gpt-4.1]
        令 $\xi_{m}' = \xi_{n+1-m}$,其中 $1 \leq m \leq n$.
    \end{dfn}
    
    
    
    \begin{thm}
        [Distribution-Equivalence-between-Minimum-Position-and-Maximum-of-Reversed-Random-Walk]
        {最小位置与最大反向随机游走的分布等价}
        [Distribution Equivalence between Minimum Position and Maximum of Reversed Random Walk]
        [gpt-4.1]
        (i) $S_n - m_n \overset{d}{=} W_n$.
    \end{thm}
    
    
    
    \begin{thm}
        [Relation-between-Minimum-Position-and-Maximum-of-Reversed-Random-Walk]
        {最小位置与反向随机游走最大值的关系}
        [Relation between Minimum Position and Maximum of Reversed Random Walk]
        [gpt-4.1]
        (ii) $S_n - m_n = \operatorname*{max}(S_0', S_1', \ldots, S_n')$.
    \end{thm}
    
    
    
    \begin{crl}
        [Limiting-Distribution-and-Maximum-of-Reversed-Random-Walk]
        {极限分布与反向随机游走最大值的关系}
        [Limiting Distribution and Maximum of Reversed Random Walk]
        [gpt-4.1]
        (iii) 结论:当 $n \to \infty$ 时,$W_n \Rightarrow M \equiv \operatorname*{max}(S_0', S_1', S_2', \ldots)$.
    \end{crl}
    
    
    
    \begin{prf}
        [Proof-that-$\Delta-u-=-0$]
        {关于 $\Delta u = 0$ 的证明}
        [Proof that $\Delta u = 0$]
        [gpt-4.1]
        通过将导数与积分交换,可以得到 $\Delta u = 0$.
    \end{prf}
    
    
    
    \begin{prf}
        [Proof-that-the-Boundary-Condition-is-Satisfied]
        {满足边界条件的证明}
        [Proof that the Boundary Condition is Satisfied]
        [gpt-4.1]
        下一步是证明其满足边界条件.
    \end{prf}
    
    
    
    \begin{prf}
        [Proof-of-Independence-of-$\int-d	heta\-h-	hetax-y$-on-$x$]
        {$\int d	heta\, h\_	heta(x, y)$ 关于 $x$ 独立性的证明}
        [Proof of Independence of $\int d	heta\, h_	heta(x, y)$ on $x$]
        [gpt-4.1]
        显然,$\int d\theta\, h_\theta(x, y)$ 与 $x$ 无关.
    \end{prf}
    
    
    
    \begin{prf}
        [Proof-of-Independence-of-$\int-d	heta\-h-	hetax-y$-on-$y$]
        {$\int d	heta\, h\_	heta(x, y)$ 关于 $y$ 独立性的证明}
        [Proof of Independence of $\int d	heta\, h_	heta(x, y)$ on $y$]
        [gpt-4.1]
        为了证明其与 $y$ 无关,令 $x = 0$,并对变量进行变换 $\theta_i = y \phi_i$ ($1 \leq i \leq d-1$),得到
\[
\int d\theta\, h_\theta(0, y) = \int \frac{C_d y}{(y^2 |\phi|^2 + y^2)^{d/2}} \cdot y^{d-1} d\phi = \int d\phi\, h_\phi(0, 1) = 1
\]

    \end{prf}
    
    
    
    \begin{prf}
        [Proof-that-the-Integral-Tends-to-$0$-in-the-Limit]
        {极限过程下积分趋于 $0$ 的证明}
        [Proof that the Integral Tends to $0$ in the Limit]
        [gpt-4.1]
        进行变量变换 $\theta_i = x_i + r_i y$ 并使用主导收敛定理,
\[
\int_{D(x, \epsilon)^c} d\theta\, h_\theta(x, y) = \int_{D(0, \epsilon / y)^c} dr\, h_r(0, 1) \to 0
\]

    \end{prf}
    
    
    
    \begin{crl}
        [Conclusion-Following-from-Theorem-9]
        {由定理9推出的结论}
        [Conclusion Following from Theorem 9]
        [gpt-4.1]
        由于 $P_{x, y}(\tau < \infty) = 1$ 对所有 $x \in H$ 成立,最后的结论由定理9推出.
    \end{crl}
    
    
    
    \begin{thm}
        [Strong-Law-of-Large-Numbers-for-Additive-Functionals]
        {加性泛函的强大数定律}
        [Strong Law of Large Numbers for Additive Functionals]
        [gpt-4.1]
        
假设 $p$ 是不可约并具有平稳分布 $\pi$ 的马尔可夫链.设 $f$ 是满足 $\sum |f(y)| \pi(y) < \infty$ 的函数.记 $T_{x}^{k}$ 为第 $k$ 次返回 $x$ 的时刻.

(i) 设
\[
V_{k}^{f} = f(X(T_{x}^{k})) + \cdots + f(X(T_{x}^{k+1} - 1)),\quad k \geq 1
\]
则 $V_{k}^{f}$ 依分布独立且同分布,且 $E|V_{k}^{f}| < \infty$.

(ii) 设 $K_{n} = \inf\{k : T_{x}^{k} \geq n\}$,则有
\[
\frac{1}{n} \sum_{m=1}^{K_{n}} V_{m}^{f} \to \frac{E V_{1}^{f}}{E_{x} T_{x}^{1}} = \sum f(y) \pi(y) \quad P_{\mu}\text{-a.s.}
\]

(iii) 有 $\max_{1 \leq m \leq n} V_{m}^{|f|} / n \to 0$,因此
\[
\frac{1}{n} \sum_{m=1}^{n} f(X_{m}) \to \sum_{y} f(y) \pi(y) \quad P_{\mu}\text{-a.s.}
\]
对任意初始分布 $\mu$ 成立.

    \end{thm}
    
    
    
    \begin{dfn}
        [Definition-of-Positive-and-Negative-Measures-under-Hahn-Decomposition]
        {Hahn分解下的正测度与负测度的定义}
        [Definition of Positive and Negative Measures under Hahn Decomposition]
        [gpt-4.1]
        
设 $\Omega = A \cup B$ 是一个Hahn分解,则定义
\[
\alpha_{+}(E) = \alpha(E \cap A) \quad \text{和} \quad \alpha_{-}(E) = -\alpha(E \cap B)
\]
其中 $A$ 是正集,$B$ 是负集,则 $\alpha_{+}$ 和 $\alpha_{-}$ 是测度.

    \end{dfn}
    
    
    
    \begin{ppt}
        [Mutual-Singularity-of-Positive-and-Negative-Measures]
        {正测度与负测度的互异性}
        [Mutual Singularity of Positive and Negative Measures]
        [gpt-4.1]
        
$\alpha_{+}(A^{c}) = 0$ 且 $\alpha_{-}(A) = 0$,因此 $\alpha_{+}$ 和 $\alpha_{-}$ 是互异的(mutually singular).

    \end{ppt}
    
    
    
    \begin{prf}
        [Proof-of-Uniqueness-of-Measure-Decomposition]
        {测度分解唯一性的证明}
        [Proof of Uniqueness of Measure Decomposition]
        [gpt-4.1]
        
为证明唯一性,假设 $\alpha = 
u_{1} - 
u_{2}$ 且存在集合 $D$ 使得 $
u_{1}(D) = 0$ 且 $
u_{2}(D^{c}) = 0$.
令 $C = D^{c}$,则 $\Omega = C \cup D$ 是一个Hahn分解,由 $D$ 的选择可知
\[
u_{1}(E) = \alpha(C \cap E) \quad \text{和} \quad 
u_{2}(E) = -\alpha(D \cap E)
\]

    \end{prf}
    
    
    
    \begin{dfn}
        [Definition-of-Transition-Probability-for-Coupled-Markov-Chain]
        {耦合马尔可夫链的转移概率定义}
        [Definition of Transition Probability for Coupled Markov Chain]
        [gpt-4.1]
        定义
\[
q ( ( x _ { 1 } , y _ { 1 } ) , ( x _ { 2 } , y _ { 2 } ) ) = \left\{ \begin{array} { l l } { p ( x _ { 1 } , x _ { 2 } ) p ( y _ { 1 } , y _ { 2 } ) } & { \mathrm { if } \ x _ { 1 } 
eq y _ { 1 } } \\ { p ( x _ { 1 } , x _ { 2 } ) } & { \mathrm { if } \ x _ { 1 } = y _ { 1 } ,\ x _ { 2 } = y _ { 2 } } \\ { 0 } & { \mathrm { otherwise } } \end{array} \right.
\]
即,两坐标在未碰撞前独立移动,碰撞后共同移动.
    \end{dfn}
    
    
    
    \begin{thm}
        [Total-Variation-Bound-for-Coupling-Chains]
        {耦合链全变差界定理}
        [Total Variation Bound for Coupling Chains]
        [gpt-4.1]
        对于新链 $( X _ { n } ^ { \prime } , Y _ { n } ^ { \prime } )$,设 $T ^ { \prime }$ 为对角线的击中时刻,则有
\[
\sum _ { y } | P ( X _ { n } ^ { \prime } = y ) - P ( Y _ { n } ^ { \prime } = y ) | \leq 2 P ( X _ { n } ^ { \prime } 
eq Y _ { n } ^ { \prime } ) = 2 P ( T ^ { \prime } > n )
\]
且 $T$ 与 $T ^ { \prime }$ 的分布相同,故 $P ( T ^ { \prime } > n ) \to 0$,因此结论成立.
    \end{thm}
    
    
    
    \begin{thm}
        [Expectation-Formula-for-Exit-Time-of-Brownian-Motion-from-Ball]
        {布朗运动出球时间的期望公式}
        [Expectation Formula for Exit Time of Brownian Motion from Ball]
        [gpt-4.1]
        如果 $|x| < R$,则
\[
E_{x} S_{R} = \frac{R^{2} - |x|^{2}}{d}
\]
其中 $S_{R}$ 表示从点 $x$ 出发的布朗运动首次达到半径为 $R$ 的球面的时间,$d$ 是空间的维数.

    \end{thm}
    
    
    
    \begin{thm}
        [Theorem-on-Finiteness-of-Function-in-Open-Set]
        {关于开集内函数有限性的定理}
        [Theorem on Finiteness of Function in Open Set]
        [gpt-4.1]
        设 $G$ 是连通开集.如果 $w 
ot\equiv \infty$,则对所有 $x \in G$ 有
\[
w(x) < \infty
\]

    \end{thm}
    
    
    
    \begin{prf}
        [Proof-of-Theorem-on-Finiteness-of-Function-in-Open-Set]
        {关于开集内函数有限性的定理的证明}
        [Proof of Theorem on Finiteness of Function in Open Set]
        [gpt-4.1]
        由引理 9.8.4 可知,如果 $w(x) < \infty$,$2\delta \leq r_{0}$,且 $D(x, 2\delta) \subset G$,则 $w < \infty$ 在 $D(x, \delta)$ 上成立.

据此可知 $G_{0} = \{ x : w(x)<\infty \}$ 是 $G$ 的开子集.

为了说明 $G_{0}$ 作为 $G$ 的子集也是闭的,注意到如果 $2\delta < r_{0}$,$D(y, 3\delta) \subset G$,且存在 $x_{n} \in G_{0}$ 且 $x_{n} \to y \in G$,则当 $n$ 充分大时,$y \in D(x_{n}, \delta)$ 且 $D(x_{n}, 2\delta) \subset G$,所以 $w(y) < \infty$.

    \end{prf}
    
    
    
    \begin{thm}
        [Theorem-on-Boundedness-of-Supremum-for-Functions-on-Finite-Measure-Connected-Open-Sets]
        {有限测度连通开集上函数上界有界定理}
        [Theorem on Boundedness of Supremum for Functions on Finite Measure Connected Open Sets]
        [gpt-4.1]
        设 $G$ 是一个有限勒贝格测度的连通开集,$|G| < \infty$.若 $w 
ot\equiv \infty$,则
\[
\operatorname{sup}_{x} w(x) < \infty
\]

    \end{thm}
    
    
    
    \begin{prf}
        [Proof-of-Theorem-on-Boundedness-of-Supremum-for-Functions-on-Finite-Measure-Connected-Open-Sets]
        {有限测度连通开集上函数上界有界定理的证明}
        [Proof of Theorem on Boundedness of Supremum for Functions on Finite Measure Connected Open Sets]
        [gpt-4.1]
        设 $K \subset G$ 为紧子集,使得 $|G-K| < \mu$,其中 $\mu$ 是引理 9.8.3 中 $\theta = c^{*}$ 时的常数.对每个 $x \in K$,可取 $\delta_{x}$ 使得 $2\delta_{x} \leq r_{0}$ 且 $D(x, 2\delta_{x}) \subset G$.开集 $D(x, \delta_{x})$ 覆盖 $K$,故存在有限子覆盖 $D(x_{i}, \delta_{x_{i}})$,$1 \leq i \leq I$.显然,
\[
\operatorname{sup}_{1 \leq i \leq I} w(x_{i}) < \infty
\]
由引理 9.8.4 可知 $w(y) \leq 2^{d+2} w(x_{i})$,对所有 $y \in D(x_{i}, \delta_{x_{i}})$,于是
\[
M = \operatorname{sup}_{y \in K} w(y) < \infty
\]
若 $y \in H = G - K$,则由(6.3a)式可得 $E_{y} ( \exp ( c^{*} \tau_{H} ) ) \leq 2$,利用强马尔可夫性质,
\[
\begin{array}{rl}
w(y) = E_{y} ( \exp ( c_{\tau_{H}} ) ; \tau_{H} = \tau ) + E_{y} ( \exp ( c_{\tau_{H}} ) w(B_{\tau_{H}}) ; B_{\tau_{H}} \in K ) \\
\leq 2 + M E_{y} ( \exp ( c_{\tau_{H}} ) ; B_{\tau_{H}} \in K ) \leq 2 + 2M
\end{array}
\]
因此得到所需结论.

    \end{prf}
    
    
    
    \begin{dfn}
        [Definition-of-π-system]
        {π-系统的定义}
        [Definition of π-system]
        [gpt-4.1]
        如果集合族 $\mathcal{P}$ 在交运算下封闭,即对任意 $A, B \in \mathcal{P}$,有 $A \cap B \in \mathcal{P}$,则称 $\mathcal{P}$ 为一个 $\pi$-系统.
    \end{dfn}
    
    
    
    \begin{dfn}
        [Definition-of-λ-system]
        {λ-系统的定义}
        [Definition of λ-system]
        [gpt-4.1]
        如果集合族 $\mathcal{L}$ 满足以下条件,则称 $\mathcal{L}$ 为一个 $\lambda$-系统:
(i) $\Omega \in \mathcal{L}$;
(ii) 若 $A, B \in \mathcal{L}$ 且 $A \subset B$,则 $B - A \in \mathcal{L}$;
(iii) 若 $A_{n} \in \mathcal{L}$ 且 $A_{n} \uparrow A$,则 $A \in \mathcal{L}$.
    \end{dfn}
    
    
    
    \begin{xmp}
        [Example-Family-of-rectangles-as-a-π-system]
        {矩形集族为π-系统的例子}
        [Example: Family of rectangles as a π-system]
        [gpt-4.1]
        例如,所有形如 $(a_{1}, b_{1}] \times \cdots \times (a_{d}, b_{d}]$ 的矩形所组成的集合族是一个 $\pi$-系统.
    \end{xmp}
    
    
    
    \begin{xmp}
        [Example-of-Lebesgues-Thorn]
        {Lebesgue之刺的例子}
        [Example of Lebesgue's Thorn]
        [gpt-4.1]
        
设 $d \ge 3$,定义集合
\[
G = (-1, 1)^d - \bigcup_{1 \leq n < \infty} \left( [2^{-n}, 2^{-n+1}] \times [-a_n, a_n]^{d-1} \right)
\]
其中 $n = \infty$ 时,去掉的仅为单点 $\{0\}$.

    \end{xmp}
    
    
    
    \begin{thm}
        [A-Brownian-Motion-Will-Not-Hit-B-t^2-B-t^3-=-0-0]
        {布朗运动不会同时到达原点的平方与立方为零的结论}
        [A Brownian Motion Will Not Hit (B_t^2, B_t^3) = (0, 0)]
        [gpt-4.1]
        证明 $P_0((B_t^2, B_t^3) = (0, 0) \text{ for some } t > 0) = 0$, 因此以 0 为起点的布朗运动 $B_t$ 几乎必然不会到达

\[
I_n = \{ x : x_1 \in [2^{-n}, 2^{-n+1}], x_2 = x_3 = \dots = x_d = 0 \}
\]

的任意一点.

    \end{thm}
    
    
    
    \begin{thm}
        [Probability-Estimate-for-Brownian-Motion-Exiting-a-Set-in-High-Dimensions]
        {高维布朗运动离开集合的概率估计}
        [Probability Estimate for Brownian Motion Exiting a Set in High Dimensions]
        [gpt-4.1]
        令 $T_n = \inf \{ t : B_t \in [2^{-n}, 2^{-n+1}] \times [-a_n, a_n]^{d-1} \}$,并选择 $a_n$ 足够小,则 $P_0(T_n < \infty) \leq 3^{-n}$.

又 $\sum_{n=1}^{\infty} 3^{-n} = \frac{1}{2}$,设 $\tau = \inf \{ t > 0 : B_t 
otin G \}$, $\sigma = \inf \{ t > 0 : B_t 
otin (-1,1)^d \}$,则

\[
P_0(\tau < \sigma) \leq \sum_{n=1}^{\infty} P_0(T_n < \infty) \leq \frac{1}{2}
\]

因此 $P_0(\tau > 0) \ge P_0(\tau = \sigma) \ge 1/2$.

    \end{thm}
    
    
    
    \begin{thm}
        [Probability-One-for-Brownian-Motion-Exiting-Measurable-Set-at-Time-Zero]
        {布朗运动首次离开测度集的概率为 1}
        [Probability One for Brownian Motion Exiting Measurable Set at Time Zero]
        [gpt-4.1]
        由于 $\tau = 0$ 可测于 $\mathcal{F}_0^+$,有 $P_0(\tau = 0) = 1$.

    \end{thm}
    
    
    
    \begin{thm}
        [Brownian-Motion-in-Two-Dimensions-Almost-Surely-Never-Hits-the-Origin-Starting-from-a-Nonzero-Point]
        {二维布朗运动从非零点出发几乎不可能到达原点}
        [Brownian Motion in Two Dimensions Almost Surely Never Hits the Origin Starting from a Nonzero Point]
        [gpt-4.1]
        设 $B_t$ 是二维布朗运动,$S_r = \inf\{ t > 0 : |B_t| = r \}$,$S_R$ 同理为到达半径 $R$,$S_0 = \inf\{ t > 0 : B_t = 0 \}$.则对任意 $x 
eq 0$,
\[
P_x(S_0 < S_R) = \lim_{r \to 0} P_x(S_r < S_R) = 0
\]
由于此结论对所有 $R$ 均成立,且布朗路径的连续性保证 $S_R \uparrow \infty$ 当 $R \uparrow \infty$,因此
\[
P_x(S_0 < \infty) = 0, \text{ 对所有 } x 
eq 0.
\]
即二维布朗运动从非零点出发,几乎不可能到达原点.

    \end{thm}
    
    
    
    \begin{thm}
        [A-Sufficient-Condition-for-the-Central-Limit-Theorem]
        {中心极限定理的一个充分条件}
        [A Sufficient Condition for the Central Limit Theorem]
        [gpt-4.1]
        
设 $(X_j)$ 是适当的随机变量序列,$\mathcal{F}_0, \mathcal{F}_{-1}$ 是相关的滤子.若满足
\[
\sum_{j \geq 0} \| E ( X_{j} | \mathcal{F}_{0} ) - E ( X_{j} | \mathcal{F}_{-1} ) \|_{2} < \infty
\]
则这是中心极限定理成立的一个充分条件.

    \end{thm}
    
    
    
    \begin{thm}
        [A-Sufficient-Condition-for-the-CLT-in-Example-7]
        {例 7 中中心极限定理的充分条件}
        [A Sufficient Condition for the CLT in Example 7]
        [gpt-4.1]
        
利用上述结果可知,若 $\sum c_{k} < \infty$,则例 7 中的中心极限定理成立.

    \end{thm}
    
    
    
    \begin{thm}
        [One-to-One-Correspondence-Between-Tail-Sigma-Field-and-Space-Time-Harmonic-Functions]
        {关于尾σ-域与空间-时间调和函数的一一对应}
        [One-to-One Correspondence Between Tail Sigma-Field and Space-Time Harmonic Functions]
        [gpt-4.1]
        
设 $X_{0}$ 的初始分布为 $\mu$.则下面的方程

\[
h(X_{n}, n) = E_{\mu}(Z \mid \mathcal{F}_{n}) \quad \text{and} \quad Z = \operatorname*{lim}_{n \to \infty} h(X_{n}, n)
\]

在有界的 $Z \in \mathcal{T}$ 与有界的空间-时间调和函数 $h : S \times \{ 0, 1, \ldots \} \to \mathbf{R}$ 之间建立了一一对应关系,其中 $h(X_{n}, n)$ 是鞅.

    \end{thm}
    
    
    
    \begin{prf}
        [Proof-of-the-One-to-One-Correspondence-Between-Tail-Sigma-Field-and-Space-Time-Harmonic-Functions]
        {尾σ-域与空间-时间调和函数一一对应的证明}
        [Proof of the One-to-One Correspondence Between Tail Sigma-Field and Space-Time Harmonic Functions]
        [gpt-4.1]
        
设 $Z \in \mathcal{T}$,写作 $Z = Y_{n} \circ \theta_{n}$,定义 $h(x, n) = E_{x} Y_{n}$.则有

\[
E_{\mu}(Z \mid \mathcal{F}_{n}) = E_{\mu}(Y_{n} \circ \theta_{n} \mid \mathcal{F}_{n}) = h(X_{n}, n)
\]

由马尔可夫性质可得 $h(X_{n}, n)$ 是鞅.反过来,如果 $h(X_{n}, n)$ 是有界鞅,利用定理 4.2.11 和 4.6.7 可知 $h(X_{n}, n) \to Z \in \mathcal{T}$ 当 $n \to \infty$,且 $h(X_{n}, n) = E_{\mu}(Z \mid \mathcal{F}_{n})$.

    \end{prf}
    
    
    
    \begin{thm}
        [Construction-of-a-Local-Martingale]
        {局部鞅的构造}
        [Construction of a Local Martingale]
        [gpt-4.1]
        若 $u$ 满足条件 (a),则 $M_{t} = u(B_{t})$ 在区间 $[0, \tau)$ 上是一个局部鞅.
    \end{thm}
    
    
    
    \begin{dfn}
        [Approximation-of-Local-Martingales-with-Stopping-Times]
        {局部鞅的近似与停时序列}
        [Approximation of Local Martingales with Stopping Times]
        [gpt-4.1]
        存在一个递增的停时序列 $T_{n}$,使得 $u(B(t \land T_{n}))$ 是一个鞅.
    \end{dfn}
    
    
    
    \begin{dfn}
        [Transformation-Function-from-Local-Martingale-to-True-Martingale]
        {局部鞅到真鞅的转化函数}
        [Transformation Function from Local Martingale to True Martingale]
        [gpt-4.1]
        为将一个局部鞅转化为一个真鞅,引入函数 $\gamma : [0, \infty) \to [0, \tau)$,当其值达到 $T_{n}$ 时将在该值保持一单位时间.具体地,$\gamma(t) = t$ 当 $0 \leq t \leq T_{1}$,$\gamma(t) = T_{1}$ 当 $T_{1} \leq t \leq T_{1} + 1$,一般地有
\[
\begin{array}{l}
\gamma(t) = t - k \quad \mathrm{当~} T_{k} + k \leq t \leq T_{k+1} + k \\
\gamma(t) = T_{k+1} \quad \mathrm{当~} T_{k+1} + k \leq t \leq T_{k+1} + k + 1
\end{array}
\]
    \end{dfn}
    
    
    
    \begin{ppt}
        [Property-of-the-Transformation-Function]
        {变换函数的性质}
        [Property of the Transformation Function]
        [gpt-4.1]
        $\gamma(t)$ 是一个停时,且 $\gamma(n) \leq T_{n}$.
    \end{ppt}
    
    
    
    \begin{dfn}
        [Definition-of-the-Dirichlet-Problem]
        {Dirichlet 问题的定义}
        [Definition of the Dirichlet Problem]
        [gpt-4.1]
        设 $G$ 为开集,Dirichlet 问题是寻找一个函数 $u$,使得:
(a) $u \in C^{2}$ 且 $\Delta u = 0$ 在 $G$ 内;
(b) 在 $\partial G$ 的每一点,$u$ 连续且 $u = f$,其中 $f$ 是有界函数.

    \end{dfn}
    
    
    
    \begin{lma}
        [Lemma-on-π-a.e.x-Satisfying-the-Hypothesis]
        {关于满足假设的π-a.e.x的引理}
        [Lemma on π-a.e.x Satisfying the Hypothesis]
        [gpt-4.1]
        
Lemma 5.8.8 guarantees that $\pi$-a.e. $x$ satisfies the hypothesis.

    \end{lma}
    
    
    
    \begin{prf}
        [Proof-for-$\bar{p}$-and-Analysis-of-Convergence]
        {关于$\bar{p}$的证明及收敛性分析}
        [Proof for $\bar{p}$ and Analysis of Convergence]
        [gpt-4.1]
        
Proof In view of Lemma 5.8.6, it suffices to prove the result for $\bar{p}$. We begin by observing that the existence of a stationary probability measure and the uniqueness result in Theorem 5.8.11 imply that the measure constructed in Theorem 5.8.9 has $E_{\alpha}R = \bar{\mu}(S) < \infty$. As in the proof of Theorem 5.6.6, we let $X_n$ and $Y_n$ be independent copies of the chain with initial distributions $\delta_x$ and $\pi$, respectively, and let $\tau = \inf\{ n \geq 0 : X_n = Y_n = \alpha \}$. For $m \geq 0$, let $S_m$ (resp. $T_m$) be the times at which $X_n$ (resp. $Y_n$) visit $\alpha$ for the $(m+1)$th time. $S_m - T_m$ is a random walk with mean 0 steps, so $M = \inf\{ m \geq 1 : S_m = T_m \} < \infty$ a.s., and it follows that this is true for $\tau$ as well. The computations in the proof of Theorem 5.6.6 show $| P(X_n \in C) - P(Y_n \in C) | \leq P(\tau > n)$. Since this is true for all $C$, $\| p^n(x, \cdot) - \pi(\cdot) \| \leq P(\tau > n)$, and the proof is complete.

    \end{prf}
    
    
    
    \begin{prf}
        [Proof-of-the-Change-of-Variables-Formula-for-Joint-Density]
        {分布密度变换公式的证明}
        [Proof of the Change-of-Variables Formula for Joint Density]
        [gpt-4.1]
        我们通过变量变换 $
u = r(t)$,其中 $
u_i = t_i / t_{n+1}$ 对于 $i \leq n$,$
u_{n+1} = t_{n+1}$.其逆变换为

\[
s(
u) = (
u_1 
u_{n+1}, \ldots, 
u_n 
u_{n+1}, 
u_{n+1})
\]

对应的雅可比矩阵 $\partial s_i / \partial 
u_j$ 为

\[
\left(
\begin{array}{ccccc}

u_{n+1} & 0 & \ldots & 0 & 
u_1 \\
0 & 
u_{n+1} & \ldots & 0 & 
u_2 \\
\vdots & \vdots & \ddots & \vdots & \vdots \\
0 & 0 & \ldots & 
u_{n+1} & 
u_n \\
0 & 0 & \ldots & 0 & 1
\end{array}
\right)
\]

其行列式为 $
u_{n+1}^n$.令 $W = (V_1, \ldots, V_{n+1}) = r(Z_1, \ldots, Z_{n+1})$,则变量变换公式说明 $W$ 的联合概率密度为

\[
f_W(
u_1, \ldots, 
u_n, 
u_{n+1}) = \left( \prod_{m=1}^n \lambda e^{-\lambda 
u_{n+1} (
u_m - 
u_{m-1})} \right) \lambda e^{-\lambda 
u_{n+1} (1 - 
u_n)} 
u_{n+1}^n
\]

对最后一维积分,得到 $V = (V_1, \ldots, V_n)$ 的联合密度:

\[
f_V(
u_1, \ldots, 
u_n) = \int_0^{\infty} \lambda^{n+1} 
u_{n+1}^n e^{-\lambda 
u_{n+1}} d
u_{n+1} = n!
\]

在 $0 < 
u_1 < 
u_2 < \ldots < 
u_n < 1$ 的区间内,得到了所需的联合密度.

    \end{prf}
    
    
    
    \begin{prf}
        [Proof-of-Part-ii-of-Theorem-1]
        {定理1的第二部分证明}
        [Proof of Part (ii) of Theorem 1]
        [gpt-4.1]
        假设 $A \subset \cup_{i} A_{i}$,则
\[
\mu(A) \leq \sum_{i} \mu(A_{i})
\]
因此有 $\mu(A) \leq \mu^*(A)$.当然,我们总可以取 $A_1 = A$ 且其它 $A_i = \varnothing$,于是 $\mu^*(A) \leq \mu(A)$.

要证明任意 $A \in \mathcal{A}$ 是可测的,首先注意到当 $\mu^*(F) = \infty$ 时,不等式 (A.1.2) 是平凡成立的,因此可以无损地假设 $\mu^*(F) < \infty$.为证明当 $E = A$ 时 (A.1.2) 成立,注意到由于 $\mu^*(F) < \infty$,存在一列 $B_i \in \mathcal{A}$,使得 $\cup_i B_i \supset F$ 且
\[
\sum_{i} \mu(B_{i}) \leq \mu^*(F) + \epsilon
\]
由于 $\mu$ 在 $\mathcal{A}$ 上可加,且 $\mu = \mu^*$ 在 $\mathcal{A}$ 上,有
\[
\mu(B_{i}) = \mu^*(B_{i} \cap A) + \mu^*(B_{i} \cap A^{c})
\]
对 $i$ 求和并利用 $\mu^*$ 的次可加性,得到
\[
\mu^*(F) + \epsilon \geq \sum_{i} \mu^*(B_{i} \cap A) + \sum_{i} \mu^*(B_{i} \cap A^{c}) \geq \mu^*(F \cap A) + \mu^*(F \cap A^{c})
\]
因为 $\epsilon$ 是任意的,这就证明了所需的结果.

    \end{prf}
    
    
    
    \begin{lma}
        [Condition-for-Equality-of-State-Periods]
        {状态周期相等的条件}
        [Condition for Equality of State Periods]
        [gpt-4.1]
        如果 $\dot{\rho}_{x y} > 0$,则 $d_{y} = d_{x}$.
    \end{lma}
    
    
    
    \begin{prf}
        [Proof-of-Condition-for-Equality-of-State-Periods]
        {状态周期相等的条件的证明}
        [Proof of Condition for Equality of State Periods]
        [gpt-4.1]
        设 $K$ 和 $L$ 使得 $p^{K}(x, y) > 0$ 且 $p^{L}(y, x) > 0$.($x$ 是常返的,所以 $\rho_{y x} > 0$.)

\[
p^{K+L}(y, y) \geq p^{L}(y, x) p^{K}(x, y) > 0
\]

所以 $d_{y}$ 整除 $K+L$,记作 $d_{y} | (K+L)$.

设 $n$ 使得 $p^{n}(x, x) > 0$.

\[
p^{K+n+L}(y, y) \geq p^{L}(y, x) p^{n}(x, x) p^{K}(x, y) > 0
\]

所以 $d_{y} | (K+n+L)$,因此 $d_{y} | n$.

由于 $n \in I_{x}$ 是任意的,$d_{y} | d_{x}$.

交换 $y$ 和 $x$ 的角色可得 $d_{x} | d_{y}$,从而 $d_{x} = d_{y}$.
    \end{prf}
    
    
    
    \begin{dfn}
        [Definition-of-Empirical-Distribution-Function]
        {经验分布函数的定义}
        [Definition of Empirical Distribution Function]
        [gpt-4.1]
        
设 $X_{1}, X_{2}, \dots$ 是独立同分布的随机变量,其分布函数为 $F$.定义经验分布函数为
\[
\hat{F}_{n}(x) = \frac{1}{n} | \{ m \leq n : X_{m} \leq x \} |
\]
即在前 $n$ 个样本中,不超过 $x$ 的样本所占比例.

    \end{dfn}
    
    
    
    \begin{thm}
        [Glivenko-Cantelli-Theorem]
        {Glivenko-Cantelli 定理}
        [Glivenko-Cantelli Theorem]
        [gpt-4.1]
        
当 $n \to \infty$ 时,
\[
\sup_{x} | \hat{F}_{n}(x) - F(x) | \to 0 \quad a.s.
\]
即经验分布函数几乎处处一致收敛于分布函数 $F$.

    \end{thm}
    
    
    
    \begin{lma}
        [Lemma-on-$\sigma$-finite-Stationary-Measure-and-Induced-Measure]
        {关于$\sigma$-有限平稳测度与诱导测度的引理}
        [Lemma on $\sigma$-finite Stationary Measure and Induced Measure]
        [gpt-4.1]
        
如果 $
u$ 是 $p$ 的一个 $\sigma$-有限平稳测度,则 $
u(A) < \infty$ 并且 $\bar{
u} = 
u \bar{p}$ 是 $\bar{p}$ 的一个平稳测度,且有 $\bar{
u}(\alpha) < \infty$.

    \end{lma}
    
    
    
    \begin{prf}
        [Proof-of-Lemma-on-$\sigma$-finite-Stationary-Measure]
        {引理关于$\sigma$-有限平稳测度的证明}
        [Proof of Lemma on $\sigma$-finite Stationary Measure]
        [gpt-4.1]
        
我们首先证明 $
u(A) < \infty$.
若 $
u(A) = \infty$,则定义的第(ii)部分蕴含对所有 $\rho(C) > 0$ 的集合 $C$,有 $
u(C) = \infty$.
如果 $B = \cup_i B_i$ 且 $
u(B_i) < \infty$,根据上述结论 $\rho(B_i) = 0$,由可数次可加性 $\rho(B) = 0$,这导致矛盾.
因此 $
u(A) < \infty$,并有 $\bar{
u}(\alpha) = 
u \bar{p}(\alpha) = \epsilon 
u(A) < \infty$.
利用 $
u_P = 
u_{\cdot}$,可得
\[
u \bar{p}(C) = 
u(C) - \epsilon 
u(A) \rho(B \cap C)
\]
由于 $
u(A) < \infty$,上述减法是良定义的,进而推出 $\bar{
u} p = 
u$.
检验 $\bar{
u}\bar{p} = \bar{
u}$,我们注意到 Lemma 5.8.4 及上述结果可得 $\bar{
u}\bar{p} = \bar{
u} p \bar{p} = 
u \bar{p} = \bar{
u}$.

    \end{prf}
    
    
    
    \begin{thm}
        [C²-Function-Satisfies-Condition-a]
        {二次连续函数满足条件(a)}
        [C² Function Satisfies Condition (a)]
        [gpt-4.1]
        如果 $
u \in C^{2}$,则它满足条件 (a).
    \end{thm}
    
    
    
    \begin{prf}
        [Proof-that-C²-Function-Satisfies-Condition-a]
        {二次连续函数满足条件(a)的证明}
        [Proof that C² Function Satisfies Condition (a)]
        [gpt-4.1]
        我们学习了这个证明方法(无需使用随机微积分),来自 Liggett (2010).

令 $x \in D$,$B = B(x, r) \subset D$.如果 $\tau_{B}$ 是从 $B$ 退出的时间,则强马尔可夫性质说明
\[
u(x) = E_{x} 
u(B(\tau_{B})).
\]

利用泰勒定理
\[
\begin{array}{l}
{\displaystyle
E_{x} 
u(B(\tau_{B})) - 
u(x) = E_{x} \sum_{i=1}^{d} \frac{\partial 
u}{\partial x_{i}}(x) [ B_{i}(\tau_{B}) - x_{i} ]
}
\\
{\displaystyle
\qquad + \frac{1}{2} E_{x} \sum_{1 \leq i, j \leq d} \frac{\partial^{2} 
u}{\partial x_{i} \partial x_{j}}(x) [ B_{i}(\tau_{B}) - x_{i} ][ B_{j}(\tau_{B}) - x_{j} ] + o(r^{2})
}
\end{array}
\]

由于对于 $i 
eq j$,$B_{i}(t)$、$B_{i}^{2}(t) - t$ 和 $B_{i}(t) B_{j}(t)$ 是鞅,有
\[
\begin{array}{rl}
E_{x} [ B_{i}(\tau_{B}) - x_{i} ] = 0 \quad &
E_{x} [ B_{i}(\tau_{B}) - x_{i} ][ B_{j}(\tau_{B}) - x_{j} ] = 0 \quad \mathrm{当~} i 
eq j \\
& E_{x} [ B_{i}(\tau_{B}) - x_{i} ]^{2} = E_{0} [ B_{i}^{2}(\tau_{B}) ] = E_{x} \tau_{B}
\end{array}
\]

将这些结果代入前式得
\[
0 = E_{x} 
u(B(\tau_{B})) - 
u(x) = \frac{1}{2} \Delta 
u(x) E_{x} \tau_{B} + o(r^{2})
\]

这就证明了所需结论.
    \end{prf}
    
    
    
    \begin{dfn}
        [Definition-of-Almost-Invariant-and-Strictly-Invariant-Sets]
        {几乎不变集与严格不变集的定义}
        [Definition of Almost Invariant and Strictly Invariant Sets]
        [gpt-4.1]
        有些作者称集合 $A$ 为'几乎不变',当且仅当 $P(A \Delta \varphi^{-1}(A)) = 0$.集合 $C$ 被称为'严格意义上的不变集',当且仅当 $C = \varphi^{-1}(C)$.
    \end{dfn}
    
    
    
    \begin{ppt}
        [Property-of-Inclusion-for-Invariant-Sets]
        {不变集的包含关系性质}
        [Property of Inclusion for Invariant Sets]
        [gpt-4.1]
        设 $A$ 为任意集合,令 $B = \cup_{n=0}^{\infty} \varphi^{-n}(A)$,则有 $\varphi^{-1}(B) \subset B$.
    \end{ppt}
    
    
    
    \begin{ppt}
        [Property-of-Construction-for-Strictly-Invariant-Sets]
        {严格不变集的构造性质}
        [Property of Construction for Strictly Invariant Sets]
        [gpt-4.1]
        设 $B$ 为任意集合且满足 $\varphi^{-1}(B) \subset B$,令 $C = \cap_{n=0}^{\infty} \varphi^{-n}(B)$,则有 $\varphi^{-1}(C) = C$.
    \end{ppt}
    
    
    
    \begin{thm}
        [Equivalence-of-Almost-Invariant-Sets-and-Strictly-Invariant-Sets]
        {几乎不变集与严格不变集的等价关系}
        [Equivalence of Almost Invariant Sets and Strictly Invariant Sets]
        [gpt-4.1]
        $A$ 是几乎不变集当且仅当存在严格意义上的不变集 $C$ 使得 $P(A \Delta C) = 0$.
    \end{thm}
    
    
    
    \begin{dfn}
        [Definition-of-Entropy]
        {熵的定义}
        [Definition of Entropy]
        [gpt-4.1]
        设 $\varphi$ 为一个在 $(0, \infty)$ 上有界的凹函数,例如 $\varphi(x) = \frac{x}{x+1}$.则定义测度 $\mu$ 的熵为
\[
\mathcal{E}(\mu) = \sum_{y} \varphi \left( \frac{\mu(y)}{\pi(y)} \right) \pi(y)
\]

    \end{dfn}
    
    
    
    \begin{thm}
        [Monotonic-Increase-of-Entropy-Under-Irreducible-Transformation]
        {熵在不可约变换下单调增加}
        [Monotonic Increase of Entropy Under Irreducible Transformation]
        [gpt-4.1]
        设 $p$ 是不可约的,$\varphi$ 是有界凹函数,则对任意初始测度 $\mu$,有
\[
\mathcal{E}(\mu p) \geq \mathcal{E}(\mu)
\]
即应用一次 $p$ 后,熵会增加.

    \end{thm}
    
    
    
    \begin{prf}
        [Proof-of-Entropy-Monotonicity]
        {熵单调性的证明}
        [Proof of Entropy Monotonicity]
        [gpt-4.1]
        由于 $\varphi$ 是凹函数,且 $
u(x) = \pi(x) p(x, y) / \pi(y)$ 是概率分布,
\[
\begin{array}{l}
  \displaystyle \mathcal{E}(\mu p) = \sum_{y} \varphi \left( \sum_{x} \frac{\mu(x) p(x, y)}{\pi(y)} \right) \pi(y) = \sum_{y} \varphi \left( \sum_{x} \frac{\mu(x)}{\pi(x)} \cdot \frac{\pi(x) p(x, y)}{\pi(y)} \right) \pi(y) \\
  \displaystyle \qquad \geq \sum_{y} \sum_{x} \varphi \left( \frac{\mu(x)}{\pi(x)} \right) \frac{\pi(x) p(x, y)}{\pi(y)} \pi(y)
\end{array}
\]
由于 $\pi(y)$ 抵消且 $\sum_{y} p(x, y) = 1$,最后一项等于 $\mathcal{E}(\mu)$,故已证 $\mathcal{E}(\mu p) \geq \mathcal{E}(\mu)$.

    \end{prf}
    
    
    
    \begin{crl}
        [Structure-of-Invariant-Measure]
        {不变测度的结构}
        [Structure of Invariant Measure]
        [gpt-4.1]
        若 $p(x, y) > 0$ 对所有 $x, y$ 成立,且 $\mu p = \mu$,则 $\mu(\boldsymbol{x}) / \pi(\boldsymbol{x})$ 必为常数,否则应用 Jensen 不等式时将出现严格不等号.

    \end{crl}
    
    
    
    \begin{thm}
        [Measurability-of-Countable-Union-of-Measurable-Sets]
        {可测集可数并的可测性}
        [Measurability of Countable Union of Measurable Sets]
        [gpt-4.1]
        如果集合 $E_i$ ($i=1,2,\ldots$)都是可测集,则 $E = \bigcup_{i=1}^{\infty} E_i$ 也是可测集.
    \end{thm}
    
    
    
    \begin{prf}
        [Proof-of-Measurability-of-Countable-Union-of-Measurable-Sets]
        {可测集可数并的可测性证明}
        [Proof of Measurability of Countable Union of Measurable Sets]
        [gpt-4.1]
        设 $E_i' = E_i \cap \left( \bigcap_{j < i} E_j^c \right)$.
(a) 和 (b) 蕴含 $E_i'$ 可测,因此可不失一般性地假设 $E_i$ 两两不交.
令 $F_n = E_1 \cup \cdots \cup E_n$.
由 (b) 知 $F_n$ 可测,利用单调性和 (c) 有
\[
\begin{array} { l } { { \mu ^ { * } ( G ) = \mu ^ { * } ( G \cap F _ { n } ) + \mu ^ { * } ( G \cap F _ { n } ^ { c } ) \geq \mu ^ { * } ( G \cap F _ { n } ) + \mu ^ { * } ( G \cap E ^ { c } ) } } \\ { { \qquad = \displaystyle \sum _ { i = 1 } ^ { n } \mu ^ { * } ( G \cap E _ { i } ) + \mu ^ { * } ( G \cap E ^ { c } ) } } \end{array}
\]
令 $n \to \infty$ 并用次可加性有
\[
\mu ^ { * } ( G ) \geq \sum _ { i = 1 } ^ { \infty } \mu ^ { * } ( G \cap E _ { i } ) + \mu ^ { * } ( G \cap E ^ { c } ) \geq \mu ^ { * } ( G \cap E ) + \mu ^ { * } ( G \cap E ^ { c } )
\]
这即为所需证明.
    \end{prf}
    
    
    
    \begin{thm}
        [Conditional-Probability-Formula-for-the-Multivariate-Distribution-of-Brownian-Bridge]
        {布朗桥多维分布的条件概率公式}
        [Conditional Probability Formula for the Multivariate Distribution of Brownian Bridge]
        [gpt-4.1]
        
如果 $0 = t_{0} < t_{1} < \ldots < t_{n} < t_{n+1} = 1, x_{0} = 0, x_{n+1} = 0$, 且 $x_{1}, \ldots, x_{n} \in \mathbf{R}$, 则有
\[
P( B( t_{1} ) = x_{1}, \ldots, B( t_{n} ) = x_{n} \mid B( 1 ) = 0 ) 
= \frac{1}{p_{1}(0, 0)} \prod_{m=1}^{n+1} p_{t_{m} - t_{m - 1}}( x_{m - 1}, x_{m} )
\]
其中 $p_{t}(x, y) = (2 \pi t)^{-1/2} \exp\left( - (y - x)^2 / 2t \right )$.

    \end{thm}
    
    
    
    \begin{prf}
        [Proof-of-the-Distribution-Formula-for-Brownian-Bridge]
        {布朗桥分布公式的证明}
        [Proof of the Distribution Formula for Brownian Bridge]
        [gpt-4.1]
        
公式(8.4.4)说明了 $B_{t}^{0}$ 的概率密度函数是多元正态分布,均值为 0.由于 $B_{t} - t B_{1}$ 也具有这一性质,只需证明二者协方差相等.计算如下:
若 $s < t$,则
\[
E \left[ ( B_{s} - s B_{1} )( B_{t} - t B_{1} ) \right ] = s - s t - s t + s t = s(1 - t)
\]
对于另一个过程,$P( B_{s}^{0} = x, B_{t}^{0} = y )$ 为
\[
\frac{ \exp( - x^{2} / 2s ) }{ (2 \pi s)^{1/2} }
\cdot \frac{ \exp( - (y - x)^{2} / 2 (t - s) ) }{ (2 \pi (t - s))^{1/2} }
\cdot \frac{ \exp( - y^{2} / 2 (1 - t) ) }{ (2 \pi (1 - t))^{1/2} }
\cdot (2 \pi)^{1/2}
= (2 \pi)^{-1} ( s (t - s)(1 - t) )^{-1/2} \exp\left( - ( a x^{2} + 2 b x y + c y^{2} ) / 2 \right )
\]
其中
\[
a = \frac{1}{s} + \frac{1}{t - s} = \frac{t}{s(t - s)} \qquad b = - \frac{1}{t - s} \\
c = \frac{1}{t - s} + \frac{1}{1 - t} = \frac{1 - s}{(t - s)(1 - t)}
\]
结合第3.9节末讨论以及
\[
\left( \frac{t}{s(t - s)}, \frac{-1}{(t - s)(1 - t)} \right )^{-1} = \begin{pmatrix} s(1 - s) \\ s(1 - t) \end{pmatrix} s(1 - t)
\]
可验证协方差相等,证明了分布公式的成立.

    \end{prf}
    
    
    
    \begin{thm}
        [Theorem-on-Differentiation-Under-the-Integral-Sign]
        {可微情形下的积分与求导交换定理}
        [Theorem on Differentiation Under the Integral Sign]
        [gpt-4.1]
        设 $(S, \mathcal{S}, \mu)$ 是一个测度空间,$f$ 是定义在 $\mathbf{R} \times S$ 上的复值函数.令 $\delta > 0$,并假设对 $x \in (y - \delta, y + \delta)$,有

(i) $u(x) = \int_{S} f(x, s) \mu(d s)$ 且 $\int_{S} |f(x, s)| \mu(d s) < \infty$;

(ii) 对固定的 $s$,$\partial f / \partial x (x, s)$ 存在且关于 $x$ 连续;

\[
\int_{S} \sup_{\theta \in [ -\delta, \delta ]} \left| \frac{\partial f}{\partial x}(y + \theta, s) \right| \mu(d s) < \infty
\]

则有 $u^{\prime}(y) = 
u(y)$.

    \end{thm}
    
    
    
    \begin{prf}
        [Proof-of-Theorem-on-Differentiation-Under-the-Integral-Sign]
        {可微情形下积分与求导交换定理的证明}
        [Proof of Theorem on Differentiation Under the Integral Sign]
        [gpt-4.1]
        由定理 A.5.1,只需验证该结论中的 (iii) 和 (iv) 条件成立.因为

\[
\int_{-\delta}^{\delta} \left| \frac{\partial f}{\partial x}(y + \theta, s) \right| d \theta \leq 2\delta \sup_{\theta \in [ -\delta, \delta ]} \left| \frac{\partial f}{\partial x}(y + \theta, s) \right|
\]

所以 (iv) 条件成立.检查 (iii) 时,有

\[
| 
u(x) - 
u(y) | \leq \int_{S} \left| \frac{\partial f}{\partial x}(x, s) - \frac{\partial f}{\partial x}(y, s) \right| \mu(d s)
\]

由 (ii) 可知被积函数当 $x \to y$ 时收敛于 0.由 (iii$'$) 和主控收敛定理得结论成立.

    \end{prf}
    
    
    
    \begin{thm}
        [Martingale-Property-of-Solutions-to-the-Inhomogeneous-Heat-Equation]
        {非齐次热方程的解的鞅性质}
        [Martingale Property of Solutions to the Inhomogeneous Heat Equation]
        [gpt-4.1]
        设 $u$ 满足如下条件:

(a) $u \in C^{1,2}$ 且 $u_{t} = \frac{1}{2} \Delta u + g$ 在 $(0, \infty) \times \mathbf{R}^{d}$ 上成立;
(b) $u$ 在每个 $\{0\} \times \mathbf{R}^{d}$ 的点处连续,且 $u(0, x) = f(x)$ 为有界连续函数.

则对于 $s \in [0, t]$,

\[
M_{s} = u(t-s, B_{s}) + \int_{0}^{s} g(B_{r})\, dr
\]

是区间 $[0, t]$ 上的局部鞅.

    \end{thm}
    
    
    
    \begin{lma}
        [Exponential-Expectation-Bound-for-Exit-Time-from-Small-Measure-Open-Set-by-Brownian-Motion]
        {关于布朗运动在小测度开集退出时间的指数型期望界}
        [Exponential Expectation Bound for Exit Time from Small Measure Open Set by Brownian Motion]
        [gpt-4.1]
        
设 $\theta > 0$.存在 $\mu > 0$,使得对于勒贝格测度满足 $|H| \leq \mu$ 的开集 $H$,令 $\tau_{H} = \operatorname{inf}\{ t > 0 : B_{t} 
otin H \}$,则有
\[
\sup_{x} E_{x}\left( \exp(\theta \tau_{H}) \right) \leq 2
\]

    \end{lma}
    
    
    
    \begin{prf}
        [Proof-of-the-Exponential-Expectation-Bound-for-Exit-Time-from-Small-Measure-Open-Set]
        {关于布朗运动退出小测度开集指数期望界的证明}
        [Proof of the Exponential Expectation Bound for Exit Time from Small Measure Open Set]
        [gpt-4.1]
        
取 $\gamma > 0$ 使得 $e^{\theta \gamma} \leq 4/3$.显然,
\[
P_{x}( \tau_{H} > \gamma ) \leq \int_{H} \frac{1}{(2\pi \gamma)^{d/2}} e^{ - |x - y|^{2} / 2\gamma } d y \leq \frac{ |H| }{ (2\pi \gamma)^{d/2} } \leq \frac{1}{4}
\]
只要我们取足够小的 $\mu$.

利用马尔可夫性,可得
\[
\begin{aligned}
P_{x}( \tau_{H} > k \gamma ) &= E_{x} \left( P_{ B_{\gamma} } ( \tau_{H} > (k-1)\gamma ) ; \tau_{H} > \gamma \right) \\
&\leq \frac{1}{4} \sup_{y} P_{y}( \tau_{H} > (k-1)\gamma )
\end{aligned}
\]
因此由归纳法可得,对于所有整数 $k \geq 0$,有
\[
\sup_{x} P_{x}( \tau_{H} > k \gamma ) \leq \frac{1}{4^{k}}
\]

由于 $e^{x}$ 是递增的,且 $e^{\theta \gamma} \leq 4/3$,所以
\[
\begin{aligned}
E_{x} \exp( \theta \tau_{H} ) &\leq \sum_{k=1}^{\infty} \exp( \theta \gamma k ) P_{x}( (k-1)\gamma < \tau_{H} \leq k\gamma ) \\
&\leq \sum_{k=1}^{\infty} \left( \frac{4}{3} \right)^{k} \frac{1}{4^{k-1}} = \frac{4}{3} \sum_{k=1}^{\infty} \frac{1}{3^{k-1}} = \frac{4}{3} \cdot \frac{1}{1 - \frac{1}{3}} = 2
\end{aligned}
\]

需要注意的是,期望中遗漏了 $\tau_{H}=0$ 的情况.但根据 Blumenthal 0-1 定律,要么 $P_{x}( \tau_{H}=0 ) = 0$,此时上述计算成立;要么 $P_{x}( \tau_{H}=0 ) = 1$,此时 $E_{x} \exp( \theta \tau_{H} ) = 1$.

    \end{prf}
    
    
    
    \begin{thm}
        [Formula-for-the-Distribution-of-Extremes-of-Brownian-Bridge]
        {布朗桥极值分布的公式}
        [Formula for the Distribution of Extremes of Brownian Bridge]
        [gpt-4.1]
        
设 $B_t^0$ 为标准布朗桥,$a < b$,则
\[
P_{0}(a < \min_{0 \leq t \leq 1} B_{t}^{0} < \max_{0 \leq t \leq 1} B_{t}^{0} < b)
= \sum_{n=-\infty}^{\infty} e^{-(2n(b-a))^{2}/2} - e^{-(2b + 2n(b-a))^{2}/2}
\]

    \end{thm}
    
    
    
    \begin{thm}
        [Formula-for-the-Distribution-of-Maximum-Absolute-Value-of-Brownian-Bridge]
        {布朗桥绝对值极大值分布公式}
        [Formula for the Distribution of Maximum Absolute Value of Brownian Bridge]
        [gpt-4.1]
        
设 $B_t^0$ 为标准布朗桥,$b > 0$,则
\[
P_{0}\left(\max_{0 \leq t \leq 1} |B_{t}^{0}| < b\right) = \sum_{m=-\infty}^{\infty} (-1)^{m} e^{-2m^{2} b^{2}}
\]

    \end{thm}
    
    
    
    \begin{dfn}
        [Definition-of-Exponential-Service-Time]
        {指数分布服务时间的定义}
        [Definition of Exponential Service Time]
        [gpt-4.1]
        设随机变量 $\eta_n$ 服从 $P(\eta_n > x) = e^{-\beta x}$,且 $E\zeta_n > E\eta_n$.
    \end{dfn}
    
    
    
    \begin{dfn}
        [Definition-of-First-Positive-Sum-Time-and-Value]
        {首次正和时刻和对应值的定义}
        [Definition of First Positive Sum Time and Value]
        [gpt-4.1]
        设 $T = \inf\{ n : S_n > 0 \}$,$L = S_T$,当 $T = \infty$ 时,令 $L = -\infty$.
    \end{dfn}
    
    
    
    \begin{ppt}
        [Lack-of-Memory-Property-of-Exponential-Distribution]
        {指数分布无记忆性的性质}
        [Lack of Memory Property of Exponential Distribution]
        [gpt-4.1]
        指数分布的无记忆性意味着 $P(L > x) = r e^{-\beta x}$,其中 $r = P(T < \infty)$.
    \end{ppt}
    
    
    
    \begin{dfn}
        [Recursive-Definition-of-Maximum-Times]
        {极大值时刻的递归定义}
        [Recursive Definition of Maximum Times]
        [gpt-4.1]
        令 $M$ 为最大值,$T_1 = T$,并对 $k \geq 2$ 定义 $T_k = \inf\{ n > T_{k-1} : S_n > S_{T_{k-1}} \}$.
    \end{dfn}
    
    
    
    \begin{ppt}
        [Application-of-Strong-Markov-Property-in-Maximum-Process]
        {强马尔可夫性质在极大值过程中的应用}
        [Application of Strong Markov Property in Maximum Process]
        [gpt-4.1]
        强马尔可夫性质意味着若 $T_k < \infty$,则 $S(T_{k+1}) - S(T_k) \overset{d}{=} L$ 且与 $S(T_k)$ 独立.
    \end{ppt}
    
    
    
    \begin{dfn}
        [Definition-of-Piecewise-Linear-Process-$B-nt$]
        {分段线性过程 $B\_n(t)$ 的定义}
        [Definition of Piecewise Linear Process $B_n(t)$]
        [gpt-4.1]
        
若定义
\[
B_{n}(t) = 
\begin{cases}
(Z_{m} - m) / n^{1/2} & \text{当 } t = m/n, \ m \in \{ 0, 1, \ldots, n \} \\
\text{线性} & \text{在区间 } [(m-1)/n, m/n]
\end{cases}
\]

    \end{dfn}
    
    
    
    \begin{dfn}
        [Definition-of-Statistic-$D-n$]
        {统计量 $D\_n'$ 的定义}
        [Definition of Statistic $D_n'$]
        [gpt-4.1]
        
\[
D_{n}' = \frac{n}{Z_{n+1}} \max_{0 \leq t \leq 1} \left| B_{n}(t) - t \left( B_{n}(1) + \frac{Z_{n+1} - Z_{n}}{n^{1/2}} \right) \right|
\]

    \end{dfn}
    
    
    
    \begin{thm}
        [Limit-Result-of-the-Strong-Law-of-Large-Numbers]
        {强大数定律的极限结论}
        [Limit Result of the Strong Law of Large Numbers]
        [gpt-4.1]
        
强大数定律推出 $Z_{n+1} / n \to 1$,因此第一个因子在极限中消失.

    \end{thm}
    
    
    
    \begin{prf}
        [Proof-of-Decomposition-and-Markov-Property]
        {关于分解和Markov性质的证明}
        [Proof of Decomposition and Markov Property]
        [gpt-4.1]
        我们首先在假设存在一个时间序列 $t_n \uparrow \infty$ 的条件下证明该结论,使得
\[
P_x (S < \infty) = \sum P_x (S = t_n).
\]
在这种情况下,证明方法基本与定理5的证明相同.
我们根据 $S$ 的取值将问题分解,应用Markov性质,然后将各部分合并.
设 $Z_n = Y_{t_n}(\omega)$,且 $A \in \mathcal{F}_S$,则
\[
E_x (Y_S \circ \theta_S ; A \cap \{ S < \infty \}) = \sum_{n=1}^{\infty} E_x (Z_n \circ \theta_{t_n} ; A \cap \{ S = t_n \})
\]
现在,如果 $A \in \mathcal{F}_S$,则 $A \cap \{ S = t_n \} = (A \cap \{ S \leq t_n \}) \setminus (A \cap \{ S \leq t_{n-1} \}) \in \mathcal{F}_{t_n}$,因此由Markov性质可知,该和为
\[
= \sum_{n=1}^{\infty} E_x (E_{B(t_n)} Z_n ; A \cap \{ S = t_n \}) = E_x (E_{B(S)} Y_S ; A \cap \{ S < \infty \})
\]

    \end{prf}
    
    
    
    \begin{dfn}
        [Definition-of-the-time-of-the-kth-return-to-y]
        {第k次回到y的时刻的定义}
        [Definition of the time of the kth return to y]
        [gpt-4.1]
        设 $R ( k ) = \operatorname* { min } \{ n \geq 1 : N _ { n } ( y ) = k \}$,其中 $R(k)$ 表示第 $k$ 次回到状态 $y$ 的时刻.
    \end{dfn}
    
    
    
    \begin{dfn}
        [Definition-of-the-interval-between-kth-returns]
        {第k次回返间隔的定义}
        [Definition of the interval between kth returns]
        [gpt-4.1]
        设 $t _ { k } = R ( k ) - R ( k - 1 )$, 其中 $R(0) = 0$,则 $t_k$ 表示第 $k$ 次回到 $y$ 的间隔时间.
    \end{dfn}
    
    
    
    \begin{thm}
        [Strong-Law-of-Large-Numbers-for-Return-Times]
        {回返时间的强大数定律}
        [Strong Law of Large Numbers for Return Times]
        [gpt-4.1]
        假设 $X_0 = y$,则 $t_1, t_2, \dotsc$ 是独立同分布的,且强大数定律有

\[
\frac{R ( k )}{k} \to E _ { y } T _ { y } \quad P _ { y }\text{-a.s.}
\]

即,第 $k$ 次回到 $y$ 的时刻的平均值收敛到从 $y$ 出发回到 $y$ 的期望回返时间.
    \end{thm}
    
    
    
    \begin{thm}
        [Relationship-between-Number-of-Returns-and-Steps]
        {回返次数与步数的关系}
        [Relationship between Number of Returns and Steps]
        [gpt-4.1]
        由于 $R ( N _ { n } ( y ) ) \leq n < R ( N _ { n } ( y ) + 1 )$,有

\[
\frac { R ( N _ { n } ( y ) ) } { N _ { n } ( y ) } \leq \frac { n } { N _ { n } ( y ) } < \frac { R ( N _ { n } ( y ) + 1 ) } { N _ { n } ( y ) + 1 } \cdot \frac { N _ { n } ( y ) + 1 } { N _ { n } ( y ) }
\]

令 $n \to \infty$,且 $N_n(y) \to \infty$ 几乎必然(因为 $y$ 是常返点),则有

\[
\frac{n}{N_n(y)} \to E_y T_y \quad P_y\text{-a.s.}
\]

即,步数与回返次数的比值收敛到期望回返时间.
    \end{thm}
    
    
    
    \begin{thm}
        [Limit-Properties-of-Return-Proportions-under-General-Initial-State]
        {一般初始状态下回返比例的极限性质}
        [Limit Properties of Return Proportions under General Initial State]
        [gpt-4.1]
        对于一般初始状态 $x 
eq y$,若 $T_y = \infty$,则 $N_n(y) = 0$,且

\[
N_n(y) / n \to 0 \quad \text{on} \ \{ T_y = \infty \}
\]

若 $T_y < \infty$,由强马尔可夫性质,$t_2, t_3, \ldots$ 是独立同分布的,且有 $P_x(t_k = n) = P_y(T_y = n)$,故

\[
R(k) / k = t_1 / k + (t_2 + \cdots + t_k) / k \to 0 + E_y T_y \quad P_x\text{-a.s.}
\]

重复 $x = y$ 的证明可得

\[
N_n(y) / n \to 1 / E_y T_y \quad P_x\text{-a.s. on} \ \{ T_y < \infty \}
\]

结合 $T_y = \infty$ 的结果即得结论.
    \end{thm}
    
    
    
    \begin{lma}
        [Upper-Bound-Estimate-for-Weight-Function-Inside-Ball]
        {关于球内权函数的上界估计}
        [Upper Bound Estimate for Weight Function Inside Ball]
        [gpt-4.1]
        设 $2\delta \leq r_{0}$.如果 $D(x, 2\delta) \subset G$ 且 $y \in D(x, \delta)$,则
\[
w(y) \leq 2^{d+2} w(x)
\]
这表明 $w(x) < \infty$ 蕴含 $w(y) < \infty$,对于所有 $y \in D(x, \delta)$ 都成立.
    \end{lma}
    
    
    
    \begin{prf}
        [Proof-of-Upper-Bound-Estimate-for-Weight-Function-Inside-Ball]
        {关于球内权函数上界估计的证明}
        [Proof of Upper Bound Estimate for Weight Function Inside Ball]
        [gpt-4.1]
        如果 $D(y, r) \subset G$ 且 $r \leq r_{0}$,则强马尔可夫性质(strong Markov property)蕴含
\[
\begin{aligned}
w(y) &= E_{y} [ \exp( c_{T_{r}} ) w( B( T_{r} ) ) ] \leq E_{y} [ \exp( c^{*} T_{r} ) w( B( T_{r} ) ) ] \\
&= E_{y} [ \exp( c^{*} T_{r} ) ] \int_{ \partial D(y, r) } w(z) \pi(dz) \leq 2 \int_{ \partial D(y, r) } w(z) \pi(dz)
\end{aligned}
\]
其中 $\pi$ 是 $\partial D(y, r)$ 上的概率测度,因为出界时间 $T_{r}$ 与出界位置 $B(T_{r})$ 相互独立.
若 $\delta \leq r_{0}$ 且 $D(y, \delta) \subset G$,则将上述不等式两边乘以 $\delta^{d-1}$ 并从 $0$ 到 $\delta$ 积分,
\[
\frac{ \delta^{d} }{ d } w(y) \leq 2 \cdot \frac{1}{ \sigma_{d} } \int_{ D(y, \delta) } w(z) dz
\]
其中 $\sigma_{d}$ 是单位球面 $\{ x : |x| = 1 \}$ 的面积.
整理得
\[
\int_{ D(y, \delta) } w(z) dz \geq 2^{-1} \frac{ \delta^{d} }{ C_{o} } w(y)
\]
其中 $C_{o} = d / \sigma_{d}$ 是仅依赖于 $d$ 的常数.
    \end{prf}
    
    
    
    \begin{xmp}
        [Properties-Calculation-of-Moving-Average-Process]
        {移动平均过程的性质计算}
        [Properties Calculation of Moving Average Process]
        [gpt-4.1]
        
设
\[
X_{m} = \sum_{k \geq 0} c_{k} \xi_{m - k} \quad \text{where} \quad \sum_{k \geq 0} c_{k}^{2} < \infty
\]
且 $\xi_{i}$, $i \in \mathbf{Z}$, 是独立同分布随机变量,$E\xi_{i} = 0$, $E\xi_{i}^{2} = 1$.若 $\mathcal{F}_{-n} = \sigma(\xi_{m}: m \le -n)$,则
\[
\|E(X_{0} \mid \mathcal{F}_{-n})\|_{2} = \left\| \sum_{k \geq n} c_{k} \xi_{-k} \right\|_{2} = \left( \sum_{k \geq n} c_{k}^{2} \right)^{1/2}
\]
例如若 $c_{k} = (1 + k)^{-p}$,则 $\|E(X_{0} \mid \mathcal{F}_{-n})\|_{2} \sim n^{(1/2) - p}$,且当 $p > 3/2$ 时满足 8.3.5 定理的条件.

    \end{xmp}
    
    
    
    \begin{thm}
        [Stationary-Distribution-of-the-Bernoulli-Laplace-Model-of-Diffusion]
        {Bernoulli-Laplace扩散模型的平稳分布}
        [Stationary Distribution of the Bernoulli-Laplace Model of Diffusion]
        [gpt-4.1]
        
Bernoulli-Laplace扩散模型的平稳分布为上一题(Bernoulli-Laplace模型)所对应马尔可夫链的极限分布.

    \end{thm}
    
    
    
    \begin{dfn}
        [Definition-of-Hitting-Probability]
        {hitting probability的定义}
        [Definition of Hitting Probability]
        [gpt-4.1]
        
设 $w _ { x y } = P _ { x } ( T _ { y } < T _ { x } )$,即从状态 $x$ 出发,首次到达 $y$ 之前未回到 $x$ 的概率.

    \end{dfn}
    
    
    
    \begin{thm}
        [Relation-between-Stationary-Measure-and-Hitting-Probability]
        {平稳测度与 hitting probability 的关系}
        [Relation between Stationary Measure and Hitting Probability]
        [gpt-4.1]
        
$\mu _ { x } ( y ) = w _ { x y } / w _ { y x }$.

    \end{thm}
    
    
    
    \begin{thm}
        [Multiplicative-Property-of-Stationary-Measure]
        {平稳测度的乘法性质}
        [Multiplicative Property of Stationary Measure]
        [gpt-4.1]
        
如果$p$是不可约且遍历的,则
\[
\mu _ { x } ( y ) \mu _ { y } ( z ) = \mu _ { x } ( z )
\]

    \end{thm}
    
    
    
    \begin{thm}
        [Finiteness-of-Expected-Hitting-Time-in-Irreducible-Positive-Recurrent-Markov-Chains]
        {不可约正遍历马尔可夫链的期望首次到达时间有界性}
        [Finiteness of Expected Hitting Time in Irreducible Positive Recurrent Markov Chains]
        [gpt-4.1]
        
若$p$不可约且正遍历,则对任意$x, y$,都有$E _ { x } T _ { y } < \infty$.

    \end{thm}
    
    
    
    \begin{thm}
        [Property-of-Irreducible-Markov-Chains-with-Infinite-Stationary-Measure]
        {不可约非正遍历马尔可夫链的性质}
        [Property of Irreducible Markov Chains with Infinite Stationary Measure]
        [gpt-4.1]
        
若$p$不可约,且存在一个平稳测度$\mu$使得$\sum _ { x } \mu ( x ) = \infty$,则$p$不是正遍历的.

    \end{thm}
    
    
    
    \begin{dfn}
        [Definition_of_Probability_Measure_$\mu_{u
    u}$]
        {定义概率测度 $\mu_{u,
    u}$}
        [Definition of Probability Measure $\mu_{u,
    u}$]
        [gpt-4.1]
        
对于 $A \subset ( -\infty, 0 ) \times ( 0, \infty )$,定义概率测度如下:
$\mu_{0,0} ( \{ 0 \} ) = 1$,并且

\[
\mu_{u,
u}( \{ u \} ) = \frac{ 
u }{ 
u - u }, \qquad \mu_{u,
u}( \{ 
u \} ) = \frac{ -u }{ 
u - u }, \quad \text{对于}\ u < 0 < 
u
\]

    \end{dfn}
    
    
    
    \begin{thm}
        [Equivalence-of-Integral-Formula-and-Expectation]
        {积分公式与期望的等价}
        [Equivalence of Integral Formula and Expectation]
        [gpt-4.1]
        
有如下公式成立:
\[
\int \varphi(x)\, dF(x) = E\left( \int \varphi(x)\, \mu_{U,V}(dx) \right)
\]
该公式在 $\varphi(0) = 0$ 时已被证明,但实际上对任意 $\varphi$ 都成立.

    \end{thm}
    
    
    
    \begin{ppt}
        [Total-Mass-of-the-Measure-is-1]
        {测度的全质量为 1}
        [Total Mass of the Measure is 1]
        [gpt-4.1]
        
令 $\varphi \equiv 1$ 代入上述公式,可得由 (8.1.1) 定义的概率测度的总质量为 1.

    \end{ppt}
    
    
    
    \begin{thm}
        [Distribution-Equivalence-of-Brownian-Motion-and-Variable-$X$]
        {布朗运动的分布与变量 $X$ 的等同}
        [Distribution Equivalence of Brownian Motion and Variable $X$]
        [gpt-4.1]
        
若 $(U, V)$ 的分布如 (8.1.1) 所定义,并且在同一概率空间上有一个独立的布朗运动,则有
\[
B(T_{U,V}) =_d X
\]
即在 $T_{U,V}$ 时刻的布朗运动与变量 $X$ 同分布.

    \end{thm}
    
    
    
    \begin{dfn}
        [Definition-of-Harris-Chain]
        {Harris链的定义}
        [Definition of Harris Chain]
        [gpt-4.1]
        我们称马尔可夫链 $X_n$ 是一个 Harris 链,如果存在集合 $A, B \in \mathcal{S}$,一个函数 $q$ 满足 $q(x, y) \geq \epsilon > 0$ 对于 $x \in A$, $y \in B$,以及一个概率测度 $\rho$ 集中在 $B$ 上,使得:

(i) 如果 $\tau_A = \inf \{ n \geq 0 : X_n \in A \}$,则对所有 $z \in S$,有 $P_z(\tau_A < \infty) > 0$.

(ii) 若 $x \in A$ 且 $C \subset B$,则 $p(x, C) \geq \int_C q(x, y) \rho(dy)$.

    \end{dfn}
    
    
    
    \begin{thm}
        [Hartman-Wintner-Limit-Theorem-for-Random-Walk]
        {Hartman-Wintner随机游走极限定理}
        [Hartman-Wintner Limit Theorem for Random Walk]
        [gpt-4.1]
        
设 $X_1, X_2, \dots$ 是独立同分布的随机变量,满足 $E X_i = 0$ 且 $E X_i^2 = 1$,则有
\[
\lim_{n \to \infty} \sup \frac{S_n}{(2 n \log \log n)^{1/2}} = 1
\]
其中 $S_n = X_1 + \cdots + X_n$.

    \end{thm}
    
    
    
    \begin{prf}
        [Proof-of-Hartman-Wintner-Limit-Theorem-for-Random-Walk]
        {Hartman-Wintner随机游走极限定理的证明}
        [Proof of Hartman-Wintner Limit Theorem for Random Walk]
        [gpt-4.1]
        
由定理 8.1.2,可写 $S_n = B(T_n)$,其中 $T_n / n \to 1$ 几乎处处成立.和 Donsker 定理的证明类似,以下论证只用到这一点.只需证明
\[
\frac{S_{[t]} - B_t}{(t \log \log t)^{1/2}} \to 0 \quad \text{几乎处处}
\]
为此,先观察对于任意 $\epsilon > 0$ 和 $t \geq t_0(\omega)$,有
\[
T_{[t]} \in [ t / (1 + \epsilon),\ t (1 + \epsilon) ]
\]
估算 $S_{[t]} - B_t$,设 $M(t) = \sup \{ |B(s) - B(t)| : t / (1 + \epsilon) \le s \le t (1 + \epsilon) \}$.为控制该量,令 $t_k = (1 + \epsilon)^k$,若 $t_k \leq t \leq t_{k+1}$,则
\[
\begin{array}{rl}
& M(t) \le \sup \{ |B(s) - B(t)| : t_{k-1} \le s, t \le t_{k+2} \} \\
& \qquad \le 2 \sup \{ |B(s) - B(t_{k-1})| : t_{k-1} \le s \le t_{k+2} \}
\end{array}
\]
注意到 $t_{k+2} - t_{k-1} = \delta t_{k-1}$,其中 $\delta = (1 + \epsilon)^3 - 1$,由缩放有
\[
\begin{array}{l}
P\left( \max_{t_{k-1} \leq s \leq t_{k+2}} |B(s) - B(t)| > (3 \delta t_{k-1} \log \log t_{k-1})^{1/2} \right) \\
\quad = P\left( \max_{0 \leq r \leq 1} |B(r)| > (3 \log \log t_{k-1})^{1/2} \right) \\
\quad \leq 2 \kappa (3 \log \log t_{k-1})^{-1/2} \exp( - 3 \log \log t_{k-1} / 2 )
\end{array}
\]
利用 (8.5.1) 和 (8.5.2),对 $k$ 求和并用 (b) 得
\[
\lim_{t \to \infty} \sup \frac{S_{[t]} - B_t}{(t \log \log t)^{1/2}} \leq (3 \delta)^{1/2}
\]
回忆 $\delta = (1 + \epsilon)^3 - 1$,令 $\epsilon \downarrow 0$,即得结论,证明完成.

    \end{prf}
    
    
    
    \begin{lma}
        [Lemma-on-Lower-Semicontinuity-of-$x-\mapsto-P-x\tau-\leq-t$]
        {关于 $x \mapsto P\_x(\tau \leq t)$ 下半连续性的引理}
        [Lemma on Lower Semicontinuity of $x \mapsto P_x(\tau \leq t)$]
        [gpt-4.1]
        
如果 $t > 0$,则映射 $x \mapsto P_x(\tau \leq t)$ 是下半连续的.即若 $x_n \to x$,则有
\[
\liminf_{n \to \infty} P_{x_n}(\tau \leq t) \geq P_x(\tau \leq t)
\]

    \end{lma}
    
    
    
    \begin{prf}
        [Proof-of-Lower-Semicontinuity-of-$x-\mapsto-P-x\tau-\leq-t$]
        {$x \mapsto P\_x(\tau \leq t)$ 下半连续性的证明}
        [Proof of Lower Semicontinuity of $x \mapsto P_x(\tau \leq t)$]
        [gpt-4.1]
        
由马尔可夫性,有
\[
P_x(B_s \in G^c \text{ for some } s \in (\epsilon, t]) = \int p_\epsilon(x, y) P_y(\tau \leq t - \epsilon) dy
\]
由于 $y \mapsto P_y(\tau \leq t - \epsilon)$ 有界且可测,且
\[
p_\epsilon(x, y) = (2\pi \epsilon)^{-d/2} e^{-|x - y|^2 / 2\epsilon}
\]
由控测收敛定理可知,
\[
x \mapsto P_x(B_s \in G^c \text{ for some } s \in (\epsilon, t])
\]
对每个 $\epsilon > 0$ 都是连续的.令 $\epsilon \downarrow 0$,则 $x \mapsto P_x(\tau \leq t)$ 是连续函数递增极限,因此是下半连续的.

    \end{prf}
    
    
    
    \begin{thm}
        [Theorem-on-Hitting-Time-of-Closed-Set-Being-a-Stopping-Time]
        {关于封闭集首次进入时刻为停时的定理}
        [Theorem on Hitting Time of Closed Set Being a Stopping Time]
        [gpt-4.1]
        
如果 $K$ 是一个闭集,且 $T = \operatorname*{inf} \{ t \geq 0 : B_{t} \in K \}$,则 $T$ 是一个停时(stopping time).

    \end{thm}
    
    
    
    \begin{prf}
        [Proof-that-Hitting-Time-of-Closed-Set-is-a-Stopping-Time]
        {封闭集首次进入时刻为停时的证明}
        [Proof that Hitting Time of Closed Set is a Stopping Time]
        [gpt-4.1]
        
设 $B(x, r) = \{ y : | y - x | < r \}$,$G_{n} = \cup_{x \in K} B(x, 1/n)$,$T_{n} = \operatorname*{inf} \{ t \geq 0 : B_{t} \in G_{n} \}$.
由于 $G_{n}$ 是开集,根据定理 7.3.1,$T_{n}$ 是停时.
现证明当 $n \uparrow \infty$ 时,$T_{n} \uparrow T$.
注意到对所有 $n$ 有 $T \geq T_{n}$,因此 $\lim T_{n} \leq T$.
为了证明 $T \leq \lim T_{n}$,假设 $T_{n} \uparrow t < \infty$.
由于 $B(T_{n}) \in \overline{G}_{n}$ 对所有 $n$ 都成立,且 $B(T_{n}) \to B(t)$,可知 $B(t) \in K$ 且 $T \leq t$.

    \end{prf}
    
    
    
    \begin{lma}
        [Periodicity-Structure-of-Irreducible-Recurrent-Markov-Chains]
        {不可约遍历马尔可夫链的周期性结构}
        [Periodicity Structure of Irreducible, Recurrent Markov Chains]
        [gpt-4.1]
        设 $p$ 是不可约、遍历的马尔可夫链,所有状态的周期为 $d$.固定 $x \in S$,对于每个 $y \in S$,定义 $K_{y} = \{ n \geq 1 : p^{n}(x, y) > 0 \}$.

(i) 存在 $r_{y} \in \{ 0, 1, \ldots, d - 1 \}$,使得若 $n \in K_{y}$,则 $n = r_{y} \bmod d$,即 $n - r_{y}$ 是 $d$ 的倍数.

(ii) 设 $S_{r} = \{ y : r_{y} = r \}$,其中 $0 \leq r < d$.若 $y \in S_{i}$,$z \in S_{j}$,且 $p^{n}(y, z) > 0$,则 $n = (j - i) \bmod d$.

(iii) $S_{0}, S_{1}, \ldots, S_{d-1}$ 是 $p^{d}$ 的不可约类,且所有状态的周期都是 $d$.

证明:

(i) 令 $m(y)$ 使得 $p^{m(y)}(y, x) > 0$.若 $n \in K_{y}$,则 $p^{n + m(y)}(x, x) > 0$,所以 $d$ 整除 $n + m(y)$.令 $r_{y} = (d - m(y)) \bmod d$.

(ii) 设 $m, n$ 满足 $p^{n}(y, z) > 0$,$p^{m}(x, y) > 0$.由于 $p^{n + m}(x, z) > 0$,由 (i) 得 $n + m \equiv j \bmod d$.又 $m \equiv i \bmod d$,故结论成立.

(iii) (ii) 直接推出 (iii) 的不可约性.周期性由周期定义 $\gcd\{ n : p^{n}(x, x) > 0 \}$ 得出.
    \end{lma}
    
    
    
    \begin{xmp}
        [Example-of-a-Punctured-Disc]
        {穿孔圆盘的例子}
        [Example of a Punctured Disc]
        [gpt-4.1]
        
设 $d := 3$,且 $G = D - K$,其中 $D = \{ x : |x| < 1 \}$,$K = \{ x : x_1 = x_2 = 0, x_3 \geq 0 \}$.我们关于唯一性的结论以及三维布朗运动从不碰到一条直线的事实,意味着解必须为 $
u(x) = E_x f(B_{\tau(D)})$,其中 $\tau(D)$ 是从 $D$ 的出界时间.如果我们令 $f$ 在 $D$ 的边界上为 $0$,在 $K$ 上为 $1 - x_3$,则解 $
u \equiv 0$ 在 $G$ 上,且边界条件将不能被满足.

    \end{xmp}
    
    
    
    \begin{lma}
        [Lemma-on-Covering-Sets-and-Control-of-Outer-Measure]
        {关于集合的覆盖与外测度控制的引理}
        [Lemma on Covering Sets and Control of Outer Measure]
        [gpt-4.1]
        设 $\{\Omega_i\}$ 是一列互不相交的集合,且 $\mu(\Omega_i) < \infty$,$\Omega = \cup_i \Omega_i$.对于任意集合 $B$,令 $B_i = B \cap \Omega_i$.由引理 A.2.1 可得,对于每个 $i$ 和 $n$,存在 $A_i^n \in \mathcal{A}_{\sigma}$, 满足 $A_i^n \supset B_i$ 且 $\mu(A_i^n) \leq \mu^*(B_i) + 1/(n 2^i)$.令 $A_n = \cup_i A_i^n$,则 $B \subset A_n$ 且
\[
A_n - B \subset \bigcup_{i=1}^{\infty}(A_i^n - B_i)
\]
因此,由次可加性有
\[
\mu^*(A_n - B) \leq \sum_{i=1}^{\infty} \mu^*(A_i^n - B_i) \leq 1/n
\]
由于 $A_n \in \mathcal{A}_{\sigma}$,故 $A = \cap_n A_n \in \mathcal{A}_{\sigma\delta}$,且 $A \supset B$.由于 $N \equiv A - B \subset A_n - B$ 对任意 $n$ 都成立,单调性可知 $\mu^*(N) = 0$.
    \end{lma}
    
    
    
    \begin{prf}
        [Proof-of-the-Lemma-on-Covering-Sets-and-Control-of-Outer-Measure]
        {关于集合覆盖与外测度控制引理的证明}
        [Proof of the Lemma on Covering Sets and Control of Outer Measure]
        [gpt-4.1]
        对上述引理,证明过程如下:

1. 令 $B_i = B \cap \Omega_i$.
2. 利用引理 A.2.1, 对每个 $i$ 和 $n$,可以选取 $A_i^n \in \mathcal{A}_{\sigma}$, 满足 $A_i^n \supset B_i$ 且 $\mu(A_i^n) \leq \mu^*(B_i) + 1/(n 2^i)$.
3. 令 $A_n = \cup_i A_i^n$,则 $B \subset A_n$.
4. 由于
\[
A_n - B \subset \bigcup_{i=1}^{\infty}(A_i^n - B_i)
\]
由次可加性有
\[
\mu^*(A_n - B) \leq \sum_{i=1}^{\infty} \mu^*(A_i^n - B_i) \leq 1/n
\]
5. $A_n \in \mathcal{A}_{\sigma}$,则 $A = \cap_n A_n \in \mathcal{A}_{\sigma\delta}$.
6. 显然 $A \supset B$.
7. 令 $N = A - B$,则 $N \subset A_n - B$ 对所有 $n$ 成立.单调性可知 $\mu^*(N) = 0$,证毕.
    \end{prf}
    
    
    
    \begin{thm}
        [Central-Limit-Theorem-for-Strong-Mixing-Sequences]
        {强混合序列的中心极限定理}
        [Central Limit Theorem for Strong Mixing Sequences]
        [gpt-4.1]
        
设 $X_{n}$, $n \in \mathbf{Z}$ 是遍历的平稳序列,满足 $E X_{n} = 0$,$E | X_{0} |^{2+\delta} < \infty$.令 $\alpha(n) = \alpha( \mathcal{F}_{-n}, \sigma(X_{0}) )$,其中 $\mathcal{F}_{-n} = \sigma( X_{m} : m \leq -n )$.若

\[
\sum_{n=1}^{\infty} \alpha(n)^{\delta / 2(2+\delta)} < \infty
\]

则 $S_{n} = X_{1} + \cdots + X_{n}$ 满足 $S_{(n\cdot)} / \sqrt{n} \Rightarrow \sigma B(\cdot)$,其中

\[
\sigma^{2} = E X_{0}^{2} + 2 \sum_{n=1}^{\infty} E X_{0} X_{n}
\]

    \end{thm}
    
    
    
    \begin{dfn}
        [Definition-of-Strong-Mixing]
        {强混合性的定义}
        [Definition of Strong Mixing]
        [gpt-4.1]
        
设 $\bar{\alpha}(n) = \alpha( \mathcal{F}_{-n}, \mathcal{F}_{0}^{\prime} )$,其中 $\mathcal{F}_{0}^{\prime} = \sigma( X_{k}, k \geq 0 )$.当 $\bar{\alpha}(n) \downarrow 0$ 时,称该序列为强混合序列(strong mixing).

    \end{dfn}
    
    
    
    \begin{thm}
        [Central-Limit-Theorem-for-Strong-Mixing-Sequences-Ibragimov-Theorem]
        {强混合序列的中心极限定理(Ibragimov定理)}
        [Central Limit Theorem for Strong Mixing Sequences (Ibragimov Theorem)]
        [gpt-4.1]
        
Ibragimov (1962) 证明:若

\[
\sum_{n=1}^{\infty} \bar{\alpha}(n)^{\delta/(2+\delta)} < \infty
\]

则 $S_{n} / \sqrt{n} \Rightarrow \sigma \chi$,其中 $\sigma^{2} = \lim_{n \to \infty} E S_{n}^{2} / n$.

    \end{thm}
    
    
    
    \begin{thm}
        [Properties-of-Occupation-Time-of-Brownian-Motion-in-Different-Dimensions]
        {布朗运动占据时间在不同维数下的性质}
        [Properties of Occupation Time of Brownian Motion in Different Dimensions]
        [gpt-4.1]
        
对于任意 $x$:

\[
\begin{array}{lll}
P_x\left(\displaystyle \int_{0}^{\infty} 1_D(B_t)\,dt = \infty\right) = 1 \quad \text{当}~d \leq 2, \\
E_x\displaystyle \int_{0}^{\infty} 1_D(B_t)\,dt < \infty \quad \text{当}~d \geq 3.
\end{array}
\]

即:对于维数 $d \leq 2$,布朗运动在集合 $D$ 上的占据时间几乎处处是无穷大;而对于维数 $d \geq 3$,其期望占据时间是有限的.

    \end{thm}
    
    
    
    \begin{prf}
        [Proof-of-Properties-of-Occupation-Time-of-Brownian-Motion]
        {布朗运动占据时间性质的证明}
        [Proof of Properties of Occupation Time of Brownian Motion]
        [gpt-4.1]
        
令 $T_0 = 0$,$G = B(0, 2r)$.

对于 $k \geq 1$,定义
\[
\begin{array}{l}
S_k = \operatorname*{inf}\{t > T_{k-1} : B_t \in D\} \\
T_k = \operatorname*{inf}\{t > S_k : B_t \in G\}
\end{array}
\]

记 $\tau = T_1$,利用强马尔可夫性质,对于 $k \geq 1$ 有
\[
P_x\left(\int_{S_k}^{T_k} 1_D(B_t)\,dt \geq s \mid {\mathcal{F}}_{S_k}\right) = P_{B(S_k)}\left(\int_{0}^{\tau} 1_D(B_t)\,dt \geq s\right) = H(s)
\]

由强马尔可夫性质,$\int_{S_k}^{T_k} 1_D(B_t)\,dt$ 是一系列独立同分布的随机变量.

由于这些随机变量具有正的均值,由大数定律可得
\[
\int_{0}^{\infty} 1_D(B_t)\,dt \geq \lim_{n \to \infty} \sum_{k=1}^{n} \int_{S_k}^{T_k} 1_D(B_t)\,dt = \infty \quad \text{a.s.}
\]

从而证明了上述定理.

    \end{prf}
    
    
    
    \begin{thm}
        [Necessary-and-Sufficient-Condition-for-Positive-Recurrence-of-Markov-Chain]
        {马尔科夫链正再生的充分必要条件}
        [Necessary and Sufficient Condition for Positive Recurrence of Markov Chain]
        [gpt-4.1]
        
马尔科夫链满足 $\mu < 1$ 时,该链是正再生的(positive recurrent).
证明如下:

通过分析首次跃迁可得

\[
E_{0} T_{0} = 1 + \sum_{k = 1}^{\infty} a_{k} \frac{k - 1}{1 - \mu} = 1 + \frac{\mu - (1 - a_{0})}{1 - \mu} = \frac{a_{0}}{1 - \mu} < \infty
\]

这表明当 $\mu < 1$ 时,链是正再生的.

    \end{thm}
    
    
    
    \begin{crl}
        [Converse-Condition-for-Positive-Recurrence]
        {正再生的反面条件}
        [Converse Condition for Positive Recurrence]
        [gpt-4.1]
        
如果 $E_{0} T_{0} < \infty$,则对所有 $k$ 有 $E_{k} N < \infty$,且 $E_{k} N = c k$,其中 $c = 1/(1 - \mu)$.当 $\mu \geq 1$ 时,这不可能成立,因此链不是正再生的.

    \end{crl}
    
    
    
    \begin{thm}
        [Expectation-Formula-and-Determination-of-Constant]
        {期望公式及常数的确定}
        [Expectation Formula and Determination of Constant]
        [gpt-4.1]
        
设 $X_n$ 是每次最多递减 1 的过程,且 $x \geq 1$,则有 $E_x N = c x$,其中常数 $c = 1/(1 - \mu)$.
证明如下:

首先有
\[
x = E_x X_{N \wedge n} + (1 - \mu) E_x (N \wedge n) \geq (1 - \mu) E_x (N \wedge n)
\]
由于 $X_{N \wedge n} \geq 0$,故 $E_x N \leq x/(1 - \mu)$

为证等号,注意 $E_x N = c x$,通过
\[
E_1 N = 1 + \sum_{k = 0}^{\infty} a_k E_k N
\]
可得 $c = 1 + \mu c$,解得 $c = 1/(1 - \mu)$.

    \end{thm}
    
    
    
    \begin{dfn}
        [Transition-Probability-of-Markov-Chain]
        {马尔科夫链状态转移概率}
        [Transition Probability of Markov Chain]
        [gpt-4.1]
        
若 $X_0 = 0$,则 $p(0, 0) = a_0 + a_1$,且对于 $k \geq 2$,$p(0, k-1) = a_k$.

    \end{dfn}
    
    
    
    \begin{thm}
        [C^{12}-Function-Satisfies-Partial-Differential-Equation]
        {C^{1,2}函数满足偏微分方程}
        [C^{1,2} Function Satisfies Partial Differential Equation]
        [gpt-4.1]
        如果 $
u \in C^{1,2}$,那么它满足条件 (a).
    \end{thm}
    
    
    
    \begin{prf}
        [Proof-that-C^{12}-Function-Satisfies-Partial-Differential-Equation]
        {C^{1,2}函数满足偏微分方程的证明}
        [Proof that C^{1,2} Function Satisfies Partial Differential Equation]
        [gpt-4.1]
        当我们用 (9.4.2) 中的解时,可以考虑从时间 0 到 $t$ 运行时空布朗运动 $(t-s, B_s)$,因此有
\[
u ( t + h , x ) = E _ { x } [ 
u ( t , B _ { h } ) \exp ( c _ { h } ) ]
\]

从两边减去 $
u ( t , x )$ 得

\[
u ( t + h , x ) - 
u ( t , x ) = E _ { x } [ 
u ( t , B _ { h } ) \exp ( c _ { h } ) - 
u ( t , x ) ]
\]

在右侧加减 $
u ( t , B _ { h } )$ 并用泰勒定理

\[
\begin{array}{l}
\displaystyle 
u ( t , B _ { h } ) \exp ( c _ { h } ) - 
u ( t , x ) \approx \sum_{i=1}^{d} \frac{\partial 
u}{\partial x_i}(t,x) (B_h^i - x) \\
\displaystyle + \frac{1}{2} \sum_{1 \leq i,j \leq d} \frac{\partial^2 
u}{\partial x_i \partial x_j} (t,x) (B_h^i - x)(B_h^j - x) + 
u(t,B_h)(\exp(c_h)-1) + o(h)
\end{array}
\]

由于当 $i 
eq j$ 时,$W_i(t)$ 和 $W_i^2(t) = t$ 以及 $W_i(t) W_j(t)$ 是鞅,

\[
E _ { x } [ 
u ( t , B _ { h } ) \exp ( c _ { h } ) - 
u ( t , x ) ] \approx \sum_{i=1}^d \frac{\partial^2 
u}{\partial x_i^2} h + E_x [ 
u(t, B_h)(\exp(c_h)-1) ] + o(h)
\]

两边同时除以 $h$ 并令 $h \to 0$,得到

\[
\frac{\partial 
u}{\partial t} = \frac{1}{2} \Delta 
u ( t , x ) + c ( x ) 
u ( t , x )
\]

从而证明了所需结论.
    \end{prf}
    
    
    
    \begin{ppt}
        [Conditional-Distribution-Formula-for-Brownian-Bridge]
        {布朗桥的条件分布公式}
        [Conditional Distribution Formula for Brownian Bridge]
        [gpt-4.1]
        设 $X = B_s^0$,$Y = B_t^0$,则 $\sigma_X = \sqrt{s(1-s)}$,$\sigma_Y = \sqrt{t(1-t)}$,且

\[
\rho = \frac{s(1-t)}{\sqrt{s(1-s)t(1-t)}} = \sqrt{\frac{s(1-t)}{(1-s)t}}
\]

则有

\[
1 - \rho^2 = \frac{t-s}{(1-s)t}
\]

且条件分布

\[
(Y|X=x) = \mathrm{normal}\left( x\rho\frac{\sigma_Y}{\sigma_X},\ (1-\rho^2)\sigma_Y^2 \right)
\]

在本特例中,得到

\[
(B_t^0 | B_s^0 = x) = \mathrm{normal} \left( \frac{x(1-t)}{1-s},\ (1-t)\frac{t-s}{1-s} \right)
\]

    \end{ppt}
    
    
    
    \begin{thm}
        [Existence-of-$\sigma$-Finite-Stationary-Measure-in-the-Recurrent-Case]
        {常返情形下的$\sigma$-有限不变测度存在性}
        [Existence of $\sigma$-Finite Stationary Measure in the Recurrent Case]
        [gpt-4.1]
        
在常返(recurrent)情形下,存在一个$\sigma$-有限的、不变的测度$\bar{\mu}$,且$\bar{\mu} \ll \lambda$.

    \end{thm}
    
    
    
    \begin{prf}
        [Proof-of-Existence-of-$\sigma$-Finite-Stationary-Measure-in-the-Recurrent-Case]
        {常返情形下$\sigma$-有限不变测度的构造证明}
        [Proof of Existence of $\sigma$-Finite Stationary Measure in the Recurrent Case]
        [gpt-4.1]
        
令$R = \inf\{ n \geq 1 : \bar{X}_n = \alpha \}$,定义

\[
\bar{\mu}(C) = E_\alpha\left( \sum_{n=0}^{R-1} 1_{\{ \bar{X}_n \in C \}} \right) = \sum_{n=0}^\infty P_\alpha(\bar{X}_n \in C, R > n)
\]

重复定理5.5.7的证明可知$\bar{\mu} \bar{p} = \bar{\mu}$.

若令$\mu = \bar{\mu} 
u$,由引理5.8.4可知$\bar{\mu} 
u p = \bar{\mu} \bar{p} 
u = \bar{\mu} 
u$,因此$\mu p = \mu$.

    \end{prf}
    
    

    \section{附录}
    本文档由脚本自动生成,供 Fulcrum 数学知识条目测试使用。

\end{document}
