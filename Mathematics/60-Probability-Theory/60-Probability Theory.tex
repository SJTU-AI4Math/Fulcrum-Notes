\documentclass[UTF8]{ctexart}

\makeatletter
\def\input@path{{../../Fulcrum-Template/}{../../Operator-List/}}
\makeatother

\usepackage{FulcrumCN}
\usepackage{OperatorListCN}
\usepackage{F4Logic}
\usepackage{F4Set}
\usepackage{F4Analysis}
\usepackage{F4Probabilities}
\usepackage{F4Algebra}

% margin
\usepackage{geometry}
\geometry{
    paper =a4paper,
    top =3cm,
    bottom =3cm,
    left=2cm,
    right =2cm
}
\linespread{1.2}

\DeclareMathOperator{\st}{\text{s.t. }}
\DeclareMathOperator{\Poi}{Poi}
\newcommand{\bysorry}{\fcolorbox{black}{gray!20!white}{\texttt{\textcolor{black}{:= }\textcolor{blue}{by }\textcolor{red}{\textbf{sorry}}}}\;\;\,\texttt{\textcolor{gray}{{\footnotesize declaration uses 'sorry'.}}}}

\begin{document}

\tableofcontents
\newpage

\section{概率空间}

    \subsection{样本空间与事件}

        \begin{dfn}
            [Sample-Space]
            {样本空间}
            []
            [Dedicatia, 猫猫]
            设 \(S\) 是类型, 定义\textbf{样本空间} 为 \(S\) 上的有限集. 

            设 \(\Omega\) 是 \(S\) 上的样本空间, \(x:S\), 定义 \(x\) 是 \(\Omega\) 中的\textbf{样本点}, 当且仅当: \(x\in\Omega\). 
        \end{dfn}

        \begin{rmk}
            [猫猫]
            通常使用 \(\Omega\) 表示\样本空间. 
        \end{rmk}

        \begin{dfn}
            [Event]
            {事件}
            [Event]
            [Dedicatia, 猫猫]
            设 \(\Omega\) 是样本空间, \(S\) 是集合, 定义 \(S\) 是 \(\Omega\) 中的\textbf{事件}当且仅当: \(S\subseteq\Omega\). 
        \end{dfn}

        \begin{rmk}
            [猫猫]
            在概率问题语境中指定\样本空间 后, 不加声明地使用\事件 的概念. 
        \end{rmk}

        \begin{rmk}
            [猫猫]
            在不致混淆的前提下, 将\事件 的交 \(A\cap B\) 简记为 \(AB\). 

            除此之外, \事件 的运算及其性质与\集合 完全相同. 
        \end{rmk}

        \begin{dfn}
            []
            {决定性事件}
            []
            [Dedicatia, 猫猫]
            设 \(\Omega\) 是样本空间, \(S\) 是 \(\Omega\) 中的随机事件, 

            定义 \(S\) 是 \(\Omega\) 中的\textbf{必然事件}当且仅当: \(S=\Omega\); 

            定义 \(S\) 是 \(\Omega\) 中的\textbf{不可能事件}当且仅当: \(S=\varnothing\); 

            定义 \(S\) 是 \(\Omega\) 中的\textbf{决定性事件}当且仅当: \(S=\Omega\lor S=\varnothing\). 
        \end{dfn}

        \begin{dfn}
            []
            {对立事件 / 逆事件}
            []
            [Dedicatia]
            设 \(A\) 是事件, 定义 \(A\) 的\textbf{对立事件} 为 \(A^{\mathrm{C}}\), 记为 \(\overline{A}\). 
        \end{dfn}

        \begin{dfn}
            []
            {互斥事件}
            []
            [Dedicatia, 猫猫]
            设 \(A, B\) 是事件, 定义 \(A\) 与 \(B\) \textbf{互斥}当且仅当: \(A\cap B=\varnothing\). 
        \end{dfn}

        \begin{dfn}
            []
            {和事件}
            []
            [Dedicatia, 猫猫]
            设 \(A, B\) 是事件, \(A\) 与 \(B\) 互斥, 定义 \(A\) 与 \(B\) 的\textbf{和事件}为 \(A\cup B\), 记为 \(A+B\). 
        \end{dfn}

    \subsection{古典概率}

        \begin{dfn}
            [Classic-Probability]
            {古典概率}
            [Classic Probability]
            [Dedicatia, 猫猫]
            设 \(\Omega\) 是\样本空间, \(S\) 是\事件, 定义 \(S\) 在 \(\Omega\) 中的\textbf{概率}为: \(\dfrac{\card[S]}{\card[\Omega]}\), 记作 \(\ProbClsc[\Omega]{S}\). 
        \end{dfn}

        \begin{rmk}
            [猫猫]
            在语境中已指定\样本空间 后, 不加声明地对\事件 讨论概率, 记作 \(\ProbClsc{S}\). 
        \end{rmk}

        \begin{ppt}
            []
            {}
            []
            [Dedicatia]
            设 \(A, B\) 是\事件, 则: 
            \[P(A\cup B)=P(A)+P(B)-P(AB)\]
            
            设 \(A\) 与 \(B\) 互斥, 则: 
            \[P(A+B)=P(A)+P(B)\]
        \end{ppt}
    
    \subsection{排列组合}

        \begin{dfn}
            [Permutation]
            {排列数}
            []
            [Dedicatia]
            设 \(n, m:\N\), \(m<n\), 定义从 \(n\) 中选 \(m\) 的\textbf{排列数}为: \(\dfrac{n!}{(n-m)!}\), 记作 \(\Perm{n}{m}\) 或 \(\mathrm{A}_n^m\). 
        \end{dfn}

        \begin{ppt}
            []
            {全排列数等于阶乘}
            []
            [猫猫]
            设 \(n:\N\), 则: \(\Perm{n}{n}=n!\). 
        \end{ppt}

        \begin{ppt}
            {排列数的定义动机}
            设 \(S\) 是\集合, \(S\) 是有限集, \(n:\N:=\card[S]\), \(m:\N\), \(m<n\), 则: 
            \[\card[\{f:\{1,\dots,m\}\to S|\单射[f]\}]=\Perm{n}{m}\]
        \end{ppt}

        \begin{prf}
            从 \(n\) 个不同元素 (无放回地) 中依次取出 \(m\) 个元素并排成一列, 称为从 \(n\) 个元素中取出 \(m\) 个元素的排列, 这样的排列的种数记作 \(A_n^m\) , 有\[A_n^m=n(n-1)(n-2)\cdots(n-m+1)=\frac{n!}{(n-m)!}.\]随机抽取时, 得到的各种排列是等可能的. 
        \end{prf}

        \begin{dfn}
            [Combination]
            {组合数}
            [Combination]
            [猫猫]
            设 \(n, m:\N\), 定义 从 \(n\) 中选 \(m\) 的\textbf{组合数}为: 
            \[
            \begin{cases}
            \begin{aligned}
                & \frac{n!}{m!(n-m)!} & n\geq m \\
                & 0 & n<m
            \end{aligned}
            \end{cases}
            \]
            
            记作 \(\binom{n}{m}\) 或 \(\Comb{n}{m}\). 
            
        \end{dfn}

        \begin{ppt}
            []
            {组合数的定义动机}
            []
            [Dedicatia, 猫猫]
            设 \(S\) 是\集合, \(S\) 是有限集, \(n:\N:=\card[S]\), \(m:\N\), 则: 
            \[\card\{X|X\subseteq S\land\card[X]=m\}=\binom{n}{m}\]
        \end{ppt}

        \begin{prf}
            从 \(n\) 个不同元素 (无放回地) 中依次取出 \(m\) 个元素, 不考虑元素的排列顺序, 称为从 \(n\) 个元素中取出 \(m\) 个元素的组合, 这样的组合的种数记作 \(C_n^m\) , 有: 
            \[C_n^m=\binom{n}{m}=\frac{A_n^m}{m!}=\frac{n!}{m!(n-m)!}\]
        \end{prf}

        \begin{ppt}
            []
            {杨辉三角}
            [Pascal's Triangle]
            [猫猫]
            设 \(n, m:\N\), 则:
            \begin{enumerate}
                \item 杨辉三角左右对称: 
                    \[\binom{n}{m}=\binom{n}{n-m}\]
                \item 杨辉三角的边界总是 \(1\): 
                    \[\binom{n}{0}=\binom{n}{n}=1\]
                \item 杨辉三角中每个组合数等于上方两数之和: 
                    \[\binom{n}{m}=\binom{n-1}{m-1}+\binom{n-1}{m}\]
                \item 杨辉三角中每行之和为 \(2\) 的幂次:
                    \[\sum_{m=0}^{n}\binom{n}{m}=2^n\]
            \end{enumerate}
        \end{ppt}

        \begin{ppt}
            []
            {组合数对角和公式}
            []
            [猫猫]
            设 \(m, n:\N\), 则:
            \[\sum_{i=0}^{m}\binom{n+i}{i}=\binom{n+m+1}{m}\]
        \end{ppt}

        \begin{rmk}
            [猫猫]
            上述和式实际上可以视为杨辉三角中一个对角线上元素的求和. 
        \end{rmk}

        \begin{xmp}
            []
            {}
            []
            [Dedicatia]
            将 \(n\) 个不同的元素分成 \(r\) 组, 每组的元素个数分别为 \(n_1,n_2,\cdots,n_r\) , 且 \(n_1+n_2+\cdots+n_r=n\), 这样的分组的种数记作 \(P(n_1,n_2,\cdots,n_r)\) , 有
            \[\binom{n}{n_1,n_2,\cdots,n_r}=\frac{n!}{n_1!n_2!\cdots n_r!}\]
        \end{xmp}
    
    % \subsection{几何概型}
        
        % \begin{dfn}
        %     []
        %     {几何概型}
        %     []
        %     [Dedicatia]
        %     设样本空间 \(\Omega\) 的体积 \(m(\Omega)\) 是正数, 样本点等可能地分布在 \(\Omega\) 中, 且事件 \(A\) 是 \(\Omega\) 的一个子集, 且 \(A\) 的体积是 \(m(A)\) , 则称这种试验为\textbf{几何概率模型}, 简称为\textbf{几何概型}. 事件 \(A\) 发生的概率为\[P(A)=\frac{m(A)}{m(\Omega)}.\]
        % \end{dfn}

        % \begin{dfn}
        %     []
        %     {事件域、可测空间}
        %     []
        %     [Dedicatia]
        %     设 \(\Omega\) 是一个集合,  \(\mathcal{F}\) 是 \(\Omega\) 的一个子集族, 若 \(\mathcal{F}\) 满足: 

        %     \begin{enumerate}
        %         \item  \(\Omega\in\mathcal{F}\) ;
        %         \item 若 \(A\in\mathcal{F}\) , 则 \(\bar{A}\in\mathcal{F}\) ;
        %         \item 若 \(A_1,A_2,\cdots\in\mathcal{F}\) , 则 \(\bigcup_{i=1}^{\infty}A_i\in\mathcal{F}\) . 
        %     \end{enumerate}

        %     则称 \(\mathcal{F}\) 是 \(\Omega\) 的一个\textbf{事件域}, 称 \(\mathcal{F}\) 中的元素为事件, 称 \((\Omega,\mathcal{F})\) 为\textbf{可测空间}. 
        % \end{dfn}

        % \begin{dfn}
        %     []
        %     {Borel事件域}
        %     []
        %     [Dedicatia]
        %     设 \(\Omega\) 是 \(n\) 维的实数空间 \(\R^n\) ,  \(\mathcal{B}\) 是 \(\R^n\) 的一个子集族, 若 \(\mathcal{B}\) 满足: 

        %     \begin{enumerate}
        %         \item  \(\R^n\in\mathcal{B}\) ;
        %         \item 若 \(A\in\mathcal{B}\) , 则 \(\bar{A}\in\mathcal{B}\) ;
        %         \item 若 \(A_1,A_2,\cdots\in\mathcal{B}\) , 则 \(\bigcup_{i=1}^{\infty}A_i\in\mathcal{B}\) ;
        %         \item 若 \(A_1,A_2,\cdots\in\mathcal{B}\) , 则 \(\bigcap_{i=1}^{\infty}A_i\in\mathcal{B}\) . 
        %     \end{enumerate}
        %     就称 \(\mathcal{F}\) 是 \(\R\) 的一个\textbf{Borel事件域}. 
        % \end{dfn}

        % \begin{dfn}
        %     []
        %     {测度}
        %     []
        %     [Dedicatia]
        %     称 \(\mathcal{F}\) 上的函数 \(m\) 是 \(\Omega\) 上的一个\textbf{测度}, 若 \(m\) 满足: 

        %     \begin{enumerate}
        %         \item 非负性: 对于任意 \(A\in\mathcal{F}\) , 有 \(m(A)\geq 0\) ;
        %         \item 空间的可加性: 若 \(A_1,A_2,\cdots\in\mathcal{F}\) 两两互不相容, 则有\[m\left(\bigcup_{i=1}^{\infty}A_i\right)=\sum_{i=1}^{\infty}m(A_i).\]
        %     \end{enumerate}
        % \end{dfn}

        % \begin{dfn}
        %     []
        %     {体积}
        %     []
        %     [Dedicatia]
        %     设有 \(r\) 维向量空间 \(\mathbb{F}^r=\left\{(x_1,x_2,\cdots,x_i,\cdots,x_r)|x_i\in\mathbb{F}\right\} \) , 对于 \(\mathbb{F}^r\) 的任意子集 \(A\) , 若存在一个测度 \(m\) , 使得 \(m(A)\) 满足: 

        %     \[m(A)=\int_{A}\dd x_1\dd x_2\cdots\dd x_r\]

        %     则称这个测度为体积. 
        % \end{dfn}

    \subsection{概率空间}
        
        \begin{dfn}
            [Probability-Measure]
            {概率测度}
            [Probability Measure]
            [猫猫]
            设 \(\Omega\) 是类型, \(\mathscr{F}\) 是 \(\Omega\) 上的 \Sigma代数, \(\mathbb{P}:\mathscr{F}\to\Icc{0}{1}\), 定义 \(\mathbb{P}\) 为 \(\mathscr{F}\) 上的\textbf{概率测度} 当且仅当: 
            \begin{enumerate}
                \item \(\mathbb{P}\) 是 \(\mathscr{F}\) 上的\测度; 
                \item \textbf{规范性}: 
                    \[\mathbb{P}(\Omega)=1\]
            \end{enumerate}
        \end{dfn}

        \begin{str}
            [Probability-Space]
            {概率空间}
            [Probability-Space]
            [Dedicatia, 猫猫]
            定义上的\textbf{概率空间}包含以下信息: 
            \begin{enumerate}
                \item \textbf{样本空间}: \(\Omega\); 
                
                \item \textbf{事件域}: \(\mathscr{F}\); 
                    
                    \Sigma代数[\(\mathscr{F}\)][\(\Omega\)]; 

                \item \textbf{概率测度}: \(\mathbb{P}: \mathscr{F}\to\Icc{0}{1}\); 
                
                    \(\mathbb{P}\) 是 \(\mathscr{F}\) 上的\概率测度;
            \end{enumerate}
        \end{str}

        \begin{ppt}
            []
            {空集概率为 \(0\)}
            []
            [Dedicatia, 猫猫]
            设 \((\Omega,\mathscr{F},\mathbb{P})\) 是\概率空间, 则: \(\mathbb{P}(\varnothing)=0\). 
        \end{ppt}

        \begin{ppt}
            []
            {互补集概率和为 \(1\)}
            []
            [Dedicatia, 猫猫]
            设 \((\Omega,\mathscr{F},\mathbb{P})\) 是\概率空间, 则: \(\forall A\in\mathscr{F}, \mathbb{P}(A)+\mathbb{P}(\bar{A})=1\). 
        \end{ppt}

        \begin{thm}
            []
            {概率的 Jordan 公式 / 容斥原理}
            []
            [Dedicatia]
            设 \(A_1,A_2,\cdots,A_n\) 是事件, 设\[p_k=\sum_{1\leq j_1\leq\cdots\leq j_k  }P(A_i A_2\cdots A_k)\], 则有
            \[P(A_1\cup A_2\cup\cdots\cup A_n)=\sum_{k=1}^{n}(-1)^{k-1}p_k.\]
        \end{thm}

        \begin{cxmp}
            []
            {Bertrand 悖论}
            []
            [Dedicatia]
        \end{cxmp}

        \begin{dfn}
            []
            {概率极限}
            []
            [Dedicatia]
            设 \((\Omega,\mathscr{F},P)\) 是\概率空间, \(\{A_n\}:\N\to\mathscr{F}\), \(A\in\mathscr{F}\), \(A_n\downarrow\lor A_n\uparrow\), 定义 \(A_n\) 的\textbf{概率极限}为: \(P\left(\bigcup\limits_{i=1}^{\infty}A_i\right)\), 记作 \(\lim\limits_{n\to+\infty}P(A_n)\)
        \end{dfn}
        
        % \begin{thm}
        %     []
        %     {概率的可列可加性定理}
        %     []
        %     [Dedicatia]
        %     设 \(P\) 是 \((\Omega,\mathcal{F})\) 上的一个概率, 则它具备可列可加性的充要条件为: 它是连续且有限可加的. 
        % \end{thm}

    \subsection{条件概率}

        \begin{dfn}
            [Condition-Probability]
            {条件概率}
            [Condition-Probability]
            [Dedicatia, 猫猫]
            设 \((\Omega,\mathscr{F},P)\) 是\概率空间, \(A, B\in\mathscr{F}\), \(P(B)>0\), 定义在条件 \(B\) 下 \(A\) 的\textbf{条件概率}为 \(\dfrac{P(AB)}{P(B)}\), 记作 \(P(A|B)\). 
        \end{dfn}

        \begin{thm}
            []
            {全概率公式}
            []
            [Dedicatia, 猫猫]
            设 \((\Omega,\mathscr{F},P)\) 是\概率空间, \(\{B_n\}:\N\mapsto\mathscr{F}\), \(\forall i,j:\N, i\neq j\implies B_i\cap B_j=\varnothing\), \(\bigcup_{i=1}^{+\infty}B_i=\Omega\), 则: 
            \[\forall A\in\mathscr{F}, P(A)=\sum_{i=1}^{+\infty}P(A|B_i)P(B_i)\]
        \end{thm}

        \begin{prf}
            由于 \(B_i\) 两两互不相容, 所以\[A=\sum_{i=1}^{k}B_iA\]
            由可列可加性可得\[P(A)=\sum_{i=1}^{k}P(B_iA)=\sum_{i=1}^{k}P(A|B_i)P(B_i).\]
        \end{prf}

        \begin{thm}
            []
            {贝叶斯公式}
            []
            [Dedicatia]
            设事件 \(B_1,B_2,\cdots,B_k\) 是一组互不相容的事件, 且 \(\bigcup_{i=1}^{k}B_i=\Omega\) , 则对于任意事件 \(A\) , 有\[P(B_i|A)=\frac{P(A|B_i)P(B_i)}{\sum_{j=1}^{k}P(A|B_j)P(B_j)}.\]
        \end{thm}

        \begin{prf}
            使用全概率公式即可证明. 
        \end{prf}

    \subsection{事件的独立性}

        \begin{dfn}
            []
            {事件的独立性}
            []
            [Dedicatia, 猫猫]
            设 \((\Omega,\mathscr{F},\mathbb{P})\) 是\概率空间, \(A,B\in\mathscr{F}\), 定义 \(A\) 与 \(B\) \textbf{独立}, 当且仅当: 
            \[P(AB)=P(A)P(B)\]
        \end{dfn}

        \begin{dfn}
            []
            {独立事件列}
            []
            [Dedicatia]
            设事件 \(A_1,A_2,\cdots,A_n\) , 若对于任意的 \(1\leq i_1\leq i_2\leq\cdots\leq i_k\) , 有\[P(A_{i_1}A_{i_2}\cdots A_{i_k})=P(A_{i_1})P(A_{i_2})\cdots P(A_{i_k})\], 则称事件 \(A_1,A_2,\cdots,A_n\) 是\textbf{独立事件列}. 
        \end{dfn}

        \begin{ppt}
            []
            {}
            []
            [Dedicatia]
            设 \(A_1,A_2,\cdots,A_n\) 是独立事件列, 用 \(B_i\) 表示 \(A_i\) 或 \(\bar{A}_i\) , 则 \(B_1,B_2,\cdots,B_n\) 也是独立事件列. 
        \end{ppt}

\section{随机变量与概率分布}

    \subsection{随机变量}

        \begin{dfn}
            [Random-Variable]
            {随机变量}
            [Random Variable]
            [Dedicatia, 猫猫]
            设 \((\Omega,\mathscr{F},\mathbb{P})\) 是\概率空间, \(X:\Omega\to\R\), 定义 \(X\) 是 \((\Omega,\mathscr{F},P)\) 上的\textbf{随机变量}, 当且仅当: 
            \[\forall B\in\Borel{\R}, X^{-1}(B)\in\mathscr{F}\]
            
            设 \(P\) 是 \(\R\) 上的一元谓词, 将 \(\mathbb{P}(\{\omega:\Omega|P(X(\omega))\})\) 记作 \(\mathbb{P}(P(X))\). 
        \end{dfn}

        \begin{rmk}
            [猫猫]
            \随机变量 是将样本空间上的实值函数, 可测集的原像是事件. 
            
            在语境中已指定\概率空间 后, 不加声明地使用\随机变量 的概念. 
        \end{rmk}

        \begin{ppt}
            []
            {随机变量的实数身份}
            []
            [猫猫]
            设 \((\Omega,\mathscr{F},\mathbb{P})\) 是\概率空间, 则: \((\Omega,\mathscr{F},\mathbb{P})\) 上的\随机变量 具有以下结构: 
            \begin{enumerate}
                \item 交换代数; 
                \item \(L^p\) 空间; 
                \item 序. 
            \end{enumerate}
        \end{ppt}
        
        \begin{rmk}
            [猫猫]
            后续要说明多大程度上随机变量作为函数空间具有的结构与实数的等同的. 

            后面方便起见, 在不引起混淆的前提下, 允许自然地对随机变量应用实定义函数. 
        \end{rmk}

        \begin{dfn}
            [Distribution-Function]
            {分布函数}
            [Distribution Function]
            [Dedicatia, 猫猫]
            设 \((\Omega,\mathscr{F},\mathbb{P})\) 是\概率空间, \(X\) 是\随机变量, 定义 \(X\) 在 \((\Omega,\mathscr{F},\mathbb{P})\) 中的\textbf{分布函数}为: 
            \[x:\R\mapsto\mathbb{P}(X\leq x)\]
        \end{dfn}

        \begin{ppt}
            []
            {分布函数性质}
            []
            [Dedicatia, 猫猫]
            设 \((\Omega,\mathscr{F},\mathbb{P})\) 是\概率空间, \(X\) 是\随机变量, \(F\) 是 \(X\) 的分布函数, 则: 
            \begin{enumerate}
                \item 单调性:  \(F\) 在 \(\R\) 上单调增加的; 
                \item 有界性:  \(F(\R)\subseteq\Icc{0}{1}\); 
                \item 右连续性:  \(F\) 在 \(\R\) 上右连续;
                \item 左极限存在性: \(F\) 在 \(\R\) 上左极限存在; 
                \item 规范性:  
                \[
                \begin{cases}
                    \lim\limits_{x\to-\infty}F(x)=0\\
                    \lim\limits_{x\to+\infty}F(x)=1
                \end{cases}
                \]
            \end{enumerate}
        \end{ppt}

        \begin{dfn}
            [Expectation]
            {平均值 / 均值 / 数学期望 / Riemann-Stieltjes 积分 / 一阶(原点)矩}
            [Mean / Expectation / First Moment]
            [猫猫, Dedicatia]
            设 \((\Omega,\mathscr{F},\mathbb{P})\) 是\概率空间, \(X\) 是\随机变量, \(F\) 是 \(X\) 的分布函数, 定义 \(X\) 在 \((\Omega,\mathscr{F},\mathbb{P})\) 中的\textbf{数学期望}为: 
            \[\int_{-\infty}^{+\infty}x\dd F(x)\]

            记作 \(\Expct{X}\). 
        \end{dfn}

        \begin{ppt}
            []
            {数学期望的线性性}
            []
            [猫猫]
            设 \(X,Y\) 是\离散型随机变量, 则: 
            \[\forall\alpha,\beta:\R, \mathbb{E}(\alpha X+\beta Y)=\alpha\mathbb{E}(X)+\beta\mathbb{E}(Y)\]
        \end{ppt}

        \begin{ppt}
            []
            {复合数学期望的计算}
            []
            [Dedicatia]
            设 \((\Omega,\mathscr{F},\mathbb{P})\) 是\概率空间, \(X\) 是\离散型随机变量, 取值 \(x_i:1\leq i\leq n\), \(p(x_i)=P(X=x_i)\), $g:\R\to\R$, 那么
            \[\Expct{g(X)}=\sum_{i=1}^n g(x_i)p(x_i)\]
        \end{ppt}

        \begin{dfn}
            []
            {\(n\) 阶原点矩}
            [Raw Moment]
            [Dedicatia]
            设 \((\Omega,\mathscr{F},\mathbb{P})\) 是\概率空间, \(X\) 是\随机变量, \(n:\N^+\), 定义 \(X\) 的 $n$ 阶原点矩为 \(\Expct{X^n}\).
        \end{dfn}

        \begin{dfn}
            []
            {\(n\) 阶中心矩}
            [Central Moment]
            [Dedicatia]
            设 \((\Omega,\mathscr{F},\mathbb{P})\) 是\概率空间, \(X\) 是\随机变量, \(n:\N^+\), 定义 \(X\) 的 \(n\) 阶中心矩为 \(\Expct{(X-\Expct{X})^n}\).
        \end{dfn}

        \begin{rmk}
            [Dedicatia]
            矩这个命名可能源于物理. 例如力矩表示力乘以距离, 概率论中的矩表示概率分布情况关于某个距离的幂次.\\
            一阶原点矩就是期望; 一阶中心矩恒为0, 没有描述意义;\\
            二阶中心矩就是方差, 标准化后的二阶中心矩就是标准差; 同样, 标准化后的三阶中心矩就是偏度; 标准化后的四阶中心矩就是标准峰度.
        \end{rmk}

        \begin{dfn}
            [Variance]
            {方差与标准差}
            [Variance \& Standard Deviation]
            [猫猫]
            设 \((\Omega,\mathscr{F},\mathbb{P})\) 是\概率空间, \(X\) 是\随机变量, 定义 \(X\) 在 \((\Omega,\mathscr{F},\mathbb{P})\) 中的\textbf{方差}为: 
            \[\Expct{(X-\Expct{X})^2}\]

            记作 \(\Var{X}\).

            定义 \(X\) 在 \((\Omega,\mathscr{F},\mathbb{P})\) 中的\textbf{标准差}为: \(\sqrt{\Var{X}}\). 
        \end{dfn}

        \begin{ppt}
            [Variance2]
            {方差的计算公式}
            []
            [Dedicatia]
            设 \((\Omega,\mathscr{F},\mathbb{P})\) 是\概率空间, \(X\) 是\随机变量, 则
            \[\Var{X}=\Expct{X^2}-(\Expct{X})^2.\]
        \end{ppt}

        \begin{ppt}
            []
            {方差与线性运算}
            []
            [Dedicatia]
            设 \((\Omega,\mathscr{F},\mathbb{P})\) 是\概率空间, \(X\) 是\随机变量, \(a,b:\R\), 则: 
            \[\Var{aX+b}=a^2\Var{X}\]
        \end{ppt}

        \begin{rmk}
            [猫猫]
            方差衡量随机变量 \(X\) 偏离其数学期望 \(\Expct{X}\) 的程度. 
        \end{rmk}

        \begin{thm}
            []
            {Markov 不等式}
            []
            [Dedicatia]
            设 \((\Omega,\mathscr{F},\mathbb{P})\) 是\概率空间, \(X\) 是\随机变量, 且 \(X\geq 0\), 则对任意 \(\varepsilon>0\): 
            \[\mathbb{P}(X\geq\varepsilon)\leq\frac{1}{\varepsilon^\alpha}\Expct{|X|^\alpha},\quad \alpha>0\]
        \end{thm}

        \begin{thm}
            []
            {Chebyshev 不等式}
            []
            [Dedicatia]
            在 \(\alpha=2\) 时的 Markov 不等式 称为\textbf{Chebyshev 不等式}: 设 \((\Omega,\mathscr{F},\mathbb{P})\) 是\概率空间, \(X\) 是\随机变量, 则对任意 \(\varepsilon>0\): 
            \[\mathbb{P}(|X-\Expct{X}|\geq\varepsilon)\leq\frac{\Var{X}}{\varepsilon^2}.\]
        \end{thm}

    \subsection{离散型随机变量}

        \begin{dfn}
            [Discrete-Random-Variable]
            {离散型随机变量}
            [Discrete Random Variable]
            [Dedicatia, 猫猫]
            设 \((\Omega,\mathscr{F},\mathbb{P})\) 是\概率空间, \(X\) 是\随机变量, 定义 \(X\) 是\textbf{离散型随机变量}, 当且仅当: 
            \[\card(X(\R))\leq\aleph_0\]
        \end{dfn}

        \begin{ppt}
            []
            {离散型随机变量的数学期望}
            []
            [猫猫]
            设 \(X\) 是\离散型随机变量, 则: 
            \[\Expct{X}=\sum_{x_i\in X(\Omega)}x_i\mathbb{P}(X=x_i)\]
        \end{ppt}
        
        \begin{ppt}
            []
            {离散型随机变量的函数}
            []
            [猫猫]
            设 \(X\) 是\离散型随机变量, \(f:\R\to\R\), 则: 
            \[\forall y:\R, \mathbb{P}(f(X)=y)=\sum_{x_i\in f^{-1}(y)}\mathbb{P}(X=x_i)\]
        \end{ppt}

        \begin{dfn}
            [Discrete-Distribution]
            {分布列}
            [Discrete-Distribution]
            [Dedicatia]
            设 \((\Omega,\mathscr{F},\mathbb{P})\) 是\概率空间, \(X\) 是\离散型随机变量, \(\{x_i\}_{i:\N}:=X(\mathscr{F})\), 定义 \(X\) 的\textbf{分布列}为: 
            \[x_i\mapsto\mathbb{P}(X=x_i)\]

            设 \(p_i := \mathbb{P}(X=x_i)\), 记作: 
            \begin{center}
            \begin{tabular}{c|c|c|c|c|c}
                \(X\) & \(x_1\) & \(x_2\) & \(\cdots\) & \(x_n\) & \(\cdots\) \\ \hline
                \(\mathbb{P}\) & \(p_1\) & \(p_2\) & \(\cdots\) & \(p_n\) & \(\cdots\)
            \end{tabular}
            \end{center}
        \end{dfn}
        
        \begin{ppt}
            []
            {分布列的覆盖性}
            []
            [猫猫]
            设 \((\Omega,\mathscr{F},\mathbb{P})\) 是\概率空间, \(X\) 是\离散型随机变量, 则: 
            \[\sum_{X(\Omega)}\mathbb{P}(X)=1\]
        \end{ppt}

    \subsection{常见离散分布}

        \begin{xmp}
            []
            {Bernoulli 分布 / 两点分布}
            [Bernoulli Distribution]
            [猫猫]
            设 \(X\) 是\离散型随机变量, \(p:\R\), \(p\in\Ioo{0}{1}\), 定义 \(X\) 服从成功率为 \(p\) 的\textbf{Bernoulli分布}, 当且仅当: 
            \begin{enumerate}
                \item \(X\) 为二元取值: 
                    \[X(\mathscr{F})=\{0,1\}\]
                \item 成功率为 \(p\): 
                    \[\mathbb{P}(X=1)=p\]
            \end{enumerate}

            记作 \(X\sim B(1,p)\). 
        \end{xmp}

        \begin{ppt}
            []
            {Bernoulli 分布的分布列}
            []
            [猫猫]
            设 \(X\) 是\离散型随机变量, \(p:\R\), \(p\in\Ioo{0}{1}\), \(X\sim B(1,p)\) 则 \(X\) 的\分布列 为: 
            \begin{center}
            \begin{tabular}{c|c|c}
                \(X\) & \(0\) & \(1\) \\ \hline
                \(\mathbb{P}\) & \(1-p\) & \(p\)
            \end{tabular}
            \end{center}
        \end{ppt}

        \begin{ppt}
            []
            {Bernoulli 分布的数学期望}
            []
            [猫猫]
            设 \(X\) 是\离散型随机变量, \(p:\R\), \(p\in\Ioo{0}{1}\), \(X\sim B(1,p)\) 则: 
            \[\Expct{X}=p\]
        \end{ppt}

        \begin{xmp}
            [Binom]
            {\(n\) 重 Bernoulli 分布 / 二项分布}
            [Binomial Distribution]
            [Dedicatia, 猫猫]
            设 \(n:\N\), \(p:\R\), \(p\in\Ioo{0}{1}\), 定义 \(X\) 服从成功率为 \(p\) 的 \(n\) 重 Bernoulli 分布\textbf{二项分布}, 当且仅当其分布列为
            
            \[\forall k:\N, k\leq n\implies\mathbb{P}(X=k)=\binom{n}{k}p^k(1-p)^{n-k}\]
            
            记作 \(X\sim B(n,p)\)
        \end{xmp}

        \begin{ppt}
            []
            {二项分布的数学期望}
            []
            [猫猫]
            设 \(X\) 是\离散型随机变量, \(X\sim B(n,p)\), 则: 
            \[\Expct{X}=np\]
        \end{ppt}

        \begin{ppt}
            []
            {二项分布的方差}
            []
            [Dedicatia]
            设 \(X\) 是\离散型随机变量, \(X\sim B(n,p)\), 则: 
            \[\Var{X}=np(1-p)\]
        \end{ppt}

        \begin{xmp}
            []
            {超几何分布}
            [Hypergeometric Distribution]
            [猫猫]
            设 \(X\) 是\离散型随机变量, \(X\) 服从超几何分布, 当且仅当其分布列为:
            \[\forall m:\N, m\leq M\land m\leq n\implies\mathbb{P}(X=m)=\frac{\binom{M}{m}\binom{N-M}{n-m}}{\binom{N}{n}}\]
            记作 \(X\sim H(N,M,n)\)
        \end{xmp}

        \begin{ppt}
            []
            {超几何分布的数学期望}
            []
            [猫猫]
            设 \(X\) 是\离散型随机变量, \(X\) 服从超几何分布, 则: 
            \[\Expct{X}=\frac{nM}{N}\]
        \end{ppt}

        \begin{ppt}
            []
            {超几何分布的方差}
            []
            [Dedicatia]
            设 \(X\) 是\离散型随机变量, \(X\) 服从超几何分布, 则: 
            \[\Var{X}=np(1-p)\qty(1-\frac{n-1}{N-1}).\]
        \end{ppt}

        \begin{xmp}
            []
            {Poisson分布}
            []
            [Dedicatia, 猫猫]
            设 \(X\) 是\离散型随机变量, 定义 \(X\) 服从\textbf{Poisson分布}, 当且仅当: 
            
            记作 \(X\sim \Poi(\lambda)\).
        \end{xmp}

        \begin{ppt}
            []
            {Poisson 分布的分布列}
            []
            [猫猫]
            设 \(X\) 是\离散型随机变量, \(X\sim Ps(\lambda)\), 则: 
            \[\forall k:\N, \mathbb{P}(X=k)=\frac{\lambda^k}{k!}\exp (-\lambda)\]
        \end{ppt}

        \begin{ppt}
            []
            {Poisson 分布的数学期望}
            []
            [猫猫]
            设 \(X\) 是\离散型随机变量, \(X\sim Ps(\lambda)\), 则: 
            \[\Expct{X}=\lambda\]
        \end{ppt}

        \begin{thm}
            []
            {Poisson 近似}
            []
            [Dedicatia]
            在独立重复试验中, 随机变量 \(X\) 服从二项分布 \(B(n,p_n)\) , \(p_n\) 是某次试验中事件 \(A\) 发生的概率, 且 \(np_n\to\lambda \) . 则当 \(n\to\infty\) 时, 有\[B(k;n,p_n)\to\frac{\lambda^k}{k!}\exp (-\lambda) \]
        \end{thm}

        \begin{prf}
            \[
                \begin{aligned}
                    B(k;n,p_n) &= \binom{n}{k}p_n^k(1-p_n)^{n-k}\\
                    &=\frac{n(n-1)\cdots(n-k+1)}{k!}\cdot\frac{np_n}{n}\left(1-p_n \right)^{n-k}\\
                    &=\frac{(np_n)^k}{k!}\left(1-\frac{1}{n} \right)\left(1-\frac{2}{n} \right)\cdots\left(1-\frac{k-1}{n} \right)\left(1-\frac{np_n}{n} \right)^{n-k}        
                \end{aligned}
            \]
            \[\lim_{n\to\infty}np_n=\lambda,\quad\lim_{n\to\infty}\left(1-\frac{\lambda}{n} \right)^{n-k}=\exp(-\lambda) \]
            取极限, 得\[\lim_{n\to\infty} B(k;n,p_n)=\frac{\lambda^k}{k!}\exp (-\lambda) \]
        \end{prf}

        \begin{xmp}
            []
            {几何分布}
            [Geometric Distribution]
            [Dedicatia]
            \(n\) 重Bernoulli试验中, 只会出现两种结果. 每次出现某种结果的概率为 $p$, 重复试验直到出现这种结果为止. 令 $X$ 为表示试验次数的随机变量, 那么 $X$ 服从几何分布.
        \end{xmp}

        \begin{ppt}
            []
            {几何分布的分布列}
            []
            [猫猫]
            设 \(X\) 是\离散型随机变量, \(X\) 服从几何分布, 则: 
            \[\forall k:\N^*, \mathbb{P}(X=k)=(1-p)^{k-1}p\]
        \end{ppt}

        \begin{ppt}
            []
            {几何分布的无记忆性}
            []
            [猫猫]
            设 \(X\) 是\离散型随机变量, \(X\) 服从几何分布, 则: 
            \[\forall m,n:\N, \mathbb{P}(X=m+n|X>m)=\mathbb{P}(X>n)\]
        \end{ppt}

        \begin{ppt}
            []
            {几何分布的数学期望}
            []
            [猫猫]
            设 \(X\) 是\离散型随机变量, \(X\) 服从几何分布, 则: 
            \[\Expct{X}=\frac{1}{p}\]
        \end{ppt}

        \begin{ppt}
            []
            {几何分布的方差}
            []
            [Dedicatia]
            设 \(X\) 是\离散型随机变量, \(X\) 服从几何分布, 则: 
            \[\Var{X}=\frac{1-p}{p^2}\]
        \end{ppt}

        \begin{xmp}
            []
            {幂律分布}
            [Power-Law Distribution]
            [猫猫]
            设 \(X\) 是\离散型随机变量, \(\alpha:\R\), \(c:=\left(\sum\limits_{k=1}^{+\infty}\dfrac{1}{k^\alpha}\right)^{-1}\), 定义 \(X\) 服从\textbf{幂律分布}, 当且仅当: 
            \[\forall k:\N^*, \mathbb{P}(X=k)=\frac{c}{k^\alpha}\]
        \end{xmp}

        \begin{ppt}
            []
            {幂律分布的无尺度性}
            []
            [猫猫]
            \[\forall a:\N, \mathbb{P}(X=ak)=\frac{1}{a^\alpha}\mathbb{P}(X=k)\]
        \end{ppt}

        \begin{xmp}
            []
            {Pascal 分布 / 负二项分布}
            [Pascal Distribution / Negative Binomial Distribution]
            [猫猫, Dedicatia]
            \(n\) 重Bernoulli试验中, 设每次出现某种结果的概率为 $p$, 重复试验直到累计出现 $r$ 次这种结果为止. 令 $X$ 为表示试验次数的随机变量, 那么 $X$ 服从负二项分布. 记作 $X\sim NB(r,p)$.
        \end{xmp}

        \begin{ppt}
            []
            {负二项分布与几何分布的关系}
            []
            [Dedicatia]
            参数为 \((1,p)\) 的负二项分布就是几何分布.
        \end{ppt}

        \begin{ppt}
            []
            {负二项分布的分布列}
            []
            [Dedicatia]
            设 \(X\) 是\离散型随机变量, \(p:[0,1]\), \(r:\N^+\), 若 \(X\sim NB(r,p)\), 那么分布列为:
            \[\forall\,n\in\{r,r+1,\cdots\},\,\mathbb{P}(X=n)=\binom{n-1}{r-1}p^r(1-p)^{n-r}\]
        \end{ppt}

        \begin{ppt}
            []
            {负二项分布的期望与方差}
            []
            [Dedicatia]
            设 \(X\) 是\离散型随机变量, \(p:[0,1]\), \(r:\N^+\), 若 \(X\sim NB(r,p)\), 那么
            \[\Expct{X}=\frac{r}{p}\]
            \[\Var{X}=\frac{r(1-p)}{p^2}\]
        \end{ppt}

        \begin{xmp}
            []
            {}
            []
            [Dedicatia]
            \(n\) 重Bernoulli试验中, 设每次出现某种结果的概率为 $p$, 重复试验直到累计出现 $r$ 次这种结果为止. 在此之前共发生了 $m$ 次不成功的试验的概率为
            \[\sum_{n=r}^{r+m-1}\binom{n-1}{r-1}p^r(1-p)^{n-r}\]
        \end{xmp}

        \begin{xmp}
            []
            {匹配问题}
            []
            [Dedicatia]
            有 $n$ 个不同种类的元素随机地将它们放入 $n$ 个集合里, 则没有一个元素归属于自己原先的集合的概率为:
            \[P=1-\sum_{k=1}^{n}\frac{(-1)^{k-1}}{k!}\]
            当 $n\to\infty$, 
            \[P=\sum_{k=0}^{\infty}\frac{(-1)^{k}}{k!}\to \exp(-1).\]
        \end{xmp}

    \subsection{连续型随机变量分布}

        \begin{dfn}
            [Continuous-Random-Variable]
            {连续型随机变量}
            []
            [Dedicatia, 猫猫]
            设 \((\Omega, \mathscr{F}, \mathbb{P})\) 是\概率空间, \(X\) 是\随机变量, \(F\) 是 \(X\) 的分布函数, \(f:\R\to\R\), \(f\) 在 \(\R\) 上 Lebesgue 可积, 定义 \(f\) 为 \(X\) 的\textbf{概率密度函数}, 当且仅当: 
            \[\forall x:\R, F(x)=\int_{-\infty}^{x}f(t)\dd t\]
            
            定义 \(X\) 是\textbf{连续型随机变量}, 当且仅当: \(X\) 存在概率密度函数. 
        \end{dfn}

        \begin{ppt}
            []
            {连续型随机变量分布函数连续}
            []
            [猫猫]
            设 \(X\) 是\连续型随机变量, \(F\) 是 \(X\) 的分布函数, 则: \(F\) 在 \(\R\) 上连续. 
        \end{ppt}

        \begin{ppt}
            []
            {概率密度函数积分为 \(1\)}
            []
            [猫猫]
            设 \(X\) 是\连续型随机变量, \(f\) 是 \(X\) 的概率密度函数, 则: 
            \[\int_{-\infty}^{+\infty}f(x)\dd x=1\]
        \end{ppt}

        \begin{ppt}
            []
            {连续密度函数是分布函数的导数}
            []
            [猫猫]
            设 \(X\) 是\连续型随机变量, \(F\) 是 \(X\) 的分布函数, \(f\) 是 \(X\) 的概率密度函数, 则: \(F\) 在 \(\R\) 上可微, 且 \(F'(x)=f(x)\). 
        \end{ppt}

        \begin{ppt}
            []
            {概率密度函数的积分刻画事件概率}
            []
            [猫猫]
            设 \((\Omega, \mathscr{F}, \mathbb{P})\) 是\概率空间, \(X\) 是\连续型随机变量, \(f\) 是 \(X\) 的概率密度函数, \(B\in\mathscr{B}(\R)\), 则: 
            \[\mathbb{P}(X\in B)=\int_{B}f(x)\dd x\]
        \end{ppt}

        \begin{thm}
            []
            {复合连续型随机变量的密度函数}
            []
            [Dedicatia, 猫猫]
            设 \(X\) 是\连续型随机变量, 具有密度函数 \(f_X\), \(g(x)\) 严格单调且可微分, 则: 
            \[f_{g(X)}=y\mapsto
            \begin{cases}
                f_X(g^{-1}(y))\dv{\cdot(g^{-1}(y))}{y} & \exists x:\R, y=g(x)\\
                0 & \forall x:\R, y\neq g(x)
            \end{cases}\]
        \end{thm}

        \begin{xmp}
            []
            {均匀分布}
            [Uniform Distribution]
            [猫猫]
            设 \(X\) 是\连续型随机变量, \(f:\R\to\R\), \(a,b:\R\), \(a<b\), 定义 \(X\) 服从 \(\Icc{a}{b}\) 上的\textbf{均匀分布}, 当且仅当: 
            \[f(x)=\begin{cases}
                \dfrac{1}{b-a}, & x\in\Icc{a}{b}\\
                0, & x\notin\Icc{a}{b}
            \end{cases}\]

            记作 \(X\sim U(a,b)\). 
        \end{xmp}

        \begin{ppt}
            []
            {均匀分布的分布函数}
            []
            [猫猫]
            设 \(X\) 是\连续型随机变量, \(F\) 是 \(X\) 的分布函数, \(X\sim U(a,b)\), 则: 
            \[F(x)=\begin{cases}
                0, & x<a\\
                \dfrac{x-a}{b-a}, & x\in\Icc{a}{b}\\
                1, & x>b
            \end{cases}\]
        \end{ppt}

        \begin{ppt}
            []
            {均匀分布的数学期望}
            []
            [猫猫]
            设 \(X\) 是\连续型随机变量, \(X\sim U(a,b)\), 则: 
            \[\Expct{X}=\dfrac{a+b}{2}\]
        \end{ppt}

        \begin{ppt}
            []
            {均匀分布的方差}
            []
            [Dedicatia]
            设 \(X\) 是\连续型随机变量, \(X\sim U(a,b)\), 则: 
            \[\Var{X}=\dfrac{(b-a)^2}{12}\]
        \end{ppt}

        \begin{thm}
            []
            {均匀分布的幂的密度函数}
            []
            [Dedicatia]
            设 \(X\) 是\连续型随机变量, \(X\sim U(0,1)\), 则\连续型随机变量 $Y=X^n$ 的密度函数为
            \[f_Y(y)=\begin{cases}
                &\frac{y^{\frac{1}{n-1}}}{n};\quad 0\leq y\leq 1\\
                &0; y<0 \lor y>1.
            \end{cases}\]
        \end{thm}

        \begin{xmp}
            []
            {幂律分布}
            [Power-Law Distribution]
            [猫猫]
            设 \(X\) 是\连续型随机变量, \(\alpha:\R\), \(f\) 是 \(X\) 的密度函数, 定义 \(X\) 服从参数为 \(\alpha\) 的\textbf{幂律分布}, 当且仅当: 
            \[f(x)=
            \begin{cases}
                \frac{\alpha}{x^{\alpha+1}}& x\geq 1\\
                0 & x<1.
            \end{cases}\]
        \end{xmp}

        \begin{xmp}
            [Expo-Distribution]
            {指数分布}
            [Exponential Distribution]
            [Dedicatia]
            设 \(X\) 是\连续型随机变量, \(\lambda:\R^+\), 定义 \(X\) 服从 \(\R\) 上参数为 \(\lambda\) 的\textbf{指数分布}, 如果其分布函数为
            \[f(x)=\begin{cases}
                &\lambda\exp(-\lambda x),\quad x\geq 0;\\
                &0,\quad x< 0.
            \end{cases}\]
            记作 \(X\sim\mathfrak{E}(\lambda)\).
        \end{xmp}

        \begin{ppt}
            []
            {指数分布的分布函数}
            []
            [Dedicatia]
            设 \(X\) 是\连续型随机变量, \(X\sim\mathfrak{E}(\lambda)\), $F$ 是 $X$ 的分布函数, 那么
            \[F(x)=P(X\geq x)=1-\exp(-\lambda x), (x\geq 0).\]
        \end{ppt}

        \begin{ppt}
            []
            {指数分布的期望和方差}
            []
            []
            设 \(X\) 是\连续型随机变量, \(X\sim\mathfrak{E}(\lambda)\), 则
            \[\Expct{X}=\frac{1}{\lambda},\qquad \Var{X}=\frac{1}{\lambda^2}.\]
        \end{ppt}

        \begin{dfn}
            [Hazard]
            {失效率函数}
            [Hazard Rate]
            [Dedicatia]
            设 \(X\) 是\连续型随机变量, $X>0$, 其分布函数是 $F$, 其密度函数是 $f$, 定义 $X$ 的\textbf{失效率}为
            \[\lambda(t)=\frac{f(t)}{1-F(t)}.\]
        \end{dfn}

        \begin{ppt}
            []
            {分布函数可由失效率唯一确定}
            []
            [Dedicatia]
            以失效率函数 $\lambda(x)$ 可以唯一地确定它的分布函数 $F$.
        \end{ppt}

        \begin{prf}
            对 $\lambda(x)$ 在 $[0,t]$ 上积分
            \[\int_0^t \lambda(x)\dd{x}=\int_0^t \frac{f(x)}{1-F(x)}\dd{x}=-\ln(1-F(x))\bigg|_0^t=\ln(1-F(0))-\ln(1-F(t))=-\ln(1-F(t))\]
            可解得
            \[F(t)=1-\exp(-\int_0^t\lambda(x)\dd{x})\]
        \end{prf}

        \begin{ppt}
            []
            {指数分布的失效率是常数}
            []
            [Dedicatia]
            设 \(X\) 是\连续型随机变量, \(X\sim\mathfrak{E}(\lambda)\), 那么失效率 $\lambda(t)=\lambda$.
        \end{ppt}

        \begin{xmp}
            [Rayleigh]
            {Rayleigh 分布}
            [Rayleigh Distribution]
            [Dedicatia]
        \end{xmp}

        \begin{ppt}
            []
            {Rayleigh 分布的分布函数、密度函数、失效率函数}
            设 \(X\) 是\连续型随机变量, $X$ 服从Rayleigh分布, $b:\R$, 那么 $X$ 的分布函数为
            \[F(t)=1-\exp(-\frac{bt^2}{2})\]
            其密度函数为
            \[f(t)=bt\exp(-\frac{bt^2}{2})\]
            失效率函数为
            \[\lambda(t)=bt\]
            这说明失效率是线性的.
        \end{ppt}

        \begin{xmp}
            [Gamma-Distribution]
            {\(\Gamma\) 分布}
            [\(\Gamma\) Distribution]
            [Dedicatia]
            设 \(X\) 是\连续型随机变量, 定义 $X$ 服从\textbf{\(\Gamma\)分布}当且仅当其密度函数为
            \[f(x)=\begin{cases}
                &\frac{\lambda\exp(-\lambda x)\cdot(\lambda x)^{\alpha-1 }}{\Gamma(\alpha)},\quad x\geq 0;\\
                &0,\quad x<0.
            \end{cases}\]
            记作 $X\sim\Gamma(\alpha,\lambda)$. 其中 $\Gamma(\alpha)$ 是 $\Gamma$ 函数.
        \end{xmp}

        \begin{ppt}
            []
            {\(\Gamma\) 分布与指数分布的关系}
            []
            [Dedicatia]
            设 \(X\) 是\连续型随机变量, \(X\sim\mathfrak{E}(\lambda)\) 当且仅当 $x\sim \Gamma(1,\lambda)$.
        \end{ppt}

        \begin{ppt}
            []
            {\(\Gamma\) 分布的期望与方差}
            []
            [Dedicatia]
            设 \(X\) 是\连续型随机变量, \(X\sim\Gamma(\alpha,\lambda)\), 那么
            \[\Expct{X}=\frac{\alpha}{\lambda}\]
            \[\Var{X}=\frac{\alpha}{\lambda^2}\]
        \end{ppt}

        \begin{xmp}
            [Cauchy-Distribution]
            {Cauchy 分布}
            [Cauchy Distribution]
            [Dedicatia]
            设 \(X\) 是\连续型随机变量, $\theta:\R$, 定义 \(X\) 服从 Cauchy 分布当且仅当其密度函数为
            \[f(x)=\frac{1}{\pi}\frac{1}{1+(x-\theta)^2}, \qquad x\in (-\infty,+\infty)\]
        \end{xmp}

        \begin{ppt}
            []
            {标准 Cauchy 分布}
            []
            [Dedicatia]
        \end{ppt}

        \begin{ppt}
            []
            {Cauchy 分布的期望不存在}
            []
            [Dedicatia]
        \end{ppt}

        \begin{xmp}
            [Normal-Distribution]
            {正态分布}
            [Normal Distribution]
            [猫猫, Dedicatia]
            设 \(X\) 是\连续型随机变量, \(\mu:\R\), \(\sigma:\R^+\), 定义 \(X\) 服从参数为 \(\mu\) 和 \(\sigma^2\) 的\textbf{正态分布}, 当且仅当其密度函数为: 
            \[f(x)=\frac{1}{\sqrt{2\pi}\sigma}\exp\left(-\frac{(x-\mu)^2}{2\sigma^2}\right)\]

            记作 \(X\sim N(\mu,\sigma^2)\). 
        \end{xmp}

        \begin{dfn}
            []
            {标准正态分布}
            []
            [Dedicatia]
            定义\连续型随机变量 $X$ 服从\textbf{标准正态分布}, 如果 $X\sim N(0,1)$. \\
            一般将标准正态分布的分布函数记作
            \[\Phi(x)=\frac{1}{\sqrt{2\pi}}\int_{-\infty}^x\exp(\frac{-t^2}{2})\dd{t}.\]
        \end{dfn}

        \begin{rmk}
            [Dedicatia]
            \正态分布 的密度函数不具有初等的原函数,计算正态分布的概率的数值时常将正态分布随机变量线性地转换为服从标准正态分布的随机变量。
            \[X\sim N(\mu,\sigma^2)\Longleftrightarrow Z=\frac{X-\mu}{\sigma}\sim N(0,1)\]
        \end{rmk}

        \begin{ppt}
            []
            {正态分布的数学期望}
            []
            [猫猫]
            设 \(X\) 是\连续型随机变量, \(X\sim N(\mu,\sigma^2)\), 则: 
            \[\Expct{X}=\mu\]
        \end{ppt}

        \begin{ppt}
            []
            {正态分布的方差}
            []
            [Dedicatia]
            设 \(X\) 是\连续型随机变量, \(X\sim N(\mu,\sigma^2)\), 则: 
            \[\Var{X}=\sigma^2\]
        \end{ppt}

        \begin{ppt}
            []
            {二项分布的正态近似}
            []
            [Dedicatia]
            设 \(X\) 是服从\hyperref[xmp:Binom]{二项分布}的\离散型随机变量, $X\sim B(n,p)$, 那么对任意 $a<b$,
            \[\lim_{n\to\infty}P\qty(a\leq\frac{X-np}{\sqrt{np(1-p)}}\leq b)=\Phi(b)-\Phi(a)\]
        \end{ppt}

        \begin{xmp}
            [Chi-Square]
            {卡方分布}
            [Chi-Square Distribution]
            [Dedicatia]
            若\随机变量 \( Z_1, Z_2, \dots, Z_k \) 相互独立且均服从标准\正态分布 \( N(0, 1) \),则它们的平方和所构成随机变量
            \[
            X = \sum_{i=1}^{k} Z_i^2
            \]
            服从自由度为 \( k \) 的\textbf{卡方分布}, 记作:
            \[X \sim \chi^2(k)\]
        \end{xmp}

        \begin{ppt}
            []
            {卡方分布的概率密度函数}
            []
            [Dedicatia]
            若\随机变量 \(X\sim\chi^2(k)\), 其概率密度函数为:
            \[
            f(x) = 
            \begin{cases} 
            \frac{1}{2^{k/2} \Gamma(k/2)} x^{k/2 - 1} e^{-x/2}, & x > 0 \\
            0, & x \le 0
            \end{cases}
            \]
        \end{ppt}

        \begin{ppt}
            []
            {卡方分布的数学期望}
            []
            [Dedicatia]
            若\随机变量 \(X\sim\chi^2(k)\), 那么 \(\Expct{X}=k\).
        \end{ppt}

        \begin{ppt}
            []
            {卡方分布的方差}
            []
            [Dedicatia]
            若\随机变量 \(X\sim\chi^2(k)\), 那么 \(\Var{X}=2k\).
        \end{ppt}

        \begin{ppt}
            []
            {卡方分布与 \(\Gamma\) 分布的关系}
            []
            [Dedicatia]
            卡方分布是一种特殊的 \(\Gamma\) 分布:
            \[\chi^2(k) = \Gamma\qty(\frac{k}{2},2).\]
        \end{ppt}

    \subsection{随机变量的其他数值特征}

        \begin{dfn}
            [Cv]
            {变异系数}
            [Coefficient of Variation]
            [Dedicatia]
            设 \(X\) 是\随机变量, 均值为 \(\mu\), 标准差为 \(\sigma\), 定义其变异系数为 \(Cv(X):=\frac{\sigma}{\mu}\). 即:
            \[Cv(X) = \frac{\sqrt{\Var{X}}}{\Expct{X}}.\]
        \end{dfn}

        \begin{dfn}
            [Skewness]
            {偏度}
            [Skewness]
            [Dedicatia]
            设 \(X\) 是\随机变量, 均值为 \(\mu\), 标准差为 \(\sigma\), 定义其偏度为
            \[\Expct{\qty(\frac{X-\mu}{\sigma})^3}.\]
        \end{dfn}
        
        \begin{dfn}
            [Kurtosis]
            {(超额)峰度}
            [Kurtosis]
            [Dedicatia]
            设 \(X\) 是\随机变量, 均值为 \(\mu\), 标准差为 \(\sigma\), 定义其峰度为
            \[\Expct{\qty(\frac{X-\mu}{\sigma})^4}-3.\]
        \end{dfn}

        \begin{xmp}
            []
            {几种分布的峰度}
            []
            [Dedicatia]
            均匀分布 \(U(0,1)\) 的峰度为 \(-1.2\);\\
            \正态分布 的峰度为 \(0\);\\
            指数分布 \(\mathfrak{E}(\lambda)\) 的峰度为 \(6\);\\
            \(\Gamma\)分布 \(\Gamma(\alpha,\lambda)\) 的峰度为 \(\frac{6}{\alpha}\).
        \end{xmp}

        \begin{rmk}
            [Dedicatia]
            偏度用来衡量分布关于其均值的不对称性. 偏度越大越不对称;\\
            峰度用来衡量分布的尖峭程度, 峰度越大, 分布越陡峭. 其减3的原因是与\正态分布 作比较.
        \end{rmk}

\section{多维随机变量}

    \subsection{多维随机变量}
        
        \begin{dfn}
            [Product-Probability-Measure]
            {积概率测度}
            [Product Probability Measure]
            [猫猫]
            设 \(\mathscr{F}_1, \mathscr{F}_2\) 是事件域, \(\mathbb{P}_1:\mathscr{F}_1\to\R\), \(\mathbb{P}_2:\mathscr{F}_2\to\R\), \(\mathbb{P}_1\) 是 \(\mathscr{F}_1\) 上的概率测度, \(\mathbb{P}_2\) 是 \(\mathscr{F}_2\) 上的概率测度, \(\mathbb{P}:\mathscr{F}_1\otimes\mathscr{F}_2\to\R\), 定义 \(\mathbb{P}\) 为 \(\mathscr{F}_1\) 与 \(\mathscr{F}_2\) 的\textbf{积概率测度}, 当且仅当: 
            \begin{enumerate}
                \item \(\mathbb{P}\) 是 \(\mathscr{F}_1\otimes\mathscr{F}_2\) 上的\概率测度; 
                
                \item \textbf{独立性}: 
                \[\forall A_1\in\mathscr{F}_1, \forall A_2\in\mathscr{F}_2, \mathbb{P}(A_1\times A_2)=\mathbb{P}_1(A_1)\mathbb{P}_2(A_2)\]
            \end{enumerate}

            记作 \(\mathbb{P}=\mathbb{P}_1\otimes\mathbb{P}_2\). 
        \end{dfn}
        
        \begin{ppt}
            []
            {积概率测度唯一性}
            []
            [猫猫]
            设 \(\mathbb{P}_1,\mathbb{P}_2\) 是\概率测度, 则: 
            \[\exists!\mathbb{P}:\mathscr{F}_1\otimes\mathscr{F}_2\to\R, \mathbb{P}=\mathbb{P}_1\otimes\mathbb{P}_2\]
        \end{ppt}
        
        \begin{dfn}
            [Product-Probability-Space]
            {积概率空间}
            [Product Probability Space]
            [猫猫]
            设 \(n:\N\), \(\{(\Omega_i,\mathscr{F}_i,\mathbb{P}_i)\}_{i=1}^n\) 是长度为 \(n\) 的\概率空间 序列, 定义 \(\{(\Omega_i,\mathscr{F}_i,\mathbb{P}_i)\}_{i=1}^n\) 的\textbf{积概率空间}为: 
            \[\left(\prod_{i=1}^n\Omega_i, \bigotimes_{i=1}^n\mathscr{F}_i, \bigotimes_{i=1}^n\mathbb{P}_i\right)\]

            记作: 
            \[\bigotimes_{i=1}^n(\Omega_i,\mathscr{F}_i,\mathbb{P}_i)\]
        \end{dfn}
    
        \begin{dfn}
            [Multi-Random-Variable]
            {多维随机向量}
            [Multi Random Variable]
            [猫猫]
            设 \(n:\N\), \(\bigotimes\limits_{i=1}^n(\Omega_i,\mathscr{F}_i,\mathbb{P}_i)\) 是\概率空间, 定义 \(X\) 是 \(n\) 维\textbf{随机向量}, 
        \end{dfn}

        \begin{dfn}
            []
            {联合分布函数}
            [Joint Cumulative Distribution Function / Joint CDF]
            [Dedicatia]
            设 \(\bm{X}=(X_1,\cdots,X_n)\) 是一个\随机向量{\(n\)}, 定义 \(\bm{X}\) 的\textbf{联合分布函数}为:
            \[F_{\bm{X}}(x_1,\cdots,x_n)=\mathbb{P}(X_1\leq x_1,\cdots,X_n\leq x_n)\]
        \end{dfn}

        \begin{ppt}
            []
            {联合分布函数的性质}
            []
            [Dedicatia]
            联合分布函数具有以下性质:
            \begin{enumerate}
                \item 非负性: $F\geq 0$;
                \item 单调性: $F$ 关于每个变量 $X_i$ 都是单调不减的函数;
                \item 规范性: $F(+\infty,+\infty,\cdots,+\infty)=1, F(x_1,\cdots,x_{i-1},-\infty,x_{i+1},\cdots,x_n)=0$.
                \item 右连续, 左极限存在
            \end{enumerate}
        \end{ppt}

        \begin{dfn}
            []
            {(单变量的)边缘分布函数}
            [Marginal Cumulative Distribution Function]
            [Dedicatia]
            设 \(\bm{X}=(X_1,\cdots,X_n)\) 是一个\随机向量{\(n\)}, 其分布函数为 \(F\). 定义其某一分量 \(X_k\) 的\textbf{边缘分布函数}为:
            \[F_{X_k}(x_k) = F(+\infty,\cdots,x_k,\cdots,+\infty).\]
        \end{dfn}

        \begin{dfn}
            []
            {多变量的边缘分布函数}
            [Multivariate Marginal Cumulative Distribution Function]
            [Dedicatia]
            设 \(\bm{X}=(X_1,\cdots,X_n)\) 是一个\随机向量{\(n\)}, 其分布函数为 \(F\). 定义其某些分量 \(X_j,\cdots,X_k\) 的\textbf{边缘分布函数}为:
            \[F_{X_k}(x_k) = F(+\infty,\cdots,x_j,\cdots,+\infty,x_k,\cdots,+\infty).\]
        \end{dfn}

        \begin{dfn}
            []
            {分布独立}
            [Independent Distribution]
            [猫猫]
            设 \(X=(X_1,\cdots,X_n)\) 是一个\随机向量{\(n\)}, 其分布函数为 \(F\), 各分量 \(X_k\) 的边缘分布函数为 \(F_{X_k}(x_k)\). 定义其分量 \(X_1,\cdots,X_n\) 是\textbf{相互独立/独立分布的}, 当且仅当:
            \[F(x_1,\cdots,x_n)=\prod_{k=1}^{n}F_{X_k}(x_k)\]
        \end{dfn}

    \subsection{期望、协方差与相关系数}

        \begin{dfn}
            []
            {期望}
            [Expectation]
            [猫猫, Dedicatia]
            设 \(\bm{X}\) 是\随机向量{\(n\)}, 定义其期望为:
            \[\Expct{X}=(\Expct{X_1}, \cdots, \Expct{X_n}).\]
        \end{dfn}

        \begin{ppt}
            []
            {Cauchy-Schwarz 不等式}
            []
            [猫猫]
            设 \((X,Y)\) 是 \随机向量{\(2\)}, 则: 
            \[|\Expct{XY}| \leq \sqrt{\Expct{X^2}}\sqrt{\Expct{Y^2}}\]
        \end{ppt}


        \begin{dfn}
            [Covariance]
            {协方差}
            [Covariance]
            [猫猫]
            设 \((X,Y)\) 是 \随机向量{\(2\)}, 定义 \(X\) 与 \(Y\) 的\textbf{协方差}为: 
            \[\Expct{(X-\Expct{X})(Y-\Expct{Y})}\]

            记作 \(\Cov{X}{Y}\). 
        \end{dfn}

        \begin{rmk}
            [猫猫]
            协方差刻画了两个随机变量的共同变化趋势. 当协方差为正时, 两个随机变量同向变化; 当协方差为负时, 两个随机变量反向变化. 
        \end{rmk}

        \begin{ppt}
            []
            {协方差计算公式}
            []
            [猫猫]
            设 \((X,Y)\) 是 \随机向量{\(2\)}, 则: 
            \[\Cov{X}{Y}=\Expct{XY}-\Expct{X}\Expct{Y}\]
        \end{ppt}

        \begin{ppt}
            []
            {协方差的对称性}
            []
            [猫猫]
            设 \((X,Y)\) 是 \随机向量{\(2\)}, 则: 
            \[\Cov{X}{Y}=\Cov{Y}{X}\]
        \end{ppt}

        \begin{ppt}
            []
            {协方差与方差的关系}
            []
            [猫猫]
            设 \(X\) 是\随机变量, 则: 
            \[\Cov{X}{X}=\Var{X}\]
        \end{ppt}

        \begin{ppt}
            []
            {独立随机变量的协方差为零}
            []
            [猫猫]
            设 \((X,Y)\) 是 \随机向量{\(2\)}, \(X,Y\) 独立, 则: 
            \[\Cov{X}{Y}=0\]
        \end{ppt}

        \begin{ppt}
            []
            {协方差的双线性性}
            []
            [猫猫]
            设 \(X_1,X_2\) 是\随机变量, \(Y\) 是\随机变量, 则: 
            \[\forall \alpha,\beta:\R, \Cov{\alpha X_1+\beta X_2}{Y}=\alpha\Cov{X_1}{Y}+\beta\Cov{X_2}{Y}\]
        \end{ppt}

        \begin{dfn}
            [Correlation-Coefficient]
            {相关系数}
            [Correlation Coefficient]
            [猫猫]
            设 \((X,Y)\) 是 \随机向量{\(2\)}, 定义 \(X\) 与 \(Y\) 的\textbf{相关系数}为: 
            \[\frac{\Cov{X}{Y}}{\sqrt{\Var{X}}\sqrt{\Var{Y}}}\]

            记作 \(\rho_{X,Y}\). 
        \end{dfn}

        \begin{ppt}
            []
            {相关系数的规模不变性}
            []
            [猫猫]
            设 \((X,Y)\) 是 \随机向量{\(2\)}, 则: 
            \[\forall \alpha,\beta:\R^+, \rho_{X,Y}=\rho_{\alpha X,\beta Y}\]
        \end{ppt}

        \begin{ppt}
            []
            {相关系数的规范性}
            []
            [猫猫]
            设 \((X,Y)\) 是 \随机向量{\(2\)}, 则: 
            \[-1\leq\rho_{X,Y}\leq 1\]
        \end{ppt}

        \begin{ppt}
            []
            {独立随机变量的相关系数为零}
            []
            [猫猫]
            设 \((X,Y)\) 是 \随机向量{\(2\)}, \(X,Y\) 独立, 则: 
            \[\rho_{X,Y}=0\]
        \end{ppt}

        \begin{ppt}
            []
            {完全相关时相关系数为 \(\pm 1\)}
            []
            [猫猫]
            设 \((X,Y)\) 是 \随机向量{\(2\)}, \(k:=\dfrac{\sqrt{\Var{Y}}}{\sqrt{\Var{X}}}\), 则: 
            \[
            \begin{cases}
                \rho(X,Y)=1 & \iff Y=k(X+\Expct{X})-\Expct{Y}\\
                \rho(X,Y)=-1 & \iff Y=-k(X+\Expct{X})+\Expct{Y}
            \end{cases}
            \]
        \end{ppt}

        \begin{dfn}
            []
            {协方差矩阵}
            []
            [猫猫]
            设 \(\bm{X}:=(X_1,X_2,\dots,X_n)\) 是 \随机向量{\(n\)}, 定义 \(\bm{X}\) 的\textbf{协方差矩阵}为: 
            \[
            \begin{bmatrix}
                \Var{X_1} & \Cov{X_1}{X_2} & \cdots & \Cov{X_1}{X_n} \\
                \Cov{X_2}{X_1} & \Var{X_2} & \cdots & \Cov{X_2}{X_n} \\
                \vdots & \vdots & \ddots & \vdots \\
                \Cov{X_n}{X_1} & \Cov{X_n}{X_2} & \cdots & \Var{X_n}
            \end{bmatrix}
            \]

            记作 \(\Sigma_{\bm{X}}\) 或 \(\Cov(\bm{X})\). 
        \end{dfn}

        \begin{ppt}
            []
            {协方差矩阵的对称性}
            []
            [猫猫]
            设 \(\bm{X}\) 是 \随机向量{\(n\)}, 则: \[\Sigma_{\bm{X}}=\Trans{\Sigma_{\bm{X}}}\]
        \end{ppt}

        \begin{ppt}
            []
            {协方差矩阵的半正定性}
            []
            [猫猫]
            设 \(\bm{X}\) 是 \随机向量{\(n\)}, 则: \(\Sigma_{\bm{X}} \succeq 0\). 
        \end{ppt}

        \begin{ppt}
            {}
        \end{ppt}

    \subsection{多维离散型随机变量}

        \begin{dfn}
            [Multi-Discrete-Random-Variable]
            {多维离散型随机变量}
            []
            [猫猫]
            称取值离散的多维随机变量为多维离散型随机变量.
        \end{dfn}

        \begin{dfn}
            {二维离散型随机变量的分布}
            []
            [Dedicatia]
            称取值离散的\随机向量{\(2\)}为二维离散型随机变量. 其取值可以一一列出:
            \[p_{ij}:=\mathbb{P}(X=x_i, Y=y_j)\]
            可以列表表示为
            \begin{center}
                \begin{tabular}{c|cccc|c}
                    \((X,Y)\) & \(y_1\) & \(\cdots\) & \(y_i\) & \(\cdots\) & \(Y\) \\
                    \hline
                    \(x_1\) & \(p_{11}\) & \(\cdots\) & \(p_{1i}\) & \(\cdots\) & \(\mathbb{P}(X=x_1)\) \\
                    \(\vdots\) & \(\vdots\) & \(\ddots\) & \(\vdots\) & \(\ddots\) & \(\vdots\) \\
                    \(x_i\) & \(p_{i1}\) & \(\cdots\) & \(p_{ii}\) & \(\cdots\) & \(\mathbb{P}(X=x_i)\) \\
                    \(\vdots\) & \(\vdots\) & \(\ddots\) & \(\vdots\) & \(\ddots\) & \(\vdots\) \\
                    \hline
                    \(X\) & \(\mathbb{P}(Y=y_1)\) & \(\cdots\) & \(\mathbb{P}(Y=y_i)\) & \(\cdots\) & 1
                \end{tabular}
            \end{center}
        \end{dfn}

        \begin{dfn}
            []
            {联合分布列}
            []
            [猫猫]
            对于 $n$ 维离散型随机变量, 定义其联合分布列为
            \[p_{x_{1},\cdots,x_{n}} := \mathbb{P}(X_1=x_1,\cdots,X_n=x_n)\]
        \end{dfn}

    \subsection{多维连续型随机变量}

        \begin{dfn}
            [Multi-Continuous-Random-Variable]
            {多维连续型随机变量}
            []
            [Dedicatia]
        \end{dfn}

        \begin{dfn}
            []
            {联合概率密度函数}
            []
            [Dedicatia]
            设 \((X_1,\cdots,X_n)\) 是\连续型随机向量{\(n\)}, 具有分布函数 \(F(x_1,\cdots,x_n)\). 定义其概率密度函数为某函数 \(f(x_1,\cdots,x_n)\) 当且仅当:
            \[F(x_1,\cdots,x_n)=\int_{-\infty}^{x_1}\cdots\int_{-\infty}^{x_n}f(t_1,\cdots,t_n)\dd{t_1}\cdots\dd{t_n}.\]
        \end{dfn}

        \begin{dfn}
            []
            {边缘概率密度函数}
            []
            [Dedicatia]
            设 \((X_1,\cdots,X_n)\) 是\连续型随机向量{\(n\)}, 具有分布函数 \(F(x_1,\cdots,x_n)\). 定义其某个分量 \(X_k\) 的边缘密度函数为
            \[f_{X_k}(x_k)=\int_{-\infty}^{+\infty}\cdots\int_{-\infty}^{+\infty}f(t_1,\cdots,t_{k-1},x_k,t_{k+1},\cdots,x_n)\dd{t_1}\cdots\dd{t_{k-1}}\dd{t_{k+1}}\cdots\dd{t_n}.\]
        \end{dfn}

    \subsection{常见多维分布}

        \begin{xmp}
            []
            {多项分布}
            [Multinomial Distribution]
            [Dedicatia]
            设 \(\bm{X}=(X_1,X_2,\cdots,X_k)\) 是\随机向量{\(k\)}, 若其联合概率函数为
            \[P(X_1=x_1,X_2=x_2,\cdots,X_k=x_k)=\frac{n!}{x_1!x_2!\cdots x_k!}p_1^{x_1}p_2^{x_2}\cdots p_k^{x_k}\]
            其中 \(x_1,x_2,\cdots,x_k\) 为非负整数且满足 \(x_1+x_2+\cdots+x_k=n\), 且 \(p_1,p_2,\cdots,p_k\) 满足 \(p_1+p_2+\cdots+p_k=1\). 则称 \(\bm{X}\) 服从参数为 \(n\) 和 \((p_1,p_2,\cdots,p_k)\) 的\textbf{多项分布}, 记作 \(\bm{X}\sim M(n,p_1,p_2,\cdots,p_k)\).\\
            多项分布是离散型分布.
        \end{xmp}

        \begin{rmk}
            [Dedicatia]
            多项分布是二项分布的推广, 当 \(k=2\) 时, 多项分布退化为二项分布.
        \end{rmk}

        \begin{xmp}
            []
            {多维超几何分布}
            [Multivariate Hypergeometric Distribution]
            [Dedicatia]
            设 \(\bm{X}=(X_1,X_2,\cdots,X_k)\) 是\随机向量{\(k\)}, 若其联合概率函数为
            \[P(X_1=x_1,X_2=x_2,\cdots,X_k=x_k)=\frac{\prod_{i=1}^{k}\Comb{N_i}{x_i}}{\Comb{N}{n}}\]
            其中 \(x_1,x_2,\cdots,x_k\) 为非负整数且满足 \(x_1+x_2+\cdots+x_k=n\), 且 \(N_1,N_2,\cdots,N_k\) 满足 \(N_1+N_2+\cdots+N_k=N\). 则称 \(\bm{X}\) 服从参数为 \(N\), \(n\) 和 \((N_1,N_2,\cdots,N_k)\) 的\textbf{多维超几何分布}, 记作 \(\bm{X}\sim HG(N,n,N_1,N_2,\cdots,N_k)\).\\
            多维超几何分布是离散型分布.
        \end{xmp}

        \begin{xmp}
            [Multi-Uniform]
            {多维均匀分布}
            [Multi-Uniform Distribution]
            [Dedicatia]
            设定义在 \(\R^{n}\) 上的可测集 \(D\) 上的 \随机向量{\(n\)} \(\bm{X}\) 服从\textbf{多维均匀分布}, 当且仅当其联合密度函数 \(f\) 满足
            \[f(x_1,\cdots,x_n)=\begin{cases}
                &\frac{1}{m(D)}, \text{ 如果 } (x_1,\cdots,x_n)\in D;\\
                &0, \text{ 其他}
            \end{cases}\]
            记作 \(\bm{X}\sim U(D).\)
        \end{xmp}

        \begin{xmp}
            []
            {二维正态分布}
            []
            [Dedicatia]
            \随机向量{\(2\)} \((X,Y)\) 服从二维正态分布当且仅当其密度函数为
            \[f_{X,Y}(x,y)=\frac{1}{2\pi\sigma_1\sigma_2\sqrt{1-\rho^2}}\exp(-\frac{1}{2(1-\rho^2)}\qty(\qty(\frac{x-\mu_1}{\sigma_1^2})^2-2\rho\frac{x-\mu_1}{\sigma_1}\frac{y-\mu_2}{\sigma_2}+\qty(\frac{y-\mu_2}{\sigma_2})^2))\]
            其中 \(\rho\in(0,1)\). 记作 \((X,Y)\sim N(\mu_1,\mu_2, \sigma_1^2, \sigma_2^2, \rho)\).
        \end{xmp}

        \begin{xmp}
            [Multi-Normal]
            {多维正态分布}
            [Multi-Normal Distribution]
            [Dedicatia]
            \随机向量{\(n\)} \((X_1,\cdots,X_n)\) 服从 \(n\) 维正态分布当且仅当其密度函数可以表示为
            \[f(\bm{X})=\frac{1}{(\sqrt{2\pi})^n\sqrt{\det\Sigma}}\exp(-\frac{\Trans{(\bm{X}-\boldsymbol{\mu})}\Sigma(\bm{X}-\boldsymbol{\mu})}{2})\]
            其中 \(\bm{X}=(x_1,\cdots,x_n)\), \(\boldsymbol{\mu}=(\mu_1,\cdots,\mu_n)\), \(\Sigma\) 是对称正定矩阵, 这时记作 \(X\sim N(\boldsymbol{\mu},\Sigma)\).
        \end{xmp}

        \begin{ppt}
            []
            {二维正态分布的边缘密度}
            []
            [Dedicatia]
            设\随机向量{\(2\)} \((X,Y)\sim N(\mu_1,\mu_2, \sigma_1^2, \sigma_2^2, \rho)\), 那么 \(X,Y\) 的边缘密度函数为
            \[f_{X}(x)=\frac{1}{\sqrt{2\pi}\sigma_1}\exp(-\frac{(x-\mu_1)^2}{2\sigma_1^2})\]
            \[f_{Y}(y)=\frac{1}{\sqrt{2\pi}\sigma_2}\exp(-\frac{(x-\mu_2)^2}{2\sigma_2^2})\]
        \end{ppt}

        \begin{ppt}
            []
            {二维正态分布的相关系数}
            []
            [Dedicatia]
            设\随机向量{\(2\)} \((X,Y)\sim N(\mu_1,\mu_2, \sigma_1^2, \sigma_2^2, \rho)\), 则相关系数 \(\rho(X,Y)=\rho\).
        \end{ppt}

        \begin{ppt}
            []
            {正态随机变量相互独立}
            []
            [Dedicatia]
        \end{ppt}

        \begin{cxmp}
            []
            {正态随机变量的联合分布不一定是多维正态分布}
            []
            [Dedicatia]
            设 \((X,Y)\) 具有如下概率密度函数
            \[f(x,y)=\frac{1}{2\pi}\exp(-\frac{x^2+y^2}{2})(1+\sin x+\sin y)\]
            则边缘分布 \(X\sim N(0,1), Y\sim N(0,1)\), 但 \(X,Y\) 不服从\正态分布[\(2\)].
        \end{cxmp}

        \begin{xmp}
            []
            {多维t分布}
            [Multivariate t-Distribution]
            [Dedicatia]
            设 \((X_1,\cdots,X_n)\) 是\随机向量{\(n\)}, 定义其服从 \(n\) 维t分布当且仅当其概率密度函数为
            \[
            f(\bm{X}) = 
            \frac{\Gamma\left(\frac{\nu + p}{2}\right)}{\Gamma\left(\frac{\nu}{2}\right) (\nu \pi)^{\frac{p}{2}} |\boldsymbol{\Sigma}|^{\frac{1}{2}}}
            \left[1 + \frac{1}{\nu} (\bm{X} - \boldsymbol{\mu})^T \boldsymbol{\Sigma}^{-1} (\bm{X} - \boldsymbol{\mu}) \right]^{-\frac{\nu + p}{2}}
            \]
            记作
            \[(X_1,\cdots,X_n)\sim t_p(\boldsymbol{\mu}, \boldsymbol{\Sigma}, \nu).\]
            其中 \(\boldsymbol{\mu}\) 是位置向量, \(\nu\) 为自由度. 二次型
            \[
            \Trans{(\bm{X} - \boldsymbol{\mu})} \boldsymbol{\Sigma}^{-1} (\bm{X} - \boldsymbol{\mu})
            \]
            称为Mahalanobis距离.
        \end{xmp}

        \begin{ppt}
            []
            {t分布与正态分布的关系}
            []
            [Dedicatia]
            当 \(\nu\to+\infty\) 时, \(n\) 维t分布趋于\正态分布[\(n\)].
        \end{ppt}

        \begin{ppt}
            []
            {t分布的边缘分布仍为t分布}
            []
            [Dedicatia]
            设 \((X_1,\cdots,X_n)\) 是服从 \(n\) 维t分布的\随机向量{\(n\)}, 则其任意分量 \(X_j,\cdots,X_k\) 也服从 \(k-j+1\) 维t分布.
        \end{ppt}

        \begin{ppt}
            []
            {t分布的数值特征}
            []
            [Dedicatia]
            设\随机向量{\(n\)} \(\bm{X}=(X_1,\cdots,X_n)\sim t_n(\boldsymbol{\mu}, \boldsymbol{\Sigma}, \nu)\). 那么当 \(\nu>1\) 时期望存在, 为
            \[\Expct{\bm{X}} = \boldsymbol{\mu}\]
            当 \(\nu>2\) 时\协方差 矩阵存在, 为
            \[\Cov{\bm{X}}=\frac{\nu}{\nu-2}\boldsymbol{\Sigma}.\]
        \end{ppt}

        \begin{ppt}
            []
            {多维t分布的生成}
            []
            [Dedicatia]
            \bysorry
        \end{ppt}

    \subsection{多维随机变量的条件分布}

        \begin{dfn}
            []
            {离散型随机向量的条件分布列}
            []
            [Dedicatia]
            设\离散型随机向量{\(2\)} \((X,Y)\) 的联合分布列为 \(p_{ij}=p(x_i,y_j)=\mathbb{P}(X=x_i,Y=y_j)\), \(X\) 的边缘分布列为 \(p_{X}(x_i)\), 定义其在 \(X=x_i\) 的条件下 \(Y\) 的概率分布
            \[\mathbb{P}(Y=y_j\mid X=x_i)=\frac{p_{ij}}{p_{i,\cdot}}=\frac{p(x_i,y_j)}{p_{X}(x_i)}=\frac{\mathbb{P}(X=x_i,Y=y_j)}{\mathbb{P}(X=x_i)}\]
            记作
            \[p_{Y}(y_j\mid x_i)\]
            为 \(Y\) 的条件分布列.
        \end{dfn}

        \begin{thm}
            []
            {离散型随机向量的全概率公式}
            []
            [Dedicatia]
            设\离散型随机向量{\(2\)} \((X,Y)\) 的边缘分布列为 \(p_{X}(x_i), p_{Y}(y_j)\), 那么
            \[p_{Y}(y_j)=\sum_{i=1}^{n} p_{X}(x_i)p_Y(y_j\mid x_i). \]
        \end{thm}

        \begin{thm}
            []
            {离散型随机向量的贝叶斯公式}
            []
            [Dedicatia]
            设\离散型随机向量{\(2\)} \((X,Y)\) 的边缘分布列为 \(p_{X}(x_i), p_{Y}(y_j)\), 那么
            \[p_{X}(x_i\mid y_j)=\frac{p_{X}(x_i)p_Y(y_j\mid x_i)}{\sum_{i=1}^{n} p_{X}(x_i)p_Y(y_j\mid x_i)}. \]
        \end{thm}

        \begin{dfn}
            []
            {连续型随机向量的条件分布}
            []
            [Dedicatia]
            设\连续型随机向量{\(2\)} \((X,Y)\) 的联合密度函数为 \(f_{X,Y}(x,y)\), \(X\) 的边缘密度函数为 \(f_{X}(x)\), 定义其在 \(X=x\) 的条件下 \(Y\) 的概率密度函数为
            \[f_{Y}(y\mid x)=\frac{f_{X,Y}(x,y)}{f_{X}(x)}.\]
        \end{dfn}

        \begin{dfn}
            []
            {条件期望}
            [Conditional Expectation]
            [Dedicatia]
            设\随机向量{\(2\)} \((X,Y)\) 的条件分布列为 \(p_{Y}(y_j\mid x_i)\), 定义 \(Y\) 在 \(X=x_i\) 条件下的期望为
            \[\Expct{Y\mid X=x_i}=\sum_{j} y_j p_{Y}(y_j\mid x_i).\]
            如果 \(X,Y\) 是\离散型随机变量, 则定义 \(Y\) 关于 \(X\) 的条件期望为
            \[\Expct{Y\mid X=x_i}=\sum_{j} y_j p_{Y}(y_j\mid X).\]
        \end{dfn}

        \begin{rmk}
            [Dedicatia]
            条件期望是一个随机变量, 它的取值依赖于条件随机变量 \(X\) 的取值. 
        \end{rmk}

        \begin{ppt}
            []
            {多维离散型随机变量的全期望公式}
            []
            [Dedicatia]
            设\离散型随机向量{\(2\)} \((X,Y)\) 的条件分布列为 \(p_{Y}(y_j\mid x_i)\), 那么
            \[\Expct{Y}=\sum_{i} p_{X}(x_i)\Expct{Y\mid X=x_i}.\]
        \end{ppt}

        \begin{ppt}
            []
            {条件期望的期望为无条件期望 / 是常数}
            []
            [Dedicatia]
            设\离散型随机向量{\(2\)} \((X,Y)\) 的条件分布列为 \(p_{Y}(y_j\mid x_i)\), 那么
            \[\Expct{\Expct{Y\mid X}}=\Expct{Y}.\]
        \end{ppt}

        \begin{ppt}
            []
            {相互独立的随机变量的条件期望是无条件期望 / 是常数}
            []
            [Dedicatia]
            设\离散型随机向量{\(2\)} \((X,Y)\) 独立, 那么
            \[\Expct{Y\mid X}=\Expct{Y}.\]
        \end{ppt}

    \subsection{多维随机变量的像}

        \begin{thm}
            []
            {多维随机变量的像的分布}
            []
            [Dedicatia]
            设 \((X_1,\cdots,X_n)\) 是\随机向量{\(n\)}, \(g(x_1,\cdots,x_n):\R^n\to\R\), 令 \(Z:=g(X_1,\cdots,X_n)\), 那么 \(Z\) 的分布函数为
            \begin{itemize}
                \item 离散型: \[F_Z(z)=\sum_{g(x_1,\cdots,x_n)\leq z}\mathbb{P}(X_1=x_1,\cdots,X_n=x_n)\]
                \item 连续型, 具有密度函数 \(f\): \[F_Z(z)=\int\cdots\int_{g(x_1,\cdots,x_n)\leq z}f(x_1,\cdots,x_n)\dd{x_1}\cdots\dd{x_n}.\]
            \end{itemize}
        \end{thm}

        \begin{xmp}
            []
            {二维随机变量的和/差的分布}
            []
            []
            设 \((X,Y)\) 是\随机向量{\(2\)}, 定义 \(Z=X\pm Y\), 则 \(Z\) 的分布为
            \begin{itemize}
                \item 离散型: \(\forall z:\R, \mathbb{P}(Z=z)=\sum_{x_i\pm y_j=z}\mathbb{P}(X=x_i,Y=y_j)\)
                \item 连续型, 具有密度函数 \(f_{X,Y}\): 
                    \[f_Z(z) = \begin{cases}
                        &\int_{-\infty}^{+\infty} f_{X,Y}(x, z - x) \dd{x}, \text{ 如果 } Z=X+Y;\\
                        &\int_{-\infty}^{+\infty} f_{X,Y}(x, x - z) \dd{x}, \text{ 如果 } Z=X-Y.
                    \end{cases}\]
            \end{itemize}
        \end{xmp}

        \begin{ppt}
            []
            {Poisson分布具有可加性}
            [the Additivity of Poisson Distribution]
            []
            设一列独立的\随机变量 \(X_1,X_2,\dots,X_n\) 服从参数为 \(\lambda_1,\lambda_2,\dots,\lambda_n\) 的Poisson分布, 则其和 \(Z=\sum_{i=1}^n X_i\) 服从参数为 \(\lambda=\sum_{i=1}^n \lambda_i\) 的Poisson分布.
        \end{ppt}

        \begin{prf}
            设 \(Z=\sum_{i=1}^n X_i\), 则: 
            \[\mathbb{P}(Z=k)=\sum_{x_1+x_2+\cdots+x_n=k}\prod_{i=1}^n\frac{\lambda_i^{x_i}}{x_i!}\exp(-\lambda_i)\]
            其中 \(x_i\) 是 \(X_i\) 的取值. \\
            令 \(x_i=k-y_1-y_2-\cdots-y_{i-1}\), 则: 
            \[\mathbb{P}(Z=k)=\sum_{y_1+y_2+\cdots+y_n=k}\prod_{i=1}^n\frac{\lambda_i^{k-y_1-y_2-\cdots-y_{i-1}}}{(k-y_1-y_2-\cdots-y_{i-1})!}\exp(-\lambda_i)\]
            令 \(y_i=k-z\), 则: 
            \[\mathbb{P}(Z=k)=\sum_{z=0}^k\prod_{i=1}^n\frac{\lambda_i^{z}}{z!}\exp(-\lambda_i)\]
            由 Poisson 分布的性质, 得: 
            \[\mathbb{P}(Z=k)=\frac{\lambda^k}{k!}\exp(-\lambda)\]
            其中 \(\lambda=\sum_{i=1}^n \lambda_i\).
        \end{prf}

        \begin{ppt}
            []
            {卡方分布具有可加性}
            [the Additivity of Chi-Square Distribution]
            [Dedicatia]
            若\随机变量 \( X \sim \chi^2(m) \), \( Y \sim \chi^2(n) \), 且 \( X \) 与 \( Y \) 相互独立,则:
            \[ X + Y \sim \chi^2(m + n)\]
            可以推广到任意可数多个独立的卡方分布随机变量之和仍服从卡方分布.
        \end{ppt}

        \begin{ppt}
            []
            {正态分布具有可加性}
            [the Additivity of Normal Distribution]
            [Dedicatia]
            设一列独立的\随机变量 \(X_1,X_2,\dots,X_n\) 满足
            \[X_i\sim N(\mu_i,\sigma^2_i)\]
            那么其和也服从\正态分布 :
            \[Z=\sum_{i=1}^n\sim N\qty(\sum_{i=1}^n \mu_i, \sum_{i=1}^n \sigma^2_i).\]
        \end{ppt}

        \begin{cxmp}
            []
            {不独立的正态随机变量不一定可加}
            []
            [Dedicatia]
            反例如: \((X,Y)\) 具有如下概率密度函数:
            \[f(x,y)=\frac{1}{2\pi}\exp(-\frac{x^2+y^2}{2})(1+\sin x+\sin y)\]
            求取其边缘分布可知 \(X\sim N(0,1), Y\sim N(0,1)\). 但随机变量 \(Z:=X+Y\) 的概率密度函数为
            \[f_Z(z)=\frac{1}{\sqrt{2\pi}}\exp(-\frac{u^2}{4})\qty(1-\frac{\cos u}{2}+\frac{\mathrm{e}^{-1}}{2}).\]
        \end{cxmp}

    \subsection{顺序统计量的分布}

        \begin{dfn}
            []
            {顺序统计量}
            [Order Statistics]
            [Dedicatia]
            设 \(X_1,X_2,\dots,X_n\) 是一组\随机变量, 记其从小到大的排列为 \(X_{(1)},X_{(2)},\dots,X_{(n)}\), 则定义 \(X_{(k)}\) 为 \(X_{i}\) 中的第 \(k\) 个\textbf{顺序统计量}.
        \end{dfn}

        \begin{thm}
            []
            {顺序统计量的分布}
            []
            [Dedicatia]
            设 \(X_1,X_2,\dots,X_n\) 是一组独立同分布的\随机变量, 其公共的分布函数为 \(F(x)\), 密度函数为 \(f(x)\). 则第 \(k\) 个顺序统计量 \(X_{(k)}\) 的分布函数与密度函数分别为
            \[F_{X_{(k)}}(x)=\sum_{j=k}^{n}\binom{n}{j}[F(x)]^j[1-F(x)]^{n-j}\]
            \[f_{X_{(k)}}(x)=\frac{n!}{(k-1)!(n-k)!}[F(x)]^{k-1}[1-F(x)]^{n-k}f(x)\]
        \end{thm}

        \begin{crl}
            []
            {所有顺序统计量的联合分布}
            []
            [Dedicatia]
            设 \(X_1,X_2,\dots,X_n\) 是一组独立同分布的\随机变量, 其公共的密度函数为 \(f(x)\). 则顺序统计量 \(X_{(1)},\cdots, X_{(n)}\) 的联合密度函数为
            \[g(x_1,\cdots,x_n)=\begin{cases}
                & n!f(x_1)f(x_2)\cdots f(x_n), \text{ 如果 } x_1<x_2<\cdots<x_n;\\
                & 0, \text{ 其他}.
            \end{cases}\]
        \end{crl}

        \begin{crl}
            []
            {某两个顺序统计量的联合分布}
            []
            [Dedicatia]
            设 \(X_1,X_2,\dots,X_n\) 是一组独立同分布的\随机变量, 其公共的密度函数为 \(f(x)\). 则第 \(i\) 个与第 \(j\) 个顺序统计量 \(X_{(i)}, X_{(j)}\) 的联合密度函数为
            \[g_{X_{(i)},X_{(j)}}(x,y)=\frac{n!}{(i-1)!(j-i-1)!(n-j)!}[F(x)]^{i-1}[F(y)-F(x)]^{j-i-1}[1-F(y)]^{n-j}f(x)f(y)\]
            其中 \(x<y\).
        \end{crl}

        \begin{xmp}
            []
            {最大值与最小值的分布}
            []
            [Dedicatia]
            设 \(X_1,X_2,\dots,X_n\) 是一组\随机变量, 其分布函数为 \(F(x)\), 密度函数为 \(f(x)\). 则最大值 \(X_{(n)}\) 与最小值 \(X_{(1)}\) 的分布函数与密度函数分别为
            \[F_{X_{(n)}}(x)=[F(x)]^n, \quad f_{X_{(n)}}(x)=n[F(x)]^{n-1}f(x)\]
            \[F_{X_{(1)}}(x)=1-[1-F(x)]^n, \quad f_{X_{(1)}}(x)=n[1-F(x)]^{n-1}f(x)\]
        \end{xmp}

        \begin{xmp}
            []
            {均匀分布的顺序统计量}
            []
            [Dedicatia]
            设 \(X_1,X_2,\dots,X_n\) 是一组服从区间 \([0,1]\) 上均匀分布的\随机变量. 则第 \(k\) 个顺序统计量 \(X_{(k)}\) 的密度函数为
            \[f_{X_{(k)}}(x)=\frac{n!}{(k-1)!(n-k)!}x^{k-1}(1-x)^{n-k}, \quad x\in[0,1]\]
            其中 \(X_{(k)}\) 服从参数为 \(k, n-k+1\) 的Beta分布.
        \end{xmp}

    \subsection{概率生成函数与特征函数}
    
        \begin{dfn}
            []
            {离散型随机变量的概率生成函数 / 概率母函数}
            []
            [Dedicatia]
            设 \(X\) 是取值于非负整数集 \(\N\) 上的\随机变量, 定义其\textbf{概率生成函数}为
            \[G_X(t)=\Expct{t^X}=\sum_{k=0}^{+\infty} p_k t^k\]
            其中 \(p_k=\mathbb{P}(X=k)\), \(t\in(-1,1)\). 也简称为生成函数.
        \end{dfn}

        \begin{rmk}
            [Dedicatia]
            从概率生成函数的构造可以看出, 其将概率分布的信息作为系数编码在一个幂级数当中. 取 \(t\in(-1,1)\) 是为了保证收敛性.
        \end{rmk}

        \begin{ppt}
            []
            {生成函数与概率分布的关系}
            []
            [Dedicatia]
            设 \(X\) 是取值于非负整数集 \(\N\) 上的\随机变量, 则其概率生成函数 \(G_X(t)\) 的系数 \(p_k\) 与概率分布满足
            \[p_k=\frac{1}{k!}\left.\frac{\dd^k G_X(t)}{\dd t^k}\right|_{t=0}=\frac{1}{k!}G_X^{(k)}(0).\]
            其中 \(p_k=\mathbb{P}(X=k)\).
        \end{ppt}

        \begin{ppt}
            []
            {生成函数与期望的关系}
            []
            [Dedicatia]
            设 \(X\) 是取值于非负整数集 \(\N\) 上的\随机变量, 则其概率生成函数 \(G_X(t)\) 的导数与期望满足
            \[\Expct{X}=\left.\frac{\dd G_X(t)}{\dd t}\right|_{t=1}=G_X'(1).\]
        \end{ppt}

        \begin{ppt}
            []
            {生成函数与方差的关系}
            []
            [Dedicatia]
            设 \(X\) 是取值于非负整数集 \(\N\) 上的\随机变量, 则其概率生成函数 \(G_X(t)\) 与方差满足
            \[\Var{X}=G_X''(1)+G_X'(1)-[G_X'(1)]^2.\]
        \end{ppt}

        \begin{ppt}
            []
            {相互独立的随机变量的生成函数}
            []
            [Dedicatia]
            设 \(X_1,X_2,\dots,X_n\) 是一组相互独立的\随机变量, 则其概率生成函数满足
            \[G_{X_1+X_2+\cdots+X_n}(t)=G_{X_1}(t)G_{X_2}(t)\cdots G_{X_n}(t).\]
        \end{ppt}

        \begin{xmp}
            []
            {二项分布的生成函数}
            []
            [Dedicatia]
            设\离散型随机变量 \(X\sim B(n,p)\), 则其概率生成函数为
            \[G_X(t)=(1-p+pt)^n.\]
        \end{xmp}

        \begin{xmp}
            []
            {几何分布的生成函数}
            []
            [Dedicatia]
            设\离散型随机变量 \(X\sim G(p)\), 则其概率生成函数为
            \[G_X(t)=\frac{pt}{1-(1-p)t}.\]
        \end{xmp}

        \begin{xmp}
            []
            {Poisson分布的生成函数}
            []
            [Dedicatia]
            设\离散型随机变量 \(X\sim\Poi(\lambda)\), 那么其生成函数为
            \[G_X(s)=\sum_{k=0}^{\infty}s^k\exp(-\lambda)\frac{\lambda^k}{k!}=\exp(-\lambda(1-s)).\]
        \end{xmp}

        \begin{dfn}
            []
            {矩生成函数}
            [Moment-generating Function]
            [Dedicatia]
            设 \(X\) 是任意随机变量. 定义其\textbf{矩生成函数}为:
            \[M_X(t):=\Expct{\exp tX}.\]
            当 \(X\) 为\离散型随机变量 :
            \[M_X(t)=\sum_{k=1}^{n}\exp(tx_k)\mathbb{P}(X=x_k)\]
            当 \(X\) 为\连续型随机变量 具有密度函数 \(f_X(x)\):
            \[M_X(t)=\int_{-\infty}^{+\infty}\exp(tx)f_X(t)\dd{t}\]
        \end{dfn}

        \begin{thm}
            []
            {矩生成函数与各阶原点矩的关系}
            []
            [Dedicatia]
            设 \(X\) 是任意随机变量且具有矩生成函数 \(M_X(t)\). 那么其各阶原点矩可以由矩生成函数求导获得.
            \[\Expct{X^n}=M_X^{(n)}(0).\]
            例如期望:
            \[\Expct{X}=M_X'(0).\]
        \end{thm}

        \begin{xmp}
            []
            {正态分布的矩生成函数}
            []
            [Dedicatia]
            设\随机变量 \(X\sim N(\mu,\sigma^2)\). 那么其矩生成函数为
            \[M_X(t)=\frac{1}{\sqrt{2\pi}}\int_{-\infty}^{+\infty}\exp(tx)\exp(-\frac{(x-\mu)^2}{2\sigma^2})\dd{x}=\exp(\mu t+\frac{t^2\sigma^2}{2}).\quad t\in\R\]
        \end{xmp}

        \begin{rmk}
            [Dedicatia]
            矩生成函数中常要限定 \(t\) 的范围, 否则矩生成函数不存在.
        \end{rmk}

        \begin{dfn}
            [Characteristic-Function]
            {特征函数}
            [Characteristic Function]
            [Dedicatia]
            设 \(X\) 是任意\随机变量. 定义其\textbf{特征函数}为:
            \[\Phi_X(t):=\Expct{\exp(\ii tX)}=\int_{-\infty}^{+\infty}\exp(\ii tx)\dd{F_X(x)}.\]
            如果 \(X\) 为\离散型随机变量 :
            \[\Phi_X(t)=\sum_{k=1}^{n}\exp(\ii tx_k)\mathbb{P}(X=x_k)\]
            如果 \(X\) 为\连续型随机变量 具有密度函数 \(f_X(x)\):
            \[\Phi_X(t)=\int_{-\infty}^{+\infty}\exp(\ii tx)f_X(x)\dd{x}.\]
        \end{dfn}

        \begin{rmk}
            [Dedicatia]
            特征函数总是存在的, 因为 \(|\exp(\ii tX)|=1\).
        \end{rmk}
        
        \begin{rmk}
            [Dedicatia]
            可以注意到, 特征函数与原概率密度函数之间存在 Fourier 变换与反变换的关系.
        \end{rmk}

        \begin{xmp}
            []
            {常见分布的特征函数}
            []
            [Dedicatia]
            \begin{itemize}
                \item 二项分布 \(X\sim B(n,p)\):
                    \[\Phi_X(t)=(1-p+p\exp \ii t)^n.\]
                \item Poisson分布 \(X\sim\Poi(\lambda)\):
                    \[\Phi_X(t)=\exp(\lambda(\exp \ii t-1)).\]
                \item 几何分布 \(X\sim G(p)\):
                    \[\Phi_X(t)=\frac{p\exp \ii t}{1-(1-p)\exp \ii t}.\]
                \item 指数分布 \(X\sim \mathfrak{E}(\lambda)\):
                    \[\Phi_X(t)=\frac{\lambda}{\lambda-\ii t}.\]
                \item 均匀分布 \(X\sim U(a,b)\):
                    \[\Phi_X(t)=\frac{\exp \ii tb-\exp \ii ta}{\ii t(b-a)}.\]
            \end{itemize}
        \end{xmp}

        \begin{thm}
            []
            {随机变量的特征函数与分布函数相互唯一确定}
            []
            [Dedicatia]
        \end{thm}

        \begin{crl}
            []
            {由特征函数求取概率分布}
            []
            [Dedicatia]
            设 \(X\) 是任意\连续型随机变量, 其特征函数为 \(\Phi_X(t)\). 则 \(X\) 的概率密度函数为
            \[f_X(x)=\frac{1}{2\pi}\int_{-\infty}^{+\infty}\exp(-\ii tx)\Phi_X(t)\dd{t}.\]
            如果 \(X\) 是任意\离散型随机变量, 其特征函数为 \(\Phi_X(t)\). 则 \(X\) 的概率分布列为
            \[\mathbb{P}(X=x_k)=\frac{1}{2\pi}\int_{-\pi}^{\pi}\exp(-\ii t x_k)\Phi_X(t)\dd{t}.\]
        \end{crl}

\section{大数定理}

    \subsection{大数定理}
        
        \begin{dfn}
            [Probability-Limit]
            {依概率收敛}
            []
            [猫猫]
            设 \(\{X_n\}_{n=1}^{+\infty}\) 是\随机变量 序列, \(X\) 是\随机变量, 定义 \(\{X_n\}\) \textbf{依概率收敛}于 \(X\), 当且仅当: 
            \[\forall\varepsilon:\R^+, \lim_{n\to+\infty}\mathbb{P}(|X_n-X|<\varepsilon)=1\]

            记作 \(X_n\plim X\). 
        \end{dfn}
        
        \begin{thm}
            []
            {弱大数定理}
            [Weak Law of Large Numbers]
            [猫猫]
            设 \(\{X_n\}_{n=1}^{+\infty}\) 是\随机变量 序列, 
            \begin{enumerate}
                \item \协方差 非正: 
                    \[\forall i,j:\N, i\neq j\implies \Cov{X_i}{X_j}\leq 0\]
                \item 方差受限: 
                    \[\frac{1}{n^2}\sum_{i=1}^n\Var{X_i}\plim 0\]
            \end{enumerate}
            
            则: 
            \[\frac{1}{n}\sum_{i=1}^n(X_k-\Expct{X_k})\plim 0\]
        \end{thm}

        \begin{crl}
            []
            {}
            []
            []
            设 \(\{X_n\}_{n=1}^{+\infty}\) 是\随机变量 序列, 
            \begin{enumerate}
                \item \(\forall i,j:\N, i\neq j\implies X_i,X_j\) 独立; 

                \item \(\exists F:\R\to[0,1], \forall n:\N, X_n\) 的\分布函数 为 \(F\); 

                \item 方差有限: 
                    \[\exists M:\R, \forall n:\N, \Var{X_n}\leq M\]
            \end{enumerate}

            则: 
            \[\frac{1}{n}\sum_{i=1}^n X_i\plim\mu\]
        \end{crl}
        
        \begin{thm}
            []
            {大数定理}
            [Law of Large Numbers]
            []
            设 \(X_1,X_2,\dots,X_n\) 是\随机变量, \(\forall i,j:\N, i\neq j\implies X_i,X_j\) 独立, 其期望为 \(\mu\), 方差为 \(\sigma^2\), 则: 
        \end{thm}

\end{document}