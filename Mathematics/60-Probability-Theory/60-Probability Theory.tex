\documentclass[UTF8]{ctexart}

\makeatletter
\def\input@path{{../../Fulcrum-Template/}{../../Operator-List/}}
\makeatother

\usepackage{FulcrumCN}
\usepackage{OperatorListCN}
\usepackage{F4Logic}
\usepackage{F4Set}
\usepackage{F4Analysis}
\usepackage{F4Probabilities}

% margin
\usepackage{geometry}
\geometry{
    paper =a4paper,
    top =3cm,
    bottom =3cm,
    left=2cm,
    right =2cm
}
\linespread{1.2}

\begin{document}

\tableofcontents
\newpage

\section{概率空间}

    \subsection{样本空间与事件}

        \begin{dfn}
            [Sample-Space]
            {样本空间}
            []
            [马驰原, 猫猫]
            设 \(S\) 是类型, 定义\textbf{样本空间} 为 \(S\) 上的有限集. 

            设 \(\Omega\) 是 \(S\) 上的样本空间, \(x:S\), 定义 \(x\) 是 \(\Omega\) 中的\textbf{样本点}, 当且仅当: \(x\in\Omega\). 
        \end{dfn}

        \begin{rmk}
            [猫猫]
            通常使用 \(\Omega\) 表示\样本空间. 
        \end{rmk}

        \begin{dfn}
            [Event]
            {事件}
            [Event]
            [马驰原, 猫猫]
            设 \(\Omega\) 是样本空间, \(S\) 是集合, 定义 \(S\) 是 \(\Omega\) 中的\textbf{事件}当且仅当: \(S\subseteq\Omega\). 
        \end{dfn}

        \begin{rmk}
            [猫猫]
            在概率问题语境中指定\样本空间 后, 不加声明地使用\事件 的概念. 
        \end{rmk}

        \begin{rmk}
            [猫猫]
            在不致混淆的前提下, 将\事件 的交 \(A\cap B\) 简记为 \(AB\). 

            除此之外, \事件 的运算及其性质与\集合 完全相同. 
        \end{rmk}

        \begin{dfn}
            []
            {决定性事件}
            []
            [马驰原, 猫猫]
            设 \(\Omega\) 是样本空间, \(S\) 是 \(\Omega\) 中的随机事件, 

            定义 \(S\) 是 \(\Omega\) 中的\textbf{必然事件}当且仅当: \(S=\Omega\); 

            定义 \(S\) 是 \(\Omega\) 中的\textbf{不可能事件}当且仅当: \(S=\varnothing\); 

            定义 \(S\) 是 \(\Omega\) 中的\textbf{决定性事件}当且仅当: \(S=\Omega\lor S=\varnothing\). 
        \end{dfn}

        \begin{dfn}
            []
            {对立事件 / 逆事件}
            []
            [马驰原]
            设 \(A\) 是事件, 定义 \(A\) 的\textbf{对立事件} 为 \(A^{\mathrm{C}}\), 记为 \(\overline{A}\). 
        \end{dfn}

        \begin{dfn}
            []
            {互斥事件}
            []
            [马驰原, 猫猫]
            设 \(A, B\) 是事件, 定义 \(A\) 与 \(B\) \textbf{互斥}当且仅当: \(A\cap B=\varnothing\). 
        \end{dfn}

        \begin{dfn}
            []
            {和事件}
            []
            [马驰原, 猫猫]
            设 \(A, B\) 是事件, \(A\) 与 \(B\) 互斥, 定义 \(A\) 与 \(B\) 的\textbf{和事件}为 \(A\cup B\), 记为 \(A+B\). 
        \end{dfn}

    \subsection{古典概率}

        \begin{dfn}
            [Classic-Probability]
            {古典概率}
            [Classic Probability]
            [马驰原, 猫猫]
            设 \(\Omega\) 是\样本空间, \(S\) 是\事件, 定义 \(S\) 在 \(\Omega\) 中的\textbf{概率}为: \(\dfrac{\card[S]}{\card[\Omega]}\), 记作 \(\ProbClsc[\Omega]{S}\). 
        \end{dfn}

        \begin{rmk}
            [猫猫]
            在语境中已指定\样本空间 后, 不加声明地对\事件 讨论概率, 记作 \(\ProbClsc{S}\). 
        \end{rmk}

        \begin{ppt}
            []
            {}
            []
            [马驰原]
            设 \(A, B\) 是\事件, 则: 
            \[P(A\cup B)=P(A)+P(B)-P(AB)\]
            
            设 \(A\) 与 \(B\) 互斥, 则: 
            \[P(A+B)=P(A)+P(B)\]
        \end{ppt}
    
    \subsection{排列组合}

        \begin{dfn}
            [Permutation]
            {排列数}
            []
            [马驰原]
            设 \(n, m:\N\), \(m<n\), 定义从 \(n\) 中选 \(m\) 的\textbf{排列数}为: \(\dfrac{n!}{(n-m)!}\), 记作 \(\Perm{n}{m}\) 或 \(\mathrm{A}_n^m\). 
        \end{dfn}

        \begin{ppt}
            []
            {全排列数等于阶乘}
            []
            [猫猫]
            设 \(n:\N\), 则: \(\Perm{n}{n}=n!\). 
        \end{ppt}

        \begin{ppt}
            {排列数的定义动机}
            设 \(S\) 是\集合, \(S\) 是有限集, \(n:\N:=\card[S]\), \(m:\N\), \(m<n\), 则: 
            \[\card[\{f:\{1,\dots,m\}\to S|\单射[f]\}]=\Perm{n}{m}\]
        \end{ppt}

        \begin{prf}
            从 \(n\) 个不同元素 (无放回地) 中依次取出 \(m\) 个元素并排成一列, 称为从 \(n\) 个元素中取出 \(m\) 个元素的排列, 这样的排列的种数记作 \(A_n^m\) , 有\[A_n^m=n(n-1)(n-2)\cdots(n-m+1)=\frac{n!}{(n-m)!}.\]随机抽取时, 得到的各种排列是等可能的. 
        \end{prf}

        \begin{dfn}
            [Combination]
            {组合数}
            [Combination]
            [猫猫]
            设 \(n, m:\N\), 定义 从 \(n\) 中选 \(m\) 的\textbf{组合数}为: 
            \[
            \begin{cases}
            \begin{aligned}
                & \frac{n!}{m!(n-m)!} & n\geq m \\
                & 0 & n<m
            \end{aligned}
            \end{cases}
            \]
            
            记作 \(\binom{n}{m}\) 或 \(\Comb{n}{m}\). 
            
        \end{dfn}

        \begin{ppt}
            []
            {组合数的定义动机}
            []
            [马驰原, 猫猫]
            设 \(S\) 是\集合, \(S\) 是有限集, \(n:\N:=\card[S]\), \(m:\N\), 则: 
            \[\card\{X|X\subseteq S\land\card[X]=m\}=\binom{n}{m}\]
        \end{ppt}

        \begin{prf}
            从 \(n\) 个不同元素 (无放回地) 中依次取出 \(m\) 个元素, 不考虑元素的排列顺序, 称为从 \(n\) 个元素中取出 \(m\) 个元素的组合, 这样的组合的种数记作 \(C_n^m\) , 有: 
            \[C_n^m=\binom{n}{m}=\frac{A_n^m}{m!}=\frac{n!}{m!(n-m)!}\]
        \end{prf}

        \begin{ppt}
            []
            {杨辉三角}
            [Pascal's Triangle]
            [猫猫]
            设 \(n, m:\N\), 则:
            \begin{enumerate}
                \item 杨辉三角左右对称: 
                    \[\binom{n}{m}=\binom{n}{n-m}\]
                \item 杨辉三角的边界总是 \(1\): 
                    \[\binom{n}{0}=\binom{n}{n}=1\]
                \item 杨辉三角中每个组合数等于上方两数之和: 
                    \[\binom{n}{m}=\binom{n-1}{m-1}+\binom{n-1}{m}\]
                \item 杨辉三角中每行之和为 \(2\) 的幂次:
                    \[\sum_{m=0}^{n}\binom{n}{m}=2^n\]
            \end{enumerate}
        \end{ppt}

        \begin{ppt}
            []
            {组合数对角和公式}
            []
            [猫猫]
            设 \(m, n:\N\), 则:
            \[\sum_{i=0}^{m}\binom{n+i}{i}=\binom{n+m+1}{m}\]
        \end{ppt}

        \begin{rmk}
            [猫猫]
            上述和式实际上可以视为杨辉三角中一个对角线上元素的求和. 
        \end{rmk}

        \begin{xmp}
            []
            {}
            []
            [马驰原]
            将 \(n\) 个不同的元素分成 \(r\) 组, 每组的元素个数分别为 \(n_1,n_2,\cdots,n_r\) , 且 \(n_1+n_2+\cdots+n_r=n\), 这样的分组的种数记作 \(P(n_1,n_2,\cdots,n_r)\) , 有
            \[\binom{n}{n_1,n_2,\cdots,n_r}=\frac{n!}{n_1!n_2!\cdots n_r!}\]
        \end{xmp}
    
    % \subsection{几何概型}
        
        % \begin{dfn}
        %     []
        %     {几何概型}
        %     []
        %     [马驰原]
        %     设样本空间 \(\Omega\) 的体积 \(m(\Omega)\) 是正数, 样本点等可能地分布在 \(\Omega\) 中, 且事件 \(A\) 是 \(\Omega\) 的一个子集, 且 \(A\) 的体积是 \(m(A)\) , 则称这种试验为\textbf{几何概率模型}, 简称为\textbf{几何概型}. 事件 \(A\) 发生的概率为\[P(A)=\frac{m(A)}{m(\Omega)}.\]
        % \end{dfn}

        % \begin{dfn}
        %     []
        %     {事件域、可测空间}
        %     []
        %     [马驰原]
        %     设 \(\Omega\) 是一个集合,  \(\mathcal{F}\) 是 \(\Omega\) 的一个子集族, 若 \(\mathcal{F}\) 满足: 

        %     \begin{enumerate}
        %         \item  \(\Omega\in\mathcal{F}\) ;
        %         \item 若 \(A\in\mathcal{F}\) , 则 \(\bar{A}\in\mathcal{F}\) ;
        %         \item 若 \(A_1,A_2,\cdots\in\mathcal{F}\) , 则 \(\bigcup_{i=1}^{\infty}A_i\in\mathcal{F}\) . 
        %     \end{enumerate}

        %     则称 \(\mathcal{F}\) 是 \(\Omega\) 的一个\textbf{事件域}, 称 \(\mathcal{F}\) 中的元素为事件, 称 \((\Omega,\mathcal{F})\) 为\textbf{可测空间}. 
        % \end{dfn}

        % \begin{dfn}
        %     []
        %     {Borel事件域}
        %     []
        %     [马驰原]
        %     设 \(\Omega\) 是 \(n\) 维的实数空间 \(\R^n\) ,  \(\mathcal{B}\) 是 \(\R^n\) 的一个子集族, 若 \(\mathcal{B}\) 满足: 

        %     \begin{enumerate}
        %         \item  \(\R^n\in\mathcal{B}\) ;
        %         \item 若 \(A\in\mathcal{B}\) , 则 \(\bar{A}\in\mathcal{B}\) ;
        %         \item 若 \(A_1,A_2,\cdots\in\mathcal{B}\) , 则 \(\bigcup_{i=1}^{\infty}A_i\in\mathcal{B}\) ;
        %         \item 若 \(A_1,A_2,\cdots\in\mathcal{B}\) , 则 \(\bigcap_{i=1}^{\infty}A_i\in\mathcal{B}\) . 
        %     \end{enumerate}
        %     就称 \(\mathcal{F}\) 是 \(\R\) 的一个\textbf{Borel事件域}. 
        % \end{dfn}

        % \begin{dfn}
        %     []
        %     {测度}
        %     []
        %     [马驰原]
        %     称 \(\mathcal{F}\) 上的函数 \(m\) 是 \(\Omega\) 上的一个\textbf{测度}, 若 \(m\) 满足: 

        %     \begin{enumerate}
        %         \item 非负性: 对于任意 \(A\in\mathcal{F}\) , 有 \(m(A)\geq 0\) ;
        %         \item 空间的可加性: 若 \(A_1,A_2,\cdots\in\mathcal{F}\) 两两互不相容, 则有\[m\left(\bigcup_{i=1}^{\infty}A_i\right)=\sum_{i=1}^{\infty}m(A_i).\]
        %     \end{enumerate}
        % \end{dfn}

        % \begin{dfn}
        %     []
        %     {体积}
        %     []
        %     [马驰原]
        %     设有 \(r\) 维向量空间 \(\mathbb{F}^r=\left\{(x_1,x_2,\cdots,x_i,\cdots,x_r)|x_i\in\mathbb{F}\right\} \) , 对于 \(\mathbb{F}^r\) 的任意子集 \(A\) , 若存在一个测度 \(m\) , 使得 \(m(A)\) 满足: 

        %     \[m(A)=\int_{A}\dd x_1\dd x_2\cdots\dd x_r\]

        %     则称这个测度为体积. 
        % \end{dfn}

    \subsection{概率空间}

        \begin{str}
            [Probability-Space]
            {概率空间}
            [Probability-Space]
            [马驰原, 猫猫]
            设 \(\Omega\) 是类型, 定义 \(\Omega\) 上的\textbf{概率空间}包含以下信息: 
            \begin{enumerate}
                \item \textbf{事件域}: \(\mathscr{F}\); 
                    
                    \Sigma代数[\(\mathscr{F}\)][\(\Omega\)]; 

                \item \textbf{概率测度}: \(\mathbb{P}: \mathscr{F}\to\Icc{0}{1}\); 
                    \begin{enumerate}
                        \item \(\mathbb{P}\) 是 \(\mathscr{F}\) 上的\测度; 
                        \item 规范性: 
                            \[\mathbb{P}(\Omega)=1\]
                    \end{enumerate}
            \end{enumerate}
        \end{str}

        \begin{ppt}
            []
            {空集概率为 \(0\)}
            []
            [马驰原, 猫猫]
            设 \((\Omega,\mathscr{F},\mathbb{P})\) 是\概率空间, 则: \(\mathbb{P}(\varnothing)=0\). 
        \end{ppt}

        \begin{ppt}
            []
            {互补集概率和为 \(1\)}
            []
            [马驰原, 猫猫]
            设 \((\Omega,\mathscr{F},\mathbb{P})\) 是\概率空间, 则: \(\forall A\in\mathscr{F}, \mathbb{P}(A)+\mathbb{P}(\bar{A})=1\). 
        \end{ppt}

        \begin{thm}
            []
            {概率的 Jordan 公式 / 容斥原理}
            []
            [马驰原]
            设 \(A_1,A_2,\cdots,A_n\) 是事件, 设\[p_k=\sum_{1\leq j_1\leq\cdots\leq j_k  }P(A_i A_2\cdots A_k)\], 则有
            \[P(A_1\cup A_2\cup\cdots\cup A_n)=\sum_{k=1}^{n}(-1)^{k-1}p_k.\]
        \end{thm}

        \begin{cxmp}
            []
            {Bertrand 悖论}
            []
            [马驰原]
        \end{cxmp}

        \begin{dfn}
            []
            {概率极限}
            []
            [马驰原]
            设 \((\Omega,\mathscr{F},P)\) 是\概率空间, \(\{A_n\}:\N\to\mathscr{F}\), \(A\in\mathscr{F}\), \(A_n\downarrow\lor A_n\uparrow\), 定义 \(A_n\) 的\textbf{概率极限}为: \(P\left(\bigcup\limits_{i=1}^{\infty}A_i\right)\), 记作 \(\lim\limits_{n\to+\infty}P(A_n)\)
        \end{dfn}
        
        % \begin{thm}
        %     []
        %     {概率的可列可加性定理}
        %     []
        %     [马驰原]
        %     设 \(P\) 是 \((\Omega,\mathcal{F})\) 上的一个概率, 则它具备可列可加性的充要条件为: 它是连续且有限可加的. 
        % \end{thm}

    \subsection{条件概率}

        \begin{dfn}
            [Condition-Probability]
            {条件概率}
            [Condition-Probability]
            [马驰原, 猫猫]
            设 \((\Omega,\mathscr{F},P)\) 是\概率空间, \(A, B\in\mathscr{F}\), \(P(B)>0\), 定义在条件 \(B\) 下 \(A\) 的\textbf{条件概率}为 \(\dfrac{P(AB)}{P(B)}\), 记作 \(P(A|B)\). 
        \end{dfn}

        \begin{thm}
            []
            {全概率公式}
            []
            [马驰原, 猫猫]
            设 \((\Omega,\mathscr{F},P)\) 是\概率空间, \(\{B_n\}:\N\mapsto\mathscr{F}\), \(\forall i,j:\N, i\neq j\implies B_i\cap B_j=\varnothing\), \(\bigcup_{i=1}^{+\infty}B_i=\Omega\), 则: 
            \[\forall A\in\mathscr{F}, P(A)=\sum_{i=1}^{+\infty}P(A|B_i)P(B_i)\]
        \end{thm}

        \begin{prf}
            由于 \(B_i\) 两两互不相容, 所以\[A=\sum_{i=1}^{k}B_iA\]
            由可列可加性可得\[P(A)=\sum_{i=1}^{k}P(B_iA)=\sum_{i=1}^{k}P(A|B_i)P(B_i).\]
        \end{prf}

        \begin{thm}
            []
            {贝叶斯公式}
            []
            [马驰原]
            设事件 \(B_1,B_2,\cdots,B_k\) 是一组互不相容的事件, 且 \(\bigcup_{i=1}^{k}B_i=\Omega\) , 则对于任意事件 \(A\) , 有\[P(B_i|A)=\frac{P(A|B_i)P(B_i)}{\sum_{j=1}^{k}P(A|B_j)P(B_j)}.\]
        \end{thm}

        \begin{prf}
            使用全概率公式即可证明. 
        \end{prf}

    \subsection{事件的独立性}

        \begin{dfn}
            []
            {事件的独立性}
            []
            [马驰原, 猫猫]
            设 \((\Omega,\mathscr{F},\mathbb{P})\) 是\概率空间, \(A,B\in\mathscr{F}\), 定义 \(A\) 与 \(B\) \textbf{独立}, 当且仅当: 
            \[P(AB)=P(A)P(B)\]
        \end{dfn}

        \begin{dfn}
            []
            {独立事件列}
            []
            [马驰原]
            设事件 \(A_1,A_2,\cdots,A_n\) , 若对于任意的 \(1\leq i_1\leq i_2\leq\cdots\leq i_k\) , 有\[P(A_{i_1}A_{i_2}\cdots A_{i_k})=P(A_{i_1})P(A_{i_2})\cdots P(A_{i_k})\], 则称事件 \(A_1,A_2,\cdots,A_n\) 是\textbf{独立事件列}. 
        \end{dfn}

        \begin{ppt}
            []
            {}
            []
            [马驰原]
            设 \(A_1,A_2,\cdots,A_n\) 是独立事件列, 用 \(B_i\) 表示 \(A_i\) 或 \(\bar{A}_i\) , 则 \(B_1,B_2,\cdots,B_n\) 也是独立事件列. 
        \end{ppt}

\section{随机变量与概率分布}

    \subsection{随机变量}

        \begin{dfn}
            [Random-Variable]
            {随机变量}
            [Random Variable]
            [马驰原, 猫猫]
            设 \((\Omega,\mathscr{F},\mathbb{P})\) 是\概率空间, \(X:\mathscr{F}\to\R\), 定义 \(X\) 是 \((\Omega,\mathscr{F},P)\) 上的\textbf{随机变量}, 当且仅当: 
            \[\forall x:\R, \left\{\omega:\Omega|X(\omega)\leq x\right\}\in\mathscr{F}\]
            设 \(P\) 是 \(\R\) 上的一元谓词, 将 \(\mathbb{P}(\{\omega:\Omega|P(X(\omega))\})\) 记作 \(\mathbb{P}(P(X))\). 
        \end{dfn}

        \begin{rmk}
            [猫猫]
            在语境中已指定\概率空间 后, 不加声明地使用\随机变量 的概念. 
        \end{rmk}

    \subsection{离散型随机变量的概率分布列}

        \begin{dfn}
            [Discrete-Random-Variable]
            {离散型随机变量}
            [Discrete Random Variable]
            [马驰原, 猫猫]
            设 \((\Omega,\mathscr{F},\mathbb{P})\) 是\概率空间, \(X\) 是\随机变量, 定义 \(X\) 是\textbf{离散型随机变量}, 当且仅当: 
            \[\card(X(\R))\leq\aleph_0\]
        \end{dfn}

        \begin{dfn}
            [Discrete-Distribution]
            {分布列}
            [Discrete-Distribution]
            [马驰原]
            设 \((\Omega,\mathscr{F},\mathbb{P})\) 是\概率空间, \(X\) 是\离散型随机变量, \(\{x_i\}_{i:\N}:=X(\mathscr{F})\), 定义 \(X\) 的\textbf{分布列}为: 
            \[x_i\mapsto\mathbb{P}(X=x_i)\]

            设 \(p_i := \mathbb{P}(X=x_i)\), 记作: 
            \begin{center}
            \begin{tabular}{c|c|c|c|c|c}
                \(X\) & \(x_1\) & \(x_2\) & \(\cdots\) & \(x_n\) & \(\cdots\) \\ \hline
                \(\mathbb{P}\) & \(p_1\) & \(p_2\) & \(\cdots\) & \(p_n\) & \(\cdots\)
            \end{tabular}
            \end{center}
        \end{dfn}

        \begin{dfn}
            []
            {Bernoulli 分布 / 两点分布}
            [Bernoulli Distribution]
            [猫猫]
            设 \((\Omega, \mathscr{F}, \mathbb{P})\) 是\概率空间, \(X\) 是\离散型随机变量, \(p:\R\), \(p\in\Ioo{0}{1}\), 定义 \(X\) 服从成功率为 \(p\) 的\textbf{Bernoulli分布}, 当且仅当: 
            \begin{enumerate}
                \item \(X\) 为二元取值: 
                    \[X(\mathscr{F})=\{0,1\}\]
                \item 成功率为 \(p\): 
                    \[\mathbb{P}(X=1)=p\]
            \end{enumerate}

            记作 \(X\sim B(1,p)\). 
        \end{dfn}

        \begin{ppt}
            []
            {Bernoulli 分布的分布列}
            []
            [猫猫]
            设 \((\Omega, \mathscr{F}, \mathbb{P})\) 是\概率空间, \(X\) 是\离散型随机变量, \(p:\R\), \(p\in\Ioo{0}{1}\), \(X\sim B(1,p)\) 则 \(X\) 的\分布列 为: 
            \begin{center}
            \begin{tabular}{c|c|c}
                \(X\) & \(0\) & \(1\) \\ \hline
                \(\mathbb{P}\) & \(1-p\) & \(p\)
            \end{tabular}
            \end{center}
        \end{ppt}

        \begin{dfn}
            []
            {\(n\) 重 Bernoulli 分布 / 二项分布}
            [Binary Distribution]
            [马驰原, 猫猫]
            设 \(n:\N\), \((\Omega, \mathscr{F}, \mathbb{P})\) 是\概率空间, \(p:\R\), \(p\in\Ioo{0}{1}\), 定义 \(X\) 服从成功率为 \(p\) 的 \(n\) 重\textbf{二项分布}, 当且仅当: 
            
            ???
            
            记作 \(X\sim B(n,p)\)
        \end{dfn}

        \begin{ppt}
            []
            {二项分布的分布列}
            []
            [马驰原, 猫猫]
            设 \(X\) 是\离散型随机变量, \(X\sim B(n,p)\), 则: 
            \[\forall k:\N, k\in\Icc{0}{n}\implies P(X=k)=\binom{n}{k}p^k(1-p)^{n-k}\]
        \end{ppt}

        \begin{thm}
            []
            {Poisson 近似}
            []
            [马驰原]
            在独立重复试验中, 随机变量 \(X\) 服从二项分布 \(B(n,p_n)\) , \(p_n\) 是某次试验中事件 \(A\) 发生的概率, 且 \(np_n\to\lambda \) . 则当 \(n\to\infty\) 时, 有\[B(k;n,p_n)\to\frac{\lambda^k}{k!}\exp (-\lambda) \]
        \end{thm}

        \begin{prf}
            \[
                \begin{aligned}
                    B(k;n,p_n) &= \binom{n}{k}p_n^k(1-p_n)^{n-k}\\
                    &=\frac{n(n-1)\cdots(n-k+1)}{k!}\cdot\frac{np_n}{n}\left(1-p_n \right)^{n-k}\\
                    &=\frac{(np_n)^k}{k!}\left(1-\frac{1}{n} \right)\left(1-\frac{2}{n} \right)\cdots\left(1-\frac{k-1}{n} \right)\left(1-\frac{np_n}{n} \right)^{n-k}        
                \end{aligned}
            \]
            \[\lim_{n\to\infty}np_n=\lambda,\quad\lim_{n\to\infty}\left(1-\frac{\lambda}{n} \right)^{n-k}=\exp(-\lambda) \]
            取极限, 得\[\lim_{n\to\infty} B(k;n,p_n)=\frac{\lambda^k}{k!}\exp (-\lambda) \]
        \end{prf}

        \begin{dfn}
            []
            {Poisson分布}
            []
            [马驰原, 猫猫]
            设 \(X\) 是\离散型随机变量, 定义 \(X\) 服从\textbf{Poisson分布}, 当且仅当: 
            
            记作 \(X\sim Ps(\lambda)\).
        \end{dfn}

        \begin{ppt}
            []
            {Poisson 分布的分布列}
            []
            []
            \[Ps(k;\lambda)=P(X=k)=\frac{\lambda^k}{k!}\exp (-\lambda)\]
        \end{ppt}

        \begin{dfn}
            []
            {Poisson过程}
            []
            [马驰原]
        \end{dfn}

        \begin{dfn}
            []
            {几何分布}
            []
            [马驰原]
             \(n\) 重Bernoulli试验中, 只会出现“成功”和“失败”两种结果. 
        \end{dfn}

    \subsection{连续型随机变量的概率分布函数}

        \begin{dfn}
            []
            {连续型随机变量}
            []
            [马驰原]
            若随机变量 \(X\) 的取值是一个区间, 则称 \(X\) 是\textbf{连续型随机变量}. 
        \end{dfn}

        \begin{dfn}
            []
            {分布函数}
            []
            [马驰原]
            设 \(X\) 是一个随机变量, 对于任意的实数 \(x\) , 定义\[F(x)=P(X\leq x)\]为 \(X\) 的\textbf{分布函数}. 
        \end{dfn}

        \begin{ppt}
            []
            {分布函数的性质}
            []
            [马驰原]
            \begin{enumerate}
                \item 单调性:  \(F(x)\) 是单调不减的;
                \item 有界性:  \(0\leq F(x)\leq 1\) ;
                \item 左连续性:  \(F(x)\) 是左连续的:  \(F(x-0)=F(x)\) ;
                \item 极限性:  \(\lim_{x\to-\infty}F(x)=0\) ,  \(\lim_{x\to\infty}F(x)=1\) . 
            \end{enumerate}
        \end{ppt}

\end{document}