\documentclass[UTF8, 10pt]{ctexart}
\usepackage{temporalTemplateName}

% Info
\title{Title}
\author{Fulcrum4Math}
\date{\today}

\begin{document}

\begin{center}
    {\LARGE\textbf{数学分析笔记整理}}

    Zhenxuan Luo
\end{center}

\setcounter{section}{-1}
\section{引言}
    这是罗振轩的数分笔记   
    \subsection{引入}
        这里是Tao的引入内容
        \begin{thm}
            {UUID}
            {三角不等式}
            {Triangular Inequality}
            {l3}
            $a , b , \in \mathbb{R} \Rightarrow{} \left| \left| a \right| - \left| b \right| \right| \le \left| a + b \right| \le \left| a \right| + \left| b \right|$
            $a , b , \in \mathbb{R} \Rightarrow{} \left| \left| \vec{a}  \right| - \left| \vec{b} \right| \right| \le \left| \vec{a} + \vec{b} \right| \le \left| \vec{a} \right| + \left| \vec{b} \right|$            
        \end{thm}
        
        \begin{prf}
            sorry
        \end{prf}

        \begin{thm}
            {UUID}
            {平均值不等式}
            {Inequality of Means}
            {l3}
            $ 0 < x_1 , x_2 , \ldots x_n \in \mathbb{R} \Rightarrow \frac{x_1 + x_2 + \ldots + x_n}{n} \ge \sqrt[n]{x_1 x_2\ldots x_n} \ge \frac{n}{\frac{1}{x_1} + \frac{1}{x_2} + \ldots + \frac{1}{x_n}}$
        \end{thm}

        \begin{prf}
            sorry
        \end{prf}

\section{数列极限与实数基本定理}
    \subsection{数列极限}
        \begin{dfn}
            {UUID}
            {数列}
            {Sequence}
            {l3}
            所谓数列,是指按照某一种法则排列的一串有次序的数$x_1 , x_2 , \ldots x_n \ldots$ 通常记为$\{ x_n \}$
        \end{dfn}

        \begin{dfn}
            {UUID}
            {数列极限}
            {Limitation}
            {l3}
            若$\forall \varepsilon > 0 , \exists N \in \mathbb{N}^* : \forall n > N |a_n - a| < \varepsilon$,则称数列$\{ a_n \}$的极限是$a$,记作$\lim\limits_{n \to +\infty } a_n = a$
        \end{dfn}

        \begin{xmp}
            {UUID}
            {例子}
            {example}
            {l3}
            $\lim\limits_{ n \to + \infty }\frac{3 \sqrt{n} + 2}{2 \sqrt{n} + 1} = \frac{3}{2}$
        \end{xmp}

        \begin{prf}
            sorry
        \end{prf}

        \begin{ppt}
            {UUID}
            {唯一性}
            {Singularity}
            {l3}
            若数列收敛,则其极限唯一
        \end{ppt}

        \begin{prf}
            设$\{ x_n \}$收敛于极限$a$与$b$,根据极限的定义$\forall \varepsilon > 0$,
            $$\exists N_1 , \forall n > N_1 : |x_n - a| < \frac{\varepsilon}{2} $$
            $$\exists N_2 , \forall n > N_2 : |x_n - b| < \frac{\varepsilon}{2} $$
            取$N = max \{ N_1 , N_2 \}$,利用三角不等式,则$\forall n > N$,有
            $$|a-b| = |a- x_n + x_n - b| \le |x_n - a| + |x_n - b| < \varepsilon$$
            由$\varepsilon$的任意性,可知$a=b$
        \end{prf}

        \begin{ppt}
            {UUID}
            {有界性}
            {Boundedness}
            {l3}
            若数列收敛,则其必有界
        \end{ppt}

        \begin{prf}
            设$\lim\limits_{n \to + \infty} x_n = a$,由极限定义,对于$\varepsilon = 1,\exists N,\forall n> N : |x_n - a| < 1$.从而得到
            $$|x_n| < 1 + |a| , n>N$$
            取$M = max \{1+|a|,|x_1|,|x_2|,\ldots ,|x_N|\} > 0$,则有
            $$|x_n| \le  M , n = 1,2,\ldots $$
            则$\{ x_n \}$为有界数列
        \end{prf}

        \begin{ppt}
            {UUID}
            {保号性}
            {Preservation of identity}
            {l3}
            1.保号性:设$\lim\limits_{n \to + \infty} a_n = a , \alpha < a <\beta \Rightarrow \exists N , \forall n > N , \alpha < a < \beta$
            \\
            2.逐项比较性:设$\lim\limits_{n \to + \infty} a_n = a , \lim\limits_{n \to + \infty} b_n = b , a < b \Rightarrow \exists N, \forall n > N , a_n < b_n $
            \\
            3.保不等式性:设$\lim\limits_{n \to + \infty} a_n = a , \lim\limits_{n \to + \infty} b_n = b,$,若$\exists N, \forall n > N, a_n<=b_n , a \le b$
        \end{ppt}

        \begin{prf}
            1.令$\varepsilon = min\{a - \alpha , \beta - a \} , \exists N_0 , \forall n > N:|a_n - a| < \varepsilon \Rightarrow \alpha < a < \beta$
            \\
            2.$a < \frac{a+b}{2} < b, \stackrel{1.1.2.3.1}{\Rightarrow} \exists N_0 , \forall n > N, a_n < \frac{a+b}{2} < b_n \Rightarrow a_n < b_n$
            \\
            3.反证法,设$a>b \stackrel{1.1.2.3.2}{\Rightarrow} \exists N_0 , \forall n > N,a_n > b_n $矛盾,得证.
        \end{prf}

        \begin{ppt}
            {UUID}
            {夹逼性}
            {Squeeze theorem}
            {l3}
            若三个数列$\{a_n\} , \{ b_n\} ,\{c_n\},\exists N_0 , \forall n>N_0 , a_n \le b_n \le c_n$且$\lim\limits_{n \to + \infty}a_n = \lim\limits_{n \to + \infty} c_n = a$,$\lim\limits_{n \to + \infty}b_n = a$
        \end{ppt}

        \begin{prf}
            由$\lim\limits_{n \to + \infty}a_n =\lim\limits_{n \to + \infty} c_n = a$,$\forall \varepsilon > 0 ,\exists N_1,N_2,\forall n>N_1 , |a_n - a| < \varepsilon ; \forall n > N_2 , |c_n - a| < \varepsilon$
            \\
            令$N = max\{N_0 , N_1 , N_2\}:\forall \varepsilon > 0 ,\exists N, a-\varepsilon < a_n \le b_n \le c_n < a + \varepsilon \Rightarrow |b_n - a| < \varepsilon$,即$\lim\limits_{n \to + \infty}b_n = a$.
        \end{prf}
        \begin{dfn}
            {UUID}
            {邻域}
            {Neighbourhood}
            {l3}
            以$a$为中心$\varepsilon$为半径的开区间$(a-\varepsilon , a+\varepsilon )$通常被称为$a$的$\varepsilon$邻域,记为$O(a,\varepsilon )$,换句话说,$\{ x_n \}$以$a$为极限的几何意义就是对任意给定的$\varepsilon > 0$,邻域$O(a,\varepsilon )$外至多包含数列的有限项
        \end{dfn}

        \begin{thm}
            {UUID}
            {数列极限的四则运算}
            {Four arithmetic operations of sequence limit}
            {l3}
            设$\lim\limits_{n \to + \infty} x_n =a , \lim\limits_{n \to + \infty} y_n =b$,则:
            \\
            1.$\lim\limits_{n \to + \infty} (x_n + y_n) = \lim\limits_{n \to + \infty} x_n + \lim\limits_{n \to + \infty} y_n = a + b$
            \\
            2.$\lim\limits_{n \to + \infty} (x_n \cdot y_n) = \lim\limits_{n \to + \infty} x_n \cdot \lim\limits_{n \to + \infty} y_n = ab$
            \\
            3.$\lim\limits_{n \to + \infty} (\frac{x_n}{y_n}) = \frac{\lim\limits_{n \to + \infty} x_n}{\lim\limits_{n \to + \infty} y_n} = \frac{a}{b} (b \neq 0)$
        \end{thm}

        \begin{dfn}
            {UUID}
            {无穷小量}
            {Infinitesimal quantity}
            {l3}
            如果数列$\{ x_n \}$的极限为0,则称$\{ x_n \}$为无穷小量
        \end{dfn}
        \begin{ppt}
            {UUID}
            {无穷小量的性质}
            {The property of infinitesimal quantity}
            {l3}
            1.有限个无穷小量的代数和是无穷小量
            \\
            2.常数乘无穷小量仍是无穷小量
            \\
            3.有限个无穷小量的乘积是无穷小量
            \\
            4.无穷小量乘有界量仍是无穷小量
            \\
            5.数列$\{ x_n\}$以$a$为极限,则存在无穷小量$\{ \alpha_n \}$,使得$\forall n \in N^* , x_n = a + \alpha_n$
        \end{ppt}

        \begin{dfn}
            {UUID}
            {无穷大量}
            {Infinite Quantity}
            {l3}
            若$\forall G>0 , \exists N,\forall n > N,|x_n| > G$,则称$\{ x_n \}$为无穷大量
        \end{dfn}

        \begin{ppt}
            {UUID}
            {无穷大量的性质}
            {The property of infinit quantity}
            {l3}
            1.若$\{ x_n \}$是无穷大量,$\{ y_n \}$为有界量,则$\{ x_n \pm y_n \}$是无穷大量
            \\
            2.若$\{ x_n \}$是无穷大量,$\lim\limits_{n \to + \infty} y_n = b \neq 0$,则$\{ x_ny_n \}$和$\{ \frac{x_n}{y_n} \}$是无穷大量
            \\
            3.设$x_n \neq 0$,则$\{ x_n \}$是无穷小量的充要条件是$\{ \frac{1}{x_n} \}$是无穷大量
        \end{ppt}

        \begin{dfn}
            {UUID}
            {子列}
            {subsequence}
            {l3}
            对于数列$\{x_n\}$与一个严格单调增的正整数数列$\{n_k\}$,那么数列$\{x_{n_k}\}$称为$\{x_n\}$的一个子列
        \end{dfn}

        \begin{thm}
            {UUID}
            {}
            {}
            {l3}
            $\lim\limits_{n \to + \infty} x_n =a$的充分必要条件是$\{x_n\}$的任何子列都以$a$为极限
        \end{thm}

        \begin{thm}
            {UUID}
            {}
            {}
            {l3}
             $\lim\limits_{n \to + \infty} x_n =a$的充分必要条件是$\{x_{2k}\}$和$\{x_{2k+1}\}$都以$a$为极限
        \end{thm}

        \begin{thm}
            {UUID}
            {斯托尔茨定理}
            {Stolz's Theorem}
            {l3}
            设$\{a_n\}$是一列严格单调增加且极限为正无穷大的数列,$\{b_n\}$为任意数列,如果 $\lim\limits_{n \to + \infty} \frac{b_n - b_{n-1}}{a_n - a_{n-1}} =a$,则有$\lim\limits_{n \to + \infty} \frac{b_n}{a_n} =a$,这里$a$可以是正无穷、负无穷或实数
        \end{thm}
    \subsection{实数完备性六大定理}
        \begin{dfn}
            {UUID}
            {界}
            {Bound}
            {l3}
            设$S$是一个非空数集,若$\exists M$,使得$\forall x \in S,x\le M$($x\ge M$),则称$M$是数集$S$的一个上界(下界),既有上界又有下界的数集称为有界数集
        \end{dfn}

        \begin{dfn}
            {UUID}
            {上确界}
            {Supremum}
            若$\beta$是数集$S$的一个上界,且满足$\forall \varepsilon > 0 ,\exists x_0 \in S,s.t.x_0>\beta - \varepsilon$,则称$\beta$是数集$S$的上确界,记为$$\beta = \sup S$$
        \end{dfn}

        \begin{dfn}
            {UUID}
            {下确界}
            {Infimum}
            若$\alpha$是数集$S$的一个下界,且满足$\forall \varepsilon > 0 ,\exists y_0 \in S,s.t.y_0<\alpha + \varepsilon$,则称$\alpha$是数集$S$的上确界,记为$$\alpha = \inf S$$
        \end{dfn}
        \begin{thm}
            {UUID}
            {确界原理}
            {Supremum and Infimum Principle}
            {l3}
            非空有上界的数集必有上确界,非空有下界的数集必有下确界
        \end{thm}

        \begin{thm}
            {UUID}
            {单调有界原理}
            {Monotonic Boundedness Principle}
            {l3}
            单调有界数列必定收敛
        \end{thm}

        \begin{thm}
            {UUID}
            {闭区间套定理}
            {Nested Intervals Theorem}
            设闭区间列$\{[a_n , b_n] \}$满足:\\
            (1)对于任意正整数$n \in \mathbb{N}^*$,$[a_{n+1} , b_{n+1}] \subset [a_n , b_n]$\\
            (2)$\lim\limits_{n \to +\infty} (b_n - a_n) = 0$\\
            则在闭区间列中存在唯一的公共点$\xi \in \bigcap\limits_{n=1}^{\infty}[a_n,b_n]$,使得$\lim\limits_{n \to \infty} b_n = \lim\limits_{n \to \infty} a_n = \xi$
        \end{thm}

        \begin{thm}
            {UUID}
            {凝聚定理}
            {Bolzano-Weierstrass Theorem}
            {l3}
            $\mathbb{R}$中的有界数列必有收敛子列
        \end{thm}

        \begin{thm}
            {UUID}
            {有限覆盖原理}
            {Finite Covering Principle}
            {l3}
            sorry
        \end{thm}
\section{}
\end{document}
