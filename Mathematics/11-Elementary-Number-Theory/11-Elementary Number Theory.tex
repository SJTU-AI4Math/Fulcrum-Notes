\documentclass[UTF8]{ctexart}

\makeatletter
\def\input@path{{../../Fulcrum-Template/}{../../Operator-List/}}
\makeatother

\usepackage{FulcrumCN}

\usepackage{OperatorListCN}
\usepackage{F4Set}
\usepackage{F4Logic}
\usepackage{F4NumberTheory}

% margin
\usepackage{geometry}
\geometry{
    paper =a4paper,
    top =3cm,
    bottom =3cm,
    left=2cm,
    right =2cm
}
\linespread{1.2}

\begin{document}
\tableofcontents
\newpage

    \begin{center}
        {\LARGE\textbf{初等数论}}

        Fulcrum4Math
    \end{center}

    \section{整除理论}

        \subsection{整除}
            
            \begin{dfn}
                [Div]
                {整除}
                [Divides]
                [猫猫]
                \DEF*
                    [ \(a, b:\Z\), \(a\neq 0\)]
                    { \(a\) 整除 \(b\) }
                    {\(\exists[n][b=an][\Z]\)}
                    [ \(a\mid b\)]

                \DEF*
                    { \(a\) 不整除 \(b\) }
                    {\(\neg(a\mid b)\)}
                    [ \(a\nmid b\)]
            \end{dfn}
            
            \begin{ppt}
                [DivPO]
                {整除的偏序性}
                [Partial Order of Divisibility]
                [猫猫]
                整除是偏序关系. 
            \end{ppt}
            
            \begin{ppt}
                [DivLinear]
                {整除的线性性}
                [Linearity of Divisibility]
                [猫猫]
                设 \(m:\Z\), \(\{x_n\}:\N\to\Z\), \(\forall n:\N, m\mid x_n\), 则: 
                \[\forall a_n:\N\to\Z, \mid[m][\sum_{i=0}^{n}a_ix_i]\]
            \end{ppt}
            
            \begin{ppt}
                []
                {整除的绝对值性质}
                []
                [猫猫]
                \THM
                    []
                    {}
            \end{ppt}
            
            \begin{thm}
                [DivisonAlgorithm]
                {带余除法}
                []
                [猫猫]
                \THM
                    [ \(a, b:\Z\), \(a>0\)]
                    {\(\existsuniq[r][\existsuniq[n][b=an+r][\Z]][\Z/a\Z]\)}
            \end{thm}
            
            \begin{dfn}
                []
                {不完全商}
                []
                [猫猫]
                \DEF
                    [ \(a, b:\Z\), \(a\neq 0\)]
                    { \(b\) 除以 \(a\) 的不完全商}
                    {由\hyperref[thm:DivisonAlgorithm]{带余除法}给出的 \(n\)}
                    [ \(b/a\)]
            \end{dfn}
            
            \begin{dfn}
                [Remainder]
                {余数}
                [Remainder]
                [猫猫]
                \DEF
                    [ \(a, b:\Z\), \(a>0\)]
                    { \(b\) 除以 \(a\) 的余数}
                    {\(b - a\cdot(b/a)\)}
                    [ \(b\% a\)]
            \end{dfn}

        \subsection{最大公因数}

            \begin{dfn}
                [GCD]
                {最大公因数}
                [Greatest Common Divisor]
                [猫猫]
                \DEF
                    [ \(a, b:\Z\), \(a\neq 0\), \(b\neq 0\)]
                    { \(a\) 和 \(b\) 的最大公因数}
                    {\(\max\{d\oldmid d\mid a \land d\mid b\}\)}
                    [ \(\gcd(a,b)\)]

                \DEF
                    [ \(S\) 是 \(\Z\) 上的集合, \(S\neq\{0\}\)]
                    { \(S\) 的最大公因数}
                    {\(\max\{d\oldmid \forall n\in S, d\mid n\}\)}
                    [ \(\gcd S\)]
            \end{dfn}

            \begin{ppt}
                []
                {}
                []
                [猫猫]
                \THM
                    [ \(a, b, c, q:\Z\), \(q\neq 0\), \(a=bq+c\)]
                    {\(\gcd(a,b)=\gcd(b,c)\)}
            \end{ppt}

            \begin{ppt}
                []
                {}
                []
                [猫猫]
                设 \(a, b, n:\Z\), 则 \(\gcd(an,bn)=n\gcd(a,b)\). 
                设 \(\delta:\Z\), \(\delta\mid\gcd(a,b)\), 则: 
                \[\gcd[\frac{a}{\delta}][\frac{b}{\delta}]=\frac{\gcd(a,b)}{|\delta|}\]
            \end{ppt}

            \begin{thm}
                [EuclideanAlgorithm]
                {Euclid 辗转相除法}
                [Euclidean algorithm]
                [白菜]
                设\(a,b:\Z\), \(a>0,b\geq 0\), 则通过以下递归关系可以计算$\gcd(a,b)$:
                \[
                    r_0=a, r_1=b
                \]
                对$i=1,2,3,\ldots$ 计算商 $q_{i+1}$和余数$r_{i+1}$ 使得
                \[
                    r_{i-1}=r_i\cdot q_{i+1}+r_{i+1}, 0\leq r_{i+1}<r_i
                \]
                直到$r_n=0$, 则$\gcd(a,b)=r_{n-1}$.
            \end{thm}

            \begin{thm}
                [BezoutThm]
                {B\'ezout 定理}
                [B\'ezout's Theorem]
                [猫猫]
                设 \(a,b:\Z\), 则 \(\exists u,v:\Z,ua+vb=\gcd(a,b)\).
            \end{thm}
            
            \begin{rmk}
                [猫猫]
                B\'ezout 定理中有某种``共振''的意味. 互素的两数的各个倍数足够不协调, 以至于它们的线性组合可以铺满所有整数.
            \end{rmk}

            \begin{dfn}
                {互素}
                [Coprime]
                [猫猫]
                \DEF*
                    [ \(a, b:\Z\), \(a\neq 0\lor b\neq 0\)]
                    { \(a\) 与 \(b\) 互素}
                    {\(\gcd(a,b)=1\)}

                \DEF*
                    [ \(S\) 是 \(\Z\) 上的集合, \(S\neq\{0\}\)]
                    { \(S\) 互素}
                    {\(\gcd S=1\)}
            \end{dfn}

            \begin{dfn}
                [LCM]
                {最小公倍数}
                [Least Common Multiple]
                [白菜]
                设$a,b:\Z$, $a\neq 0, b\neq 0$, 定义 $a$ 和 $b$ 的最小公倍数为:
                $\min\{a\oldmid d \land b\mid d\}$, 记作 $\lcm(a,b)$.
                设 $S$ 是 $\Z$ 上的集合, $S\neq \{0\}$, 定义$S$的最小公倍数为:
                $\min\{d\oldmid\forall n\in S, n>0 \land n \mid d \}$, 记作$\lcm S$
            \end{dfn}

            \begin{thm}
                [MulEqualsGCDLCM]
                {两数之积等于其最大公因数与最小公倍数之积}
                []
                [白菜]
                设\(a,b:\Z\), 则
                \[a\cdot b=\gcd(a,b)\cdot \lcm(a,b)\]
            \end{thm}
        
        \subsection{素数与算术基本定理}
        
            \begin{dfn}
                [Prime]
                {素数与合数}
                [Prime Number \& Composite Number]
                [猫猫]
                \DEF*
                    [ \(p:\Z\), \(p>1\)]
                    { \(p\) 是素数}
                    {\(\forall a, b\in\Z, p=ab \Rightarrow a=1 \lor a=p\)}

                \DEF*
                    [ \(n:\Z\), \(n>1\)]
                    { \(n\) 是合数}
                    {\(\exists a, b\in\Z, n=ab \land a\neq 1 \land a\neq n\)}
            \end{dfn}

            \begin{thm}
                [PrimeInfinitude]
                {素数无穷性}
                [Infinitude of Primes]
                [猫猫]
                素数有无穷多个
            \end{thm}

            \begin{ppt}
                [PrimalityOfMinFactor]
                {最小正因数的素性}
                [Primality Of Minimum Factor]
                [白菜]
                设\(a:\Z\), \(a>1\), 则 \(a\) 的最小正因数是素数.
            \end{ppt}

            \begin{ppt}
                [PrimeFactorLessSqrt]
                {合数素因子不大于其平方根}
                []
                [白菜]
                设\(a:\Z\), \(a>1\), 若 \(a\) 是合数, 则 \(a\) 的最小素因子不大于 \(\sqrt{a}\).
            \end{ppt}

            \begin{ppt}
                [PrimeDividesOrCoprime]
                {素数必整除一个数, 或与其互素}
                []
                [白菜]
                设\(p, a:\Z\), \(p\) 是素数, 则 \(p\mid a\) 或 \(p\) 与 \(a\) 互素.
            \end{ppt}

            \begin{thm}
                []
                {算术基本定理}
                [Fundamental Theorem of Arithmetic]
                [猫猫]
                设\(n:\Z\), \(n>1\), 则 \(n\) 可以唯一分解为素数的乘积, 即
                \[
                    n=p_1^{\alpha_1}p_2^{\alpha_2}\cdots p_k^{\alpha_k}
                \]
                其中 \(p_1<p_2<\cdots<p_k\) 是素数, \(\alpha_1,\alpha_2,\ldots,\alpha_k\) 是正整数.
            \end{thm}

            \begin{dfn}
                [pAdicValuationFunction]
                {p-进赋值函数}
                []
                [白菜]
                \DEF
                    [ \(p:\Z\), \(p\) 是素数, \(a:\Z\), \(a>0\)]
                    { \(a\) 的 \(p\)-进赋值}
                    {\(\max\{k|p^k\mid a \land k\geq 0\}\)}
                    [ \(v_p(a)\)]
            \end{dfn}

            \begin{ppt}
                [GCDLCMUnderPrimeFactor]
                {素数分解下的最大公因数与最小公倍数}
                []
                [白菜]
                设\(a, b:\Z\), \(a>1, b>1\), 且
                \[
                    a=p_1^{\alpha_1}p_2^{\alpha_2}\cdots p_k^{\alpha_k}
                \]
                \[
                    b=p_1^{\beta_1}p_2^{\beta_2}\cdots p_k^{\beta_k}
                \]  
                其中 \(p_1<p_2<\cdots<p_k\) 是素数, \(\alpha_1,\alpha_2,\ldots,\alpha_k\) 和 \(\beta_1,\beta_2,\ldots,\beta_k\) 是非负整数. 则
                \[\gcd(a,b)=p_1^{\min(\alpha_1,\beta_1)}p_2^{\min(\alpha_2,\beta_2)}\cdots p_k^{\min(\alpha_k,\beta_k)}\]
                \[\lcm(a,b)=p_1^{\max(\alpha_1,\beta_1)}p_2^{\max(\alpha_2,\beta_2)}\cdots p_k^{\max(\alpha_k,\beta_k)}\]

            \end{ppt}

        \subsection{整数部分}
            
            \begin{dfn}
                [GaussFunction]
                {整数部分 / Gauss 函数}
                [Gauss Function]
                [猫猫]
                \DEF
                    [ \(x:\R\)]
                    { \(x\) 的整数部分 / Gauss 函数}
                    {\(\max\{n|n\in\Z \land n\leq x\}\)}
                    [ \(\lfloor x \rfloor\)]
            \end{dfn}

            \begin{dfn}
                [FractionalPart]
                {小数部分}
                []
                [猫猫]
                \DEF
                    [ \(x:\R\)]
                    { \(x\) 的小数部分}
                    {\(x - \lfloor x \rfloor\)}
                    [ \(\{ x \}\)]
            \end{dfn}
            
            \begin{ppt}
                [DivisionWithRemainderUnderGaussFunction]
                {Gauss函数意义下的带余除法}
                []
                [白菜]
                设\(a, b:\Z\), \(b>0\), 则
                \[a=b\lfloor \frac{a}{b} \rfloor + b\{\frac{a}{b}\}\]
            \end{ppt}

            \begin{thm}
                [LegendreFormula]
                {勒让德公式}
                [Legendre's Formula]
                [白菜]
                设\(n, p:\Z\), \(n\geq 0\), \(p\) 是素数, 则
                \[v_p(n!)=\sum_{i=1}^{\infty}\lfloor\frac{n}{p^i}\rfloor\]
            \end{thm}

            \begin{crl}
                [SimplifiedLegendreFormula]
                {勒让德公式的化简表示}
                []
                [白菜]
                设\(n, p:\Z\), \(n\geq 0\), \(p\) 是素数, 则
                \[v_p(n!)=\frac{n-s_p(n)}{p-1}\]
                其中 \(s_p(n)\) 是 \(n\) 的 \(p\) 进制表示下各位数字之和.
            \end{crl}

            \begin{crl}
                [CombinationIsInteger]
                {组合数为整数}
                []
                [白菜]
                设\(n, k:\Z\), \(n\geq 0\), \(0\leq k\leq n\), 则组合数
                \[\binom{n}{k}=\frac{n!}{k!(n-k)!}\]
                是整数.
            \end{crl}
    \section{不定方程}
        
        \subsection{二元一次不定方程}
        
            \begin{thm}
                []
                {二元一次不定方程整数解系}
                []
                [猫猫]
                \THM
                    [ \(a, b, c, x_0, y_0:\Z\), \(a\neq 0\), \(b\neq 0\), \(ax_0+by_0=c\)]
                    {\[\forall x,y:\Z, ax+by=c\implies\exists k:\Z, 
                    \begin{cases}
                    \begin{aligned}
                        &x=x_0+\frac{b}{\gcd(a,b)}k\\
                        &y=y_0-\frac{a}{\gcd(a,b)}k
                    \end{aligned}
                    \end{cases}\]}*
            \end{thm}

            \begin{thm}
                []
                {二元一次不定方程有解的充要条件}
                []
                [猫猫]
                \THM
                    [ \(a, b, c:\Z\), \(a\neq 0\), \(b\neq 0\)]
                    {\[\exists x,y:\Z, ax+by=c \iff \gcd(a,b)\mid c\]}*
            \end{thm}
            
            \begin{thm}
                [typeofsolutionoflinearDiophantineEquation]
                {二元一次不定方程解的形状}
                []
                [白菜]
                设\(a, b, c:\Z\), \(a\neq 0\), \(b\neq 0\), 且 \(\gcd(a,b)\mid c\). 则二元一次不定方程
                \[ax+by=c\]
                的整数解可以表示为
                \[
                    \begin{cases}
                    \begin{aligned}
                        &x=x_0+\frac{b}{\gcd(a,b)}k\\
                        &y=y_0-\frac{a}{\gcd(a,b)}k
                    \end{aligned}
                    \end{cases}         
                \]
                其中 \(k\in\Z\), \((x_0, y_0)\) 是该方程的一个特解.                    
                
            \end{thm}

            \begin{rmk}
                    特解可以通过扩展欧几里得算法求出.
            \end{rmk}

        \subsection{多元一次不定方程}
            
            \begin{thm}
                [conditionofsolvabilityofmultivariablelineardiophantineequation]
                {多元一次不定方程有解的充要条件}
                []
                [白菜]
                设\(a_1, a_2, \ldots, a_n, N:\Z\), 且不全为零. 则多元一次不定方程
                \[a_1x_1+a_2x_2+\cdots+a_nx_n=N\]
                有整数解的充要条件是
                \[\gcd(a_1, a_2, \ldots, a_n)\mid N\]
            \end{thm}

            \begin{rmk}
                    求解该不定方程时, 可以先求以下二元一次不定方程组的解:
                    \[
                        \begin{cases}
                        \begin{aligned}
                            &a_1x_1+a_2x_2=d_2t_2\\
                            &d_2t_2+a_3x_3=d_3t_3\\
                            &\cdots\\
                            &d_{n-1}t_{n-1}+a_nx_n=N
                        \end{aligned}
                        \end{cases} 
                    \]
                    其中 \(d_i=\gcd(d_{i-1}, a_i)\). 中间变量 \(t_i\) 可被消去, 从而得到原多元一次不定方程的解.
            \end{rmk}
        \subsection{勾股方程}
            \begin{lma}
            [CoprimeProductIsSquareNecessaryCondition]
            {互素积为平方的必要条件}
            []
            [白菜]
            设\(a, b:\Z\), 且 \(a\) 与 \(b\) 互素. 若 \(ab\) 是完全平方数, 则 \(a\) 和 \(b\) 均为完全平方数.
            \end{lma}

            \begin{rmk}
                当我们考虑方程\(x^2+y^2=c^2\)时, 可以不妨假定\(x\)与\(y\)互素. 此时\(x,y\)必定一奇一偶, 因为若都是偶数与互素矛盾, 若都是奇数, 则\(z\)为偶数, 
                则\(x^2+y^2\equiv 2\mod 4\), 但$z^2$模4只能为0或1, 矛盾. 故后面考虑勾股方程的解时, 可以假定\(x,y\)一奇一偶, 且两者互素.
            \end{rmk}

            \begin{thm}
            [InternalSolutionsOfPythagoreanEquation]
            {勾股方程的整数解}
            []
            [白菜]
            设\(x, y, z:\Z\), 且 \(x, y, z>0\), \(\gcd(x,y=1)\), \(2\mid x\). 则方程
            \[x^2+y^2=z^2\]
            的整数解可以表示为
            \[
                \begin{cases}
                \begin{aligned}
                    &x=2mn\\
                    &y=m^2-n^2\\
                    &z=m^2+n^2
                \end{aligned}
                \end{cases}
            \]

            \end{thm}
            
            \begin{prf}
                方程即
                \[(\frac{x}{2})^2=(\frac{z+y}{2})(\frac{z-y}{2})\]
                可知\(\gcd(\frac{z+y}{2},\frac{z-y}{2})=1\), 再由引理即得定理.
            \end{prf}
    \section{同余}
        \subsection{同余的基本概念}
            \begin{dfn}
                [Congruence]
                {同余}
                [Congruence]
                [白菜]
                \DEF*
                    [ \(a, b, m:\Z\), \(m>0\)]
                    { \(a\) 与 \(b\) 模 \(m\) 同余}
                    {\(m\mid (a-b)\)}
                    [ \(a\equiv b \mod m\)]
            \end{dfn}

            \begin{ppt}
                [CongruenceEquivalenceRelation]
                {同余是等价关系}
                [Equivalence Relation of Congruence]
                [白菜]
                \THM
                    {同余关系是等价关系}
            \end{ppt}

            \begin{ppt}
                [CongruenceLinearity]
                {同余的线性性}
                [Linearity of Congruence]
                [白菜]
                \THM
                    [ \(a, b, c, d, m:\Z\), \(m>0\), \(a\equiv b(\mod m)\), \(c\equiv d(\mod m)\)]
                    {\(\begin{cases}
                    \begin{aligned}
                        &a+c\equiv b+d \mod m\\
                        &-a\equiv -b \mod m\\
                    \end{aligned}
                    \end{cases}\)}
            \end{ppt}

            \begin{ppt}
                [CongruenceMultiplicativity]
                {同余的乘法性质}
                [Multiplicativity of Congruence]
                [白菜]
                \THM
                    [ \(a, b, c, d, m:\Z\), \(m>0\), \(a\equiv b\mod m\), \(c\equiv d\mod m\)]
                    {\(ac\equiv bd \mod m\)}
            \end{ppt}

            \begin{dfn}
                [ResidueSystem]
                {剩余系}
                [Residue System]
                [白菜]
                设\(m:\Z\), 定义模 \(m\) 的\textbf{剩余系} 为 \(\Z\) 在模 \(m\) 同余关系下的商类型, 记作\(\ZnZ{m}\).
            \end{dfn}

            \begin{dfn}
                [OperationOnResidueSystem]
                {剩余系上的运算}
                [Operation on Residue System]
                [白菜]
                设\(m:\Z\), 则模 \(m\) 剩余系上的乘法定义为:
                \[  
                    \begin{aligned}
                        \cdot: \Z/m\Z\times \Z/m\Z &\to \Z/m\Z\\
                        (\bar{a}, \bar{b}) &\mapsto \overline{ab}
                    \end{aligned}
                \]
                模 \(m\) 剩余系上的加法定义为:
                \[  
                    \begin{aligned}
                        +: \Z/m\Z\times \Z/m\Z &\to \Z/m\Z\\
                        (\bar{a}, \bar{b}) &\mapsto \overline{a+b}
                    \end{aligned}
                \]
            \end{dfn}

            \begin{thm}
                [ResidueSystemIsCommutativeRing]
                {剩余系上的交换环结构}
                []
                [白菜]
                设\(m:\Z\), 则模 \(m\) 剩余系 \(\Z/m\Z\) 在上述加法与乘法下构成一个交换环.
            \end{thm}

            \begin{rmk}
                [白菜]
                当问题中只有环的运算时, 我们会说"在模\(m\)的意义下", 即将整数视为模\(m\)剩余系中的元素来进行运算.
            \end{rmk}

            \begin{dfn}
                [ReducedResidueSystem]
                {即约剩余系}
                [Reduced Residue System]
                [白菜, 猫猫]
                设 \(m:\Z\), 定义模 \(m\) 的\textbf{即约剩余系}为: 
                \[\SetOf{\bar{a} : \ZnZ{m}}{\gcd(a,m) = 1}\]
            \end{dfn}
            
            \begin{thm}
                []
                {既约剩余系是交换群}
                []
                [猫猫]
            \end{thm}

            % \begin{thm}
            %     []
            %     {互素整数诱导即约剩余系的自同态}
            %     []
            %     [白菜]
            %     设\(m, a:\Z\), 且 \(\gcd(a,m)=1\). \(A\subset \Z/m\Z\), 为模\(m\)的即约剩余系, 则映射
            %     \[ 
            %         \begin{aligned}
            %             f: A &\to A\\
            %             \bar{x} &\mapsto \overline{ax}
            %         \end{aligned}
            %     \]
            %     是模 \(m\) 的即约剩余系到其自身的一个双射.
            % \end{thm}

            \begin{rmk}
                或者说, 将模\(m\)即约剩余系中的每一个元素乘以与\(m\)互素的整数\(a\), 仍然得到模\(m\)即约剩余系中的元素, 且没有重复.
            \end{rmk}
            
            \subsection{欧拉函数与欧拉定理}

            \begin{dfn}
                [Euler-Totient]
                {欧拉函数}
                [Euler Function]
                [白菜]
                定义\textbf{欧拉函数}为:
                \[\card{\SetOf{\bar{a} : \ZnZ{m}}{\gcd(a,m) = 1}}\]

                记作 \(\Euler\). 
            \end{dfn}

            \begin{ppt}
                [NumberOfElementsInReducedResidueSystem]
                {即约剩余系的元素个数为\(\varphi(m)\)}
                []
                [白菜]
                设\(m\), \(A\subset \Z/m\Z\), 为模\(m\)的即约剩余系, 
                则\(|A| = \varphi(m)\).
            \end{ppt}

            \begin{ppt}
                [MultiplicativityOfEulerFunction]
                {欧拉函数的乘性}
                [Multiplicativity of Euler Function]
                [白菜]
                设\(m_1,m_2:\Z\), \(\gcd(m_1,m_2)=1\), 则\(\varphi(m_1m_2) = \varphi(m_1)\varphi(m_2)\).
            \end{ppt}

            \begin{crl}
                []
                {欧拉函数的显式公式}
                []
                [白菜]
                设\(m:\Z\), 则
                \[
                \varphi(m) = m \prod_{p \mid m} \left(1 - \frac{1}{p}\right)
                \]
            \end{crl}

            \begin{thm}
                [EulerTheorem]
                {欧拉定理}
                [Euler's Theorem]
                [白菜]
                设\(m:\Z\), \(a:\Z\), 且 \(\gcd(a,m)=1\), 则
                \[
                    a^{\varphi(m)} \equiv 1 \mod{m}
                \]
            \end{thm}
            
            \begin{rmk}
                [猫猫]
                Lagrange 定理的推论. 
            \end{rmk}

            \begin{crl}
                []
                {费马小定理}
                [Fermat's Little Theorem]
                [白菜]
                设\(p:\Z\), \(a:\Z\), 且 \(p\)为素数, \(\gcd(a,p)=1\), 则
                \[
                    a^{p-1} \equiv 1 \mod{p}
                \]
            \end{crl}

            \begin{thm}
                [WilsonTheorem]
                {威尔逊定理}
                [Wilson's Theorem]
                [白菜]
                设\(m:\Z\), 则 \(m\)为素数当且仅当
                \[
                    (m-1)! \equiv -1 \mod{m}
                \]
            \end{thm}
            
            \begin{rmk}
                [猫猫]
                在剩余类中互逆元配对即可.
            \end{rmk}

    \section{同余方程}
        \subsection{基础概念与一次同余方程}
            % \begin{dfn}
            %     [ModuleEquation]
            %     {同余方程}
            %     [Module Equation]
            %     [白菜]
            %     设\(f(x)=\sum_{i=0}^n a_i x^i\), \(a_i \in \Z\), 则同余方程指的是形如 
            %     \[f(x) \equiv 0 \mod{m}\]
            %     如果对于\(\bar{x}\in \Z/m\Z\), 有\(f(\bar{x}) = 0 \)在\(\Z/m\Z\)上成立. 则称其为上述同余方程的一个解
            % \end{dfn}

            \begin{thm}
                []
                {一次同余方程的解}
                []
                [白菜, 猫猫]
                设\(a,b:\Z\), 则: 
                \begin{enumerate}
                    \item 有解的充要条件: 
                        \[\exists ax \equiv b \mod{m}\iff\gcd(a,m) \mid b\]
                    
                    \item 解数:
                        \[\card\SetOf{x \in \Z/m\Z}{ax \equiv b \mod{m}} = \gcd(a,m)\].
                \end{enumerate}
            \end{thm}

            \begin{crl}
                []
                {乘积为\(1\)的同余方程的解}
                []
                [白菜]
                设\(a,m:\Z\), 则同余方程
                \[
                    ax \equiv 1 \mod{m}
                \]
                有解的充要条件是\(\gcd(a,m) = 1\), 且解在模\(m\)意义下唯一.
            \end{crl}

            \begin{dfn}
                []
                {模\(m\)逆}
                [Inverse modulo \(m\)]
                [白菜]
                设\(a,m:\Z\), \(\gcd(a,m)=1\), 则在模\(m\)意义下, 方程
                \[
                    ax \equiv 1 \mod{m}
                \]
                的唯一解称为\(a\)模\(m\)的逆. 记作\(a^{-1}\). 在模\(m\)意义下, 记\(ab^{-1}=b^{-1}a=\frac{a}{b}\).
            \end{dfn}

            \begin{ppt}
                []
                {模\(m\)的逆兼容有理运算}
                []
                [白菜]
                设\(a_1,a_2,b_1,b_2,m:\Z\), \(\gcd(b_1,m)=\gcd(b_2,m)=1\). 则
                \[
                    \frac{a_1}{b_1} \cdot \frac{a_2}{b_2} = \frac{a_1 a_2}{b_1 b_2} \mod m
                \]

                \[
                    \frac{a_1}{b_1} + \frac{a_2}{b_2} = \frac{a_1 b_2 + a_2 b_1}{b_1 b_2} \mod m
                \] 

            \end{ppt}

            \begin{rmk}
                在求解同余方程
                \[ax+b\equiv 0 \mod{m}\]
                时, 我们可以通过求解方程
                \[
                \frac{a}{\gcd(a,m)} x \equiv \frac{b}{\gcd(a,m)} \mod{\frac{m}{\gcd(a,m)}}
                \]
                的唯一解, 于是得到在方程\(ax+b\equiv 0 \mod{m}\)中的\(\gcd(a,m)\)个解.
                这一步即通过求解\(\frac{a}{\gcd(a,m)} \)模\(\frac{m}{\gcd(a,m)}\)的逆元并消去后完成.
            \end{rmk}

        \subsection{一次同余方程组与中国剩余定理}
            \begin{thm}
                []
                {同余方程组有解的判定}
                []
                [白菜]
                设\(m_1, m_2, \ldots, m_k:\Z\), \(a_1, a_2, \ldots, a_k:\Z\). 则同余方程组
                    \[
                        \begin{cases}
                        \begin{aligned}
                            x &\equiv a_1 \pmod{m_1}\\
                            x &\equiv a_2 \pmod{m_2}\\
                                &\vdots\\
                            x &\equiv a_k \pmod{m_k}
                        \end{aligned}
                        \end{cases}
                    \]
                    有解的充要条件是对于任意的 \(i, j\) 有
                    \[
                        a_i \equiv a_j \mod{\gcd(m_i, m_j)}
                    \]
                    若存在解, 则解在模 \(M = \lcm(m_1, m_2, \ldots, m_k)\) 的意义下唯一.
            \end{thm}

            \begin{thm}
                [ChineseRemainderTheorem]
                {中国剩余定理}
                [Chinese Remainder Theorem]
                [白菜]
                设\(m_1, m_2, \ldots, m_k:\Z\), 且两两互素, \(a_1, a_2, \ldots, a_k:\Z\). 则同余方程组
                    \[
                        \begin{cases}
                        \begin{aligned}
                            x &\equiv a_1 \pmod{m_1}\\
                            x &\equiv a_2 \pmod{m_2}\\
                                &\vdots\\
                            x &\equiv a_k \pmod{m_k}
                        \end{aligned}
                        \end{cases}
                    \]
                    在模 \(M = m_1 m_2 \cdots m_k\) 的意义下有唯一解. 该解可以表示为
                    \[
                        x \equiv \sum_{i=1}^{k} a_i M_i (M_i)^{-1} \pmod{M}
                    \]
                    其中 \(M_i = \frac{M}{m_i}\), \((M_i)^{-1}\) 是 \(M_i\) 模 \(m_i\) 的逆元.
            \end{thm}

        \subsection{高次同余方程}
            \begin{thm}
                []
                {模素数幂同余方程的升幂解法}
                []
                [白菜]
                设 \(p:\Z\), \(p\) 为素数, \(k,n:\Z\), 且 \(k\geq 1,n\geq 1 \). 
                \(f(x)\)为\(n\)次整系数多项式. 则同余方程\(f(x) \equiv 0 \mod{p^k}\)满足\(x \equiv x_0 \mod{p^{k-1}}\)的解为
                \[
                    x \equiv x_0 + t_j p^{k-1} \mod{p^k} (1\leq j \leq r)
                \]
                其中 \(t_j\) 为满足
                \[
                    f'(x_0) t_j \equiv -\frac{f(x_0)}{p^{k-1}} \mod{p}
                \]
                的全体解

            \end{thm}

            \begin{prf}
                考虑泰勒展开
                \[
                    f(x_0 + t p^{k-1}) \equiv f(x_0) + f'(x_0) t p^{k-1} \mod{p^k}  
                \]
                则
                \[
                    f(x_0 + t p^{k-1}) \equiv 0 \mod{p^k} \iff f(x_0) + f'(x_0) t p^{k-1} \equiv 0 \mod{p^k}
                \]
                即
                \[
                    f'(x_0) t \equiv -\frac{f(x_0)}{p^{k-1}} \mod{p}
                \]
                故得证.
            \end{prf}

            \begin{rmk}
                当\(f'(x_0) \not\equiv 0 \mod{p}\)时, 这次升幂可以得到唯一对应解; 
                当\(f'(x_0) \equiv 0 \mod{p}\)且\(-\frac{f(x_0)}{p^{k-1}} \equiv 0 \mod{p}\)时, 
                这次升幂可以得到\(p\)个解; 否则无解.
            \end{rmk}

            \begin{thm}
                []
                {模素数幂同余方程的降幂}
                []
                [白菜]
                设\(p:\Z\)为素数, \(k,n:\Z\), 且 \(k\geq 2,n\geq 1 \).
                \(f(x)\)为\(n\)次整系数多项式. 
                则求解同余方程\(f(x) \equiv 0 \mod{p}\)的解与求解同余方程\(r(x) \equiv 0 \mod{p}\)的解等价.
                其中
                \[
                    r(x) \equiv f(x) \mod{x^{p}-x}
                \]
                为一个次数不超过 \(p-1\) 的多项式.
            \end{thm}

            \begin{prf}
                由于对任意\(x \in \Z\), 有
                \[
                    x^p - x \equiv 0 \mod{p}
                \]
                即证
            \end{prf}

            \begin{thm}
                []
                {多项式按根分解}
                []
                [白菜]
                设\(p:\Z\)为素数, \(n:\Z\), 且 \(n\geq 1 \).    
                \(f(x)\)为\(n\)次整系数多项式. 则在模\(p\)意义下, 有
                \[
                    f(x) \equiv \prod_{i=1}^{k} (x - a_i)^{r_i} g(x) \mod{p}
                \]
                其中 \(a_1, a_2, \ldots, a_k\) 是模 \(p\) 意义下的不同根, \(r_i \geq 1\), 
                且 \(g(x)\) 不含模 \(p\) 意义下的根.
            \end{thm}

            \begin{crl}
                []
                {多项式根的个数上界}
                []
                [白菜]
                设\(p:\Z\)为素数, \(n:\Z\), 且 \(n\geq 1 \).    
                \(f(x)\)为\(n\)次整系数多项式. 则同余方程
                \[
                    f(x) \equiv 0 \mod{p}
                \]
                在模 \(p\) 意义下的不同根的个数不超过 \(n\).
            \end{crl}

            \begin{thm}
                []
                {解数等于次数的条件}
                []
                [白菜]
                设\(p:\Z\)为素数, \(n:\Z\), 且 \(1\leq n \leq p-1\).    
                \(f(x)\)为\(n\)次整系数多项式. 则同余方程
                \[
                    f(x) \equiv 0 \mod{p}
                \]
                在模 \(p\) 意义下有 \(n\) 个不同根的充分必要条件是
                \(f(x)\)除\(x^{p}-x\)的各项系数模 \(p\) 为零.
                也即若
                \[x^p-x=f(x) \cdot g(x)+r(x)(\deg r(x) < n)\]
                则\(r(x)\)各项系数模 \(p\) 为零.
            \end{thm}
    \section{二次剩余}
        \subsection{二次剩余与欧拉判别准则}
            \begin{dfn}
                [QuadraticResidue]
                {二次剩余}
                [Quadratic Residue]
                [白菜]
                设\(p:\Z\)为奇素数, \(a:\Z\), 且 \(\gcd(a,p)=1\). 若同余方程
                \[
                    x^2 \equiv a \mod{p}
                \]
                有解, 则称 \(a\) 为模 \(p\) 的一个\textbf{二次剩余}; 否则称 \(a\) 为模 \(p\) 的一个\textbf{二次非剩余}.
            \end{dfn}
        
            \begin{thm}
                [EulerCriterion]
                {欧拉判别准则}
                [Euler's Criterion]
                [白菜]
                设\(p:\Z\)为奇素数, \(a:\Z\), 且 \(\gcd(a,p)=1\). 则
                \[
                    a \text{ 是模 } p \text{ 的二次剩余} \iff a^{\frac{p-1}{2}} \equiv 1 \mod{p}
                \]
                \[
                    a \text{ 是模 } p \text{ 的二次非剩余} \iff a^{\frac{p-1}{2}} \equiv -1 \mod{p}
                \]
            \end{thm}
        
            \begin{crl}
                []
                {模素数的二次剩余个数}
                []
                [白菜]
                设\(p:\Z\)为奇素数, 则模 \(p\) 的二次剩余共有 \(\frac{p-1}{2}\) 个. 
                且恰为\(1^2, 2^2, \ldots, \left(\frac{p-1}{2}\right)^2\)
            \end{crl}

        \subsection{勒让德符号}
            \begin{dfn}
                [LegendreSymbol]
                {勒让德符号}
                [Legendre Symbol]
                [白菜]
                设\(p:\Z\)为奇素数, \(a:\Z\). 则勒让德符号定义为
                \[
                    \left( \frac{a}{p} \right) =
                    \begin{cases}
                        0 & \text{若 } p \mid a\\
                        1 & \text{若 } a \text{ 是模 } p \text{ 的二次剩余}\\
                        -1 & \text{若 } a \text{ 是模 } p \text{ 的二次非剩余}
                    \end{cases}
                \]
            \end{dfn}

            \begin{ppt}
                []
                {勒让德符号的欧拉判别准则表示}
                []
                [白菜]
                设\(p:\Z\)为奇素数, \(a:\Z\). 则有
                \[
                    \left( \frac{a}{p} \right) \equiv a^{\frac{p-1}{2}} \mod{p}
                \]
            \end{ppt}

            \begin{ppt}
                []
                {勒让德符号的乘性}
                []
                [白菜]
                设\(p:\Z\)为奇素数, \(a,b:\Z\). 则有
                \[
                    \left( \frac{ab}{p} \right) = \left( \frac{a}{p} \right) \left( \frac{b}{p} \right)
                \]
            \end{ppt}

            \begin{ppt}
                []
                {-1的勒让德符号}
                []
                [白菜]
                设\(p:\Z\)为奇素数. 则有
                \[
                    \left( \frac{-1}{p} \right) = (-1)^{\frac{p-1}{2}} =
                    \begin{cases}
                        1 & \text{若 } p \equiv 1 \mod{4}\\
                        -1 & \text{若 } p \equiv 3 \mod{4}
                    \end{cases}
                \]
            \end{ppt}
        \subsection{Gauss引理与二次互反律}
            \begin{thm}
                [GaussLemma]
                {Gauss引理}
                [Gauss' Lemma]
                [白菜]
                设\(p:\Z\)为奇素数, \(a:\Z\), 且 \(\gcd(a,p)=1\). 记集合
                \[
                    S = \{a, 2a, 3a, \ldots, \frac{p-1}{2} a\}
                \]
                中模 \(p\) 意义下的最小正剩余中大于 \(\frac{p}{2}\) 的元素个数为 \(n\). 则有
                \[
                    \left( \frac{a}{p} \right) = (-1)^n
                \]
            \end{thm}

            \begin{crl}
                []
                {2的勒让德符号}
                []
                [白菜]
                设\(p:\Z\)为奇素数. 则有
                \[
                    \left( \frac{2}{p} \right) = (-1)^{\frac{p^2 - 1}{8}} =
                    \begin{cases}
                        1 & \text{若 } p \equiv 1 \text{ 或 } 7 \mod{8}\\
                        -1 & \text{若 } p \equiv 3 \text{ 或 } 5 \mod{8}
                    \end{cases}
                \]
            \end{crl}

            \begin{lma}
                [QuadraticReciprocityLemma]
                {二次互反律引理}
                []
                [白菜] 
                设\(p:\Z\)为奇素数. \(\gcd(p,2a)=1\), 记集合
                \[
                    S = \{a, 2a, 3a, \ldots, \frac{p-1}{2} a\}
                \]
                中模 \(p\) 意义下的最小正剩余中大于 \(\frac{p}{2}\) 的元素个数为 \(n\). 则有
                \[
                    n \equiv \frac{a-1}{2} \cdot \frac{p-1}{2} \mod{2}
                \]
            \end{lma}

            \begin{thm}
                [QuadraticReciprocityLaw]
                {二次互反律}
                [Quadratic Reciprocity Law]
                [白菜]
                设\(p,q:\Z\)为奇素数. 则有
                \[
                    \left( \frac{p}{q} \right) \left( \frac{q}{p} \right) = (-1)^{\frac{p-1}{2} \cdot \frac{q-1}{2}}
                \]
            \end{thm}

        \subsection{Jacobi符号}
            \begin{dfn}
                [JacobiSymbol]
                {Jacobi符号}
                [Jacobi Symbol]
                [白菜]
                设\(n:\Z\)为奇正整数, 且 \(n=p_1^{e_1} p_2^{e_2} \cdots p_k^{e_k}\)为其素因子分解. 
                对于任意\(a:\Z\), 定义Jacobi符号为
                \[
                    \left( \frac{a}{n} \right) = \prod_{i=1}^{k} \left( \frac{a}{p_i} \right)^{e_i}
                \]
                其中 \(\left( \frac{a}{p_i} \right)\) 为勒让德符号.
            \end{dfn}

            \begin{ppt}
                []
                {Jacobi符号的乘性}
                []
                [白菜]
                设\(n, n_1,n_2:\Z\)为奇正整数, \(a,b:\Z\). 则有
                \[
                    \left( \frac{ab}{n} \right) = \left( \frac{a}{n} \right) \left( \frac{b}{n} \right)
                \]
                \[
                    \left( \frac{a}{n_1 n_2} \right) = \left( \frac{a}{n_1} \right) \left( \frac{a}{n_2} \right)
                \]
            \end{ppt}



            \begin{thm}
                []
                {Jacobi符号的二次互反律}
                []
                [白菜]
                设\(m,n:\Z\)为奇正整数. 则有
                \[
                    \left( \frac{m}{n} \right) \left( \frac{n}{m} \right) = (-1)^{\frac{m-1}{2} \cdot \frac{n-1}{2}}
                \]
            \end{thm}

            \begin{rmk}
                此时, 即使\(\left( \frac{a}{n} \right) = 1\), 也不能保证同余方程
                \[x^2 \equiv a \mod{n}\]
                有解. 但是我们一定可以通过Jacobi符号的二次互反律计算legendre符号
            \end{rmk}
        \subsection{合数模的二次剩余}
        \subsection{表素数为二平方和}
    
    \section{代数数}

        \subsection{代数数与代数整数}

            \begin{dfn}
                [Algebraic-Number]
                {代数数}
                [Algebraic Number]
                [猫猫]
                设 \(\xi:\R\), 定义 \(\xi\) 为\textbf{代数数}, 当且仅当: 
                \[\exists f \in \Q[x], f \neq 0, f(\xi) = 0\]

                定义 \(\xi\) 为\textbf{代数整数}, 当且仅当:
                \[\exists f \in \Z[x], f \text{首一}\land f(\xi) = 0\]

                代数数类型记作 \(\A\). 
            \end{dfn}
            
            \begin{ppt}
                [MinimalPolynomial]
                {极小多项式}
                [MinimalPolynomial]
                [猫猫]
                设 \(\xi:\A\), 则: \(\existsuniq f\in\Q[x]\)
                \begin{enumerate}
                    \item \(f\) 首一; 
                    \item \(f(\xi) = 0\);
                    \item 极小:
                    \[\forall g \in \Q[x], g(\xi)=0\implies f \mid g\]
                \end{enumerate}
            \end{ppt}

        \subsection{代数整数环}
        \subsection{代数整数环的理想}



\end{document}