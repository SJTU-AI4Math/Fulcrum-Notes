\documentclass[UTF8]{ctexart}

\makeatletter
\def\input@path{{../../Fulcrum-Template/}{../../Operator-List/}}
\makeatother

\usepackage{FulcrumCN}

\usepackage{OperatorListCN}
\usepackage{F4Logic}
\usepackage{F4Type}
\usepackage{F4NumberTheory}

% margin
\usepackage{geometry}
\geometry{
    paper =a4paper,
    top =3cm,
    bottom =3cm,
    left=2cm,
    right =2cm
}
\linespread{1.2}

\begin{document}
\tableofcontents
\newpage

    \begin{center}
        {\LARGE\textbf{初等数论}}

        Fulcrum4Math
    \end{center}

    \section{整除理论}

        \subsection{整除}
            
            \begin{dfn}
                [Div]
                {整除}
                [Divides]
                [猫猫]
                \DEF*
                    [ \(a, b:\Z\), \(a\neq 0\)]
                    { \(a\) 整除 \(b\) }
                    {\(\exists[n][b=an][\Z]\)}
                    [ \(a\mid b\)]

                \DEF*
                    { \(a\) 不整除 \(b\) }
                    {\(\neg(a\mid b)\)}
                    [ \(a\nmid b\)]
            \end{dfn}
            
            \begin{ppt}
                [DivPO]
                {整除的偏序性}
                [Partial Order of Divisibility]
                [猫猫]
                \THM
                    {整除关系是偏序关系}
            \end{ppt}
            
            \begin{ppt}
                [DivLinear]
                {整除的线性性}
                [Linearity of Divisibility]
                [猫猫]
                \THM
                    {整除关系是线性关系}
            \end{ppt}
            
            \begin{ppt}
                []
                {整除的绝对值性质}
                []
                [猫猫]
                \THM
                    []
                    {}
            \end{ppt}
            
            \begin{thm}
                []
                {余数唯一性}
                []
                [猫猫]
                \THM
                    [ \(a, b:\Z\), \(a>0\)]
                    {\(\existsuniq[r][\existsuniq[n][b=an+r][\Z]][\Z/a\Z]\)}
            \end{thm}
            
            \begin{dfn}
                []
                {不完全商}
                []
                [猫猫]
                \DEF
                    [ \(a, b:\Z\), \(a\neq 0\)]
                    { \(b\) 除以 \(a\) 的不完全商}
                    {}
                    [ \(b/a\)]
            \end{dfn}
            
            \begin{dfn}
                [Remainder]
                {余数}
                [Remainder]
                [猫猫]
                \DEF
                    [ \(a, b:\Z\), \(a>0\)]
                    { \(b\) 除以 \(a\) 的余数}
                    {\(b - a\cdot(b/a)\)}
                    [ \(b\% a\)]
            \end{dfn}

        \subsection{最大公因数}

            \begin{dfn}
                [GCD]
                {最大公因数}
                [Greatest Common Divisor]
                [猫猫]
                \DEF
                    [ \(a, b:\Z\), \(a\neq 0\), \(b\neq 0\)]
                    { \(a\) 和 \(b\) 的最大公因数}
                    {\(\max\{d|d\mid a \land d\mid b\}\)}
                    [ \(\gcd(a,b)\)]

                \DEF
                    [ \(S\) 是 \(\Z\) 上的集合, \(S\neq\{0\}\)]
                    { \(S\) 的最大公因数}
                    {\(\max\{d|\forall n\in S, d\mid n\}\)}
                    [ \(\gcd S\)]
            \end{dfn}

            \begin{ppt}
                []
                {}
                []
                [猫猫]
                \THM
                    [ \(a, b, c, q:\Z\), \(q\neq 0\), \(a=bq+c\)]
                    {\(\gcd(a,b)=\gcd(b,c)\)}
            \end{ppt}

            \begin{ppt}
                []
                {}
                []
                [猫猫]
                \THM
                    [ \(a, b, n:\Z\)]
                    {\(\gcd(an,bn)=n\gcd(a,b)\)}
            \end{ppt}

            \begin{thm}
                [EuclideanAlgorithm]
                {Euclid 辗转相除法}
                [Euclidean algorithm]
                [白菜]
                设\(a,b:\mathbb{Z}\), \(a>0,b\geq 0\), 则通过以下递归关系可以计算$\gcd(a,b)$:
                \[
                    r_0=a, r_1=b
                \]
                对$i=1,2,3,\ldots$ 计算商 $q_{i+1}$和余数$r_{i+1}$ 使得
                \[
                    r_{i-1}=r_i\cdot q_{i+1}+r_{i+1}, 0\leq r_{i+1}<r_i
                \]
                直到$r_n=0$, 则$\gcd(a,b)=r_{n-1}$.
            \end{thm}

            \begin{thm}
                [BezoutThm]
                {Bezout 定理}
                [Bezout's Theorem]
                [猫猫]
                设$a,b:\mathbb{Z}$, 则存在正整数$u,v:\mathbb{Z}$, 使得
                \[ua+vb=\gcd(a,b)\]
            \end{thm}

            \begin{dfn}
                {互素}
                [Coprime]
                [猫猫]
                \DEF*
                    [ \(a, b:\Z\), \(a\neq 0\lor b\neq 0\)]
                    { \(a\) 与 \(b\) 互素}
                    {\(\gcd(a,b)=1\)}

                \DEF*
                    [ \(S\) 是 \(\Z\) 上的集合, \(S\neq\{0\}\)]
                    { \(S\) 互素}
                    {\(\gcd S=1\)}
            \end{dfn}

            \begin{dfn}
                [LCM]
                {最小公倍数}
                [Least Common Multiple]
                [白菜]
                设$a,b:\mathbb{Z}$, $a\neq 0, b\neq 0$, 定义 $a$ 和 $b$ 的最小公倍数为:
                $\min\{a\mid d \land b\mid d\}$, 记作 $\lcm(a,b)$.
                设 $S$ 是 $\mathbb{Z}$ 上的集合, $S\neq \{0\}$, 定义$S$的最小公倍数为:
                $\min\{d|\forall n\in S, n>0 \land n \mid d \}$, 记作$\lcm S$
            \end{dfn}

            \begin{thm}
                [MulEqualsGCDLCM]
                {两数之积等于其最大公因数与最小公倍数之积}
                []
                [白菜]
                设\(a,b:\mathbb{Z}\), 则
                \[a\cdot b=\gcd(a,b)\cdot \lcm(a,b)\]
            \end{thm}
        
        \subsection{素数与算术基本定理}
        
            \begin{dfn}
                [Prime]
                {素数与合数}
                [Prime Number \& Composite Number]
                [猫猫]
                \DEF*
                    [ \(p:\Z\), \(p>1\)]
                    { \(p\) 是素数}
                    {\(\forall a, b\in\Z, p=ab \Rightarrow a=1 \lor a=p\)}

                \DEF*
                    [ \(n:\Z\), \(n>1\)]
                    { \(n\) 是合数}
                    {\(\exists a, b\in\Z, n=ab \land a\neq 1 \land a\neq n\)}
            \end{dfn}

            \begin{thm}
                [PrimeInfinitude]
                {素数无穷性}
                [Infinitude of Primes]
                [猫猫]
                素数有无穷多个
            \end{thm}

            \begin{ppt}
                [PrimalityOfMinFactor]
                {最小正因数的素性}
                [Primality Of Minimum Factor]
                [白菜]
                设\(a:\Z\), \(a>1\), 则 \(a\) 的最小正因数是素数.
            \end{ppt}

            \begin{ppt}
                [PrimeFactorLessSqrt]
                {合数素因子不大于其平方根}
                []
                [白菜]
                设\(a:\Z\), \(a>1\), 若 \(a\) 是合数, 则 \(a\) 的最小素因子不大于 \(\sqrt{a}\).
            \end{ppt}

            \begin{ppt}
                [PrimeDividesOrCoprime]
                {素数必整除一个数, 或与其互素}
                []
                [白菜]
                设\(p, a:\Z\), \(p\) 是素数, 则 \(p\mid a\) 或 \(p\) 与 \(a\) 互素.
            \end{ppt}

            \begin{thm}
                []
                {算术基本定理}
                [Fundamental Theorem of Arithmetic]
                [猫猫]
                设\(n:\Z\), \(n>1\), 则 \(n\) 可以唯一分解为素数的乘积, 即
                \[
                    n=p_1^{\alpha_1}p_2^{\alpha_2}\cdots p_k^{\alpha_k}
                \]
                其中 \(p_1<p_2<\cdots<p_k\) 是素数, \(\alpha_1,\alpha_2,\ldots,\alpha_k\) 是正整数.
            \end{thm}

            \begin{dfn}
                [pAdicValuationFunction]
                {p-进赋值函数}
                []
                [白菜]
                \DEF
                    [ \(p:\Z\), \(p\) 是素数, \(a:\Z\), \(a>0\)]
                    { \(a\) 的 \(p\)-进赋值}
                    {\(\max\{k|p^k\mid a \land k\geq 0\}\)}
                    [ \(v_p(a)\)]
            \end{dfn}

            \begin{ppt}
                [GCDLCMUnderPrimeFactor]
                {素数分解下的最大公因数与最小公倍数}
                []
                [白菜]
                设\(a, b:\Z\), \(a>1, b>1\), 且
                \[
                    a=p_1^{\alpha_1}p_2^{\alpha_2}\cdots p_k^{\alpha_k}
                \]
                \[
                    b=p_1^{\beta_1}p_2^{\beta_2}\cdots p_k^{\beta_k}
                \]  
                其中 \(p_1<p_2<\cdots<p_k\) 是素数, \(\alpha_1,\alpha_2,\ldots,\alpha_k\) 和 \(\beta_1,\beta_2,\ldots,\beta_k\) 是非负整数. 则
                \[\gcd(a,b)=p_1^{\min(\alpha_1,\beta_1)}p_2^{\min(\alpha_2,\beta_2)}\cdots p_k^{\min(\alpha_k,\beta_k)}\]
                \[\lcm(a,b)=p_1^{\max(\alpha_1,\beta_1)}p_2^{\max(\alpha_2,\beta_2)}\cdots p_k^{\max(\alpha_k,\beta_k)}\]   

            \end{ppt}

        \subsection{整数部分}
            
            \begin{dfn}
                [GaussFunction]
                {整数部分 / Gauss 函数}
                [Gauss Function]
                [猫猫]
                \DEF
                    [ \(x:\R\)]
                    { \(x\) 的整数部分 / Gauss 函数}
                    {\(\max\{n|n\in\Z \land n\leq x\}\)}
                    [ \(\lfloor x \rfloor\)]
            \end{dfn}

            \begin{dfn}
                [FractionalPart]
                {小数部分}
                []
                [猫猫]
                \DEF
                    [ \(x:\R\)]
                    { \(x\) 的小数部分}
                    {\(x - \lfloor x \rfloor\)}
                    [ \(\{ x \}\)]
            \end{dfn}
            
            \begin{ppt}
                [DivisionWithRemainderUnderGaussFunction]
                {Gauss函数意义下的带余除法}
                []
                [白菜]
                设\(a, b:\Z\), \(b>0\), 则
                \[a=b\lfloor \frac{a}{b} \rfloor + b\{\frac{a}{b}\}\]
            \end{ppt}

            \begin{thm}
                [LegendreFormula]
                {勒让德公式}
                [Legendre's Formula]
                [白菜]
                设\(n, p:\Z\), \(n\geq 0\), \(p\) 是素数, 则
                \[v_p(n!)=\sum_{i=1}^{\infty}\lfloor\frac{n}{p^i}\rfloor\]
            \end{thm}

            \begin{crl}
                [SimplifiedLegendreFormula]
                {勒让德公式的化简表示}
                []
                [白菜]
                设\(n, p:\Z\), \(n\geq 0\), \(p\) 是素数, 则
                \[v_p(n!)=\frac{n-s_p(n)}{p-1}\]
                其中 \(s_p(n)\) 是 \(n\) 的 \(p\) 进制表示下各位数字之和.
            \end{crl}

            \begin{crl}
                [CombinationIsInteger]
                {组合数为整数}
                []
                [白菜]
                设\(n, k:\Z\), \(n\geq 0\), \(0\leq k\leq n\), 则组合数
                \[\binom{n}{k}=\frac{n!}{k!(n-k)!}\]
                是整数.
            \end{crl}
    \section{不定方程}
        
        \subsection{二元一次不定方程}
        
            \begin{thm}
                []
                {二元一次不定方程整数解系}
                []
                [猫猫]
                \THM
                    [ \(a, b, c, x_0, y_0:\Z\), \(a\neq 0\), \(b\neq 0\), \(ax_0+by_0=c\)]
                    {\[\forall x,y:\Z, ax+by=c\implies\exists k:\Z, 
                    \begin{cases}
                    \begin{aligned}
                        &x=x_0+\frac{b}{\gcd(a,b)}k\\
                        &y=y_0-\frac{a}{\gcd(a,b)}k
                    \end{aligned}
                    \end{cases}\]}*
            \end{thm}

            \begin{thm}
                []
                {二元一次不定方程有解的充要条件}
                []
                [猫猫]
                \THM
                    [ \(a, b, c:\Z\), \(a\neq 0\), \(b\neq 0\)]
                    {\[\exists x,y:\Z, ax+by=c \iff \gcd(a,b)\mid c\]}*
            \end{thm}
            
            \begin{thm}
                []
                {二元一次不定方程解的形状}
                []
                [猫猫]
            \end{thm}

        \subsection{多元一次不定方程}
            
            \begin{thm}
                []
                {}
                []
                []
            \end{thm}

    

    \section{同余方程}

    \section{原根与指标}
        
        



\end{document}