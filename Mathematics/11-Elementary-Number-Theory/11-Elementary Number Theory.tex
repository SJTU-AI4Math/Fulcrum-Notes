\documentclass[UTF8]{ctexart}

\makeatletter
\def\input@path{{../../Fulcrum-Template/}{../../Operator-List/}}
\makeatother

\usepackage{FulcrumCN}

\usepackage{OperatorListCN}
\usepackage{F4Logic}
\usepackage{F4Set}
\usepackage{F4Type}
\usepackage{F4NumberTheory}

% margin
\usepackage{geometry}
\geometry{
    paper =a4paper,
    top =3cm,
    bottom =3cm,
    left=2cm,
    right =2cm
}
\linespread{1.2}

\begin{document}
\tableofcontents
\newpage

    \begin{center}
        {\LARGE\textbf{初等数论}}

        Fulcrum4Math
    \end{center}

    \section{整除理论}

        \subsection{整除}
            
            \begin{dfn}
                [Div]
                {整除}
                [Divides]
                [猫猫]
                \DEF*
                    [ \(a, b:\Z\), \(a\neq 0\)]
                    { \(a\) 整除 \(b\) }
                    {\(\exists[n][b=an][\Z]\)}
                    [ \(a\mid b\)]

                \DEF*
                    { \(a\) 不整除 \(b\) }
                    {\(\neg(a\mid b)\)}
                    [ \(a\nmid b\)]
            \end{dfn}
            
            \begin{ppt}
                [DivPO]
                {整除的偏序性}
                [Partial Order of Divisibility]
                [猫猫]
                \THM
                    {整除关系是偏序关系}
            \end{ppt}
            
            \begin{ppt}
                [DivLinear]
                {整除的线性性}
                [Linearity of Divisibility]
                [猫猫]
                \THM
                    {整除关系是线性关系}
            \end{ppt}
            
            \begin{ppt}
                []
                {整除的绝对值性质}
                []
                [猫猫]
                \THM
                    []
                    {}
            \end{ppt}
            
            \begin{thm}
                []
                {余数唯一性}
                []
                [猫猫]
                \THM
                    [ \(a, b:\Z\), \(a>0\)]
                    {\(\existsuniq[r][\existsuniq[n][b=an+r][\Z]][\Z/a\Z]\)}
            \end{thm}
            
            \begin{dfn}
                []
                {不完全商}
                []
                [猫猫]
                \DEF
                    [ \(a, b:\Z\), \(a\neq 0\)]
                    { \(b\) 除以 \(a\) 的不完全商}
                    {}
                    [ \(b/a\)]
            \end{dfn}
            
            \begin{dfn}
                [Remainder]
                {余数}
                [Remainder]
                [猫猫]
                \DEF
                    [ \(a, b:\Z\), \(a>0\)]
                    { \(b\) 除以 \(a\) 的余数}
                    {\(b - a\cdot(b/a)\)}
                    [ \(b\% a\)]
            \end{dfn}

        \subsection{最大公因数}

            \begin{dfn}
                [GCD]
                {最大公因数}
                [Greatest Common Divisor]
                [猫猫]
                \DEF
                    [ \(a, b:\Z\), \(a\neq 0\), \(b\neq 0\)]
                    { \(a\) 和 \(b\) 的最大公因数}
                    {\(\max\{d|d\mid a \land d\mid b\}\)}
                    [ \(\gcd(a,b)\)]

                \DEF
                    [ \(S\) 是 \(\Z\) 上的集合, \(S\neq\{0\}\)]
                    { \(S\) 的最大公因数}
                    {\(\max\{d|\forall n\in S, d\mid n\}\)}
                    [ \(\gcd S\)]
            \end{dfn}

            \begin{ppt}
                []
                {最大公因数非负}
                []
                [猫猫]
            \end{ppt}

            \begin{ppt}
                []
                {}
                []
                [猫猫]
                \THM
                    [ \(a, b, c, q:\Z\), \(q\neq 0\), \(a=bq+c\)]
                    {\(\gcd(a,b)=\gcd(b,c)\)}
            \end{ppt}

            \begin{ppt}
                []
                {}
                []
                [猫猫]
                \THM
                    [ \(a, b, n:\Z\)]
                    {\(\gcd(an,bn)=n\gcd(a,b)\)}
            \end{ppt}

            \begin{thm}
                []
                {Euclid 辗转相除法}
                []
                [猫猫]
            \end{thm}

            \begin{thm}
                []
                {Bezout 定理}
                [Bezout's Theorem]
                [猫猫]
            \end{thm}

            \begin{dfn}
                {互素}
                [Coprime]
                [猫猫]
                \DEF*
                    [ \(a, b:\Z\), \(a\neq 0\lor b\neq 0\)]
                    { \(a\) 与 \(b\) 互素}
                    {\(\gcd(a,b)=1\)}

                \DEF*
                    [ \(S\) 是 \(\Z\) 上的集合, \(S\neq\{0\}\)]
                    { \(S\) 互素}
                    {\(\gcd S=1\)}
            \end{dfn}

            \begin{dfn}
                []
                {最小公倍数}
                [Least Common Multiple]
                [猫猫]
            \end{dfn}

            \begin{thm}
                []
                {两数之积等于其最大公因数与最小公倍数之积}
                []
                [猫猫]
            \end{thm}
        
        \subsection{素数}
        
            \begin{dfn}
                [Prime]
                {素数与合数}
                [Prime Number \& Composite Number]
                [猫猫]
                \DEF*
                    [ \(p:\Z\), \(p>1\)]
                    { \(p\) 是素数}
                    {\(\forall a, b\in\Z, p=ab \Rightarrow a=1 \lor a=p\)}

                \DEF*
                    [ \(n:\Z\), \(n>1\)]
                    { \(n\) 是合数}
                    {\(\exists a, b\in\Z, n=ab \land a\neq 1 \land a\neq n\)}
            \end{dfn}

            \begin{ppt}
                {}
                []
                [猫猫]
            \end{ppt}

            \begin{ppt}
                {}
                []
                [猫猫]
                \THM
                    []
                    {}
            \end{ppt}

            \begin{ppt}
                {Eratosthenes 筛法}
                []
                [猫猫]
            \end{ppt}

            \begin{thm}
                []
                {素数无穷性}
                [Infinitude of Primes]
                [猫猫]
                \THM
                    {}
                    {}
            \end{thm}

            \begin{thm}
                []
                {算术基本定理}
                [Fundamental Theorem of Arithmetic]
                [猫猫]
            \end{thm}

        \subsection{整数部分}
            
            \begin{dfn}
                []
                {整数部分 / Gauss 函数}
                [Gauss Function]
                [猫猫]
                \DEF
                    [ \(x:\R\)]
                    { \(x\) 的整数部分 / Gauss 函数}
                    {\(\max\{n|n\in\Z \land n\leq x\}\)}
                    [ \(\lfloor x \rfloor\)]
            \end{dfn}

    \section{不定方程}
        
        \subsection{二元一次不定方程}
        
            \begin{thm}
                []
                {二元一次不定方程整数解系}
                []
                [猫猫]
                \THM
                    [ \(a, b, c, x_0, y_0:\Z\), \(a\neq 0\), \(b\neq 0\), \(ax_0+by_0=c\)]
                    {\[\forall x,y:\Z, ax+by=c\implies\exists k:\Z, 
                    \begin{cases}
                    \begin{aligned}
                        &x=x_0+\frac{b}{\gcd(a,b)}k\\
                        &y=y_0-\frac{a}{\gcd(a,b)}k
                    \end{aligned}
                    \end{cases}\]}*
            \end{thm}

            \begin{thm}
                []
                {二元一次不定方程有解的充要条件}
                []
                [猫猫]
                \THM
                    [ \(a, b, c:\Z\), \(a\neq 0\), \(b\neq 0\)]
                    {\[\exists x,y:\Z, ax+by=c \iff \gcd(a,b)\mid c\]}*
            \end{thm}
            
            \begin{thm}
                []
                {二元一次不定方程解的形状}
                []
                [猫猫]
            \end{thm}

        \subsection{多元一次不定方程}
            
            \begin{thm}
                []
                {}
                []
                []
            \end{thm}

    

    \section{同余方程}

    \section{原根与指标}
        
        



\end{document}