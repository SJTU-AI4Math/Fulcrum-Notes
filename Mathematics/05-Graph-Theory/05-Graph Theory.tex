\documentclass[UTF8]{ctexart}

\makeatletter
\def\input@path{{../../Fulcrum-Template/}{../../Operator-List/}}
\makeatother

\usepackage{FulcrumHabitCN}

% ams package
\usepackage{amsfonts}
\usepackage{amssymb}
\usepackage{amsthm}
\usepackage{amsmath}

% margin
\usepackage{geometry}

% \dd
\usepackage{physics}

% Boldface
\usepackage{bm}

% Tikz
\usepackage{tikz}
\usetikzlibrary{calc}

% Gaussian Elimination
\usepackage{gauss}

% Commutative Graph
\usepackage[all]{xy}

% Comment
\usepackage{comment}

\title{Title}
\author{Fulcrum4Math}
\date{\today}

\geometry{
    paper =a4paper,
    top =3cm,
    bottom =3cm,
    left=2cm,
    right =2cm
}

\linespread{1.2}

\begin{document}
    \begin{center}
        {\LARGE\textbf{图论笔记}}

        Fulcrum4Math
    \end{center}

    \tableofcontents

    \newpage

    \section{基本概念} % uuid:"11GT.Cat.Section.SectionName"

        \subsection{图} % uuid:"11GT.Cat.Subsection.SubsectionName"
            
            \begin{dfn}
                []
                {图}
                [Graph]
                [猫猫]
                集合对 \(G:=(V,E)\) 称为是一个\textbf{图}, 若 \(\card(V)\in\N\), \(E\subseteq \N\times V^2\). 其中 \(V\) 中元素称为\textbf{顶点 (Vertices)}, \(E\) 中元素称为\textbf{边 (Edges)}. 

                称边 \(e\in E\) \textbf{连接 (Connect)} 顶点 \(a,b\in V\), 若 \(\exists i\in\N, (i,a,b)\in E\), 此时称顶点 \(a,b\) \textbf{相邻 (Adjacent)}. 

                连接顶点与自身的边称为一个\textbf{自环 (Loop)}.
            \end{dfn}
            
            \begin{dfn}
                []
                {子图}
                [Subgraph]
                [猫猫]
            \end{dfn}
            
            \begin{dfn}
                []
                {诱导子图}
                [Induced Subgraph]
                [猫猫]
            \end{dfn}

        \subsection{简单无向图}

            \begin{dfn}
                []
                {简单图}
                [Simple Graph]
                [猫猫]
                图 \(G:=(V,E)\) 称为是一个\textbf{简单图}, 若 \(G\) 中不存在自环, 且任意两顶点之间至多有一条边. 
            \end{dfn}
            
            \begin{dfn}
                []
                {无向图}
                [Undirected Graph]
                [猫猫]
                图 \(G:=(V,E)\) 称为是一个\textbf{无向图}, 若 \(E\) 对称. 

                此时可将 \(E\) 视为其关于对称关系的商集. 
            \end{dfn}
            
            \begin{dfn}
                []
                {邻域}
                [Neighbourhood]
                [猫猫]
                设 \(G:=(V,E)\) 是简单无向图, 顶点 \(v\in V\), 全体与 \(v\) 相邻的点称为 \(v\) 的\textbf{邻域}, 记作 \(\mathcal{N}(v)\). 
            \end{dfn}
            
            \begin{dfn}
                []
                {度}
                [Degree]
                [猫猫]
            \end{dfn}
            
            \begin{ppt}
                []
                {握手定理}
                [The Handshaking Theorem]
                [猫猫]
            \end{ppt}
            
            \begin{dfn}
                []
                {路径}
                [Path]
                [猫猫]
            \end{dfn}
            
            \begin{dfn}
                []
                {圈}
                [Circuit]
                [猫猫]
                图 \(G:=(V,E)\) 中的\textbf{圈}, 是指一个路径, 其首尾相连, 且不包含重复顶点. 
            \end{dfn}
            
            \begin{dfn}
                []
                {连通图}
                [Connected Graph]
                [猫猫]
            \end{dfn}
            
            \begin{dfn}
                []
                {连通分支}
                [Connected Component]
                [猫猫]
            \end{dfn}

        \subsection{简单有向图}

            \begin{dfn}
                []
                {有向图}
                [Directed Graph]
                [猫猫]
                图 \(G:=(V,E)\) 称为是一个\textbf{有向图}, 若它不是无向图. 
            \end{dfn}
            
            \begin{dfn}
                []
                {出 / 入邻域}
                [In / Out-Neighbourhood]
                [猫猫]
                设 \(G:=(V,E)\) 是简单无向图, 顶点 \(v\in V\), 全体与 \(v\) 相邻的点称为 \(v\) 的\textbf{邻域}, 记作 \(\mathcal{N}(v)\). 
            \end{dfn}

            \begin{dfn}
                []
                {出 / 入度}
                [In / Out-Degree]
                [猫猫]
            \end{dfn}
            
            \begin{ppt}
                []
                {握手定理}
                [The Handshaking Theorem]
                [猫猫]
            \end{ppt}
            
            \begin{dfn}
                []
                {有向路径}
                []
                [猫猫]
            \end{dfn}
            
            \begin{dfn}
                []
                {有向圈}
                [Directed Circuit]
                [猫猫]
            \end{dfn}
            
            \begin{dfn}
                []
                {强连通图}
                [Strongly Connected Graph]
                [猫猫]
            \end{dfn}
            
            \begin{dfn}
                []
                {强连通分支}
                [Strongly Connected Component]
                [猫猫]
            \end{dfn}

        \subsection{Hamilton 路径与圈}

        \subsection{\(n\)-部图与完全图}
            
            \begin{thm}
                []
                {Turan 定理}
                [Turan's Theorem]
                [猫猫]
            \end{thm}

        \subsection{Ramsey 问题}

    \section{树}

        \subsection{基本概念}
            
            \begin{dfn}
                []
                {树}
                [Tree]
                [猫猫]
            \end{dfn}
            
            \begin{ppt}
                []
                {树是最大无圈图}
                []
                [猫猫]
                对树添加边将产生圈. 
            \end{ppt}
            
            \begin{ppt}
                []
                {树是最小连通图}
                []
                [猫猫]
                对树删除边将破坏连通性. 
            \end{ppt}
            
            \begin{ppt}
                []
                {树的边数固定}
                []
                [猫猫]
                设 \(T\) 是一个树, 则 \(\card(E)=\card(V)-1\). 
            \end{ppt}
            
            \begin{dfn}
                []
                {子树}
                [Subtree]
                [猫猫]
            \end{dfn}
            
            \begin{dfn}
                []
                {\(m\)-叉树}
                [\(m\)-ary Tree]
                [猫猫]
            \end{dfn}
            
            \begin{ppt}
                []
                {\(m\)-叉树的节点数可控}
                []
                [猫猫]
            \end{ppt}

        \subsection{有根树}
            
            \begin{dfn}
                []
                {有根树}
                [Rooted Tree]
                [猫猫]
            \end{dfn}
            
            \begin{dfn}
                []
                {高度}
                [Height]
                [猫猫]
            \end{dfn}

        \subsection{最小生成树}
            
            \begin{dfn}
                []
                {生成树}
                [Spanning Tree]
                [猫猫]
            \end{dfn}
            
            \begin{ppt}
                []
                {无向图连通性的生成树判定}
                []
                []
            \end{ppt}
            
            \begin{dfn}
                []
                {最小生成树}
                [Minimal Spanning Tree]
                [猫猫]
            \end{dfn}
            
            \begin{dfn}
                []
                {Prim 算法}
                [Prim's Algorithm]
                [猫猫]
            \end{dfn}
            
            \begin{ppt}
                []
                {Prim 算法正确性}
                []
                [猫猫]
            \end{ppt}
            
            \begin{ppt}
                []
                {Prim 算法完备性}
                []
                [猫猫]
            \end{ppt}
            
            \begin{ppt}
                []
                {Prim 算法复杂度}
                []
                [猫猫]
            \end{ppt}
            
            \begin{dfn}
                []
                {Kruskal 算法}
                [Kruskal's Algorithm]
                [猫猫]
            \end{dfn}

            \begin{ppt}
                []
                {Kruskal 算法正确性}
                []
                [猫猫]
            \end{ppt}

            \begin{ppt}
                []
                {Kruskal 算法完备性}
                []
                [猫猫]
            \end{ppt}

            \begin{ppt}
                []
                {Kruskal 算法复杂度}
                []
                [猫猫]
            \end{ppt}

        \subsection{树的遍历}
            
            \begin{dfn}
                []
                {深度优先搜索}
                [Depth First Search (DFS)]
                [猫猫]
            \end{dfn}
            
            \begin{ppt}
                []
                {回边总是连接祖孙节点}
                []
                [猫猫]
            \end{ppt}

        \subsection{Huffman 编码}
            
\end{document}