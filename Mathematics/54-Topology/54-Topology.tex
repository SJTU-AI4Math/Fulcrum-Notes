\documentclass[UTF8]{ctexart}

\makeatletter
\def\input@path{{../../Fulcrum-Template/}{../../Operator-List/}}
\makeatother

\usepackage{FulcrumCN}

\usepackage{OperatorListCN}
\usepackage{F4Logic}
\usepackage{F4Set}
\usepackage{F4Topology}
\usepackage{F4Analysis}

\DeclareMathOperator{\st}{\text{s.t. }}

% margin
\usepackage{geometry}
\geometry{
    paper =a4paper,
    top =3cm,
    bottom =3cm,
    left=2cm,
    right =2cm
}
\linespread{1.2}

\begin{document}

\tableofcontents
\newpage         

    \section{拓扑空间与连续函数}

        \subsection{基本概念}

            \begin{dfn}
                [Topological-Space]
                {拓扑与拓扑空间}
                [Topology and Topological Space]
                [猫猫]
                设 \(S\) 是类型, \(\mathcal{T}\) 是集合, \(\mathcal{T}\subseteq\PS[S]\), 定义 \(\mathcal{T}\) 是 \(S\) 上的\textbf{拓扑}, 当且仅当: 
                \begin{enumerate}
                    \item \(\mathcal{T}\) 包含空集, 全集: 
                    \[\varnothing, S\in\mathcal{T}\]
    
                    \item 对任意并运算封闭: 
                    \[\forall[\mathcal{X}][\mathcal{X}\subseteq\mathcal{T}\implies\bigcup\mathcal{X}\in\mathcal{T}]*\]
    
                    \item 对交运算封闭: 
                    \[\forall[X, Y][X, Y\in\mathcal{T}\implies X\cap Y\in\mathcal{T}]*\]
                \end{enumerate}

                定义 \((S, \mathcal{T})\) 为\textbf{拓扑空间}, 当且仅当 \(\mathcal{T}\) 是 \(S\) 上的拓扑. 
            \end{dfn}

            \begin{rmk}
                [猫猫]
                拓扑空间是刻画集合中元素间``亲疏''关系的结构. 
            \end{rmk}
            
            \begin{rmk}
                [猫猫]
                在不引起混淆的情况下, 常将拓扑空间 \((S,\mathcal{T})\) 简记为 \(S\). 
            \end{rmk}

            \begin{dfn}
                [Open-Closed-Set]
                {开集与闭集}
                [Open \& Closed Sets]
                [猫猫]
                设 \((S,\mathcal{T})\) 是\拓扑空间, \(X\) 是集合, \(X\subseteq S\), 
                
                定义 \(X\) 是 \((S,\mathcal{T})\) 上的\textbf{开集} 当且仅当: \(X\in\mathcal{T}\). 

                定义 \(X\) 是 \((S,\mathcal{T})\) 上的\textbf{闭集} 当且仅当: \(S\setminus X\in\mathcal{T}\). 
            \end{dfn}

            \begin{rmk}
                定义集合上的拓扑空间, 只需指定一个幂集上的一元谓词来定义什么是``\开集''. 
            \end{rmk}

            \begin{rmk}
                在语境中能够自然推断所讨论的拓扑空间时, 我们直接说某集合是\开集 或\闭集. 
            \end{rmk}

            \begin{ppt}
                []
                {开集与闭集的运算对称性}
                []
                [猫猫]
                设 \((S,\mathcal{T})\) 是\拓扑空间, \闭集[\(X, Y\)], 则\闭集[ \(X\cup Y\)]. 

                设 \(\mathcal{Z}\) 是集合族, \(\mathcal{Z}\subseteq\mathcal{T}\), \(\forall [Z][Z\in\mathcal{Z}\implies\闭集[Z]]*\), 则: 
                \[\闭集[\bigcap\mathcal{Z}]\]
            \end{ppt}

            \begin{prf}
                证明: 运用DeMorgan公式即证. 
                \(\square\)
            \end{prf}

            \begin{rmk}
                [猫猫]
                \开集 和\闭集 的概念实际上是对称的, 可以通过指定\闭集 来同理定义\拓扑空间. 
            \end{rmk}

            \begin{dfn}
                [Neighbourhood]
                {邻域}
                [Neighbourhood]
                []
                设 \((S,\mathcal{T})\) 是\拓扑空间, \(x:S\), \(N\) 是\集合, \(N\subseteq S\), 定义 \(N\) 是 \(x\) 的\textbf{邻域}, 当且仅当: 
                \[\exists O, \开集[O][(S,\mathcal{T})]\land x\in O\land O\subseteq N\]

                定义 \(x\) 在 \((S,\mathcal{T})\) 上的\textbf{邻域族}为 \(\{N|\邻域[N][x]\}\), 记作 \(\Nbr{x}\). 
            \end{dfn}

            \begin{rmk}
                [猫猫]
                邻域是一个点的``安全区'', 即该点可以在自身周围自由活动而保持在某个开集之内. 
            \end{rmk}
                
            \begin{dfn}
                [Interior]
                {内部}
                [Interior]
                [猫猫]
                设 \((S,\mathcal{T})\) 是\拓扑空间, \(E\) 是\集合, \(E\subseteq S\), 定义 \(E\) 的\textbf{内部}为: 
                \[\bigcup[\{X|\开集[X]\land X\subseteq E\}]\]
                记作 \(\overset{\circ}{E}\) 或 \(\intr E\). 

                设 \(x:S\), 定义 \(x\) 是 \(E\) 的\textbf{内点}当且仅当: \(x\in\intr E\). 
            \end{dfn}

            \begin{rmk}
                [猫猫]
                集合的内部是集合的``最大''开子集, 即包含于该集合的所有\开集 的并. 
            \end{rmk}

            \begin{ppt}
                []
                {集合的内部是开集}
                []
                [猫猫]
                设 \((S,\mathcal{T})\) 是\拓扑空间, \(E\) 是\集合, \(E\subseteq S\), 则: \(\开集[\intr E]\). 
            \end{ppt}

            \begin{ppt}
                []
                {内点的等价定义}
                []
                [猫猫]
                设 \((S,\mathcal{T})\) 是\拓扑空间, \(E\) 是\集合, \(E\subseteq S\), \(x\in E\), 则: 
                \[x\in\intr E\iff E\in\Nbr{x}\]
            \end{ppt}

            \begin{ppt}
                []
                {集合是开集当且仅当其内部为自身}
                []
                [猫猫]
                设 \((S,\mathcal{T})\) 是\拓扑空间, \(E\) 是\集合, \(E\subseteq S\), 则: 
                \[\开集[E]\iff E=\intr E\]
            \end{ppt}

            \begin{dfn}
                [Closure]
                {闭包}
                [Closure]
                [猫猫]
                设 \((S,\mathcal{T})\) 是\拓扑空间, \(E\) 是集合, \(E\subseteq S\), 定义 \(E\) 的\textbf{闭包} 为: 
                \[\bigcap[\{X|\闭集[X]\land E\subseteq X\}]\]

                记作 \(\bar{E}\) 或 \(\cl E\). 

                设 \(x:S\), 定义 \(x\) 是 \(E\) 的\textbf{闭包点}当且仅当: \(x\in\cl E\). 
            \end{dfn}

            \begin{rmk}
                [猫猫]
                集合的闭包是包含该集合的``最小''\闭集, 即包含集合的所有\闭集 的交. 
            \end{rmk}

            \begin{ppt}
                []
                {集合的闭包是闭集}
                []
                [猫猫]
                设 \((S,\mathcal{T})\) 是\拓扑空间, \(E\) 是\集合, \(E\subseteq S\), 则: \(\闭集[\cl E]\). 
            \end{ppt}

            \begin{ppt}
                []
                {闭包点的等价定义}
                []
                [猫猫]
                设 \((S,\mathcal{T})\) 是\拓扑空间, \(E\) 是\集合, \(E\subseteq S\), \(x:S\), 则: 
                \[x\in\cl E\iff\forall U\in\Nbr{x}, U\cap E\neq\varnothing\]
            \end{ppt}

            \begin{ppt}
                []
                {集合是闭集当且仅当其闭包为自身}
                []
                [猫猫]
                设 \((S,\mathcal{T})\) 是\拓扑空间, \(E\) 是\集合, \(E\subseteq S\), 则: 
                \[\闭集[E]\iff E = \cl E\]
            \end{ppt}

            \begin{dfn}
                [Deleted-Neighbourhood]
                {去心邻域}
                [Deleted Neighbourhood]
                [猫猫]
                设 \((S,\mathcal{T})\) 是\拓扑空间, \(x:S\), \(U\) 是\集合, \(U\subseteq S\), 定义 \(U\) 是 \(x\) 的\textbf{去心邻域}, 当且仅当:
                \[U\cup\{x\}\in\Nbr{x}\land x\notin U\]

                定义 \(x\) 在 \((S,\mathcal{T})\) 上的\textbf{去心邻域族}为 \(\{U|\去心邻域[U][x]\}\), 记作 \(\DelNbr{x}\). 
            \end{dfn}

            \begin{dfn}
                [Limit-Point]
                {极限点 / 聚点}
                [Limit Point / Accumulation Point]
                [猫猫]
                设 \((S,\mathcal{T})\) 是\拓扑空间, \(E\) 是\集合, \(E\subseteq S\), \(x:S\), 定义 \(x\) 是 \(E\) 的\textbf{极限点}, 当且仅当: 
                \[\forall U, U\in\DelNbr{x}\implies U\cap E\neq\varnothing\]
            \end{dfn}

            \begin{dfn}
                [Derived-Set]
                {导集}
                [Derived Set]
                [猫猫]
                设 \((S,\mathcal{T})\) 是\拓扑空间, \(E\) 是\集合, \(E\subseteq S\), 定义 \(E\) 的\textbf{导集}为: \(\{x|\极限点[x][E]\}\), 记作 \(\DrvSet{E}\). 
            \end{dfn}

            \begin{ppt}
                []
                {集合与导集的并等于闭包}
                []
                [猫猫]
                设 \((S,\mathcal{T})\) 是\拓扑空间, \(E\) 是\集合, \(E\subseteq S\), 则:
                \[\cl E=E\cup\DrvSet{E}\]
            \end{ppt}

            \begin{dfn}
                []
                {自密集}
                []
                [猫猫]
            \end{dfn}

            \begin{dfn}
                []
                {完全集}
                []
                [猫猫]
            \end{dfn}

            \begin{dfn}
                [Isolated-Point]
                {孤立点}
                [Isolated Point]
                [猫猫]
                设 \((S,\mathcal{T})\) 是\拓扑空间, \(E\) 是\集合, \(E\subseteq S\), \(x\in E\), 定义 \(x\) 是 \(E\) 的\textbf{孤立点}, 当且仅当: \(x\notin\DrvSet{E}\). 
            \end{dfn}

            \begin{ppt}
                []
                {孤立点的等价定义}
                []
                [猫猫]
                设 \((S,\mathcal{T})\) 是\拓扑空间, \(E\) 是\集合, \(E\subseteq S\), \(x\in E\), 则: 
                \[x\in\cl E\setminus\DrvSet{E}\iff\exists U\in\DelNbr{x}, U\cap E=\varnothing\]
            \end{ppt}
            
            \begin{dfn}
                [Boundary]
                {边界}
                [Boundary]
                [猫猫]
                设\((S,\mathcal{T})\) 是\拓扑空间, \(E\) 是\集合, \(E\subseteq S\), 定义 \(E\) 的\textbf{边界}为: \(\cl E\cap\cl(S-E)\), 记作\(\bd A\). 
            \end{dfn}

            \begin{xmp}
                [RealNumberTopology]
                {实数拓扑}
                [Topology on Real Number]
                [Dedicatia]
                这可能是比较熟悉的例子:\\
                实数$(\R,\tau)$是\拓扑空间 , 其中$\tau$是任意开区间、开区间的任意并和有限交.\\
                其中,开区间, $\varnothing$, $\R$, $\R\backslash\{0\}$, $(0,+\infty)$ 都是\开集 ; $\varnothing$, $\R$, 闭区间, $\Ico{0}{+\infty}$ 等是\闭集 ;\\
                对于集合$E=\Ico{0}{1}$, 其\hyperref[dfn:Interior]{内部}是$(0,1)$, \hyperref[dfn:Derived-Set]{导集}$E'=[0,1]$, \hyperref[dfn:Closure]{闭包}$\cl E=[0,1]$, \hyperref[dfn:Boundary]{边界}是$\{0,1\}$.\\
                实数的拓扑将作为一元微积分的基石。
            \end{xmp}
            \begin{xmp}
                [DiscreteTopology]
                {离散拓扑}
                [Discrete Topology]
                [Dedicatia]
                对任意集合$S$都可以赋予\textbf{离散拓扑}: $(S,\tau)$, 其中$\tau$是$S$的幂集, 也就是所有的子集组成的集合。\\
                在离散拓扑中,任意集合都是\开集 , 任意集合都是\闭集 ;\\
                设$S=\{a,b,c\}$, 对于集合$E=\{a\}$, 其\hyperref[dfn:Interior]{内部}是$\{a\}$, \hyperref[dfn:Derived-Set]{导集}$E'=\varnothing$, \hyperref[dfn:Closure]{闭包}$\cl E=\{a\}$, \hyperref[dfn:Boundary]{边界}是$\varnothing$.\\
                $a$的\邻域 只有一个,就是$\{a\}$. 在$a$的\邻域 中不包含任何异于$a$的点。这表明$S$中所有的元素都不是\hyperref[dfn:Limit-Point]{极限点}.\\
                离散拓扑的本质就是``完全分离'', 没有一个点可以粘在其他点周围。
            \end{xmp}
            \begin{xmp}
                [TrivialTopology]
                {平凡拓扑}
                [Trivial Topology]
                [Dedicatia]
                对任意集合$S$都可以赋予\textbf{平凡拓扑}: $(S,\tau)$, 其中$\tau=\{\varnothing,S\}$. 也就是说取了空集和全集作为\拓扑 .\\
                在平凡拓扑中,\开集 只有$\varnothing$和$S$. \闭集 也只有$\varnothing$和$S$. 其他集合既不开也不闭。\\
                任何$x\in S$, 都只有一个\邻域 就是$S$. \\
                如果$E$是$S$的非空真子集, 那么$\intr E=\varnothing$. 因为其中的元素的\邻域 $S$不是$E$的子集.\\
                对于任何包含多于一个点的集合$E$,其\hyperref[dfn:Derived-Set]{导集}都是$S$. 这表明每个点都无限接近其他点。$E$的\hyperref[dfn:Closure]{闭包}是$S$, $E$的\hyperref[dfn:Boundary]{边界}是$S$.\\
                这些性质清晰地表明,在平凡拓扑下,所有点都粘在一起,没有任何办法区分其中的点。
            \end{xmp}

            
            \begin{dfn}
                [Dense-Set]
                {稠密集}
                [Dense Set]
                []
                设 \((S,\mathcal{T})\) 是\拓扑空间, \(E\) 是集合, \(E\subseteq S\), 定义 \(E\) 是\textbf{稠密集}, 当且仅当: \(\cl E=S\). 
            \end{dfn}

            \begin{dfn}
                [Limit]
                {极限}
                [Limit]
                [猫猫]
                设 \((S,\mathcal{T})\) 是\拓扑空间, \(x:S\), \(\{x_\cdot\}:\N\to S\), 定义 \(\{x_\cdot\}\) 以 \(x\) 为\textbf{极限}, 当且仅当: 
                \[\forall U, U\in\Nbr{x}\implies\exists N:\N, \forall n:\N, n\geq N\implies x_n\in U(x)\]
                
                记作 \(\lim\limits_{n\to+\infty}x_n=x\). 
            \end{dfn}

            \begin{dfn}
                [Converge]
                {收敛}
                [Convergence]
                [Dedicatia]
                定义\拓扑空间 上的序列 \(\{x_\cdot\}\) 收敛于 \(x\), 当且仅当序列以 \(x\) 为极限。这时称该序列为\textbf{收敛点列}.
            \end{dfn}

        \subsection{连通性}
            
            \begin{dfn}
                [Connectedness]
                {连通性}
                [Connectedness]
                []
                设 \((S,\mathcal{T})\) 是\拓扑空间, \(E\) 是集合, 定义 \(E\) 是\textbf{连通集}, 当且仅当: \(\开集[E]\land\闭集[E]\). 

                定义连通的, 若 \(S\) 即开又闭的子集只有 \(S\) 和 \(\varnothing\), 即: 
                \[\PP(S)\cap\{X|X\in\mathcal{T},(S-X)\in\mathcal{T}\}=\{\varnothing,S\}\]
            \end{dfn}

            \begin{rmk}
                这意味着 \(S\) 无法写成两个不相交的非空开集之并, 即\(S\)的\textbf{分割}不存在. 
            \end{rmk}
            
            \begin{ppt}
                []
                {}
                []
                []
                设\拓扑空间 \(S\) 的子集 \(X\) 是连通的, 则 \(Y(X\subseteq Y\subseteq\cl X)\) 是连通的. 
                
                特别地, \(\cl X\) 是连通的. 
            \end{ppt}
            
            \begin{ppt}
                []
                {}
                []
                []
                设 \(X\) 是连通的拓扑空间, \(Y\)是拓扑空间, 则局部常值函数 \(f:X\to Y\) 是常值函数: 
                \[f:X\to Y(\forall x_0\in X, \exists U(x_0)(\forall x\in U(x_0), f(x)=C))\implies\forall x\in X, f(x)=C\]
                
                即, 若定义域连通, 则所有点分别保持常值的函数, 整体保持常值. 
            \end{ppt}

            \begin{dfn}
                [Linear-Continuum]
                {线性连续统}
                [Linear Continuum]
                []
            \end{dfn}

            \begin{dfn}
                [Path-Connected]
                {道路连通}
                [Path Connected]
                []
                设 \((S,\mathcal{T})\) 是\拓扑空间, \(x,y\in S\), 称映射\(\gamma:[0,1]\to S\)是\(S\)上的\textbf{道路}, 使得 \(\gamma(0)=x,\gamma(1)=y\) 且 \(\gamma\) 连续
                拓扑空间 \((S,\mathcal{T})\) 称为是道路连通的, 若 \(S\) 的任意两点间存在道路, 即:
                \[\forall x,y\in S, \exists\gamma:[0,1]\to S,\gamma(0)=x,\gamma(1)=y,\gamma\in\Con([0,1],S)\]
            \end{dfn}

            % \begin{example}
            %     设\(S=\{x\times\sin(\frac{1}{x})|0<x\leq1\}\), 显然S是连通集\((0,1]\)的一个连续像, 故S的闭包\(\cl S\)是连通的, 但它不是道路连通的. 

            %     \begin{tikzpicture}
            %         \begin{axis}[
            %             domain=0.01:1, % 定义域
            %             samples=1000, % 采样点数
            %             xlabel=\(x\),
            %             ylabel={\(x \sin\left(\frac{1}{x}\right)\)},
            %             grid=both,
            %             width=10cm,
            %             height=6cm
            %         ]
            %         \addplot[blue] {sin(deg(1/x))};
            %         \end{axis}
            %     \end{tikzpicture}

            %     \begin{center}
            %         图1. \(S=\{x\times\sin(\frac{1}{x})|0<x\leq1\}\)
            %     \end{center}
            % \end{example}
            
            \begin{prf}
                不妨设 \(f:[0,1]\to \cl S\) 是连接 \((0,0)\) 点到 \(\cl S\) 中一点的道路.

                显然, 存在 \(\{x_n\},x_n\downarrow>0\), 使 \(f(x_n)={(-1)}^n\neq0\)
                \[\because f(0)=0\]
                \[\therefore f\notin\Con([0,1],\cl S)\]
                \[Absurd!\]

                故\(\cl S\)不是道路连通的.
            \end{prf}

            \begin{ppt}
                []
                {道路连通蕴含连通}
                [Path Connectedness Implies Connectedness]
                [Dedicatia]
                一个集合是\道路连通 的,那么它是\连通 的。
            \end{ppt}

            \begin{cxmp}
                []
                {连通不蕴含道路连通}
                []
                [Dedicatia]
                考虑 \(\R^2\) 上的通常拓扑, 定义集合 \(E:=A\cup B\), 其中 \(A=\{(x,y):y=\sin\frac{1}{x}, x\in\Icc{-\pi}{\pi}\}\), \(B=\{(x,y):x=0, y\in\Icc{-1}{1}\}\). \(E\) 是\连通 的. 但天然的映射 \(y=\sin\frac{1}{x}\) 在 \(x=0\) 处具有振荡间断点, 它不是\道路连通 的. 
            \end{cxmp}

        \subsection{紧性}
            
            \begin{dfn}
                [Open-Cover]
                {开覆盖}
                [Open Cover]
                [猫猫]
                设 \((S,\mathcal{T})\) 是\拓扑空间, \(E\) 是集合, \(\mathcal{O}\) 是集合族, \(E\subset S\), 定义 \(\mathcal{O}\) 是 \(E\) 的\textbf{开覆盖}当且仅当: 
                \begin{enumerate}
                    \item \(\mathcal{O}\) 是开集族: 
                        \[\mathcal{O}\subseteq\mathcal{T}\]
                    \item 覆盖性: 
                        \[E\subseteq\bigcup[\mathcal{O}]\]
                \end{enumerate}
            \end{dfn}

            \begin{dfn}
                [Compact]
                {紧性 / 紧致性}
                [Compactness]
                [猫猫]
                设 \((S,\mathcal{T})\) 是\拓扑空间, \(E\) 是集合, 定义 \(E\) 是 \((S,\mathcal{T})\) 中的\textbf{紧集}当且仅当: 
                \[\forall[\mathcal{O}][\开覆盖[\mathcal{O}][E]\implies\exists\mathcal{O}', (\card\mathcal{O}'\in\N)\land(\开覆盖[\mathcal{O}'][E])]\]

                定义 \((S,\mathcal{T})\) 是\textbf{紧拓扑空间}, 当且仅当: \(S\) 是 \((S,\mathcal{T})\) 中的紧集. 

                紧性也被称为\textbf{Heine-Borel性质}。
            \end{dfn}

            \begin{rmk}
                [猫猫]
                紧集是指其任意开覆盖均存在有限子覆盖的集合. 
            \end{rmk}

            \begin{ppt}
                []
                {紧空间的闭子集是紧的}
                []
                [猫猫]
            \end{ppt}

            \begin{thm}
                [HB]
                {Heine-Borel定理}
                [Heine-Borel Theorem]
                [Dedicatia]
                设 $E$ 是$\R^n$ 的子集,那么以下三条性质等价:
                \begin{enumerate}
                    \item $E$ 是有界的\闭集 ;
                    \item $E$ 是\紧 ;
                    \item $E$ 是序列紧的。
                \end{enumerate}
            \end{thm}

            \begin{cxmp}
                {有界闭集与紧集不等价}
                考虑在连续函数空间 $C[-\pi,\pi]$ 上定义度量
                \[d(f,g)=\sqrt(\int_{-\pi}^{\pi}|f(t)-g(t)|^2\dd{t})^{\frac{1}{2}}\]
                考虑集合
                \[E:=\{\sin nt\}^\infty=\{\sin t,\sin 2t,\cdots,\sin nt,\cdots\}\]
                对任意 $n\in\N^*$:
                \[d(\sin nt,0)=\]
            \end{cxmp}

            % \begin{thm}
            %     []
            %     {管状引理}
            %     [Tube Lemma]
            %     []
            %     考虑积空间 \(X\times Y\), 其中 \(Y\) 是紧的. 对于每一个包含 \(x_0\times Y\) 的开集 \(N\), 存在 \(x_0\) 的邻域 \(U\), 使得 \(U\times Y\) 被 \(N\)包含. 

            %     注意\(Y\)是紧的, 若否, 则管状引理不一定正确. 如:取\(X=Y=\R,x_0=0,N=\{x\times y|\left|x\right|<\frac{1}{y^2+1}\}\)
            %     % \begin{figure}[h]
            %     %     \centering
            %     %     \begin{tikzpicture}
            %     %         \begin{axis}[
            %     %             domain=-1:1, % 定义域
            %     %             samples=1000, % 采样点数
            %     %             xlabel=\(x\),
            %     %             ylabel=\(y\),
            %     %             grid=both,
            %     %             width=10cm,
            %     %             height=6cm,
            %     %             restrict y to domain=-1:1,
            %     %             ytick={-1, -0.5, 0, 0.5, 1},
            %     %             xtick={-1, -0.5, 0, 0.5, 1}
            %     %         ]
            %     %         \addplot[blue, thick] gnuplot {1/(x**2+1)};
            %     %         \addplot[blue, thick] gnuplot {-1/(x**2+1)};
            %     %         \end{axis}
            %     %     \end{tikzpicture}
            %     % \end{figure}                
            % \end{thm}

        \subsection{Hausdorff 空间}

            \begin{dfn}
                [Hausdorff-Space]
                {Hausdorff 空间}
                [Hausdorff Space]
                [猫猫]
                设 \((S,\mathcal{T})\) 是\拓扑空间, 定义 \((S,\mathcal{T})\) 是\textbf{Hausdorff 空间}, 当且仅当: 
                \[\forall x,y:S, x\neq y\implies\exists[U_x, U_y][U_x\in\Nbr{x}\land U_y\in\Nbr{y}\implies U_x\cap U_y=\varnothing]*\]
            \end{dfn}

            \begin{rmk}
                [猫猫]
                \Hausdorff空间 是强调任意两点间可用邻域区分的拓扑空间. 
            \end{rmk}

            \begin{rmk}
                []
                \Hausdorff空间 存在最细基, 这表明改空间的元素间存在一定独立性
            \end{rmk}
            
            \begin{ppt}
                []
                {Hausdorff 空间中点列极限唯一}
                []
                [猫猫]
                设 \((S,\mathcal{T})\) 是 \Hausdorff空间, \(\forall(S,\mathcal{T})\) 中的收敛序列\({\{x_n\}}_{n=0}^{+\infty}\), 满足上述条件的\(x\)是唯一的. 
            \end{ppt}
            
            \begin{prf}
                运用反证法. 

                若\((S,\mathcal{T})\)是Hausdorff的, 且假设\(\exists{\{x_n\}}_{n=0}^{+\infty}\), 对于该序列有: 
                \[\exists x,y\in S: x\neq y, \forall U(x), U(y), \exists N\in\N: \forall n>N(N\in\N), x_n\in U(x)\cap U(y)\]
                
                由\((S,\mathcal{T})\)的Hausdorff性质可知\(\exists U(x), U(y): U(x)\cap U(y)=\varnothing\). 

                取满足上述条件的\(U(x), U(y)\), 则当\(n>N\)时, 有\(x_n\in U(x)\cap U(y)=\varnothing\), 矛盾. 

                \(\therefore\)此时只能存在一个满足条件的\(x\). 
            \end{prf}

            \begin{ppt}
                []
                {Hausdorff 空间中紧集是闭的}
                []
                [猫猫]
            \end{ppt}

            \begin{thm}
                []
                {}
                []
                [猫猫]
            \end{thm}

            \begin{crl}
                []
                {紧集套定理}
                []
                [猫猫]
            \end{crl}

        \subsection{拓扑空间的构造}

            \begin{dfn}
                [Generated-Topology]
                {生成拓扑}
                [Generated Topology]
                [猫猫]
                设 \(S\) 是集合, \(\mathfrak{B}\) 是集合族, \(\mathfrak{B}\subseteq\PS[S]\), 
                \begin{enumerate}
                    \item 覆盖性: 
                    \[\bigcup[\mathfrak{B}]=S\]
                    \item 生成性: 设 \(B_1,B_2\) 是集合, \(B_1, B_2\in\mathfrak{B}\), \(x\in B_1\cap B_2\), 则: 
                    \[\exists[B_3][(B_3\in\mathfrak{B})\land(x\in B_3)\land(B_3\subseteq B_1\cap B_2)]*\]
                \end{enumerate}
                定义由 \(\mathfrak{B}\) \textbf{生成的拓扑}为: 
                \[\left\{X\middle|\exists B\subseteq\mathfrak{B}, X=\bigcup[B]\right\}\]

                定义由 \(\mathfrak{B}\) \textbf{生成的拓扑空间}为: 
                \[(S, \生成拓扑{\mathfrak{B}})\]
            \end{dfn}

            \begin{xmp}
                []
                {通常拓扑}
                []
                []
                在扩充实数集 \(\bar{\R}\) 上依据偏序关系 ``\(<\)'' 定义开区间: 
                \[(a,b)=\{x|a<x<b\}(a,b\in\R\cup\{+\infty,-\infty\})\]
                
                并如下定义拓扑 \(\mathcal{T}\), 称为\(\R\)上的\textbf{通常拓扑}: 
                \[O\in\mathcal{T}\Longleftrightarrow\forall x\in O, \exists(a,b): x\in(a,b)\subseteq O\]

                \(\mathcal{T}\)中的非空元素必可表示为有限个或可数无限个互不相交的开区间之并: 
                \[\forall O\in \mathcal{T}(O\neq \varnothing), \exists\bigcup\{(a_i,b_i)|i\in I, \card(I)\in\mathbb{N^+}\cup\{\aleph_0\}\wedge\forall i,j, (a_i,b_i)\cap(a_j,b_j)=\varnothing\}=O\]
            \end{xmp}
            
            \begin{xmp}
                [Order-Topology]
                {序拓扑}
                [Order Topology]
                []
                若\(S\)是一个全序集, 设\(\mathcal{B}\)为由下述所有集合构成的族:

                (1)\(S\)的所有开区间\((a,b)\)

                (2)所有形如\(\Ico{a_0}{b}\)的区间,其中\(a_0\)为\(S\)的最小元 (存在的话)

                (3)所有形如\(\Ioc{a}{b_0}\)的区间,其中\(b_0\)为\(S\)的最大元 (存在的话)

                则\(\mathcal{B}\)是\(S\)的某拓扑的一个基, 此拓扑称\textbf{序拓扑 (Order Topology)}
            \end{xmp}

            \begin{dfn}
                [Topological-Basis]
                {拓扑基}
                [Topological Basis]
                [猫猫]
                设 \((S,\mathcal{T})\) 是\拓扑空间, \(\mathfrak{B}\) 是集合族, \(\mathfrak{B}\subseteq\mathcal{T}\), 定义 \(\mathfrak{B}\) 是 \((S,\mathcal{T})\) 的\textbf{拓扑基}, 当且仅当: \[\mathcal{T}=\生成拓扑{\mathfrak{B}}\]
            \end{dfn}

            \begin{cxmp}
                []
                {拓扑空间的基不唯一}
                []
                []
            \end{cxmp}

            \begin{dfn}
                [Subspace-Topology]
                {诱导拓扑 / 子空间拓扑}
                [Induced Topology / Subspace Topology]
                [猫猫]
                设 \((S,\mathcal{T})\) 为\拓扑空间, \(S'\) 是集合, \(S'\subseteq S\), 定义由 \(S'\) 诱导的 \((S,\mathcal{T})\) 的\textbf{子空间拓扑}为: 
                \[\{O'|\exists O\in\mathcal{T}, O'=O\cap S'\}\]

                定义 \((S',\mathcal{T}')\) 为由 \(S'\) 诱导的 \((S,\mathcal{T})\) 的\textbf{拓扑子空间}, 当且仅当: \(\mathcal{T}'=\) 由 \(S'\) 诱导的 \((S,\mathcal{T})\) 的子空间拓扑. 
            \end{dfn}

            \begin{rmk}
                [猫猫]
                由子集诱导的子空间拓扑是从原拓扑中以该子集为边界``截取''出来的. 
            \end{rmk}

            \begin{ppt}
                []
                {子空间拓扑是合法拓扑}
                []
                [猫猫]
                设 \((S,\mathcal{T})\) 为\拓扑空间, \(S'\) 是集合, \(S'\subseteq S\), 则由 \(S'\) 诱导的 \((S,\mathcal{T})\) 的子空间拓扑是 \(S'\) 上的\拓扑. 
            \end{ppt}

            \begin{ppt}
                []
                {}
                []
                []
                设 \((S,\mathcal{T})\) 是\拓扑空间, \(T\) 是 \(S\)的一个子空间. 那么, \(T\)是紧的当且仅当对于\(T\)中的每一个开覆盖, 存在一个有限子集是\(T\)的开覆盖. 
            \end{ppt}
            
            \begin{prf}
                若\(T\)是紧致的, 设\(\mathcal{A}'=\{A_{a}\} '\)是\(T\)由\(T\)中开集构成的一个开覆盖. 选取合适的\(\{A_{a}\}\),有\(A_{a}'=A_{a}\cap T\)
            \end{prf}
            
            \begin{dfn}
                [Product-Topology]
                {积拓扑}
                [Product Topology]
                [猫猫]
                设 \((S_1, \mathcal{T}_1), (S_2, \mathcal{T}_2)\) 是\拓扑空间, 定义 \((S_1, \mathcal{T}_1)\) 与 \((S_2, \mathcal{T}_2)\) 的\textbf{积拓扑}为: 
                \[\{O|O\subseteq S_1\times S_2\land O.1\in\mathcal{T}_1\land O.2\in\mathcal{T}_2\}\]

                设 \(\mathcal{T}\) 是 \((S_1, \mathcal{T}_1)\) 与 \((S_2, \mathcal{T}_2)\) 的积拓扑, 定义 \((S_1, \mathcal{T}_1)\) 与 \((S_2, \mathcal{T}_2)\) 的\textbf{积空间}为: \((S_1\times S_2, \mathcal{T})\), 记作 \((S_1, \mathcal{T}_1)\times(S_2, \mathcal{T}_2)\).
            \end{dfn}

    \section{连续映射}

        \subsection{连续映射}

            \begin{dfn}
                [Continuous-Mapping]
                {连续映射}
                [Continuous-Mapping]
                [猫猫]
                设有拓扑空间 \(X,Y\), 映射 \(f:X\to Y\); 设 \(x\in X, f(x):=y\in Y\). \(f\) 称为是在 \(x\) 处\textbf{连续}的, 若: 
                \[\forall U_y\in \mathcal{V}(y), \exists U_x\in \mathcal{V}(x)(f\left(U_x\right)\subseteq U_y)\]

                \(f\) 称为是 \(X\) 的连续映射, 若 \(f\) 在 \(X\) 中的每一点连续. 

                将 \(X\to Y\) 全体连续映射的集合记为 \(\Con(X,Y)\)
            \end{dfn}
            
            \begin{ppt}
                []
                {连续映射将紧集映射为紧集}
                []
                []
            \end{ppt}

            \begin{ppt}
                []
                {连续映射的等价定义}
                []
                []
                设拓扑空间\((X,\mathcal{T}_X), (Y,\mathcal{T}_Y)\), \((f:X\to Y)\in\Con(X,Y)\)有两种等价描述: 
                
                (1)开集的原像是开集: 
                \[\forall S(f(S)\in\mathcal{T}_Y)\implies S\in\mathcal{T}_X\]
                
                (2)闭集的原像是闭集: 
                \[\forall S((Y-f(S))\in\mathcal{T}_Y)\implies(X-S)\in\mathcal{T}_X\]
            \end{ppt}
            
            \begin{prf}
                证明: 先证明开集情况. 
    
                (1)\(\implies\): 
                    \[f(S)\in\mathcal{T}_Y\implies(\forall y\in f(S)\implies f(S)\in\Nbr{y})\]
                    \[\forall x_0\in S, y_0:=f(x_0)\]
                    \[(f:X\to Y)\in\Con(X,Y)\implies\exists U(x)\in\Nbr{x}(\forall x\in U(x)\implies f(x)\in Y)\]
            \end{prf}
            
            \begin{thm}
                []
                {}
                []
                []
                Hausdorff空间之间的连续映射将收敛数列映射为收敛数列. 

                设拓扑空间\(X,Y\)是Hausdorff的, 映射\(f:X\to Y\)在\(x\in X\)处连续, 则: 
                \[\forall{\{x_n\}}_{n=0}^{+\infty}(x_n\in X(\forall n\in\N)\wedge\lim_{n\to+\infty}x_n=x)\implies\lim_{n\to+\infty}f(x_n)=f(x)\]

                此定理的另一种表述方式是: \(\lim\limits_{n\to+\infty}\) 与 \(f\) 可交换, 若 \(f\) 连续且 \(X\) 与 \(Y\) 都是Hausdorff的. 
            \end{thm}
                
            \begin{prf}
                \[y:=f(x); f\text{在\(x\)处连续}\implies\forall U(y)\in\Nbr{y}, \exists U(x)\in\Nbr{x}: f(U(x))\subseteq U(y)\]
                \[\lim_{n\to+\infty}x_n=x\implies\exists N\in\N: \forall n>N, x_n\in U(x)\Rightarrow f(x_n)\in f(U(x))\subseteq U(y)\]
                \[\therefore\lim_{n\to+\infty}f(x_n)=f(x)\square\]
            \end{prf}
            
            \begin{thm}
                []
                {连续的复合传递性}
                []
                []
                设拓扑空间\(X,Y,Z\), 映射\(f:X\to Y\)在\(x\in X\)处连续, 映射\(g:Y\to Z\)在\(f(x)\in Y\)处连续, 则\(g\circ f:X\to Z\)在\(x\)处连续. 
            \end{thm}
                
            \begin{prf}
                \[y:=f(x), z:=g(y)\]
                \[\forall U(z)\in\Nbr{z}[Z], g\text{在\(y\)处连续}\implies\exists U(y)\in\Nbr{y}[Y]: g(U(y))\subseteq U(z)\]
                \[f\text{在\(x\)处连续}\implies\exists U(x)\in\Nbr{x}[X]\subseteq X: f(U(x))\subseteq U(y)\]
                \[\therefore g\circ f(U(x))\subseteq g(U(y))\subseteq U(z)\implies g\circ f\text{在\(x\)处连续}\square\]
            \end{prf}

            \begin{dfn}
                [Homeomorphism]
                {同胚}
                [Homeomorphism]
                []
                设拓扑空间\((X,\mathcal{T}_X), (Y,\mathcal{T}_Y)\), \(f:X\to Y\), 定义 \(f\) 是 \(X\) 到 \(Y\) 的\textbf{同胚}, 当且仅当: \(f\)是双射, 且 \(f\in\Con(X,Y), f^{-1}\in\Con(Y,X)\). 

                定义 \(X\) 与 \(Y\) \textbf{同胚}, 当且仅当: \(\exists f:X\to Y, f\) 是从 \(X\) 到 \(Y\) 的同胚. 
            \end{dfn}
            
            \begin{thm}
                []
                {}
                []
                []
                设拓扑空间\((X,\mathcal{T}_X),(Y,\mathcal{T}_Y)\), \(f\in\Con(X,Y)\)是双射, 则\(f\)是\(X\)到\(Y\)的同胚当且仅当: 
                \[\forall S\in\mathcal{T}_X\implies f(S)\in\mathcal{T}_Y\]
                \[\forall S((X-S)\in\mathcal{T}_X)\implies(Y-f(S))\in\mathcal{T}_Y\]
            \end{thm}

                

    \section{度量空间}
        
        \subsection{基本概念}
            
            \begin{dfn}
                [Metric-Space]
                {度量空间 / 距离空间}
                [Metric Space]
                [猫猫]
                设 \(S\) 是类型, \(d:S^2\to\R\), 定义 \(d\) 是 \(S\) 上的\textbf{度量}, 当且仅当: 
                \begin{enumerate}
                    \item 正定性: 
                    \[\forall x,y:S, d(x,y)\geq 0\]

                    \item 同一性: 
                    \[\forall x,y:S, d(x,y)=0\iff x=y\]
    
                    \item 对称性: 
                    \[\forall x,y:S, d(x,y)=d(y,x)\]
    
                    \item 三角不等式: 
                    \[\forall x,y,z:S, d(x,y)+d(y,z)\geq d(x,z)\]
                \end{enumerate}

                定义 \((S,d)\) 为\textbf{度量空间}当且仅当 \(d\) 是 \(S\) 上的\度量. 
            \end{dfn}

            \begin{dfn}
                [Open-Ball]
				{开球}
                [Open Ball]
                [猫猫]
                设 \((S,d)\) 是\度量空间, \(x:S\), \(\delta:\R\), 定义以 \(x\) 为中心, 以 \(\delta\) 为半径的\textbf{开球}为: 
                \[\{y:S|d(x,y)<\delta\}\]
                
                记作 \(\Ball{x}{\delta}\). 
		    \end{dfn}

            \begin{dfn}
                [Metric-Topology]
                {度量诱导的拓扑}
                [Metric-Induced Topology]
                [猫猫]
                设 \((S,d)\) 是\度量空间, 定义由 \(d\) 诱导的 \(S\) 上的\textbf{度量拓扑}为: 
                \[\{O|\forall[x][x\in O\implies\exists\delta:\R^+, \Ball{x}{\delta}\subseteq O]*[S]\}\]
            \end{dfn}

            \begin{rmk}
                [猫猫]
                度量诱导的拓扑判定全体元素都有``安全区''的集合是开集. 
            \end{rmk}

            \begin{ppt}
                []
                {度量诱导的拓扑是合法拓扑}
                []
                [猫猫]
            \end{ppt}

            \begin{ppt}
                []
                {度量诱导的拓扑空间是 Hausdorff 的}
                []
                [猫猫]
            \end{ppt}

            \begin{ppt}
                []
                {度量空间中开球的邻域身份}
                []
                [猫猫]
            \end{ppt}

            \begin{thm}
                []
                {度量空间中极限点附近有无穷个点}
                []
                [猫猫]
                
            \end{thm}

            \begin{crl}
                {度量空间中有限集没有极限点}
            \end{crl}

            \begin{dfn}
                {列紧集}
            \end{dfn}

            \begin{thm}
                []
                {Bolzano-Weierstrass 定理 / 聚点定理}
                []
                [猫猫]
            \end{thm}

            \begin{dfn}
                []
                {有界集}
                []
                [猫猫]
            \end{dfn}

        \subsection{拓扑空间的分离性}

            \begin{dfn}
                [Kolmogorov-Space]
                {Kolmogorov空间}
                [Kolmogorov Space]
                [Dedicatia]
                定义一个\拓扑空间 $(X,\tau)$ 为\textbf{Kolmogorov空间}当且仅当其中的任意两点都可以被一个\开集 区分。
                \[\forall\,x,y\in X, \exists\,D, x\in D,y\notin D.\]
                又叫做\textbf{T0空间}.
            \end{dfn}

            \begin{xmp}
                [Sierpinski-Space]
                {Sierpinski空间}
                [Sierpinski Space]
                [Dedicatia]
                \textbf{Sierpinski空间}指的是\拓扑空间 $(S,\tau)$, $S=\{x,y\}$, $\tau=\{\varnothing, \{y\}, S\}$. 这时我们可以用一个开集 $\{y\}$ 区分 $x,y$. 说明它是\textbf{T0空间}. 但是它不再满足更强的分离性。
            \end{xmp}

            \begin{dfn}
                [Frechet-Space]
                {Frechet空间}
                [Frechet Space]
                [Dedicatia]
                定义一个\拓扑空间 $(X,\tau)$ 为\textbf{Frechet空间}当且仅当其中的任意两个不同的点都可以用各自的开\邻域 区分。
                \[\forall\,x,y\in X, \exists\Nbr{x}\st x\in\Nbr{x}, y\notin\Nbr{x}. \]
                又叫做\textbf{T1空间}.
            \end{dfn}

            \begin{ppt}
                []
                {Frechet空间的充要条件}
                []
                [Dedicatia]
                一个\拓扑空间 是\textbf{T1空间}的充要条件是每个单点集都是\闭集 .
            \end{ppt}

            \begin{xmp}
                [Cofinite-Topology]
                {无限集上的有限补拓扑空间}
                [Co-finite Topological Space]
                [Dedicatia]
                设 $X$ 是无限集合,定义\拓扑
                \[\tau:=U \text{ 如果满足 }\begin{cases}
                    &U=\varnothing,\\
                    &X\backslash U \text{ 是有限集。}
                \end{cases}\]
                这时,所有的单点集的补集都是\开集 , 所以单点集都是\闭集 . 这是一个\textbf{T1空间}. 但它不是T2空间,因为任意两个非空的\开集 必定相交:\\
                设 $U,V$ 是两个非空开集,那么 $X\backslash U$, $X\backslash V$ 是有限的。考虑 $U\cap V$, 那么其补集 $X\backslash(U\cap V)=(X\backslash U)\cup (X\backslash V)$ 应该也是有限的。如果 $U\cap V=\varnothing$, $(X\backslash U)\cup (X\backslash V)=X$ 就是有限集。这与 $X$ 是无限集矛盾。\\
                具体来讲这个空间有以下的性质:
                \begin{itemize}
                    \item 具有\紧 ;
                    \item 具有\连通 性;
                    \item 空间中的序列可以收敛到多个极限,实际上是无穷多个极限。因为这其中的点的开邻域都是无限集。
                \end{itemize}
            \end{xmp}

            \begin{dfn}
                [Hausdorff-Space2]
                {Hausdorff空间}
                [Hausdorff Space]
                [Dedicatia]
                定义一个\拓扑空间 $(X,\tau)$ 为\textbf{Hausdorff空间}当且仅当其中的任意两个不同的点都可以用不相交的各自的开\邻域 区分.
                \[\forall\,x,y\in X, \exists\Nbr{x},\Nbr{y},\st x\in\Nbr{x},y\in\Nbr{y}, \Nbr{x}\cap\Nbr{y}=\varnothing. \]
                又叫做\textbf{T2空间}.
            \end{dfn}

            \begin{ppt}
                []
                {度量空间都是T2空间}
                []
                [Dedicatia]
            \end{ppt}

            \begin{dfn}
                [Regular-Space]
                {正则空间}
                [Regular Space]
                [Dedicatia]
                定义一个\拓扑空间 $(S,\tau)$ 为\textbf{正则空间}, 当且仅当对任意 $x\in S$ 和\闭集 $C$, 其中 $x\notin C$, 存在不相交的\开集 $X,Y$ 使得 $x\in X$, $C\subseteq Y$. 
                正则空间又叫\textbf{T3空间}.
            \end{dfn}

            \begin{dfn}
                [Normal-Space]
                {正规空间}
                [Normal Space]
                [Dedicatia]
                定义一个\拓扑空间 $(S,\tau)$ 为\textbf{正规空间}, 当且仅当对任意两个不相交的\闭集 $C,D\in\tau$, 存在两个不相交的\开集 $X,Y$ 使得 $C\subseteq X, D\subseteq Y$.
            \end{dfn}

        \subsection{完备性}
            
            \begin{dfn}
                [Cauchy-Series]
                {Cauchy列}
                [Cauchy Series]
                [Dedicatia]
                定义 $\{x_n\}^\infty$ 为\度量空间 $(X,d)$ 上的\textbf{Cauchy列}, 当且仅当对任意 $\varepsilon>0$, 存在 $N\in\N$, $m,n>N$ 时均有 $d(x_m,x_n)<\varepsilon$. 
            \end{dfn}

            \begin{dfn}
                [Complete-Metric-Space]
                {完备度量空间}
                [Complete Metric Space]
                [Dedicatia]
                定义一个\度量空间 是\textbf{完备度量空间}, 当且仅当其中的每一个Cauchy列都是\hyperref[dfn:Converge]{收敛点列}。
            \end{dfn}

            \begin{xmp}
                []
                {Euclid空间是完备度量空间}
                []
                [Dedicatia]
                $n$ 维Euclid空间 $(\R^n,d)$ 是\完备\度量空间 . 其中
                \[d(\bm{x},\bm{y})=\sqrt{\sum_{i=1}^n(x_i-y_i)^2}\]
            \end{xmp}

            \begin{xmp}
                []
                {最大值度量下的连续函数空间是完备度量空间}
                []
                [Dedicatia]
                连续函数空间 $C[a,b]$ 在度量 $d(f,g)=\max\limits_{x\in\Icc{a}{b}}\{f(x),g(x)\}$ 下的\度量空间 是\完备 的. 因为这种意义下的收敛就是一致收敛。一致收敛保持连续性。
            \end{xmp}

            \begin{cxmp}
                []
                {积分度量下的连续函数空间不是完备度量空间}
                []
                [Dedicatia]
                连续函数空间 $C[a,b]$ 在度量 $d(f,g)=\int_{a}^b|f(x)-g(x)|\dd{x}$ 下的\度量空间 不是\完备 的。反例为:构造函数列$\{f_n\}$, 其中$f_1=1$, $n\geq 2$时
                \[f_n=\begin{cases}
                    &0, 0\leq x\leq \frac{1}{2}-\frac{1}{n},\\
                    &n\pqty{x-\pqty{\frac{1}{2}-\frac{1}{n}}}, \frac{1}{2}-\frac{1}{n}< x\leq \frac{1}{2};\\
                    &1, x>\frac{1}{2}.
                \end{cases}\]
                也就是说,该函数在$0\leq x\leq \frac{1}{2}-\frac{1}{n}$上是0,在$\frac{1}{2}-\frac{1}{n}< x\leq \frac{1}{2}$线性地由0变到1,在$x>\frac{1}{2}$上是1.\\
                由对称性设$m\geq n$, 通过画图可知该积分
                \[d(f_m,f_n)=\int_0^1|f_m-f_n|\dd{x}\leq\int_0^{1/2}|f_n|\dd{x}\leq\frac{1}{2}\cdot\frac{1}{n}\cdot 1\]
                与$\frac{1}{n}$同收敛。所以显然是Cauchy列;\\
                但是这个序列的极限是
                \[f(x)=\begin{cases}
                    &0, 0\leq x\leq\frac{1}{2};\\
                    &1, x>\frac{1}{2}.
                \end{cases}\]
                事实上,该函数在$x=\frac{1}{2}$时从0跃变到1, 这是不连续的。函数列的极限$f\notin C[0,1]$中,说明该度量空间不是完备的。
            \end{cxmp}

        \subsection{拓扑空间的度量化}

            

\end{document}