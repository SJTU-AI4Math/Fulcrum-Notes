\documentclass[UTF8]{ctexart}

\makeatletter
\def\input@path{{../../Fulcrum-Template/}{../../Operator-List/}}
\makeatother

\usepackage{FulcrumHabitCN}

\linespread{1.2}

\begin{document}

\tableofcontents
\newpage

\section{微分方程的基本概念}

    \subsection{常微分方程的定义及概念}

        研究自然知识与社会现象离不开数学这一强大的工具。而数学模型最常见的范式就是包含自变量与未知函数关系的函数方程。假如这类方程中还包含有该未知函数的导数,它就是一类特殊的函数方程——\textbf{微分方程}。

        现在我们给出明确的定义:
        \begin{dfn}
            [ODE]
            {常微分方程}
            [Ordinary Differential Equation]
            []
            联系自变量 $x$ 与这个自变量的未知函数 $y=y(x)$ 以及它的导数 $y^{(n)}=y^{(n)}(x)$ 在内的方程
            \[F(x,y,y',\cdots,y^{(n)})=0\]
            叫做常微分方程。
        \end{dfn}

        \begin{rmk}
            [Dedicatia]
            之所以给微分方程冠以“常”字,是因为我们研究的函数是一元的。如果是多元函数,就会出现偏导数。这时我们称之为\textbf{偏微分方程}.
        \end{rmk}

        \begin{dfn}
            []
            {常微分方程的阶}
            []
            []
            某个常微分方程中导数的最高阶数 $n$ 称作该常微分方程的\textbf{阶}。
        \end{dfn}

        \begin{dfn}
            []
            {线性常微分方程}
            [Linear Ordinary Differential Equation]
            [Dedicatia]
            如果常微分方程 $F(x,y,y',\cdots,y^{(n)})=0$ 对于未知函数 $y$ 及其各阶导数 $y^{(n)}$ 都是线性的, 就称该常微分方程为\textbf{线性常微分方程}.
        \end{dfn}

        定义了方程必定要定义解。我们有:

        \begin{dfn}
            []
            {常微分方程的解}
            [Solution to an ODE]
            []
            如果在区间 $D$ 上的函数 $\varphi(x)$ 是连续函数且具有 $n$ 阶导数,将其代入上述方程 $F=0$ 中得到关于 $x$ 的恒等式
            \[\forall x\in J\subseteq D,\,F(x,\varphi(x),\cdots,\varphi^{(n)}(x))=0\]
            就称 $\varphi(x)$ 是常微分方程 $F=0$ 在 $J$ 上的一个解。
        \end{dfn}

        \begin{dfn}
            []
            {通解}
            [General Solution]
            []
            $n$阶的常微分方程的解$\varphi$具有如下形式:
            \[y=\varphi(x,C_1,C_2,\cdots,C_n)\]
            $C_1,\cdots,C_n$是\textbf{独立的任意常数},我们就称其为该常微分方程的\textbf{通解}.
        \end{dfn}

        \begin{dfn}
            []
            {特解}
            [Particular Solution]
            []
            称不包含任意常数的解为特解。一旦任意常数由实际情况确定了下来之后,通解也就变成了特解。
        \end{dfn}

    \subsection{一阶微分方程的几何意义}-                           

        \begin{dfn}
            []
            {积分曲线}
            []
            []

        \end{dfn}

\section{微分方程的初等积分解法}

    \subsection{可构造全微分的微分方程}

        \begin{dfn}
            []
            {恰当方程}
            [Exact Equation]
            [Dedicatia]
            考虑一阶微分方程可以变为
            \begin{align*}
                P(x,y)\dd{x}+Q(x,y)\dd{y}=0\tag{$\star$}
            \end{align*}
            形式. 假如存在可微函数$\Phi(x,y)$满足
            \[\dd\Phi(x,y)=P(x,y)\dd{x}+Q(x,y)\dd{y}\]
            也就是说,
            \[\pdv{\Phi}{x}=P(x,y),\quad\pdv{\Phi}{y}=Q(x,y)\]
            则称上述微分方程为\textbf{恰当方程}. 
        \end{dfn}

        \begin{dfn}
            []
            {通积分}
            [General Integral]
            []
            将恰当方程改写为全微分的形式,有
            \[\dd\Phi(x,y)=P(x,y)\dd{x}+Q(x,y)\dd{y}=0\]
            从而 $\Phi(x,y)=C$. 将其写为显式函数,我们便得到了一个通解。称这个 $\Phi$ 为原常微分方程的\textbf{通积分}.
        \end{dfn}

        \begin{thm}
            []
            {微分方程恰当的充要条件}
            []
            []
            设函数 $P(x,y)$ 和 $Q(x,y)$ 在区域 $R:\{\alpha<x<\beta, \gamma<y<\delta\}$ 上连续且具有一阶的连续偏导数。则微分方程$(\star)$是恰当方程的充要条件是等式
            \[\pdv{P(x,y)}{y}=\pdv{Q(x,y)}{x}\]
            在 $R$ 上恒成立。

            当满足该等式时,方程 $(\star)$ 的通积分为
            \[\int_{x_0}^{x}P(x,y)\dd{x}+\int_{y_0}^{y}Q(x_0,y)\dd{y}=C\text{ 或 }\int_{x_0}^{x}P(x,y_0)\dd{x}+\int_{y_0}^{y}Q(x,y)\dd{y}=C\]
            其中 $(x_0,y_0)$ 是区域 $R$ 中任意取定的一点。
        \end{thm}

        \begin{dfn}
            []
            {变量分离方程}
            []
            []
            考虑一阶微分方程
            \begin{align*}
                P(x,y)\dd{x}+Q(x,y)\dd{y}=0 \tag{$\ast $}
            \end{align*}
            中的$P(x,y)$和$Q(x,y)$可以表示为$x$的一元函数与$y$的一元函数的乘积,就称$(\ast)$为\textbf{变量分离方程}。
        \end{dfn}

        \begin{ppt}
            []
            {变量分离方程的一般解法}
            []
            []
            在$(\ast)$中,只要令$P(x,y)=X_1(x)Y_1(y)$, $Q(x,y)=X_2(x)Y_2(y)$, 得到
            \[X_1(x)Y_1(y)\dd{x}+X_2(x)Y_2(y)\dd{y}=0\]
            我们写成
            \[\frac{X_1(x)}{X_2(x)}\dd{x}+\frac{Y_1(y)}{Y_2(y)}\dd{y}=0\]
            这时我们就称完成了分离变量。它的通积分为
            \[\int\frac{X_1(x)}{X_2(x)}\dd{x}+\int\frac{Y_1(y)}{Y_2(y)}\dd{y}=C\]
            这就解出了该微分方程。

            注:这里要先验证常函数$x=a,y=b$是否满足原先微分方程。
        \end{ppt}

    \subsection{一阶线性微分方程}

        \begin{dfn}
            []
            {一阶线性微分方程}
            [First-order Linear Differential Equation]
            []
            对于微分方程具有如下形式
            \[\dv{y}{x}+p(x)y=q(x)\tag{$\star$}\]
            其中$p(x)$和$q(x)$在$D:(a,b)$上连续,则称该微分方程为\textbf{一阶线性微分方程}.
        \end{dfn}

        \begin{dfn}
            []
            {一阶齐次线性微分方程}
            [First-order Homogeneous Linear Differential Equation]
            []
            如果上面$(\star)$中的$q(x)$恒为0,即
            \[\dv{y}{x}+p(x)y=0\tag{$\star\star$}\]
            则称 $(\star\star)$ 为一阶齐次线性微分方程.\\
            否则称为一阶非齐次线性微分方程,并称$q(x)=0$时的微分方程为相应的齐次线性微分方程.
        \end{dfn}

        \begin{ppt}
            []
            {一阶齐次线性微分方程的解法}
            []
            []
            将$(\star)$写为对称形式
            \[\dd{y}+p(x)y\dd{x}\]
            显然为变量分离方程。我们先考虑$y\neq 0$. 分离变量,得
            \[\frac{\dd{y}}{y}+p(x)\dd{x}=0\]
            由此积分后,得到了$(\star)$的通解
            \[y=C\exp(-\int p(x)\dd{x}).\]
            由于这里我们限定了$y\neq 0$, 所以这里的任意常数$C\neq 0$. 当然,$C=0$时,有一个特解$y=0$也满足该微分方程,因此补上这个特解。故最终的通解形式中的$C$可以是任意常数。
        \end{ppt}

        \begin{ppt}
            []
            {一阶非齐次线性微分方程的解法}
            []
            []
            先写为对称方程,即
            \[\dd{y}+p(x)y\dd{x}=q(x)\dd{x}\tag{$\ast$}\]
            一般情况下这个方程不是恰当方程。但我们可以同乘以$\mu(x)=\exp(\int p(x)\dd{x})$
            \[\exp(\int p(x)\dd{x})\dd{y}+\exp(\int p(x)\dd{x})p(x)y\dd{x}=\exp(\int p(x)\dd{x})q(x)\dd{x}\]
            这时左右两边各自凑出了全微分形式
            \[\dd(y\exp(\int p(x)\dd{x}))=\dd\int q(x)\exp(\int p(x)\dd{x})\dd{x}\]
            直接积分便得到通积分
            \[y\exp(\int p(x)\dd{x})=\int q(x)\exp(\int p(x)\dd{x})\dd{x}+C\]
            这样就得到了微分方程$(\ast)$的通解
            \[y=\exp(\int p(x)\dd{x})\cdot\qty(C+\int q(x)\exp(\int p(x)\dd{x}))\]
            有时为确定起见,常将结果中的不定积分写成变上限的定积分
            \[y=C\exp(-\int_{x_0}^{x}p(t)\dd{t})+\int_{x_0}^{x}q(s)\exp(-\int_{s}^{x}p(t)\dd{t})\dd{s}\]
        \end{ppt}

        \begin{dfn}
            []
            {积分因子}
            []
            []
            称上述解题过程中使用的因子$\mu(x)=\exp(\int p(x)\dd{x})$为\textbf{积分因子}。像这种找到合适的积分因子凑全微分的方法我们叫做\textbf{积分因子法}。
        \end{dfn}

        \begin{ppt}
            []
            {与0的关系}
            []
            []
            齐次线性方程的解或者恒等于0, 或者恒不等于0.
        \end{ppt}

        \begin{ppt}
            []
            {解的线性性}
            []
            []
            1. 齐次线性方程的任意解的线性组合仍是它的解.\\
            2. 齐次线性方程的任一解与非齐次线性方程之和仍是这个非齐次线性方程的解.\\
            3. 非齐次线性方程任意两个解的差是相应的齐次线性方程的解.\\
            4. 非齐次线性方程的特解与相应的齐次线性方程的通解之和构成了这个非齐次线性方程的通解.
        \end{ppt}

        \begin{ppt}
            []
            {解的唯一性}
            []
            []
            给定初值条件后一阶线性微分方程的解存在且唯一.
        \end{ppt}

    \subsection{一阶线性微分方程的变体}

        \begin{dfn}
            []
            {比值微分方程}
            [Homogeneous Differential Equation]
            [Dedicatia]
            定义微分方程
            \[P(x,y)\dd{x}+Q(x,y)\dd{y}=0\]
            是\textbf{比值微分方程}, 当且仅当其中的函数 \(P(x,y)\) 与 \(Q(x,y)\) 都是 \(x,y\) 的 \(m\) 次齐次函数. 即
            \[\forall\,n\in\R, P(nx,ny)=n^mP(x,y), \; Q(nx.ny)=n^mQ(x,y).\]
        \end{dfn}

        \begin{ppt}
            []
            {比值微分方程的等价定义}
            []
            []
            如果一阶微分方程可以写成
            \[\dv{y}{x}=f\qty(\frac{y}{x})\]
            的形式,则称该微分方程为\textbf{比值微分方程}.
        \end{ppt}

        \begin{ppt}
            []
            {比值微分方程的解法}
            []
            [Dedicatia]
            对于比值微分方程,我们令
            \[u=\frac{y}{x}\Rightarrow y=ux\]
            为新的未知函数. 那么
            \[P(x,y)=P(x,ux)=x^mP(1,u)\]
            \[Q(x,y)=Q(x,ux)=x^mQ(1,u)\]
            因此原方程可写为
            \[x^mP(1,u)\dd{x}+x^mQ(1,u)\dd{(ux)}=0\]
            这是一个变量分离方程. 由此我们便可以解出该微分方程的通积分.
        \end{ppt}

        \begin{thm}
            []
            {比值微分方程积分因子的求法}
            []
            [Dedicatia]
            对于比值微分方程
            \[P(x,y)\dd{x}+Q(x,y)\dd{y}=0\]
            其积分因子可取为
            \[\mu(x,y)=\frac{1}{xP(x,y)+yQ(x,y)}.\]
        \end{thm}

        \begin{xmp}
            []
            {求解比值微分方程的实例}
            []
            [Dedicatia]
            求解微分方程
            \[\dv{y}{x}=\frac{x+y}{x-y}.\]
            不妨令 \(u=\frac{y}{x}\). 则
            \[\dv{y}{x}=u+x\dv{u}{x}=\frac{1+u}{1-u}.\]
            整理得
            \[\frac{\dd{x}}{x}=\frac{1-u}{1+u^2}\dd{u}\]
            由此积分,得
            \[\ln|x|=\frac{1}{2}\ln(1+u^2)-\arctan u+C.\]
            将 \(u\) 换回 \(y/x\), 得到该微分方程的解为
            \[\sqrt{x^2+y^2}=C\exp(\arctan\frac{y}{x}).\]
            这时如果写成极坐标形式 \(x=r\cos\theta, y=r\sin\theta\), 则解可写为
            \[r=C\exp\theta.\]
            这是一族以原点为极点的对数螺线.
        \end{xmp}

        \begin{dfn}
            []
            {Bernoulli 方程}
            [Bernoulli Equation]
            [Dedicatia]
            形如
            \[\dv{y}{x}+p(x)y=q(x)y^n\]
            其中 \(n\neq 0,1\) 的微分方程称为\textbf{Bernoulli 方程}.
        \end{dfn}

        \begin{ppt}
            []
            {Bernoulli 方程的解法}
            []
            [Dedicatia]
            对于 Bernoulli 方程, 在两端同时乘 \((1-n)y^{-n}\), 得
            \[(1-n)y^{-n}\dv{y}{x}+(1-n)p(x)y^{1-n}=(1-n)q(x)\]
            注意到
            \[\dv{y^{1-n}}{x}=(1-n)y^{-n}\dv{y}{x}\]
            所以令 \(v=y^{1-n}\). 则上式变为
            \[\dv{v}{x}+(1-n)p(x)v=(1-n)q(x)\]
            这是一个关于 \(v\) 的一阶线性微分方程. 
        \end{ppt}

        \begin{dfn}
            []
            {Riccati 方程}
            [Riccati Equation]
            [Dedicatia]
            形如
            \[\dv{y}{x}=p(x)y^2+q(x)y+r(x)\]
            的微分方程称为\textbf{Riccati 方程}.
        \end{dfn}

\section{微分方程解的存在性与唯一性}

    \subsection{}
        
        \begin{thm}
            []
            {}
            []
            [猫猫]
            设 \(f:\R^2\to\R\), \(f\) 在 \(\R^2\) 连续, \(\exists M:\R^+, |f(u,v_1)-f(u,v_2)|\leq M|v_1-v_2|\), 则 \(y'(x)=f(x,y(x))\) 满足初始条件 \(y(x_0)=y_0\) 在 \(x_0\) 的某个邻域上存在唯一解. 
        \end{thm}

        \begin{prf}
            通过 Banach 压缩映射定理证明. 
        \end{prf}

\section{高阶微分方程}

\end{document}