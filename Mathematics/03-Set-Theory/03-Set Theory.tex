\documentclass[UTF8]{ctexart}

\makeatletter
\def\input@path{{Fulcrum-Template/}{Fulcrum-Template/OperatorList/}}
\makeatother

\usepackage{FulcrumHabitCN}

% ams package
\usepackage{amsfonts}
\usepackage{amssymb}
\usepackage{amsthm}
\usepackage{amsmath}

% margin
\usepackage{geometry}

% \dd
\usepackage{physics}

% Boldface
\usepackage{bm}

% Tikz
\usepackage{tikz}
\usetikzlibrary{calc}

% Gaussian Elimination
\usepackage{gauss}

% Commutative Graph
\usepackage[all]{xy}

% Comment
\usepackage{comment}

\title{Title}
\author{Fulcrum4Math}
\date{\today}


\geometry{
    paper =a4paper,
    top =3cm,
    bottom =3cm,
    left=2cm,
    right =2cm
}

\linespread{1.2}

\begin{document}

\section{Zermelo-Fraekel公理集合论(承认选择公理, 即ZFC公理体系)}

    \begin{axm}
        []
        {外延公理}
        []
        []


        若两集合拥有相同元素, 则两集合相等. 
        \[X=Y\Longleftrightarrow\forall x\in X, x\in Y; \forall y\in Y, y\in X\]
    \end{axm}
    
    \begin{dfn}
        []
        {空集(Empty Set)}
        []
        []


        假设$X$是任意集合, 
        \[\varnothing:=\{x\in X|x\neq x\}\]
    \end{dfn}
    
    \begin{axm}
        []
        {配对公理}
        []
        []


        可以将两个集合配对组成一个新集合. 
        \[\forall x,y, \exists\{x,y\}\]
    \end{axm}
    
    \begin{dfn}
        []
        {有序对(Ordered Pair)}
        []
        []

        \[(x,y):=\{\{x\},\{x,y\}\}\]
    \end{dfn}
    
    \begin{dfn}
        []
        {Catesian乘积}
        []
        []

        \[X\times Y:=\{(x,y)|x\in X, y\in Y\}\]
    \end{dfn}

    由以上两条可归纳定义有限多元素组及Catesian乘积. 

    \begin{dfn}
        设$P$为关于集合的一个性质, 以$P(x)$表示集合$x$满足性质$P$. 
    \end{dfn}
    
    \begin{axm}
        []
        {分离公理模式}
        []
        []


        可以从一个集合中筛选出满足某个条件的元素组成一个新集合. 
        \[\forall X, \exists Y=\{x\in X|P(x)\}\]
    \end{axm}
    
    \begin{axm}
        []
        {并集公理}
        []
        []


        可以将一个集族中全体集合的元素合并为一个新集合. 
        \[\forall X, \exists\bigcup X:=\{x|\exists y\in X: x\in y\}\]
    \end{axm}
    
    \begin{axm}
        []
        {幂集公理}
        []
        []


        可以将集合的全体子集取出组成一个新的集合. 
        \[\forall X, \exists\PP(X):=\{x|x\subseteq X\}\]
    \end{axm}
    
    \begin{dfn}
        []
        {归纳集}
        []
        []


        集合$X$称为是一个\textbf{归纳集}, 若$\forall x\in X, x'\in X$, 其中$x'$称为$x$的后继. 
    \end{dfn}
    
    \begin{axm}
        []
        {无穷公理}
        []
        []

        
        存在无穷集. (通过定义$x$的后继$x'=x\cup\{x\}$实现. )
        \[\exists X: \varnothing\in X\wedge\forall x\in X, x\cup\{x\}\in X\]
    \end{axm}
    
    \begin{axm}
        []
        {替换公理模式}
        []
        []

    \end{axm}
    
    \begin{axm}
        []
        {正则公理}
        []
        []


        任何非空集都含有一个对从属关系$\in$极小的元素. 
    \end{axm}
    
    \begin{thm}
        不存在无穷的从属链: 
        \[\nexists \{x,x_1,x_2,\cdots\}: x\ni x_1\ni x_2\ni \cdots\]
    \end{thm}
    
    \begin{axm}
        []
        {选择公理}
        []
        []


        对由非空集合组成集族, 可构造选择函数, 从每个集合中选择一个元素组成一个新集合. 
        \[X: \forall x\in X, x\neq\varnothing\Longrightarrow\exists g:X\to\bigcup X: \forall x\in X, g(x)\in x\]
    \end{axm}

\section{序结构与序数}

    \begin{dfn}
        []
        {偏序关系}
        []
        []


        定义在集合$S$上的二元关系$"\leq"$称为是一个\textbf{偏序关系}, 若它具有: 

        (1)自反性: 
        \[\forall x\in S, x\leq x\]
        
        (2)传递性: 
        \[x\leq y\wedge y\leq z\Longrightarrow x\leq z\]

        (3)反对称性: 
        \[x\leq y\wedge y\leq x\Longrightarrow x=y\]

        此时称结构$(P,\leq)$为\textbf{偏序集}. 

        若二元关系$"\leq"$满足$(1), (2)$, 称为\textbf{预序关系}, 相应的结构$(S,\leq)$称为\textbf{预序集}. 
    \end{dfn}
    
    \begin{dfn}
        []
        {保序映射}
        []
        []


        设结构$(S,\leq_S),(T,\leq_T)$是预序集, 映射$f:S\to T$称为是\textbf{保序映射}, 若: 
        \[\forall x,y\in S: x\leq_S y\Longrightarrow f(x)\leq_T f(y)\]
    \end{dfn}

    方便起见, 记$x\leq y\wedge x\neq y$为$x<y$, 及引入$\geq, >$符号. 
    
    \begin{dfn}
        []
        {全序集/线序集(Linearly Ordered Set)/链}
        []
        []


        偏序集$S$称为是一个全序集, 若: 
        \[\forall x,y\in S, x\leq y\vee y\leq x\]
    \end{dfn}

\section{无穷递归原理}

\section{基数}  
    
    \begin{dfn}
        []
        {等势}
        []
        []


        集合$X,Y$称为是\textbf{等势(have the same Cardinality)}的, 若$\exists$双射$\phi:X\to Y$. 
    \end{dfn}
    
    \begin{thm}
        []
        {}
        []
        []
        $\forall$不可数集$X$, 
        \[\card X^2=\card X\]
    \end{thm}

    证明: 

    % 以下这段是从旧的数分部分拷过来的:
			
			\begin{dfn}
				[]
				{可数集(Countable Set)}
				[]
				[]
                集合$S$称为是一个可数集, 若$\card(S)=\aleph_0$
			\end{dfn}
				
			\begin{ppt}
                []
                {}
                []
                []
				可数集的子集是可数集. 
			\end{ppt}
			
			证明: 
				
				设任意可数集$S=\{a_1,a_2,\cdots,a_n,\cdots\}$
				\[\forall S'=\{a_i|a_i\in S\}\subseteq S: |S'|=+\infty, \]
				\[f(a_i):=|S'\cap\{a_1,a_2,\cdots,a_i\}|\]
				
				有$f:S'\to\mathbb{Z}^+$是双射, 即$S'$是可数集. $\square$
			
			\begin{thm}
                []
                {}
                []
                []
				设集族$\bar S=\{S_i|S_i\mbox{是可数集}\}$是可数集, 则: 
				\[S=\bigcup_{S_i\in \bar S}S_i\mbox{是可数集. }\]
			\end{thm}
			
			证明: 
				
				$\because S_i\in\bar S$是可数集, $\Longrightarrow$可设$S_i=\{a_{i,1},a_{i,2},\cdots, a_{i,n},\cdots\}$
				
				将$a_{i,j}$写成矩形数阵形式: 
				\[\begin{matrix}
				\xymatrix{
				a_{1,1}\ar[d] & a_{1,2}\ar[ddl] & a_{1,3}\ar[dddll] & a_{1,4}\ar[ddddlll] & \cdots & a_{1,n} & \cdots\\
				a_{2,1}\ar[ur] & a_{2,2}\ar[ur] & a_{2,3}\ar[ur] & a_{2,4}\ar[ur] & \cdots & a_{2,n}& \cdots\\
				a_{3,1}\ar[ur] & a_{3,2}\ar[ur] & a_{3,3}\ar[ur] & a_{3,4}\ar[ur] & \cdots & a_{3,n}& \cdots\\
				a_{4,1}\ar[ur] & a_{4,2}\ar[ur] & a_{4,3}\ar[ur] & a_{4,4}\ar[ur] & \cdots & a_{4,n}& \cdots\\
				\vdots\ar[ur] & \vdots\ar[ur] & \vdots\ar[ur] & \vdots\ar[ur] & \ddots & \vdots & \cdots\\
				a_{m,1} & a_{m,2} & a_{m,3} & a_{m,4} & \cdots & a_{m,n} & \cdots\\
				\vdots & \vdots & \vdots & \vdots & \vdots & \vdots & \ddots
				}
				\end{matrix}\]
				
				按照折线方向构造一个元素序列$\{a'_n\}$, 其中: 
				
				\[a_{m,n}\mbox{的序数为}\frac{(m+n-1)(m+n-2)}{2}+n\]
				
				\[f(a_{m,n})=|\{a'_1,a'_2,\cdots,a'_{\frac{(m+n-1)(m+n-2)}{2}+n}\}|\]
				
				有$f$为双射, $\Longrightarrow S$是可数集. $\square$
				
			\begin{thm}
                []
                {}
                []
                []
				有理数集$\mathbb{Q}$是可数集. 
			\end{thm}
				
				证明: 
				
					\[\mathbb{Q}=\bigcup_{p\in\mathbb{Z}}\{\frac{p}{q}|q\in(\mathbb{Z}-\{0\})\}\]
					
					其中$\mathbb{Z}, (\mathbb{Z}-\{0\})$都是可数集, $\Longrightarrow\mathbb{Q}$是可数集. $\square$

    \section{二元关系}

    \subsection{二元关系}
        \begin{dfn}
            [UUID]
            {二元关系}
            [Binary Relations]
            [czy]
            如果 \( R \subseteq A \times B \),则 \( R \) 是从 \( A \) 到 \( B \) 的\textbf{二元关系}。\\
            如果 \( R \subseteq A \times A \),则 \( R \) 是 \( A \) 上的\textbf{二元关系}。\\
            \( a R b \iff (a, b) \in R \),其中 \( a \in A \land b \in B \)。
        \end{dfn}

        \begin{dfn}
            [UUID]
            {二元关系的性质}
            [Properties of Binary Relations on \( A \)]
            [czy]
            令\( R \subseteq A \times A \),则R可能拥有的性质如下:
            \begin{enumerate}
                \item \textbf{自反性(Reflexive)}:\( \forall a \in A, \; a R a \)。
                \item \textbf{对称性(Symmetric)}:\( \forall a, b \in A, \; a R b \implies b R a \)。
                \item \textbf{反对称性(Antisymmetric)}:\( \forall a, b \in A, \; a R b \land b R a \implies a = b \)。
                \item \textbf{传递性(Transitive)}:\( \forall a, b, c \in A, \;  a R b \land b R c, \implies a R c \)。
            \end{enumerate}
        \end{dfn}

        \begin{xmp}
            [UUID]
            {二元关系举例及其性质}
            [Examples of Binary Relations and Their Properties]
            [czy]
            \begin{enumerate}
                \item \textbf{小于等于关系}
                    设 \( A = \mathbb{R} \)(实数集),定义关系 \( R \) 为:
                    \[
                    R = \{ (a, b) \mid a \leq b \}
                    \]

                    \begin{itemize}
                        \item \textbf{自反性}:是。因为 \( \forall a \in \mathbb{R}, \; a \leq a \)。
                        \item \textbf{对称性}:否。例如,\( 1 \leq 2 \) 但 \( 2 \nleq 1 \)。
                        \item \textbf{反对称性}:是。如果 \( a \leq b \land b \leq a \),则 \( a = b \)。
                        \item \textbf{传递性}:是。如果 \( a \leq b \land b \leq c \),则 \( a \leq c \)。
                    \end{itemize}
                \item \textbf{整除关系}
                    设 \( A = \mathbb{Z}^+ \),定义关系 \( R \) 为:
                    \[
                    R = \{ (a, b) \mid a \mid b \}
                    \]
                    
                    \begin{itemize}
                        \item \textbf{自反性}:是。因为 \( \forall a \in \mathbb{Z}^+, \; a \) 整除 \( a \)。
                        \item \textbf{对称性}:否。例如,\( 2 \) 整除 \( 4 \),但 \( 4 \) 不整除 \( 2 \)。
                        \item \textbf{反对称性}:是。如果 \( a \) 整除 \( b \land b \) 整除 \( a \),则 \( a = b \)。
                        \item \textbf{传递性}:是。如果 \( a \) 整除 \( b \land b \) 整除 \( c \),则 \( a \) 整除 \( c \)。
                    \end{itemize}
                \item \textbf{空关系 (Empty Relation)}
                设 \( A \) 是一个集合,定义关系 \( R \) 为:
                \[
                R = \emptyset
                \]
                即 \( R \) 不包含任何元素。
                
                \begin{itemize}
                    \item \textbf{自反性}:否。除非 \( A = \emptyset \),否则 \( R \) 不满足自反性。
                    \item \textbf{对称性}:是。因为 \( R \) 中没有元素,对称性条件自动满足。
                    \item \textbf{反对称性}:是。因为 \( R \) 中没有元素,反对称性条件自动满足。
                    \item \textbf{传递性}:是。因为 \( R \) 中没有元素,传递性条件自动满足。
                \end{itemize}
            \item \textbf{全关系 (Universal Relation)}
                设 \( A \) 是一个集合,定义关系 \( R \) 为:
                \[
                R = A \times A
                \]
                即 \( R \) 包含所有 \( A \) 中的元素对。
                
                \begin{itemize}
                    \item \textbf{自反性}:是。因为 \( \forall a \in A, \; (a, a) \in R \)。
                    \item \textbf{对称性}:是。如果 \( (a, b) \in R \),则 \( (b, a) \in R \)。
                    \item \textbf{反对称性}:否。除非 \( A \) 只有一个元素,否则存在 \( a \neq b \) s.t. \( (a, b) \in R \land (b, a) \in R \)。
                    \item \textbf{传递性}:是。如果 \( (a, b) \in R \land (b, c) \in R \),则 \( (a, c) \in R \)。
                \end{itemize}
            \end{enumerate}
        \end{xmp}
        
        \begin{dfn}
            [UUID]
            {逆关系}
            [Inverse Relation]
            [czy]
            如果 \( R \subseteq A \times B \),则 \( R \) 的\textbf{逆关系} \( R^{-1} \subseteq B \times A \) 定义为:
            \[
            R^{-1} = \{ (b, a) \mid (a, b) \in R \}
            \]
        \end{dfn}

        \begin{ppt}
            [UUID]
            {}
            []
            [czy]
            \( (R^{-1})^{-1} = R \)
        \end{ppt}

        \begin{dfn}
            [UUID]
            {关系的复合}
            [Composition of Relations]
            [czy]
            \( R \subseteq B \times C \), \( S \subseteq A \times B \),则 \( R \) 和 \( S \) 的\textbf{复合} \( R \circ S \subseteq A \times C \) 定义为:
            \[
            R \circ S = \{ (a, c) \mid \exists b \in B, \; (a, b) \in S \land (b, c) \in R \}
            \]
        \end{dfn}

        \begin{ppt}
            [UUID]
            {}
            []
            [czy]
            \( (R \circ S)^{-1} = S^{-1} \circ R^{-1} \)
        \end{ppt}

        \begin{ppt}
            [UUID]
            {}
            []
            [czy]
            \( R \circ (S \circ T) = (R \circ S) \circ T \)
        \end{ppt}

        \begin{ppt}
            [UUID]
            {性质的等价表示}
            [Properties in Terms of Relations]
            [czy]
            \begin{itemize}
                \item \( R \) 是\textbf{自反的} \( \iff I_A \subseteq R \),其中 \( I_A = \{ (a, a) \mid a \in A \} \)。
                \item \( R \) 是\textbf{对称的} \( \iff R = R^{-1} \)。
                \item \( R \) 是\textbf{传递的} \( \iff R \circ R \subseteq R \)。
                \item \( R \) 是\textbf{反对称的} \( \iff R \cap R^{-1} \subseteq I_A \)。
            \end{itemize}
        \end{ppt}


        \begin{dfn}
            [UUID]
            {等价关系}
            [Equivalence Relation]
            [czy]
            如果关系 \( R \subseteq A \times A \) 满足自反性、对称性和传递性,则 \( R \) 是 \( A \) 上的\textbf{等价关系}。
        \end{dfn}

        \begin{dfn}
            [UUID]
            {等价类}
            [Equivalence Class]
            [czy]
            如果 \( R \) 是 \( A \) 上的等价关系且 \( a \in A \),则 \( a \) 的\textbf{等价类}定义为:
            \[
            [a]_R = \{ b \in A \mid a R b \}
            \]
        \end{dfn}

        \begin{thm}
            [UUID]
            {}
            []
            [czy]
            设 \( R \) 是集合 \( A \) 上的等价关系,\( a, b \in A \),则:
            \[
            a R b \iff [a]_R = [b]_R
            \]
        \end{thm}
            
        \begin{prf}

                “\(\Rightarrow\)”: \( a R b \implies b R a \) \( \forall c \in A \),\\
                \(
                c \in [a]_R \iff c R a \iff c R b \iff c \in [b]_R
                \)
                \( \therefore [a]_R = [b]_R \).

                “\(\Leftarrow\)”:\( b R b \implies  b \in [b]_R = [a]_R \) \( \therefore a R b \).\qed
        \end{prf}

        \begin{ppt}
            [UUID]
            {}
            []
            [czy]
            设 \( R \) 是集合 \( A \) 上的等价关系,\( a, b \in A \),则:
            \[
            a R b \iff [a]_R \cap [b]_R \neq \emptyset.
            \]
        \end{ppt}
        
        \begin{prf}
                “\(\Rightarrow\)”:显然。

                “\(\Leftarrow\)”: \( [a]_R \cap [b]_R \neq \emptyset \implies c \in A , c \in [a]_R \land c \in [b]_R \),\\
                \(
                \therefore c R a , c R b \implies a R c , c R b \implies a R b.
                \)
                \qed            
        \end{prf}

        \begin{dfn}
            [UUID]
            {划分}
            [Partition]
            [czy]
            集合 \( P \subseteq \mathcal{P}(A) \) 称为 \( A \) 的\textbf{划分},如果满足:
            \begin{enumerate}
                \item \( \forall B \in P, \; B \neq \emptyset \)。
                \item \( \forall B_1, B_2 \in P, \; \text{如果 } B_1 \neq B_2, \text{ 则 } B_1 \cap B_2 = \emptyset \)。
                \item \( \bigcup P = A \)。
            \end{enumerate}
        \end{dfn}

        \begin{lma}
            [UUID]
            {}
            []
            [czy]
            \( P \) 是 \( A \) 的划分, \( a \in A \) \(\implies \exists ! B \in P s.t. a \in B \)。
        \end{lma}

        \begin{thm}
            [UUID]
            {}
            []
            [czy]
            如果 \( R \) 是 \( A \) 上的等价关系,则等价类的集合 \( \{ [a]_R \mid a \in A \} \) 构成 \( A \) 的一个划分。
        \end{thm}
        
        \begin{thm}
            [UUID]
            {}
            []
            [czy]
            设 \( P \) 是集合 \( A \) 的一个划分,定义关系 \( R \) 为:
            \[
            R = \{ (a, b) \in A \times A \mid \exists B \in P, \; a \in B \text{ 且 } b \in B \}
            \]
            则 \( R \) 是 \( A \) 上的等价关系,且 \( P = \{ [a]_R \mid a \in A \} \)。
        \end{thm}

        \begin{prf}
            \begin{itemize}
                \item \textbf{自反性}:\(\forall a \in A \),\(\exists B \in P \) s.t. \( a \in B \),\(\therefore a R a \)。
                \item \textbf{对称性}:\(\forall a, b \in A \),如果 \( a R b \),则\(\exists B \in P \) s.t. \( a \in B \land b \in B \),\(\therefore b R a \)。
                \item \textbf{传递性}:\(\forall a, b, c \in A \),如果 \( a R b \land b R c \),则\(\exists B, B' \in P \) s.t. \( a, b \in B \land b, c \in B' \)。由于 \( b \in B \cap B' \),且 \( P \) 是划分,故 \( B = B' \),\(\therefore a R c \)。
            \end{itemize}

            \(\therefore\) \( R \) 是 \( A \) 上的等价关系.

            下证 \( P = \{ [a]_R \mid a \in A \} \).

            \begin{itemize}
                \item “\(\subseteq\)”:\(\forall B \in P \),\(\exists a \in B \),只需证明 \( [a]_R = B \)。
                \begin{itemize}
                    \item (i) \(\forall b \in [a]_R \),有 \( a R b \),\(\therefore \exists B' \in P \) s.t. \( a, b \in B' \),故 \( B = B' \),\(\therefore b \in B \)。
                    \item (ii) \(\forall b \in B \),有 \( a R b \),\(\therefore b \in [a]_R \)。综上,\( [a]_R = B \)。
                \end{itemize}
                \item “\(\supseteq\)”:\(\forall a \in A \),\(\exists B \in P \) s.t. \( a \in B \),只需证明 \( [a]_R = B \)。
            \end{itemize}
            \qed
        \end{prf}

    \subsection{闭包}

        \begin{dfn}
            [UUID]
            {闭包}
            [Closure]
            [czy]
            设 \( R \subseteq A \times A \),对于性质P,\( R' \) 是 \( R \) 的\textbf{闭包},当且仅当:
            \begin{enumerate}
                \item \( R \subseteq R' \);
                \item \( R' \) 具有性质P;
                \item \(\forall T \subseteq A \times A \),如果 \( R \subseteq T \) 且 \( T \) 具有性质P,则 \( R' \subseteq T \)。
            \end{enumerate}
        \end{dfn}

        \begin{dfn}
            [UUID]
            {}
            []
            [czy]
            设 \( R \subseteq A \times A \),定义:
            \[
            R^1 = R, \quad R^{n+1} = R^n \circ R, \quad R^+ = \bigcup_{n=1}^{\infty} R^n, \quad R^* = \bigcup_{n=0}^{\infty} R^n
            \]
        \end{dfn}

        \begin{lma}
            [UUID]
            {}
            []
            [czy]            
                如果 \( R \subseteq R' \land T \subseteq T' \),则 \( R \circ T \subseteq R' \circ T' \)。
        \end{lma}

        \begin{lma}
            [UUID]
            {}
            []
            [czy]
                如果 \( R_1, R_2, \ldots, T \subseteq A \times B \),且\(\forall n \in \mathbb{Z}^+ \),有 \( R_n \subseteq T \),则 \( \quad \bigcup_{n=1}^{\infty} R_n \subseteq T \)。
        \end{lma}

        \begin{lma}
            [UUID]
            {}
            []
            [czy]
                \( R^n \circ R^m = R^{n+m} \)(\( n, m \in \mathbb{Z}^+ \))。
        \end{lma}

        \begin{lma}
            [UUID]
            {}
            []
            [czy]
            \( R^+ \) 是传递的。
            \end{lma}

            \begin{prf}
            假设 \( a R^+ b \land b R^+ c \),则\(\exists n, m \in \mathbb{Z}^+ \) s.t. \( a R^n b \land b R^m c \)。
            \[
            \therefore a (R^n \circ R^m) c \quad \text{即} \quad a R^{n+m} c
            \]
            由于 \( R^{n+m} \subseteq R^+ \),故 \( a R^+ c \)。\( \therefore R^+ \) 是传递的。
            \qed
        \end{prf}

        \begin{lma}
            [UUID]
            {}
            []
            [czy]
            如果 \( R \subseteq T \subseteq A \times A \),且 \( T \) 是传递的,则 \( R^+ \subseteq T \)。
        \end{lma}

        \begin{prf}
            \(\forall n \in \mathbb{Z}^+ \),证明 \( R^n \subseteq T \):
            \begin{itemize}
                \item (1) 当 \( n = 1 \) 时,\( R^1 = R \subseteq T \)。
                \item (2) 假设 \( n = k \) 时 \( R^k \subseteq T \),则 \( R^{k+1} = R^k \circ R \subseteq T \circ T \subseteq T \)(因为 \( T \) 是传递的)。
            \end{itemize}
            根据定义,\( R^+ = \bigcup_{n=1}^{\infty} R^n \subseteq T \)。
            \qed
        \end{prf}

        \begin{thm}
            [UUID]
            {}
            []
            [czy]
            \( R^+ \) 是 \( R \) 的传递闭包。
        \end{thm}

        \begin{prf}
            由引理已证。
            \qed
        \end{prf}

    \subsection{函数}

        \begin{dfn}
            [UUID]
            {函数}
            [Functions]
            [czy]
            \( F \subseteq A \times B \) 是一个\textbf{函数},当且仅当满足以下条件:
            \begin{enumerate}
                \item \(\forall x \in A \),\(\exists y \in B \) s.t. \( x F y \)。
                \item \(\forall x \in A \),如果 \( x F y_1 \land x F y_2 \),则 \( y_1 = y_2 \)。
            \end{enumerate}

            记法:\( F: A \rightarrow B \) 表示 \( F \) 是从 \( A \) 到 \( B \) 的函数。

            \(\forall x \in A \),\( F(x) \) 表示唯一的 \( y \in B \) s.t. \( x F y \)。
        \end{dfn}

        \begin{ppt}
            [UUID]
            {}
            []
            [czy]
            设 \( F, G: A \rightarrow B \) 是两个函数,则 \( F = G \Longleftrightarrow\forall x \in A , F(x) = G(x) \).
        \end{ppt}

        \begin{thm}
            [UUID]
            {}
            []
            [czy]
            设 \( F: B \rightarrow C \) 和 \( G: A \rightarrow B \) 是两个函数,则:
            \begin{enumerate}
                \item \( F \circ G: A \rightarrow C \) 是一个函数。
                \item \(\forall x \in A \),\( (F \circ G)(x) = F(G(x)) \)。
            \end{enumerate}
        \end{thm}
    

\end{document}